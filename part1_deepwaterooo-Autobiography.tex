% Created 2015-12-12 Sat 15:42
\documentclass[12pt]{book}
\usepackage{graphicx}
\usepackage{xcolor}
\usepackage{xeCJK}
\setCJKmainfont{SimSun}
\usepackage{longtable}
\usepackage{float}
\usepackage{textcomp}
\usepackage{geometry}
\geometry{left=1.5cm,right=1.5cm,top=2cm,bottom=1.5cm}
\usepackage{multirow}
\usepackage{multicol}
\usepackage{listings}
\usepackage{algorithm}
\usepackage{algorithmic}
\usepackage{latexsym}
\usepackage{natbib}
\usepackage{fancyhdr}
\usepackage[xetex,colorlinks=true,CJKbookmarks=true,linkcolor=blue,urlcolor=blue,menucolor=blue]{hyperref}


\lstset{language=Java,numbers=left,numberstyle=\tiny,basicstyle=\ttfamily\small,tabsize=4,frame=none,escapeinside=``,extendedchars=false,keywordstyle=\color{blue!70},commentstyle=\color{red!55!green!55!blue!55!},rulesepcolor=\color{red!20!green!20!blue!20!}}
\author{deepwaterooo}
\date{\today}
\title{The Autobiography of deepwaterooo\textasciitilde{} \linebreak Part 1: 成长的故事 --- 我和舅舅}
\hypersetup{
  pdfkeywords={},
  pdfsubject={},
  pdfcreator={Emacs 24.3.1 (Org mode 8.2.7c)}}
\begin{document}

\maketitle
\tableofcontents


\chapter{前言与杂记}
\label{sec-1}
\section{前言}
\label{sec-1-1}

发信人: Dreamer (不要问我从哪里来), 信区: Dreamer

标  题: 成长的故事 -- 我和舅舅

发信站: BBS 未名空间站 (Fri Nov  4 16:07:51 2011, 美东)

  今天,中午,刚刚出门一趟办事,回到桌边。到这一刻,我终于意识到自己的历史、故
事对今天的自己造成了多么沉重的精神压力和负担。虽然长到今天我32岁了,但是总有
那么几个时刻,借用一网友一句话,“我的小宇宙随时都有暴发的可能”,但是今天,
我终于还是要爆发了。或许我不善于保护自己,或许就像自己以为的,因为我不漂亮而
又太引侧目豆遭到太多恶俗眼神,就像多年前一位算命先生所说,我的逆势太多了。

  今天,当无言的猜忌又一次地将自己推向风口浪冲,我想也该是时候稍微澄清一下自己
了。

  我曾想过写一本书,一本关于自己的传记,但我没有过人的才华,也没有任何的成就,
这书写出来,也未必能有读者。但我的生命里也确实发生了很多故事,对生命的感受有
一些。虽然后面还有大半生要过,但我觉得我最有借监意义的故事事情,也早在这前半
生都已经发生了。今天处在风口浪尖,把必要的有意义的、值得思考的故事写出来,也
算是把这些故意的借鉴意义奉献给了大家,也算是完成了曾经那个写书的心愿。茫茫人
海,人的思想千奇百怪,能有一小批能够理解的读者,也算是达到目的了。

  有些故事已经久远了,可以说的我会说出来;有些事情发生在当下,我也只能尽自己最
大努力来比较客观地陈述事实。有些事情虽然过去已有些年头,但今天说出来也必将会
伤害到当事人,我只陈述我可以陈述的部分,希望将不必要的伤害降到最低。

\section{关于一个瘦女孩的记忆  }
\label{sec-1-2}

  高中过去很久了,很多人很多事都已忘记,但是有一个女孩,一直印在我的脑海里。

     高一的时候,我和她分在一个班,很有幸地坐在她前一桌。她清瘦、高挑、白皙,
常穿着不太合身的两三套衣服,或者袖子、裤腿短一截,或者穿着把人显得臃肿,但她
好像并不在意这些。后来,她也有两套夏日的长袖衣裤,像是手工裁制的,但穿得很有
气质。

     班上还有一个白肤也很白、长相一般、个儿不算高的女孩和她很要好。语文老师常
拿她的作文当范文在课堂上读。可真正让我注意到她的,却是因为一个苹果。

     我们这些走读的学生,早中晚餐一般也都是回家去吃。小胖女孩一般中午很早就带
着一个苹果来到学校。什么人不在家里吃东西,却要把个苹果带来学校吃呢?我很好奇
。瘦女孩一般都到得要晚很多。那日中午那瘦女孩到校后,就看见小胖女孩拿出一把小
水果刀,把早就洗干净的苹果一切为二,稍大的一半给了瘦女孩。以后再多留意她们,
就知道胖女孩带到学校的所有零食都会分给瘦女孩一半儿。
 
     还有一天上午第一节课下课,离早餐结束已经一个小时,离中饭时间也还有三堂课
,有一个老奶奶提着两小袋零售,一袋饼干,一袋像是什么甜点来找她。听她们谈话才
知道,早餐时间瘦女孩回家去时,负责开关大院门的这位奶奶睡过回头觉,早餐时间竟
没能开门。接下来的几堂课,那两袋零食,女孩都不曾动,静静地躺在她的课桌里。只
是中午一放学,她就拿着这些零食急急地出去了。

     后来听说,这瘦女孩没有爸爸,在她很小的时候爸爸因为一场车祸永远地离开了她
们。她妈妈凭借着裁缝的手艺,一个人艰难地抚养着她和弟弟。

     瘦女孩人很好,长得也漂亮,什么时候见到她都是副笑脸,乐呵呵的。我有什么不
懂的题目问她,她都能很快帮我讲明白。语文课我最头大了,她却知道很多文言典故、
唐诗宋词类的。她学习上花的时间少,可学习成绩一直比我好很多。听一个初中都与她
同班的同学说,她算聪明不勤奋,考试题目简单,她倒考得一般,倒是题目越难,她倒
考得越好!

     很快就到了高二,该分文理班了。只有小胖女孩读了文科,我们两个也分开了,我
被分到了别的班。听同学说,女孩的妈妈和老班的老婆极要好,是我们高一老班不惜一
切代价把她留在了自己的班。其实班主任欣赏一个学生,把她留在自己班,应该不算过
分的事吧。在我眼里,她从老班那里享受到的特殊待遇,只有座位一件。女孩儿很高,
高一就有一米七,可视力很好,在80\%学生都戴眼镜的班里,她该是唯一一个学习好却
不太近视的了。老班可能也比较怜惜吧,我们按高矮次序坐座位的,一直把这个一米七
放在第三排坐着。高考她也不负众望,六百五十多分,至少在年级前十名。普通人家的
女儿,或许未必就要考清华、北大吧,或者学医是好民的志向,她读了同济医科大。

     以后的事情,全都是听同学说的了。大学里,她也一直申请减免学费,有了一个男
朋友,帮她减轻各种负担,后来成了她老公。她也继续读了研,现在北京一家医院工作
,绝对算是工作顺利、家庭幸福了。小胖女孩后来上了哪所大学,现在哪里,在做什么
,我一直没能打听到。高二分班后我们基本就没再见面了,但是瘦女孩的故事、关于她
的记忆却伴随了我很多年。

\section{青梅竹马}
\label{sec-1-3}

  上小学一年级的时候,班上转来三个复读书。因为年龄比我们大,所以他们三个光
荣地成为了我们班第一批少先队员。当鲜红的红领巾戴到脖子上的时候,我很羡慕他们
。或许从那个时候开始,我慢慢地注意到了他。

     他并不帅,只是皮肤很白,个头很高,可能是当时班上最高的吧。注意到他,也有
地利之便。他住临村,上学要比我远一倍。除了走与我同样的路,他们上下学还要翻过
两道堤坝,而水渠上只有一道没有扶栏的窄桥。他是村里的娃娃头,很能照顾低年级的
小弟弟小妹妹。也很有亲和力,即使比我们年级高的哥哥姐姐,也喜欢与他们这帮小P
孩儿一起上下学。我虽不与他们同村,但我从来都是等到他们的部队到达才去上学,放
学再随他们一起回家。那会儿大家最大的兴趣就是评论前一天晚上的电视剧,什么<绝
代双骄>、<霍元甲>、<小李飞刀>啊,绘声绘色地描述精彩情节,对于谁谁的武功高,
谁谁厉害,谁谁人好,谁是为了救谁才可能受到了伤之类的总是能争个脸红脖子粗。我
原本该是加入到他们的行列来个一争高下,但我看电视很少。房间里枯黄的灯光下妈妈
拉鞋底,爸爸看小书小报,我写作业竟是我小学傍晚最深刻的记忆。所以我总是默默地
走、静静地听。有时他说得精彩了,会情不自禁投上羡慕的一瞥。就这样每天两个来回
(中午我们回家吃饭的),走了六年。到现在,不是迫不得已,我都不太多说话,多多少
少都有那个年代的印记吧。

     孩童时的我无忧无虑,胖乎乎的,每天闷闷的却也很快乐,老师们大多也都喜欢我
。所以虽然没当过班长,但也一直都是小组组长的那种。记忆里好像他一直都是与我坐
同一组。虽然没有刻意去想什么歪主意,但潜意识中我还是不折不扣地玩弄了权术,利
用我是小组长之便,对他进行百般刁难。比如,老师让小组长检查背课文、笔记什么的
,组里别的同学背个差不多就让他们过了,可是他,背错一个字,我也会要他重背。没
什么严格要求的理论,只是简简单单地觉得,呵呵,要是背不过,你就得再来一遍啊。
所以,日积月累,天长地久,就形成了一种习惯:当组里所有的同学都背完,他会最后
一个出现在我的面前;即便如此,我也不依不饶,照样打发回去重背;第二遍,外孙打
灯笼,照旧,而这时他也会乖巧地把我的笔记拿回去背;到第三遍,看他被我折磨得够
可怜,态度也还不错,也快到了下早自习的时间,于是非常富有人情味地就此放行。长
大后我回想,为什么这个男孩从来都没有向老师告过我的状呢?
 
     他也有不少“恶习”。那会儿老班喜欢让我们带盆子给他家的菜地端水浇菜,每到
这种情况,我总是早早地吃饭中饭好去干活(老师的话可真是圣旨啊);而他们村的三个
总是到上课打预备的时候才见拿着个盆朝学校走。也只有在这种情况下,我才不会等他
们上学;而相反地,我们中午上课迟到无数,偶尔会为迟到愧疚,但更多的还是觉得好
玩儿。冬天化雪的早上,我们带过炭火盆上学;无数个夏天的中午,我被他按趴在课桌
上,只因为我不爱午休干别的。这个好按人家头的毛病跟他说过无数次,只是从来没见
他改过!玩疯的时候,也会装神弄鬼,以百米冲刺的速度逃回家省得被鬼撞了;轮着我
值日打扫卫生的时候,却也帮我带过午饭、雨伞之类。

     时间慢慢过去,到了六年级,换了个看上去很严厉的男老班,还是语文老师,我很
怕他。好像我们的坐位也被调开。除了上下学大部队一起走,其它的记忆都淡忘了。临
近毕业的一件事却让我至今印象深刻。不是他,而是班上的一个女生问我,“XXX,班
上有你喜欢的同学吗”?至今我仍清楚地记得当时我脑海里立刻就印出了他的形象,但
是唐突地被问到那种问题,还是违心地说,“不知道啊,应该没有吧!”谁知她嘴很快
,“你不说我也知道,是XXX吧?”我心里顿时就像打翻了调味瓶,酸甜苦辣都没法描
述,难受到了极点,打个地洞钻下去的心都有。长大后,多少个午夜梦回,回想那一幕
,我当时难受的,是因为第一次意识到“喜欢”了,还是因为这份喜欢被人说破,竟终
究不能分辩。

    小学毕业的暑假,镇上开英语班。我去找他想问他去不去上,叔叔说他去亲戚家了
,让我给他留个字条。于是进到他的房间,草稿纸上,满纸都写着“student”,让我
很受鼓舞了一番。后来,我们被分到了不同的初中,就断了联系。中考后,听妈妈说他
考得不好,只得了五百分左右,上了中专,武汉水利水电学校。

     大一的寒假,他来我家拜年,大家聊天,“促膝长谈”了两个小时,交换了联系方
式,于是我们开始写信;暑假,我没回家;我们又断了联系;大二的寒假,我去他家他
不在,去海南工作了。大四下学期,只知道考研成绩,还没划线不知道结果的时候,我
把我的简历寄给了他,希望他能帮我在他的周边地区投出一些。再后来,我到北京读研
,他常打电话给我,但后来,最终,我们还是断了联系。回想起来,他每周到是在固定
的时间打到我宿舍,而那会儿我最是没心没肺了,居然连续错过了好几次。

     如果我稍微聪明灵性一点儿,或许,大四,我就不该再打扰别人了。但那会儿我压
根儿就不曾多想过,他可是我儿时最要好的伙伴儿啊! 因为自己的不开窃,让我对这件
事愧疚了好一阵子。只是,当愧疚、歉意、遗憾慢慢成为生活的常态,我也学会了淡忘
,也慢慢地忘了有这么回事。只是早前一段时间打电话回家,无意间听妈妈说他结婚了
,觉得很开心,心里很祝福他找到了自己的幸福吧!
\section{August 8, 2007  }
\label{sec-1-4}

  今天是我们班(大学班)帅哥和美女的生日,巧啊,80年8月8日。他们的生日又勾
起我对大学生活的回忆。

  大学对于踏入社会的人来说还是校园生活,可它也是一个小社会,形形色色的人来
到这里,他们稚嫩、羞涩,对于未来的事情一无所知,却又青春张扬,指点江山,激 
扬文字,相信“我能”,相信自己的将来无限美好!和自己一起生活的姐妹、发生在自
己身边的一个又一个的故事,都变为灵魂的活水。今天将这些故事写下来也是因为大学
生活结束已有数年,相信自己对那些故事已经消化通透,相信自己不致于像是针对刚刚发
生过的事情那样,说什么都辞不达意。

  怎么说呢,从同住一宿舍四年的姐妹说起吧。

  我们班上有学生29,女生7人,男女比例3:1。故事就围绕着这7个女孩 子展开,
看似再平常不过,而我即使今天回想起来还觉余韵犹存,历历在目,如同发生在昨天。

  薇是系里的美女,80年8月8日生,家就是武汉周边,父母经常开着小汔车来看她。
初到校园,就开始流传,咱们新一级系里有两大美女,云南的和咱们班上的薇。传说云
南美女家里比咱家薇更有钱点儿,不过咱们的薇长相、性格各个方面也都不弱。感觉她
们也都还在暗地里较着劲呢,呵呵。小龙女除了没有这么多外在美以外,别的条件也都
好,很像薇。

  凯身材纤纤,皮肤白白,眼睛深遂,鼻梁很挺,很像西方人的面孔,很有型。或许
潜意识地将她当作了睿,我的初中同学,俺那个时候最最要好的朋友,我四年里基本有
点儿像是她的fan,对她很要好。这种fan的关系还挺多,纳纳是薇的fan,大学四年一
直都是。

  小亚的家也住在附近的镇上,由于从小一直生长在小镇,用“八面玲珑”来形容这
个女孩一点儿也不过分,真的是聪明伶俐。还剩一个活宝是来自南方的桢,在看过她之
前,我从不曾想象过亚热带地区的阳光会有那么执灼。如果我说她的家乡在广西,我想
你能够反应出她的狭窄的双肩,纤细的身材,典型的高挑女,面部的轮廓也 很美。这
里的故事都围绕她展开,而这么一个人也对我影响很深。
\section{我的十(二)年感言}
\label{sec-1-5}

看到我QQ好友的好友们一篇篇<我的十年感言>,很唏嘘,我也来整一篇。

九八年,高考。同学们大部分都进了理想的大学,我,迫不得已,进了农校;
九九年,过英语四级,糊糊涂涂又一年。
二千年,过英语六级。记得那个冬日的上午,武大绘会展在广场展开,看到很多很美的
作品,感觉很清新、很奇妙!
零一年,害了一场病,作了个手术,考了一次研;
零二年,学五笔输入,到北京读研,开始实验室试验;
零三年,第一个专业报告;
零四年,英语考试年;
零五年,硕士毕业,南下工作;
零六年,准备继续读书;
零七年,继续实验室实验;
零八年,换专业,继续读书;
零九年,读完书,毕业;
一零年,开始工作,终于开始挣两分钱------

\chapter{童年记忆}
\label{sec-2}
\section{我的童年}
\label{sec-2-1}

  记得我的QQ空间里有一篇写自己是野娃娃的日志,但今天找不见了,可能是过去什
么冲动的时候直接删了。那就只好再从自己的出生重新写一遍了。

    1979年夏天里的一天,我出生了。父母满心地希望我是个男孩的愿望伴随着我的哭声最
终还是肥皂泡一般破裂了。我曾经是他们最后的希望。奶奶听爸爸说我是女孩,脸一扭
就走了,连妈妈的房间都不进去一下。后来听妈妈说,刚出生时的我,脸搭在肩膀上。

  这是一个普通的农民家庭。我的上面有三个姐姐,分别是8岁,5岁和两岁。妈妈在
我二姐之后曾流产过一个。我曾经是他们想要儿子的最后希望。七十年代末已经开始由
公社制集体劳动改为个体制分产到户,同时也实午了计划生育。而在执行计划生育的时
候,妈妈已经怀上了我,可能过了月数也不好流产,就生了下来。后来听爸妈说,我一
生下来,就罚了150元,据说是当时一辆自行车的钱。

  我不是聪明小孩,对自己早些年月的事情都不大记得。只是听他们说,小时候想喝
汤叫“糠,糠,要吃糠”;大姐把我从摇篮里不小心摔下来我对爸妈告状说“姐姐抱我
ban(摔)呀(得)jidong(叮咚)”;本来是想向婶婶(叔叔的爱人)炫耀一下自己围在
面前的漂亮兜兜的,结果被婶婶和妈妈一起把我那么漂亮的白色带褶皱边的兜兜哄走给
了堂妹;和姐姐们一起在池塘里玩水摘莲蓬,结果我不知什么时候耳朵一阵呜咽就云里
雾里了,后来姐姐们把我弄起来,给我洗个头,哄我说,“妈他们回来就说我们给你洗
头了,他们也不知道你掉进水里了”,我不知是因为后怕呢,还是觉得姐姐们没把我看
好,哭啊哭的哀号了大半天。

  大家都说爹妈喜欢的是幺儿子幺女,可能也是因为我是家里最小的,很受父母宠爱
。最小的姐姐只比我大两岁,也是一个不懂事儿。任何事情,我们姐妹俩总是争个你死
我活,但当爸爸来审判事情的时候,总是姐姐大,应该让着我才对。这样,以后有什么
事情我只要学会向爸爸告状就可以了,就连我们姐妹俩都挨打的时候,同一根荆条,也
是姐姐挨得重,爸爸打我出手轻。这样我总是缕缕免受责难。不过事情发展到极端就会
有恶果。比如改天爸妈需要一起出门去哪里,要我们姐妹俩在家里看门儿,他们会早早
地买好好吃的犒劳我们。但是往往的结果就是,他们走了,我们是把门儿看好了,但他
们前脚一走,姐姐总算又傣住机会把我狠狠皱一顿,打得我哭天呛地的,是要得好几个
小时才能平复受伤情绪的。爸妈一回来我就告状,他们又要磨几个小时嘴皮好好教育我
们。

  不过,我觉得爹妈宠我还有一个重要原因是我小时候很乖。他们大人因为分田到户
要好好管理田地,会分配一些小事给我们做,比如场子上晒粮食了要赶走小鸡捣乱,偶
尔也需要我们去放一下牛,姐姐总是不听话,而这些都成为我的绝活,因为我把这些小
事做好,晚上爸妈忙完回来就会表扬我很喜欢我。最终,我六岁上学前班(一年,相当
于一年的城市上的幼儿园之类的吧)之前,还在堂哥等大小孩的带领下放过两年的牛。

  以后就是上学后的故事了。
\section{我的小学}
\label{sec-2-2}
\subsection{我的小学(1)}
\label{sec-2-2-1}

  按照规定,我六岁时就去村里上学前班了,再加六年的(九年制义务教育,六年制
小学)义务教育,到小学毕业,十三岁。如同上学前的幸福时光,如同每个作家所回忆
的,童年是一个人最幸福快乐的日子,我的小学也是,很快乐,童年里的黄金时代。虽
然以自己成年后的总结,自己性格里光明阴暗面也都在这七年里形成。

  我并不是聪明小孩。学前班的时候(上学第一年),就发生了一件被别人笑到我小
学毕业的事儿。那时比我大五岁的二姐已经上小学四五年级(她那级还是五年制小学义
务教育),放学回家的路上,二姐班上有个男生逗我,问我“1+1等于几”,在这个陌
生的大男孩面前我很紧张,本来想说“二(耳)儿”的,结果一紧张,张口就变成了“
万儿”。后来我跟村里的大小孩儿、哥哥们到长渠堤坝上放牛,碰见认识我爸妈的长辈
们,还要拿1+1等于几来打趣儿我。
  
  二姐比我大太多,不跟我一般见识,也不喜欢同我玩。她有很多朋友,她下课了就
同她们班的女孩子们抓五颗石头子玩儿。偶尔我去找姐姐,总是惹得她们班的男生女生
笑,姐姐也会跟着变腼腆。多年后我来到美国,姐姐同我QQ聊天,她原来的账号叫“水
中鱼儿”,取意如鱼得水,自由自在;可我后来的成长她觉得我总是形单影只,就像天
空中孤独飞翔的大雁,她便改了网名为“孤雁”以抚慰我在异国他乡孤寂的心灵,让我
很是感动。
  
  虽然小时候和三姐经常打架,但上学了姐姐学会了照顾我。小时候姐姐很爱美,大
姐三姐长得都像爸爸,又瘦又漂亮,二姐同我像妈,胖但聪明一点儿学习好。三姐常从
爸妈放钱的地方偷偷拿出些钱花(为此常受爸妈开小灶单独教育),比如从贺郎那里买
好看的头绳发饰;冬天的早上,下课了,校门口总会有卖热馒头、面包的叔叔阿姨,爸
爸没来得及起床为我和姐姐做早饭的早上,三姐就会买来热馒头给我。爸爸很宠爱我们
姐妹,一年四季都早早起床为我和姐姐煎热乎乎的油饼用作早餐,用纸包了我和姐姐走
在上学的路上边走边吃。冬天天冷,偶尔爸爸也会睡忘了,补救办法就是给我和姐姐一
人一毛五可以买一个热馒头或者面包。馒头热,吃着暖和;面包甜,吃得享受。我常在
馒头和面包面前犹豫,不知该买哪一样。馒头结实,吃得饱;面包好吃,但就不那么饱
了。偶尔我也会恶一天,这样接下来一次我攒够三毛的时候就可以两个一起买,来一次
奢侈的享受。不过我从不向爸爸多要钱,我觉得一毛五够了,能吃上一样就很好了,有
很多一起上学的伙伴从来不买,一年到头儿来也吃不上一样。

  周六的下午,二姐上中学住校放周末回来带菜洗澡换衣服,妈妈会做很多好吃的来
改善姐姐的饮食,所以周六晚上我家都是一成不变的听爸妈讲故事――讲远亲舅舅家表
姐们努力学习考上大学吃商品粮端铁饭碗生活幸福的故事以鼓励上中学的二姐好好学习
,争取二姐考上中专跳出农村农门。我和三姐还小,出访里时进时出,房间里也是出出
进进上窜下跳。爸妈对我和三姐小学并不重视,任由我们玩儿。这样我的小学,从学前
班到小学毕业都一直是原地踏步的第三名。我也贪玩,只要期末考试我能考进前三名,
暑假过年我能拿回一副年画奖状挂在家里父母就很满足,我也就很知足了。可能因为长
得胖,不好动不爱动,运动上一直没有什么长项,小伙伴玩儿的时候我就显得笨拙一点
儿。那时候跳皮筋蹦键子我都不是很强,一般是玩得好的当葫芦,带领大家跳(这样其
它人玩得轻松些),也可以在大家跳“环掉”的时候为他们补(葫芦重新再跳一次)回
来。姐姐为了让我玩,经爸爸允许,从家里带了像皮筋,利用她是娃娃头(比我们班的
小P孩大多了),让我当葫芦,可惜我跳不好,后来就跳得更少了。

  想写该写的很多,下一篇吧
\subsection{我的小学(2)}
\label{sec-2-2-2}

  其实小学除了玩之外,印象深刻的也就是<青梅竹马>里那篇了。不管是因为小地
方,老师学生谁是我姐姐几个姐姐都认识之类的,或许二姐学习好的光环也自然而然地
加到了我的头上,老师们大多很喜欢我,当然我学习也好,同小时候一样也很乖(老师
让浇菜园就浇菜园),但我主要同青梅竹马那帮男生玩,女生要少一些(这个性格在上
国内硕士时更明显一些)。记得小学快毕业的时候,他们有四个女生组成了班上的四大
姐妹,我不是成员。有一个张老师是代课老师,我同他女儿小学时比较要好,上初中就
生分了。老师们不管大官小官总是让我当一个什么官,比如小组组长,检查背书之类的。

  我不喜欢语文,语文七拐八拐我不喜欢。记得一篇什么课文里面主人问一个仆人什
么人之类的“你这么晚还没有休息吗?”语言老师让我们想,这个主人实际上是想问什
么?我很头大,想不出来,四大姐妹中一员举手答对了,主人是想问“你这么晚还没休
息,是不是在厨房偷吃东西”。相反,我很喜欢数学,因为它们的答案不是对就是错,
我可以确信地知道我是对了还是错了(虽然后来大学学了门混沌的入门课,从图形上感
觉可以也有对错之间的中间层)。那时数学考试常常会问,一个边长多少的正方形,在
它里面画一个最大的圆,请问圆的面积是多少,后来那个比例,正方形里最大圆的面积
是正方形面积的pai(3.1415926)/4 是我发现告诉老师的。

    四大姐妹都是语文好,语文老师很喜欢她们;因为我数学好,数学老师很喜欢我。
以到于四五年级时有一次镇上教育组织有人来学校检查一个什么宣传之类的访谈,当时
数学老师在给我们代课,他想都没有想就派我和一个男生去了。那个男生是从四川转学
过来的,比我们稍大一两岁。这个数学老师就是小学里我们所有人练书法字的鼻祖,我
有画过全班得分最高的池塘荷叶画,但没练会书法字,但这个男生练好了,写得一笔好
字。数学老师对我们俩的亲睐可见一斑。但接下来,我就干了一件看上去极其完美,却
十分错误的事。

镇里教育组织里来的检查一个什么宣传之类的活动访谈,说实在话,我们学校也没有做什么像样的宣传,我基本想不出任何的活动来,那个男生可能因为大些成熟懂事,所以谨言慎行,不多说话,所以就剩下我,我办法,我只能使劲浑身解数地编啊,天南地北地编,但得编得团圆,不能被他问出破绽。他是一个人来检查,我们两个人应对,我卡壳的时候那个男生会帮我说些什么,办公室里只有另外一个看起来很严谨的语文老师,这个老师后来成为了我六年级的语文老师(有了那个语文老师之后我的日子就直接下地狱了)。------总之,就是在这一个小时的访谈里,90\%的情况我在说话,而我说的话里,99\%都是编的。

后来数学老师告诉我,说我编得很好。但是他们没有任何人纠正我,做人需要诚实。没有任何人,我爸妈都并不知道学校里有发生这么一件事,而我自己,当时,以后很长一段时间里,也都没有意识到这件事情会对我造成多少恶劣的影响。今天回忆起来,这是我从小到大第一次撒下的大谎。虽然当时没有伤害到任何人,但多年以后(六年以后),这个没有被纠正的性格缺点最终还是伤害了我自己。
\section{我眼中之大姐谈恋爱}
\label{sec-2-3}

        小学的我对大姐基本是没有什么记忆的,可能当时她已经上了初中,初中毕业后又
了襄樊市哪里去做事。只记得小学有一阵儿我也很臭美,一心想要留头发,留成一个小
小的马尾马,这样我就可以梳妆打扮自己。心发还没留长,有一段时间,我天天盼着大
姐回家,姐姐一回家,我就向她讨头饰,要她去城里的时候帮我带一个什么什么样的发
夹好不好。而往往的结果是,姐姐忙,没能给我带。这样好几次之后,姐姐发火说,“
你头发没长两根,为什么天天要头饰?有那点儿心思就不能好好学习,把学习弄成个第
一名什么的?”尝到了姐姐的厉害,以后就不敢再问她要什么了。姐姐的说法也体现了
父母在我们要求面前一贯的做法:只从学习上鼓励,其实方面绝不盲目顺从。秋到的道
场早被爸整理得非常平整,农忙的间隙我和三姐就拿爸爸的28永久牌自行车练习学骑自
行车。爸爸的自行车和那里邮递员的自行车有得一拼,又老又笨重,我和姐姐常常在爸
妈面前絮叨,“爹,你什么时候给我和三姐买辆新的26的不带前面横杠的小自行车,好
不好?”那时妈妈总是抢爸爸一步(可能知道爸爸一贯宠我们)一口咬定地说,“那不行
,爸妈辛辛苦苦劳动的钱,只够供应你们上学读书,没有多余的钱买自行车”。不过为
了鼓励我们,妈妈也会退一步地安慰我们,“不过嘛,如果你们像表姐一样考上中专或
者大学了,爸妈高兴,省吃俭用也一定给你们买自行车”。虽然这种话听起来败兴,但
在我们心里,至少在我心里树立了一份永久的愿望:希望有一天我能够考上中专或者大
学,能让爸妈高兴,能够拥有一辆属于自己的自行车。

扯远了,还是先说小学眼中我看大姐谈恋爱吧。

        前面说过了,大姐三姐像爸,瘦并且很漂亮。加上姐姐出社会早,初中毕业没再继
续读书就出去做事了,差不多十五六岁就踏入社会。因为长相条件好,出去做事又有机
会接触人,家庭条件又好(爸妈因为我们姐妹四个都是女孩儿家,没有个男孩儿,没体
力做农活,也怕被农村其它有儿子的家庭瞧不起,所以发奋劳作,在我五六岁的时候家
里就早早地盖起了上下四间两层楼高的八间房,希望我们长大了不会被人瞧不起),姐
姐就东挑西选,早早地谈起了恋爱。姐姐谈恋爱的一贯做法是,先把一个陌生男孩领回
家给爸妈看,两人骑同一辆自行车回来,暂且称他们哥哥吧,一个陌生哥哥骑自行车把
姐姐载回家,吃一餐饭,哥哥再把姐姐载走。下一次姐姐回家就一个人骑自己的小行车
回来,就会问妈妈爸妈觉得这个朋友怎么样,同妈妈交换一些必要的信息,妈妈说好,
就继续,妈妈说哪里哪里不好,同姐姐讲道理。姐姐听懂了听明白了。过段时间,就会
换个朋友载姐姐回家给爸妈看。爸妈觉得姐姐条件好,爸妈为她创造的外在条件也好,
有的是时间和机会好好挑选,就这样,几年时间下来,姐姐领回家不下二十个我们姐妹
唤作“哥哥”的陌生朋友。

        不过比较搞笑的是,几年时间下来,姐姐的男朋友又换回成了她的第一个领回家的
在镇上软木钻厂工作过的朋友。追姐姐追得可紧了,姐姐早不理他了,因为他家不肯让
他作我们家的上门女婿,又没机会见面,三姐那会儿上初中,听说哥哥好多次拦住周六
放学回家的三姐,让她给大姐带封信,要大姐亲启。后来这个哥哥成了我大姐夫。一次
我和三姐割完喂牛吃的青草,背着背笼走在回家的路上下大雨了,听大姐说哥哥远远地
看见我们小姊妹俩个背着背笼,便匆匆地跑去接我们,帮我们背。其实他们结婚时作不
作我们家上门女婿似乎也没有说定,可能是哥哥的做法最终感动了姐姐吧。至于姐姐同
爸妈之间有没有过什么激烈的对抗,我那时小,不知道也不懂得问什么的,只是后来听
说姐姐第一次同他分手,他连喝了九十颗安眼药(那时的我觉得姐夫喝药很懦弱,男子
汉大丈夫干这种事太不值得了),后来再怎么被救醒的就不知道了。

        姐姐的恋爱并没有对我造成男女情爱方面的任何影响,但姐姐领朋友回家的那几年
却还是给我造成了心理上的阴影。因为姐姐姐姐漂亮,又过于听从爸妈的话,爸妈很喜
欢姐姐。加上姐姐领朋友回家,家里待客人的都是好菜好饭,因为姐姐回家爸妈的盛情
款待,我开始有了心理上的阴影。我小的时候是爸妈宠我,爸爸尤其宠爱我;可是等我
长大了,我还是那么听话,爸妈却更爱大姐了,好吃的都等姐姐回来才吃,好穿好看的
都留给了姐姐穿,而我这个老幺,就只能沾着姐姐的光吃点儿好吃的,捡着姐姐们不再
穿的破烂吊吊穿一下保暖而已。但是这个想法我憋在心里从来没说,觉得爸妈还是喜欢
我的,只是不像从前那么喜欢了,他们现在更喜欢大姐而已;另外两个姐姐对我也还好
,爸妈辛辛苦苦劳动挣来的钱他们自己也没有享受,而是家里盖房供我们姐妹读书了。
所以虽然要穿捡得姐姐们的吊吊穿,也没有太多直接呈现。
\section{记忆中的第一次委屈}
\label{sec-2-4}

        快要跳过小学这个快乐的篇章了,才想起长大后受过多少次的委屈,我只能哑
巴吃黄莲,苦只能往自己口中咽,小学时也发生过一件。

        可能是因为家中姐妹众多,而爸妈要管理庄稼、田地,辛苦劳作挣钱养家好让
我们能够接受教育脱离农村苦海。自打我记事起,爸妈都会分配我们作一些力所能及的
活儿。比如最大的姐姐被安排过照看我们妹妹,大姐二姐农忙季节时都被安排过做饭,
而家里地里大大小小能做的活,爸妈也都会安排给我们,希望我们能够体会他们劳作的
艰辛,也好激励我们好好学习。我小的时候冒傻的事情很多,比如插秧时我和姐姐打秧
行,别人右手食指中指两个指头轻轻一点就把秧苗送进水田软泥中,我右手握着拳头往
泥里送,后来又被姐姐嘈笑纠正;爸爸出门了,晚上我们害怕,会把门插上。爸爸走之
前早早地交待我这个乖乖女,晚上爸爸叫门时要起来给爸开门。大冬天的晚上,我和三
姐睡一张床,盖一床被,我也早早地把火柴放在棉衣口带里,可是爸爸叫门时我左掏右
掏多少个口袋都掏过了就是掏不到火柴,没办法就只好黑灯瞎火地披着棉衣摸索着出去
给我爸开门。可是没要姐姐起来开门不说,她还拽着我的棉袄不许我出去给爸开门,我
急了跟我爸告状,我胖啊壮啊,姐姐挣不过我,我就去披着棉袄去给爸开门了。第二天
早上姐姐告诉我们大家,冷得要命那个胖XX披着背子去开门,她被冷飕飕地晾在床上。

        有一个夏天中午,妈妈要求我和姐姐吃完中饭后一人扯完一畦干草籽再去上学
。草籽是豆科植物,苗期把它们耕翻在地里可以欧淝,长老了干枯了桔杆也可以用来粉
糠喂猪。妈妈分派的正是要我们把干掉的桔杆扯断堆起来,这样改天可以拖了拉去有机
器的家里粉糠喂猪。一畦的地为什么有这么宽这么长啊,好不容易干完了,姐姐的一畦
还有好多,姐姐瘦弱干不动,我一定得帮姐姐的,就这样我们俩个干完来到学校,别人
午休都快结束了。

        来到学校我才知道,因为镇里又有什么破烂组织要来检查学生作业了,我身体
好没生病没怎么请过假,老师要我们好几个学习比较好、很少请假矿课的同学加作两三
次作业好应付检查。值日同学告诉我,我需要做哪些哪些作业的时候,平时的作业对我
来说从来都不算什么,可是今天我好累,偏偏今天要加三次作业,那份委屈一下子就化
作眼泪掉了下来。正好检查午休的老师经过,看见我哭,我没说什么,她以为是因为我
被回作业,就告诉我让我加只是因为我作业完整,并不是因为我字写得好看(我的字,
方格本子,一个字只能占方格下面中间的1/6,太小了)。于是她就又自作主张地把我的
作业本交给四大姐妹中的一个,让她把三次作业做到我的相对比较完整的作业本上。因
为我一直哭没说什么,猜测到老师认为我懒,就更委屈,哭得更伤心了。

        那是我第一次当着那么多同班同学的面、委屈地哭,后来成长过程中,因为逆
势比较多吧,受过很多的委屈,也哭过好多次,人前的或人后的。属相上说,属羊的孩
子有很多的泪水,因为他们心地善良。不管别人相不相信,我自己相信这一点。
\section{委靡凋谢的青少年(1)}
\label{sec-2-5}

        正如我被人肉后大家所猜测到的,童年青少年时期,确切地说小学毕业的暑假
,是有一件事,直接影响到了我初中高中六年的生活,甚至严重一点儿说,如果没有这
件事,我的人生可能完全会成为另外一种样子。大家的猜测是对的,这件事与性骚扰/
伤害有关。我承认这是一次意外,但又不是传统意义上的性侵犯。鉴于我还是一位未婚
女子,尚处在找男朋友谈婚论嫁的年龄,敬请大家不要过于纠结事情的细节,请大家原
谅我对这件事情的保留,能够思考一下这种事情的借鉴意义,也算是我的这个故事终究
发挥了它的正面意义。

事情发生之后我害怕极了,但最近几年大姐谈恋爱从爸妈的反应来看,我觉得爸妈偏心,他们喜欢大姐,不喜欢我,我不愿意把这样一件事情告诉他们。那时候,没有觉得爸妈是自己最值得信任的人,所以我的事情也并不愿意分享。现在回过头看,什么叫害怕?害怕就是一个十三岁左右的小孩晚上辗转反侧,不能入眠,最消极的一次是一个晚上整晚没能睡着。后来我想,我要把这件事情放下来,不要再去想它,但是我做不到。就这样,青春发育期,我越来越多地注意到自己的身体发育,本能地觉得我和别人不一样,我的身体发育也同别人不一样,心里慢慢沉淀了自卑。再加上一些封建的余毒,妈妈只上过两年小学,不懂科学,也比较迷信,无形中潜艇默化地(加上我又发生了这么一件事)为我加上了沉重的心理负担。虽然爸爸一点儿都不迷信,是彻头彻尾的无神论者,但是他不知道我身上有发生这么一件事。就这样,我开始变得绝望。

有一次初一的好朋友提起与之相关的什么事来,我告诉她,女生上厕所的时候被男生看见会怀孕。她不信。她当时表达了不信之意,但并没有完全反搏我。后来她回家问她妈妈告诉我那不可能。她说她也不知道什么情况会怀孕,但是我说的那种情况绝不可能。我很想同她多说点儿什么,但却她起疑,就打住了。

好在,初中的功课并不是很难,我的学习上因为这件事情分神,但并没有到什么不可救药的地步。初中三年考得最差是初一刚进校门时的月考还是期中考试,全镇两所初中,按横穿小镇的铁路线划分,我在新建初中(我们那一级是第二级,也就是我刚上初中那年只有初二学生,初三学生还没有)考试上了年级四十多位。爸妈觉得我考得很不理想,加上我也上初中了,学习上就该重视起来,于是以前每个周末对二姐三姐的磨牙功夫就用到了我头上。那时我也还算听话,我知道学习能够改变我的命运,它已经改变了二姐的命运,二姐考上了中专成了金凤凰,她以后都不用在苦海农村里种田干活了,所以我也用心学习,这样努力努力很快地成绩就稳在年级前十名,至少也在二十名之内。

那个张老师家的小女儿和我同一所初中,但初一在不同的班,后来初二还是初三重新分班我忘记了(可能是初二结束之后吧,又觉得在后一个班里时间长,那就是初一结束分的了),反正我被分到了新的班里。就同张老师家的小女儿又在同一个班了,但是我们已经生分了。她觉得我学习好,老师喜欢我不喜欢她,她不跟我玩儿了。我也没有刻意挽回什么,而是结识了一位新朋友。这个朋友就是让我起死回生、把我安全送进重点高中的一生中最重要的两个朋友之一。
\section{一生中最好的朋友}
\label{sec-2-6}
\subsection{一生中最好的朋友(1)}
\label{sec-2-6-1}
        其实如果我没有记错的话,我们是初一结束就重新分班了,也就是初二我就在
新的班级,有机会结识新的朋友。但是记忆中又好像只有初三的记忆,或许初二被我选
择性忘记了?

        初三时我们班里来了个帅气的小帅哥复读生,不过他大概只能考襄樊市重点高
中,好像我们本县城里不招收复读生。因为他复读,学习比我们好。我不知道初中化学
算不算是数学的什么姊妹学科,反正是一进初三一次什么考试,还不到期中考试,不到
月考之类的,化学单科考,我错了一个填空,得了98分全年级最高分(整个年级的化学
老师只有一位,垄断啊);小帅哥错了两个选择,得了96年级第二;其它班上再以下的
成绩就不知道了,但我们班上其它所有人的分数都低于90。这就成了我初三的第一个神
话,学校不大,初三年级的老师很快都知道了这事。化学是初三新开的科目,大概这次
考试之后,那个漂亮的孔雀女就慢慢成为我初三最要好的朋友。

        孔雀女一家(她和哥哥妈妈)是从襄樊市里搬来的,她妈妈在镇上工作。我们大
部分学生是农村的孩子,每周都从家里带菜来吃,但她妈妈要求她从食堂打新鲜菜,她
哥哥没事的时候吃饭时间还专门为她送来午餐晚餐什么的。不过她对我很要好,有时候
她看我吃得带的菜太少,她一打新鲜蔬菜就会分一些给我。而我在吃的方面常常没有什
么可以分享给她的。但学习上我们常常交流学习方法什么的。

        她长得漂亮,皮肤尤其的白,家庭条件好,衣服穿得很好看,学习成绩又好,
大概是班上很多男生暗自喜欢的对象。但没有任何的优越感,没有傲气,知识面很广。
感觉她很喜欢文科,看过很多的书,慢慢和她了解后就越来越信任她。那折磨了我两年
的心事,我终于第一次向这位朋友敞开了心扉,虽然给她讲的是一个我编的自动降级版
的(我告诉她说我很小很小的时候,上小学前遭遇过一次性侵犯)故事(看吧,小撒谎还
在继续,虽然只是为了减轻自己的尴尬)。她立即告诉我,这事不算什么,很多人小时
候都发生过这种事情,蓍名的人物里也有;她告诉我,这事不会对我造成任何的影响,
我应该忘掉烦恼,开开心心地生活。我非常相信她,因为她看过很多书,我课外读物儿
童文学什么的不曾读过一本。

        就这样在她的鼓励下,我慢慢变得快乐起来,学习成绩也越来越好,后来小考
小闹的搞了个门门课都得了班上第一(忘了语文有没有除外)。班主任告诉我,我们分班
的时候她不惜一切代价一定要把我选到她班上(我猜我二姐同她们一样也是铁饭碗就在
镇上医院工作她们也知道,有一个语文老师还是我二姐的初中同学)。就像我现在所回
忆的,同此孔雀女的友情是我一生中最为真挚的友谊。她就像是春雨及时雨,默默地滋
润着我两年来枯竭干涸的心灵。我有任何的想法包袱都会告诉她,她就一定会鼓励我。
她也借书给我看,是一本撒切儿夫人传,感觉她喜欢文科,将来早晚是当官的料。看了
这本传记后,以后我对人物传记就有一种特殊的感情,非常喜欢,是个什么不错不一般
的人的传记,我都想找来看一看。

        我从小到大唯一一次带朋友回家玩就是初三寒假的时候请她到我家玩(小学时
候同村里的伙伴在我家吃饭不算)。她不会骑自行车,这让我很奇妙了一段时间,因为
不理解这世界上居然还有不会骑自行车的大小孩,后来还是理解了。记得我早早地告诉
爸妈明天我要带一个好朋友来家里吃餐中饭,想请她来我们家转转。一大早我就骑车去
她家,然后把车放她家里,我同她一起一路聊天走七八里路到我家,在我家楼上楼下看
,我的房间转转,去我们农田里草埂上走一走。中午在我家吃饭后,下午我们再一起走
回她家,我最后骑爸爸的自行车回家。记得冬天里没有特别好的菜,除了菜园里的绿叶
菜外,妈妈把作年货用的鱼煎了一条大的款待我朋友。我爸很怜惜我这个长得天仙般的
朋友,让她吃鱼吃鱼再吃鱼,妈妈也对我这个长得武大山粗实在不像个娃娃的娃娃感到
很无奈。
\subsection{一生中最好的朋友(2)}
\label{sec-2-6-2}

        就这样我的学习越来越好,期中考试是年级第一;妈妈说期中考试很多同学可
能没有足够重视,要期末考试也考第一、升学考试也考第一才叫厉害。于是我期末又考
了第一,慢慢地,我忘了烦恼,忘了自卑,对学习变得非常自信,不过可能还是不喜欢
语文。以至于后来上课的时候,老师喜欢在台上讲说,“你们谁能保证你就一定能考上
中专高中?”意思是说没人能保证,所以要好好学习。但每次任何老师在台上讲,我就
在自己心里小声嘀咕,“我就相信我一定能考上高中!”但我不骄傲,嘀咕完了继续好
好学习。

        那份自信是真正的自信,后来我们那一级加试体育。我是学校里唯一一个获得
县里张自忠三好学生初三年度200元奖学金的学生。所谓三好众所周知,德、智、体。
前两样初三的我还行,可是体育就太不沾边了吧?可见学校里老师们对我的偏爱。真正
中考要考体育的时候,30分满分,我们班有男生拿满分的,我的好朋友孔雀女得了17分
,我只得了11分,估计是全县体育最低分。不过虽然只得了11分,我没有半点儿紧张和
不安,实际上是跟本没把它当回事儿,连家长都没有告诉,因为我觉得就算我是体育0
分光凭文化课我也一定能考上高中。后来,初三年级所有的大考我一直是年级第一,中
考也是,582,全镇文化课第一名;第二名是好朋友孔雀女,578。我俩都很自信,我对
外宣称我是全镇第一,因为文化课第一,高考不考体育;我的朋友对外宣称她是全镇第
一,因为她全面发展。

        请大家宽容一下我对初三成绩的炫耀吧,因为只有初三时我有最好的朋友这颗
救星,后来高中就没有了。初三是我学生生涯最光彩夺目的一年,以后的学习中在学习
上的自信都最早从这一年形成。有一个全国范围内可比性的成绩就是数理化奥林匹克竞
赛。我们学校我同那个体育满分的男生进入了化学复赛,同另外一男生进入了数学复赛。

        因为复赛要到襄樊去考,我就有了理直气壮的资本问妈妈要新衣服,因为我没
像样的衣服穿。妈妈很理解,因为初三年级我的成绩突飞猛进,她们很多事情就开始顺
从我。她一农民家的不知道现在小姑娘们都穿什么样的,就让大姐给我买衣服。姐姐给
我买了件粉红色的上衣西装外套,直接送到我学校让我试穿看合不合身,很合身我也很
喜欢。不过我有一个担心,太久没穿新衣服了我怕我会因为这件衣服而沾沾自喜,考试
时不能发挥我的最高水平。于是为了我的复赛成绩着想,我可是把这件新衣服狠狠地穿
了两三个星期适应了才穿去襄樊参加复试的。后来镇上两所初中只有我化学复赛得了湖
北省三等奖,其它所有的复赛都没有出线。

        我说过我不是聪明小孩,后来高考、大学英语四六级考试、考研、TOEFL、GRE
考前我都是很紧张,常常考试第一天的前一天晚上午夜都要等到两三点钟才能入睡。但
中考对我来说实在是没有紧张,最最正常水平的发挥。统计后来成为我来美国后的专业
,提一下数学单科,那时小学升初中99分,错一填空;考中升高中是128.5,A卷错一填
空(扣三分,但A卷除以2)。中考数学还是很高以至于高一时班主任根据中考成绩选我当
数学科代表。
\section{初三它解}
\label{sec-2-7}

        是的,初三我是单纯快乐的,没有任何思想包袱;初三学习非常好也是事实。
但这学习好事实的背后,隐藏的是另一个问题:我不会想问题,确切地说,从来不想生
活上学习以外的问题。事实是,我每天24小时除了吃饭,睡觉之外,我想的所有的事情
都是学习上的事情,或者就我作了一年学习机器。我家小猪每天吃了睡睡了吃,不想任
何事情,我觉得自己那一年就像一头快乐的猪,学习累了吃睡,吃饱休息好就学习,我
的生活里再没有其它。但就是因为初三生活,我变得很喜欢猪。后来上硕士时流行<两
只蝴蝶>、<猪之歌>的时候,听猪之歌就成为我的一大爱好。不过好在只有一年,如果
一个人整个学生生涯都像我初三那样,我估计又一个学习上的天才,生活中的傻子诞生
了。不过也还好,我没有那么天才,也没有那么傻,只是有那么点学习天才生活傻子的
味道。

        我们化学老师的风格是,在讲台上给我们念题目听,然后给我们一定的思考时
间,再然后就要我们举手回答他刚才念出来的问题。有一次可能没有其它人举手吧,老
师就直接把我叫了起来要答案,那我就说了。只可惜啊,我是知道计算方法的,但因为
一个步骤的马乎结果计算错了。当时就没觉得有什么,就想着以后再做这些计算的时候
要小心就行了。然后这事我就忘了。可是偏偏下课数学老师还是谁要我去办公室拿同学
们的作业本。进往办公室,就听见化学老师在绘声绘色地描述我举手了却答错了。而我
一进到办公室里,化学老师的话就嘎然而止了。知道结果怎么样么?我傻呀,我根本就
不去想学习之外的事情,化学老师在办公室同其它老师的谈论对我没有任何伤害,完全
就当它是过眼云烟,转头就忘。

        其实爸妈原本是想让我读中专,但95年那会儿我80\%的同学都上高中,我也想
上。爸妈想让我上中专是有点儿稍微自私点儿想上中专三年就结束了,但上大学又得他
们多少年的辛苦劳作。但是姐姐们回家作了爸妈的思想工作,说家里就只有我一人念书
了,即使将来哪天父母实在没有能力供应了,姐姐们不会对我不管不顾的。就这样,作
为家里的老幺我有了上高中的机会。

        但我想上高中最根本的原因是因为我的数学老师的意见,她同时也是我们班主
任老师。她希望我的孔雀女朋友上襄樊市重点高中,希望我能上县城高中。我对这位老
师有一种深入骨髓的信赖。她的老家在临镇上,家里只有她和弟弟两姊妹,上的中专师
范,毕业后来我们学校执教。属相是什么我不知道,但同我一样,狮子女。我们两个凤
凰女(虽然当时我不是)有着本能的深入灵魂的相通。她很照顾弟弟,到我们学校执教后
,她把弟弟转到我们初中来读书,同时照顾他的学习和生活;可能是因为对我的信任吧
,她把弟弟安排与我同桌。据我们小道消息,初一朋友告诉我她们老班喜欢我们老班,
想追我们老班但我们老班看不上他们老班。他们老班个儿高1米8,瘦,帅,我们老班不
到1米6,胖点儿,不漂亮,所以我初一朋友他们就很悲愤。但是后来就知道我们老班早
心有所属。是她一初中同学,对她非常好,当时是在武汉打工,农村户口可能是。但就
是这样最终她们结婚在一起生活。我非常敬重这位老师,也常感叹平常百姓家的爱情也
可以如此平淡而幸福。

        初三上学期的时候,因为临近过年,农村里那时候基本上还家家户户都杀猪的
,所以也间接形成了一种风气就是,很多学生请代课老师们周六晚上去他们家里吃饭。
可能因为学习太好吧,或者我前两有二姐这个铁饭碗光环的笼罩,我觉得我不需要,就
没有让家里破费。可是不过呢,等初三结束了,忽然觉得这一年自己过得实在是太幸福
快乐了,是久违了的小学生活的感觉,学习、朋友、生活各方面都很满足。真的是很感
激老师们对我的厚爱,于是就回家同爸妈商量,想请他们暑假来我们家吃餐饭。爸妈欣
然同意,让在镇上医院工作的二姐二姐夫到那天回家陪客,妈妈也没忘交待我把食堂里
有个叫大勇的哥哥也叫上一起来我们家。于是我就回学校同化学老师说了。那天走进办
公室找化学老师,有一位很眼熟却不认识的大哥也在,我就问化学老师有没有见到那个
大勇哥哥?办公室里他们两人同时一阵大笑,食堂里天天给我们打饭打菜的大勇哥哥我
居然不认识!够傻吧,但那就是我的初三生活。

\chapter{我的高中}
\label{sec-3}
\section{我的高中(1)}
\label{sec-3-1}

        当年是爸爸和大哥(大姐夫)骑两把自行车载米载被子把我送进初中报名的,爸
爸说报名的人太多很挤,感觉他的肋骨都快被人挤断了;可高中报名的时候,虽然我已
早早地来到县城住在大哥大姐家,但报名当天,爸爸来是特意赶来送我入学。分班是按
中考成绩分十个班,我我名字排在高一(8)班第二名。孔雀女朋友分到了其它的班,我
们也自然而然地生分了。

        可能大家忘了,我也快忘了,小学毕业那个暑假的不定时炸弹又开始在心里隐
隐作痛,我再一次地慢慢滑向低谷。高中我的学习很一般,最好成绩只有一次排在年级
四十多名,一般可能在一百名左右徘徊。政治课上我们背“资本主义社会的经济危机是
不可避免的,每隔一段时间就会发作一次”,我心里的那个隐痛就像资本主义社会的经
济危机一样,时常发作。遇见朋友退去了的自卑在我学习生活中再次形成,我常常哭,
基本见不着笑。考试考得不好会哭,有时晚上回家休息也会哭。虽然我走读住大姐家,
但我晚上休息的地方离姐姐休息的地方远,即便我晚上休息有时会哭,她也不知道;我
就只一日三餐匆匆忙忙上她家去吃饭,吃完就去上学,基本与她0交流。因为高中走读
,与女同学们接触机会也不多,没有机会交到好朋友,就这样我高中基本没有好朋友,
只是同自己的同桌稍微熟一些。就这样,我的生活压抑、沉默极了,我需要一个像初三
的朋友那样能够帮助我的人,解救我脱离苦海。我还真遇见一个。

        高二的暑假,忘了是什么原因我到远亲大舅舅、舅母家玩,撞见了一个陌生人
,舅母告诉我这是回国探亲的二舅(大舅同父同母的弟弟),这可是个大人物,是我崇拜
的偶像。我早就听妈妈讲起过,大舅家大表姐上高中考上了大学,小表姐高中的时候就
被二舅带到美国去,也是学习非常发奋,现在在美国过得很幸福。传说中的二舅啊,我
今天可算是见着真人版的了。

        或许是因为妈妈对于他们有交化受过教育不用辛苦劳作家庭的崇拜,或许我对
妈妈所描述的小表姐在美国的幸福生活我向往,当我亲眼见着这么一个大人物的时候,
我本能地“冲”了上去,我希望这个偶像级人物能够鼓励鼓励我。我请二舅陪我单独出
去走一会儿。虽然是舅舅,虽然是偶像,对于小学毕业那件事,我还是守口如瓶,只是
编了一个有点儿早恋味道的什么故事(具体细节我都忘了,97年,十四年了)。二舅真的
安慰鼓励了我,他拿胳膊搂了搂、拍了拍我的肩头,告诉我这只是因为我生长的环境太
小,我的世界太小,所以让我觉得这是个事儿,真正在大环境中,在美国,据他所知,
这种事情很普通,没有什么的,让我不要在意,好好学习。回到大舅家,二舅还上楼去
特意拿来一小袋巧克力送给我以示鼓励。

        那是我在国内第一次见本文题目所指的舅舅。偶像的魅力还是巨大的,我颇为
炫耀地把巧克力拿到学校,分给班上我的女同学们一起吃,笑容也从此再一次回到我的
脸上。我的生活开始变得有盼望、有希望。而且,我对舅舅口中那个那种事情不算个事
的美国有了一种神奇的向往。
\section{我的高中(2)}
\label{sec-3-2}

        忘记了是什么时候老班把我们班的小美女(大概1 米55吧)调到与我同桌,小巧
玲珑型的,娇滴滴的,可能过于注重她自己的外貌,我那会儿不怎么喜欢;元旦的时候
,课桌里收到一张卡片,是班上一个男生放的,不过也没多想什么。再后来到下学期五
月份前后的时候,我意识到一件很意外的事情:大家说我早恋,我有吗?但我心里清楚
地知道,真正打倒我的,是因为我撒谎了。在知道那年夏天(98年)舅舅又回到大舅家度
假之后,我不知道哪根筋搭错了,对班主任老师说舅舅在给我办手续要带我去美国。请
大家原谅我那一次吧,因为之后,这个缺点就改正了。而我,为那一次的错误也付出了
沉重的代价。

五月份,认识到自己错了时,我崩溃了。

        以前我都是匆匆忙忙连走带跑地去姐姐家吃饭,吃完饭就去学校学习,但是,
现在,我不想去学校了,也不想动了,我只想静静地躺在床上休息。我睡着了吗?没有
。我只是躺在床上这个世界上最安全的地方去想一些问题。想什么呢?想我错了,我不
应该撒谎,做人要诚实;想我为什么撒谎了,因为我太虚荣;想我为什么虚荣,因为我
自卑;想我为什么自卑,因为我觉得自己有生理缺陷;想我为什么会有生理缺陷,因为
我小学毕业发生了一次意外;而那种意外为什么会发生在我身上?因为我笨;那你还有
什么缺点?个子矮,长得胖,眼睛小,单眼皮,鼻梁低,遗传的全是爸妈身上的缺点--
----那你有没有什么优点,没有。那么,活着的希望/盼望/意义是什么?没有希望,没
有盼望,没有意义。你考得上大学吗?考不上。为什么考不上?因为算命先生说反话,
他说我是个拿笔杆的,实际上是说我是拿锄头杆、镰刀杆的;那你为什么要相信算命先
生的话?因为他算得准;有什么能证明他算得准?因为他说过什么什么就发生了什么什
么------就这样,我一天24小时地躺着想,却想不出一个未来。

        我出事之后,大姐大哥把我领回家,交到爸妈手上,看爸妈决定怎么办。爸爸
下令说,家里的一切事情你们都不用管,要妈妈陪我们一起回县城、要我回学校上学,
要妈妈把我看着。就这样我又重新回到了学校。我的思考并没有因为妈妈的到来而结束
。这一次,到这种情况下,我终于一个人撑不住了,所有发生过的事情、现在心里的想
法统统向妈妈、姐姐们交待清楚。学医的二姐告诉我,人只有在三种情况下不能怀孕:
精子存活率过低;精子卵子不能结合成受精卵;受精卵不能成功着陆,并一一向我解释
清楚;姐夫向我举例说明算命先生的话可以有多种理解,他们是见风使舵的主儿。二姐
也从客观事实和科学的角度向我解释了妈妈曾经说过的谁谁谁不能怀孕是领养的孩子,
后来又能生子的原因。妈妈也找到了姨父问那次算命到底是怎么回事;他们尽了他们能
尽的一切努力想要说服我,但我实在是太绝望了。

        姐姐说,学校是菜园子门啊,你想上学就上学,想不上学就不上学?但后来它
真的就变成了我这里的菜园子门。因为绝望,因为确信自己一定考不上大学,我不要再
去上什么学!但过了一两天,最终我还是去上学了,是被老妈使用武力逼的。妈妈说,
“你今天上午要是再不去学校,我就是打死你我去坐牢我也要把你打死先,你这么恨人
,我今天不打死你我不解恨!”在精神痛苦和肉体痛苦面前,我以为精神上的痛苦远胜
于肉体痛苦,可能是懦弱吧,老妈一顿狠打把我打回了学校。

        我多么希望学校里老师们同学们能好好鼓励鼓励我,但是基本没有。英语老师
说了一句,“每个人都会犯错误,知错就改便是好学生。人要学会坚强,如果首先自己
就把自己打倒了,那就没有别人能够扶得起你。”班长在我书桌里放了一封信,里面写
得也都是鼓励我的话。但那会儿我太绝望了,我已经先把自己打倒了。

        我思想走极端并没有因为回到学校而停止,相反它还在进一步地滑坡,我知道
自己一无是处,我想到了死,但没有死的勇气,就算我没本事作了叫化子,也还是想看
看这个世界;但我又不想连累家人,所以我想走,离家出走,隔舍掉所有亲情,走后自
生自灭,永不再回来。但我没钱,我要先攒钱。在纷纷扰扰痛打落水狗的外界舆论下,
家人为我筑起了一座玻璃温室,妈妈、大姐一家、三姐三姐夫、姨和姨父等都会在我回
家中午晚上吃饭的时间在家吃饭、打麻将,其乐融融。他们没有人追问我学习的情况,
只要我上学就好。不知道他们怎么发现我的小算盘的,他们所有人赢的钱都给我。就在
我攒够了到武汉的公汽钱,早上望了妈妈最后一眼,准备离家出走的中午,妈妈和姐姐
家的孩子已经在教室外等我了,我没走成,有点儿感动。

        就这样,我最终还是上了高考的考场。考试完我就回到了老家,爸妈不让我作
什么事情,只让我休息。考完相当长一段时间里,我还在继续进行着我的思考。姐姐姐
夫们偶尔回来看看爸妈看看我,但他们没有任何人追问我的成绩、想法什么的,可能是
怕给我造成压力,不想刺激我,希望我在爸妈身边能够好好静养,也把想要去思考的问
题好好一次想个清楚。终于是八月的一天,爸从农田忙完回来,我对我爸说,如果我这
次我没能考上大学,我想重新复读一年,争取明年能考上大学。爸爸看着我的眼睛湿润
了,说了一个字,“好!”
\section{我的高中(3)}
\label{sec-3-3}

  不得不承认,这几天所讲的故事都早已是陈年往事,但要把它们真正写出来却并不
容易。原来回忆也是一件这么痛苦的事情,会影响情绪、影响休息。原本周六要加班的
,却因为我周五晚休息不好不得不临时向老板请假,请其它同事代班。有限的这几篇里
,写到姐姐恋爱造成的心理阴影、没新衣服穿都忍不住掉泪,而昨天写到自己一生中最
痛苦的精神创伤,实在难以自持,哭了好几通。

  一个人的一生中,或者总有那么一两次让你过不去的槛,你需要停下来去想,等你
想通了,你以后一辈子都是坦途了,因为,经历了这两次,已经没有什么可以把你轻易
打垮。

  在思想游走、昏昏乎乎的那几个月里,因为自己的意志垮掉,我的同桌小美女也成
为我欣赏的对象,因为在高考压力面前她的意志没有垮掉,我对她比以前要好很多。

  后来上了大学,又经过一段时间的灾后重建,我开始有了大病初逾般的脆弱清新和
芦笋雨过天晴朝气勃发的生命力。我开始想问题,对生活中的事情常常能有自己独立思
考而来的观点。初中高中六年里的心里疙瘩我并没能彻底忘掉,但我学会了放下。那三
四个月的精神梦游让我想明白了一个道理:人生是一个过程,生命的意义大概就在于为
向往的生活付出踏踏实实的努力。即使生活中有缺憾,我还有其它事情可以做,可以让
自己变得充实,可以让自己的人生变得有意义。

  后来,除了有过一次情感上的轻微创伤,很多的艰险考验我都一步一步走了过去。

  其实我最开始也不理解为什么说农村学习条件艰苦,但经历了高考那次我就明白了
。如果没有家人及时发现,或者今天的我就变成了另一个“犀利哥”或是“憨厚姐”。
之前我对亲情的理解很平淡,就像当初我没有认为父母应该是最值得我信任的人。但这
件事情之后,我原谅了父母的偏心,从心底诚挚地包容了我成长过程中他们做得不足的
地方。亲情在我心里开始转变成一种精神支柱,一种在任何困难面前支撑我坚强、勇敢
地走下去的力量。

  但我有过一点疑惑,爸妈没有受过高等教育,在我最困难的时候,他们怎么可以做
到那么好?我怀疑他们这方面的能力。后来大哥告诉我,是因为我的班主任老师,他因
为我的精神状态跟到姐姐家同大哥聊过很多次。而这些,当初的我并不知道,得知这件
事的时候心里充满了感激,而当初我还对他有些误会。

  事后得知的另一件事让我难受了很久。上大学后同以前的老同桌联系了一下,她告
诉我当年我们班有一个学生没有参加高考。我很差异,连我这样的脑震荡型的都参加高
考了,还有什么人会没有参加高考?她告诉了我那个男生的名字,是当初大家认为喜欢
我的那个男生。考前老师没能及时发现,他是住校生,家长也不知道情况,据说后来他
家里也没能让他再复读。这件事后来我每次想起都会觉得难受。

\chapter{我的大学}
\label{sec-4}
\section{我的大学(1)}
\label{sec-4-1}
  我想过很多次,逆势到底是什么?高三那年的创伤让我收获了些许精神层面的灵气
和通透,但世俗社会里我还是以前那个的冒点儿傻气的孩子。现在能回忆起来的第一次
逆势大概是因为大一下学期的一句话。

  高三毕业暑假的八月,我重新获得了以往通过知识改变命运的想法和勇气。上大学
后的几个月里,我反复地回忆总结过刚刚过去的那场灾难。下学期的一次什么班会上,
我说了一句话,每个人都会犯错,我要通过自己的努力重新站起来。我知道,虽然经历
了一场灾难,但内心里我对于舅舅口中那个自由的国度还是有着一如既往的向往。说那
句话是为了鼓励自己吗?但在一个农业院校,在一个刚刚升入大学,大部分同学来自农
村的同学班会里,我真真切切与同学分享的,发自内心的愿意,却像当初社会里一个女
人的轻浮,犯下了大忌。班上绝大部分同学不相信我,认为我是一个夸夸其谈、品行极
端顽劣的人。那时的我已经有了死后重生的生命力,他们如此种种我已经能够轻松做到
不以为意。但有一个人相信了我。在那个<疯狂英语>疯狂流行的年代,班长这个体育
生学习成绩全班可能能倒着数,却买了一本借到我们宿舍,辗转借到我手上。那期疯狂
英语里有一个申请出国留学的步骤介绍。

  上了大学,我还是像以前小学一样贪玩,学习成绩不太好也不在意,只是因为心里
有一个想法,只对英语单科比较重视。我说过我不是聪明孩子,四级是一次过的,六级
考了两次到快大学毕业时才过。像所有没有学习压力的大学生一样,我没事看看小说,
去小影厅看看电影,周末再同宿舍里的姐妹去打打乒乓球,日子过得不亦乐乎。

  好好玩了两年多后,到了大三下学期,我那久违的理想就不得不提上日程来。但这
个理想太大太沉重了,我没有十足的信心和勇气,却又跃跃欲试。而大四上学期元月份
的考研也是一种现实的出路。在考研还是考托福出国的选择上,那个春天,我纠结极了
。我告诉同宿舍一个室友说,我就像空气中舞动的尘埃,浮躁极了。我现在每天闲暇时
候最想做的事情就是回到宿舍,冲到水龙头下,好好淋上半个小时,好让自己变得滋润
清新。

  在我拿不定主意的时候,我给远地美国的舅舅写了封电子邮件。这里必须交待一下
,98年的夏天我出事后没再去找舅舅;但大一还是大二的暑假二舅在大舅家探亲度假时
,在大舅家我有请二姐陪我一起上楼到舅舅们的书房问二舅要联系方式,舅舅有把他的
工作邮箱留给我。那几年几个老年人,舅舅的妈妈、大舅母的妈妈都还健在,舅母同当
了医生的二姐一家走动还很频繁,二舅那几年回国度假的暑假,二姐二姐夫也有陪舅母
、舅舅们去爬家乡家门口的小秃山。

  在写给舅舅的邮件里,我告诉他,我很希望能有机会到外面的世界去看一看。我以
已经大三快结束了,我想申请留学美国,但以我的专业和学历,我担心我申请不到奖学
金。我问舅舅假如我申请不到奖学金,舅舅有没有可能、能否资助我读完一个硕士?还
是,考研对我来说更切合实际一些?我用四六级的英语水平给舅舅写了这封信,但我没
有等到回信。

  大三下的春夏,我的纠结、浮躁迟迟不能尘埃落定。但一场病、一个手术结束了我
的痛苦选择。
\section{我的大学(2)}
\label{sec-4-2}

  大三暑假,周围的同学朋友都去上了华师大考研补习班,为几个月后的研究生考试
作准备,我去了华师大的书店,买了一套新东方托福考试复习资料,想看看难度到底有
多大。我查到武汉地区那年8月6日可以报名托福考试(2001),但在我风风火火准备开
工时候,我生病了,7月29日在武汉作了阑尾炎手术。手术后伤口感染,8月6日还是7日
出的院,出院后我的同学每天轮流用自行车载我去校医院清洁伤口换药,十一期间伤口
彻底长好。住院期间二姐二姐夫有来武汉看我,我学会了依赖亲人,告诉他们我想考托
福出国。姐夫给我泼了盆凉水,他问我,“你知道什么叫心比天高,命比纸薄么?”在
他们的劝阻下,我决定考研。

  我的学习不太好,如果报考本校研究生,会有往年考题和专业课上的优势,但竞争
激烈。我即使上线了,可能也得读自费,但这是我的家庭条件所绝不允许的;我想考外
校,我想到了北京-新东方的故乡。而且将来申请留美读博的话生物类比室外大田作物
类要好拿奖学金多了。我要找个作分子的实验室,挑来挑去,只有农科院最适合我。当
年那个导师在全国范围内只招一个学生,整个研究所那年也只有两个导师共招两个学生
。但既然我打定主意不报本校,那就破斧沉舟来个冒险的,考得上就好,考不上工作了
改天再卷土重来。

  就这样,我没有报任何补习班,外带手术后还没愈合的伤口,自学一门<分子生物
学>专业课,考了315分。如果我告诉你当年农科院的分数线就划在315分,那我算不算
已经out没戏了?如果我再告诉你,我所报考的那个导师刚好就只我一个人上线,那我
算不算in?就这样,我以农科院当界所有考生最差成绩(刚好压线)考上了公费研究生
(除了因为关系,比如院长等,复试时从各大院校当年上线自费生中加招的几个学生之
外),每月会有600多的生活补助。重点是,我考上公费研究生了,我考上了!我开创
了我们学校报考农科院的先河,下一年就有一两个师弟“追随我”而来。

  考研时有两件事古物了我,一件是初三快乐的猪的精神,另一件是一场篮球赛。那
场球赛上半场都结束了,我们系的比分远远落后,很多同学、系友都难掩脸上的失望。
但我没有。那买过疯狂英语的班长也注意到了我。下半场,当队里其它成员还打得半死
不活的时候,班长先独自发起了进攻,后来,队友们都被他带动起来开始有了新迹象。
台下敌方观众就开始渺视了,说我们系照样打不赢,我可偏不信这个邪,与说那话的人
争了起来。他们打得很努力,再加上裁判一点儿偏心,最终我们系赢了。这场球赛很鼓
舞了我一段时间。

  或许就像每个人都有那么一段喜欢上喜欢一个人的感觉,我也有这么一段。这种感
觉,我喜欢了三四年。后来,大学毕业,环境将我们分开,也就放下了。

\chapter{研究生}
\label{sec-5}
\section{研究生(1)}
\label{sec-5-1}
  就要写国内研究生生活了,却觉得很沉重,迟迟理不出头绪。很多方面的事情都需
要写,需要有被提及,仅只文章篇副的组织对我来说就很成问题。而一些情感方面的事
情还需要特别谨慎,白纸黑字写出来的话,没有客观公正的态度都终将伤害当事人和我
自己。尽自己的最大努力吧,希望写出后的成品能不枉费大家与我共度的年华,也不违
背自己写成长故事的初衷。

  伴随着考研、复试的紧张,比常人多一点儿幸运,我考上了公费研究生,来到北京
,我的生活也翻开了新的篇章,开始了一段自己人生中最激情飞扬的奋斗。

  硕一的我延承了自己一贯贪玩的个性,上上课,去实验室做做实验,日子过得轻松
自在。高考那年劫后重生,我的世界从此打开,性格也重新开朗起来。英语课上是敢做
敢说,课外活动、后来非典时期非典生活也能玩个淋漓尽致舒畅痛快。也同对门宿舍的
一河北女孩比较要好。硕一下她们宿舍另一女生问我,我们班那谁谁谁你觉得怎么样,
要不要牵线搭桥你们作男女朋友?我认真地想了想说,我还是想考G考托申请出国,可
能没有精力谈恋爱,暂时就不考虑了。

  那会儿我实在是没心谈恋爱,没有觉得自己生活里缺少、需要男朋友这个角色。但
硕二进实验室后我却误打误撞,在感情的漩涡中泥足深陷。

  当年我可是自学了分子生物学的课才进到这个实验室的,对于能做一些PCR,构建
载体,AFLP银染测序之类的分子实验,我很是自豪,在所里其它两个女同学面前表现
出了十足的优越感。加上从一开始就把这个专业作为终身职业重视,信奉凭实力生存,实
验一开始又有好几个优秀的老师、师兄师姐帮忙指导实验、实验记录,我的课题进展顺
利,当着所长、导师同学等人的面,开题报告也能作得洋洋洒洒,收放自如,觉得自己很优秀,初三时的自信又被自己慢慢找了回来。报考了04年6月的GRE和05年
1月的托福,我说过我不是聪明小孩,同往常四六级考研考试一样,成绩不好不坏,自
己还比较满意。打算工作一年攒点儿钱就申请出去。

  我对自己的专业有着一定的觉悟和认识,但并不是天生的。硕二进实验室我的第一
次实验报告就遭遇了挫折。当时实验做得不错,幻灯片准备得也很周全,但报告结
束后老板的严厉批评实在让我出乎意料又饱受痛苦。当时就被整哭了,删除幻灯片并清
空回收站,会后躲进厕所又哭了好久。接下来的一周里,筒子楼实验室走道里很多次地
遇见了老板,我都避开了自己的目光。气头上我祈祷,请让我作个冷血的学生,只求毕
业就好,千万别再招惹我!但事实是,接下来他变得对我很好。实验还要继续,生活还
在继续。几个月后的十月,因为老板的一句话,我一头栽进了洗不清的黄河。
\section{研究生(2)}
\label{sec-5-2}

其实早在硕一下的时候就隐隐觉得有点儿什么不对劲,但又说不上来;进实验室第一个
报告的那次挨批也的确够伤人的,前篇写的读起来可能有点儿悲愤,但那个星期心里确
实有过这样的想法。后来做实验需要养拟南芥,温室里也有种些什么苗子。那天早上,
我一如往常来到所里第一件事就是去温室照看我的实验苗。不知什么时候老板来了,就
想赶快拔脚躲掉,但他叫住了我,很真诚地说,“XX,我不会让你觉得委屈!”我想,
这人看起来是老了些,但好歹还算有点儿担当,至少如果事情发展不顺利,他不会让我
觉得委屈。于是,便说服自己试着去了解、接纳这个人。

同一个实验室里,每天抬头不见低头见,批评养的苗长不好时说过“我跟你说,养拟南
芥就跟养孩子一样”blahblah,也接到过关切的目光。鼓励考试时说过“要是托福考个
满分即使去不了美国,别的国家争着要”,博士生联合培养学术交流会、课题开题报告
上也提携过我,学术会散会后也要我帮他背过电脑包。

在一个比较开放的国家重点实验室里做实验,而老板就是自己导师,我多少还是受到不
少实惠的。之前硕一硕二秋冬季节实验室都举行过什么什么之类的培训班,硕三还将继
续进行,而就在这个向往过的培训里,我却被整蒙了。

我清楚地知道我是被管实验的张老师找个借口开出去了,开出到整个培训之外。但我却
没能想明白究竟是为什么,心中充满了委屈。培训结束后一次不知什么原因到办公室,
说什么事情说着说着眼泪就掉下来,办公室里其它老师就都出去了。他说了些什么我都
忘了,还是安抚了一下我的情绪,周末带我们去涮小肥羊。

虽然开培训班时相当委屈,但他也有安慰过,加上还要考托福也很快就忘掉了。一月份
考完试,我的生活起了变化。

如果说大三下我清楚地知道自己的纠结浮躁,那么现在,我实在不知道我是怎么了。我
没有学习压力,打算等一年再申请出去,没有实验压力,实验进展顺利,整个学期基本
写写论文就可以了,但我却坐立不安,晚上辗转反侧不能入眠。宿舍楼水房里说过“一
个冬天过去了,感觉自己老了十年”的话(当然依照我一贯风格,绝对带了夸张的),
春天来了,失眠还在继续。我想我还是纠结的。老板对一实习老师的做法暗示了我留下
实习一年,我却做不了决定。虽然他说过blahblah什么什么事是他宠我,留下来并不会
太委屈我,但我却并不真正喜欢实验室。如果我选择,我情愿刮民主的风,呼吸自由的
空气。中间一次出去吃饭两杯啤酒不到就醉了。我每天只能睡着几个小时, 白天很清醒
,但人很累。我不知道是怎么了,尝试过去改变现状,却没有任何效果。快毕业时一同
学问我“你怎么瘦得这么干净?”听得我无限苍凉,一时语塞。人还是有底线的。经历
了高考我好不容易活过来,不能因为喜欢一个人去变成疯子吧?!实验室一定不是我的
选择,我选择了去山青水秀的广西养病。
\section{研究生(3)}
\label{sec-5-3}

因为一份工作,我来到了广西,尝尽了各式各样的当地小吃。米粉就能有几十种不同名
称和做法,还有各种粥(咸粥、皮蛋粥)和烧烤,以及白切鸡,水煮鸭等等。在那里,
长这么大我第一次尝尽了当地各种小吃,忘掉了烦恼,前半年瘦下去的肉又重新都长了
回来。一农科院同学的朋友问我要不要留下,不要,一定不要。我还是想申请学校,还
有那个想要出去看看外面的世界的愿望。因为当地受条件限制、很多事情都不方便,我
要回北京。

从南宁走之前,有打电话告诉大姐。电话里,姐姐要我保证回到北京后只把申请学校的
事情做好,感情的事一定放下。在南宁走之前,我是向姐姐保证了,但回到北京,我的
保证我真的能做到吗?

回到北京,为时尚早,先找了份工作干着。后来申请学校,公司老板听说这件事情后不
愿花时间和精力培养干不长久的员工,我被解雇了。我便全心全意只做一件事。我是想
全心全意的做,但我能做到吗?

可能是二三月份吧,等我申请完,就找了另一份小公司的事情干着。我终于还是没能忍
住,去找了学生时的老板。但他有一师姐作保护伞,去他家找他时,也撞见过师姐一人
在他家。那年的五一,极为尴尬。

我打电话到他家里,是一个女人接的电话;我问她是谁,她说她是他爱人。如果说以前
我一直不明白为什么老板总是放不开那个师姐,这一刻我算是彻底明白了。但是别人在
电话里告诉你人家是老婆,我该怎么办,我还能怎么办?稳了稳情绪,我本能地向她道
歉:“对不起,是我错了。”尔后挂断了电话。

我错了吗?我真的错了吗?后悔,还来得及吗?恨,咬牙切齿地恨,但别人告诉你她是
老婆。那句久远的承诺啊,是在说利益?除了那句,他还对你说过什么?师姐就像一座
雕像,永远地站在我们中间,而我却一直只理解为他老男人性格阴郁有些话说不出来,
怕被拒绝。我该怎么办,我到底该怎么办?小宇宙爆发的时候我真想作一泼妇,跑他家
去闹一通,把他与那师姐孤男寡女共处一室的事情也泼出去,把他搞臭,但那都是我的
自尊所不允许的。既然他承诺的是利益,那你得到了吗?我得到了,我得到了时间,在
我要考试的时候他没有逼我做实验。

成长的代价啊,我要付出多少,才能真正长大?

第二天我强忍着怒火,套了件最能显幼稚的衣服在身上出去转了一圈。院里很清静,后
来听说他们当众牵手秀过恩爱。

我的日子一如即往,看不出任何的悲伤。公司里翻译的工作好歹还对自己的胃口,作得
也还马马乎乎,一切都显得那么平静。
\section{研究生(4)}
\label{sec-5-4}

  2006年6月6日,在公司上班。刚吃过午饭,有点昏乎。打开自己的私人信箱,几封
新邮件,居然有封英文的,变得有点儿紧张。当我点开,意识到这就是我梦寐以求的、
传说中的OFFER时,我的手开始抖,那一刻真想哭,八年了,为了这个理想,我付出了
八年!

  等自己稍稍平静下来后,我开始打电话,迫不及待地想把这个好消息分享给姐姐们
。晚上下班回来,也在公用电话亭打电话给父母。是妈妈接的电话,我告诉她,我拿到
了奖学金,这次,我真的要出去了。电话那头,妈妈哭了,说舍不得我去那么远的地方。

  我接着上班,就像刚上大学总结刚刚过去的那场生死劫难一样,我总结了自己在北
京的这三四年。所有的付出都无怨无悔,但想到毕业那年瘦掉二十斤、想到一个多月前
自己的尴尬,觉得真的很不容易。但没有想到,又一场舆论战开始了。

  几天后的一天下班,回到院里,对面很近的地方走过来一老头,快要与我擦肩而过
时,老头发出一记出乎意料的响亮咳嗽吓得我心惊,能猜出他们怀疑我在撒谎。我禁不
住一阵冷笑,19岁的我在思想封闭六年后神经搭错会撒谎,今天我还会吗?恐怕我还没能
修练到这个份上吧?或许心里的嘲笑和鄙夷会自然而然地反映到脸上,那老头怔了我一
眼,默默地走开了。

  真正大面积范围内获得舆论胜利是在接下来的周末。周五我已向公司老板请假回武
汉办护照。周日同表妹在武汉逛街压马路,我接到了原研究生院打来的电话,告诉我有
封信寄错地方寄到他们那里,他们帮我打开,知道是我的OFFER LETTER,对我很重要
,所以特意打电话告诉我。

  但这还并没有结束。我那铁饭碗当了医生的二姐二姐夫一家不允许我出国,原因是
我太幼稚,出去没有生存能力。大姐大姐夫一家是支持的。我二姐是真心不愿意我出去
。三个姐姐都是初中毕业早早地进入了社会,他们对社会世俗的了解和把握远胜过了我
,我,充其量不过是一直生活在校园里的愣头青。

  因为我爸妈年级也大了,并没有多少劳力可以种田。在早早踏入社会的姐姐们眼里
,我早该自强自立经济独立了。体谅到爸妈劳作的辛苦,我们姐妹商量出一个折衷的办
法就是:从我大学(大概是)大三开始,爸妈不再供应我上学费用,我可以向姐姐们借,
但记在自己上学的账目上,等我工作了有能力归还时再不计利息归还他们。所以,到那
时,我已经从大姐一家借了大概二万左右。

  二姐有一项可以要协我的武器就是我要出去他们不会借钱给我。我没有想到都到这
头上了,他们不支持我,所以也放下狠话:如果他们不借我钱,我就算是向自己的同学
借,我也一定会出去的!后来是爸爸站出来发话,要二姐借我钱让我出去。后来除了机
票,二姐姐夫为我换好了1600美元让我带上以备急用。

  7月份签过了签证,走之前我回家了两三个周。临走时,爸爸特意交待过我一番话。

  爸爸说,我在国内表现很好,他们作父母的很为我感到骄傲。但是到了美国之后,
人生地不熟,语言上、学业上、生活上可能都会遇到困难,要我学会坚强。爸爸说,万
一学业上语言上我遇到不可战胜的困难,一定要想得开,要我平平安安地回家来,回到
他们身边,他说爸爸妈妈永远是你最坚强的后盾,任何时候你都不能做傻事!我哭着答
应了爸爸。

\chapter{留学生活}
\label{sec-6}
\section{留学生活(1)}
\label{sec-6-1}
  初到美国,一切都觉得新鲜。但熟悉了新环境、新鲜劲过了后,日子便显得漫长。
我初中开始学英语,听力还可以,上课基本能听得懂课,但口语不够好,常常讲出来的
话对方不仔细听会听不懂。我按步就班去上课,休息时间也让自己好好消息。学习基本
上还过得去。

  如果说在国内的这些年我有一个理想,我一直是为自己心中的想法在努力在奋斗,
那么今天真正来到这个国度,忽然觉得很空虚,有些不能适应。我真正想要的生活是什
么,我的下一个努力目标又是什么?我没能想得很清楚,学习上也显得庸懒。

  这是一个极为偏远的小镇,小镇上的人也显得民风纯朴保守。不知道是这个学校不
懂事的小留太多还是其它什么原因,我来这里不多久便爆发了我是小三的舆论。或许以
前在农科院的时候还都知道是怎么回事吧,所以我出事的时候大家反应不大;但在美国
这样一个小镇,我感觉很受伤害。后来穿了件美国T恤衫,中国传统棉质七分布裤出去
转了圈,相当于也是在说你们这些美国人也并不知道我在中国的经历,你们是否应该为
你们胡乱judge人感到耻辱呢?后来就平静了。

  但这件事从根本上打击了我找男朋友嫁人的希望。原本来到美国,我可把国内一切
的不幸彻底忘掉,不曾想到就这么件事还传到这个小镇,我又何尝没有一个希望能遇到
一个好人相亲相爱过一辈子的愿望,但这件小三舆论爆发后,我还有多少希望呢?加上
与同来上学的大部分学生相比,我年龄要比他们大出不少。这里绝大部分是读语言上本
科的小留,上硕士博士以及作博士后的有一部分,要么比我小很多要么就已经有了家庭
,真正年龄相仿的还要嫌我背负着“小三”的绰号,想要在这么个地方嫁出去还真不容
易。

  异国他乡可能显得分外孤独吧。隐约中记得那个二舅可能就在旁边的学校,但当我
把二舅名字的全拼输进这个学校主页的搜索栏,我却没能找到只好作罢。二姐是我们姐
妹里经济意识最强的一个。她问过我可不可以打工,我告诉她国际学生有奖学金的话不
允许打工。姐姐说,既然你每月只能拿这么多钱,又不能打黑工增加收入,那么你就只
有一个办法可以攒钱,就是省吃俭用可以不买的东西一定不买。姐姐给我讲过攒钱的重
要性,所以我谨遵她的教诲,认认真真地做起了守财奴。我没有电话,常常在电话亭打
电话给学校里的同学,也不常打电话回家里。我一般从加油站买五元钱的电话卡可以打
一百多分钟,然后用实验室的电话打回家去同姐姐爸妈稍微聊一聊。

  当初出国时,爸爸叔叔这边所有亲人坐拖拉机把我送到镇上,而后同二姐姐夫一家
三口经县城、襄樊市分别同亲人们告别,他们再陪我坐火车来到北京,把我送上飞机。
一路上,姐夫不下十次地问过我要不要请自己硕士时的老板吃饭,每一次,我都回答得
及时干脆,“不请,坚绝不请!”当时的我并没多想我这么做是为什么, 但到美国后,
我明白了。
\section{留学生活(2)}
\label{sec-6-2}

  在国内的时候,因为那份尴尬,我本能地掩饰着自己的痛苦。那时想,对这种伤害
的最悲愤最鄙视的作法就是把它当作过眼云烟,转头就忘,就像从来不曾发生、没有存
在过。但来到美国,在那份对待学习的庸懒里,在异国他乡的孤独寂莫里,那份刚刚过
去的尴尬回忆就像是毒品,越来越上瘾。多少个躺在床头的夜晚、多少个午夜梦回,我
一遍又一遍地流泪,一遍又一遍地添噬着自己的伤口。

  从结局看,他其实从来没有对你有过过多期待,因为他一直在婚姻中。他承诺了你
利益,他也做到了。你还有什么不满呢?感情世界里我受到了极大的伤害。那感情世界
里那你的角色又是什么?我的角色就是逼宫,把她从国外给逼回来。她是给逼回来了,
但他有没有从一开始就向你明示你的角色呢?反复回忆后,摸着自己的良心说话,XXX
,你就不该否认,温室那句承诺后的一个周末,他带师姐们出去吃饭,故意没带你。这
也该算是暗示吧,只是当时的你没能明白。那任务结束后呢?什么时候是真正故事的结
束?是第三次实验室培训我被赶出来了?如果是培训自己被赶出来时,那他之后为什么
还要碰我的手呢?我找不到一个确切的时间点、终点。

  在那段看不到找男朋友的希望、每天昏天暗地想要解脱的日子里,我的生活也变得
混混恶恶,有些混乱。只是因为爸爸交待过的话,还没有真正完全放弃自己。我休息不
好,不愿学习,不爱学习,与同一层楼上一大我一岁的国际男生混在一起,虽然我清楚
地知道他一直想打我歪主意。我很多时间与他耗在一起,偶尔会与他一起做餐饭吃,或
者出去买些东西什么的。

  我执着地纠结在那个找不到的时间点,每次都纠结在那个终点,每次都恨得咬牙切齿泪流满面不能自已。在我百思不得其解的情况下,终于在一次电话里我悲愤地向二姐述说了我的不满。姐姐骂了我,电话里破口大骂,她说我怎么这么糊涂,别人一直在婚姻中,别人从不曾对你说过什么,是你自己不成熟阅历不够在诱惑面前一头栽进去,能怪得了谁呢?姐姐说她们都觉得那件事后我早该自己想明白,她们不追问也是怕会让我尴尬,没想到过了这么久,我还在为那件事痛苦。电话里姐姐加足了砝码,就像一个个大石头砸向我,个个都击中要害,想不清醒都难。

  如果说我成长的早期妈妈的封建迷信涂毒了我的心灵,(那么经过几年的纠正与释
放,改正了不少。)那这时封建迷信也帮了我一个小忙。我从星座上查到他的星座上写
着“他让你爱,他让你恨”,意思是说他这个星座的人本能地不知道如何终止感情或者
是说诱惑。我的成长脆弱而艰辛,我也有很多性格缺点比如好哭耳根软容易忌妒等等,
但我还需要坚强勇敢地活下去,所以最终,就像原谅自己自身也存在性格缺点一样,我
原谅了这个人,这件事。这只是一件特殊历史困境下发生的个体事件,我得到了利益(
虽然不是我的出发点)受到了感情的伤害,师姐是坐享其成者,但她也是老板棋局里不
可缺少的棋子。我相信她们孤男寡女住在他家也是出于社会世俗所需的保护,并没有任
何见不得人的勾当,因为我与他最亲密的接触也不过是第三次培训被开后办公室电脑键
盘上他刻意碰过我的手。基于人性的善与恶,这个故事应该在历史的仓河中被淹没遗忘。
(五年后因一句率直而欠思考的话我被人肉,故事再次被打涝而浮上水面。我因被过度
误解被迫作此自传,我所能做的也只能是还原历史本来面目,请读者自作思考。)

  或许自己处在困境中的时候总是希望有人能来主动帮助,当我从感情的阴影中走出
来后,便断开了与那个国际学生的所有联系。“eecs”就像一串密码神奇地浮现在我的
脑海,那是二舅留给我的工作邮箱里的几个字母。我从网上进入舅舅学校的这个系,找
出教授清单,人群里,我一眼认出了舅舅,便去他办公室找他。舅舅来我的住所探望过
我一次,建议我买部二手车。后来我买了车,又认了舅舅,彻底告别了那段阴森晦暗的
日子,感觉自己从此挺直了脊梁骨。
\section{留学生活(3)}
\label{sec-6-3}

  一年级结束时,我开始意识到这个年龄段找男朋友比节俭更重要,所以新学期开始
时就早早地联系好了四个新生共用一个750分钟的family plan。他们四个男生,农村城
市的都有,有个比我小两岁的老乡,也有一个十八九的,大家相处愉快。回想自己刚来
时的种种困难,我尽量扮演一个姐姐的角色,知道他们刚来时住在学校招待所,只能从
外面买食物吃还不习惯,便请他们到自己租住的小窝吃过几餐。

  舅舅来我的住所看我时,建议我买一辆二手车。遵循姐姐要我节俭的教诲,我得选
辆价格公道经济实惠型的。之后经舅舅参谋便买了现在这辆有过明显磕碰痕迹舅舅可以
帮我再加工的车。有一次在房东家吃饭,我便告诉房东,舅舅已经帮我把车外壳陷进去
的地方吸出来了,重新喷了漆,现在整个外观已经很漂亮了,也教我学开车,但感觉舅
舅总不愿把车钥匙交给我。房东说这你还不明白,别人帮你选了车,别人帮你修了车,
你不都还没买礼物表示感谢吗?之后我就按房东教我的主意去试探一下舅舅有没有什么
特殊需要或者喜好。我告诉舅舅我买了件奢侈品,一台\$25的面条机,从bedbath\&
beyond买的,很好用,舅舅却说喝豆浆对身体很好,进而提到豆奖机。我一开始是像我
以往买任何东西一样找便宜的,就找了台七十多块的,下了订单。但转而又想,舅舅年
级大了,别人都说在美国一分钱一分货,我买个贵点儿的好歹舅舅用着方便。便又及时
取消订单,重新从walmart官网上买了一台百元左右的邮寄到local店(加税在100到110
之间)。那天舅舅陪我一块儿去店里取的,他打开包装箱看了看就抱回去了,我连豆浆
机长什么样都不知道。房东有带我去costco买过两瓶还是四瓶老年人舒活筋骨健关节的
保健丸,我问舅舅不知道舅母吃不吃这类东西,舅舅说她吃,我便送了过去。

  舅舅带给我的另一样转机就是转专业。受同学影响,我告诉舅舅我太笨了,这个专
业我熬不出头。舅舅解释说中国学生一来就转专业,在系里影响非常不好,要我等。我
很多时候想,一学生来上一学期转专业与上一年半两年转专业区别真的有那么大吗?但
我没办法只能等,加上我总是眼高手低选难课又考不好,最终也几乎是不得不转专业。
后来08年春季新学期时转到同一学校统计专业,舅舅是经济担保人。

  08年夏天舅舅陪我一起驾车到加州表姐家。舅舅是表姐的亲叔叔,我妈妈的爷爷同
舅舅的爷爷是同一个人,当年表姐还才读高中的时候就是这个舅舅把她带到了美国上学
。路上我开车时65限速我一定卡着75开,舅舅不许我开快。后来我真正慢下来,我们便
山南海北地聊,我把自己除了小学毕业那件和“小三”那件之外的所有成长经历都讲给
了舅舅听,包括从大三开始向姐姐借学费。舅舅也给我请他那大家闺秀的妈妈,他的哑
巴妹妹怎样聪明以及早早离世的原因。这些都聊完了,什么圣经、红楼梦、以及其它一
些著名的文学作品,舅舅知道的,我感兴趣的全聊了一遍。继而现在的年轻人不攒够钱
就买房子,奢侈浪费家外面冻得要死房子里热得要命,等等,一路聊到了加州。
\section{留学生活(4)}
\label{sec-6-4}

  其实早在去加州之前,舅舅就帮我换好机油,我洗车,而后舅舅帮我一起打蜡去蜡
。那时我们也聊过很多。我听妈妈说舅舅家有三个表哥,我便问舅舅他们家亲人近况。
舅舅说舅母出门去了(其它情况下有时候说去教会了),大表哥在西雅图,二表哥在韩
国,小表哥结婚了住在我所在的小镇。那大概是我们第一次有时间好好聊天,聊什么呢
?我告诉舅舅我长这么大感触最大的就是我爸妈辛辛苦苦把我们姐妹四人拉扯大,他们省
吃俭用地供应我们读书,而他们自己从来都不舍得吃不舍得穿,我真真切切地觉得父母
对子女的爱太无私太伟大了。对他们我最大的愿望就是希望等我有能力的时候能够让父
母过好晚年。舅舅接过我的话说了些什么,而后说他们(指他与舅母)同国内的父母不
一样,国内父母干涉子女私生活,他们是有文化受过教育的人,决不会干那种事情。我
当时只是感觉有点儿噎,傻傻地也没好说什么,听过也就忘了。来到加州后,表姐问,
老二还没结婚啊?还没有舅舅答。我想表哥在韩国即使还没结婚,韩国美女那么多,而
且年龄也有那么大了,在韩国一定有他实实在在的生活,所以也从不多想。后来求学的
几年里还带过一个男生到舅舅家,希望他能看看也帮我参谋一下。

  08年春天四五月份的一天,我第一次因为经济压力情绪崩溃了。转专业时我就打听
过这个专业所有的国际学生都免了外州费,所以我认定了转这个专业。虽然我苦苦地争
取了一个学期的奖学金,但当我的同学们都收到下一年的TA offer,我连外州费都不能
免时痛苦极了。我跑去系里,小秘说系主任不在具体情况她不清楚,所以只能等系主任
回来再说。系里碰见我导师他建议我去IPO问一下他们能否帮忙免外州费,我去了,IPO
说关键人物也不在,不过希望不大。我背着书包跑到图书馆去趴了会儿稳一下情绪。那
天傍晚回家的路显得格外漫长,一路上所有的花草树木都暗淡无光暗然失色,downtown
熙熙攘攘的人群也都有着他们各自的落寞。我费了九牛二虎之力终于走回家,关上门,
一屁股跌坐在地上大哭起来,不知道接下来的生活要怎么才能继续。哭了大概有半个小
时吧,最终还是只能站起来,做点儿吃的,因为苦也好累也好一切都还得继续,不是吗
?后来系里还是给我免了外州费,我猜想系主任这么拐一下大概只是希望我能认识到免
除外州费本身已是一种恩赐,我应当学会感恩吧。

  舅舅给我灌输的概念也是人要感恩。舅舅给我讲他家那条狗如何讲究卫生,主人出
门了它便不吃不喝也不拉,等到主人回来拿食物给它它才吃。舅舅说他会养那条狗到终
老,只要他自己有口饭吃,那条狗就一定不会饿着。舅舅向我表达过带小表姐出国他情
感受到伤害的意思。因为小表姐读书出来工作以后,从不曾表达过感谢过舅舅。舅舅说
若不是不放心我,想亲自送我去加州,他绝不会上表姐家。

  舅舅不愿打搅别人,中午带我去JackintheBox吃饱吃好下午才上表姐家去。我一头
热想学包饺子,便真正和了面,不多不少大概三四十个饺子的量。表姐见我真正想学,
便赶快剁虾仁切韭菜。舅舅鼓励我学这门手艺说,等改天我学会了到他家,我可以让舅
母不用动手,自己亲自做一盘热气腾腾的饺子请舅母品尝。当时大舅大舅母也在表姐家
。舅舅对他们说我靠自己的能力出来很了不起。当晚舅舅喝酒不知醉了没醉,一直哭,
说他命苦什么的,最后除了表姐家两小孩尝过几个,舅母一次煮八个十个,几锅煮下来
,舅舅把那些个饺子全吃了。

  那时我钻在钱眼里,对表姐们宣称说我回学校的时候准备装一车干粮回去贩卖,舅
舅说那好啊,但他对我说话的眼神语气加上他留自己的行李箱让我帮他带回去,我无法
做到两手空空地归还舅舅的行李。从大舅母那里打听到舅舅的一大爱好便是品茶,而且
只喝好茶,超市里卖的大众化的他根本不喝。于是从天仁茗茶\$60一磅的冻顶乌龙茶培
火的与清香的各买了一磅,加上从大华买了几袋湖南腊肉(我们陪大舅母一起去超市时
舅舅挑选过这一样)和其它一些干粮杂物,这样08年夏天我回学校时便有买\$150-\$170
的礼物送给舅舅。
\section{留学生活(5)}
\label{sec-6-5}

  07年夏天从一段阴影走出来后,还是觉得自己日子很空,朋友太少,于是便在家旁
边的一家美国教会呆过一段时间。07年感恩节附近还有随他们一起去本州北部的一小镇
参加他们的活动。活动本身包吃住,两天三夜大概是,自带sleep bag,每人交\$40。他
们为鼓励广大国际学生参与,说确实有困难的国际学生可以不用交,07年我应该还是守
财奴,所以我就没有交那钱,但一直参加他们的活动。后来,08年春季快结束的时候,
因换了专业,学业也基本还算进展顺利,实在还是觉得自己不是能够成长为基督徒的料
,便离开了。也实在是觉得他们那个组织不会去赚别人的钱,所以走时有捐\$60给他们
的一个活动作为对那次活动\$40的归还。

  08年底的时候有坐飞机再次去加州一个朋友家,但限于时间匆忙,那年圣诞,我并
没有给舅舅带任何礼物。回来后有给舅舅写封邮件,告诉他我回来了。他在邮件里说“
Welcome home.”

  如果说之前小镇上的人们对我印象还基本过得去,马马乎乎,那么09年春季,因为
一场恋爱,因为一件不相干的事,讨厌我的人更多。

  那年农历春节的时候在房东家party,认识了一个住同一座楼上大我一岁的男生。
和大家想的一样,我恋爱了。

  我把从网上能找到的他的属相、星座与血型以及我自己的、网上可以查到的组合的
全打印出来,然后跑跑跑跑到舅舅那里,问舅舅对这人是什么看法。舅舅说绝大部分可
能他是个骗子,我不信,觉得那时因为舅舅的一句判断就分手太痛苦了。舅舅见我坚持
,便说也有可能,怎么怎么样。舅舅还给我讲了一个他所认识的一个人恋爱了九年最终
还是结婚了的故事。可是舅舅这个极端的故事我一定是不喜欢的,本能地排斥着。

  那时也听说我们family plan有个男孩因为功课还是实验被老板开了,他没回家,
也没离开我们小镇,据说在校招待所呆着。当时他欠了我大概\$160的电话费没给,组里
有其它成员找了一个女生来用他的手机、手机号来代替他。那个女生我比较熟,还对她
说过如果万一哪天他不知道哪里去了,我可能也不会问他要那些电话费,因为他实在是
处于困境,等哪天我闲了会去找找他看他是怎么想怎么计划的。

  后来,事实证明舅舅说对了,那个人确实并不值得我投入。当我获得足够的信号意
识到这一点儿,便与他闪电分手了。

  分手的那天晚上(好像是周五),接到一个来自上海的陌生电话,知道是那男孩的妈
妈,是family plan 的那女生将我的号码告诉他爸妈的。她问我男孩的近况。想到那男
孩独自一个徘徊在十字路口,就像看见当年那个19岁的自已,当时我有亲人的关心爱护
,尚且苦苦思索了几个月,何况他是一个人。他先前的老板可以不喜欢他,但我们算作
朋友的又怎能不管不顾?就算我作个恶人,他的情况一定得让他家人知道。我便告诉了
他妈妈我所知道的一切。他妈妈当时电话里就忍不住哭了,要我明天白天一定去找找他
,要他给他家里打电话。我答应了他妈妈。

  那天晚上在水龙下冲了很久很久,仿佛有千年的污垢累积,无论如何都洗不干净。
很晚了终于勉强入睡。第二天一大早,无数个闹钟将我炸醒。拿手机一看上面全是男孩
爸妈打过来的电话,只好赶紧去找男孩。
 
  我找到了他,并说服一住一室一厅的朋友收留男孩住到他家。后来白天里接到男孩
在美国一叔叔的电话说他已订好机票明天就到准备送男孩回国,怕男孩在最后关头思想
想不开出现意外,叔叔让我看着他,我答应了。一个失恋的人同一个失意的人在一起还
真是有共同语言。他说他妈妈有高血压,我便向他道歉昨晚对他妈妈讲话时没能考虑到
这些。他说他什么什么人很闷骚,我接过话说“我也很闷骚”,我为自己终于第一次能
够在陌生异性面前承认自己闷骚感到很开心。

  后来,男孩的一切顺风顺水,叔叔将他送回国,亲自交到他父母手上。叔叔将他欠
我的\$160也写支票给了我,还请我吃了餐饭。叔叔做事干净利落的作风也给我留下了深
刻的印象(叔叔将男孩堂哥也带来,这样他们能相见,但我感觉叔叔更主要是不愿麻烦
我送他们去机场,所以由堂哥来操办这些杂役)。

  陪着男孩的这个周末,我很开心。男孩走后,我终于回归到自己的失恋情绪中,一
片荒芜。
\section{留学生活(6)}
\label{sec-6-6}

  这段恋情极其短暂,有如烟花,灿烂绽放,瞬间即逝。恋情开始是因为承诺,分手
后我对恋情、对那个前男朋友也没有留恋,以最短的时间忘掉它是爱自己的最好方式。

  有一种杀手叫隐形杀手,有一种痛是潜伏心底的痛。称他们为隐形的潜伏的,是因
为当你生活正常、开心快乐时,他们不显示任何症状;但是一旦有什么风吹草动,它们
就会蹦出来充当帮凶,成为压垮你击溃你的最后一根稻草。

  非常不幸的,我有这样一块“肿瘤”。19岁那年,向亲人坦白小学毕业那件事后,
姐姐就已经从理论上帮我解释得很清楚了。但被我夹着腋着藏了六年发酵了六年的成品
,姐姐解释得再清楚,心理上的阴影不是一时半会儿能根除掉的。上大学时晚上习惯听
广播,有时候会收到一些什么医疗台。所以就有那么一个晚上,我躲在外面等节目一开
始,我便打电话进去请教,医生说不会有事,也可以作一些配合检查。但我一直没有机
会。刚分手时因身体不适去过医院,医院说除了有近期轻微感染,没有大障,开了些药
就好了。但这次分手将我的情绪扔到无底洞后,我想彻底宰了那个“隐形杀手”。

  我是学生,还可以享受学校里的心理咨询服务。我便去见了医生。大概见了十多面
吧,从分手的恋情起头,到自己小学毕业那件事结束。但没有特别深的感触。医生对我
的结论是我冷血。医生说正常情况下如果你见一个医生数十面,然后她告诉你她要出差
了,而你没有任何留恋,能说明什么呢?虽然我不是很承认自己冷血,但我觉得她有帮
我纠正一个观念就是那件事不是我的错,那是意外,而之前我从来都是怪自己笨,如果
我能聪明点儿,那件事可能不会发生。和以前一样,心理的阴影是没法轻易消除了,所
以最终就还是进行了生理检查,结果是一切正常。

  我解脱了吗,轻松了吗?也没有。那件事以及造成的心理阴影就像心底一个若有若
无的影子但却客观存在。我只能说我尽力了,但这件事、这道阴影,我终究只能是无能
为力。

  那段时间里,“民间” 流传着一种我与那前男朋友的分手另类说法,说我是故意
这么设计好了与前男朋友分手的。这样我分手后“下一秒”便找到一个更年轻更帅气的
男生共处把他比下去!我有这么荒诞吗?很多时候我也只能哀叹自己命不好,逆势太多
,受到太多无缘无故的伤害。
\section{留学生活(7)}
\label{sec-6-7}

  09年秋还发生了一次与这次逼迫我写自传同等严重的摧残。虽然我清楚地知道这次
是因为感情事件,知道是谁亲手把我推上了断头台。但那一次,情况很复杂。

  早在春季学期的时候,我就已经递交了毕业申请。秋季开学后因注册part time的
学费也很贵,所我就只注册了一个学分维持身份。后来收到IPO的邮件说我有免除外州
费的优势,如果不用过期作费,便赶快重新改注成full time学生。后来不知道经历过
了到底多少次舆论的酝酿与发酵,等有一天我意识到大家(学校的师生和小镇上的人)
怀疑我撒谎,已是硝烟滚滚,战火连天。第二天早上一上课,代课老师先分享一个故事
,故事讲的是他撒谎了,可能希望我不显得那么尴尬。问题是,我没有啊?有了高考那
年的经历,借我十个胆子,看我敢不敢?极度抑郁不平下,我便给从前一起去过教会的
美国女孩用英文写邮件解释事情的整个经过。

  没有想到,这封充满不平、不能理解大家为何怀疑我的邮件又把自己推向了另一个
火山口。后来知道,因为在我希望获得奖学金以及注册full time还是part time学生这
件事上,我推想系主任怀疑小秘办事不力,有可能是因为这个原因把她调到了校其它地
方工作,我只知道系里小秘换了。大家舆论所不平的,根源大概是这个。所以写给女孩
的邮件里,我具体解释了事件的整个经过,小秘没有什么做得不对、说得不清楚的地方
。我只是因为不喜欢系里那个新来的女老师,所以即使我申请十二月立即毕业,也不是
很愿意跟她做研究。信里我说,她每天来系里都打扮得花枝招展,又不好好教我们(我
们五个中国学生有上她的课一年),我有时候都忍不住想她是不是花瓶。美国人的思维
到底是什么样的,我真搞不懂,这句话怎么就成了她与系主任有某种特殊关系的证据。
到后来事实证明没有那层关系时,我又成了人们讨厌的对象。

  我想当时写那封邮件时的自己,情绪是处于大家极力怀疑甚至是肯定我撒谎的情况
下,我非常不平地写下那些话。抑郁之下对自己不够喜欢的那女老师自然是怎么看都不
会顺眼。我应该表达的不过是,作为代课老师,作为代第一年课的老师,她对我们要求
不够严格,她有能力、原本可以在代课上表现得更好,而作为我们学生,至少我是对这
个老师这一年的代课有所不满的。另一个我当时没写出来的原因便是那年春季刚同一个
骗人的前男友分手,不是自己的导师,不是我能够真切信任的导师,我又岂敢同她做研
究?多年来的经历、初三、考研、考TG出国,转专业,不从来都是从自身从内在发掘力
量吗,相信什么都没有相信自己实实在在的努力来得有勇气。

  当然我的执扭自然是惹恼了系主任,下学期我没能得到系里的任何经济帮助(除了
免外州费),舆论也有为我不满过,系主任的挽救办法是,冲我选临校一门课与他的
seminar课在时间上冲突时,他允许我不用上他的seminar课,这样我有机会选另一门应
用课。

  小镇上的人们很讨厌我,这件事后来也是不了了之。去年我收到学校寄来的邀请捐
款的信。那时我终于从多年的学习压力中解脱出来,玩得很high,想到自己受惠于这个
免外州费的program, 便捐了\$1000给系里,希望能帮助到真正有需要的人。
\section{留学生活(8)}
\label{sec-6-8}

  从07年夏天认了舅舅到毕业后离开的两三年里,因为舅舅建议买的这辆车,我们有
机会对对方有所了解。这中间也发生过几件小事。

  舅舅教我学车,12(11?)月18日我拿到驾照。之后舅舅就把钥匙给了我。08年5
月第一次去加州前有换机油洗车打蜡。差不多半年后(舅舅觉得我开得少,每三个月太
浪费了),我买好机油、滤器同舅舅约好时间,他会帮我换机油,再打蜡。

  有一次我去舅舅那里,舅母还是不在。快中午了,舅舅说出去要买一个什么东西,
我就坐在门口的水泥阶梯上等。后来舅舅买回来一条很长的什么味道什么牌子的面包,
舅舅说他很喜欢,和一盒烤鸡。我们舅侄俩个就坐在门前的水泥阶上吃起来。以我扎实
的饭量,我一定可以吃很多,但在舅舅家,我终究是觉得拘谨,匆匆啃了条鸡腿吃了节
面包就打住了。舅舅可能有点儿扫兴,也不再多吃,就收了。

  不记得是不是同一天了,换油洗车打蜡一切都做好了,我很不争气,下坡时不长眼
睛撞到了邻居家的curb,右前方车轮爆了。我清楚地记得舅舅步履蹒跚走回他家车库拿
工具的背影。那一刻恨死自己了,真是恨不得抽自己几下。舅舅找来一个以前不什么时
候准备的车轮,帮我换上,带我去店里做了balance,才算彻底解决问题。

  还在我流恋着钓鱼的岁月里,有一次打电话给妈妈提到美国的鱼很笨很好钓怎么怎
么的,妈妈说让我钓到鱼了也给舅舅送些去。对啊,我怎么就没想到。于是去冰箱里翻
,从清洗干净了已经冻成冰块的冰柜里捡最大的挑了12条装了一大袋打电话后给舅舅送
去。我选12条是因为我们那里一个license一次只能钓六条,12条也就意味着两次满载
而归而且自己一条不吃的收获累积量。不过这只是自己的想法,交给舅舅时,我并没有
说这么多废话。

  还有一次快圣诞节还是刚过这个节,下了整整一周的雪,从walmart 拐去买菜店刚
进左转道,车子直接来了个180度调头,我惊得嘴巴张得比保龄球还大,眼看就要撞上
对面一辆车了,还好那车主及时刹住了车。那美国男生车主比较nice, 特意示意我到
停车场停下,问我没事没有任何问题安抚了我一会儿他才离开。可我这么溜了一下一定
是惊动未定的,回去就给舅舅发邮件叫嚎着要换雪胎。舅舅开我的车载着我去他之前搜
到的几家“胎”主那里去看。但最终舅舅都不满意还是没能选定。可是舅舅也带我去看
去挑选过了,受惊吓的心情也该平复了,剩下能做的就只能是自己开车小心了,不是吗?

  09年春天换机油时我买好了机油、滤器和雨刷,谁知舅舅也买了滤器和雨刷,但舅
舅坚持用了他买的,我自己买的扔在了车上。我曾经问过舅舅冻顶乌龙茶,茶培火的与
清香的,他到底喜欢哪一样?舅舅说他两种都喜欢,没有特别偏好。

  09年夏天我有机会去到加州。是我给舅舅买礼物已经买成惯性了吗?那年夏天,我
又买了两磅茶,几袋湖南腊肉和一些菌类干粮,和08年夏天一样,价格在\$160-\$180。
和往常一样,去之前先给舅舅打电话,车开到他工作楼附近,而后找到舅舅,他开我车
将我载到他车旁边。等所有的礼物都转到他车里,这次,舅舅说,“现在是学生买这些
还不算什么,等以后工作了更该加倍地买。”
\chapter{表哥与加州生活}
\label{sec-7}
\section{表哥(1)}
\label{sec-7-1}

  09年秋毕业前的最后一个学期,是对界压力最大的一学期,但自己过得很充实。那
时几个月前与前男朋友“设计好”分手的流言余毒还在,后有谎言风波以及后来很多人
都讨厌我。也快毕业了,该做的事情一定得做好,尤其是专业上的两个认证,一定得考
,后来也算都过了。

  09年夏去加州前,舅舅问我什么时候毕业,我说十二月;舅舅问还能再推吗?我不耐烦地说我都申请毕业了。秋天我回来后,没想到舅舅把表哥从韩国搬了回来。那学期有被舅舅正式地邀请到他家have dinner三次。第一次大概是八月份,开学前还是开学后不久,这餐饭舅母和表哥两人同时在我的世界里第一次出现。舅母一头花白长发披散在肩上,看起来还很慈祥。饭桌上见到了表哥,四角方桌他坐我旁边,只觉得他很瘦,脸部轮廓分明,眼睛显得很深遂,但我不敢也不愿看。就算他是个帅哥,与我又有什么关系?饭桌上有舅舅特意准备的生鱼片,看舅舅表哥特喜欢的样子,我不太敢吃,便说吃不习惯。这餐饭上我明显感觉到舅舅对表哥的格外偏爱,只能暗叹自己又算得了他们家的什么人,自然无法得到他们家宝贝儿子的待遇!

  忘了是第二次还是第三次吃饭了,大家似乎都等着我说话,可是舅舅刚才叫他什么
?我没听清楚,也想不起来,又不好问,只好叫“表哥”。估计舅母是瞪大了眼睛盯着
我,问我刚才叫他什么,这世上怎么有这么不解人意的人啊,人家好不容易可以糊弄过
去,还被舅母抓现的,没办法,只好再重复一遍“表哥”,眼睛只敢盯着桌子角。后来
反正是匆匆地把饭吃完了,表哥饭桌上讲了一个什么笑话我既听不懂又不好笑。不过这
人应该也不算太差。临走时舅母让我带一袋超市里买的饺子回去。还有一次舅母给我带水
果,说是表哥从校园里摘的试验品种的苹果,回去尝一个,是辣的,再尝一个,味道还
是很奇怪。

  后来应该是最后一次吃饭吧,舅母问我我属什么,我说属羊,舅母说属羊怎么就
没见你温柔呢?我特冲地说,千万别把我惹恼火了,把我惹火了我脾气可大着呢!晚饭
结束后,我终于是鼓起勇气问舅舅表哥有多大,舅舅说了,我算了算,比我大13岁。13
岁到底意味着什么我没多想但表哥走进房间时的一个喷嚏让我自主后退十万八千里。表
哥饭桌上还有几句话,等吃完饭他一般就钻进了他房间,我一个女孩子又不好意思闯他
房间,所以一般吃完饭,帮舅母把餐具洗了我就直接回家了。后来也没再联系。

  毕业后休息了一段时间,准备来加州好好找工作。临走时将一个\$25在亚州店买的
蒸锅送给了前房东(09年秋季住了五个月的房东)。舅舅在当地居住了几十年,他应该
不缺这类东西。我把一部分不能随行的衣物书籍等放到舅舅那,并把一些买了没来得及
吃掉的菌类干粮等,每样都>2/3瓶的香油、花椒油、生抽老抽等很多瓶乱七八糟的都
搬到舅舅家并说明这些我不要了,让他们帮我用掉不扔了可惜就行,剩下的<2/3瓶的
各类杂物、炒菜锅电饭锅餐盘等都留给了当时的房东,自己只留一个喜欢的汤锅和新买
的电饭褒放在了舅舅家。朝舅舅家橱房搬东西时,突然想起我还有两盒很好的银耳我只
吃了一团忘了拿,舅舅说我可以明天早上走之前放在他家门口。临走时舅舅送我一个睡
袋(sleep bag),我对舅舅说我没机会用这东西,舅舅一定要我拿着。我没办法,只好
拿着。

  第二天早上我去舅舅家放东西,却看见表哥从他车里钻出来,很快地去抚摸他家的
狗。我很意外,又觉得那一刻表哥显得特别有亲和力,我嘴里说着不知道什么乱七八糟
的话,本能地向他家门口看了一眼(总觉得舅母的眼睛在哪个窗角盯着我们),就匆匆
离开了,还要赶路。

  不过早上与表哥匆匆一别,既意外又兴奋起来,心里面推敲着表哥到底喜不喜欢我
,一路山歌唱到了加州。 
\section{加州生活}
\label{sec-7-2}

  来到加州后,很快找到一份工作,不好不坏,但自己还比较满意。

  表姐家表姐们面前我多次提到那个临走时还见了一面的表哥,每次大表姐就喝斥我
:当初你们在一个地方时,也没见你们有点儿什么,现在分了十万八千里了,你还再想
这事作什么?表姐说表哥性格不好,怪得很,要我多出去参加活动,多交友,表姐说我
现在学位有了,工作有了,再找个男朋友,生活就完美了。表姐说这些话一定是为我好
的,我没有不听的道理,便很快忘了表哥,如火如荼地投身到找男朋友的“事业”中。
  
  或许也因为二三十年来终于第一次从学习的压力中解放出来,整个人就玩得特high
,一周打一两次羽毛球,周五周六的晚上去跳过一段时间的舞,也参加过一两次每月一
次的舞会。后来觉得自己实在不是什么跳舞的料,就不再去了。但hiking,附近有组织
的一些插得上手的活动一般都参加了。或许自己太high太放松,或许自己过于直率,终
于是在说了句不经过大脑的脑残无厘头的话后,我被人肉了。

  再后来,等我的OPT第一年快用完,得申请17个月延期时,学校IPO因为只有一位老
师负责做这件事情,而她正好又出去开过会,而且生过病,所以我的申请也被一拖再拖
。我请我手机family plan的一男生去IPO帮我崔一下那老师,但那同学问后给我打电话
说那老师说我的issue没有那么urgent,所以要我耐心地等。我一定是没有等的耐性的
,一如那年等免外州费,等可以杀了我!所以同那老师约好时间,我向老板请假,打算
回学校去准备所有材料。这样我又回到了以前那个小镇。

  忘了交待,一年半的农学博士研究生期间虽然那时我作了很久的守财奴,但攒的钱
并不多,后来转专业成为自费研究生,虽然免外州费,有亲朋好友接济,压力还是很大
,到最后学期时不得不向舅舅分两次共借了$4000,向family plan的这个男生借了$
1000(\$995),最终才勉强度过难关。工作后,向同学借的数目少,很快就还了。到回学
校时,我手上有可以还舅舅的钱,但手上钱不多,也害怕自己有什么意外时又将不得不
再向舅舅借钱,所以同舅舅说,先放一放,等我手头再稍微宽余些就还舅舅,他同意了。

  那天舅舅说要我加倍地买礼物时,我心里早犯起了嘀咕,牙齿还真长,怎么可能?
!但因为舅舅的钱我还没还,加上工作的第一年自己确实开心,再孝敬舅舅一次也不过
分,我还真就真真切切地给舅舅双倍地买了一次礼物:\$120一磅的冻顶乌龙茶,培火的
与清香的各300克(店主推荐说买300克包装好的比称散装的好),\$80一磅的黄山毛峰
(是毛峰还是毛尖我记混了,但更大可能是毛峰)一磅包成了两大包(茶的总价在\$220
-\$230)。按照一贯的习惯,大华里的湖南腊肉、另一样瘦肉多的腊肉、各种香菇花菇
稚融菌类等等装满了一大购物车,账单在\$130-\$150。这样带着总价在\$350-\$380的沉甸
甸的礼物,我回到了舅舅家。
\section{表哥表哥(1)}
\label{sec-7-3}

        我住的地方到舅舅家有一千多迈(1038),正常情况下开过去要十六七个小时。为了
不让舅舅久等,那天周六凌晨二点我就出发了,后来一路“泥泞”到晚上九点半才到。
舅舅家橱房亮着温暖的灯,转眼舅舅就出来迎接我了。

        舅舅问我饿不饿,他可以给我做吃的。其实早前打电话给舅舅能够感觉到舅舅对我
回他家是有期待的,但与表哥的事情我还没想好,不想给他老人家太多幻想,所以一口
回绝“不饿不吃”。表哥刚从厕所出来,上面套着件带帽外套,下面裹了条浴巾,叫了
声“表哥”,他说了句“回来了?”“恩,回来了”就睡他的回头觉去了。我也是昏昏
乎乎,拿到舅舅给我的房门钥匙,告诉他明天早上舅母不用叫醒我,让我休息够了再起
来,匆匆洗漱便休息去了。

        我以为我能睡到十二点再起,谁知九点多就醒了。家里像一座空城,除了找到舅母
,舅舅表哥已不知去向。我问舅母,她说你知道舅舅办公室吗?你找到舅舅就可以找到
表哥了。忘了什么情况下问的舅母了,舅舅一把年级了周末、晚上还去办公室干什么?
舅母说舅舅在写一本书。我也想写一本书,一本关于自己的书,我们舅侄两个还真像。

        舅舅的办公室还是有凌乱的地方,但已经比07年我第一次找到舅舅时好多了。他随
手拿起一颗巧克力放进嘴里。我问舅舅表哥在哪里,舅舅问我找表哥做什么?我就把自
己想考一个专业认证没有软件界面练习的情况对舅舅说了,他还装模作样地在一张白纸
上写啊记地,我就直说,“舅舅我对你说这些说了你也不懂,你带我找表哥去。”

        舅舅带我来到表哥的student office,表哥看见我就先笑了。舅舅先简单地说了一
下,表哥便用他的电脑在网上搜我那软件到底是什么东西,我站他旁边不远的地方,他
一边搜一边拧着嘴傻笑。后来大家决定表哥带我去图书馆电脑上看一下他们学校有没有
这个package,舅舅说我办完事还可以去他office上网学习。临走前表哥去洗手间,我
便耍小聪明讨讨舅舅的欢心,我说以前是学生,给舅舅买\$60一磅的茶,这次我工作了
,给舅舅买的茶也升了一个等级,这次给舅舅买了些好茶。我话还没说完,舅舅一脸不
屑地拉长了脸,倾刻间头也扭向一边侧背着我不愿看见我。当时我极为尴尬,加倍买礼
物是他要求的,我不过是把它说出来好让他开心,他怎么就这么个表情?!当时能想出
来的就只上次临走时他让表哥等我,他们可能觉得表哥对我有情有义我却从来不曾考虑
过吧。所以那时就想,为了让舅舅开心,也给自己一个机会,打算这次与表哥处处看。
表哥回来后他们每人拿了颗巧克力我们就与舅舅分开了。

        我们学校以前有坡的地方我都常常“爬”得气喘吁吁,表哥学校的坡更多。我问表
哥还有多远,他也说不清有多远,我就让表哥帮我背书包,里面放的是公司的笔记本电
脑(其实完全可以放表哥office,但不知当时哪根筋搭错了竟然背着)。去了可能是上午
不开门吧,下午几点后才开,我们就又折回来。我自然是不去舅舅办公室了,便在表哥
这里呆下来。他给我找了个笔记本我可以上网。上网上烦了,我就开始骚扰表哥了。
\section{表哥表哥(2)}
\label{sec-7-4}

       我问表哥什么血型,他说是O;过会儿我对表哥说(右手搭在他左肩上),我中午约
好了几个上学时的朋友一起吃中饭,表哥你陪我一块儿去好不好。表哥说好。我满足了
跑回去上网了。过了会儿,我又跑回来说表哥你这里有什么好玩的?“好玩儿的啊?”
表哥想了想,打开一个装满照片的文件夹,我搬把椅子坐到表哥右边,表哥就帮我讲解
起那些个动物园的小动物们来。大象很大,讲讲讲讲到一个园子里斑马和邻居孔雀的故
事时,表哥说,他们在一个园子里相处得久了,他们之间不说什么不做什么行动上也有
了默契。他给我讲他拍那张照片时的情景。表哥说最开始那只孔雀(早上想来?刚吃饱
饭?)懒洋洋地一边傻站着,斑马就愣是不长眼睛朝孔雀的方向走过来,眼看着斑马就
要踩着孔雀了,没有早一步,也没有晚一步,孔雀挪动了一小步就避开了。我听着想着
觉得好玩,没想到表哥这个汉语讲得结结巴巴的ABC居然还能讲出这么好玩儿的故事,
便津津有味地听着。后来是一些小动物,飞鸟之类的。我想对表哥说,大象斑马太大了
,小鸟麻雀太小了,我只喜欢猫兔子大小的就好了。但怕打击表哥积极性,就没说。后
来表哥打开一些大表哥家两小孩的照片给我看,很可爱。还有就是表哥自己拍的小鸟和
夏天舅舅捉回他们家院子的野兔(舅舅知道我喜欢兔子,去加州的路上给他讲过农科院
时园子里看见过一只干干净净的小白兔地上吃草顿时眼前一亮),我总算知足了。

        表哥给我讲那天他怎么改机票特意去了动物园(最后一学期舅舅家吃第一餐饭时舅
舅就提到过小动物,但他怎么说的全忘了),而后他如何如何走了很远很远,走了至少
一到两迈才领到当地两颗free的巧克力。表哥走近他的小冰箱,拿出一小袋里面只有两
颗、装在一个充了气鼓啷啷的塑料袋里的巧克力给我。我接过来拿在手里揣摩端祥着,
当时确实有向表哥表白并吃掉一颗的冲动,但这一切还是太快太突然了,我还没同家里
打好招呼,我还得再想想,便很无奈地把巧克力还给了表哥。或许表哥热切地注视过我
,或许他真的失望了,折回来后,我们还坐在并排的椅子上,椅子之间相隔的距离也不
曾改变,但表哥开始写他的code,有一种明显的台风过境的疏离。我是自私的,我怎能
容许表哥就这么从我的世界里消失?!就算没表白我也还是有想法的嘛,我双手抓了下
表哥的右胳膊,他不动,继续写他的code。当时心里只有一个想法,我是喜欢表哥的,
所以什么都不用怕,我就继续抓,他不动我还抓,从大胳膊顺势往下抓到了他右手,又
抓住了他左手。这下他满意了,很开心地说,“我们去吃饭。”我有点儿惊魂未定,好
险!

        我们到朋友们还没到,我先去洗手间;出来时朋友们和表哥分坐在两个桌。朋友说
他还问过我表哥,我又何尝知道你的名字?!朋友们聚了聚,聊聊天,吃了饭,表哥是
最后的考试周,他可能比较急,我们吃完饭也就很快散了。来到office,我问表哥你需
要写code还是陪我去图书馆,表哥陪我去图书馆。雨过天晴,滋润清新;春风拂面,芳
草戚戚。在表哥坚定的目光里,我就像一个舞动的精灵,轻盈灵动,旋转在空旷辽阔的
大草原上。到图书馆表哥先帮我找洗手间,出来时表哥站在“橱窗栏”前,胳膊上搭着
衣服,那个等我的意境感觉很美,一如他等了我十个月等到我归来。喜欢着一个人的世
界竟是如此的完美,完美到容不下图书馆地面上的一张卡片,也容不下归来路上的任何
一片垃圾。回到表哥office已经三点多了,我答应过舅母早点儿回去,便与表哥匆匆作
别,“表哥,我先回去,你晚上早点儿回来!”自顾自地跑掉了。
\section{舅母}
\label{sec-7-5}

        昨天路上赶了一天,很辛苦,到早上起床都没洗澡,早上起来问舅母,舅母却说她
早上洗浴,我下午再用浴室。不好同舅母争辩什么,就只能忍受到下午了。三四点钟回
来第一件事先把自己洗干净。

        等我忙完,舅母还在准备晚餐。等所有蔬菜准备停当(其实也就两盘蔬菜,炒时加
稍许瘦肉),要开炒了,舅母说这炉子还有点儿小姐脾气,时好时不好的。舅母用煮汤
锅炒菜,用筷子当锅铲,看得我那个着急。但这是别人的家,我又能说什么呢?说不准
我走了,别人就不用这汤锅了。舅母把我买回来的湖南腊肉蒸了一袋瘦的,蒸熟了舅母
也没再改刀切片,大大小小一疙瘩一疙瘩的,她让我尝,知道是舅舅的最爱我不敢多吃
,捡了块最小的尝了尝。

        我问舅母,之前舅舅说小表哥结婚了住在我上学的小镇,那他现在也还在那里?舅
母说小表哥住在楼下,表哥他们住楼上。等舅母把晚饭准备好,我把可以先洗的餐具洗
掉,舅母就给表哥打电话,问他什么时候回来,说是六点,还有一个小时。舅母就先给
舅舅盛了一盘打发他先吃,舅母也先吃了,可能是希望给我和表哥创造机会吧。

        这次开车回来非常辛苦,中间一段山林地带雨水很多,视线不好底下又滑,并且刹
车灯又亮了,刹车都刹不稳。我对舅舅说,要他帮我看一下。舅舅当时就去看了,我同
舅母在living room看球赛。舅舅很快就跑到living room来说是要找一个什么油并告诉
我刹车灯已经不亮被他修好了。真的吗,我很欣喜,但对舅舅来living room找什么汽
车油感觉很奇怪。

        舅母说起家附近一个什么类似”工厂”的地方,表哥毕业后,舅母说希望他就在附
近能在那里上班就好。舅母给我讲那时候她对表哥非常严格,从来都要求他自强自立,
从多大起就自己攒钱养活自己。舅母说因一件什么对表哥用钱格外苛刻的事她现在还有
点儿后悔,如果当初她不对表哥有那么严格,表哥或许不会远走他乡(具体是不是远走
他乡,是什么事情其实我没明白透)。

        舅母一边看电视,一边往手上脸上擦护肤品。看我在看她,就势便来了句你皮肤不
好什么的。这话把我给气得脸上红一阵白一阵的,接下来的球赛,她再帮我请解什么的
我都云里雾里灵魂出壳了。
\section{曾经的朋友}
\label{sec-7-6}

        好不容易等到表哥回来,舅母交待过我给他盛汤,我便照做了,也给自己盛了碗。
四角方桌我们面对面坐下来。兴头上我说表哥我想喝果汁。表哥绕过来,把冰箱里有的
四五种果汁全拿出来摆在我面前让我选,我选得很受用,选了瓶蓝霉果汁,表哥说这个
剩很少了,你可以抱着瓶子都喝了,我便照做了,心里暖暖的。可是我给表哥只舀了汤
他便只喝汤。表哥放下了碗筷我也不好意思再吃,晚饭就没吃饱。

        总觉得同表哥还是有些话需要讲,可是怎么讲呢?洗餐具前我就问表哥,家里有没
有无线网,呆会儿等我忙完,我们一起看个小电影说说话好不好?表哥说好,他要先问
一下舅舅无线网的密码什么的。我很开心就赶快收拾橱房。

        等我忙完时表哥去洗澡了。舅舅不知是出去溜了狗还是散步很快便回来了。再见到
表哥时,他穿了高领长袖上衣,牛仔裤,同我类似衣服的小清新相对照,他的衣服穿得
有那么点小性感。

        等我同舅舅说话报怨到感觉接下来一年很辛苦时,忽然想起好久不见了哥哥,便扔
下舅舅,这次就直闯表哥房间了,打开灯,表哥已经躺下了。我走到表哥床头,右手冰
了冰他的脖子,说,“表哥,就你不能鼓励鼓励我吗?”表哥睁开眼,看看我幽幽地说
,“你早点睡吧!”无奈便只能折回去,毕竟别人要休息了。走着走着踩上一脚什么东
西,低头一看,表哥刚穿过的衣服就随便扔在地上,乱七八糟。

和舅舅打好招呼我也该休息了。躺在床上,想起白天里的表哥,也想起了一个朋友。

       她属马,双子座,O型血,和表哥的完全一样,她是我硕士时最要好的朋友。星座
上我从来没有查到过这两个属相星座组合(属马双子与属羊狮子)的结果,但一直感觉我
们两个很亲密。在北京硕士三年,后来我差不多又工作一年,四年的时间她有很多朋友
,也有了男朋友,我只有她一个朋友,但她一直陪着我。她说过她和我作朋友只是因为
我单纯,从来没有坏心眼去害别人。她也对我说过,“XX,你知道吗,你身上最宝贵的
品质就是善良,不管遇到什么困难,不管在社会上经历多少磨难,你都要保存保护这一
点,永远不要失去它。”我们一起吃饭、逛街,骑车去金五星买水果。硕一她还没男朋
友我们都还没得到过什么宠物时,也干过各花15元(rmb)买一个毛茸茸的熊送给对方过
。只是相对于初三我的精神需求来说,硕士时的我已经很强大了,对别人的精神依赖几
乎没有(除了父母),对她,我享受着她友情的陪伴,但我对她的回报,回忆起来终究觉
得还是太少。对这份友情最早的回忆留恋是在来这边第一年的孤独痛恨里,但环境将我
们分开,过去的也就只能过去了。我们在QQ上淡淡地联系着。我想,等明年我有机会回
国,应该会去见这位朋友,也看看她的宝贝孩子。

        表哥,从我这个朋友的经历上讲,是大一号的异性朋友(大一轮),那表哥这个12岁
、性别,与我曾经的朋友相比,又会带来什么不同?想着想着我便睡着了。第二天,同
表哥的告别,我得到了答案。
\section{Moment}
\label{sec-7-7}

  接下来的周一,我去学校把自己该办的事情办好了。回到舅舅家,我一边请舅母帮
我热点吃的,一边赶紧去收拾自己的行李。舅母说中午都不到吃什么饭,类似这类大打
折扣的话,我很难受说“算了算了,我自己路上买了吃”便不再理采她。过了会儿她自
觉无趣,又凑过来问给我热点什么吃,我便让她热点剩饭剩菜不用我路上浪费时间找吃
的就行。过了不知多久,舅母来了句,“你上次走的时候那些个橱房里的瓶瓶灌灌还要
不要?”我答得轻松,“舅母,那些我走的时候就对舅舅说过我不要了,你们看着帮我
处理了就行。”舅母随口一句“我把它们都扔了!”正常受过教育知书达礼的人会说这
么刻薄的吗?既如此,我二月还没喜欢表哥,搬那些瓶瓶灌灌那天走时你又为何要巴巴
地倚着门框目送我离开?对舅母的预防针在她说这句话时算是打下了。临走了,她又说
,“我把车库门打开了,你的行李如果需要你可以拿。”我想着虽然生舅母气,但同表
哥还没完没了,等下次吧。舅母又说,“反正也这时候了,不耽误那几分钟,吃饱了再
走。”她这话提醒了我,反正这时候了,不差这几分钟,一定要再见一下表哥!

  从家里出发前就背着舅母给表哥打了电话,很快就来到表哥楼前,表哥出来接我去
他office。Office里没有别人,想表哥抱抱我,他不肯;拉着表哥的手,带着哭腔说,
“表哥,我晚上没休息好,我心里难受,我不想走!” 蹲在地上快哭出来。表哥在给
一个什么人打电话,我也管不了那么多了,靠在表哥背上哭起来。哭了好几分钟吧难受
得也差不多了,便松了表哥的手,从后面抱住了哥哥。两手臂上一阵温热,哥哥还是徒
然地放下了他试图掰开我的手。我在后面嘟嘟囔囔地说,“表哥,我觉得接下来的一年
好辛苦!”边说边把侧靠着的头调了个方向,就这样静静地抱着。我还有要紧话要对表
哥说,便转到前面来,表哥也不再躲闪,顺着我,我顺势双手揽住了他的腰,身体贴着
他说出了最亲密的话,“表哥你喜欢我吗?”“我把你当妹妹。”没防备表哥会说话,
话音刚落,“可是如果我也喜欢你呢”我的话已崩出来。“可是我还没想好,我不知道
该选什么样的人。”我接着说,“以前都是舅舅支持我,表哥,以后你要支持我、鼓励
我。”表哥这里很温暖,紧挨着的头又调了个方向,想了想我又说,“接下来的一年,
我没心思谈恋爱,等我把工作换了转了身份,我会想谈恋爱,会考虑感情问题,到那时
我应该也会想清楚了。”我知道自己干了件世界上最自私的事,想了想又定定地说,“
我知道舅舅、舅母对我俩这事的态度,等我想好了,表哥,不管我有没有选你,我一定
回来跟你说清楚!”为什么我会说这么多的话,为什么表哥都不肯抱我?我终于还是耐
不住了,“表哥,就算你把我当妹妹,你就不能抱抱我吗?”边说双手在表哥后背也忙
碌起来。可是表哥还是不肯。(在此奉劝广大单身女生们,没把握千万别试这个,否刚
你后背刮阴风别怪我!)无奈我就只能再拉拉表哥的手。表哥很温柔地说,“没休息好
应该中午回去睡一下!”我智障吗?所谓“大跌眼镜”,眼珠都快掉下来描述的应该就
是我当时的感受吧,想来昨晚我走时表哥听到我略带试探的话可能也是这个反应吧,所
以他才拖拖拉拉很晚回家!我本能地迎向哥哥的目光,说,“基本上还能开得回去。”
表哥导师进来了,我们及时松开了手。“我该走了,表哥你送我出去吧!”表哥给我带
错了门,“从这里出去我找不到我的车。”表哥停下来问我,为什么接下来一年会辛苦
,我就解释了一下工作的事; “要一年吗?” 表哥问得真诚真切充满期待,我知道自
己干了件最自私恶毒的事,本能地想要减轻他的痛苦,答说,“半年,大半年!”“你
呆会儿还回去吗?”“不回去了。”“路上不要超速,开车要小心!”表哥带我找正门
,我们牵手了。看见第一个人时我们松开了,但终究还是紧紧地握在了一起,对走道里
的学生视而不见,世界仿佛只剩下我俩!到门口,我说,“表哥,我要走了!”“小心
开车!”我扣上外套,走出了大门。回头望时,表哥还定定地站在那里,眼里充满期待
,我一阵心酸,眼底升起一股迷雾,眼前已是一片蒙胧。

\section{后遗症}
\label{sec-7-8}

  是渴望表哥一个拥抱时,还是在开往加州的车里,我想明白了表哥一定是喜欢我宠
着我的,他那句拿我当妹妹的话说得是那么地言不由衷。而我,在表哥坚定的目光里变
得轻盈自如自由自在,与表哥之间的亲密又让我觉得我们牵手、拥抱,无论怎样都不过
分、不生分。与表哥的这场分别里我的娇气、表哥的宠爱、我的那份享受宠爱的逍遥自
在,有那么一个moment,我觉得表哥就是那个我内心里渴望的人,照顾我,宠爱我,让
我变回小时候那个被父亲宠爱的女孩。

  我,一个有着快乐的童年、烦恼的少年、脆弱成长的青少年,到今天三十一二岁了
一直还是一个不懂感情的狂妄“少年”,但与表哥这短短一两天的相处我所有的隐性情
感基因都在表哥的诱导下、环境刺激下一一表达了,仿佛身体里的每一个细胞都在呐喊
:“我喜欢表哥,我感觉很幸福!”与表哥的这场告别,我终于明白自己心底的渴望,
也开始明白为什么恋人分开时需要他们的moment, 明白表哥就是我心中的“沧海”。

  在表哥学校图书馆,我没找到我要的软件。来到加州后,我”贼心不死”,同表哥
学校我专业的小秘联系了一下,知道有一个教室学生可以练习使用那个专业软件。我就
写邮件给表哥,请他去这个地方帮我查一下他们学校有没有我想要package。邮件里,
我告诉表哥他浴室里的小红面巾是我从中国带来的,这里买不到,我很娇气地对表哥说
,“Dear cousin, could you please send me my towel after it is dry? 表哥,表
哥\textasciitilde{}~\textasciitilde{}”

  等啊盼啊,12月21日我收到了表哥的回信。表哥说他去那个教室帮我找了,没有找
到那个package,我的面巾他写信这天也寄出来了,我应该很快能收到。表哥的信写得
很平淡,言简意垓,没有任何感情表露,读起来却散发着阵阵温情。

  我心里有了表哥这个秘密后,便迫不及待地想要同姐姐们分享,也想征求她们的意
见。家里所有人的意见一致,表哥比我大13岁大太多了,又是近亲结婚。给二姐打电话
,姐姐说,“就算你觉得13岁不算什么,万一将来生个傻子怎么办?”我回答得很干脆
,“现在医学已经这么发达了,很多症状怀孕期间已经可以诊断;我们家簇没什么遗传
病,生傻子也只是小概率事件;万一万一退一万步说,真正生出个带点傻的来,傻人自
有傻福气,我们能够享受到自己爱情的幸福就该知足了,天才孩子也好,傻孩子也罢,
他们自有他们自已的人生。”我还给姐姐举了我们五姨(我妈妈的亲妹妹)与五姨父和
我一样四代近亲结婚,表妹只是两脚各有六个趾头(后来手术割除了)而已好让姐姐信
服。因为遇事我有自己的主见,所以就像说服二姐一样,我说服了家里所有的人。他们
心里可能还是有那么点不顺,但也不好再说我什么。

  13岁差距究竟意味着什么我不懂,我只知道我们是相互喜欢的,无论如何我舍不得
表哥。
\section{立场 }
\label{sec-7-9}

  我问过了家里所有人的立场和态度,即便他们有反对过的,全被我说服了。但舅舅
、舅母这边的立场真的像我所看到、所感受以为的,他们一定会支持吗?我不确定。当
表哥成为我心里、日常生活的一部分时,我与表哥的亲缘关系也成了我重要的参考资料
。我向妈妈打探了我们家族关系所有的细枝末节,结论都一样:妈妈的爷爷与舅舅的爷
爷是同一个人!妈妈的意思是说我与表哥近亲关系太近,她说我不信可以自己问舅舅。
但她这话正好被我用来试探了舅舅、舅母的立场。

  从舅舅家走后不久,一天他们晚饭后时间我就打电话到舅舅office, 舅舅在,我
就同他聊天。我对舅舅说,当年斗地主,舅舅随舅爷(舅舅的父亲)离家躲走时舅舅大
概只有十岁左右,妈妈当年还没出生,妈妈说她记不清家族里的那些关系了,我问舅舅
我们这近亲关系到底是怎样的?舅舅说出了同妈妈一样的答案,舅舅说我的外公是外公
他们四弟兄里最不爱学习的,所以大舅、他们(舅舅指他自己家)家都出人才,我外公
这边的妈妈那一辈基本都流落为农民。我顺势问了舅舅,假如我同表哥恋爱了,他们作
为父母、舅舅舅母会是什么态度,舅舅说,“我既不支持,也不反对。”可后来的事实
证明,正是这个说着既不支持,也不反对的舅舅在不该支持的时候支持,不该反对的时
候反对。

  我也向舅舅打听关于表哥我想知道的内容,比如,表哥会不会牵女孩子的手。舅舅笑了,他说他也不知道,表哥应该会的吧。除了这些,舅舅还简单地讲了些表哥曾经的恋爱对像们,有航空公司的空姐,有国内高干子女,有身高多少的模特,也有什么什么朋友的什么人。舅舅在个什么不打紧的情景下不打紧地加了句,“他以后结婚了不要小孩都可以”,我不能理解为什么表哥结婚了可以不要小孩,但大家众所周知的原因,听到舅舅这句话,我心里那个舒坦劲,实在是无以言表。

  其实早在舅舅家吃三餐饭(09年秋)的时候,舅母讨好我向我报怨过表哥作息时间
不科学,又高又瘦呢又不锻炼身体,舅母盯着我问我觉得表哥这习惯好吗?当时对表哥
身材好长得又帅更多的是一种羡慕,只能很惭愧地对舅母说,“表哥这样挺好的,我太
胖了,我该运动减点儿肥才对。” 话虽这么说,我还没从表哥那里得到足够的信号,
怎么可能为他减肥,只顾忙自己的考试去了。到这次见表哥时,表哥还是那时我说话时
的样子,不黑不胖不白不瘦。可是现在,表哥在我心里扎了根,就想起那个13岁来,表
哥现在锻炼吗?12月30日,我收到了舅母帮我从邮局寄来的毛巾,装在一个牛皮纸信封
里的无色透明塑料袋里。我很开心地给表哥写邮件,告诉他我收到了自己心爱的红色毛
巾,并说“Dear cousin, recently my head is always full of somebody. 表哥,do
you exercise nowadays?”
\section{分裂}
\label{sec-7-10}

  从舅舅家与表哥相处一两天回来后,我有过一段非常幸福快乐的时光。表哥就像我
脑海里、心底的一个影子,工作的间隙、休息时他都会崩出来。当然最幸福的,还是晚
上洗涮完毕,静静地躺在床上,与表哥相处的一幕幕场景就会像放电影一样在我脑海里
一遍遍反复重现,陪伴我甜蜜入睡。这些场景里,有表哥诱导我表白未遂时我认认真真
看他的第一眼,有以为我不喜欢他疏离时我的强力挽留,有楼梯间我蹦蹦跳跳跟在表哥
后面时表哥放慢脚步刻意的等待,那时真有冲动想挽住表哥的胳膊;有与表哥告别时的
每一个动作、每一句话,表哥的每一个反应,也有牵手时表哥拇指在我手背上的摩嗦,
回想起来,很长一段时间里,我手背上还一直停留着那一瓣手指的温热。(回想起来,
国内硕士期间同自己导师之间那点儿破事儿究竟算不算个事儿,又能算是个什么事儿,
现在想起来真是可笑!)

  这段时间我心里很美,美极了,但我的理智却始终还是与情感状态步调不一。是的
,我用自己的主观感觉、开明态度和坚强意志说服了我周围所有亲人和朋友,但我心里
真的就没有哪怕一丝一毫的犹豫吗?有的,一定是有的。我对表哥非常满意,但我对他
的家庭充满畏惧。

  因为申请出国时受到小表姐及其家人的帮助(小表姐为我提供了我可以考虑申请的
学校名单;我将申请材料寄到小表姐后,大舅大舅母帮我分装到信封里邮寄到几所不同
学校),我来美国后就给小表姐家打了电话。06年快圣诞节我打电话时,是舅母接的电
话,舅母为人处世极其圆润,便嘘寒问暖,问我舅舅跟我联系没有,我答没,舅母很奇
怪,就告诉我说舅母一听妈妈说我来美国就告诉二舅要他同我联系,舅母说她很奇怪这
么久了舅舅都没同我联系。当时学校里“小三”舆论应该已经暴发,可能是自己自卑吧
,我当时也没多想。

  去年12月去舅舅家时还有一件小事忘提。走之前我告诉表姐们我要开车去舅舅家,
大表姐说要我开车小心,去了给她打个电话。那个周日傍晚当我给表姐打电话,舅舅正
好也在,我问舅舅要不要同表姐讲话,舅舅接过电话,我在旁边不远,说是无心就当我
有意吧,听见舅舅说他的晚年就指望表姐们了。这,不是舅舅又一次如同要求我加倍买
礼物一样对表姐们赤裸裸的要求吗?说良心话,作为他们家事的外人,我不知道事情详
情,只是觉得舅舅把小表姐带出来读书,如果小表姐真正就从来不曾感谢过舅舅,那她
有点儿错;但从另一方面,圆润的大舅母从他们长辈这个情分上也多多少少有帮小表姐
还了些情吧。97年,98年,我所知道的,至少01或02年中的一年,三四个暑假舅舅都是
在大陆大舅大舅母家探亲度假,大舅母很会做人,客人在他家从来都是客人点菜,大舅
大舅母对舅舅的情应该也算不薄了吧?我想打电话的两人和我三个人都清楚地知道说那
话是舅舅一厢情愿,虽然舅舅对大表姐这么要求也只是为了减轻我的心理负担,但考虑
到人性的另一面,我感觉到的是舅舅有点儿可怕。

  后来,见了舅舅很多面,第一次买豆浆机感谢舅舅时还给舅母买了保健品,但我一
直连舅母是不是中国人都不确定,因为从来没见过。舅母是伴随表哥同时出现在我的世
界里的。认了舅舅接近两年的时间里我应该到过舅舅家四五次吧,从来不曾见舅母,要
见便是同表哥一起见。舅母这个在我的世界从来不起任何作用的人,舅舅借我两个\$
2000支票上却是写着她的名字。小表哥当初住我上学的小镇,及至这次与表哥相处问及
舅母,他已住楼下,表哥他们住楼上。我不谙世事时给舅舅买第一份礼物是舅舅暗示(
逼)的,到我快毕业,舅舅问我要加倍的礼物也开始用说的。

  这生活中的细节点滴真的就只是偶然吗?我很难相信这一点感到很害怕。在外界舆
论压倒性地认为我应该理所当然同表哥在一起时,我果敢地跑出去和朋友们打球,并与
他们共进晚餐,因为我心里实在是有所不安。但我的果敢又再一次地激起了人们的义愤
,“小三”,自私,没有感情,――,又一场更为深入的人肉运动开始了。

\chapter{我与大舅一家人}
\label{sec-8}
\section{我与大舅一家人(1)}
\label{sec-8-1}
  自打我记事起,我们家基本上是不怎么与大舅一家有来往的。原因很简单,他们家
富贵,我们家贫穷,我们家攀不起也不大愿意攀这样的亲戚。记得有一次在家吃晚饭,
妈妈说刚刚看见大表姐从我家门前经过。那时我才上小学,大表姐应该也只二十多岁吧
,妈妈猜测她们一帮人是去走表姐外婆这边的亲戚。因为妈妈给上初中的二姐讲表姐们
热爱学习发奋图强的故事,心里好渴望一识庐山真面目,但妈妈这样我也不敢,只能把
想法埋在心里。二姐中考报志愿时,妈妈也会打发姐姐来舅舅家请教舅舅,到底该报什
么样的志愿才好。在舅舅的建议下,姐姐读了医。

  虽然外公这边妈妈一辈的远近亲戚与大舅大舅母家都不大走动,但大舅大舅母却实
实在在是在我生命里帮助过我的人。

  大舅是个热爱学习的人,后来表姐有机会走出门,舅舅那个年代可能是学俄语的人
也自学起了英语,他们家有电脑,也有很多有声读物。我这个远亲的侄女也跟着沾了点
儿光。初一寒假,我告诉妈妈我想去舅舅家用电脑学英语,妈妈同意了。舅舅舅母也愿
意帮我。这样,我每天早上骑自行车或走路到舅舅家,就只用电脑听听力看书学习,中
午在舅舅家吃饭,下午傍晚的时候回家,这样打扰了他们大概过年前两三个礼拜。过年
我们家杀猪时妈妈给他们家送了条哪个部位的猪肉和一条猪腿。

  后来高二高三的事情前面也都有提及,那段时间只有二姐一家同舅舅舅母家走得近
,后来也疏远了。大学的某个暑假舅母要我教大表哥家孩子学数学,就在舅舅家玩了好
几周,舅母送我了一些表姐不再穿的衣服。06年元月时(大概是这个时间)舅舅舅母动
身到美国小表姐这里打算呆三年,经过北京时,舅舅舅母还招呼我与他们家其它亲友一
起到餐馆吃餐饭。舅舅说如果我的早请材料准备好了,他们可以帮我带到美国帮我分寄
。我当时可能还没准备完备,但后来UPS寄到表姐家舅舅还是帮了我。

  从08年暑假到毕业前我先后到加州三次,每次回来都能找到舅母在我箱子里悄悄塞
下的红包,分别是\$200、\$200和\$150元。因为我也有打搅到他们家生活,08年临走时我
同大表姐一起去超市买菜,那个\$130-\$150的账单是我付的。09年舅舅舅母回大陆后,
我也让妈妈买过100个鸡蛋、50个鸭蛋和一袋新出的大米去看望舅舅舅母。

  或许真像舅舅所说,我有错吧,后来在与表姐们的交往中逐渐有了磨擦,关系疏远
。
\section{我与大舅一家人(2)}
\label{sec-8-2}

  后来我来到加州工作,舅舅舅母已经离开回国了,但我与表姐(大表姐)的矛盾却
迸发出来。在我找到正式工作前我在小表姐家住了一个礼拜(大表姐也在小表姐家住着
),周五我的manager打电话面试我,周日大表姐就把我“赶”了出来。别人也没有说
一定不让你住,但不论你说什么做什么别人都同你作对,你还好意思再住下去么?而周
日我搬出去住后,周一我就接到了工作offer。但那时我还没有同他们做到决绝地分开。

  我没工作时经常周六大表姐带我去豆腐店买豆制品。她一般一次大概买一壶豆浆,
两大盒豆腐脑,几个春卷,一块炸豆腐以及变着花样的一两样东西。她买得相对多,我
与大表姐拿回家前一般合吃一盒豆腐脑。剩下的他们四口之家小家庭享用,也给他们留
些私人空间。工作后,我便照做。周末如果去表姐家便先上豆腐店买好类似的东西再过
去。但问题是五六次过去都是大表姐一人在家,我走后小表姐面前这点儿乱东西算谁买
的还不知道,买了几次就兴趣索然,不想动了。小表姐可能觉得我从来不回报也生分了
。有一次大表姐要回加拿大(她们小家庭加拿大国籍),带我去买衣服。大表姐说让我
办张卡好省10\%,她160多刀账单也要我刷卡结。买完东西她问我回家还是怎么办,我就
说去表姐家玩。我一直不提走的话,后来没办法她说给我支票,我说你写160就可以了
。我拿到支票后才回家的。

  后来恶化这段关系的是去年临近圣诞节。表姐来我家找我出去逛街买圣诞礼物。08
年圣诞节我来了加州,所以两个表姐有给我准备礼物,大表姐买了一件衬衣(非纯棉)
,小表姐的是一小袋化妆品。我知道我一定是欠大舅、大舅母的情的,但欠他们长辈的
情我宁愿我明年回国自己带礼物去看望他们,而不是任随大表姐宰羊子。这圣诞节的礼
物我是只买两表姐的,还是要买小表姐一家的?有大表姐陪驾,他们的礼物我能轻慢吗
?我告诉表姐下个月我要考试,还没复习好,我没时间去。

  表姐不甘心,逼我还她\$2000。这钱是她带我去开CITI账户向她借的。她说开账户
能挣\$75bonus。她带我去BA开账户,小表姐refer她,她refer我,这样小表姐只得25,
她75,我50(新开账户的人得50,refer的人得25);CITI好像是类似挣point。开账户
是她要带我去开的,还钱也是她逼我还的。我当时刚从舅舅家回来,路费、礼物、申请
费、下个月的考试费用,一张罚单、考试机票,CITI REQUIRE 6000存款,若还了表
姐的钱,我的钱一定是不够的。我对表姐说,下个月今天我就可以还你,但今天我的钱
不够,能否再晚一个月,她逼着我写支票给她。写完那张支票我不送客,她自己走了。

  我想,等我有机会回中国,我还是一定会去看望舅舅、舅母的,但对表姐们,心里
的这两道伤痕一直都有。
\section{起争执}
\label{sec-8-3}

  那段时间想念表哥的时候其实也给舅母打过电话。对舅舅、舅母我还不敢认我与表
哥之间的这些小密秘。有一次试问舅母,问她表哥有女朋友吗?舅母说,表哥有两个女
朋友,他上大学时的哪里有一个,韩国还有一个。舅母说,表哥都快订婚了哦!我说如
果表哥订婚了要舅母告诉我一声。

  可是我信吗?我不信,表哥才不会喜欢别人呢。在舅舅家吃三餐饭时,舅母在饭桌
上想故意刺激我,问表哥之前提到过的一个什么学生后来有没有联系,当我的面,表哥
说哪里有,根本没那回事儿!

  有一次同舅舅聊天舅舅问到表姐他们,我就把发生的事情说了一遍。舅舅当时的反
应是什么我忘了。后来一段时间因为元月我确实有考试,所以也没再多打电话。到考完
试回来,同朋友吃饭时,又聊起了舅舅舅母。以前舅舅、舅母说什么做什么我只当他们
是对我有过些帮助的人,我只需做到我应尽的份数就可以了。但如果能与表哥发展顺利
,那么他们就将变成我的公公婆婆,所以很多之前不需要格外考虑的问题都不得不将他
们提升到公公婆婆的立场上来考虑。

  舅舅带我上表姐家时,说是不愿意打扰别人,带我吃饱了下午才去,可是真正在表
姐家,舅舅不是照样喝酒、喝到醉哭,把饺子几乎他一个人全吃了吗?我学包饺子,当
时舅母我都还没见过,舅舅不是也先就暗示(“按排”)我做饺子侍候舅母吗?说话不
是也摆到台面上说要我加倍地买礼物吗?如果舅舅是这样一个爱发号施令的人,我这样
一个同样强势、同样有着强烈主观意志的人,舅舅还要我们同住一个屋檐下,早晚我可
能都会同舅舅吵架干起来。而这样一个想作我将来公公的人,在我来这边的第一年,就
因为外界舆论认为我是“小三”,他就不来找我,他到底是怎么想的?

  当时舆论分两派,道德论者认为就算没有爱情我也应该因为感恩同表哥在一起;人
权论者认为爱情是我的权利,有分歧的情况下,只能与道德分开考虑。我也相应地同朋
友说了我的观点,我这样一个对亲情、友情有着深刻体验、清晰感受的人,又如何能在
爱情上将就?如果同表哥没有感情,就一定不能强求我同表哥将就。想作我将来公公的
舅舅在我来第一年不肯找我的事,我想不通,也同朋友说了说。当时气头上,话就说得
很重。大概是说舅舅明知道表哥在韩国有段混乱的日子,他明明知道我就在旁边学校,
舅舅却不肯来找我,就好像舅舅就是要故意这么做好让我有机会作践自己,最终我才不
得不放低我的标准才能将就接受表哥一样。

  这气头上说得究竟算什么话,就算是当时气头上说的,现在想来这话也实在是太重
了,舅舅自然不是这样的人。可是就是现在此时此刻敲下这些字的时候,我还是不得不
承认,这个当初想不通的事,最终还是被心机重重的舅舅设计成了对我的伤害,当然这
是后话。

  应该是流言传到了舅舅的耳朵里,后来我给舅舅打电话时他态度180转变,说我性
格不好,嫁不出去,没人要,说我是骗子。当时的我应该没能想通这些,当时是生气极
了,虽然对表哥恋恋不忘,觉得曾经沧海难为水,以后自己可能都没法幸福了,但自己
自尊心受到伤害,觉得与表哥那等待大半年的承诺也该是时候结束了。因为有过承诺,
我的行李留下了;现在该结束承诺了,我也该去把我的行李拖回来了。
\section{二月归去}
\label{sec-8-4}

  就像大家所感觉到的,有一个事实我一直不愿意客观面对的就是,我与表哥本人的
联系很少。除了(except, 不是besides)给他打的第一个电话,给他写的第一封邮件
他回了之外,后来再给他打电话他不接,给他发邮件他不回。但那会儿我还在兴奋头上
,总是麻醉自己说是舅舅拿着电话呢(他们父子俩共同一手机)。

  那段时间可能更适合用“惊艳”来形容我对表哥的感情吧,惊艳于表哥这个“大一
号”的异性朋友对我稍微的关心照顾爱护一举一动都是那么地契合我心底的渴望丝丝入
扣,以至于回到加州后我执着了很长一段时间坚定地认为这个世界上茫茫人海里只有表
哥对我是真心真意不打任何折扣其它所有人待我只是因为我有工作我很上进我现在或不
远的将来能挣不少钱,也常心里暗叹人与人之间的感情是如此地不同。

  也正是因为这份执着,虽然因为刻薄的舅舅打算放弃了,但还是觉得我应该让表哥
知道我的想法,我放弃之前也应该给他一个表明他自己立场的机会。于是回舅舅家拖行
李前,我还是一如既往地给表哥写了封邮件。我告诉他,“Dear 表哥,I have 
something HUGE I want to let you know, but I would rather tell you face to 
face.”

  不到两个月前刚给舅舅买过一份双倍的礼物,谁有多少钱要天天买礼物?所以这次
去舅舅家就只买了四袋干蘑菇和两袋湖南腊肉,总价小于\$40,只求做到不空手到舅舅
家就可以了。

  那天是周六凌晨十二点(还是一点左右,大半夜的)出发开的车,一路泥泞到晚上
六七点钟到的舅舅家。按门铃后,开门的是一个矮个儿结实点儿的男人我不认识,我问
表哥在吗,那人进去招呼后出来的是舅母,舅母说,“你不要再找表哥了!”在我眼里
她就是个被舅舅宠坏的女人,我才懒得理她,就进去living room找表哥,他与舅舅在
看球赛,表哥长胖了点儿,能看得出脸上饱满多了。上次回来表哥说他在韩国的时候偶
尔会自己做面条吃,表哥当时坐在我上次坐过的地上,我就拖着表哥的胳膊要他帮我煮
面条,一拖就知道表哥变结实了。表哥不愿意,我有话要问他,自然找别的借口把他带
出去,我要表哥载我去超市买明天回家路上的干粮。舅舅装腔作势地劝表哥不要带我出
去(为了反衬这家人里就只表哥待我最好?)还特意交待表哥,不许带我去表哥office
。我很奇怪难道表哥两颗巧克力的事舅舅也知道?但疲惫交加也顾不上想太多,就坐进
表哥的车走了。
\section{表白}
\label{sec-8-5}

  在车里面对的只是表哥一个人,我便央求表哥带我去办公室。开回舅舅家的一路上
,我又何尝不曾想过向表哥表白后的种种可能的后果,我又何尝不渴望得到表哥一个
kiss,一个拥抱,那地点当然是表哥office最好,因为那里有两颗巧克力,一颗巧克力
开始一段完美恋情,另一颗引领我们走进婚姻。表哥不愿带我去office。就要结束了,
那会儿还真是不到黄河不死心啊,我告诉他我有话对他讲,要他带我到一个我可以讲话
的地方,他在停车场停下了息了车。

  我想我是真诚的,表哥的表情可以照鉴。他微侧着头,静静地看着我,皮肤白皙润
泽,目光清澈明亮。我告诉他,我喜欢他,想和他以后永远生活在一起。我向他解释上
次说我没有想好是因为我觉得这件事情太大了,关系到两个原本有着亲缘关系的家庭,
所以我需要慎重考虑。回去后我同家里所有人都讨论过了,大家都同意。表哥不相信我
似的说,“他们都同意了?”我老老实实地回答,“我对家里所有人所有的姐姐爸妈都
说了,他们最终都同意了。”那一刻表哥双眸明亮波光流转我永远都记得。

  “我十年之内都不会结婚!”表哥顾左右而言他。那一刻我瞬间“白发”,低头眼
泪一下子就涌出来。即如此,当初你又何苦搞两颗巧克力在那里诱惑来诱惑去的?披散
的长发遮住了我的脸,坐在副驾位置的我哽哽噎噎地哭起来。等我平复了情绪,毕竟与
表哥亲密,转过头来,破涕为笑地说,“好丢人啊,现在我姐姐他们都知道了,回头她
们要笑我了!”表哥见我不哭了就问我上次走时是怎么回事,他老先生对上次告别的事
居然不认,你不认我也不认,我就也不把它当回事儿地说,“很早之前有个算命先生说
我三十二岁这年有小人害我。我不知道是什么小人,能把我害到什么程度什么地步,我
害怕!”一边说一边“瑟瑟发抖”地望着表哥,表哥又问了些细节之后就带我去超市买
东西了。

  刚刚过去的尴尬被我很快忘掉,我们推一辆够物车在各走道里穿行。即便有时我自
己推车,表哥也会伸出一支手来援助我。我们像极了兄妹,亲密快乐;超市里的路人可
能猜测我们像极了情侣,亲密无间。我们还是很引人侧目,不过谁还有精力去理会那么
多呢?选瓶装水时,一瓶高的一瓶矮的,高的挺拔俊俏,矮的敦厚老实。我一口咬定我
喜欢高的,表哥说他喜欢矮的。表哥比划着说高的重心不稳很容易倒,水流得到处都是。
我看看表哥笑着说,“表哥,你也像这瓶水一样细细长长的,”我一边比划一边笑,“
你有没有想过有一天你也会像这瓶水一样一不小心倒掉的?”表哥听我这么说,大家都
笑了。

  最后一学期刻苦学习时吃过几次核桃很好吃,我就央求表哥带我去我之间的小镇去
买。谁知表哥随便问个人很快就找到袋装的核桃。我很扫兴啊,就唠叨我那核桃夹子也
不知在哪,这拿回去可怎么吃啊?谁知表哥早打起了核桃的主意,告诉我说其实就用手
捏也能捏破的。我不信,那得要多大的力气才行?表哥就给我示范,他一只手拿住袋里
两核桃,稍一用力就听见炸开了,我们一阵大笑,我也对表哥佩服得不得了。
\section{理解}
\label{sec-8-6}

  那天我头很痛,感觉快要炸开了。忘了是在去超市购物前还是之后开动的车里,昏
乎中听见表哥提醒我一句,“其实我也可以带你去office”,引得我又强力思索一番。
舅舅带我第一次去office找他时,他傻笑了两次;表哥只接我第一个电话,只回我第一
封邮件。之前我说他就那样挺好的,他不是一年多来都不曾黑一点儿白一分长一磅少一
两吗?我12月30日写邮件问他现在锻炼吗,他现在就长白长胖长结实了。想清楚这些后
,我就对他说,“表哥,我不信,你今天说过的所有的话我都不信。”表哥见我清醒过
来,就给我讲了他弟弟从大陆江浙一带的哪里讨了个媳妇过来,说是出去打工,后来与
家里再没有联系了。按表哥的意思,他们家是一朝被蛇咬,十年怕蛇毒。表哥说希望我
能理解一下。

  我计划的是回来拖行李的,在彻底放弃前希望表哥知道我的想法,给他表明选择他
立场的机会,而他却要我理解一下。表哥的话说得真诚肯切,我沉默了。也许表哥面前
当时的自己是默许的,但回家之后,在舅舅的一手操纵下,表哥眼前的默许终究是变成
了舅舅面前的默拒。

  一路上我还是在琢磨这事。这次回来,表哥明确变化的一点就是不许我再“碰”他
。我们之间亲密,之前不论是拉拉表哥的手,碰到哪里他都不会说什么,但这次,他明
确要求我不许碰他(这次一回家就拖他胳膊除外)。我猜测这一定是舅舅的杰作。

  变化面前我显然没有足够的心理准备,回家后我就说了明天一早就走,现在就开始
把行李往车里装。他们家人世俗,就像舅舅总会装模作样装腔作势无非是想在世俗社会
里免受舆论谴责,有一样感触最感的就是他们家人特会讨好人。可能是听我说觉得舅舅
贪心吧,舅舅帮我搬行李时,把他们家车库里的其它东西也往我车上搬。舅舅真没注意
吗?我宁愿相信舅舅是故意的。可能是因为我气头上把舅母称作“老太婆”,老太婆就
讨好我,给我煮面条,里面加了三个饺子,可能还有一个鸡蛋记不清了。到我回加州后
清理车上杂物时发现,舅舅不知什么时候把我上次买到他家的部分没折包装的干粮装了
一小袋又悄悄地塞回我车里(舅舅有我车的备份钥匙)。

  爱情的伟大之处就在于1+1远大于2。能够找到一个情投意合、性格相辅、共鸣默
契的 soulmate便能充分调动人的主观能动性,激发无限潜能。而这个投契的程度便决
定了是等于、大于、远大于,还是无穷大。而怕就怕,两者相加变成负值,双方的缺点
都被无限放大。

  自认为表哥与我,至少从我个人感觉上说,应该属于远大于。而舅舅与我这对外人
看来的“忘年交”朋友,在面对我与表哥的感情这件事上,朋友的关系也变成了负值。
\section{泪水}
\label{sec-8-7}

  这次回来,舅舅家饭厅的四角方桌不见了,添置新红木样式的长形轮廓大餐桌。搬
完行李的心情沉重复杂难受。就像有强迫症一样,我掏出书包里的支票本,给舅舅写了
四千的check以还清上学期间从他家借出的债务。至此,这次回来的任务才算基本完成。

  舅母要我吃面条,舅舅却要求舅母陪他一起坐到饭桌上来给我上政治课。舅母虽然
平时对我不怎么样,但她早看出我的难受,她才不会坐上来”羞辱”我,她橱房里穿进
穿出劳心劳力地在整理我房间的床铺被子。

  舅舅在我旁边呈90度角坐下来,开口就批评我怎么可以那么aggressive,我怎么可
以抱表哥?是不是人疲倦的时候更容易掉眼泪,舅舅这话一出,我的眼泪就断了线的珠
子一样涌出来,落进面汤里。我怎么就不可以抱表哥,我与表哥亲密你们又不是看不见
?!舅舅接着说,我理解你一个人的艰难,可是这并不意味你可以不择手段blahblah,
我的眼泪又一阵儿大珠小珠落玉盘;那时对舅舅是眼冒金星的恨,我所有的事情除了小
学毕业那件和 “小三”我觉得耻辱不曾告诉你之外,我的什么事情你不知道,什么性
格你不明白,我什么时候不择手段过?这个舅舅,因为他一直对我有照顾,我也从来就
只是顺着,从来不曾顶嘴忤逆过,可他这话说得,我一个字也吐不出来,唯有泪千行。

  我以为我沉默舅舅就会打住了,他却接着说,“表姐的事情也是,我看就是你错了
!”我好不容易嚼上两口,舅舅这句话,我的眼泪又开始淅沥哗啦地掉;“大表姐就算
了,小表姐多好的人!”舅舅有完没完,今天我得承载多少顶帽子多少条罪证他才善罢
甘休?我就算不心疼我的眼泪我还心疼我的面条呢,我都还没吃饱。“我不care行吧?
!”我开始试图堵住舅舅的嘴。“小表姐多善良的人,你怎么跟她也闹翻?”是啊,小
表姐多善良的人啊,是谁对我说他因小表姐感情受伤害!你自己也知道大表姐什么样的
人,你不知道小表姐家住着个大表姐小表姐一切杂事都有大表姐护航啊?“我说过我不
care了,你还有必要再多说什么吗?”“你就不能把这里当成一个度假胜地,闲的时候
遇事不顺利的时候来舅舅这里坐坐修养身心?”“我不愿意行吧?”终于是小半碗汤面
往桌上一砸再也吃不下去了。“那算是我们有缘相识一场!”话说到这份上,我也是再
说不出一个字来,洗了碗便该走了,控制不住的,除了对表哥的感情,还有泪水。

  正是因为舅舅的极端话语,我自尊心受伤害回来拖行李;我自然是敏感脆弱的,我
自己的爸妈姐姐们批评我从来都是陪尽小心,可舅舅偏不理会这些,一顶顶帽子压得我
喘不过气。这里的”家”、这份亲情在我受伤害的情绪里早已变成了牢笼,多呆一分钟
一秒钟都是谋财害命荼毒生灵。舅舅要留我当晚在他家住下,我不愿意。花五十块钱在
旅馆住下会比住他家轻松自在很多。就要走到尽头了,人与人相识也是一种缘份,舅舅
跟着我的车送我很远。

\section{血雨腥风}
\label{sec-8-8}

  有一次打电话回家我告诉二姐一人我被人肉了。姐姐问我能不能去他们人肉我的论
坛把一些事情解释清楚。我就对姐姐说网上什么样的人都有,隐藏在马甲之后说任何话
只图说得痛快就行,根本不用负任何责任。就算我真正找到那个人肉我的论坛,估计最
先崩溃、最受不了的人终究只能是我自己,何况我还真不知道大家人肉别人到底是在哪
个什么样的论坛。姐姐认同我的说法,也就作罢。

  一波波一轮轮滚滚浪潮之后,最新实事下的不择手段论、新小三论又开始了。而追
其原因却是因为二三月份我换新工作有一位同专业的前辈帮忙refer了我,中国人,中
国男人。人们指责的原因是周末我们两人共吃过一餐午饭。那餐饭确实是我的错,因为
我没能考虑到那么多。我在国内共工作了大半年的时间,共计三份小工作,小公司,工
资不高,那时工作也只图有件事作,能力所能及地养活自己便可,找得毫不费力,不涉
及任何高深复杂的人际关系论。来美国后那同专业前辈帮我refer的是我第二份正式工
作,之前的一份也找得轻松。经验也好,教训也罢,我明白了就是了。

  在一波波浪潮来临时,之前一起打羽毛球的一位朋友一直在帮助我、鼓励我。

  早在去年十二月我来美第一年极端痛苦下不人不鬼的经历也被曝光后,他就“打”过我。那段时间,很多常规打球的人那段时间球场上都看不到了,他一直都在。那天下午,三四个人双打,也是打球打累了,我回头对搭档的他说,“XX,我打不动了!” 几分难受求助,几分脆弱娇柔。接下来打球不多久,他“不小心”打到了我的头,接下来的一下午,他多次问我还好吧,打得重不重痛不痛。真的不是很痛,即使头皮会痛心里也是温暖的,不是吗?

  三四月份(三月底四月前几号),当血雨腥风再次来临,他,又一次地帮了我。他
有一个女粉丝小梯队,他利用他自己的影响力号召力大家都来打球,而我两之间关系又
显得稍微近一点儿。

  一波尚未平息一波又起,他,我们毕竟只是普通朋友,他应该是顺风顺水长大的普
通孩子,比起一般同龄的年青人他已经算是很成熟了,可是自己处在困难面前,我又如
何能够强求他一定得支持我?他,有他的自由。

  在全世界都认为我滥不择手段新“小三”的时候,我知道有一个角落能够容纳我,
那个因为我office里一句话被订下千里迢迢只为与他发生一夜情罪证而我们什么也没有
、那个一朝被蛇咬十年怕蛇毒希望我能理解一下的表哥。舅舅也说过希望我能当他们家
是一个可以为我遮风挡雨修养身心的地方。我该回“家”了。
\section{四月归来}
\label{sec-8-9}

  与表哥相处的机会总是很少,那天也是情绪比较冲动下班后(七点左右)就直接出
发去舅舅家了,这样第二天(星期六)中午十一点左右就到了表哥office,表哥在。我
开口就说“表哥,我太困了,我想在你这里先休息一下!”表哥意外惊喜又烦恼了一下
,不知跑出去哪里干嘛去了。

  表哥的小床上铺着一个纯棉布料的sleep bag,但不知为什么他把乱七八糟的衣服
全铺在上面。我要休息,这么多乱衣服可怎么办?我就只好稍微折一下堆了一厚叠放在
床头。然后在sleep bag杂层中躺下来,可是还嫌不够暖和就把表哥蓝色外套盖在sleep
bag上很快睡着了。

  醒来后已经是一两个小时之后了,表哥在写他的code。见我醒来,表哥先问我想不
想吃东西,他帮我用微波炉热了两块pizza,里面加了几片他切好的黄瓜片。我穿红色
高领毛衫、蓝色牛仔裤、运动鞋,表哥也穿了高领的毛衫,蓝色牛仔裤和运动鞋,和我
穿得几乎一样。想到我一来他就跑了,可能跑回家换过衣服了吧。我吃pizza的时间表
哥既说他忙,他却开始收拾他的桌面抽屉,把所有的打印资料书本等都排放整齐。

  这个人是两个月前我执着了两个月的这个世界上待我最真心诚意的人,虽然他希望
我理解的时候我被舅舅的帽子仗轰走了,虽然我们之间,我心里对他的热情(激情,惊
艳感觉)早已消逝,但再见到这位表哥的感觉却还是一见如故。我还是忍不住要问他那
个他一直不曾直接回答我的问题。

  表哥嫌我烦,他说我上次(去年十二月)每隔一段时间就拿什么乱七八糟的破问题
去烦他,他最烦那种了。那我就答应表哥我只需要占用他几分钟,问完这个问题我就再
不烦他。他坐到门口离我坐在床上十万八千里的地方让我问,我便强求他坐近坐到他办
公桌前的椅子上,这样他的表情、反应我也才能看得清楚。他照做了。

  “表哥,你到底喜不喜欢我?”这是我第几次问他了,我也快问烦了。表哥说,“
我早对你说过了,我把你当妹妹。”你那第一次说的当妹妹能算数吗,有一边说当妹妹
一边要问中午应该回去睡一下的吗?可是这里是office,能够感觉周围来学习办公的师
生不少,这话也就只能在自己心里恼一下,望着眼前近在咫尺的表哥,任凭眼泪落下哽
咽起来,“我接下来几年都不回来了!”我当时说这话绝不是赌气,因为要彻底忘掉这
个真心真意待我我又实实在在感动过的人太难了(现在全世界就剩下他这一个是我心里
的依靠了)。记忆里当时表哥一脸痛苦,一如表白那晚他清澈明亮的双眼让人过目不忘。

  两个小时我是一定没睡够的,既然别人并不喜欢自己,那把自己的痛苦掩藏在被子
里不是更好吗?我躺下来,哽咽还在继续,等我平息下来,脑袋里一根神经隐隐作痛,
像无数个来到美国这边第一年的晚上一样,我哭累了很快就睡着了。
\section{希望还在}
\label{sec-8-10}

  手机是上午休息时定下的下午四点,闹钟把我闹醒前接过一个工作中介的电话,朦
胧中告诉他我找到工作不需要了,又睡着了。后来分不清是电话还是手机闹钟,最终醒
来时已是四五点钟。休息前是因为表哥只把我当妹妹弄得自己哭得哽哽噎噎,哭累了才
睡着的。可是等我醒来,还没坐起来,侧躺在床上,看见表哥,我又看见了希望。

  表哥也不坐椅子,就蹲(跪蹲?)在自己脚上,两胳膊刚够得着桌子就都趴在桌子
上了,头离电脑显示器不到半尺,眼睛更是瞪大了紧紧地盯着屏幕。于是之前发生过的
与表哥之间如此种种又在我脑海里过了遍电影:他特意改迁机票去了动物园,他走了一
两迈去拿巧克力,第一次来office找他时他傻笑了两次,他诱惑我表白未遂时台风过境
的疏离,图书馆归来路上的垃圾,那场告别那个moment,表哥为我维持了一年现有身材
以及后来应我要求积极锻炼,他也有近期恋爱结婚的打算,以及刚刚过去的他也收拾自
己的东西和我说几年之内不再回来他的痛苦,于是我再一次地相信,表哥是喜欢我的,
表哥一定是喜欢我的!表哥可能只是说不出来,或者对他来说时机还不到,总有一天,
他一定会亲口告诉我的。想着想着,觉得很开心,也开始觉得自己就一小P孩什么事都
不懂浅薄得很连个问题都不会想!

  “表哥给我两颗巧克力好不好?”我对表哥的两颗巧克力恋恋不忘。表哥说不行,
不过他递给我一大块长方块巧克力让我吃。看见包装纸还没折开过,我吃得很开心。不
过,人家说大专生要房子才结婚,博士生用根草环婚就结了,表哥真用两颗free巧克力
就把我搞定了?这两颗巧克力对他真有那么重要吗?“表哥,我就看一下可以吧?”表
哥同意了我就跑去冰霜,说是迟那时快,一找到那袋子我就开始拆(掰),可惜啊腰间
冒出来一只手来一把把它夺走了,还差点儿打我,看他认真,斗智斗勇斗不过表哥就只
能承认它的重要性了。

  接下来我就问了表哥他接下来会有什么打算。表哥说他暑假会找工作,能找到工作
他就毕业,找不到他就继续读。之前十二月正热同舅舅打电话聊天的年月里问过舅舅表
哥什么时候才能毕业啊,舅舅一口打消我的顾虑说,表哥有六十多篇文章,想什么时候
毕业什么时候就能毕业,想来即使十二月不毕业,明年夏天也该毕业了吧,胜利在望。

  大概是我先问了舅舅在不在家吧,表哥就势问我,“舅舅是谁,你晚上自己找旅馆
住去!”我知道表哥是在生气说气话,就想着先给舅舅、舅母打个电话。他们都没有接
电话,不知道是不是舅母号码变了(手机在表哥这里的话,舅舅如果不在办公室我就找
不到舅舅)我便问了表哥。表哥反问我,“谁是舅母?”我很生气,不就是那次气头上
叫了舅母“老太婆”吗,至于气这么久吗,便答说,“你的妈妈!”“谁,谁是舅母?
”表哥拿足了气势又问了一遍,“你的妈妈!”我的语气更重也快生气了。他见我还理
得清,就翻了舅母的号码告诉我。没有变,我再试一次,舅母还是不接。我就只好对表
哥说,“表哥,呆会儿晚上回家吃饭时你带我回舅舅舅母家,好不好?”我基本算是承
认错误,态度也还不错,表哥答应了。于是我便静下来用表哥给我找的电脑上网,等到
表哥说回家我才再开口说话。
\section{甜蜜}
\label{sec-8-11}

  回家舅舅舅母都没有说什么。舅母做了很少的吃的,反正是吃了点儿东西没饿着。
饭桌上还是只有我和表哥。我要表哥把之前那个三餐饭时我没听懂的笑话表哥又给我讲
明白了,忍不住想笑。期间舅母在饭厅里穿过几趟,不过我已经对她视而不见了,我猜
想舅母肯定有份特殊的类似侦察员的工作负责专门向舅舅汇报我与表哥恋情进展。

  饭后前院里,表哥逗他家的狗做各种表情动作我看得很开心。后来表哥就给了它
treat,也喂了它些晚餐。之后看见舅舅他就要去溜狗了。之前舅舅身强体壮健朗得很
,但那时看来已经变成了一小老头,消瘦憔悴了很多。我对舅舅说这次回来匆忙没来得
及给舅舅买礼物,下次回来会补上。舅舅说我能回来就挺好的,不需要我买什么礼物,
舅舅说话时显得很真诚。

  再之后,就见到二月回来时帮我开门的矮个儿结实点的男人,表哥对我介绍说,“
这个也是表哥,他是你小表哥!”表哥边介绍边看我,他明明知道“表哥”在我这里是
专有名词,他还要说其它人也是表哥,气得我对他直翻白眼。他就跑藏回家里了。

  我再回来时,表哥答应我晚上会再去office呆会儿。我的衣服穿得不够,晚上出去
会冷,表哥就把他自己新买的一件深色外套让我套上。我们穿着同样款式的衣服各背一
个书包走在校园小道上,远处不时有学生侧头,叽叽喳喳“指指点点”,大概在说我们
快结婚了吧。

  我们来到office后表哥自己也想再休息一会儿。他把那些个我折起来的衣服又全都
铺回去了说那样才睡得更舒服,说着从床底掏出一床被子。我佯装生气,“表哥你有这
么好的被子都舍不得给我用!”“我回来时你已经睡下了。”表哥见我想气便真给我点
儿气受,他把被子往地上一扔,又去整理个什么东西,我急啊,“表哥你怎么可以把被
子随便扔地上!”与表哥恋爱的感觉可真好啊,没有他这么一扔,我又如何知道我还可
以如此娇嗔?表哥躺下了,我帮他掖好被角,看他闭上眼睛。表哥可真帅啊,尤其是 
那两道高耸的眉毛。简爱那话是怎么说来着?blahblah我一定把你变得像我现在离不开
你一样离不开我!作家的经历穿越时空此时的我也能感同身受。“走开啦!”表哥把我
轰走,我便关了灯,静静地在电脑上作自己的东西。表哥休息了一个小时左右,手机闹
钟响后他自己也在床上又赖了会儿。表哥后来又作了两个小时的project,到十点我们
就回家了。

  我老说舅舅什么事情我不理解,我自己还不是常常说话做事只说到一半害得大家瞎
猜。十二月时就因为表哥打电话问舅舅无线网密码舅舅很快就回来,害得我该说的话没
得说,第二天跟表哥告别把话说了舅舅又嫌我“不择手段”;这次吧晚饭时问舅母她什
么时候休息,舅母说了几点几点,我就是想问一下看我晚上回来洗澡的话会不会吵到她
们,结果我后半句没说,到我们十点回来,舅母房间的灯是亮着的,估计她跟舅舅一样
又想歪了。

  舅舅舅母都休息了,我与表哥就只能用一个洗手间了。表哥让着我推说他用得久让
我先用,我洗涮完毕来到表哥房间里,表哥坐在椅子里,感觉目光很温柔。我告诉表哥
我已经摘了隐形眼镜,看不大清楚。同表哥度过了轻松愉快的一天,就要分开了,我一
定是舍不得表哥的,就站到表哥面前贴得很近对表哥带着哭腔撒娇说,“表哥,还有,
还有,我明天早上四点起来走之前想洗个澡,我那时候洗会不会吵到舅母吵到你们?”
表哥安慰我说不会,让我不用担心,不会吵到他们的。“表哥你明天会四五点钟起来送
我吗?”“我可能会起来。”有了白天“妄自菲薄”的反省,我相信表哥这么说他明天
一早就一定会起来送我的,“那我去休息了!”娇滴滴地同表哥打好招呼就该走了,表
哥很宠我,一句“Good night”释放出无限柔情,我心里美滋滋的甜蜜极了,头一挨着
枕头就睡着了。
\section{再试探}
\label{sec-8-12}

  第二天起床有点儿失望。我四点起床收拾好房间里的东西才去洗澡。等我洗完,表
哥房间还没有任何动静。我进去,打开灯,表哥穿着毛衫的大半个上身露在外面睡得还
正香呢!那会儿还真是有过冲动想要干点儿什么浪漫的,比如爬到表哥身边躺一会儿,
可是这个“床”字又太敏感了点儿。二月舅舅想留我当晚住下时说过他早上五点就醒,
没准儿现在墙的那边早竖起了一对耳朵,那事可千万干不得。无奈之下就只能拉了拉表
哥的手,嚷嚷着表哥起床,要表哥快起来等我出去他好锁门。表哥一句知道了便又睡着
了。我又唠叨了一遍想想舅舅可能也该起床了,表哥实在想睡就让他吧,便自己一人走
了。回加州我一路上都愤愤不平,表哥这次为什么不起来送我?

  更愤愤不平的事还在后面呢。等我来到加州,我早上走表哥没起来送我也就罢了,
他老先生居然上午十点多快十一点时发封邮件严厉谴责我闯他房间很rude。我就开始在
心里好好琢磨这事儿:去年十二月回来周六的晚上同舅舅说话说着说着就想起好久不见
了表哥,便闯他房间从他床上找到了他,跟他说完话从重新回去找舅舅,舅舅和表哥都
没有说什么不是;二月回去时在living room舅舅和表哥都在看电视球赛时我拖表哥胳
膊,舅舅和表哥也都没说什么;这次连office里给他掖被角他都没说什么,只在我发花
痴的时候轰我走开,没有感觉有什么太离谱的呀?

  这事事发突然,应该同舅舅没有关系,那就只能从表哥身上找原因。表哥,表哥,
昨天晚上表哥的“Good night”很温柔,昨天晚上表哥的目光也很温柔,我摘了隐形眼
镜也还能感觉到这一点儿,表哥昨晚会不会是对我有所期待?想想十二月走时表哥说过
的话,二月我是被舅舅帽子仗轰走的所以就完全没有机会。如果真是这样,邮件里表哥
真要把我只当妹妹严格要求了,表哥的邮件我该怎么回?在无助的时候我就跑去查星座
了,星座上说双子座由爱到性是自然而然的事情,表哥喜欢我,那他有那样的想法对我
有所期待不是很正常吗?

  表哥的邮件是把我当妹妹,批评我不守妹妹作为客人应有的礼仪私闯他房间。这从
另一个方面理解,会不会是表哥在push我,相比于十二月同表哥告别时说“我不知道该
选什么样的人”,现在是不是也到了我该做一个彻底的选择、该做这么一个决定的时候
了?我是喜欢表哥的,那回表哥这封邮件,我一方面要守住自己作妹妹的立场以免引起
表哥和我以后相处的尴尬,另一方面也一定要让他知道他在我心目中的特殊地位,一定
得让他知道我只有在得不到他的爱的前提下才能甘心屈居妹妹的角色。

  这些思路都理清楚之后,我就洋洋洒洒地给表哥回邮件了。我很想把这次来回邮件
贴出来,但在与舅舅一次次“战争”之后我一次次盛怒之下我与表哥舅舅的过往邮件就
只剩下有迹可循的最后两封。所以现在能写下也只有我写过记忆比较深的几句:Dear 
cousin, you are the most cherished cousin in the world, please don’t drive me crazy. Fine, I will obey your rules. I will knock your door before I enter 以及解释那天早上我必须敲他门的原因是我无法从外面锁门,而我又礼不应当吵醒舅舅舅母,所以表哥是我唯一可以敲门的人了。
\section{再决定}
\label{sec-8-13}

  表哥应该是从这封信里获取了他想要得到的信息,后来到五月底回去之前,这一两
月里我大概又写过三四封信给表哥,主要是告诉他我喜欢他。四月份血雨腥风回舅舅家
时office里表哥就责怪过我回去之前也不先打好招呼,他反问过我我就不怕哪天回去他
不在啊?我打算五月底长假的时候再回去一趟,也确实担心表哥暑假会出差,所以就给
表哥写了邮件,告诉他I am planning the trip to visit you in the end of May。后来提前一两周同老板请假,并及时向表哥更新了我的行程规划。

  以前在表哥office里表哥大概都是让我借用他老板的笔记本让我上网,但这次回去
,老板批准的是我周四和周五可以work from home。所以周四我赶到表哥那里时是需要
工作的,而我这次就一定得用自己的工作笔记本。邮件里我特意交待表哥查询好一切无
线网申请步骤,我到后需要十几分钟、半个小时之内可以上网。我告诉表哥,如果有必
要,我可以提供我笔记本的physical IP address.

  可能我见色忘友吧,自从去年十二月迷上了表哥,舅舅这位昔日一直帮助我的朋友
就被我本能地打回到退居二线,舅舅这位原本和蔼可亲的“谦谦君子”在对我与表哥的
感情事情的处理上在我心里也开始变得面目可憎。十二月与表哥告别时嘴上讨好表哥我
感激的是“以前一直都是舅舅支持我”,而我心里叹的却是之前舅舅从来都没有真正走
进过我的内心,从来都没有从真正意义上成为过我的精神支柱,遇见表哥,我找到了我
想要的依靠,遇见表哥,真好!

  因为后来同舅舅产生矛盾,这里需要特别交待一下。十二月走后一直到二月回去前
那次舅舅骂我“性格不好嫁不出去没人要是骗子”的电话之间,是我最狂热迷恋表哥的
时段,也是我与舅舅打电话最多的时候。后来感觉舅舅总是好暗示我、把我框得太紧,
同他的电话就越来越少。五月中旬在进行回舅舅家的购物准备时,有一天大概也是舅舅
吃完晚饭在办公室的时间我打电话给舅舅,稍微聊了聊表哥。电话里我告诉舅舅我有五
月底回来一趟的行程安排,电话里,舅舅淡淡地说了一句,“我不欢迎”。但我与表哥
正在兴头上,我也写信告诉过表哥,表哥没回邮件便是默许的。所以,电话里舅舅的话
我并没太在意。五月底的星期三下午,我早早地兴冲冲地出发来舅舅家了。
\section{长途旅行}
\label{sec-8-14}

  也说说这几次长途开车自助旅行吧。十二月我是被逼回去的,被自己无法等待的性
格逼回去的。学校IPO老师因公开会出差,加上她自己生病,还有就是她一个人处理所
有这些事务,忙不过来,情有可原。我猜测既使我打发了family plan的朋友帮我催她
也没有加速办理,这里面当时可能有希望能够帮助成就一桩美事的美好意愿在。二月的
那趟是早晚必回的,因为承诺;四月的回去是我自愿的,因为血雨腥风,身边没有一个
可以支撑抚慰我的人心里确实非常难受。

  这三趟里,开车最难的是十二月,回家途中几百迈山林地带雨水极多,车轮打滑而
且视线模糊,一路泥泞真的是开车开到想哭,太难受了。回去时没来得及车检,走的前
一天family plan的这位朋友还要我去那些店里找找看能不能找到防滑带,他说可以防
滑,下雨也可以用。我嫌麻烦也没时间就没去。舅舅对我寒冬腊月里开车回北方冰冻地
带关心的话一个字也没有。十二月回去时刹车就亮起了红灯,舅舅说他帮我修好了,回
来时没开几迈灯就又亮了。对车舅舅是有着极大爱好的,但应该只是业余水平,所以舅
舅修不好刹车灯我没有丝毫可以怪罪他的地方,一定要说不满,应该是对他跑到living
room去找什么汽车油邀功最为不满吧。回来时因自己过于急躁吃了一张加州\$355罚单。

  二月回去前我就特意做好了车检,换了机油,也换上了放在车里很久的雨刷,并把
车轮作了对调,修车师傅说我车后面两个轮子比前面的好, 对调过来用防滑效果更好
,二月回去除了雨刷摆动不够顺畅外,其它基本还算一路顺利。放行李时舅舅坚持我的
书等两箱重物应该放在前面副驾位置开车更稳,结果第二天早上刚出住旅馆的小镇就出
现了踩刹车车会哽死的症状。刹车刹不稳,一路开回去陪尽了小心很难受。一回加州就
去店里检查了车,换了刹车皮。师傅说车会哽死可能是发动机太老太陈旧了,将就着开
段时间也该换车了。四月回舅舅家前也检修了车,换机油换了新买的雨刷。四月的天气
已经大好,这次往返开车都很轻松,除了我还是每走200迈都要加一次油比较浪费时间
之外(因为油表不准,怕油烧没了)。

  这次五月底回去,我还是老老实实地作好车检换机油。为方便自己的旅行,我买了
五加仑的汔油壶,想省时间免得我一路上单程都得加油四次。但后来听同事说大夏天车
存放汔油在车上很不安全,加上自己不会用汽油壶,就基本买了也没用。不过五月底这
次回去我按照同事的建议耐心地试就发现我车的油表还是在走的,只是因为米表活塞有
阻塞,所以走得不均匀。我往返路上一次比一次多开一点儿一直试,回加州后再上网查
了一下我车的盛油量竟然有16.9加仑。容我小叹一下,这真是一次伟大的试验和发现,
因为之后我再也不用频繁地加油还真是省事啊!
\section{五月归来}
\label{sec-8-15}

  遇见表哥后我对舅舅家基本就没什么留恋,我每次回去也都是先去表哥office楼附
近停车先找到表哥,表哥再在前面开车带路把我领回家,待我把车停到了舅舅家,表哥
就会充当司机我再不用开车,我接下来只要跟着表哥混就好了。

  这次回来,一切照旧。大上午快中午的,我找到表哥便嚷嚷着要表哥带我回家,因
为那段时间我住Chinatown,那次回去除了茶(besides,天仁茗茶每磅\$60的冻顶乌龙
茶培火的与清香的各300克,店家推荐300克包装好的比称散装的好,茶总价\$80左右)
,这次我没去超市买干粮,而是从Chinatown采的“鲜货”,我住学校时不方便购买的
各式中国蔬菜水果,叉烧肉、一只卤鸭,两袋冰冻大虾仁和两条冻成冰块的鱼。这次买
的礼物不是很值钱,装在一个大冰盒里,总价只在\$150-\$170,同去年十二月自己开开
心心时买回去的礼物是没法比的。表哥说其实我也可以不必中午回去,他说他知道什么
地方可以免费停车,但我想着冰盒里只有一袋冰,都过了这么久了,还是拿回家放冰霜
里的好。于是表哥就带我回去了。

  舅舅一开始是不在的,舅母在,舅母见我买了礼物就说要给我买礼物的钱,我买得
东西很简单买的时候我也压根儿就没曾想过他们会给我什么礼物的钱,之前也从来没说
给过,谁会要她的什么礼物钱呢?表哥说她身体不好要去哪里跟医生有个什么
appointment,所以舅母急着出去我也不怎么在意。我就让表哥帮把东西放进冰霜。

  舅舅家他们长辈说什么我都只顺着,可能我自己个性还是比较冷傲孤僻一些,所以
一直不喜欢同别人分辩argue什么,也一般不能包容别人的缺点。但同表哥因为心里喜
欢,有什么话我都会同他讲,也基本能做到包容。大夏天的开了一夜的车,身上很不舒
服,我就同表哥商量说,“表哥,你急着去office吗?你能不能等我一二十分钟,我身
上不舒服去之前我想先洗个澡。”我说话语气真诚恳切,与表哥已经很亲密了,站得离
他也很近,表哥就答应了。我洗好出来时表哥在逗他的狗,看见我,就领我到橱房指着
餐桌上的蛋糕一定要我尝一块,表哥说那是他知道我要回来亲手做的。蛋糕本身烤得并
不是很好吃,但我应该吃得还算有过感动吧。家里面表哥浴室的外层装饰性浴帘和橱房
里的窗帘由以前白色的换成了大(深)红庄重典雅色的了。

  坐进表哥车里看见舅舅回去了。我们就快开离家门了,表哥说舅舅在把我买的东西
往门口摆。我一下子就火了,舅舅就只我四月回来时说过不需要我买什么礼物的话,之
前他一直都是自己要礼物的,我怎么能及时反应得过来?当时自己说过什么我已经不记
得了,应该是有过报怨吧,表哥说,“他们两人闷家里斗,可能最终舅母会把舅舅摆出
来的东西再捡回去吧。”表哥即这么说我也就不往心里去了,就想着,只要我们晚上回
来你们老两口折腾够了把东西捡进家里了我就当没这回事。当时路上我大概也借这件事
向表哥表达过我对舅舅的不满,具体说了舅舅什么我忘记了;就舅母给钱的事,我自尊
心强又好面子,也补了句,“以前我给舅舅买礼物他也没说过要给礼物钱啊;再说了,
就算舅母要给礼物钱,我什么时候给舅舅买过七十块的礼物?我什么时候买礼物少过一
百块钱(其实有的,二月份回来拖行李时就<\$40)?”表哥接过我的话说,“是七十
啊,我还以为是四十呢!”表哥这话到底是在帮我还是在帮他妈?这钱原本也不是什么
大事儿,我就打住没再说什么了。
\section{五月归来(2)}
\label{sec-8-16}

  来到office表哥老板和一个同学都在。这个表哥欺人太甚,以前我巴巴地央求过他
多少次要他帮我申请他们学校的无线网,他就作作样子试了两下从来都没真正帮我解决
问题;这次因为我要工作,表哥三下五除二不到两分钟就把无线网帮我连好了,很神速。

  中午我问表哥我都快一天没吃过米饭了,我们出去吃炒饭好不好,表哥嫌烦,不好
!我再说吃面条,也不好!最后我问那pizza吧,结果表哥就把责任推给了他老板,最
后是表哥的导师带我和表哥同学出去买pizza的,表哥闷在office里不知道他在干嘛。

  表哥的老板人极好,很热情,他让我品尝他家乡的茶(好像是碧螺春),看上去绿
油油的,表哥后来也尝了杯,真的是好清香,比我给舅舅在店里买的\$60名茶好上十倍
都不止。老板说那还只是往年陈茶,每年产的新茶比那个还要好喝。我小时候看电视电
视剧中间总是插播中国酒文化,喝了老板的茶开始体会中国也该有茶文化,中国的茶文
化应该也是博大精深吧。表哥ABC,汉语就讲得结巴一点儿。老板知道我小丫头片子就
迷些什么偶像剧啊电影啊之类的,Office里我们三个地地道道的中国人就聊起了各种电
视剧,经典电影,人文的人性的,爱情的可能没涉及。就在我们大侃<士兵突击>各种经
典名片的时候,表哥还接到一个骚扰电话,只听见表哥“Hello, this is XX”招牌语言之后就挂断了电话,这不知是哪家的姑娘找上头来,我狠狠地翻了他两个白眼,也注意到表哥今天没有像以往一样与我穿同样款式的衣服。老板和表哥同学对那些经典片的理解感觉比我要深刻得多,我还请表哥同学帮我从家里带一部特经典的什么电影的DVD(我忘了名字)到office来这样第二天(星期五)我就可以抽时间看。不过后来回舅舅家后的事情证明,没有第二天了。

  那天因为要工作work from home,我得把老板布置的那点儿活干了,加上星期四,
表哥office里老师同学都在,我是没法休息的,只是云里雾里的干完活就听他们侃大山
了。后来对表哥说想晚上早点回去吃饭,想晚上能吃上米饭,想傍晚好好休息一会儿。
表哥答应了。

  回去路上我问表哥要十二月份他秀给我看的望远镜,我还想再看看。记得当时我没
拿望远镜多望什么,只是拿在手里揣摩,心扑哧扑哧地跳,我从来都对浩瀚星空星星月
亮充满了莫名神奇的向往,感觉我握着的不是望远镜,而是一份美好向往,茫茫人海里
,我遇见了你,携手走完我们或许并不十分精彩却相亲相爱的一生,散发着秋收的的质
朴气息。我很想再看看那个望远镜,表哥却说那个不在,不知道放哪里了,改天找到了
再拿给我看。
\section{落荒而逃}
\label{sec-8-17}

  回来的路上我还是一如既往地掏了表哥那个裤子大腿上的口袋,一如既往地掏出了
表哥的记事本,表哥说是舅舅给的。表哥还说了些关于舅舅的什么话,我脑袋晕晕乎乎
什么也没听进去。

  有一点观点,我始终没能想明白,为什么舅舅家舅舅舅母从来都是希望我追表哥,
为什么就一定得是我追?我自然是喜欢表哥的,这个毋庸多疑。这桩婚姻会很困难,因
为近亲、因为年龄差距,但为什么不是表哥和我一起战胜困难并肩作战,为什么舅舅舅
母从来就是哄着追着逼着我跳出来扛大梁先?

  回家后,一看到家门口的礼物袋,我就傻了眼。舅舅家门口水泥阶梯上放着的分明
是摆放整齐的茶叶礼品袋、我的蓝色化妆包和舅舅家新买的装了我Chinatown所买杂物
的白色冰盒,但我看到的却是舅舅扔出来躺了几个小时、街坊邻居都看到的满地的果蔬
、看到的是两年前舅舅讲给我的九年故事的分手场景,今天我厚着脸皮留下来,几年后
我却不得不自己卷铺盖走人,什么也没得到什么也没留下,没任何人会同情理解自己的
遭遇!我还看到了舅舅作为长辈的推卸责任。XX,你不是口口声声地嫌舅舅把你框得太
紧,你不自由吗,现在没有了舅舅的护航,你自由了,你还敢吗?扪心自问,我不敢,
我一定不敢,我有太多的害怕,我怕我身材长相上配不上表哥,我怕表哥花心,我知道
自己这次回来意味着什么,我怕自己输不起,我选择了落荒而逃。

  我沮丧着脸对表哥说,“表哥,我是不是该走了?”表哥没有留我。舅舅是否就在
橱房等着我们回来,我对表哥的话音刚落,舅舅说,“你为什么没事就一直往回跑?”
我回来,我与表哥的事情,舅舅你就真不知道吗?你这种装腔作势掩人耳目的伎俩到底
还要干多久?气头上的我什么话重捡什么话说,“舅舅,我今天跟你说,就算全世界其
它所有男人都死光了,就算我一辈子打光棍嫁不出去,我也绝不作你们家的媳妇,绝不
会嫁给表哥!”对舅舅说这些话的我,咬牙切齿,面部表情极度扭曲,恨不得把舅舅给
吃了。

  当着表哥的面,我从车里把去年二月走时舅舅送的sleep bag也拿出来,借口说我
买东西给舅舅他不要,那他给我的东西我也不要,把它扔到表哥车顶上,那一刻,我像
极了,不,那一刻,我原本就是一泼妇。

  可是我终究还是难受的,我抹着眼泪对表哥说要他找工作找到加州来找我,表哥说
好,但他加了一句,“如果我找到工作离你很近的话”,这句不够,表哥又加了一句,
“不过我是把你当妹妹去找你!”我没胆量留下来,我又能如何强求表哥不把我当妹妹
?只能昧着良心说好。这次分开,就真该是接下来几年都不再见面了,“表哥,最后一
次,你抱我一次,好不好?”我央求表哥最后再宠我一次,他却不愿意。想到强要的拥
抱后背也会起阴风,想到十二月那次告别允许我抱也是因为他喜欢我,我无法再强求,
表哥也不送我,只能自己一人泪如泉涌地离开了。
\section{归途}
\label{sec-8-18}

    其实我多少还是有点儿承认我自己是冷血的,不断冷血,而且杀亲。在遇到困难、
遇到波折的时候,从亲人、朋友那里得不到理解和鼓舞的时候,那这亲人朋友就会被我
几乎永久性地打入冷宫,一如在我狂热迷恋表哥的同时,因为想不通我来美第一年舅舅
不去找我,我敢冒天下之大不韪、毅然决然地跑出去打球与朋友吃饭;一如二月在舅舅
家当他给我戴高帽时我一分一秒也不愿再停留。

  但是小匆见大匆,那天我也算是真正见识了我们家族里更冷酷的冷血。正常社会里
遇到需要帮助的人,相当一部分的人会表现出冷漠,事不关已高高挂起;从磨难中走过
来有过一定生活经历的人,有的人对别人的困境会感同身受,一如当年我会本能地去帮
助family plan那个男孩,但是也有一些人,更愿意明哲保身,不给自己带来一丝一毫
的麻烦,一如当年我来美第一年的舅舅,他可是战乱年代10岁左右就随大外公外逃躲避
斗地主灾难,他的生活阅历会比我少吗?

  到那天(星期四)下午,我已经是几乎24小时不曾合眼了(夜里特困的时候可能也
在车里躺过几个小时),那天傍晚,表哥不留我,我愤然离开,这个作舅舅的、作舅母
当长辈的,就真有那一百二十个放心我不会有丝毫的意外?

  二月里那次被舅舅气走后,我去住了我当时学校小镇的旅馆(也是那个family 
plan的男孩曾经住过的);可是五月底走的那次,连旅馆我也不愿住,直接上路朝家开
了。一路上一阵阵眼泪扑朔而下,恨死那个舅舅,恨死那个表哥了。恨到节骨眼上,恍
惚间真想直接加足了油门,开下山崖去,让舅舅表哥终生遗憾,永远得不到解脱!

  亲情之所以可以成为精神支柱也更能体现在这种极端意念间。在我有这些念想的时
候,我也还是不可避免地想起了久违的来自父母的关爱,我想起了自己出国临走时给过
爸爸的承诺,想起了临走前爸爸特意交待过我的话,也想起了高考那年我对爸爸说如果
我考不上大学想再复读一年时爸爸湿润的眼睛。不,我不是为表哥舅舅而活,他们不配
,我是为了自己的父母,不管我有钱没钱,在美国顺不顺利,不管我的境遇有多差,我
始终承载着父母的期待和盼望,我是属于父母的,我不能自私,我一定不能做傻事!

  想明白这一点儿,我意识到自己精神状态非常不好,情绪也很不稳定,就开始有意
识地控制情绪,找个rest area停下来,再穿上两三条裤子,把所有的裙子衣物都盖到
身上,好好休息几个小时,平复一下自己的情绪。就这样在连续作战(开车)极度疲乏
下回来时用了差不多二十五六个小时,周五下午六七点钟平安地回到了真正属于自己的
家。
\section{余震}
\label{sec-8-19}

  回到加州后,因为我没能及时完成老板布置的任务,没能及时与老板取得沟通,在
老板、中介联系人和我三方妥协下我被开了,失去了工作。

  我能理解是自己的原因没能胜任工作所以才会被开,不管外界舆论如何猜度揣测中
介的为人以及他与我之间的关系,这位统计前辈在为我refer的这份工作里始终是干干
净净清清白白,始终是真正帮助过我的一位贵人。所以即使自己失去了工作,我还是用
自己的私人邮箱写邮件给我这位老板和这位统计前辈,告诉他们我很理解是因为自己的
不善沟通导制了工作上的失误,我对失去这份喜爱的工作没有任何的不解和埋怨,我很
感激他们为我提供了这次工作锻炼的机会。即使明知道自己说的是官场话空话,还是在
邮件里同他们说了如果可能希望将来有机会还可以继续合作的话。

  失恋加星期一当晚失去工作后,觉得自己的眼神如此空洞,无法聚焦,目光所到之
处一片茫然。在感情的痛苦里,我在床上躺了很多天不想动,终于想起自己的父母来。

  我的爸爸个头高很瘦,还一直骑着他的老式二八自行车,骑得很吃力。妈妈稍胖偏
重,可能是爸爸人也老了吧,爸爸骑自行车想要载着妈妈出去赶集上街的话会非常吃力
。这样妈妈基本也就只能常年困在家里。拿到offer还没有来美国的时候,我就对爸爸
说等我来后真正拿到奖学金,会给爸爸买辆电动自行车。后来06年九月,我买到笔记本
后敲了封长信给姐姐姐夫,请他们帮忙把爸妈照顾我,寄了\$500给爸爸买电动车。爸爸
对他新买的电动车非常满意非常开心,姐姐还把爸妈坐电动车的照片寄给了我。

  后来因为转专业成为自费学生,姐姐们没有给我施加任何的压力,我也一直不曾表
示过什么,直到去年我工作了又寄了\$500给爸妈作生活费,希望能够帮助他们减轻经济
压力,能够帮助他们放开了舍得用钱消费,能够改善他们的晚年生活。

  与其说五年来我对自己父母感恩太少感到愧疚,所以我才急于做点儿什么让自己不
那么难受,不如说在失恋痛苦下的自己是那么的心空而冷,所以才急于想要从与父母的
些许联系里去说服自己父母对我的爱从未走远,它一直都在,想要用与父母的那些实实
在在的联系来温暖自己。在床上躺了十天后,我把早先从costco买到的四瓶保健品(总
价大概\$60-\$80)花了\$60寄回大陆。花了这六十大刀的邮费后我终于是觉得自己变强大
了,因为父母的爱与我同在,开始了自己又一次找工作的旅程。
\section{可恶的舅舅}
\label{sec-8-20}

  去年十二月那次回舅舅家前在茶馆买茶,大概是因为我买的茶多、又再贵点儿,老
板娘问了问我的大致学习生活状况、身份问题,故意讨好我说,“不过你也不用担心,
到时你舅舅自然会帮你想办法。”是哦,舅舅不是帮我把表哥都搬回来了吗,舅舅把表
哥搬回来不就是为了让他作我的靠山吗?想当年09年夏天我要去加州前,舅舅不是赶鸭
子上架一样追问我什么时候毕业,及至我说我都已经申请毕业了,舅舅还不是不早不晚
地在我最后一学期把表哥给搬回来了?没准儿舅舅还真是这么想的呢!

  可是事实果真是这样吗?记得去年二月走之前我有试问过舅舅,万一我找不到工作
会该怎么办,舅舅说现在中国发展很快,回中国也是一种很好的选择。记得当时的我就
对舅舅很失望,没什么好声色给他看,人家明明是比较喜欢美国的,很自由,你却偏偏
要把人家往中国赶,还要装模作样地搬回个表哥来,舅舅你这明摆着是气死人不偿命嘛!

  十二月回去时,舅舅满心欢喜地希望我能打开心扉试着与表哥交往,可是我第二天
去办公室找到他时他却还要装模作样地自己在纸上写啊记地,装着他要作联系人交待表
哥为我办事却不允许我直接去找表哥,幸好我一句话捅破他那个烂楼子;都到表哥
office了吧,舅舅还要我与表哥办完事后到他office里去上网学习,不允许我打扰表哥
;毕业前他要我工作后要加倍地买礼物,我唯一一次加倍买礼物了,想对舅舅表表功,
讨好讨好他吧,他老人家还摆出一张臭脸,一脸的不屑一顾,想要讨得他老人家一点儿
欢心为什么就这么难呢?

  十二月周六晚上吃饭也是他们两老故意避开,好给我们年青人留点儿空间;可是晚
上吃过饭我想和表哥好好说说话,想借个什么小电影把气氛调得缓和点儿,表哥一个电
话打过去问无线网密码,电话刚放下,舅舅前脚就踏进了家门,表哥乖乖地早早地睡下
了,可我却是对舅舅藏了一肚子的怨,吃饭是要给我们留空间,我与表哥说说话就不要
空间了吗,你那么着急跑回来干嘛?人家的好话没得说,第二天在舅母的泓惑下跑去跟
表哥告了个别;人家好不容易从与表哥的告别里尝到点儿甜头儿求交往吧,你老人家可
好,一顶顶破帽子扔过来人都给砸傻了,表哥求理解面前我情意绵绵的默许就生生变成
了这破落户破猴头眼前的伤离别。是你一句要我把你家当作我修养身心的地方把我从四
月的血雨腥风里带回来,是你说不需要买什么礼物,我能回你家就挺好的了,是你从来
不曾送我任何东西却在二月走时给我一个sleep bag要我常回来,现在,我真正回来了
,回到你家,舅舅,还是你,棒打鸳鸯把我与表哥生生折散。当今社会的父母长辈啊,
若都像舅舅您这样奸吝狡诈,我与表哥这样的年青人又该情何以堪,何以成活?
\section{万恶的表哥}
\label{sec-8-21}

  表哥,两年多前的八月里,我,第一次见到了你。四角方桌与你相临而坐,我看见
你五官分明,眼神深遂。表哥你这么帅这么俊朗,我土噶嗒自惭形晦,小鹿撞怀却无论
如何不敢看你的眼睛,我怕我这花痴从此迷失在你的眼波里。

  此后,不管舅母替你送苹果送杏子,还是你自己饭桌上讲笑话,我都要hold住,心
扉紧锁,绝不能在表哥你面前露出破绽来。二月临走时你的意外出现终于给足了我信号
,也让我一阵狂喜,一阵忐忑,嘴上说着牛头不对马嘴不着边际的话,眼睛却在勘测敌
情,落荒而逃。一路上我回想着与你的过往种种,大快朵怡,引吭高歌,到加州后却终
于是禁不住表姐狂轰乱炸,落回到找男朋友的现实。

  我以为我可以从此忘了你,我以为我们再也不会有什么火花,可是表哥,你要去开
会就去开会,为什么还要跑去拍什么小动物;拍拍小动物也就罢了,为什么还偏要给我
看;让我看看也就算了,为什么你还要讲那么煽情的故事?讲讲故事也没所谓,为什么
你们还要父子合谋弄出个关于巧克力的长篇来?

  我好不容易喜欢上你,二月来同你表白,你却用个十年不结婚来折磨我;折磨我一
下也就算了,折磨完了,你又要我理解;我是想理解的吧,可惜你的老爸太可恶,好歹
给我留点儿自尊心作救命稻草得以存活总可以吧。

  二四月间我是见过其它朋友,但那也是因为你老爸啊,谁让你二月我表白时你没给
我个明确态度呢,再说困难里我能不见朋友吗?四月当我知道你才是我真正值得我信赖
的人,表哥我回来又问过你,为什么你就总是这么藕断丝连呢?喜欢我你也不好好说,
天天到那儿瞎折腾。这次邮件里你也默许了,我自己都回去了,你就不能说一句留住我
的话吗?

  敢不成你从来都是要骗我的,十二月是,四月也是,这次回去你还没得逞,你们父
子两个就是这么设计的,所谓不主动、不拒绝、不负责,如此是也!舅舅一把年龄的人
了竟然还要为你如此包庇!真是一家极品,我还是早死早超生的好!
\section{背叛-临界-反省}
\label{sec-8-22}

  五月底从舅舅家逃跑时,表哥脾气还算好,他还在妥善地联系着我与他家的关系,
他帮我把我的物品清好装好,也把我扔出去的sleep bag抱回他家里;我要他来加州找
我,他也带限定语的答应了。来到加州后看见舅母放在我化妆包里的\$70块钱,而我买
的礼物没有留下。五月买茶时我还特意买了\$25每磅的冻顶乌龙茶清香的\$4一小袋,想
自己试着喝喝看。从舅舅家气跑回来后,想着自己挣钱给舅舅买礼物,自己的父母我也
没能孝敬多少,我为什么要孝敬那不近人情的老头?便拿把剪刀把给舅舅买的清香茶剪
开,自己泡上一杯好好品起来。

  后来想想,舅舅对我还是有恩的,舅母的\$70礼钱都放进了我的化妆包,我还是该
把买给他们的礼物寄给他们。于是给自己的爸妈寄出保健品后,我又跑一次邮局,把上
次买给表哥的半盒巧克力(另半盒被我吃了,既然他们父子希望巧克力的童话能继续,
我也帮大家继续)、买给舅舅的两袋茶(一袋剪开一袋没剪开)一起包进表哥的巧克力
盒寄到了表哥家落款表哥收。包装时巧克力盒很大,包得很困难,大概里里外外包了三
层花了大半个小时,等包完了才想起我应该把\$70或者写张支票放里面,再折开太麻烦
,另放信封吧,赤眉白眼也很尴尬,所以索兴就没再多作什么。他们家表哥收到我寄来
的包裹后是如何处理的我并不知晓。

  从失恋的痛苦中慢慢走出来后,我去找了以前的房东,想搬回去过正常生活,
Chinatown太乱没安全感,那晚顺路也去见了以前打球的朋友,也见到了那个曾经帮助
过我的他。他穿了件上面印有“rebuilding together”的T恤衫。初见他时有过以往的
亲切,看见T恤上印的字母有过感动,可是坐在椅子上观看他们打球的我却觉得自己的
心很空。

  这位朋友十二月时给过我最初的感动,所以后来回舅舅家当时已经很喜欢表哥了却
也只能对表哥坦白“我不知道该选什么样的人”;二月回舅舅家前我对表哥狂热暗恋,
却因为舅舅在我来美第一年不去找我、因为舅舅的心机以及表哥家庭的原因毅然决然地
跑出去与他们打球,打完球后和他以及另外两个朋友一起出去吃过饭。四月的舆论出来
后他帮助我时我心里对他产生过阵阵涟漪,在外界对我找男朋友的舆论鼓舞下,我又何
尝不想忘掉过去曾经有过的所有的阴影,真正像所有其它正常人一样激情飞扬地去爱一
场!困境时他的帮助让我感觉很sweet,假以时日或许他也能够给予我表哥那样值得依
赖的感觉吧!我对他有过期待,在他与表哥之间有过比较与平衡,但我们毕竟都年轻,
在狂风暴雨大到超出承受能力的时候,在我心里固有的阴影、不安全感暴发的时候,所
有的幻想也都该结束了。我想我的成长有过重大精神创伤,后来也有过忧郁,我更需要
一个成熟、懂得爱的人来照顾我,而不是我去培养等待别人长大,那时我想表哥应该才
是我真正的归宿。

  现在,我又见到了这个人,他是我除了表哥之外关系最近的异性朋友了,现在表哥
都要被我打入冷宫了,这位朋友以他的生活和阅历,也没有任何做得不好不对不鼓舞我
的地方,可是为什么我的心还是这么空?

  那晚他有个粉丝脚扭了。借着脚扭的痛,那粉丝抱着另一个女孩哭了。刚出来工作
的学生学习生活各方面的压力都很大,我是学生我也很能理解。是因为实在太年轻不够
坚强吗,还是因为我的临时出现扰乱了她的步骤?面对这样一个对他有情有义的情敌,
我竟是如此的平静,平静到能视她为妹妹,能深刻理解她的痛苦,能为她拿冰块做什么
都可以,心甘情愿。我想我心里是有愧疚的,我一直以来都拿他作了保护伞,他并不知
道我与表哥这分分合合的过往,他是无辜的,而到今天我才算真正明白我们不合适!不
合适就是不合适,没有缘由,晚分早分早晚都得分,希望他能早日遇到他的那个真命天
女吧。

  那晚我没有像以往一样一直呆到晚上十一点大家都打完球后一起走,而是同几个其
它的男生聊了聊天,后来自己不到十点独自走掉了。后来听说,那晚他送那女孩回的家
,或许他也明白了吧。

  我的生活是什么,与表哥的关系到底算怎么回事,我接下来的生活该如何安排,我
还需要仔细反省。
\section{反省-闪婚(1)}
\label{sec-8-23}

  来到美国后,我那先天性上眼皮下塌的眼睛被我苦巴巴地戴隐形眼镜掰大了很多,
眼神也还算清澈,去年自己的皮肤也还好,抛开自己肉乎乎胖乎乎的身材不讲,面孔基
本上还算看过得去吧。现在,望着镜中的自己,是什么时候,上眼皮塌得更多,眼睛变
小,眼周也平添几条小纹,甚至眼袋也若隐若现,面部皮肤干涩。我是为了谁,我辛辛
苦苦开长途跑了这么多次又是为了什么?望着镜中仿佛不认识的自己,我发誓,以后我
再也不开长途了,任何人都休想!

  在找工作的忧虑里,在对表哥爱情的回望里,我的情绪起伏不定,焦躁烦闷。房东
的后院有个小鸡笼,我就跑去店里捉了几只憨萌小鸡来养着。院里跑动的小鸡、爬上架
的豆角腾都给我带来很多快乐。

  快乐情绪里的表哥也变得棱角分明起来,舅舅、表哥应该远远没有我所想象的那么
可怕。舅舅会觉得自己的孩子是天才,表哥会问我接下来一年为什么会辛苦,他们都是
天生的乐天派,而我却是悲观的。遇事我先想到最差的,再付出一点儿自己实实在在的
努力,那么结果就不大可能太糟糕,也会变得容易接受得多。舅舅从来不曾十恶不赦地
干过任何伤害我的事情,虽然那天我愤然离开,相信他们只是疏忽,我相信他们心里也
一定有过担心。如果表哥果真是个恶魔,他应该会告诉我的。那舅舅不曾说过什么,表
哥也就一定没有那么坏,虽然他说不出来,虽然一拖再拖,把我弄得七荤八素,我还是
相信他一定是喜欢我的,虽然我想不透那个具体原因。认识到这些,我找工作也变得相
对轻松不少。

  因为第二份工作的工作内容、专业方向我比较喜欢,待遇也还不错。这次找工作我
就给自己定位得比较高,方向我一定要喜欢,待遇不能低于XX才能考虑。其实接下来的
一两个月到七月底,也接到过上百封来回邮件,差不多有过phone screen二三十个吧,
两三个onsite。之前的那位统计前辈也帮我refer过一个phone screen,但后来他告诉
我面试者认为我的background 不够strong。我想这应该也是那次找工作不太顺利一个
重要原因,除了一开始主观定位过高外。

  找工作找了一两个月不太顺利,很多朋友怂恿我去找表哥,嫁给表哥就什么都好了
。我也自然而然地想起表哥来。

  人生是一个过程我19岁那年就大彻大悟了,所以工作上我不会有太高的追求,只想
保持自己的上进心,在力所能及的范围内折腾折腾,能折腾出点儿什么来是庆幸,折腾
不出来也不遗憾,享受这个过程就可以了。在工作上我可以想怎么折腾就怎么折腾,但
感情上我从来都是寻求安定的。在美国的五年学习工作能力干劲我都大不如以前国内,
那时能一心一意用功不觉得生活中缺少什么,但来美国后的日子如同无根浮萍飘浮不定
,很多时候处于不想学习工作也没能玩好的梦游状态,这几年是有缺失不完整的,那学
习工作效率就可想而知了。我走我想我渴望有一个男朋友能让我从此安心,而表哥就是
那个我一直在寻找的人。
\section{反省-闪婚(2)}
\label{sec-8-24}

  就像茶店老板娘说的,我最初也以为舅舅搬回表哥来就是为了给我作靠山的,但舅
舅坚定地否绝了。我讨好过舅舅,他要我加倍我就买过双倍多,也按照他的意愿“疯疯
颠颠”又跑回去过两三次,人都累傻了;加上我从来都是喜欢表哥的,表哥舅舅也都知
道。大家不要被我的表象所迷惑,我应该是一开始就惊艳于表哥高大帅气风度翩翩气宇
不凡的,小丫头的小心思里也早早地无意识地选择性地、充分发挥主观能动性地打一开
始就没打算没想要记住表哥的名字,所以后来处于尴尬境地时顺其自然叫了“表哥”,
要知道,我眼睛盯着桌子角再重复一遍“表哥”的时候,一脸娇羞,这个称呼把我与大
帅哥表哥之间13年的生硬距离闪电缩短,心里美着呢!回去翻我第一餐饭吃酸醋那段儿
吧!

  我从来都是喜欢表哥的,从我与表哥有限的交往看,表哥也从来都对我照顾有加,
也很宠我,他应该也是喜欢我的吧。他也有近年恋爱结婚的打算,想来上次我也确实太
冲动了,至少听舅舅把话说完说清楚。如果当初我稍微有点儿耐性稍微不那么暴躁,没
准儿今天就会是另一副样子。

  今年元月我飞去其它州参加了我们一个专业考试,没过。表哥是帮我找到练习软件
的人,我把成绩单forward到表哥邮箱,并把我的LinkedIn profile的link同一封邮件
里发给了表哥,表哥好像视而不见,没什么动静。后来四月回去与表哥巩固了感情之后
,因为喜欢自已加州的工作,我也够狠地发邮件问表哥问他会为了他自己,也
potentially为了我会找工作找来加州吗?表哥没回。找工作期间,不知什么时候我搜
到表哥的profile,他的connection当时已经有4个,前三个时我没觉察到。后来我见到
表哥的联系人从四个到五个,从五个到六个,但六个之后就没再增加了。既然我在加州
找不到工作,那我就该试着先去适应表哥的小城生活。表哥的联系人停滞在六个,或许
他也是这么想的吧,所谓心有灵犀不点自通!

  爱情里我大概从来都是闪婚族吧,觉得两个人只要相互喜欢就可以结婚了,至于婚
后的柴米油盐酱醋茶也就是一个相互包容协调维持的过程。与前男朋友能够开始不也是
因为轻信一个关于(爱情)婚姻的承诺吗,小叹一下人骨子里的东西还真是难改,撞过
南墙了,我竟然还是这么轻率!

  那我现在就是真的心甘情愿地嫁给表哥,没有丝毫顾虑了吗?有的。我不放心舅舅
眼中的“叼蛮公主”,此人擅长察言观色,又随时可以拿捏人,我怕我会死在她手下;
舅舅在我面前把她藏了两年多,想来藏还是有必要的,要不然我早知道有这样的婆婆,
打死我也不嫁表哥;表哥的家人在我面前的出场顺序都是有安排有讲究的,甚至后来小
表哥,未见其人,先闻其声,十二月舅母告诉我他住楼下,二月见他第一面,四月表哥
把他正式介绍给我,舅舅的内心庭院深深深几许啊!

  与表哥的这桩婚姻,从我的角度看,一定有婚姻绿卡的原因在。这么说吧,如果同
样这个表哥H1B在美国但工作一般身份没有房没有车没有,我应该一开始就不会考虑的
。我何尝不是恋着望着表哥的绿卡一不小心跌进舅舅精心设计的陷井里,再也爬不出来
! 好在掉进去有表哥把我接住,对我也疼爱有加,外面的世界很精彩,我却想作青蛙
王子,坐井观天,哟喝唉哉地唱唱情歌享受自己的水底乐园了。

  舅舅早早地就把我培养成了贤慧准儿媳,他们这样的家庭找到我这样一个儿媳,他
们会亏吗?我的成长环境与表哥类似,我也有爸妈需要照顾,我也有自己的姐姐需要稍
微贴补照应,我从来都没有觉得小表哥会成为我们很大的累赘,只是舅舅那曲径通幽处
的手段我更害怕,更怕舅舅的心机、垄断统治和舅母的刁钻而已。

  再说舅舅吧,大的事情上他好像也没太为难过我。大三下的时候写信给他他没帮忙
,也没回邮件,虽然我用英语写的邮件,但不排除系统by default把它当作垃圾邮件的
可能。再说,我们毕竟是远亲,连大表姐他都没再往外带,我也实在不能强求这远亲舅
舅把我带出来,在那事上我也还真没指望过他,那时的自己更多的是无法做决定彷徨纠
结。转专业时舅舅一拖再拖,但拖了一个学期,他还是作了经济担保人帮我转了专业不
是?加上之前已经对舅舅作了肯定的反省,我想我努力去作一个贤慧本份的媳妇,将心
比心,舅舅舅母应该不会对我差到哪里去吧!

  想了差不多一个周,想着就很开心,7月24日那天就轻轻松松地给表哥写了封邮件
,告诉他我的想法,鉴于平时我打电话他不接,写邮件他从来不回,我在邮件里说,他
要是不给我及时回邮件,我就当他默许,我会几天内就直接跑回去了。

  我坐在门口望啊望,等待邮差的到来,但我等来的却是两封拒信。
\section{两封拒信}
\label{sec-8-25}

Date Sun, Jul 24, 2011 at 4:24 PM

Xxxxx:
  I have already told you my answer on your last 3 visits. 

  I don’t want you.  You are too pushy, and you don’t listen to me.  
Already, it fails my test for a harmonious relationship. 

  As I have already told you, we are too closely related.  We are first 
cousins, with the same grandfather.  Any children between us would be 
severally at risk for birth defects. 

  Marriage is absolutely out of the question.  I don’t even like your 
recent behavior. 

  I don’t consider you to be my girlfriend, and never have.  You are 
simply a cousin.  When you visit Xxxxx, you are a guest.  That is how I have
treated you, and that is how I expect you to behave. 

  If you have other valid reasons for your school or career that happens 
to  bring you to the Xxxxx-Xxxxxxx area, e.g. a test of U. Xxxxxxx paperwork
that you can’t do anywhere else, then you can stop by for an incidental 
visit while you are in town.  Otherwise, you should not be here.  I am not 
the primary purpose of your visit.  Do not expect to spend even 1\% of the 
time with me.  I have my own work and activities.  You must have yours 
already arranged, before you come. 

  Do not lie to, nor hide from, your own employer to trick him or her into
thinking that you are still in town.  That is not a meaningful visit.  Do 
things in a proper, professional manner. 

  Do not come to my office during working hours.  When I am at work, I am 
working.  You should not interrupt me with frivolous non-work requests every
10 minutes.  I really hate a person who nags me like that. 

  This is the fourth time I am telling you: I don’t want you as a 
girlfriend. 

  If you fully understand that, and fully consider everything I have 
written here, then maybe you can visit us as a cousin and guest in our house
.  Tell us about things you have done, and interesting activities you will 
be doing in the Xxxxx-Xxxxxxx area.  Do not ask us to join in them, and do 
not expect me to accompany you.  You must be independent. 

  If you can’t do that, then I prefer that you stay away. 

--
XXXX(表哥的名字)

cc    表哥邮箱地址
date     Sun, Jul 24, 2011 at 5:08 PM
subject     Your Harassment
mailed-by     YYYYYY.edu
Important mainly because of the words in the message.
    hide details Jul 24 

我的full name:

Your behavior is already a harassment to me and to my family.
As I told you repeatedly during your last several visits here,
YOU ARE NOT WELCOME!
Now, I even do not consider you as a relative.
Please stay away.

If you continue to harass us, I will

REPORT YOU TO THE POLICE AND TO THE U.S. IMMIGRATION OFFICE
TO GET YOU DEPORTED FROM THE U.S.
IT'S THAT SERIOUS. You better consider the consequences.

Please behave yourself properly.

舅舅full name
\section{伤}
\label{sec-8-26}

  我的英文不够好,之前狂热暗恋时有写信给表哥要他回信时帮我纠正我发给他邮件
的用词与语法。表哥的这封拒信是他写给我的第二封信,相比于我的原件,他的回复像
我高中写的小论文。他给出了论点论证论据,也把他作为兄长提醒我工作、生活需要注
意的地方在信里作了延伸。这是论文,理当生硬刻板却被表哥妙笔生花,选词造句有理
有据有情有义,读起来像抒情散文。原来英语里两个人之间的孩子是用“between”啊
,这个英语介词让我禁不住浮想连篇,意绵绵邮件生香。

  他的答复在我看来就是,与他结婚完全不可能,作他女朋友也至少暂时没得考虑,
他要我从此死了这条心吧。这封回信信息量过大,至于内容里的细节,我怎么就变成了
他的first cousin,他是如何把我作为备胎,考虑过与我之间爱情的可能性,又是在什
么时候他就退却止步了,我比较笨,要彻底消化理解想通想透这些至少得要我花上十天
半个月到一个月吧。

  我知道了表哥的答案,此时,我还真没有心思去理会他信里的这些细节,因为,收
到表哥邮件的43/44分钟后,舅舅的一封警告信直接送我go to hell! 不要问我怎能做
到独立不依赖,被拒第二天LinkedIn上的connection就可以狂增,那全是拜舅舅所赐,
都是被他气得逼的!

  人的成长中或多或少都会经历一些来自social network的情感方面的伤害,比如来
自父母亲人、同学朋友、恋爱对象以及领导同事等等。我也有过,咀嚼过父母偏心的痛
苦,独自承担过五六年的孤独,浸受过绝望的洗礼,也品尝过爱情的毒药。后来经过学习教育感受来自父母亲人的关爱,这些伤害的结果,父母的偏心被我原谅了,亲情成了我最坚实的精神支柱,也从绝望中获得新生的力量重新站了起来,独自承载的五年孤独成为了一道心灵阴影,爱情毒药一度成为我的内伤。今天舅舅的一封邮件,可能会成就我今生最大的内伤!
\section{我和舅舅}
\label{sec-8-27}

  97年夏天舅舅在我的生命中第一次出现,作为我的偶像,他用他的知识阅历和一袋
巧克力为我打开了一个童话世界,地球另一端舅舅口中那个自由的国度让我心弛神往;
九年后,我踏上了这片热土,也开始了找舅舅的旅程。

  06年8月刚到的时候,我输入舅舅名字的大陆全拼(没意识到台湾拼音与大陆不同
)在他学校的网站上找,无果;06年12月大舅母告诉我二舅(舅舅)就在我旁边学校,
她没有告诉我舅舅的电话号码,因为“小三”舆论暴发,我很自卑,也不多问;再半年
后的07年5月,我内伤基本痊愈,”eecs”几个密码神领我找到了阔别快十年的舅舅(中
间大学某年夏天见过一次要邮箱,但后来邮箱地址被我弄丢了),一代偶像,再见到舅
舅的这一刻,我觉得自己从此挺起了脊梁骨。

  舅舅是偶像,是精神支柱,他建议我做什么我就做什么,除了转专业是自请的。舅
舅建议我买一辆二手车。从个人主观意愿出发,我想买吗?我不想,我想听姐姐的话老
老实实作个守财奴。我一直都有二手自行车,那是我孩童时期的奢望,大学有,国内硕
士时有,到美国后也有。在被人们讥笑为小三的那一年里,除了后来房东的偶尔照应,
没有人载我去买菜,我就自己坐校车或者骑自行车去。若没有认舅舅,我一定会等到工
作后才买车学车的。但我在这里孤单漂零,我情愿听舅舅的。买了一辆车,这辆车又引
领我们走入新的里程。

  07年舅舅帮我选好车,修好车后,房东载我去当地costco买了四瓶(两瓶应该太少
,我想我应该买的是四瓶,真不记得了)保健品,打算作为送给舅舅舅母的礼物。后来
试控舅舅想要的是豆浆机,我也顺了他的愿买了个货真价实的,并把早先买好的保健品
也一并送了他与舅母。

  同他聊天时舅舅一句他们是受过高等教育的人不会干扰小家庭私生活会给年青人留
空间的话噎住我了。后来房东给我介绍一个男朋友,我们交换照片、电话号码一切进展
顺风顺水时,却因为房东“无意”的一句认为我家庭负担重,不但要照顾自己父母姊妹
,可能还有个美国舅舅晚年要我养,把活脱脱一个前来探望的准男朋友生生吓跑了。

  后来换过几次机油,洗过几次车,舅舅亲自把我送到过加州,在冰调车头的惊慌里
,在撞爆车胎的羞愧里,在舅舅藏匿着舅母两年多的时光流逝岁岁年年的礼物里,我们
建立了“忘年交”朋友般的信任。

  09年夏天去加州前舅舅仅只坚持用了他买的雨刷和滤器,而summer回来我还是一如
既往地给舅舅买了往年同样份量的礼物。真的是我给舅舅买礼物买成惯性了吗?不,在
那还是学生精神压力尚未解放的年月里,所有的礼物、所有的感激都是发自肺腑,仿佛
舅舅是我的救命恩人,他是我在美国远离亲人时唯一的支柱。那时的礼物与感激都是那
么地真诚纯粹,不带有丝毫的杂念。

  09年秋舅舅从韩国搬回一个表哥来。表哥、舅母等舅舅家庭成员的加入打翻了一坛
酸醋,惊起一场声势浩荡的讨伐争掳。
\section{我和舅舅(2)}
\label{sec-8-28}

  记得我的大姐在谈恋爱时,与男朋友先一起回一次家,再自己单独回来问妈妈,父
母对她朋友的看法。似乎姐姐们的恋爱里,说什么事情都只同妈妈讲,有意地避开了爸
爸。我,一个风雨中摸爬滚打长大的孩子,却始终没什么性别观念。08年春天有点儿迷
国内一朋友时就同舅舅讲了,舅舅建议我有机会summer回去探亲,与他相处一下,合适
的话回来的时候就可以把婚事订下来,后来舅舅送我去了加州,回国没能成行;09年与
前男朋友相处的时候前面提过,我是把能查到的所有两人的生肖星座血型打印出来,跑
去找舅舅问他怎么想,舅舅说给他时间与他相处不要push,最终还是会走到一起,并给
我讲了那个九年的故事。

  舅舅与我,在我眼中,更像是红孩儿与姥姥的角色。是什么时候,我与他建立了朋友的信任,又是什么时候,我在依赖?表哥是他的孩子,表哥回来舅舅对他好也是理所当然的事情,我却对表哥妒忌得要命,暗叹我又算得了他们家的什么人,自然无法得到他们家宝贝儿子的待遇,肚里酸醋翻滚(十二月回去见表哥时对他提起过这段感受)。舅母是舅舅宠爱着的“公主”,她却是我这个有“恋舅舅”情结侄女眼中的“后母”,与我争抢舅舅的爱,面对情敌我无意识地时时刻刻渺视她,根本就不把她放在眼里(潜意识里我们两个女人常常把家里搞得鸡飞狗跳,内战暗战无数吧,嘻嘻),害得后来四月回去时表哥耳提面命我得尊重舅母。

  我真的是恋着望着表哥的绿卡掉进舅舅设计的陷井吗?我很怀疑。我本来就喜欢他
家这个宝贝儿子,见第一眼就喜欢,叫第一声就是“表哥”;就像小时候我能把放牛晒
粮食赶走小鸡小鸟之类的小事做好做成我的绝活,这样我可以得到爸爸的宠爱,我当然
希望按照舅舅的意愿试着去喜欢表哥来获得舅舅的“宠爱”。那两股力量相加,我本来
就喜欢,舅舅又希望我去喜欢,我能不喜欢表哥吗?想要不喜欢表哥恐怕都难吧!只是
我不敢看进表哥的眼睛,表哥又只能通过舅母借花献佛,平时吃完饭就躲进了他的房间
,根本就从来不曾表达过情意,跟那些个死gay有什么区别?二月来加州找工作时舅舅
策划的表哥送别让我开心了一路,但到了加州爱情的星火还是被大表姐掐灭。

  十个月后的十二月回来,舅舅看出了门道,我们有了第一次的火花,舅舅舅母齐力
撮合,我终究还是没能逃脱与表哥告别时被点燃的命运。

  狂热的那两个月打电话给舅舅时舅舅说表哥在国际上发表过六十多篇文章,随时可
以毕业找到工作。可能当时真是“热”吧,我自己从google以及后来查到表哥在韩国的
学校里去查,凭我有限的搜索能力,只找到25篇左右的文章,国际上还是韩国内的我搞
不清楚。二月跑回去我问表哥有篇他们实验室文章表哥的名字的那个中间名是什么意思
,我那会儿还真是怀疑过表哥会不会在那边还有个孩子什么的。表哥说我弄错了,那个
人应该不是他,后来也就忘了。
\section{我和舅舅(3)}
\label{sec-8-29}

  人很多时候都在选择。选择又有主动和被动。如果当初我没有选择离开,如果我的
主动选择是留在实验室,在没有考试压力下,我会否留意到情感的异样,我会不会就早
死早超生了?如果当初看到舅舅摆出来的礼品袋(盒)时,我没有主动选择,没有问表
哥我是不是应该离开,而是舅舅强加到我头上强力说服我务必不得不离开,我还会有今
天读到舅舅邮件的愤怒吗?

  二月是被舅舅用帽子砸走的;五月底从舅舅家走时是我没安全感,自己主动选择离
开的。舅舅就算气也只能气我把买的礼物也都搬走了,而舅母还在我化妆包悄悄塞了\$
70。可是我当时那么做也只是希望舅舅明白有一天真把我惹急了,我还真会像那天舅母
问我属相面前的冲话一样“把我惹火了我脾气大着呢”,气头上我又没多想,后来礼物
不是也寄给表哥了吗?再说二月气走后四月表哥逼我去住旅馆什么的,我不是态度好好
的相当于也认错了不是?就因为我把礼物搬走,舅舅至于气成这样吗?舅舅这是冷暴力!

  内伤是什么?内伤是爱恨交织缠绕成的旋转皮鞭,鞭鞭都抽打在五脏六腑,胸膛内
部阵阵痉挛,心内绞痛滴血。就拿爱情毒药来说吧。我之所以会恨也是因为我心中有爱
。我尊敬崇拜这样一位导师,他有自己的事业理想,他希望自己桃李满天下,他希望他
能为自己国家的农业发展添砖加瓦贡献力量,而他当时的生活却倍受煎熬。而我会恨却
也是因为这样一位我如此尊敬崇拜的老师给过我的承诺却仅关利益。无论我如何迫不得
已接受这个事实,我终究是心不甘情不愿意难平,而道义上我却不得不本能地道歉,把
痛留给自己。好在时间是良药,一年后我便康复了。

  舅舅这事儿,我可是从来都把舅舅当救命恩人般的对待,就算是在表哥这事上,我
是在气头上才做出搬走礼物的事情,我有“何德何能”竟能把这老头气成这样?他什么
时候真正警告过我、让我真正明白过来我不该回来?当初,06年他知道我来他都不来找
我,难道就是因为在他的如意算盘里迟早都会有今天?

  我要回去闹一场吗?要,一定要,太有必要了。有了当年一年内伤的痛苦教训,我
也学会了保护自己,与前男朋友分手时该甩出去的拳头一定要甩(不提倡,美国法律不
允许打人),与表哥五月底走时该扔的sleep bag也一定要扔!但我暂时还没有工作,
我回去会被别人笑话我图谋不轨。

  人总有一样支撑自己得以生存的性格,对于狮子座来说大概就是尊严吧。06年的五
一,当别人的爱人告诉我她是老婆,我本能地向她道了歉,为了自己的自尊心把泼妇一
回的怒火强压下去,那道内伤一年多后痊愈。现在我没有工作,为了自己的尊严,我忍
受屈辱,一定不能回去!如果当初我一直没能找到工作直接回中国去了,我想,我与舅
舅的关系,终将成为我终生的内伤。好在,后来我找到一份工作。

  我有了工作便有了维护自己人格尊严的资本。那个周五中介问我能否周一工作,我
推到周三,说我还有事情要做。很多人以为我口衔橄榄枝为和平而归,我知道我是心怀
仇恨,怒气冲冲地杀回去的!
\section{决战}
\label{sec-8-30}

  舅舅的信是警告,但我读出的却是威协恐吓。我是天生的赌徒,考研报研究所赌过
一次、出国、转专业、按时毕业,以及这次冒死去找舅舅决战。我是一个天使,却被仇
恨折磨成魔鬼,我一定要与舅舅针锋相对、针尖对麦芒地决一死战来平息自己的怒火变
回仇恨前快乐的自己。而我敢自动送上门去找死,我唯一的赌注就是舅舅知道我付出了
多少代价来到美国,我就不信他一个911把我遣送回中国后他的晚年能过得安心。

  还是先去到表哥的office。表哥不在,门没锁,我就进去爬表哥床上休息了。表哥
的枕头调了个头儿,床的枕头那头地上放了一个黑乎乎的大仪器,不知道又是表哥的什
么新玩意儿,体积很大立式站着。那时心里有事怎么也睡不着。刚躺下没多久,表哥就
回来了。他翻了翻我的书包,哼了两声,就坐到办公桌前忙着处理电脑里什么东西,透
过被子的缝,能偷瞥见表哥穿着那条我常掏口袋的裤子,再多的窥看我也不敢了。我心
里有鬼不知道接下来会发生什么,所以虽然我没睡着,却也什么都不敢同表哥讲,不多
久表哥就走了。

  可能真是心里有鬼吧,那天怎么也睡不着,就坐到表哥办公桌前想看看他的台式机
。鼠标关了,按下显示器开关,电脑没锁,屏幕上当前工作台面是Adjust 
Date/Time窗口,后面是他窗口最大化的hotmail,不过inbox和delete里所有的邮件都
删除了,没有其它应用程序。我在他的电脑里翻了翻,把那时他秀给我看的小动物的文
件夹找到了了,看见里面所有的文件都详细填加了照片名称解释。我也找到那个有小兔
子的文件夹。不过表哥电脑里实在是没什么好玩的,我又很累,就打算出去洗刷一下回
来后还是好好休息一下要紧。

  我始终没有动过表哥office的门,但等我从洗手间回来,他的office已经锁上了,
我进不去,手机也锁在里面。我是回来亲自观察家里的变化的,并不知道接下来会发生
什么,就趴在走道的办公桌上等啊等,期待着表哥会再次出现。多少个人从门口经过,
但都不是表哥。我的耐性耗尽,找到一个professor借用他office的电话打给表哥,他
没接。后来又跑出去公用电话亭打过好几通电话,他还是没接。大概等了一两个小时吧
,我就算是被逼上梁山,这次也不得不走路上舅舅家了。

  舅舅家我去过好多次了,但路我总不记得。加上极度疲乏下走路,又绕了弯,又耗
了差不多一个小时才总算找到了舅舅的家。走进去舅母在橱房,她还真是恶俗,看见我
就鄙视了一下,“老太婆”这个称呼就像机器里刚炸熟的爆米花噼里啪啦往脑子里崩,
我小声在心里叫上千遍万遍,不这么称呼她太对不起我们文明的祖先发明这个词了。

  用了一下洗手间,洗手间里忍不住眼泪就掉下来,太累太辛苦了。擦干眼泪,跑出
去敲表哥的门,里面没人应,打开门看,表哥穿着背心短裤平躺在床上休息,见我开门
,稍微抬了抬头。我站在门口盯着躺在床上的表哥看了又看,这身穿着、这个姿态这种
表情的表哥我突然觉得陌生,一如两周前他的拒信来得让我没有任何防备。之前我从来
没见过表哥光胳膊或者小腿什么的,可能他会担心我觉得他年龄大皮肤粗糙皱纹多吧。
穿很少衣服的表哥还是对我造成了致命的吸引力,很想要走前几步仔细看看表哥摸摸他
的胳膊,但那封信这个人让我陌生得害怕,往前一步是天灾,多说一字便是人祸。再说
万一床上躺着的是别人,那岂不是天大的笑话!我一个字也没说,关上表哥的门,可以
确定他们家在装修房子。

  真快真巧啊,出门就看见小表哥坐在橱房桌上吃饭,转眼舅舅就牵狗回来了。他们
快,我更快,眨眼间就跑到表哥隔壁房间,躺到了与表哥床平行的舅舅床上去休息,被
子掩面,静观其变。
\section{决战(2)}
\label{sec-8-31}

  舅舅回来了,他气啊,气得心潮澎湃,说话声音发抖,感觉不像是装出来的,但我
却无论如何没法理解我竟然可以把他气成这样;可是我还有气呢,你是舅舅,我对你也
从来不薄,你居然对我施加冷暴力。我就不动,一动也不动,一个字也不说,就躺在床
上闭着眼睛睡自己的。

  舅舅是装的吗,他居然平息得这么快,他开始说话,说他可以拿枪打死我;说什么
blahblah一大堆,我早累得昏昏乎乎,快睡着了。后来他就掏出电话打911了。

  舅舅打911电话时的语气如此平缓温和,完全就是一谦谦君子,那时我相信刚才那
几分钟的生气一定是舅舅装出来的。电话里舅舅说他是37年9月5日生日,我记得之前他
说他是36年重阳节,阳历就是9月24的了,不过在那个动荡年代,生日的纪录有错也是
有可能的,我的小学同学很多档案上的生日与真正生日还不符呢。我接着听舅舅说,舅
舅对policeman说我是骗子,说他离家多年,不知道家乡有我这么一个亲戚,说我是表
哥的first cousin,blahblah,我再也听不下去了。

  舅舅是我心中到那时已经存活了十四年的偶像和精神支柱,五月底时他责问我为什
么跑回去我只认为他世俗虚伪,心里一套,面对社会大众又伪装出另一套,很好理解所
以后来就没当回事。直到舅舅发恐吓信最后通谍算是把我惹火了,我一腔热血空对月?
极度不平下一定要回来找死。而就在今天这一刻,我亲眼目赌,不,应该是亲耳听见了
舅舅撒谎,打911对policeman撒谎,我心目中的一代偶像,终于如地震时的高楼瞬间坍
塌在地,一片狼籍,惨不忍睹。

  Police来舅舅家后舅舅又说了同样的话。我英语口语不好,我只是一个国际学生,
在美国还没能扎根,我无法对Policeman argue说舅舅在说谎,舅舅作为名校计算机专
业的资深教授,他有他的社会地位,我说什么,说到天花乱坠都无法改变这一客观事实
,就只能都忍了。Police问了我的基本情况,我告诉他我知道我与表哥关系变了,我只
是想回来看看到底是怎么回事;Police后来又跑去找表哥作纪录,我继续躺在舅舅床上
亲耳听见表哥以火暴的脾气说我是他的first cousin,说他从来都没有说过喜欢我,blahblah。对他们父子,我受够了。那时突然瞬间我获得一股强大的生命力,心里有了十足的勇气,告诉自己不用担心,就算是要打官司一切就都兵来将挡水来土掩吧。但舅舅自己放弃了。那police比较nice,他要表哥亲自再告诉我一遍我与表哥关系的经过与结果,但我下意识地扭过了头,说我不要再听表哥说任何话,说我今天来舅舅这个家也只是表哥设的陷井,他锁了我的书包和手机,这是trap是seduction。表哥辨解了一下。我说话时也故意不看表哥,他后来又穿的什么衣服ware什么表情,如何开车领police去他office楼取我的书包手机,一切都已经与我无关。我与police在楼下等我的书包,知道表哥会送书包出来,我也特意背过身去,哪怕是用眼睛的余光也绝不再多看他一眼。

  Police拍了我的照片作了纪录,交待我以后不可以再去那个地方,除非先与他们联
系好。我回答说我这辈子再也不会与他们有任何的联系了。我与舅舅、表哥一家人的缘
份应该到此也就结束了吧。
\section{大浪淘沙}
\label{sec-8-32}

  反正我周三才开始工作,为了让自己衰老得慢点儿,为了给自己足够的时间休息,
那个被打911的周六晚我就住到了学校小镇以前我住过的旅馆。回舅舅家时GPS坏了,还
好是快到他家时坏的,没坏我的大事。到了旅馆把自己清洗干净,舒舒服服地穿条裙子
出去买只热腾腾的烤鸡(和一些水果)回来吃,打算第二天买个新GPS休息一下第三天周
一一大早就可以开车回家了。

  战争是需要付出代价的,我陪上了\$280的油费、快到舅舅家GPS坏掉时收到\$150的
罚单,以及连续两次开1000多迈的衰老。但对我来说,战争还是有进步意义的。这场战
争的原本目的就是我与舅舅拼个你死我活。现在从法律上讲,我不得再入侵北方半步;
但这场战争也真正彻底地解放了思想。当初我可是作好了必死的准备北上挑战的,但那
个冷血冷酷暴烈、曾经我恨他恨得几乎要一命乌乎的老头也并没能像他信里威协的那样
把我赶出美国不是?所以这场战争的伟大意义就在于推翻了长期的垄断统治,心里上暂
时就不说渺视他,至少与这老头站到同一水平面上,撕毁了曾经那尊敬崇拜的面纱,我
终于是可以做到对他视而不见听而不闻,可以对他的话左耳进右耳出,置若罔闻。摧毁
了偶像,我的仇恨也没有了,怎一个爽字了得?

  这次回去,战争的实际意义就是让我亲耳听见了舅舅撒谎和表哥的暴烈否认。表哥
很狡猾,他一定是这世上最著名最彻底的恶魔,居然有这么多的手段。回到加州后我还
是忍不住看了看表哥的profile,他的联系人几天之内已经增加到9个,接下来的两周又
各增一个,最后可能以为我网上说鸡犬相闻老死不相往来是说他吧,波澜不惊地停留在
11个。怪不得表哥会有那么多骚扰电话,表哥所谓能让那些个女人欲罢不能的手段也不
过是藕断丝连,就像给我的拒信要称我们是first cousin,还要给个曾经喜欢过我的影
像,就连这次911战争他还要摆出个特大号超豪华望远镜立在床头,怕我不知道是望远
镜还要在police面前故意提起。我现在可总算是鸿毛一身轻松了,桃花源是陶渊明的理
想生活,鸡犬相闻老死不相往来是他那个时代的局限,但这老死不相往来可是我眼中表
哥一家与我关系的世外桃源理想境界啊,我可一定得好好珍惜自己历尽千辛万苦通过战
争获得的自由啊!每次想到这里就想唱那春节晚会上谁唱过的“我(们)走进了新时代
,勤劳勇敢的中国人,意气风发走进了新时代\textasciitilde{}~”

  回去后刚工作的我,一时半会儿也没太大的压力,便喝喝汤减减肥,打扮打扮,有
点儿小闲的时候就做做面膜,把战争年代暴晒过的皮肤再重新改造回来。知道表哥藕断
丝连,知道舅舅家在装修,知道我到舅舅家时他们家人能够短时间内齐刷刷地聚集,舅
舅还牵条狗,都是因为他们认定希望我能回归那个家庭,但我会吗?奸诈狡猾的老头儿
加刁钻跋扈的老太婆,我心里还是有点儿害怕的。

  我想我只要做好自己,努力工作,忘掉表哥,我就能走进自己的新时代。后来居然
还有一个别州的on-site,跑去玩儿了一趟权当旅游了(人家也没真正想要我不是?)
。可是为什么我越来越空虚,总是这么空荡荡的,对表哥的思念却越来越深?
\section{大浪淘沙(2)}
\label{sec-8-33}

表哥真如我所想象的那么暴烈不堪吗?多少个夜晚我沉迷执着在对表哥的回忆里。
\linebreak
\linebreak
十二月回去,舅舅带我去表哥office找他;表哥贡献两次傻笑;

用自己小P孩问题骚扰他,问星座时右手搭他左肩上;

问他有没有什么好玩儿的,他摆出了动物园动物、巧克力诱惑;

诱惑我表白未遂,他有过台风过境的疏离;我自私抓住他的手想要留住他;

我捡起图书馆地面上的卡片,他捡起回office延途的垃圾;

楼道里他等待诱惑我捞他胳膊未遂;临走时受舅母启发与他告别。
\linebreak
\linebreak
给舅舅打过n通电话,舅舅说我们是近亲,对我与表哥恋爱不支持也不反对;

舅舅说表哥是天才,国际上发表了六十多篇文章;

舅舅说表哥想什么时候毕业就什么时候毕业,毕业后随时可以找到工作;

舅舅说表哥普通话讲得好,万一不想在美国工作,还可以去中国;

舅母说表哥有两个女朋友,表哥快订婚了;

打电话给舅舅,我提到表哥一篇文章是电脑模拟打网球扫描聚焦的?

对朋友讲舅舅贪心,恋爱是我的权利,对表哥的家庭很担心怕嫁过去会受气;

舅舅电话里骂我性格不好嫁不出去没人要是骗子,挂我电话;

二月归拖行李表白,表哥求理解,舅舅扔帽子砸人,我去住旅馆,与表哥,必分。
\linebreak
\linebreak
四月归来,温馨甜蜜,表哥性诱惑未遂,要把我当妹妹严格要求,被搏回;

考验表哥,会否找工作找来加州,未回;

电话舅舅淡然一句“我不欢迎”,表哥默许五月底归,归;

表哥给卧室钥匙,请尝蛋糕;舅舅质问为何归,没安全感,要走,表哥不留;

在舅舅舅母摄合下喜欢上表哥,反被舅舅责问,速回,恨死他们父子。
\linebreak
\linebreak
丢工作,找工作不顺;表哥LinkedIn停于6 connection;

发信问表哥婚事意愿,被否,first cousin藕断丝连,遭舅舅邮件911谴返恐吓;

收到offer北上决战,舅舅打911并放弃谴返,绝别;

表哥大型望远镜,LinkedIn connection诱惑,遭鸡犬相闻老死不相往来拒。
\linebreak
\linebreak
十二月,二月,四月,六月(五月底),七月底的拒信和八月,任何一次我回去他都不曾真正彻底地拒绝过我,这究竟算是为什么,就只是为了考验我吗?
\section{小小潜力股}
\label{sec-8-34}

  在对表哥的不断琢磨里,我也在反复不停地琢磨着自己。我是一个什么样的人,我
期待怎样的生活,我喜欢表哥什么,他在我的爱情、将来的婚姻以及以后的生活中到底
都扮演怎样的角色,我与表哥的爱情、生活模式会是怎样?

  几个月前一位在国内创业的海归朋友问我,要不要和他一起创业,我拒绝了。拒绝
的理由是,他有婚姻有小孩有家庭,生命中最美的过程最美的年华他都体会过了,所以
现在是他创业的黄金时代;但是我,虽然活到32岁了,对感情的感悟只有6岁小P孩的理
解,男朋友没有爱情没有,婚姻家庭小孩更是遥遥无期,我都还没有体会过生活中的美
,我又有什么心思动力激情去搞那些更挑战人的创业?虽然我傻到现在还没好好谈过恋
爱,但我还没傻到对自己连个基本的认识都没有。 

  我是一个不算聪明(当然还是有点儿小聪明,不然怎么考上研究生以及来美国),
但实事求是、脚踏实地的人。虽然我只能以蜗牛、乌龟的速度向前爬行,但我还算是有
毅力的,保不准哪天我幸福了,我还真能超过兔子。

  国内硕士三年是我最激情飞扬、努力拼搏的三年;但美国的五年却是我最浑浑恶恶
、彻底沉沦的五年,因为我得不到想要的爱情,我不完整我缺“心”眼。

  如果一年多前的十月我没说错话没被人肉,如果学校老师顺利帮我办了STEP延期我
没回舅舅家没有遇见表哥没被点燃,那我会不会就像任何正常人一样顺着生活原本的轨
迹平平淡淡地走下去,凭借长相工作等外在条件挑选恋爱对象,恋爱结婚“生子”,波
澜不惊平平淡淡地生活?但遇见表哥我跌入了“万劫不复”的深渊。

  爱情是一个永恒的话题。现在还活着的千千万万遇见、错过、得到并享受爱情的人
们,你们有没有一个moment会觉得这个人就是我一直想找的人,若不是迫不得已,生离
死别,你都不愿意轻易放弃?爱情很美,但它也很折磨人、带有毁灭性。
 
  幸运又很不幸地,这个表哥被我遇见。遭遇爱情,像任何体验过的人所感悟的一样
,我也经历过犹豫彷徨徘徊不定,被舅舅帽子砸走一次,气走一次,从当时对表哥家庭
的诸多担心,到911时舅舅这个我心头的精神傀儡“自杀”后,我越来越清楚地认识到
,我喜欢表哥,真正喜欢,发自心底的喜欢,不因为舅舅逼迫舅母摄合,不因为表哥诱
惑,而是因为我自身的需要,因为我对幸福的渴望。

  19岁那年的灾难让我对人生有了基本的认识,在学习工作上绽放出蓬勃的生命力,
但初中高中那压抑了五六年的情感思维却未随生命力的迸发而变得成熟。我的情感成熟
得太慢了太慢了,比蜗牛爬还要慢。这种智慧与情感思维发展的不平衡大概是我今天或
者以后生活不幸福的根源吧。
\section{小小潜力股(2)}
\label{sec-8-35}

  我小的时候很听话,能干很多力所能及的小活儿,爸妈很宠我,爸爸尤其宠爱我;
而不管是上学前还是上学后我随堂哥带领的小分队去放牛,哥哥也一直很照顾我;我上
小学初中,学习成绩都很好,前面又有二姐铁饭碗光环的笼罩,初三获得了一生中最值
得回忆的友情,老师们也都很照顾(因为与二姐的朋友关系)、很喜欢我。上高中没人
喜欢我了吧,我的思维又处于停滞不动状态。所以在上大学前我基本就处于被关心被照
顾状态,上硕士时还享受导师的提携“宠爱”,和那好朋友四年的关心照顾。我的精神
,更确切地说,我的情感什么时候独立过?

  我是依赖表哥的,情感上是绝对的依赖。我有过十九岁的精神重创,有过国内硕士
毕业时的忧郁,表哥以他属马人O型血特有的光明乐观坚定驱走我心头的阴霓,让我放
下所有伪装的坚强,呈服在他的保护羽翼下。

  是的,我非常享受表哥的呵护照顾,那是我喜欢表哥最主要的原因。那我能因为这
一点就嫁给他吗?他能为我提供、我们能共同创造基本物质生活条件吗?可以的。表哥
也是能屈能伸的,他可以按舅母的意愿在他家乡附近找到工作,也可以随我到大些的城
市;我现在找工作不太顺利,我也可以先随他到他家乡呆几年,毕竟两人都大龄了,早
点儿要孩子是重点。等这些都解决了,我年轻可以工作的年限要多些,表哥可以再稍微
多牺牲一下嘛,随我到大些的城市这样两人都可以工作,经济压力也会小很多。

  不可否认,最初几年有家上有老下有小,一人工作生活略显拮据在所难免,但一切
困难都只是暂时的。与表哥安于贫穷不同的是,狮子座AB血型的我是有着物质欲望的,
但这份欲望是建立在自已努力的基础上。也就是说,我不因表哥贫穷而选择表哥我是选
择了自己主观感受的幸福快乐(物质放在次要地位),那我希望与我生活的表哥也同样
是幸福快乐的,他可以不为钱困,随心所欲地做他喜欢的科研。而我反过来,我获得了
情感上的极大满足,我也终于是可以随心地去追寻物质追寻梦想。其实我所谓的梦想也
不过是等一切盛世爱情激情落幕,去开开公司创创业。至于能不能挣钱,挣多少钱不是
第一位的,排第一位的是我那个年少时的梦想我去努力过了(至少也得八年十年之后吧
)。我们高三老班喜欢晚饭后上自习前念报纸给我们听。其实热爱学习的同学估计都只
做作业学习去了,我却是那个被老师牵引喜欢听报纸的人。那时的故事比较多的大概是
宣扬年青人的努力与激情奋发吧。我骨子里好歹也还算是个热血青年,我与故事里的那
些主角是同一国的。

  表哥,对于我来说,一定是幸福的选择,是痛并快乐着的选择。但是,我,对于表
哥来说呢,他会像我如饥似渴般想要得到他一样,他会希望我成为他生活、蓝图的一部
分吗?
\section{表哥的拒信}
\label{sec-8-36}

  表哥的拒信也是半真半假的,一些话一些句子显得真实,可以理解,而一些话一些
句子就是睁着眼睛说瞎话,对于他睁眼瞎的话我又该如何理解?

  他说我去他家的最后三次他都给了我答案,但那三次给我的答案却与这封拒信截然
相反。十二月、二月和四月他都通过语言或者行动给了我希望,但这次的邮件里他明确
地说不想我作他的女朋友。

  对于我来说,对于他来说,我一定是pushy了,因为我的闪婚情结;而且我的确没
有listen to him。可是我能听他的吗?我若是听他的,十二月第一次回去不就得发生
点儿什么,还有四月?他可能在等我与他发生点儿什么好让他心理上不把我当妹妹,可
我又何尝不是在等待他承认点儿什么,比如作他女朋友,好让他所想望的那个发生点儿
什么变得合乎情理些?

  而且五月底回去,我一定是没有听他的。那次是舅舅与表哥的第一次严重分裂,虽
然我口口声声向表哥表达着我对舅舅的不满,但在这个分裂口上,我不敢,我对表哥的
信任不够,每次office,家里想要同他聊点儿什么都被他拒绝,我对他了解得太少了。
而如果我听他的,如果我对他想得悟得足够透彻,我就该知道我说过的话,问过的问题
,他都有放在心上。十二月我问他会做饭吗,他说在韩国时他也自己煮面条。二月我要
他煮的面条被舅母代劳了,五月底回去待我洗完澡他便急着让我品尝他做的蛋糕,虽然
蛋糕并不美味,但那是他尝试着、学着做的,仅凭这点儿就很可贵了不是吗?还有他八
月立在床头的望远镜。

   表哥说他根本就不喜欢我最近的行为,应该电指二月走和五月底走的这次。可是这
两次我心里有多慌有多乱他一点儿也体会不到吗?二月走后为什么会分?毕竟只见一两
次,感情基础也不够,还被那破老头砸得连做人的基本人格尊严都没有,那还能不分吗
?五月底走这次现在也是想得清楚得不能再清楚了,我也很后悔当初没能真正走进表哥
的世界,没能真正悟出那些深意啊?我两次气头上的行为他当然不喜欢,我自己也还不
喜欢呢,可是之前十二月,中间的四月我“正常”的时候我们不得都处得好好的吗?表
哥也从来没说过他不喜欢我十二月的表现。

  表哥的拒信里,“When I am at work, I am working.”中的”working.” 和“
This is the fourth time I am telling you: I don’t want you as a girlfriend.
”中的”I don’t want you as a girlfriend.”用了Bold加粗。遥想十二月回去我们
从图书馆出来,迎面而来的几老美学生本能地鄙视,我终于是等不及表哥为我开门,自
己气乎乎地推门冲出去了。我气什么,不就是气老美学生世俗,气我并不是他们想象的
那样,我毕竟还是喜欢表哥的吗。瞧,我在反抗老美鄙视的同时,自己不也同样在猜测
别人的想法吗?(世俗到底是什么算什么?)回到表哥office后,表哥自己没话找话地
插了句,“我现在还是学生。”以前表哥是学生不愿意谈恋爱,现在如同拒信里所说,
他现在是在working了,我还是不能作他女朋友吗?

  他现在要把我当妹妹说得天花乱坠,什么”I really hate a person who nags me
like that”十二月最开始我有那样,可是后来四月我不都改了吗?什么”Do not 
spend even 1\% of the time with me.”,四月回去时他自己向我强调平时晚上他一定
不去office,那天他是为了陪我他才晚上去office的,他自己什么时候只花过1\% of the
time在我身上?既然要把我当作妹妹,四月问他的时候我就早说过了几年之内不再回来
,他还希冀我回去吗?我就算像他所说有valid reason可以回去,我又该要如何面对这
宠然大物的表哥?”stay away”才是对自己来说迫不得已的最好选择啊。他信里的”
first cousin”, ”valid reason”回去都是狐狸尾巴,我想把它们当作壁虎尾巴扯断
,却终究显得那么力不从心。

  表哥的拒信被我翻译得一塌胡涂,正面与反面的,肯定与否定的信息混作一团,我
理不清头绪。只是这是一次书面正式的拒绝,那八月回去以前硬闯的房间得先敲门了(
为什么他明明醒着听见我敲门却故意不应还要我自己推门?),以前即便他不回我也会
随便发的邮件终于也在舅舅书面警告下停止了,与他断了一切联系。可我内心的矛盾纠
结啊,何时为了?
\section{结语}
\label{sec-8-37}

  每个人都有缺点,表哥也不例外。表哥是残忍的,他言语不多,却用有限的交谈把
人三六九等分明,他用冷酷到极点的阈值(例如舅母说要给我\$70他却说还以为是\$40呢)
把人拒绝于千里之外,秒杀周遭美女无数。他能冷酷地筛选到这世上最淳最美的爱情,
但终究难掩一路艰辛。表哥能45岁还不结婚,他自有他的软实力累积。可能曾经沧海难
为水、高处不胜寒吧,他也有他的爱情(和事业)理想,有他的深遂和博大精深。我想我
们都还在等待。若有一天真如早前舅母电话里所说,表哥订婚了或是结婚了,我那时会
怨恨表哥吗?我想我应该不会,我能有的应该也只能是忠心的祝福,希望他能幸福。

  再说舅舅。我以自己这专家鉴定的冷血去挑战他那十岁就开始混社会的冷血,我终
究还不是他的对手。表哥是敏感算计的(十二月我清楚地告诉过表哥我的工作情况),
舅舅却更是恶俗势利的。现在的困境究竟是表哥的主观意愿还是舅舅强迫所至,我分不
清楚。从前去年十二月我不是在同表哥一人谈恋爱,我是在同表哥一家人谈恋爱,现在
分手我也不是与表哥一人分手,我是在同表哥一家人分手。所谓大难临头各自飞大概如
此吧。舅舅同时也是暴烈自私冷酷的,是恶俗到泯灭人性的。若容我愤闷头上出口恶气
,容我再稍微发挥一下自己的发散思维和极端想象力,舅舅是否会恶俗到八年十年后我
买好房可以接他们家人享轻福过轻松日子的时候才允许我与表哥结婚?现在年轻人向往
的幸福,舅舅会理会吗?他老人家多年磨砺残忍之至,他才不会考虑我们想不想要孩子
的吧?现在的我究竟是该怀疑表哥的爱情还是该怀疑表哥的生理功能(全是气话大家忘
了吧)?但不管我如何极尽所能地想要把舅舅描述到不可理喻的地步,我终究无法抹杀
一个事实就是,他是那么无私地爱着他自己的孩子,他是一个伟大的父亲。

  我是冷血的,暴烈的,反复无常的,但我心里也是有爱的。我的多重性格,狮子座
与肖羊属性的冲突,AB血型的两面性,我的彷徨犹豫徘缓不定,把原本美好的感情折腾
得支离破碎。我一步一步认清自己,表哥却是步步为营节节后退。我已经寻找了五年,
我已经32岁了。我是应该嘘唏这擦肩而过的爱情,还是应该把它牢牢守住?遇见表哥之
前,我可以理直气壮地凭长相、工作工资等物质条件去挑选恋爱对象寻找爱情,现在真
正明白相遇是一种缘份,爱情可遇而不可求,任凭我自己单方面努力都终将显得无能为
力力不从心。现在,我还能做什么?坚持与放弃又有什么不同?我究竟是该像表哥一样
坚持爱情理想,还是在现实面前将就寻得平淡生活?张爱玲说,生命是一席薄绢,上面
爬满了虱子。面对这悲怆的人生,我又还能说什么呢?

  亲爱的读者,我的故事到这里就结束了。斗争的最高形式是战争,但战争也是残酷
的。舅舅对我使用冷暴力,我以暴制暴地利用战争缴杀了心里的仇恨,付出了物质的代
价,但这场战争同时也把舅舅和自已推上了舆论的风口浪尖。可是我又能说什么呢?故
事的开始,我是多么急于想要证明自己的清白,而写到结束,清白什么都不再重要了,
甚至结尾不惜要抹黑一下舅舅以平愤懑。由于年龄以及积累沉淀的不同,小时候的故事
闭上眼睛也可以写好多篇,到结尾却只能挤篇副,说什么都显得言不由衷词不达意。写
这个系列故事,对自己作了必要的反省或许是我写故事的最大收获吧。故事里的每个人
都真实存在,有些想法是我极端暴烈的产物,大家看过就忘了吧。

\chapter{第二次写}
\label{sec-9}
\section{前言}
\label{sec-9-1}

发信人: deepwaterooo (梦魇), 信区: Dreamer

标  题: Re: 成长的故事 -- 我和舅舅

发信站: BBS 未名空间站 (Thu Apr 19 01:42:37 2012, 美东)

去年写这个故事,我清楚地知道是因为911事件,是舅舅亲自把我推到了舆论的风口浪
尖。为了自己的生存,也为了澄清曾经的冤案,一部分内容写得非常狠毒。当时的想法
只有一个,既然网络暴力能把一个正常人的生活撕毁到“出不了门,上了不班”的地步
,那,形势所迫,就让一切尽可能客观地还原历史真相。而还原这个真相所付出的“流
血牺牲”,或许是平息这场暴力,共建和谐社会必须付出的代价吧。也请大家原谅我对
曾经的冤案无法再守口如瓶,原谅我那时的狠毒。

同样客观讲述的,还有我与家簇亲人关系、与舅舅表哥关系的描写。那时的纠结也客观
真实。

如果说之前写这个故事是打保卫战,正当防卫,那么现在续写这个故事,是为了不愧对
自己的良心,给故事作一个更为客观深刻完整的理解与结局。必须承认,想要续写这个
故事,需要太多的勇气。去写和写完它,培养自已与挑战自我相克相生。这一部分,会
同样客观真实地描写家簇亲情、爱情关系,会试着深度挖掘主人公的性格和人物之间的
关系。希望我能把它写好。
\section{考试}
\label{sec-9-2}

对表哥的思念越来越深,无以为藉,一时半会儿又不能回去。日子过得空荡失落,回想
起四月还是五月底回去时表哥问我后来考过那个认证没有,我答没有。之前确实想再考
一次的,但后来工作正好换了比较感兴趣的方向,我自以为是地认为我已经有了实战经
验,大概不需要再考那个什么认证了吧,所以去年四月有一次家门口考试的机会也没有
去。想到表哥提起过这件事,便跑去翻它们的日历,十月正好还有一次家门口的考试机
会,反正没事,便报了名。考试本身没多大难度,只是考前受版权限制没法熟悉考试所
用的软件界面。后来那天请了半天假,周五中午开车过去,顺利地过了考试。

高中老班一次晚自习前给我们读小报说XXX名人喜欢春天出去打猎,秋天在家读书。这
个观点投射到我身上就是秋天好好学习,春天好好玩。我喜欢秋天。夏天的高考、04年
6月的GRE都考得比较差(中考除外),秋冬季节的考研考试、TOEFL都还不错。感觉秋天
的萧瑟寒气把脑袋冻得很清醒,春天就太困太浮躁了,不太容易静得下心来做事情。

我久违的韩剧啊,考完试的周末便找了个短剧翻了翻犒劳一下自己。忙考试的这几周也
没顾得上打电话给家里。生活总是在你最漫不经心的时候给你致命的打击,这话用到我
身上再正确不过了。有了快一年笑容的那年高考就是最好的例证。那个周末,打电话到
家里,妈妈接起的电话。妈妈说家里一切都好,让我不用担心。可是这次电话里妈妈故
作镇定言谈间却流露出无法抑制的悲伤,我知道,家里一定出事了!
\section{爸爸}
\label{sec-9-3}

给妈妈的电话里,我不好追问,怕她不说还会引得她又伤心。挂断妈妈的电话,便赶快
打到二姐手机上。电话里,姐姐的声音也带着哭腔。她没有直接回答我的问题。她说这
几年,人多车多,村子里、周围庄上很多人,与爸妈上下年龄的人离去了,也有好几个
年轻人出车祸也去了。姐姐的话说得客观、似乎与爸无关,我却听得一头雾水,不祥的
预感笼罩心头。姐姐接着问我,近期能不能回国一趟。我答说要看工作看具体情况看老
板给不给批假,同时还涉及到身份签证问题。姐姐接着问我,我回国之前,爸妈若有一
人离去,我将来会不会遗憾。这是我长这么大以来还从来不曾想到的状况,爸妈都还这
么年轻,爸爸的身体一直都很好。“会的,一定会的。我爸,我爸他到底怎么了?”我
追问着姐姐,她也忍不住抽搐起来。“我已经是长这么大的人了,我也是爸爸的女儿,
爸爸在家里出什么状况,我也跟你们一样有权知道!”姐姐说,爸爸病了在住院,情况
很不乐观,要我有点儿思想准备,却没说爸爸得了什么病,到底严重到什么程度。电话
里姐姐再也控制不住情绪,大哭起来并匆匆挂断了电话。

爸爸是我心目中最值得尊敬的人,我想回去看他、希望能在病床前照顾他。对我来说,工
作可以以后再找,工资可以以后再挣,这些对我都不重要,唯有爸爸最为重要。我打算
向老板请假。只是回国期间身份对我来说还是问题。我相信表哥是喜欢我的,我与表哥
最终还是会结婚的。我与表哥现在的问题归根结底只是舅舅一人在中间作梗。那时的我
很傻很天真,以为只要我能说服舅舅让表哥与我把结婚证领了,我就可以回家几个月好
好照顾爸爸,直到他病好,婚姻问题也解决了,两全其美。可是当我电话打到他办公室
,舅舅一句话就把我解决了,他说表哥感情不到位! 接着又blabla说了一堆学生回国签
证的问题。Full time学生回国与F1 OPT回国能一样吗?可是是我打进的电话,又不能
多说什么,就忍着默默地听。我不再多说任何话,电话就该结束了,舅舅又加了句,
911多打几次就能把你直接谴返了!冷不防舅舅来了这么一句,我的小宇宙终于又爆发了
。我是直接挂断的电话、还是直接关机我已经不大记得了,我只知道我出离愤怒了。我
说过要去你家了吗?我想过要去你家了吗?难道你作为一个大学教授就可以这么肆意侮
辱别人的人格吗?去年二月回去是被几天前舅舅一个骂我是骗子的电话激回去的,二月
回去拖走了我所有的行李,还了从他家借的\$4000块钱,了断了与他家所有的联系,临
走前还要被他上堂政治课羞辱一番;5月底回去被他丢了礼品吓跑了,8月却又被他气得
回去决一死战。为什么自打与表哥相处两天后,与舅舅的沟通交流都变得这么困难,为
什么有了表哥后每次与舅舅打交道我都感觉这么憋屈?这一次的出离愤怒,我终于认识
到,舅舅对于我就如同观音菩萨之于孙悟空头上的紧箍咒,烦都烦死了!那以后我再也
没打过舅舅的电话。别问我对他是否还有牵挂,我也需要保护我自己的生活。
\section{爸爸(2)}
\label{sec-9-4}

几经周折,终于打听到,爸爸出车祸了。我这边时间考试那周周末周六晚上(北京时间
16日下午),爸爸给住在县城的三姐送些自家菜园产的菜,在姐姐那里吃过午饭回来的
路上,骑自行车安分守已地走在前面,却被后面一个歪么子(技术不高)骑摩托车拐弯不
减速的人撞上,把爸爸撞飞起来,落下时头先着地,伤着了大脑。三姐最先赶到现场的
,姐姐说,“爹,我是玲儿,你看看我,叫我一声好不好?”爸爸早已昏迷。送去医院
,当晚作了开颅手术,因脑内另一部位出血,第二天上午又作了第二次开颅手术。医生
说,因伤了脑神经中枢,老年人很难苏醒康复;若能苏醒,一定是奇迹;但就算醒来,
最好的情况是会成为一个植物人。妈妈非常后悔,觉得应该让爸骑电动车,因为爸爸骑
电动车一定会带头盔。但那天爸爸骑的是人力脚踏车,头部没有得到任何保护。我几乎
没有恨过任何人,但得知爸爸的状况,我恨死了那个骑摩托车的人。妈妈说那骑摩托车
的人水平非常差,在爸出事之前他已经撞过好几个人了。爸爸的身体一直非常好,除了
年轻时得过一场重病,晚年基本没生过病。可就是这样一个几天前还鲜活完美、攫烁抖
擞的爸爸却因为这个人现在不得不躺在病床上,接受生死考验,或许还能保存生理生命
,却永远无法再苏醒过来。

妈妈听医生说,现在爸爸还在特护病房,看半个月左右爸爸能否苏醒。亲属每天只能下
午固定时间去探望半小时。家里大姐姐辞了工作,每天24小时候在病房外。妈妈说让我
稍等一段时间再回去,看爸爸的情况是否有好转。妈妈说希望等半个月爸爸醒来后我再
回去我还能陪爸爸说说话,好让爸爸记起我们来。

爱恨一念间。尽管之前电话里舅舅无限同情无比遗憾地说表哥感情不到位,希望我能理
解,希望我能再等等。可是我爸病成这样,我都还不能回去,让人怎么理解?事情发展
到今天这个地步,舅舅与表哥之间,至少有一个是假的。电话里既然舅舅已经这么说了
,那表哥一定是假的,骗人的,藕断丝连,玩弄别人的感情,是这世界上最可恶的恶魔
。爸爸生病的这段时间里,对爸爸的爱有多深,对表哥的恨就有多深,天知道我有多爱
自己的爸爸!
\section{我和爸爸}
\label{sec-9-5}

我在前面交待过了,中学时代的我因为小学毕业那年暑假的一次不愉快经历变得忧郁孤
僻,但上初中之前童年的我还是有着非常多的幸福快乐,那时的快乐总是有着爸爸的参
与。

农家的小孩力所能及地参与劳动总是不可避免无可厚非。清楚地记得有一次随爸爸去地
里用板车拖梆成捆的小麦,爸爸抓住一只刺猬。那时的乡村还没有环保、热爱大自然的
概念,加上小野生动物泛滥殃及庄稼,村子里有不少人会寻觅野味,兔子、野鸡、刺猬
、蛇都可以成为桌上餐。爸爸把那刺猬用细绳拴住一只腿,捆在了板车车轮钢条上。那
时的我玩心大发,把刺猬从钢条上解开来,还对在帮爸爸搬麦捆的二姐说,“姐姐,姐
姐你快看,我把刺猬弄趴着了!”二姐比我大五岁,也比我懂事很多,她也没搭埋我,
帮爸干活累着呢!可是不大一会儿功夫,等爸爸捆好一板车的麦子,要回家了,我看着
的刺猬却也不知道哪儿去了,带着腿上的那根绳儿跑了!想来刚刚趴着的那会儿已经打
算逃跑了!我们三人在附近走了一圈,没找着,就回家了。那件事爸爸后来批评了我,
又想玩儿,又不看好它,原本一桌美味儿就这么没有了!

与妈妈相比,爸爸对我们姐妹更为慈爱。小的时候走亲戚,主要是去镇上外公外婆舅舅
家,基本上下雨天大家都打着赤脚片,难免有时候脚上扎刺或被玻璃划了。记得那时候
可能我好撒娇,或者因为我最小,有时候确实走不动了,爸爸就会把我扛坐在他肩头,
剩下姐姐们在地上跑。上学了,想要一双雨胶鞋的愿望就特别强烈。经过我和姐姐几次
三番地絮叨,后来爸爸答应了我们,给我和三姐各买了一双黑色短筒雨胶鞋,姐姐的比
我大两个号。我清楚地记得买回来那天是晴天我们也在家里穿着,怕沾上灰外面套两个
塑料袋,晚上休息时我和姐姐都穿着那新买的鞋睡了一夜到第二天才淡下瘾来。那时的
我们是多么快乐!

还有一年冬天,乡亲们就说今年这么大的雪,出去肯定能逮得着兔子。村里有年轻人出
去了,爸爸那时四十多岁也按捺不住蠢蠢欲动的心,我想着好玩儿坚持要爸爸把我捎上
,妈妈便同意了。吃罢早饭我穿了自己的小雨胶鞋,外面用塑料皮捆了绑腿一直捆到膝
盖上,拿了长竹杆和网,就跟着爸爸出发了。爸爸走在前面,跨过一圻大沟,我眼望着
周围白皑皑纯净透明的世界,心里想着美味的兔子,吹呼雀跃,眼大心空,没等爸爸回
头搭救我,腿一抬就掉进沟里了。还好大冬天沟里没水,爸爸把我拉起来拍拍雪就没事
儿,我们继续向着目标前进。那天我们顺着兔子的足迹找了很多个小拱桥状的沟洞,就
那最后一个里面有只肥肥的大兔子。爸爸让我收声敛气,他在桥洞的一端把带有两根竖
棍的网插好,让我扶紧竖棍把它看好,爸爸在另一头用竹杆赶,很快兔子就成为了我的
网中猎物。待我抓住了兔子,爸爸把它拴好,我们也该收工了。回到家已经半下午了,
爸爸感慨说要是早上没捎上我,没准还真逮不着那兔子。那天的晚饭吃得特别香。
\section{我和爸爸(2)}
\label{sec-9-6}

爸爸一生中最大的爱好就是捕捉鱼类,他的钓“鱼”水平也远近闻名。小时候泥鳅鳝鱼
比较多。记忆里爸爸有着长的短的各种各样自制的钓鱼钩,是用长钢丝在磨刀石上磨尖
了顶端再弯曲成钩状,尾端还栓上一绳和木制水漂。这些都是爸爸用来钓鳝鱼的。妈妈
给我讲过爸爸历史最佳成绩是钓到一条七八两重的鳝鱼(方言称“黄鳝”),因为大,
卖了个特好的价钱。想来爸爸也有着他自己的执着,妈妈说那是爸爸发现池塘里那个鳝
鱼洞后屡败屡战,钓了五次才钓上钩的。我的脑海里也有爸爸钓鳝鱼的记忆,不过那时
我P大点儿小孩,不懂事还直接搞破坏,爸爸一般都不愿带着我。

那时候的乌龟也特别多,并且没人要买乌龟,它们就成了我儿时的玩具。与邻居家的小
朋友相比,爸爸帮我抓的乌龟总是比她的大,让我多少有些飘飘然。爸爸帮我用绳子拴
住乌龟的一条腿,我提着乌龟绳子走。我也会把它拴养在稻田的一角。然而对于那些乌
龟,我一般大概也只有两三天的热情。那些惨死(饿死)在我手下的乌龟冤魂没有上百也
有几十吧。而最终饿死的乌龟又都会被爸爸拿了龟壳卖给货郎或换作了它用。二十年后
在美国,一次钓鱼钓起一只乌龟。我对它并没有什么想法,只是可惜自己的一只鱼钩被
它衔在嘴里。于是我以大欺小、以强欺弱,一脚踩了它的龟壳,一手用手钎夹住了我的
鱼钓想要把它解救出来。然而我的牛脾气够倔,却终究抵不过它的狠,它以弱小身躯区
区几颗小牙平衡了我有杠杆作用的钳子,咬断了我的鱼钩,威武不能屈。至此,我只好
作罢,放它回到了它的清水世界。

那时割麦插秧的农忙后季节,我的一大乐事就是随爸爸去沟里刨泥鳅。一条水沟只要通
过水,就会有泥鳅鳝鱼的驻扎繁衍。待插完秧不再忙的时候,爸爸就会找那些水沟,把
水放干,然后从那泥浆里刨出一条条活蹦乱跳的泥鳅来,有时也有些鳝鱼。我小P孩帮
不上什么大忙,便趴守在田梗上,争抢着要帮爸爸转装泥鳅的盆子。待一两个钟头过去
,爸爸刨完一条水沟,盆里已有一盆底的收获,那是家里通常会有的没有成本的牙冀。

稍长大后便有了磁电机,用硫酸溶液充电,爸爸欣然前往地买了一个。一次充饱电后大
概可以持续放电八个小时。到那时我就成了爸爸的小跟P虫。晚春夏天的晚上吃过饭等
到天黑,我就把妈妈特意缝制的鱼口带挂到脖子上,随爸爸一起出去打鱼。爸爸把两根
可以放电的竹竿放进水里,按一下放电按扭,顺势将电昏了正死窜乱跳的小鱼兜到网里
。等爸爸把网兜提出水面,我人小个儿小手也矮,就抬抬手把网里的猎物抓到自己面前
的鱼口带里,省去爸爸弯腰折腾的时间,我们父女两个相互也有个照应。夜里露水降下
来,空气滋润清新,我的小脑袋瓜子那时候也总是特别清醒,有时候我也会提醒爸爸网
里像是有条蛇(一般都是水蛇没毒的),爸爸仔细看看,就会把那蛇扔向远方。我们一般
待到夜里两三点钟,估算着电机可能会没电了就回家。那时每晚的收获连小仓鱼带泥鳅
鳝鱼的大概有五到八斤的样子。第二天一大早,爸爸就会赶到集市上将贵的泥鳅和鳝鱼
卖掉,剩下小仓鱼留在家里我们自已吃。

记忆里夜暮中的爸爸与我,连同那有着青蛙鸣叫的青草世界,一切都显得那么和谐。后
来我上了初中,爸爸要求我以学业为重,就再也没有了夜晚随爸爸出行的经历。那夜幕
中的清草味道就永远地定格在了我的记忆里。
\section{爸爸}
\label{sec-9-7}

得到爸爸生病的消息后,我想不管我什么时候回去、最终能否回去,我都要先跟老板打
好招呼。妈妈让我再等一等,我查了查机票,两三天内的基本都还有票,便也没着急订
机票。

家里三个姐姐都已结婚,各有一个小孩,他们也都有了各自的小家庭。对于家里,除了
之前提到过给家里寄过两次\$500,寄过一次包裹,我刚工作不久,刚还了从舅舅那里借
下的钱,手上的积蓄不多,加上之前在国内读书欠下姐姐们的钱都还没有还,也就再没
有向家里表示过什么了。这次爸爸意外住院医疗费应该挺贵的。不管姐姐们会不会拿出
些积蓄来,我最小又是一个人,没有小家庭的矛盾问题,我都应该多出些。于是接下来
的周六到邮局给大姐夫寄了一张\$8000的支票,折合五万人民币,应该能稍微维持一段
时间了。我想过,如果万一没能用完,也可以用剩余的钱还欠下姐姐们的债,能还多少
是多少,所以一定会派上用场的。电话里便对姐夫说让他们尽最大努力给爸治病,不用
担心钱的问题,如果不够我还能再寄些。

把向老板请假的邮件发出去后,我就只能天天数着日子过。每天都会打个电话回去问一
下爸爸的近况。姐姐们说爸爸基本还是之前刚做完手术的样子,一直没有苏醒的迹象。
学医的二姐说她拿手电筒照到爸爸眼睛上,爸爸连最基本的光反应都没有。我劝姐姐说
,爸爸年级大些,医生说等半个月,我们就耐心些,多等些时日,让爸爸有足够的时间
能够醒来。我自己也在网上查到了很多这种意外事故的康复苏醒案例。快的年轻的三五
天就能醒,年级大些的最长的也有快八个月才醒过来的。我想爸爸一向身体很好,我也
深有体会爸爸从来都是乐观坚强的,我一边作妈妈姐姐们的思想工作,一边祈祷着希望
爸爸能顺利地挺过这一关,早日醒来。
\section{爸妈的晚年生活}
\label{sec-9-8}

到这次爸爸生病,我离开家已经五年了,五年里我一次家也没回过。最开始一年半是有
奖学金的,寒暑假回国最方便,但爸妈不舍得我奢侈浪费不让我回;08年夏天是最有热
情想回去的,想回去看看爸妈,也见见我的同学,但那时已是自费学生,舅舅是我的经
济担保人,自然也要听从舅舅的意见,最终暑假回国的计划没能成行。再后来毕业后
OPT回国就显得更困难了。

这五年里,其实爸妈已经出现了些衰老的症兆。09年和10年的冬天,妈妈都因为夜间起
床受凉导致轻度中风和面瘫。两次生病前后各住了一个月左右的医院。这些都是姐姐们
在照顾,我也没太往心里去。我的性格很像爸爸口直心快,有时候说出来的话就真像是
没有经过大脑的,干起事来也毛毛燥燥。这五年里,爸爸从香椿树还是从枣树上摔下来
过一次,摔断了胳膊,打石膏带子把胳膊挂脖子上挂了一百多天;农忙时在家做饭剁骨
头不小心伤到了手指,听姐姐说伤得挺严重的,前前后后拖了可能也有一个月。妈妈说
爸爸身体很好,基本不生病,那时的我总是很欣慰。

妈妈多年来在家里大门不出、二门不迈,06年走之前我对爸爸说给他买辆电瓶车,他很
开心,妈妈也因此终于可以随爸爸在远亲近临间稍微走动。听姐姐们说,爸爸有了电瓶
车后,其实心里一直痒痒地在打摩托车的主意。家里大姐夫、二姐夫之前各有一辆摩托
车,大姐夫的骑车技术还不错,二姐夫就稍差一点儿。他们以各自骑车的经验教训,坚
决反对爸爸买摩托车。这件事情我也站到了爸爸的对立面,爸爸的这个希望的肥皂泡最
终扑了空。妈妈说晚年的爸爸喜欢攒钱,我问妈妈爸爸攒钱作什么,妈妈说她也不知道
。我自己也问过爸爸,他只简单地说,“我攒钱也没什么用,我不攒钱。”我也不好再
多追问爸爸些什么。

几年前,二姐找朋友帮爸妈做了好几个不同款式的便携式农用小推车、拖车,不需要爸
妈再抬或扛粮食,只需推拉就行,爸妈用得很方便。为我们姐妹四人辛劳了一辈子的爸
妈在10年我工作后终于不再种地,改成了为三姐种。三姐三哥(三姐夫)想种家里的地,
大姐夫就安排三姐一家种,爸妈帮他们常年照看地里的庄稼,除了给爸妈些口粮,其余
所有收获归三姐一家。虽然爸妈还是没能彻底从农活里解放出来,但和以前相比也轻松
不少。

农村逐渐机械化后,家里便不再喂牛;猪的肥肉太多吃了不健康也没人吃,猪瘟也多,
家里便也不再喂猪;这样家里的小动物便只剩下狗和小鸡。妈妈饲养小鸡很用心,总想
着喂大了他们好等姐姐们回来给姐姐做美味吃。

晚年的爸爸一如既往地喜欢他的电瓶喜欢晚间出去晃一圈,被我电话里几次三番地劝,
人老了天又黑,身边又没个什么人,万一出事出现意外怎么办?后来爸爸答应我只傍晚
出去下濠子(长圆锥形竹笼里面放蚯蚓,蟮鱼只能进不能出),天黑之前就回家,第二天
早上天亮后再去取濠子。
\section{睁开眼睛}
\label{sec-9-9}

等待特护病房里爸爸的消息显得特别煎熬,姐姐每天24小时守候在病房外,而周围的亲
人,妈妈的姊妹、侄男侄女,姐姐姐夫们每天能真正接触感受到爸爸的时间也只有半个
小时,护士也还容易不耐烦。半个月15个24小时我数得很辛苦,从姐姐们那里我知道,
爸爸渐渐地眼睛能睁开条缝,左手手指能稍微动一动;若用指甲去掐爸爸的脚趾头,爸
爸感觉到痛了腿也还是会稍微侧一下。但医生说,这都不是意识苏醒的症状。医生建议
病人转入普通病房,姐姐们也希望能够更好地照顾爸爸,大家便照做了。

三姐同她的同事换班挤班后一周大概能有一两天的空闲在医院,二姐二哥除了医院的正
常上班家里还有私人诊所常年工作忙外加一个上高中的孩子,二姐一周也会去照顾爸爸
一两个晚上;大哥周六周日倒班也会呆在医院里让大姐能够稍微休息,住在医院附近的
四姨也会时不时地帮姐姐帮我们照看一下爸爸,剩余的时间也全都是大姐一人挡在医院
里。妈妈有遗传性高血压,年级也大了,暂时不需要她照顾爸爸一人呆在家里,四姨五
姨姐姐姐夫和我的电话常打回去关心一下妈妈的近况。

我问过二姐到底有哪些手术并发症会给爸爸带来危险,姐姐说,很多原因很多可能,肺
部感染、体质电解质紊乱、多器官衰竭、脑死亡都有可能。因为爸爸现在处于插氧气状
态,肺部插了根支管连到外边来,需要每天抽痰化痰。以我有限的理解,肺部感染是四
大原因里最危险也最容易感染的后遗症。我问二姐,我们作为家属,能否要求医生定期
,比如每两天每三天或每一周为爸爸作一次肺部检查,以保证我们给爸爸足够多的时间
康复。姐姐说不用,这些是住院的固定项目,包括体内的电解质平衡,都是定期要抽血
检测化验的。听姐姐这么说,我也稍微缓了口气,不是只有我一个人在盯嘱着爸爸的康
复。

爸爸转入普通病房后,有一次打电话过去,三姐在医院里,姐姐说爸爸醒着,我便要姐
姐把手机放到爸爸右耳边(爸爸伤在左脑,怕辐射也怕左耳左脑不好使),对爸爸说了几
句话。我告诉他我是XX(爸妈唤我的小名),我一时半会儿还回不去。我告诉爸爸,高考
那年是他和妈妈救了我,若不是因为爸爸我不知道自己现在会是在天涯海角的哪个角落
流浪。我深深地体会过爸爸的辛劳和坚韧,“这是爸爸您一生最艰难的考验,爸爸您一
定要像当年盼我参加高考盼我走出低谷一样坚持住挺过来,妈妈、姐姐和我,我们所有
的人都盼着等着您醒过来!爸爸您还没花过我挣的工资,没穿过我买的衣服,您还没来
美国转一圈,我们还有好多事情没有做,爸爸只要您挺过这一关,我们还有好多年能一
起过!”同爸爸讲这些话的我,眼泪噼里啪啦往下落,鼻涕身上面前到处都是,同爸爸
说了这短短几句话,我仿佛又重新经历了当年的高考,经历了一遍爸爸的人生。
\section{爸爸的故事}
\label{sec-9-10}

如同当年我的实验、开题报告做得好不是天生的,爸爸的博大坚韧也是经受过历练的。
爸爸的一生可以写一部传奇。

爸爸这边的亲人我只能追塑到爷爷这一辈,再长些的,妈妈或许讲过,但我都不记得了
。据妈妈说,曾祖父还算得上是文化人。爷爷走得早,妈妈听村里人说,爷爷大字不识
一个,居然当了一辈子的村支书。这显然很矛盾,我是不信的。我五岁时家里盖了新楼
房,给奶奶搬家时还发现一个四角砚台,大概是写毛笔字用来磨墨或是蘸墨的,不知道
是曾祖父还是爷爷用过的。我把砚台收藏过一段时间后来还是忘了(又或许现在还躺在
家里的某一角落)。

爸爸有弟兄三人,伯伯和叔叔都是普通农民,伯伯个儿小,叔叔英俊魁梧些。弟兄三人
里最能干的当属爸爸。听爸爸讲过,年轻时他尝试过很多事情,最喜欢的还是当邮递员
。当邮递员的一大好处就是可以有好自行车骑。我对自己骑自行车比较自信,爸爸的骑
车技术更高。爸爸快一米八的个头,很瘦,轮廓分明,我想爸爸年轻时一定很帅。

以爷爷村里村支书的地位,爸爸家庭条件算好的。然而爷爷只留伯伯娶了伯母在家给他
们养老,自作主张地将爸爸和叔叔都妱到了女方家。不知道是因为他们夫妇感情不好婚
姻实在没有幸福可言,还是结婚时爸爸过于年轻,尚未定性,婚后已经育有一个男孩子
的爸爸孤身一人回到了爷爷家。

我那个家境很好,在村里有钱有地位的爷爷却怎么也没能够想通想明白爸爸这件事。想
来当年爸爸回去时爷爷与爸爸之间一定是有过激烈论战的。但他们可能谁也没能真正说
服了谁。爷爷是那个年代里执迷不悟的家长,爸爸却是醒悟领悟了,是“众人皆睡而我
独醒”的年青人。在我的理解里,爷爷受环境条件限制他的眼界格局大概也并不宽广,
他未必有爸爸看得清楚透彻。大概爷爷觉得爸爸放着好好的日子不过了跑回家来,是非
常不孝让他很没面子,又或者挑战了他的权威,扫了他的威信、不可原谅不可纵容的事
情吧。他们父子间激烈争战的结果便成了,一天,在爸爸和奶奶都没有丝毫防备的情况
下,爷爷用一根绳索了结了自己当时非常年轻的生命。

爸爸多大年岁结的婚,爸爸离开那个家时那个哥哥有多大,爸爸是如何补偿当时的妻子
和孩子,这些事情我只间或听村里人提起过,但我从来没有正面直接问过爸爸妈妈。村
里大人们说,主要还是爸爸觉得爷爷指望伯伯一个人养老是指望不住的。爸爸因为内心
的善良因为对长辈的孝敬决意回到自己家,却不得不面对爷爷的离去和他自己极其不孝
的后果。爸爸大概也是真看清楚了想清楚了的,爷爷以那种方式离开后,爸爸还是没有
回家。

我想那段时间爸爸一定痛苦极了。当时的爸爸是否内心充满了自责和后悔,但爷爷已然
离去,爸爸再做什么都无济于事。不知道当时的爸爸是否像高考当年的我一样,对自己
进行了360度的否定,看不到生活的希望,不得不问自己,人活着、生命的存在到底是
为了什么。而现在,除非爸爸能够恢复意识苏醒过来,我将永远没有机会从爸爸这里得
到我想要的答案。

村里人说那时的爸爸是个浪子。爸爸还和结婚前一样种地放牛,但却总是吊儿郎当,做
任何事都打不起精神来,也很有些自暴自弃。远亲近临里没有谁舍得把自已的女儿嫁给
爸爸糟踏,直到有一天,爸爸遇见了妈妈。
\section{爸妈的爱情}
\label{sec-9-11}

我的妈妈出生在一个知识人家庭。与外公同父同母的共有弟兄四人,分别被命名为学尧
、学舜、学禹和学汤。外公是排行最小的。妈妈说外公个性闲淡,当了一辈子人民教师
,拉得一手凄婉哀凉凄切动人的二胡。外婆个儿小,我见过并清楚地记得外婆的一双小
脚还真是三寸金莲,比正常人的脚小太多了!外公外婆在中国亿万普通人家庭里算是幸
福的。妈妈共有姊妹六人,大姨、舅舅,妈妈和三个小姨。妈妈小时候也上过两年学后
来辍学在家。

解放后六七十年代,在家排行老三的妈妈被当作家里的劳动力代表与同村人一起被派送
到爸爸所在的村子做工,兴修水渠,具体的工作大概就是清理村子旁边那条溪水渠的淤
泥吧。而像妈妈这样的姑娘家大概不会真正去渠里劳动,她们在爸爸村子里固定一两户
人家家里住下来后负责为上渠做工的同村人准备伙食。

听妈妈后来说,她做工住到爸爸所在村后,在一次休息时打牌的牌桌上第一次见到了爸
爸。当时的爸爸英俊魁梧,妈妈年轻漂亮有着白皙的皮肤。我猜想过妈妈当时的表现,
是像我后来长大后体验过的脸飞红云,还是能把自己掩饰保护得很好,我想那个年岁的
妈妈可能会表现得更像前者。而爸爸这边已经是有过一段婚姻经历的人,大概能够一眼
看出妈妈当时的小心思吧。

妈妈说做完工前脚刚回到家,爸爸后脚就跟进家来向外公外婆提亲了。外公外婆大概还
想再观察一下爸爸的表现。偏巧爸爸村里有其它人听说爸爸去提亲后也赶紧赶到了外公
外婆家提亲。经历过一次失败婚姻和爷爷离世后深刻反省的爸爸很坚韧执着,他大概清
楚地知道自己想要的是什么,听说同村里有年轻人搞破坏便急急地跑到外公外婆家为他
自己的幸福又作了进一步的争取。妈妈没有告诉我爸爸到底讲了哪些什么话,但不难猜
到可能的内容吧。有了几次与爸爸的接触和基本了解,妈妈大概觉得爸爸还是很有些与
普通年轻人不一样吧,亦然决然地嫁给了爸爸。结婚时爸爸二十四,妈妈二十。

小时候客厅里略带装饰性的柜子、八仙桌和两只同样大小的衣箱看起来是一个系列,都
涂了大红油漆,衣箱上赫然写着“为人民服务”五个大字,妈妈说那是她的嫁妆。同为
嫁妆的还有一床新弹制的棉被、双人床单、被单和大红稠子田心(被单和田心用线装起
来后功能上相当于后来的被罩),以及外公特意找朋友为妈妈打制的一把不锈钢锅铲和
一把切菜片刀。

妈妈的到来给爸爸带来一颗定心丸,爸爸从此安心,也带给他们一段四十多年一生一世
的婚姻。
\section{我的妈妈}
\label{sec-9-12}

小的时候,我比较亲妈妈,与二姐坚决地站成了对立派。二姐亲爸爸,对奶奶也非常好
,而我那时总是对自己的奶奶不冷不热。因为奶奶还没有要伯伯和爸爸养老时,偶尔奶
奶家里杀鸡或是有什么好吃的,伯伯家有两个孩子,奶奶全叫上;我们家有四个孩子,
奶奶就只叫大姐和二姐两个,我眼巴巴地想着馋着奶奶家里的好吃的,却什么也吃不上。

妈妈一生顺遂,外公外婆也随了她的心把她嫁给了爸爸。妈妈对外公外婆和她的姊妹从
来都有着深厚的感情,闲来无聊时她也常常对我讲外公家簇这边的各种故事。在我的理
解回忆里,外公外婆的爱情是幸福的,外公的性格闲淡,他们对子女也没有过高或是很
严格的要求,以至于这个人民教师的六个孩子全部“垮掉”全部务农,除了四姨接了外
公的铁钣碗。按照当时的年龄,妈妈说应该是五姨接外公的职位,但她们姊妹感情好,
四姨个儿小像外婆,做不动农活,便同五姨商量,说每月给五姨五块钱,求五姨让着她
让她接外公的职位,五姨也是豪爽派的,区区一个职位跟本就没放在眼里,对四姨说,
“我才不想去当老师呢,我干得动农活,你想去你去!”

但外公外婆给予妈妈他们姊妹的爱是健全完整的。后来的事实证明,几个小家庭后来都
挺幸福的(三姨除外,以后再表)。妈妈给我讲三年大灾害大饥荒时,所有的人都吃那个
蚕豆藤尖煮水喝,她闻着那个味就着呕,宁愿饿着也不吃。外公心疼妈妈,拿家里一件
什么稀罕物件找朋友给妈妈换来几碗小米,煮粥给妈妈一个人喝。妈妈给我讲外公外婆
的故事讲她小时候的故事时,就像我现在回忆小时候同爸爸在一起的故事一样,总是感
觉很幸福。

后来妈妈说,她一辈子就一件事情吓着她了,那便是那年爸爸生病。
\section{早期记忆}
\label{sec-9-13}

有的人说性格是基因是DNA决定的,但受环境影响会改变。但我常常会很矛盾地想,一
个人对自己早期的记忆,是因为DNA决定了性格,以至于决定了记忆早期只有与性格相
关的经历最终存在了人的记忆里,还是,这些偶然存在记忆里的早期经历决定了后来的
性格?我会颠三倒四反复想这个,也是因为我的记忆与自己后来的性格非常相关。

那年冬天,爸爸生病了,病得很严重。医生说要给爸爸输血。妈妈便对外公说了。那时
候大家都还没有血型的概念,妈妈想来想去,姐姐们身体单薄,家里就我最胖傻乎乎胖
乎乎的(我小时候听妈妈说我生下来时脸搭肩上),便要外公把我接去给爸爸输血。长大
后我听妈妈说,那时四姨家两个月前刚给我添了个表妹,大人们心疼说这么小的孩子要
去抽血,四姨拿了两条表妹过满月喜酒时别人送的毛衫布料给舅母,舅母为我做了一件
新棉袄。可是长大后对捡姐姐们的旧衣服吊吊恨得牙痒痒、喜欢新衣服的我却没能记住
那件棉袄,记住了另外一个场景。

想来外公对于那时我这个小胖墩儿,是没有任何办法的,不会骑自行车也没有自行车,
抱不动、背不动、扛不动。外公牵来一头牛,我是如何同姐姐们告别跟着外公走,外公
离世多年他的音容笑貌我也都不记得了。记住的场景是,一位个头不高的长辈(外公) 
牵着牛绳走在堤埂前面两三米远,肥肥的牛背上驮着一小P孩(我),双手紧紧地抓住牛
鬃毛,远近的村庄变得朦胧遥远,身边渠里溪水清清,萧瑟的秋冬寒气打在脸上手上,
坐在牛背上的我懵懵咚咚,对爸爸所在的医院心生向往,又不知道等待我的将会是怎样
的命运。牛背上那种懵咚的感觉与十四年后遇见舅舅(题目所指舅舅)怀揣一个对陌生自
由国度的向往何其相似!

后来的事情我还是不记得。妈妈说等外公把我送到医院,医生说这么小的孩子怎么能够
抽血,外公又把我送了回来。那年没能给爸爸输血,后来我长大上大学后倒是有机会献
过几次血。后来同妈妈校对,妈妈说我们家是八四年秋天盖新房,八三年冬爸爸生病,
爸爸那会儿的病叫“出血热”。妈妈说那年她很怕,怕爸爸万一有个什么三长两短,她
一个人拖着四个孩子日子可要怎么过啊!

我早说过我不是聪明孩子,爸爸生病那会儿,我快四岁半。搬进新家之前,我们一家住
在新房后的老土屋里。我对自己最早的记忆也就停留在这老土屋,起始在三岁多四岁左
右。

另一个记忆犹新的片段发生在老土屋里。老土屋里我们一家六口住两间房,一间客厅兼
厨房,一间卧室被拦腰隔成两半放两张床。我们一家六口便挤在这两张床上。这是一个
大冬天的早上,爸妈姐姐们都起床了,听着房外寒冬怒号,我是无论如何也不要起床的
。妈妈快做好早饭了,最小的姐姐吆喝着命令我起床三次,再哄着我起床哄三次,我还
是不要起,一小人儿自个儿在床上打滚。那会儿还穿开档裤,可能偶尔还会尿床。不起
床的结果最终就变成了,妈妈拿着我的棉袄棉裤来到床头,哄我说她已经把我的衣服烤
得很暖和了,要帮我穿。我想着自己在床上滚了那么久,肚子也饿了,迟早还是要起床
的,便听了妈妈的话,坐在床上穿了棉袄,衣服果然暖和,便站起来,两手搭在妈妈双
肩上,提起腿儿来让妈妈帮我穿棉裤,烤得暖烘烘的棉袄甚至有点儿烫肉,但得到了妈
妈的宠爱,觉得心里也暖烘烘的。
\section{出院后的爸爸}
\label{sec-9-14}

从爸爸住病后,大姐长期守在病房外,大姐和爸爸的餐食也都一并是在住医院附近的四
姨家解决的。爸爸一日三餐都是喝清汤,鱼汤、煮得粘稠到化掉的米粥和果汁水。爸爸
在普通病房住了两三周后,姐姐们还是觉得住医院里爸爸大家吃东西都不方便。姐姐说
二姐二姐夫都是医生,医院里医生开的什么药姐姐也都知道,可以直接从那个医院进药
在家给爸爸打吊针。所以把爸爸弄回家与爸爸住在医院不会有本质的区别。我打电话回
去时,姐姐说他们在医院时给爸爸拔了氧气观察了几天没有问题后,已经在2011年11月
11日把爸爸接回家去了。

听妈妈说回到家里去后,一间房里并排摆了三张床,妈妈大姐和爸爸一人一张床,爸爸
躺中间。二姐批了两箱点滴在家里放着,她每隔一天晚上回来给爸爸打针上药。爸爸的
情况看来还比较稳定。爸爸生病后二姐夫很少到医院照顾爸爸,后来听妈妈说,姐夫找
公安局最后把爸爸这件车祸案子跑下来了,造事方需要赔偿爸爸目前花费的所有医药费
用。

爸爸出院大概一周左右,我又打电话回去时听妈妈说爸爸抽痰化痰已经抽不出来了。我
问妈妈在爸爸出院之前有没有给爸爸作一个全身系统性的检查,妈妈说没有。当着妈妈
,我没有当场发作,可是接下来打给二姐的电话里,我的脾气来了。我对二姐说大姐会
帮我还她两万块钱,我之前读书和出国时向她们家借下一万八千多块钱,我还她两万,
我借了这么多年,那多的一两千块钱就当是利息了。电话里我对二姐说,“大姐帮我把
这两万块钱还给你,我也就不欠你什么了!”电话里,我大有恩断义绝,与自己的亲姐
姐老死不相往来之势。

第二天我再打电话回去,家里所有人的予头都指向了我。我也对妈妈解释了,“我早就
对姐姐说,要给爸爸最好的治疗,要给爸爸足够的时间,要要求医院为爸爸作肺部检查
,为什么二姐还是学医的,爸爸出院之前她就不能引导大家要求医生为爸爸彻底检查一
下?”妈妈大姐姐夫都为二姐说话,妈妈要我打电话给二姐道歉。为了不让妈妈难受这
个电话后来我还是打了。二姐没忍心等我说出道歉的话,电话里她向我解释了她的看法
。二姐说,“你离得远,大家没人对你有什么要求,”但现在这种情况,每个人能尽的
都只是自己那份孝心,只要她们自己觉得对爸爸尽到孝,对得起爸爸了,她们也就不遗
憾了。

听到二姐解释的话,我觉得悲怆苍凉,原来姐姐是这么想的。可是仔细想想,我又有什
么资格权利要求别人要对爸爸如何,我自己不是从知道爸爸生病到现在,明明老板已经
批假了,我却因为身份签证的问题迟迟没有动身,到现在也还没有回去吗?我又为爸爸
做了些什么?

与二姐之间这点儿不愉快被我很快忘在脑后,可是每想起现在的爸爸,和姐姐的表现,
我都控制不住眼泪沽沽地往下流。这段时间,我开始真真切切地准备回家,向公司要
offer letter,准备返回签证时需要的材料。
\section{出院后的爸爸(2)}
\label{sec-9-15}

三姐在城里有个工作做,爸爸回家后,她也还是积极主动地与她的同事换班,这样一周
里她可以回家两天,帮忙换一下大姐。在我的再三央求下,三姐11月21日用QQ给我发了
四张爸爸的照片,一张爸爸健康时的,三张生病后的。

生病前的照片是爸爸站在家门口照的。爸爸说他后来买了水泥,把家门口水泥阶梯坏掉
的地方,和门口走路的小道全泥了起来,与家后面的水泥板公路相接。这张照片我看见
了家门口的变化,爸爸本来就瘦,又穿了稍大一点儿的衣服就更显得瘦骨伶仃,但照片
里的爸爸笑得很开心。另三张里,因为是头部近照,看见爸爸右上嘴唇上还有黑色血痂
,嘴唇略张露出雪白的牙齿。眼睛睁开一条缝,目光清澈,两内眼角和眼周都还淤青。
两天后我又收到了三姐QQ帮我传过来的爸爸最新照片更新。姐姐们帮爸爸刮了胡子,上
嘴唇的血痂已经没有了,嘴唇真张妈妈所说,很红有了颜色,照片里爸爸微张着嘴巴,
牙齿显得更白了。

爸爸出院后,我打电话到家里很方便,每次电话里我都会问妈妈爸爸现在每天吃什么,
吃多少,大小便是否正常,是否还在打针等,有时候还会建议妈妈像待婴儿一样喂养爸
爸,每四个小时给爸爸打(用注射器注射到胃管里)一顿流食。妈妈和姐姐坚持过一段时
间,后来是什么原因就改为之前的四餐了。我问妈妈是否还在给爸爸打点滴,妈妈说没
有了,原因是给爸爸打针,爸爸总是昏昏沉沉,精神没有不打针时好。我犹豫着,妈妈
强调说村里很多人都来看过爸爸了,大家都说爸爸这段时间比刚回家时气色好了很多,
眼周的淤青没有了,脸上也有了颜色。听妈妈这么说,我自己也亲眼看见了照片,心里
也还比较欣慰,心里想着我的爸爸一辈子都这么坚强,照着这种变化继续,只要爸爸挺
过这一关,爸爸还是有很大希望应该能够醒过来的。
\section{爸爸的故事(2)}
\label{sec-9-16}

爸爸经历了第一段失败的婚姻,经历了失去爷爷的痛苦,可能还有过一两年的空窗期,
再遇见妈妈后,便一把抓住了那份幸福,和妈妈一起安分守已地过日子。

而我的妈妈,虽然她得到的爱是健全完整的,但她也还是有遗憾的。我小的时候,妈妈
对我讲过,说她当时学习也很好。但一次上课不知道什么原因(妈妈讲过具体原因,我
自已忘记了)她捣乱了,她的代课老师用手指头关节敲了两下妈妈的脑袋,妈妈一气之
下便再也不要上学读书了。但后来当妈妈的好多同学都不用在农村劳动,妈妈一定是后
悔遗憾的。所以当我们姊妹四人还很小的时候,爸妈便确定好共同目标,趁孩子们还小
的时候盖好房,等我们姊妹长大后,他们会尽最大努力供我们读书,希望我们姊妹将来
不用在农村辛苦一辈子。

人还是要有归属感的。爸爸结了婚也安了心,接下来便学习了很多种不同的手艺。

在我刚记事的时候,爸爸是个木匠。我的记忆里我们家的工具特别多。(钓鱼的工具除
外,)家里有各式各样的刨子、打孔工具、钻子、钜子以及一些我叫不上名字的。我看
过爸爸用钻子打孔,我就照葫芦画瓢地悄悄用爸爸的钻子在一根小竹杆上打了一个洞,
问堂哥要了一节抽去内芯的电线绝缘皮为自己上学之前的两年放牛岁月自制了一个时髦
的皮鞭。爸爸的木匠手艺是从哪里学的我并不清楚,但那会儿爸爸的手艺大概也只是解
决日常需求,比如家里差个桌子就做一个,以及做些坐的椅子等等。由于当时的市场需
求量并不大,爸爸的这门手艺便停留在了自给自足自娱自乐的层面。后来听妈妈说,我
们家盖新房所用的门窗等全是爸爸和他的一个小徒弟自己做的。这应该算是爸爸这门手
艺的最大用处了。

后来,随着邓爷爷在中国的南海边画了一个圈,随着农村分田到户大家逐渐富裕起来,
爸爸不知从哪里学的师又干起了沏匠。或许爸爸有些领导才能吧,或许爸爸性格好做事
认真负责,爸爸带领同村和邻村的一些年青人组起了一个沏匠班子,专门负责为周围村
庄上有需求的人建房。到八四年秋我们家自己建房时,爸爸就带领他的一众徒弟直接帮
我们家建了,不用付工钱,只妈妈每日两餐(中饭和晚饭)准备好食物,不能让大家饿
着就可以了。

还记得家里盖房那会儿,我一小P孩问妈妈什么是大工什么是小工,妈妈说需要拿刀沏墙的是大工,和灰提灰的便是小工;我问妈妈我是什么工,妈妈说我是小搬砖工。受了妈妈的启发,我便真成了搬砖工,那些叔叔哥哥们的沏墙沏到拐角处谁需要半块砖的,只要吆喝我一声,我便两手各提块半砖头屁颠屁颠地送过去。我的小伙伴更好玩儿。他爸爸也帮我们家盖房子,晚上他便也常来我家吃饭。爸爸他们大人们晚上吃饭喜欢划拳绞酒,绞酒就是明明还可以喝的,可是嘴上却一定说,“不能喝了,真的是不能再喝了”,也就为劳累了一天的疲惫磨磨舌根子绞个趣儿。我的这个小朋友,听他爸爸说不能再喝了,害怕他爸爸喝醉,便一手将酒杯抢过来,仰起头来一杯白酒咕咚咕咚就灌进了肚里,第二天,听他爸爸说这个小伙伴喝醉了,把我们大家给乐得笑呵呵的。
\section{舅舅和姨姨们}
\label{sec-9-17}

我呆在美国的五年里,有两位亲人已经离开了,一位是我的舅舅(妈妈的亲哥哥),另一
位是我的叔叔。

对舅舅的记忆与外公外婆和姨姨们联系在一起。小时候过八月十五,爸妈有时候只带两
个大的姐姐去,我们小姊妹两个在家里看门。爸妈买给外公外婆的是两袋月饼,回家带
回来的往往有三四袋月饼,舅舅外公外婆都让着妈妈,说我们家小孩多,要爸妈多带些
回来给我们吃,于是在家看门的我和三姐往往一人可以得到一整袋月饼,外加可以尝剩
下不同口味的一两袋。

小时候我很喜欢到舅舅家走亲戚,一来是因为舅舅会为我们准备很多好吃的,再则逢年
过节到舅舅家会很热闹,众多的姊妹聚集到一起,有得好玩儿。大姨家有三个表哥,舅
舅家我有一表哥一表妹,我们家姐妹四个,三姨家一表姐一表妹,四姨家和五姨家各一
表妹。大姨家的表哥们比我们大得多些,我的大姐也往往不同我们一起玩儿,我们小姊
妹九个在二姐和表哥的率领下一块儿一起玩儿也算是热热闹闹、声势浩荡。到在外婆走
亲戚,妈妈他们大人们总喜欢聚在一起拉家常,我们小孩也喜欢聚在一起自个儿玩,玩
各种各样好玩儿的,捉迷藏、打牌、脑筋争转弯,以及各种游戏。有一次我们在外婆隔
壁房间的响声太大了,外婆批评我们说我们像“响水青蛙”一样,结果笨笨小小的我回
到家向妈妈告状说,“外婆说我们是响水娃娃!”害得姐姐们又笑话我。

在我的记忆里舅舅舅母生活得不错,但也一直受妈妈、四姨及五姨的接济。就像外公外
婆对子女的教育都没有严格的要求,他们对舅舅这唯一的儿子也是加倍宠爱的。

舅舅留在我的记忆里更多的是对我们晚辈的慈爱。我高考没考好,大学上了农校,即使
后来上了公费研究生,始终没能摆脱掉农校的底色。06年我准备出去了,妈妈招待了爸
妈两边的至亲,舅舅给了妈妈1000礼钱,舅舅舅母很为我开心欣慰,我很感激。然而到
舅舅59岁因脑溢血离去时,我始终没能抱答过舅舅什么。从妈妈那里知道舅舅离世的消
息后,我难受了很长一段时间。
\section{舅舅和姨姨们(2)}
\label{sec-9-18}

妈妈说,舅舅当年娶舅母是换亲。舅舅像外婆,个头小,可能也稍带点儿特殊年代(特
殊历史条件下)成份不好,不容易找到合适的对象。我的舅母身村高挑,年轻时可能条
件也算不错吧。所以舅舅当年最终能够娶到舅母的前提条件便成了,把三姨嫁给舅母的
弟弟作媳妇。我的三姨像外公像妈妈,也是很聪明的,三姨夫个头虽大,但没有丝毫灵
气,三姨自然心里很委屈。但这两段婚事为了舅舅考虑,在外公外婆作主下,最终也只
能委屈了三姨。

三姨可能自始自终都与姨夫没有任何感情。记得小时候逢年过节,当妈妈姊妹们都在外
婆家里拉家常,只有三姨一人常年拉着她的板车水果滩在镇上新街口卖水果(三姨也住
镇上),等到外公外婆的饭做好了,表姐叫了三姨,她才过来同大家一起吃饭,吃完饭
又急急地去卖她的水果了。三姨结婚二十多年后,姨夫离世。三姨找到了第二春,终于
有机会体验爱情婚姻的幸福。

大姨的婚姻很普通很平凡平淡也很幸福。听妈妈说大姨与大姨夫一辈子也不曾红脸的。
大姨五十多岁便离去,后来大姨夫还常来我们家与爸妈拉家常,大姨夫为人应该也还不
错。四姨当年接了外公的钱饭碗,也顺风顺水地嫁给四姨夫这个政府官员。生活原本幸
福,无奈四姨夫因肝癌离世,离开时表妹只有十一二岁。表妹在四姨、以及以舅舅妈妈
五姨为代表的亲人的呵护下长大。等表妹上了大学后,四姨也找到了第二春。

妈妈姊妹里最幸福的当属五姨。五姨当年将铁饭碗让给了四姨,似乎不能把握自己的命
运,但在我的终极理解里五姨是妈妈众多姊妹里最聪明的。有一次我到舅舅家做客,那
时五姨可能还没出嫁,我去舅舅邻居家找到五姨,撞见五姨和邻居女主人在吃蒸红薯,
见我来女主人便赶快也给我拿了半个。我见五姨丝丝屡屡地撕下红薯皮,慢幽幽地享受
那半个红薯,而大家可以猜测若看见我的吃法吃相就应该会相信这世上是有饿狼传说的
吧。

妈妈说年轻时五姨爱美,不吃东西把自己饿得很漂亮。五姨夫的妈妈与外婆是堂姊妹还
是表姊妹妈妈讲过我不大记得了,大家对彼些家的孩子也知根知底,五姨又出落得窈窕
水灵,便想亲上作亲,五姨夫的妈妈说让五姨从她家的三个儿子里挑一个五姨喜欢的。
五姨还没来得及见上他们三弟兄一面,弟兄三个里五姨夫是读书人,急急地就给五姨写
了情书,好像还不止一封。这段姻缘便也最终定了下来。五姨夫后来有正式的工作,五
姨一生相夫教子。而且五姨对表妹的教育相当周全,甚至不惜发扬“孟母三迁”的精神
,把表妹教育得非常好。

五姨的婚姻是妈妈众多姊妹里最幸福的,小表妹也是我们这一辈里最幸福、最有出息的
,找到投合相契的男朋友,随了本心学了艺术,在上海生活得很好。而我们这一辈的其
它姊妹,却酸甜苦辣,个中滋味,各自不同。
\section{我们众姊妹}
\label{sec-9-19}

大姨和大姨父都是踏踏实实过日子的人,他们的婚姻也算幸福美满,三个表哥后来只有
大表哥一人在家务农,二表哥成为了人民教师,三表哥在县城里建筑公司上班。那时候
我很小,我对大姨家的记忆也就显得久远,他们的生活质量生活模式到底是什么样的,
我知道得并不清楚。二表嫂也是一个老师,三表哥在他公司里找到婚姻的归宿。两表哥
结婚后在大姨和姨父的教导下为人处世也都是有理有节,逢年过节亲戚间礼尚往来,该
有的礼数从不曾落下什么。大表哥大概是三个表哥里受到教育最浅的,结婚后完全被表
嫂同化,大表嫂多少有些自私,对她娘家的亲戚很好,但对大姨及姨父就很差。妈妈好
在我面前唠叨,常常会说大表哥是娶了媳妇忘了娘。

十三姊妹里,接下来便是我的大姐排行第四。大姐的恋爱经历在前半部分已有涉及,不
再冗述。大姐的恋爱史在我看来更多的是相亲经历,虽然试图接触过二三十个备选对象
,大姐对于他们的了解应该也止限于蜻蜓点水、浅尝则止。姐夫是第一个真正让她动心
的人,也是她后来历经一生去了解、真正理解的人。

二姐属虎水瓶座,在我眼里姐姐略有一点儿事业狂,对感情博爱而浅淡。当年我上初中
时二姐问妈妈,她选男朋友是选个同行医生好,还是选那个当初的同学兼现在的老师比
较好,妈妈说让她自己的事情自己做主。后来姐姐跟了与她同行的医生姐夫,而我初中
时被同学们封为“冷美人”的铁饭碗语文代课老师后来嫁给了二姐这个作了民办教师的
同学,朴实真挚的爱情每天都在发生。我长大出国后,二姐对我半真半假地叹过,说她
一辈子从来没有轰轰烈烈忘记一切地去爱过谁,安慰我也好,发自肺腑地感慨也好,每
个人的生活都是我们自己选择的。当初姐姐以工作为重心,她也从工作中获得了她想要
的回报。

四姨家我的表妹十一二岁时姨父就离世了,四姨和表妹的生活显得孤苦伶仃。妈妈这边的亲人觉得表妹这么小的孩子没有了爸爸太可怜了,大家都非常宠爱表妹。每次表妹到我们家作客,妈妈都会尽最大努力做各种各样好吃的来款待这个表妹。当然,除了妈妈、舅舅、五姨和大姐、大姐夫等我们这边亲人的宠爱外,她爸爸那边表妹的叔叔、姑姑和堂哥们待表妹也极好。虽然表妹在幼小的年龄不得不面对四姨父离去的现实,但表妹的生活里基本上不缺少男性长辈和兄长的关爱,加上天生情商高,从小又受作为政府官员的姨父影响,表妹成长过程中情感上是完整的不缺失什么。长大后高挑、单皙伶俐的表妹寻了一普通长相的男孩作老公,她老公对她也非常好。
\section{我们众姊妹(2)}
\label{sec-9-20}

托尔斯泰说,幸福的家庭总是相似的,不幸的家庭却各有各的不同。具体说来,三姨家
的两个孩子,舅舅家的表哥和表妹,以及三姐与我,我们六人的不幸却轻重缓急、各自
不同。

表姐长得像三姨父,个头条子比较好。表姐也上了中专学医,后来镇上医院被私人买断
后,当年二姐、二姐夫和表姐同时失了业。塞翁失马,焉知祸福。失业后姐姐和姐夫开
了私人门诊发家致富,二姐后来也进修过好几次,现在姐姐和姐夫都有正式工作。表姐
失业后,正处谈恋爱谈婚论嫁的大好年华,嫁给了同一镇上的一位大卡货车司机。再后
来下岗失业的正多,也不多表姐一个,表姐便在镇上同当初五姨一样,过上相夫教子的
日子,幸福与否,我并不知晓。妈妈说最近几年来,三姨家的表姐表妹与我们家簇亲人
逐渐疏远,有淡出了我们亲人圈的趋势。听到妈妈这话,我很淡然。当一个人的心中没
有足够的温暖,我们作为他们小家庭之外的人又能对他们强求什么呢?

与表姐相比,三姨家的表妹就又多些波折了。我小时候有一次表妹到我们家作客,妈妈
意外发现表妹背上的脓包,慈心大发,连烧了两锅开水,帮表妹洗头洗澡,把表妹打扫
得干干净净。大概在那时的妈妈和我们姊妹眼里,表妹虽然父母双全,却没能得到足够
的关爱。长大后的表妹读书读到几年级,后来在做什么工作,我都不是很清楚。只是听
妈妈说后来表妹还没结婚已经怀孕了,还好,她男朋友还是负责任的人,最终奉子成婚
。当初上学不多的三姨大概也是世俗的。妈妈说,表妹怀孕后,三姨恨铁不成钢,对表
妹很冷漠。表妹感冒生病了三姨也对她不管不顾。后来表妹生下一个男孩,医生检查说
小孩有先天性呼吸系统还是某内脏发育不全,那个孩子最终离去了,两年后表妹产下一
女婴。想着她老公和公婆都是还算不错的人,现在的表妹应该幸福着吧。

前几年听妈妈说,舅舅家那个表妹也离婚了(舅舅家的表哥多年前就离婚了,以后再表
)。就像舅舅当年离开时我难受了很长一段时间,表妹离婚后,我又难受了一阵子。与
三姨家表姐和表妹可能没能得到的足够的家庭温暖相比,表哥表妹反而是在外公外婆以
及舅舅舅母的宠爱下长大的。表哥离婚了,我像是还能接受,可是表妹又是因为什么?
因为在婚姻里得到的爱远不及从小到大来自父母和爷爷奶奶的宠爱吗?因为对婚姻中物
质的不知足吗?我不知道,我也想不明白。妈妈说后来表妹嫁了一个年长的,年长表妹
多少,妈妈没说,我也没问。之前听过一个女孩的观点说,年轻的时候她已经轰轰烈烈
地爱过了,体会过了爱情的幸福,所以以后便不再奢求什么,遇到一个对自己还不错的
人便可以轻松嫁了。或许她是吧,可是表妹那么年轻,爱情也是可以这么轻易放弃的吗
?我只能说,多年没有联系,我并不真正了解表妹,无论如何,希望她能幸福吧。

三姨曾说表姐婆家很稀罕表姐,说表姐长得像电影明星一样。这话传到妈妈耳朵里,妈
妈很不平衡,对爸爸说,表姐就只是个头高些,真要说长相,表姐是比不上三姐的,妈
妈说三姐才是真正长得像明星一样星光闪耀。当时说这话的妈妈本能地认定三姐也会像
大姐一样寻得好人家,会过得很幸福。但说那话的妈妈不知道,也从来不曾意识到姐姐
与我的痛。
\section{我的表哥}
\label{sec-9-21}

我对自己大学四年的一个室友提过,我喜欢电影里面的小混混。他们很帅,他们有点儿
放浪不羁的小坏,但若你能有幸成为他们的朋友,你就会发现他们总有一面、总有一天
会感动你。从舅舅家的表哥身上,我咀嚼体会了穿透岁月的“暗恋”。

小时候,我们在舅舅家作客,表哥和二姐率领我们一众姊妹一直玩儿得非常开心,对他
们我很佩服、羡慕,我总是盼着自己快快长大。我的话从来不多,我只是一个默默的看
客。

小时候放暑假,我一般也会上县城里四姨家玩儿一个周左右。那年的二姐,中专快毕业
在县医院实习;那年的表哥初中已毕业,与他的铁哥门儿们四处晃荡。

二姐比表哥大两岁。同在四姨家作客的我,犹记得邻居阿姨问起四姨他们两个,四姨答
得云淡风轻,说一个是亲侄儿一个是亲侄女。但二姐与表哥间的“欢欣愉悦”映在我的
脑海里。

几个月后,二姐与同在医院工作的二哥谈起了恋爱。四姨父准备送表哥去参军。临行前
的表哥与二姐是如何道别的,我没见到但反复猜测过。当时脑袋笨笨不大开窍的我心里
若有所失却不知道是为什么,也不知道该做些什么。

小我四岁的四姨家的小表妹,为表哥出足了气。在我们当年的小脑袋瓜子里,二姐大概
是背叛了表哥,但二姐背叛表哥是因为二哥!于是乎,小表妹对二哥百般叼难,在医院
里见到二哥一次就吵一次,小表妹与二哥吵架,我也总是暗自开心!

第二年二姐结了婚。我参了军的表哥,还在军队训练,还要一两年才能回来。我有时候
也会去想,表哥在军队日子过得怎么样,每天会有怎样的心思情绪,表哥还会对什么事
魂牵梦扰吗?我没有答案无从知晓。我唯一知道的便是,四姨有舅舅存折的账户信息。
在军队参军的表哥只要给四姨打电话,四姨的汇款便随叫随到。据四姨说,参军的两三
年里表哥所有的袜子,甭管质量好坏,对表哥来说,它们从来都是一次性的。

参军复原回家后的表哥成功升华为我心目中准男朋友的完美形象,表哥成为远近镇上县
里不折不扣的混混,听说表哥一伙儿打架斗欧干群仗,几乎是无所不为。至于表哥有没
有彻底升级为混混头目,我没那么高的欣赏水平,我也不关心那个。至于那时的表哥有
没有几个漂亮小妞儿,偶那时也很单纯,从来没有想到过这些,所以也就无从知晓了。
\section{我的表哥(2)}
\label{sec-9-22}

后来一天在我们镇上,一群像表哥一样年轻体壮的混混狠狠地收拾了表哥,一把匕首捅
在了表哥的胸膛,我心目中先前有着伟岸形象的表哥这次也终于像电影里演的一样倒在
了血泊中。

二姐夫(先前的二哥)的亲妹妹在镇上开餐馆,她打了急救电话,一个箭步冲上去将表哥
送到医院去抢救。

医生说表哥需要输血,表哥的血型是A,大姐夫一马当先,400CC的血液及时进入了表哥
的身体。每一个遭受摧残的生命都有亲人的关爱和盼望,表哥也不例外。除了姐夫的热
血、老妈的鼻涕眼泪,四姨舅舅舅母的守候与等待,还有我说不出口的牵挂。

表哥伤口痊愈后,安分了一段时间,给我娶了个与表哥相像的花花小姐般的嫂子,舅舅
舅母也扭不过他,允许他们结婚,把镇上家里的楼房卖了,在县城给他们买了套单元房
。几个月后家里添了个小丁丁。

结婚后的表哥大概还是野着的吧。一次不知道什么原因表哥又出去打群架了。这次他们
人多势众,欺负别人过了头,有个人死掉了。自此,表哥曾经的那群“铁”哥们作鸟兽
散。我们亲戚里也没什么有权有势的人,表哥就成了这次群架的唯一“负责人”,在牢
里呆了好几年。

这期间,舅舅突发脑溢血急急地离去了。临走前表哥赶回来与舅舅勉强见了一面。待表
哥真正出来后,表嫂已将当年舅舅为他们买的婚房卖掉,告诉表哥说房子的钱早就花光
了。表嫂与表哥离了婚,将小丁丁甩给了表哥。

有时候我也会想,一个人的顽劣究竟到什么时候才会终止,若没有高考那年的痛定思痛
,我后来的日子又会是什么样?之前我一点儿也不喜欢那个表嫂,但当表嫂将舅舅的房
钱浪掉,与表哥离婚并拍拍屁股留下小丁丁后,谢天谢地,我的浪子表哥,终于反省悔
悟了,或许是因为他爸爸的离去,或许是因为他爸妈辛劳一生的房钱,也或许是因为他
自己那几年的青春岁月。

大家都住县城里,表哥与大姐、姐夫和四姨就来往得比较密切。听妈妈说,表哥有了新
女朋友,女方家里好像不大乐意;后来表哥结了婚,再后来表哥新添了一个女儿。听妈
妈说,表哥让女儿随她妈妈姓,表嫂的爸妈也终于是慢慢转变态度,越来越稀罕俺表哥。

当表哥的故事浸透岁月,当我将对表哥的这段心事纪录下来,我也终于明白,表哥早已
不再是我那儿时心心念念掩藏心底的暗恋,不知不觉中表哥早已成为我的亲人。
\section{大舅一家人}
\label{sec-9-23}

爸爸生病期间,我也常打电话回去。有一次打回去,妈妈说那天傍晚大舅(题目所指舅
舅的亲哥哥)母打电话过来,妈妈同大舅母很聊了一段时间。妈妈说她打电话的时候爸
爸是醒着的,待电话打完,她去看爸爸,爸爸鼻子喉咙抽啊抽的、像想哭一样伤感了大
半天。听到这话的我,也随了爸爸伤感了好久。妈妈讲的这件小事让我更加坚定信念,
一生中历经磨难的爸爸现在虽然不能说话,但他心里是明白的,以爸爸的坚强乐观,我
相信爸爸一定会挺过这一关的!

爸爸的伤感起源于大舅母的电话。爸爸住院后,多年不曾走动的大舅领着大表哥和当时
回国探亲的大表姐去医院探望过爸爸一次。若爸爸真还是清醒的,爸爸应当还会记得我
10年12月到舅舅家一趟后的涟漪。当时家里人最开始是有些不乐意的,无奈我的立场太
坚定,家里人也拿我没什么办法。我清楚地记得我当时说服了所有的人,包括我的爸爸
。电话里,我对爸爸说,我们可以把目光放得长远些,一时的贫困并不意味着永远贫穷
。找到一个真正相知相爱的人能一辈子过得幸福才是最重要的。若爸爸心里真明白,若
爸爸还能记得早前我特意对他说过的话,爸爸应该就能明了大舅表哥和表姐去探病意味
着什么,我想爸爸对我应该是会放心的。

爸爸生病后,表姐说大舅母也病重,很快大表姐就回国探亲去了。临走时我对表姐讲过
爸爸也出意外在住院。大表姐回国后待舅母出院把舅母安顿好,便随了大舅和大表哥一
起到医院探望爸爸了。之后大表姐同舅母一起还请了家簇里大舅这边之前有些疏远的远
亲到餐馆吃了餐中饭。我的表哥(妈妈的亲侄儿,我舅舅家的表哥)也是见过世面的人,
那天晚上便在同一餐馆回请了表姐舅母、四姨等家簇亲人,在大表姐的带动下我们远亲
间的联系也稍微活络起来。

既然这部分具体写爱情、写亲情,写到这里,我就得对大舅一家再作些详细介绍了。前
面提到,外公共有弟兄四人,外公最小;大舅的爸爸排行第二,后文我就称呼为二外公
了。听妈妈说,三外公走得早,大概还没结婚就离去了,具体原因我不清楚。那年陪我
驾车前往加州的路上,舅舅就对我讲过,他的妈妈是大家闺秀。大舅共有姊妹三人,国
内的大舅、现在美国的二舅和已经离世的小姨。二舅极善察言观色,以为我喜欢佩服聪
明人,便总在我面前秀聪明,说自己的孩子是小天材之外,便也说小姨自小生得聪明秀
丽,只是她小时候生病延误了就医了的最佳时间,后来长大后耳朵听不见,便也不太会
说话。我对小姨还有些记忆。那时大舅村子里相临住着的两家人,我有一个哑巴舅母(
是大外公的一个儿媳妇),也有一个哑巴小姨,各有两把编得很整齐的村姑长辫子。那
时我太小,走亲戚之前妈妈会教我说到了称呼谁谁什么什么,可是真正到了之后,她们
两个我还是常常弄混,不过大人们也没谁同我计较什么。小姨一生可能都没结过婚。听
妈妈说小姨很勤劳,后来一次在镇上的铁路上捡煤(货车运煤从镇上经过,可能偶尔会
掉些煤块下来吧)没能注意到火车来了,便早早地离去了。小姨走时有多大,我不清楚
,大概大舅二舅应该会知道吧。
\section{我的叔叔}
\label{sec-9-24}

那年的八月十五,姐姐们在叔叔家过节。节后打给妈妈的电话里我向妈妈打听姐姐们为
叔叔出的礼钱。妈妈说,姐姐们每个小家庭都是只给爸妈一百元钱,但是这次给叔叔家
比较多,二姐给了五百,大姐和三姐也每家出了两三百。电话里我没对妈妈瞎说什么,
但放下电话,我自个儿暗自神伤了一阵儿,觉得自己压力大没能给爸妈寄点儿钱,姐姐
们竟也都这么小气,连累得爸妈心里可能也会受委屈。后来过了一段时间才知道,原来
叔叔病了,病得很严重,到了癌症晚期,时日不多了。

叔叔与三妈(叔叔的爱人)是一个平和的家庭,有一双可爱的女儿。我小的时候有过一段
时间对自己怎么来到这个世界上很好奇,每次问妈妈,妈妈都说我是爸爸上山砍柴捡回
来的娃娃,我自然不信。不过问多了没有答案也就不问了。小时候放牛时听村里人说叔
叔家的堂妹(大妹妹)是领养来的孩子,我有些不信,便回去问妈妈。妈妈这次没有骗我
。妈妈说,堂妹是抱的别人家的娃娃。妈妈说人的命各有不同,有的人命里无孩,便一
辈子也没个自已的孩子,就只能领养别人家的了;有的人引子伢子,领养一个孩子后便
可以有自己的孩子。妈妈说叔叔三妈结婚多年没个孩子,便从市医院抱了一个别人不要
的女娃回来养,引了这个女娃后他们便有了自己的小孩。并且妈妈告诉我说,堂妹上面
有两个亲姐姐,可是她的亲生父母想要个男孩,所以生下她是女孩后便把她送给了别人
,希望将来再怀一胎男孩。幼小的我觉得 “引子伢子”很神奇,但堂妹的生事妈妈说
得这么真切也由不得我不信。

妈妈说三妈领养堂妹之前,叔叔把二姐要去跟了他们,算是他们的孩子,爸妈也都答应
了。二姐跟了叔叔三妈一段时间,又自个儿跑回家来了,对妈妈说三妈很小气,连吃饭
时要点儿猪油和饭吃三妈都不肯。猪油饭我小时候也吃过,很香很好吃,尤其是就着淹
辣椒,是儿时岁月里吃得很享受的牙祭。妈妈觉得既然已经答应把孩子给了他们,就应
该断了姐姐的念想,于是便带了罐猪油,妈妈领着二姐又把她送了回去。可是送回去不
久,叔叔又把二姐送回来了,原因是三妈觉得二姐太聪明,怕这个孩子养大了还是咱们
家的,成不了他们家的人!三妈不敢再要二姐了,二姐便又成了我们家里的人,叔叔和
三妈后来才又领养了表妹。

后来的叔叔家便很幸福,一团和气。在我的记忆里三妈可能难免会偏爱自己的女儿多一
些,但叔叔待两个妹妹从来都是一碗水端平。两个表妹也都稍微读了点儿书后来辍学在
家。长大点儿后,大妹到市里找了个工作在做,小妹一直在家呆着,后来别人介绍了个
条件不错的人家就嫁了。倒是大妹一些年来寻寻觅觅,到小妹结了婚她都还没能安定下
来。
\section{爸爸}
\label{sec-9-25}

三姐上次给我QQ传了四张照片后的一个星期左右,姐姐被我催急了,又给我传了一组爸
爸的照片。不知姐姐寄给我的第一组照片是什么时候拍的,两组照片我收到的时间相差
只有一个星期,但爸爸与上周相却比,完全瘦变了样,看得我心里非常难受。

我们姐妹四人里,就我上了大学,二姐没得到读高中的机会,读了中专,大姐只读到初
中毕业,大姐添小侄儿的那年三姐初中还没毕业就不再读了,去帮姐姐看孩子了。姐妹
四人里除了大姐有过更多的相亲经历,三个姐姐基本上都只谈一场恋爱便顺利结婚;姐
妹四人里,大姐二姐有机会在我们市里呆过,三姐三哥在广州打过一年左右的工。我的
成长过程历经磨难,四姐妹里大概数我最能深刻体会爸妈一生的辛劳和爸爸的坚强乐观
吧。于是乎,爸爸危难之际,我把自己变成了这场意外后爸爸能否康复的整个家庭的精
神支柱,也同时把自己变成了一个小监工。

在妈妈的强迫下我隔一天才能打个电话回去。打电话回去的我,会仔细询问爸爸的饮食
起居,还不时加些自己的建议。妈妈说爸爸小便很正常,她每天早上起来都会倒掉满满
一个尿袋。但爸爸无法正常每天排大便,要两三天才会有一次。我告诉妈妈说很多水果
都能帮助清肠利尿,若妈妈能每天固定钟点给爸爸饮水果浆汁,帮助爸爸养成好的钦食
和排泄生物钟,爸爸的身体和精神健康都会好很多。

照片里爸爸瘦得我很难受,虽然还没能最后决定动身,但作为一种鞭策手段,我对妈妈
姐姐说我打算近期回去照顾爸爸。我这么说是因为我觉得妈妈姐姐可能还是希望我能在
这边生活。我说要自己回去亲自照顾爸爸了,可能能迫使他们把爸爸照顾得更尽心些。

后来下次的电话里,妈妈说要告诉我一个好消息,妈妈说爸爸能张开嘴了。妈妈说以前
每天早晚给爸爸洗牙,爸爸嘴巴都不肯张开,妈妈就只能用干净纱布帮爸爸擦擦外边牙
周。那天早上三姐帮爸爸洗口,顺口说“爹,你把嘴张开,我给你洗牙好吧”,爸爸就
真张开了嘴巴。三姐把妈妈大姐都叫进去看,爸爸果然能张嘴的。爸爸能张开嘴巴这个
事实对我来说是个阶段性、决定性胜利,我们接下来还可以做很多事情。我们可以帮助
爸爸练习咀嚼食物,可以多陪爸爸说话,让他彻底清醒过来。我把自己能想到的所有思
路都一一交待给妈妈,让妈妈和姐姐试着观察执行。

妈妈一生都是很爱爸爸的,我从每次打给妈妈的电话里也能感觉到妈妈照顾爸爸的用心
和投入。后来妈妈说爸爸基本上能每天正常排泄了,我暂时放下心来。我从中介公司拿
到了offer letter,把签证表打印填好,以备急用。接下来的日子里,我追踪着爸爸的
进展,也追着打电话给三姐,要她帮我传爸爸的照片。
\section{MS Computer Science}
\label{sec-9-26}

爸爸这次病得这么严重,我都迟迟没能动身,主要还是身份问题,怕OPT回国后签证签
不回来,我自己似乎也心理上还没能完全准备好彻底回国。那时OPT也差不多用到剩最
后一年,那我接下来的打算是什么呢?

我知道我会尽自己最大努力去找工作,不去找或者找得不努力不尽心都太对不起自己了
,高考那步弯路迈得太痛苦,我不想再为自己留遗憾。可是万一我还是找不到工作我又
要怎么办?我需不需要申请一个Ph.D或者是Master学位接着读书呢? 

我知道小的时候我们姐妹学习上的问题、二姐报考中专志愿,妈妈都会打发我们上大舅
家去,问大舅大舅母我们该怎么做。妈妈觉得她和爸爸都没读过多少书,对这些学习上
的事情他们不懂,所以宁愿我们去请教他们眼中的专业人士。是惯性吗,我都这么大了
,因为舅舅一句不称心的话就再也不给他打电话的我,居然还是想问问舅舅和表哥的意
见。于是乎,一封邮件洋洋洒洒、点击一下“Send”便轻飘飘地发出去了。

发送时间:2011-12-14 11:42;收件人:舅舅;抄送:表哥;

Dear Uncle, Dear Cousin, 

I will try my best to look for new jobs, but at the same time, I am thinking
I will take another master degree from XXXX computer science department. 
Another choice would be XXXX EMBA in XXXX, but there are too much risk for 
EMBA and is expensive \$37000 tuition totally, I would rather think about MS 
computer science. It is almost deadline now, and I appreciate a lot if you 
can offer some suggestions. 

这封邮件发出去后表哥没有理我,舅舅第二天早上回了四个字“We have no 
suggestions”。舅舅这是在跟赌气吗?不过现在是我自己处在危难当头,没时间想那
么多了,先解决自己的事情要紧。

可舅舅的四个字还是让我忍不住得问自己:这真的是我想要的吗?不是,一定不是。当
年因为想要毕业想要工作连延期一个学期我都不想再读了,一个三十多岁的人因为没有
解决恋爱婚姻问题飘浮不定,连读书工作都不能静下心来,他还真会有心要回去读书吗
?读书只是维持合法留美身份的一种手段方式,读书绝不是这时我内心想要的,它不应
该成为我的选择。

我想,尽自己100\%的努力找工作,万一找不到工作即使表哥的家庭到时还是不能接纳我
,我正确的选择也应该是回国,而不是为了留在美国而去再读什么书。人终究还是要面
对自己的,我不能自欺欺人。最终,我没有申请任何学校,准备好好找工作。
\section{大表姐}
\label{sec-9-27}

过圣诞前,我总是心理上觉得别人快过年了,公司不会再管什么招人不招人的事,于是
一时半会儿都还没有急着投简历。既然要破釜沉舟找工作了,给自己充足电也是必须的
。受父母影响,我对得到专业、专家、官方认可、专家认证一类一向比较认可。考完上
个专业认证,接着在看同专业另一个方向的书,希望找工作时能多一些机会。碰巧在一
个专业认坛里看见一同行考了这个认证,便同他联系了一下,希望他能帮忙分享一些考
试经验。那个认证也不难,新年前轻松过了。接下来不打算再考更多的认证,也是时候
该好好投简历找工作了。

一段时间以来,因为考试一直很忙,待这一季第二个认证考过,我也该去找找我那好久
不见的表姐了。因为住得近,来往会稍微频繁一些,去时我还是一如既往的两袋豆制类
食品,也就是一二十块钱的东西,从10年春天工作后既使我到表姐家,别人不留,我也
从来不在她家吃饭,所以我只需要保证不空手去礼节上也就可以了。

那天也是周末上午十点多钟到的表姐家,放下东西,同08年暑假、之后两三次到表姐家
、没工作前一样,先在大表姐的带领下收拾几个卧室的脏衣服、换床单整理床铺,用吸
尘器吸地,把家里收拾得差不多,外面又把我的车、表姐们的两辆车清洗干净。干完这
些休息一会儿,大表姐说出去吃饭。小表姐说要帮我们找什么地方的一个coupon,我心
里有点儿开始犯愁了。

08年夏天要来加州之前我给家里打过电话,爸爸提醒交待过我,到别人家里要学会、注
意察言观色,要尽量不要给、少给别人添麻烦。08年去时大舅舅母也在表姐家探亲(他
们一次来可以住三年),小表姐一家四口、大表姐、外加舅舅舅母我们四个,九个人的
家,我当然是尽自己最大努力陪舅母买小菜、帮舅母做饭打下手、饭后自己清洗碗筷收
拾厨房,这是我应该做的。09年舅舅母回国后,再到表姐家,大表姐很尊重小表姐一家
人的私人生活空间,所以基本我只随大表姐一起活动。工作前偶尔到大表姐家去(最多
一周去一次,最长的一次是两三个月去了一次),大表姐也只是鸡内脏鸡杂糁蔬菜炒两
个菜吃,我还要被大表姐劝减肥少吃点儿。08年夏天走之前为表姐付了那个\$130-\$150
的买菜账单,舅舅把我带到他们家,我的确打扰了别人的生活,但毕竟有这层远亲关系
在,这个账单我出,是感谢他们的帮忙,也希望以后再到表姐家时他们会帮忙吧。但我
真的不认为我吃到或是花到他们这么多钱。在我心里这更多的是为将来还会打扰到他们
的一笔预支付、预开支。而10年工作后的一二十块钱的豆制品随手小礼我也买过好几次
。
\section{大表姐(2)}
\label{sec-9-28}

大表姐在北边一个小城买了房子后,一个周六带我去看过一次。那天是打算在那边过夜
,大表姐的本意大概是我们去那里打扫卫生吧。对于为表姐们付出一点儿自己的劳力,
我从来都不介意,毕竟自己挣钱太少,能从体力、劳力上帮他们点儿什么,我自己心里
也会开心。但那天去后房间还是太乱,无从下手,当天傍晚就回来了。大表姐提了一下
说晚上要不要去吃饭,我拒绝了。我们的积蓄都不多,大表姐即使今天请我吃饭为我花
钱了,早晚我也还得花钱把表姐这个礼还回来。我对表姐说,看看她的新家我很开心,
我们又没做什么体力活也不累,我回家自己简单做点儿吃的就可以了,便回自己家了。

仔细想想大表姐还帮我过什么,带我开了两个银行账户。小表姐说要帮找coupon时我就
猜到小表姐大概是想让我请大表姐,我原以为我不空手到他们家就可以了,这餐饭在我
的意料之外。可是我又有什么?我不给自己的父母寄钱,我省吃俭用只为了今天这么铺
张浪费吗?同之前一次一样,我不打算浪费我们的钱。表姐坚持要出去吃饭后,我对表
姐说,我们不要去那个XXX贵的,我们就去XX超市旁边那个小店(二十块钱三个菜随便点
)简单吃点儿就可以了。表姐依了我的主意,后来剩一个菜打了个包,后来还是表姐结
了账。

后来接下来的事情就更离谱了。确实是在超市附近吃饭,我们两人再次一起上了次超市
,表姐只买了捆青菜,一袋碎肉提在手里我们就出来了。后来东西放车里,表姐说我们
走去那个XXX店看看有没有LED电灯泡。周末的大街上人来人往,我们站在街边等车流过
去,螃蟹般横穿马路。等车时落寞的我突然开窃意识到,这是一场戏,我今天最重要的
使命应该是陪着大表姐演戏,把我之前文章里写她的坏话一一用行动为她纠正回来!

若这是戏,或许我真该演这戏。人活在这个世界上有多少的无奈,主动的、被动的,人
又这么渺小,总是活在自己的悲痛里。这种痛苦,我自己又不是没有体会过。在美国这
样的一个讲究credit的国度里,表姐也需要工作,也需要同事、朋友、邻居的认可,我
之前说过的话,无疑对她伤害太大了。之前第一部分关于她的坏话是我写出来的,若表
姐认为这样能让她心理、工作环境、朋友关系变得好一点儿,那我也愿意成全她。我也
是一个吃货,之前同表姐礼尚往来吃过最多的也只是越南粉。后来的两三次接触里,表
姐总把我往人多的地方带, 越南粉也不吃了,去Mall里, 去Mall里吃人山人海的盒饭吃
法。表姐通过自己的努力应该让她的形象回复了不少吧。后来我们的联系终于还是因为
一些其它的原因淡下来。而我自己,慢慢定心,后来同一个校友出去吃过几餐后,我自
己也已经好几个月不再出去吃东西了。
\section{大舅一家人(2)}
\label{sec-9-29}

前面写到大表姐,还是禁不住得写表姐的家人,二外婆和舅舅舅母以及表哥表姐们。

我对大的历史事件不够敏感,听妈妈说当年共产党要斗地主时,作为当时镇上的地主,
大外公呆在原地没动,妈妈说大外公后来是在牢里离去的;二外公外婆当时也已有三个
孩子,二外公大概聪明些,领着当时只有十岁左右的二舅跑了。剩下二外婆和大舅小姨
母子女三人相依为命。二外婆同舅舅小姨的日子是怎么过来的,妈妈讲得少,我思量着
作为地主家庭,即使二外公这个当家人离开了,二外婆大概也还会从她的父母亲人那里
受些接济吧。妈妈说二外公走时二外婆只有29岁。我本能地以为二外公与二外婆接下来
的婚姻会成为一个凄美的爱情故事,便问妈妈“二外公离开了,二外婆也没再找个人家
?”妈妈说,那时大户人家娶媳妇一般也只娶好人家的女儿,那些女孩子十几岁就嫁过
去当媳妇了。二外婆那时29岁三个孩子,其实想嫁也不那么容易了。我想着那时二外婆
心里可能也抱着希望,等那阵儿风过去,二外公和舅舅还会回来的吧。

生在地主家庭,生在那个讲究成分的年代,大舅的日子大概也并不好过。大舅比妈妈要
大十五六岁,大舅的早年经历妈妈大概也并不清楚。妈妈只给我讲过大舅的爱情故事。
妈妈说,那时大舅成分不好,不容易找对象,到了结婚年龄,还是连个女朋友也没有。
后来二外婆作主,要把二外婆娘家的一个亲侄女许给大舅。二外婆姓姚,这里暂且称这
亲侄女“姚表妹”吧。大舅年龄大些,大概已经工作,作了一名教师;姚表妹尚小,还
在读书。妈妈说大舅用自己的工资供那表妹读书,但读完书的姚表妹最终还是不愿意同
大舅结婚。后来大舅没办法,年龄也大了,随了大舅母。

听妈妈说,大舅母当时是离婚了,带了一个当时已有五六岁的大表哥。很多年前我问过
妈妈大舅母为什么会离婚,妈妈说好像是因为她老公坐牢了。而为什么坐牢,别人不说
,我们也就不知道。大舅母是极其聪明的人,不是说做学问很有科研精神的聪明,而是
一种待人处世的圆滑,或者说很会忽悠,会见人说人话见鬼说鬼话吧。大舅的爱情、婚
姻或许来得太不容易吧,同舅母结婚后,舅舅这边我们这些穷亲戚就给甩开,倒是舅母
那边的亲戚后来表姐们都还是走着的。
\section{大舅一家人(3)}
\label{sec-9-30}

一二十年后,表哥表姐们都长大。可能那时家里条件还不算太好吧,表哥娶了个大表嫂
舅母一直不太满意,妈妈猜想说舅母可能觉得表嫂不够圆滑、说话做事稍显木纳。大表
哥也还是很有才的,后来表哥小有成绩,在县里开了个建筑公司,前面提到我大姨(妈
妈的亲姐姐)家的三表哥就是在这个表哥公司里上班。家里条件改善后,大表哥离了婚
,重新娶了个年轻漂亮的。大表哥是我们周围亲人圈里第一个离婚的人,妈妈这边的亲
人都一致认为是大舅母一手操纵的结果。而妈妈姨姨们情感上是偏向同情前表嫂的,她
们会念及表嫂的好,心善,周济过我们穷人(后来被大舅母制止)。多年后大表姐说表哥
后来找的那个表嫂也没让表哥少费事。听表姐说那话,我心里感慨,嘴上却什么也说不
上来。

七十年代末八十年代初的时候,已在美国安家定居的二舅回到他童年时的故乡,找到了
二外婆和大舅。作为知识分子家庭(舅舅舅母都是教师),八十年代的出国热他们也不
会不知道,大概是舅母的主意,要二舅把小表姐带出来读书。大表姐当时已经上了大专
,小表姐在读高一还是高二,舅母要舅舅带小表姐出来应该是因为小表姐像舅舅学习上
聪明多了,比大表姐过去更容易扎下根来。小表姐来到美国后上了几年高中,上了大学
,读了硕士,博士读了一半放弃学业直接工作。六四期间也上街游行顺利拿了血卡。她
的恋爱史我不清楚,表姐夫是当时的房地产商炒地皮的,结婚时小表姐表姐夫各有一套
房子。08年夏天到表姐家后大舅母希望我体恤表姐的辛苦说,表姐已经把房贷还完了,
但是为了让表姐夫做生意,小表姐帮表姐夫贷了一百万美金,但被表姐夫炒砸了,我有
些不大相信。在舅舅舅母眼里,他们始终觉得自己的女儿是最好的,认为表姐夫工作、
心智上都配不上表姐,大家包括二舅都很为表姐感到遗憾。或许是有点吧,但我觉得情
感上俩人儿还是很不错,从来没争过嘴吵过架,养了两个儿子。

与小表姐刻苦读书、单薄得无从知晓的恋爱史相比,大表姐就显得懂得享受生活得多了
。大表姐大专毕业后,在市里工作。在那个提倡晚婚晚育、优生优育的年代,表姐享受
了多年的恋爱感觉,后来却因为男朋友出车祸眼睛瞎了痛苦分手。当时表姐已经老大不
小了,同舅母一样个头又高,想找到个合适的对象还真是不容易。偏巧那事前后有个有
自信、有勇气的喜欢小表姐的小表姐国内high school sweetheart写信到舅舅家想追求
在美国发展的小表姐,被大表姐拦腰横截,变成了大表姐夫,成就了一对姐弟恋、高矮
配。两人都上过大学,很和气,心灵上可能也还比较相通吧。

多年后10年我到加州后,表姐劝我找个男朋友。我很多方面表现出的执扭、顽冥不化可
能让大表姐觉得孺子不可教吧,感慨地对我说,我们从小长到大受太多苦,已经让我们
性格扭曲了,已经不知道该如何追求幸福了。这话对我来说有点儿重,心底掠过一丝不
快,本能地反问大表姐,她受过什么苦,扭曲成了什么样。表姐很坦诚地说,他们小时
候舅母同二外婆天天吵架,吵得家里鸡飞狗跳的,烦都烦死了。表姐说了这话,我也释
然了。表哥表姐们的婚姻,舅舅舅母心里感到尤其遗憾的小表姐的婚姻,说到底还是舅
舅舅母亲自导演的,虽然当时的当事人或许并没能意识到这一点。
\section{父母的婚姻}
\label{sec-9-31}

当我讲述完几乎所有其它亲人的故事,我终于还是不得不转向父母的婚姻,交待自己童
年时的故事。去年11月当我第一次用<成长的故事>为自己的传记故事命名的时候,我还
不曾想到有一天,我需要去回忆自己童年的痛苦经历,去揭开那早已结痂的伤疤,想着
揭开后鲜血淋漓血肉模糊的画面我总显得迟疑。现在早已度过那一关的我,再来讲述那
段经历、把我体验总结的教训分享给大家,我是可以做到的。但讲述之前,我需要先申
明一下,我深爱我的父母,希望读者能用一颗体贴、宽容的心辩证地看待那个年代大背
景下的那些人、那些事。

爸妈从相识到结婚,前面已经交待过了,婚后的爸妈应该也是挺幸福的,第二年大姐就
落地了,三年后二姐也出生了。爸妈一辈子都勤勤恳恳,是踏踏实实过日子的庄稼人。
用我长大后成熟的心智来推算,爸妈婚后的前八年十年可能都还挺幸福的。接下来的事
情,同大家猜的一样,我也没能逃脱大表姐总结出的“恶咒”,与表姐们的经历相似,
我也是在一个鸡飞狗跳的记忆里长大的。

我小时候(上小学六七年里的事)的记忆里充斥了太多爸妈两人吵架的场景。他们两人只
要吵起来,谁也不让着谁。事情往往只是由鸡毛蒜皮的小事开始,由于双方都不作任何
让步,激怒、愤愤不平的情绪下,双方都捡狠话往下策往狠里说,大家说出来的话也就
只会是越来越难听,越来越置对方于死地不管不顾。吵架时爸妈的声音都很大,动辄就
要砸穿西。两人吵到不可开交,便说要把房子拆了,要把电视砸了,日子也不要再过了
,妈妈说要大姐和我,爸爸的话紧跟着就会说二姐三姐跟着他过就跟着他过,谁怕谁! 
吵到那个份上,他俩儿倒是精诚合作、高度统一! 可再吵也会有吵累的时候,吵完架后
的爸爸情绪总是相对稳定,会做简单的饭菜给姐姐和我吃,妈妈却往往是气得眼泪鼻涕
横飞、会独自躺在床上哭很久。吵完后的爸爸倒是也知道他气着妈妈了,往往会打发三
姐或是我捧着加进了菜的饭碗和筷子到妈妈床头,要我们劝妈妈吃点儿东西。而气头上
的妈妈一般又是不会吃的,我和姐姐的眼泪就会紧跟着又掉下来。

爸妈吵架的时候,三姐与我,我们都不知道该做什么,该怎么办,我一般会随着姐姐一
起在被窝里躲起来。我与姐姐睡一张床,盖一床被子长大(差不多到姐姐上初中时才分
开)。爸妈吵架的时候我不知道姐姐是怎么想的,蜷缩在床上的我每到那种时候就只能
默默地掉眼泪,任凭泪水如泉水般涌出来,顺着脸颊流进耳朵里。而每次爸妈说要砸电
视的时候,就又会想电视砸了明天的电视剧没得看怎么办,会想什么时候才能悄悄躲进
客厅把电视藏起来,或者以后要想个什么主意要求爸妈把电视就放到我们房间来。我小
的时候爸爸带我去看医生、治过很多次耳朵,还用农村的偏方土方法把一种很嫩的植物
藤茎挤出水来浇进耳朵涂在耳朵周围。爸爸奇怪过我的耳朵怎么总不见好,爸爸却不明
白那个孩子为什么、怎么就有那么多眼泪。
\section{父母的婚姻(2)}
\label{sec-9-32}

没有吵架时的爸妈又非常好,家里其乐融融,爸妈随便谁烧饭都是一把好手,有时候他
们也一起一边做饭一边村里长、亲戚里短地拉家常。一家人出门在外也都是热热闹闹,
显得和幕,看不出有什么异样。

小时候的一件小事,让我尤其记得三姐的聪明。那天老妈心血来潮,说是爸爸眉巴怎么
什么时候就一下子长得乱七八糟。说罢拿把裁剪衣服用的大剪刀要给老爸修理眉毛。爸
爸在我们孩子们面前似乎还很害羞,迫不得已躲到客厅门后面,说是怕来个什么人看见
要多不好意思。老妈依了他,便对着从门缝透进来的一缕亮光为爸爸修剪起来。我懒懒
地倚在客厅沙发里,眼见着三姐走出去在外面场子里转了圈就折回客厅,边走边急急地
说,“妈,来客了,我看长渠沿路上好像五姨他们来了!”老妈一恻,这眉巴才只剪了
一根,家里来了贵客可要如何是好?便急急地跑出去看,我也跟着出去迎客了,可路上
连个人影儿也没有!老妈随即回到客厅门后顺手拿起一把扫帚,边往外跑边呵斥说,“
你给我站住!”扬起扫帚来姐姐已经一溜烟儿跑掉了。站场子里看热闹的我肚子都笑痛
了。别看我们是小P孩,亲戚间亲疏远近我们明白着呢!

小时候幼小的我也奇怪过爸妈怎么就能三天两头地吵,难道他们就不能相互让一让吗?
有什么事情是非得吵架不可的呢?爸妈吵得特别厉害的时候妈妈也上过长渠,当年就是
因为妈妈住这个村修长渠爸妈才认识的,但吵架时的妈妈想到的却是一了百了,被暗地
里跟着她的爸爸给扭了回来。有一次他们吵得特别凶,家里面妈妈的话带着浓重的哭腔
鼻音机关枪般往外冒。也是那次吵架我知道了事情的原委。

大概是有段时间了,妈妈奇怪地发现奶奶总是背着妈妈在爸爸旁边滴滴咕咕。几十年后
长大后的我能想明白,妈妈算是有些聪明的,留心逮住了爸爸同三妈将要或是正在发生
什么,抓住了爸爸的把柄,也警告了他们不可再犯;可是妈妈也不足够聪明,一件事磨
在心里多年,折磨了她自己一辈子,因为这事与爸爸时常发作,毁了他们的爱情婚姻,
也毁了后来三姐与我的幸福。

回到儿时,三姐当时是睡着了还是醒着的,三姐知不知道那件事,我从来没有问过姐姐
。那事在我一个几岁的孩子眼里,我本能地认定爸爸是错的,爸爸怎么可以做那种事情
。大概从爸妈那次吵架后,他们再吵架,我还是会有泪水,但我变得越来越麻木。情感
上我也只和妈妈亲,对爸爸淡淡的,对那个奶奶也是不冷不热。对于那个时常会吵架的
家,我也不再抱幻想,他们若还要再吵,随他们吵去吧!
\section{成长的历程}
\label{sec-9-33}

我上小学时二三年级时大姐就已经在镇上软木砖厂上班,外婆舅舅家也住镇上。姐姐就
会时常带来些外公外婆舅舅家里长短的话啊事的告诉妈妈。有一次姐姐回来对妈妈和我
说,舅舅邻居家的那个谁谁把舅舅家那棵桔子树上结的两个大青桔子摘了,表哥知道后
跑到邻居家去闹,邻居家里的大人说要给表哥多称几斤桔子还他赔他,表哥依然又哭又
闹(表哥只比我大三岁)。姐姐模仿着表哥的腔调对着妈妈和我说,“树上就只有这两个
桔子,长了几个月了,我这么好吃的我都没舍得摘,你倒好,把它们摘得一个也不剩!
谁要你买来的桔子,我要你给我赔,你把它们给我长回去!”姐姐妈妈大概觉得表哥小
孩子好笑好玩,她们却不知道我同表哥是一条心的,别说表哥,我这么好吃的,见表哥
还留着它们,我都没舍得摘,凭什么那个人就把它们给摘了?表哥当然该闹,我当时心
里对表哥的做法可是暗自叫好、拍手叫绝的呢!

小学加学前班七八年的时间,记忆里爸妈一路吵架吵到我小学毕业。上初中后,他们大
概也还是吵的,不过初中我已经开始住校,一个星期只有星期六傍晚回家,星期天下午
再返回学校。我上初中后,爸妈也一直贯彻执行、努力实现他们那一辈的理想,和当年
对待姐姐们一样为我树立学习上的榜样,给我讲大表姐、小表姐好好学习上大学出国的
故事。我上初中后爸妈都变得很能控制情绪,一周回去一次他们一般也就不会再吵什么
,妈妈也会为我准备很多干粮蔬菜供我接下来六天在学校里吃。学校放寒暑假的时候他
们偶尔还是会吵的吧,但比我小学时六七年好了很多。

小时候记事后的我和妈妈是很亲的,可是到小学高些年级时大姐已经开始谈恋爱,会隔
三差五地带朋友回家。我心里仅存的那点儿同妈妈之间的温暖也慢慢给冲刷没了。小学
毕业那年暑假我发生意外后,我反复地考虑过要不要告诉妈妈。但心理上我还是觉得妈
妈偏心,觉得她不够爱我。经再三琢磨,我同妈妈谈了一次,我问妈妈为什么她和爸爸
总是偏心、喜欢大姐,不喜欢我们。妈妈说十个手指头伸出来还有长短,一娘养九子九
子各不同,他们作父母的又哪能保证说对我们姐妹就能真正做到完全平等。我的问题把
妈妈问哭了,妈妈的眼泪也把我弄哭了。可是妈妈的哭泣和泪水并没能让我真正理解他
们,没能打消我心里的固有的成见。那天中午与妈妈谈过这话题之后,我把我的心事压
了下去,可能还是觉得爸妈偏心,与其如此,不如自已承担吧,那是我长这么大最后一
次提到说爸妈偏心。

经历了漫长的小学前后七八年爸妈吵架的困扰,我对他们吵架也早就变得司空见惯见怪
不怪了。我惯性地以为爸妈接下来会一直吵下去,吵到终老病死,吵完一生一世,但他
们没有。他们吵了十多年的架,最终却因为一个人而终止改变。
\section{我的大哥}
\label{sec-9-34}

大姐夫比大姐只大两岁,他们结婚前我称呼他“大哥”,后来因为习惯了,他们结婚后
我和两个姐姐还是称呼他大哥,四姨家我时髦的小表妹叫他“大哥大”他也不介意什么。

大姐初中毕业便踏入社会,小小年级早早地相亲谈起了恋爱。姐夫是大姐的第一个相亲
对象,接下来的两三年里,姐姐转了一大圈,简单地相处了总共不下二三十人,最终又
回到了起点,与大哥谈恋爱并早早地结婚,第二年我们家里就添了个小不点儿。

和我们家有姐妹四人差不多,大哥家里也有三姊妹,他排行老二,有一个姐姐和弟弟。
姐姐很多年前已经嫁人,姊妹三人里只有大哥没有务农,上了中专学了橱师,有商品粮
户口,同姐姐结婚时姐姐还是农村户口来着。后来政策放宽了,不只是考上中专大学、
端上铁饭碗才可以转户口,出得起钱也可以买了,爸妈替他们小家庭考虑,便花了五千
块钱为姐姐买了一个,省得他们小家庭一高一低地牵扯着。

小时候,有一次爸妈一起做饭。爸爸在灶堂前生火烧水,妈妈边切菜边对爸爸说,家里
新进个大哥真是称心,那天就只是无意中提了一下家里菜板不太好用,大哥下次回来就
买个新的带回家来。妈妈说要给他们钱,大哥便说只是个小东西花不了几分钱不在话下
。大姐告诉妈妈大哥买了两个一模一样的,两边老人一家一个。大哥大姐结婚后,我惊
喜地发现家里包饺子的次数越来越多,后来才知道妈妈得知大哥也爱吃饺子后,逢年过
节知道大哥大姐会回来的时候便准备好包饺子,家里爸爸、二姐和我这些原本就爱吃饺
子的就跟着沾光了。

大姐大哥还谈恋爱的时候,大哥来找姐姐带来的小零食,瓜子辣子肠什么的,常常会被
我们小姊妹几个瓜分掉。有一次我和三姐背着背笼割牛草回来时下雨了,大哥远远地看
见我们便跑去接我们。初三那年寒假过年,大姐拿一个深绿色绸缎封面很精致的硬皮抄
笔记本和一支钢笔送给我。我问从哪儿来的,大哥便说是他们公司发的,他和大姐都用
不上,就拿给我用。后来我拿它当了日记本,写了满满一本,现在还在姐姐家箱子的某
个角落躺着。上高中时快开学的前几天爸爸就把我送到了住在县城的姐姐家。那天碰巧
舅舅三姨家的表妹们都在四姨家作客,我们小姊妹几个便一块儿玩儿。大哥带我们到楼
下给我们买吃的,他给三个表妹各买了根五毛钱的普通绿豆雪糕,却给我买了个一块钱
的“娃娃脸”(那年比较流行的一种雪糕),我就奇怪了问大哥干嘛不给我买跟她们一
样的,大哥当时就重新给我们换成了一个一个娃娃脸。

那个年代大哥大姐在县城的小家也是我们小姊妹几个呆在县城的中转站。二姐当年中专
快毕业,最后一年在县医院实习,二姐便在大哥大姐家吃饭吃了一年;后来三姐去帮他
们看小孩,三四个人一起闹腾了一年多;一两年后我上高中,这是个大工程,我一日三
餐都在姐姐家吃饭,一吃便是三年。但对于我们这些小姊妹的打扰,大姐大哥从来都没
有任何怨言。
\section{我的大哥(2)}
\label{sec-9-35}

高中三年我是走读,吃住都在姐姐家,但我与他们交流很少。姐姐家我晚上是住在另一
栋楼上,白天一日三餐我一般都是打机关枪般吃完饭就去学校,但那时的我是不会思考
问题的。我还是更喜欢数学一些,有一次问数学老师一道题目,老师想把我骂醒,可惜
当时只是把我骂哭了,高考之后自己醒悟才真正明白老师当时的苦心。

高中时姐姐姐夫都不对我说什么,但上大学后大哥开始以一种比较轻松地方式教育我。
新年里大哥带领大家包饺子,大哥一边忙和面也会一边给我们小姊妹小表妹们讲述一些
有趣好玩的故事,大哥也会借他同事的故事间接地教育我,让我意识到自己需要改进的
地方。表哥出来后与大哥姐姐们走得近。大哥也常教育表哥。妈妈说大哥对表哥说过,
“你还记得不,你身上流的有我的血!你不好好混对得起当年输给你的血吗?”

大学毕业那年的暑假,我抱着毕业离校时的签名册和班上同学送的照片,给姐姐讲述我
的大学生活。讲完我自己的故事,姐姐也给我讲她的生活,讲大哥的故事。说大哥一个
哑巴女同事,在QQ上认识了一个四川男孩网友,去见了,但回来后不久发现怀孕了。但
那网友显然无意继续什么。女孩发邮件给大哥说若大哥不帮忙陪她去医院做人流,她第
二天就去找同公司对她有意的一哑巴男生随便过日子了却此生。大哥不敢轻率答应女孩
什么,但第二天仔细观察,那女孩真要去找哑巴男孩。大哥看不下去了,带女孩回到姐
姐家,同姐姐讲清楚事情的来笼去脉,后来姐姐陪女孩去医院做了人流。不知道后来那
个女孩幸福了没,但姐姐说大哥是个可靠的人,万一那女孩有个什么小心思,大哥若有
一点儿把持不好分寸,事情可能就会成为另外一种样子。

妈妈早就说过家里这个大哥的好,我总是无动于忠,总想当然地以为我又为什么做不到
。大姐大哥结婚以后,意外发现了家里这偶尔出现的一大奇观,便开始作爸妈的思想工
作,说一辈子那么多年都过来了,还有什么过不去的坎非得要吵着过。爸妈当面当然是
听的,但背后可能还是会吵的。所以后来就有那么一天,姐姐们回到家爸妈在吵架,妈
妈边吵边包饺子,大哥自个儿在床上休息,等爸妈煮好了饺子,大哥无论如何也不吃。
当爸妈用最真的心做大哥爱吃的饺子他却不吃时,他们大概才最终意识到大哥苦口婆心
劝了那么多次的不要吵架一事对大哥到底有多重要。第二天大哥吃了些前天晚上剩下的
饺子,但自此以后,爸妈再也没大吵过了,小争两句可能还是有的。
\section{无标题}
\label{sec-9-36}
故事的主人公19岁那年从一场浩劫中醒来。醒来后意识到十多年来的沉睡原来也只能归
结在她自己身上;经历磨难后的她重获亲情、也获得一点儿生命力,开始努力学习考研
出国,来到她向往过的国度,转了专业,读研毕了业,工作了。

工作后的自己因为亲人(远亲)的关系偶然相遇一表哥,便从此惦记上了。相遇的绽放如
此美丽,以至于双方都往相反的方向逃。主人公对于表哥的家庭、表哥的亲人、表哥家
鸡毛蒜皮的小事耿耿于怀,而表哥的爸爸也一口咬定主人公其滥无比不择手段。双方在
时间的沧河中僵持着。

一两年后主人公的父亲不幸离世。仍残留着一点儿孩童时迷信的主人公固执地认为父亲
始终以另一种方式陪在她身边,以至于因为一件不相干的小事想要放弃这段关系。好在
及时认识到一旦放弃就意味着永远失去,而表哥始终还是那个茫茫人海里主人公享受过
恩宠、不可替代的the one。于是放弃自尊,回到表哥家同表哥的父亲进行了一场穿透
时空的对话。

最后的最后,主人公将自己的成长故事写出来,与大家分享那个在扭曲的环境中长到19
岁,又用十四五年的时间通过学习将自己的个性慢慢扭向回归线的故事。劳动创造美
,而孩童时情感上的引导、心智教育会改变人。

主人公情感上的归宿自然还是在表哥这里。而该如何解决问题,尚在思考中。。。
\linebreak
\linebreak
最近一段时间有点儿累,昨天吃完晚饭八九点钟就休息了。午夜梦回,来回你的贴子。
以后应该会尽最大努力把这个故事写完,写到可以打暂停号的地方,以这样的速度可能
还需要半个月左右。

你指出的挺对的,上大学时有了与同宿舍一南方女孩的比较才认认到自己竟然都不懂得
关心人。。。从来没有想到过自己会因一句不小心说出的过分的话而被人肉,会受到大
家的关注。故事里也对主人公的某些个性作了剖析,33岁茅塞顿开,能把自己看得清楚
透彻,想来以后的生活都会顺遂的,很欣慰。

主人公是在扭曲的环境下长大的,写这个故事的过程也让自己收获很多,通过组织思考
故事的经络也把自己梳理得很柔顺。这个亲身经历的故事也只是希望提醒没能意识或是
没能足够意识到孩子情感教育的家长多花心思在他们的子女身上,为他们的孩子提供一
个温暖健康的身心成长环境。而绝大部分读到这个故事的人对于如何教育自己的孩子可
能早已拥有足够的觉悟,所以大可不必浪费时间在这样一个故事上。谢谢你的关注,希
望大家都好好的。。。
\section{成长的历程(2)}
\label{sec-9-37}

上初中后我耳根清净,便也好好学习,初三还交到一个非常要好的朋友,顺利地上到了
高中。我当年没把那件事情坦白地告诉妈妈,而接下来的五六年里,我心里萦绕的最多
最大的问题便是我将来长大成婚后能不能生小孩,因为妈妈讲过的关于人各有命,因为
一路走来仔细观察过的自已的身体发育。那五六年里,我大概从来没有觉得自己是个正
常女孩吧。独自承担总是要付出代价的,那五六年里,除了初三有很要好的朋友,心里
一直比较苦,总是有那么多的思想顾虑隐隐作痛。高一高二我是不笑的,当我压抑到极
致,压抑到无法承受,97年暑假我遇见了舅舅,舅舅好好地鼓励了当年的我。 

高三那年的五月,当我绝望无力无法再承受一切的时候,当我不得不向亲人们坦承我一
切的思想包袱后,爸爸妈妈和姐姐们帮了我。当我告诉姐姐我担心将来自己不能生小孩
,我觉得自己活在这个世界上没有意义的时候,学医的二姐向我清楚地解释了叔叔三妈
“引子伢子”不是命中注定。姐姐告诉我三妈家的小堂妹不是叔叔的孩子。我听不明白
,姐姐继续告诉我说,当年的叔叔三妈完全可以再领养一个孩子,可能三妈可能还是想
生一个自己的,所以小堂妹生理遗传意义上的父亲另有其人,是三妈那个地方的一个什
么人。第一次听到姐姐这样的解释,我相当震撼,回想小时候爸妈吵架的原因,回想妈
妈提到过的叔叔三妈是有去看过医生的,我相信姐姐说的是真的。可是我仔细观察了那
么多年的身体变化啊,我将来真的可以生小孩吗?我还是不确定。或许每个女人都会想
要体验一回恋爱、结婚、怀孕、生子作母亲的幸福吧,姐姐给我解释那件事后我的生命
还是一片荒芜。我只是学会了将这件事暂且放下。

那次我在生死关头,爸爸以一种沉着大气、深沉内敛的人生处世智慧解救了我,也让我
从心底消除了以往所有的顾虑,原谅了我成长过程中爸妈对我照顾不周的地方。亲情从
此转化为一种生命力量支撑着自己。
\section{我的叔叔(2)}
\label{sec-9-38}

我们家里爸爸这边的亲情显得冷淡,之前奶奶允许吃东西也只能是伯伯家两个孩子我们
家也只能去两个孩子,似乎一切都计算得很清楚。而妈妈那边的亲人,四姨五姨待我们
从来不会因为我们家姊妹众多而礼轻计较。相反,小时候五姨姨父在我们家作客五姨都
还常会背着我爸妈给我和三姐一人打发十块钱,说如果把这件事告诉爸妈就证明我们不
打算好好学习,若不告诉他们就说明我们是爱学习的好孩子,我常常会被五姨的难题难
倒。

而爸爸这边我们最亲的也就只有叔叔了。当年把二姐送回我们家之后的叔叔和三妈心理
上又始终觉得二姐还是他们家的人,后来二姐考上中专成为那时的铁饭碗,叔叔三妈待
二姐比大姐三姐都亲,常常是送三个姐姐一点儿什么农村土特产,二姐的那份总比另外
两个姐姐的多,姐姐们也常取笑二姐是叔叔三妈家的孩子。我不清楚二姐跟了叔叔三妈
到底多长的时间,心理上二姐似乎也始终记得叔叔待她的好,对叔叔可能也是一直心存
感激吧。所以多年来只要二姐放假在家我们会有一个固定节目就是,二姐带领我和三姐
,有时候也叫上堂哥,一起走路到叔叔家去吃餐中饭。

我们到叔叔家去一般会挑下雨天去,因为叔叔同爸爸一样,也是把弄鱼的好手。到了之
后二姐三姐会帮三妈做饭,而我一般会随了叔叔去打鱼。叔叔所在的镇县有很多溪水鱼
塘,与爸爸用钓鱼钩磁电机不同,叔叔用渔网。随叔叔出去打鱼也是我小时候的一大乐
事。每次一到叔叔家,叔叔就会从房梁上取下渔网,我便又会提着个小渔篓屁颠屁颠地
跟着叔叔来到广袤的大自然中去呼吸雨过天晴的新鲜空气。叔叔身板很结实,一张巨大
沉重的网,他两胳膊把它拉开,右胳膊往前轻轻一带将整个网甩出去,凭它是江河湖渠
,收起网来总会有些收获。爸爸收获的鱼里泥湫鳝鱼居多,细细长长深色的,而叔叔网
到的鱼多是白肚皮、有鳞有翅的传统意义上的鱼。我总把大鱼小鱼全都捡起来,叔叔却
让我只捡大的小的可以不要。回到叔叔家这些鱼就会成为我们美味的午餐。有时候我们
的渔获比较多,叔叔也会留几条大的不做午餐,让我们下午带回家晚上爸妈就也可以吃
上。
\section{我的叔叔(3)}
\label{sec-9-39}

我孤身在外,舅舅和叔叔临终前都几乎没能得到我任何的孝敬。叔叔被诊断癌症晚期后
,姐姐们感念叔叔一生都没有自己的孩子,妹妹们尚小,叔叔基本也没能享到堂妹们什
么福份,那年八月十五便聚在叔叔家陪在床前,希望他能感到些许安慰。

听妈妈后来说,叔叔临走前妹妹们守在床前,叔叔大概还是不大放心大堂妹没有男朋友
、婚事未定吧,叔叔走得不太平静。妈妈说大堂妹从来不同小妹计较谁待父母更好,家
里需要什么就买什么,待叔叔三妈实际上比小妹要好、周全很多。有时候我会想,假如
大四毕业舅舅资助我顺利来美读硕士了,我的生活向前迈进五年半该有多好;有时候我
又想,假如高二暑假我没能遇见舅舅,高三的我一如高一高二一潭死水日子又将会过得
有多痛苦。感恩的心,我们到底有没有,相信在大堂妹报答叔叔三妈的养育之恩上已然
体现。而我自己,高考之后,不管之前自己成长得多么艰苦扭曲,我都应该彻底原谅他
们,应该感谢他们用辛劳和汗水给予了我求学的机会,他们能做到这一点已经很可贵了
。社会的和谐运转有它自己的规律,那些学得精明只索取不回报的人或许迟早还会经受
社会的考验吧。

我深信大堂妹的世界相比她妹妹更为精彩。妈妈说后来大堂妹嫁了一个男孩子,父母过
早离世,他与姐姐相依为命。姐姐在外打工帮弟弟在家乡建好一栋房子才嫁人。我本能
地相信身世如此相近的大堂妹与妹夫一定会幸福的。他们与我一样拥有亲情。
\section{爸爸(2)}
\label{sec-9-40}

在上次三姐给我传照片接下来的四个周里,每次打电话问她她都说忙,没顾得给爸爸拍
照片。我脾气急,电话里就同姐姐吵过一次,姐姐哭着说妈妈不准给我传照片。那次吵
架一周后的12月28日,我终于又收到姐姐寄来的一组照片共12张,和四个短的视频。照
片里昔日精神掘说的爸爸只剩下皮包骨头,眼睛眯成一条缝。妈妈把爸爸用军大衣包好
放在外面阳台上沙发椅里晒太阳。视频里爸爸确实有张口嘴巴打了一个大哈欠,妈妈和
姐姐说爸爸可以张嘴了应该没有骗我。

同以前一样,我每收到一次爸爸的照片心里难受一次。我心里难受的直接反应就是打电
话给妈妈询问各种细节。或许心理上我一直觉得同爸妈比同姐姐亲吧,以前我打电话都
是只对妈妈说,同大姐讲话很少。这次,打给妈妈的第二天,我忍无可忍,对妈妈要求
让大姐接电话,妈妈去橱房做饭了。这次电话里同姐姐哭了一个多小时,给姐姐讲爸爸
年轻时的离婚再婚,中年时的出血热,爸爸一生作过的邮递员木匠沏匠,爸爸钓鱼打鱼
下壕子,我对姐姐说爸爸一生这么坚强,所有的困难都挺过来了,只要我们把爸爸照顾
好,帮助爸爸挺过这最后一关,爸爸还可以陪我们还好多年!姐姐被我哭得没办法,说
她敬到她的孝心,她尽到她的努力就可以了。

接下来的几天里,妈妈和姐姐按照我提的建议给爸爸食物加量,并每天定时给爸爸饮果
汁。1月2日的中午我去银行取钱准备交房租,我的破车开始出现踩刹车会哽死症状。三
号晚上打电话回去姐姐抱怨说爸爸小便排泄还算正常,大便有几天没有了。我劝姐姐没
事儿守在爸爸身边,多仔细观察爸爸的情况。四号周三上午公司里老板对我们一共三个
同事说由于公司内部改组,我们三个职位被砍掉了,两天后的周五是我们最后一天(我
在这个公司工作很久了,很板对我也很好,我对公司的决定没有任何怨言)。这几天我
的车还是一直坏时不时地熄火。星期四晚上,从公司到家十迈的单程,freeway上踩刹
车熄火了几次,local红灯前又熄火了好几次。Local上有好心的姐姐吆喝我提醒我说我
车灯坏了。回到家房东帮我看了看车。

周四晚电话打到家里,妈妈说那天从早上起爸爸神清目明,眼珠清亮,嘴里念念有词念
叨着我的乳名。我问妈妈姐姐的手机有电没有,我要同爸爸说几句话。姐姐把手机放在
爸爸耳边,电话里爸爸唤我一声,我就应一声,我随着爸爸的节奏说话,让他知道我就
是他嘴里念叨着的那个女儿。我哽咽着对爸爸说那时候我小,临走前只答应了爸爸遇到
任何困难都不能放弃,却没想到该同爸爸要一个同样的承诺,要爸爸好好的等我回来。
现在我的工作问题还没能解决好,我一时半会儿不能回来看爸爸了。电话里,我又细数
了一遍爸爸一生中的几件大事,对爸爸说是爸爸的爱和坚强支撑着我不要倒下,也希望
爸爸能坚持住,挺过这一关,等我找到工作一定尽早回来看爸爸。电话里爸爸叹了口气
,我想爸爸应该是答应了我。电话里我嘱咐姐姐同妈妈好好照顾爸爸,放下电话也该要
准备好好找工作了。

第二天我看到三姐QQ留言“爹等不到你他走了”,瞬间崩溃,泪如雨下。爸爸走了,我
的身体还在,心却随爸爸飘远了。
\section{我的爸爸}
\label{sec-9-41}

高二高三时我有一个女同学,她的爸爸是镇上杀猪卖肉的屠夫。那天下午我们放月假,
她的爸爸早早地来到学校教室门口等她放学。放学后她们父女两个手牵着手下楼梯齐步
并驱走在前面,我同学回过头看见走在后面不远处的我,很不好意思地挣脱了她爸爸的
手。而眼前刚刚逝去的那一幕却让我若有所失。之前我从来不曾真正想过国内读硕士时
为什么会愿意、想要去了解一个大自己那么多的老板。来到美国后,在异国他乡的孤寂
里我有一年左右的时间周末在教会度过。教会里每次看到有小女孩嘴巴贴着她爸爸的耳
朵说悄悄话我总会情不自禁地看傻呆掉。

我与爸爸之间比较close的记忆停留在自己三四岁时。可能我小时候很会撒娇吧,记得
那是妈妈把我平抱在怀里,爸爸却拿他的胡茬扎我的脸。扎完了还要问我喜不喜欢,小
P孩可能多少有点儿可爱吧,撇着嘴扭扭捏捏地说不喜欢。爸爸还要问为什么不喜欢,
我便也会说爸爸的胡茬很扎肉的。妈妈就会帮我把爸爸打跑。而这点记事竟成为我与爸
爸之间仅有的最亲密的记忆。

今天当我写下这段话的时候,我终于明白,我同爸爸之间的父女亲情还是略有缺失的。
或许也正是因为那种近乎天然的缺失才让我后来对父爱的体会显得深刻。

那年夏天我没有把自己的心事告诉妈妈,接下来的五六年一个人一路走到黑,走到无法
承担精神崩溃。绝望下的我体会了亲情,也深刻体会了爸爸的沉着深遂、大气淡定而又
内敛。当年姐姐姐夫把我领回家交到爸妈手上,爸爸的坚毅、指挥若定,几个月后我告
诉爸爸若我考不上学还想再复读时爸爸湿润的眼睛和那一个“好”字,出国临行前爸爸
特意交待过我的话都永远地刻在了我心上。很难想象长大后深刻体会着父爱的我高考之
前对爸爸却一直都只是淡淡的。而爸爸对我,对姐姐的爱从来都没有变过啊。
\section{我的爸爸(2)}
\label{sec-9-42}

伴随着家里盖了楼房添置了各种家电,大姐有了很好的婚姻归宿,二姐早早地端上了铁
饭碗,我们家的日子越过越好,爸爸对叔叔和伯伯的心疼体谅就越来越多。妈妈说只要
叔叔到我们家作客,爸爸总是争着要亲自做饭做菜,虽然不至于为叔叔杀家里的鸡,但
打鸡蛋、切瘦肉炒肥肉从来都不吝啬,吃饭时平时还有点儿饭量的爸爸也会舍不得动筷
子,总是让着让叔叔多吃些。妈妈知道爸爸心疼叔叔家过得艰难,从来不说爸爸什么,
烧火做饭也只是默默地帮爸爸打下手。叔叔离世后,爸爸又将他那点儿兄弟情谊转嫁到
了伯伯身上。伯伯家只有一个堂姐一个堂哥。堂姐早就嫁人了,堂哥身高体帅年轻时挑
挑捡捡要高的要白的要漂亮的要年轻的要到最后落得四十出头了也没找到个对象。家里
后来也盖了楼房,堂哥却待伯伯伯母极差,烟不能抽酒不能喝卖到村里来的零食也不能
换。爸爸后来在哥哥姐姐们的劝导下为身体健康考虑从善如流先后戒了烟也戒了酒,但
为了让伯伯能过酒瘾,家里便常年备有一两斤白酒,等改天伯伯撞上我们家吃饭时可以
喝。

爸爸一生都是孝子。年轻时他的孝表现得事与愿围,爸爸认定爷爷奶奶指望伯伯一个人
养老是指望不住的,执意离婚,却直接导致了爷爷的离去。小时候我同妈妈亲,妈妈很
多时候对奶奶让我们家只准去两个孩子吃东西表现得很是愤愤不平,气极了妈妈也会说
,“你现在只让两个吃,等哪天你死了我们家也只准两个跪,两个小的到时候不许给你
下跪!”长大后我能明白爷爷走得早,奶奶杀一只鸡做菜,七个人吃每人能吃到的都不
多(奶奶伯伯爸爸外加四个孩子),要是九个人就更少了,再说奶奶也从来没有说要独自
吃啊。可惜小时候我不懂,随妈妈不喜欢奶奶。奶奶年级大了后爸爸和叔叔轮流照顾奶
奶,每家包吃包住半个月轮班。爸爸总是把奶奶该到我们家的日子记得非常清楚,前一
天晚上就上伯伯家去帮奶奶扛铺盖卷,早早地把奶奶接到我们家来,平时吃穿用度更是
对奶奶尽心极了。

爸妈都是心肠很好的人,加上爸爸手艺多,给这家修个橱灶,那家做个桌子什么的,远
亲近邻的帮过很多人。淳朴的农村人帮人家点儿小忙也不会好意思要工钱,爸妈都喜欢
幺儿子幺女的,我便成为那个先后坐在爸爸二八红旗自行车的前杠、后座上,吃着百家
饭长大的幸运小孩。要请爸爸帮忙的人一般前一天就会到我们家同爸妈打好招呼,第二
天早上我们吃好早饭,爸爸就会把我放在前杠上、带上劳动工具骑着他的大自行车去帮
忙。到了主人家,看我是小孩子,女主人一般都会把家里有的零食拿些出来给我吃,即
使家里实在没什么吃的,也会为我煮几个白鸡蛋。有时候我会在主人家吃一两个,有时
候我也会耍乖乖嘴把它们带回家去。我们在主人家吃中饭,待爸爸忙完一天,该帮的忙
帮完帮好了,爸爸就载着我回家,到那时妈妈在家一般也把晚饭做好了,我们就开饭。
有一次在邻县县城里为一个远亲的姨奶奶家定地基准备盖房。大夏天的我见不远处有卖
西瓜的便对爸爸嚷嚷着说口渴,那时西瓜还分片卖,那天爸爸为我前后连续买了五片西
瓜之后才把我领到水龙头下喝自来水。爸爸待我那么好,儿时的我却不懂得体会。

当我敲下这些字的时候,我也不免问自己,一叶障目,两耳塞豆,要怎样的残忍绝决、
固执狭隘才能把待自己那么好的亲爸爸拒绝在心门之外?爷爷受他那个时代的局限,生
养了那个抛妻离子的忤逆儿子,一根绳索便了结了生命;奶奶受她那个时代的局限,为
了一个姓氏为了家簇的传宗接代,叔叔有生理问题,便指挥起了爸爸;爸爸有生活阅历
看得开明却也有天生的死穴,面对爷爷离世后孤苦伶仃的奶奶只想“孝”“顺”,终究
是“背叛”了妈妈。若时光可以倒流,我多么希望我从来都不知道这一切的过往!
\section{我的爸爸(3)}
\label{sec-9-43}

稍微平静后,我打电话回去问妈妈爸爸临走时的情况。妈妈说老古言说生老病死,死也
是人生的一庄喜事,爸爸临终时走得很安详。前一天傍晚爸爸身体已经开始水(浮)肿,
二姐也回来照看了爸爸一个晚上。第二天早上姐姐骑电瓶车去上班,却怎么也走不动,
平时只要十几分钟的那天早上路上耗了半个小时。那天中午一点半我打完电话后,妈妈
给爸爸喂了一袋牛奶。可能几天来的食物没能太消化好,爸爸吐了些黄水。妈妈为爸爸
清洗干净后,大姐接到大哥的电话说第二天晚上公司有晚宴,要姐姐休息一下到市里去
一趟。爸爸大概听得明白舍不得姐姐走吧,便追着姐姐我挂断电话仅只三个小时左右爸
爸就走了。妈妈说看着看着爸爸就只能呼气不能吸气,前后十分钟不到就平静地离开了
。想着叔叔走时不太放心大堂妹的婚事,想来爸爸还是知道我的,知道我有归宿,所以
会走得放心。

但我终久还是舍不得爸爸的。我有困难时农村季节爸爸一个人在家里看家,要妈妈把我
看好,高考结束后把我留在家里静养;而爸爸有困难时,他却没能指望得上我们。回想
国内硕士毕业那年的春夏我严重失眠,每天白天吃得很好,晚上在床板上翻无数个来回
,几个月下来就已经瘦得很干净。爸爸只能吃流食,脑昏迷后也不能保证每天八个小时
睡眠,那这场昏迷对爸爸的身体消耗该有多大?若我们不那么早让爸爸出院,保证爸爸
每天的几瓶点滴营养和一日三四餐的流食,爸爸的身体虚耗会不会减少,爸爸是否就能
坚持得再久一点儿坚持到过他正月里的生日?妈妈说男怕生前女怕生后,爸爸坚持到过
了生日没准就挺过来了。可爸爸有我们四个女儿却最终都没能指望上。打哈欠与主动张
嘴能一样吗,爸爸能张嘴是真的吗,还是终究只是妈妈姐姐商量着要稳住我统一口径撒
下的谎?若我不空中瞎指挥,不给爸爸食物加量,爸爸是不是也不至于消化不良?我只
保证了资金到位,却对爸爸关心不够,多个环节上一错再错,觉得自己太愧对爸爸了。

家里妈妈很强势,她绝对不会允许自己辛苦劳动挣得的钱花在别人身上。那爸爸有没有
想过想要认他当年的那个儿子?我可以理解晚年的爸爸打摩托车的主意是因为一直以来
他骑车水平都很高,想挑战摩托车也是他一生积极进取这一个性的直接体现。可晚年的
爸爸攒钱又是为了什么?爸爸对我那个half brother 会有遗憾吗?妈妈强势,而我这
个自认为能够理解爸爸的女儿又为爸爸作过什么?很长一段时间以来,每次一想到爸爸
,我都无法控制自己的泪水无声地流淌,我知道这一辈子我亏欠爸爸的太多了,而且这
种遗憾永远无法弥补。
\section{找工作}
\label{sec-9-44}

爸爸离开后我心里一直空荡荡地很失落,虽然我也没有掉块肉但精神上一定少了些什么。
正月里爸爸生日的前一天晚上妈妈因为稍微多喝了大半碗饺子汤第二天消化不良吐了,
加上老年人颈椎部位什么病,打点滴打了两三天才稍微好点儿。

妈妈在医院治病时我有打电话到农村老家,是大姐家的小侄儿接的电话。他跟我说及妈
妈的几个不良习惯,说妈妈也不知道改一下。我不容丝毫犹豫地告诉他,奶奶已经是上
了年级的人了,不要希冀她去改变什么,有什么事情稍微包容一下做晚辈的忍一忍顺着
老人家也就过去了。当然我说话的语气气势可能还是让他难受了吧。结果接下来下次打
到农村老家的电话是大姐接的,大姐让我打电话给大姐夫,我便打了。电话里姐夫说到
与小侄儿说过类似的问题,我还是当仁不让地将姐夫反搏了回去。我说我与爸爸一起生
活过的时间很短,也基本就到初中,后来在家的日子越来越少,爸爸走了我尚且难过,
为什么妈妈作为那个同爸爸生活了一辈子的人想起爸爸伤心难过就一定不应该、不可以
,这是妈妈自己能够控制得了的吗?我脾气急,电话里与大哥针锋相对,话说得非常直
。妈妈病好后,大姐也该回去上班了,他们打算把妈妈一个人留在村子里。我心里忍得
难受,但自己的工作也还没有定下来,也只能在心里默默地祈祷妈妈能想开一点儿。

找工作是爸爸临终前我告诉爸爸我不能回去看他的原因,可能也是爸爸与我通电话后三
个小时就速速离去的,想让我了无牵挂地好好努力的愿望吧。几个星期后我开始到一个
小公司上班。因为有身份的顾虑,面试前别人说想要的是临时工,到要起草合同时我追
问了一下能否办身份,别人说临时工去留无意,都是at will的。话说到此,双方也就
心知肚明,签了八个周的intern。谁又能知道接下来的命运会是怎样?有份工作总比没
有的好,这活儿我得接了。
\section{新工作}
\label{sec-9-45}

面试的时候我听说这个公司大概有十五个左右的全职员工,半职的据说也有一二十个。
真正工作后了解到公司成立的几大cooperator都是有过多年业内经验,对行业非常熟悉
,以一生所积累学以致用创办了这样一个公司。十多个全职员工每个至少都有五年以上
的工作经验,全都没有身份问题,除了我之外。我翻过一遍他们的profile,做得好的
一个同事可能拥有十个左右的专利。他们三十出头四五十岁,婚姻家庭生活稳定,不为
养家糊口辛劳,享受的更多的可能是创造性工作所带来的心理满足。若是十年后的我来
这样的公司工作,倒是一个美妙去处。而现在在这样一个工作环境,我显得尤其格格不
入。小公司给我的感觉与想象中乔布斯刚刚起步时的苹果公司很像。我的那些同事们,
他们能够放弃高薪;放弃大公司来这里工作,我相信他们对公司的理念、运作模式非常
认同,怀抱着同样的理想想要共同打江山。

世界也很小,我的新同事里就有一个大半年前在一家大公司面试过我的面试官。半年来
他略有发福。我刚毕业不久工作一两年,也只在两三家公司工作过,对于大公司可能多
少还是有些向往吧。我问过这个同事他为什么会放弃那份非常体面的工作来到这个名不
见经传的小公司。我以为他会帮我请解公司的理念,运作模式,他的回答却非常出我意
料。他说原因很简单,因为之前同他一直合作得很愉快的老板离开了,他就离开了。我
很不以为然,因一个好领导的离开他便离开一个很好的职位,是否过于轻率?我暗自告
戒着自己以后若我遇到同样的选择,我要考虑周到、谨慎决定。

工作的第一周我就接到前面这家大公司的面试电话,是个半年长的临时职位,但中介会
帮忙办身份。因为小公司与我都心知肚明的身份问题,我有面试有进展有offer也就直
接告诉了招我进公司的同事。他们说想招临时工也只是因为他们想找到真正合适他们的
人,若组里觉得我还不错,也会考虑帮我办身份。同事说即使我的起薪不高,他们也会
给我分些股票什么的作为迷补。但对一个完全没有概念的人提这些如同鸡同鸭讲实在不
通啊!工作经验不多的我对于大公司的向往还是比较显然的,加上也担心小公司想留我
只是出于礼貌并无真心。所以最终明知道小公司待我还很不错,我却还是无比遗憾地选
择去了大公司,离下这个八个周的intern只干了两个周,做了三四个小project(后来他
们招了另一个劳力这是后话)。
\section{“熬成婆”}
\label{sec-9-46}

新的工作我签在中介公司,中介公司的client,我的直接工作公司是前面提到的这家大
公司,工作合同一年,中介帮办工作身份。所以身份问题最终解决下来。

多年的媳妇熬成婆,工作的问题终于解决后,我也花了些小钱稍微改善了一下自已的生
活。以前我一直用房东的床,这次我在Craigslist上找到一张床,这个size的大床我向
往了很久,后来我有添置些许床上用品也全都是按这个size买的;更换成一张L-shape
办公桌和一把office chair;找到一块圆玻璃和合适大小的IKEA Lake Table用作卧室
里的小餐桌;也找到张low profile的显卡,再加一个monitor,将家里的台式机成功升
级为双屏。这一系列的改善除了显卡是新的,其余都是二手的,总共花了我不到三百块
钱。但得到自己向往过、喜欢的东西,觉得自己的生活很圆满很知足了。

我这次找工作期间注意到表哥的profile又变了,他的名字改变成了英文名中夹着中文
名字,那个中文名字按照发音我听妈妈说起过。拼写还是与大陆拼写不一样,一如当年
来美第一年我用大陆拼写搜不到舅舅任何信息。妈妈听二外婆说,二舅家共有三个表哥
,大表哥的叫“心远”,隔我们国家远;二表哥叫“心选”,是有心选择的一个国家;
三表哥叫“心围”,二外婆说那是围着我们(?)(舅舅的姓氏)家户转!之前同舅舅聊天
时舅舅大概也为我解释过一遍,不过意思相差不远,我还是只记往了妈妈转述二外婆的
生动版本的解法。

找到工作后不知道舆论从哪里传起,传说我与表哥早已分手我也早已有了新欢,实在是
树大招风,不知道该怎么说。多年来的逆势让我对这种空穴来风早已司空见惯,但我与
表哥相距太远,还是应该尽量减少不必要的麻烦和潜在的误解。于是后来background 
check通过之后我就将网站上自己的profile内容和名字先后都改得与表哥的神似,后来
据说被人批说我有暴力倾向,不过那些说法在我看来也都无所谓了。
\section{写邮件}
\label{sec-9-47}

找到新工作后在我到新岗位报到前一个周,我就借机给表哥写邮件。可惜他还是同以前
我写给他和舅舅的“MS Computer Science”一样,不肯理我。我也不多想,继续给姐
姐们打电话,告诉她们我在这边的消息,告诉他们我工作的问题已经解决了,但还是需
要时间等待身份。现将写给表哥的邮件附在这里。

\subsection{发送时间:2012-02-22 11:20AM;}
\label{sec-9-47-1}
收件人:表哥;

Subject: jobs \& visits

Dear cousin, 

I did not apply for any Ph.D program last year finally, and instead now I 
found a full time position who will sponsor H1B for me. From coming week I 
will be working for a famous staffing company, and I will work for its 
client XXXX for at least 6 months (which is within 5 miles from where I live
, pretty close). And after that, XXXX may extend my contract to be one year.
If it does not, the staffing company will fit me to some other big 
companions! I feel excited about my opportunities!

The oral offer came last week, after one week background check, and the 
formal offer came yesterday. 

I bought a second hand queen size wood bed last Sunday, \$130 for bed and \$30
for transportation. I feel like it a lot, and hoping you will come to CA to
see me soon. When you come, you can help me pick a new mattress, or even a 
new bed if we want to. 

I am writing to let you know all my good news, and let's have all our best 
wishes, 

Sincerely, (myname)

给表哥写完邮件,那天晚上我也给妈妈打了电话。可以想像,这次打给妈妈的电话会同
以往不同。我告诉妈妈我找到可以给我办身份的工作了,递交申请后三个月左右的时间
身份就会批下来。等到十月一号新的身份一生效,我就回家去看妈妈,并打算这次把妈
妈接来与我同住一段时间。妈妈若来她签证的有效时间可能是半年,如果从这边能再延
半年的话,那妈妈就可以来一年。这件事我向往了很久,希望有一天我能把爸妈接过来
与我同住半年一年的,让一生中好歹也算是见多识广的爸爸也能有机会来看看美国,看
看我,看看他的小女儿在美国的生活。可惜这个愿望对爸爸来说已经永远地不可能了。

但我没能想到的是,我这个急切地想要与妈妈分享喜悦、分享愿望的电话,却让妈妈饱
受劫难、饱受病痛的折磨。
\section{妈妈生病}
\label{sec-9-48}

我下次再打电话回去妈妈就又生病了,感觉病得很严重。姐姐们把妈妈送到医院,医院
检查说是脑溢血。前面我没能作具体交待,外公一直有高血压,59岁时突发脑溢血抢救
无效离世的;舅舅那天在四姨姨父家作客,晚间突发脑溢血,第二天早上才发现,送去
医院作开颅手术但抢救无效,也是59岁离开的;妈妈今年62岁,检查也一直说是有高血
压,有时需要吃些降血压的丸药。这次这个家族遗传病般的病便就发生在了妈妈身上。

妈妈这次生病的前后过程我们姐妹想起来都后怕。那天晚上我打完那个给妈妈报喜的电
话,第二天早上一起床妈妈就明显感觉到了身体不适,她以为是颈椎病什么的又犯了,
上午便自己走路去村子里打了吊针;打完吊针勉强走回家的妈妈还是感觉身体疲乏无力
,便没吃中饭躺在床上休息;大姐早上已知道妈妈生病,想着她去打完吊针应该会好点
儿;等下午一两点钟大姐再打电话回去妈妈没接电话,大姐便将电话打到了邻居家,请
邻居那个嫂子帮忙上我们家去看看妈妈的情况。邻居回头回来说妈妈像是有点儿犯迷糊
。二姐三姐离家近,大姐火速通知她们,姐姐们也赶紧打120车回家把妈妈送往医院。
姐姐们回到家时妈妈已经失去知觉,大小便失禁,医院里医生及时诊断出是妈妈这是脑
溢血。而得了脑溢血这样的病,时间就是生命。外公和舅舅不都是因为抢救不够及时离
去的吗?我能找到工作、妈妈生病还好能被及时发现进行抢救,我想爸爸大概在暗中保
护我们吧。

妈妈发病的那天二姐守在医院照顾了妈妈一夜,经治疗,凌晨四五点钟的时候妈妈已经
清醒过来,知道是二姐在照顾她;我早上六点打电话过去,妈妈同我讲过电话,虽然说
话显得还很吃力,但意识是完全清醒的。后来电话里我同二姐说,爸爸走后,妈妈接二
连三地病了好几次了,是不是会不会爸爸临走时还有什么放心不下的?我要姐姐们改天
找个机会回家到爸爸坟上为爸爸烧些纸,也替我问候一下爸爸。
\section{表哥的回信(需要再整理一下格式)}
\label{sec-9-49}

接下来的周一,我就该到新的工作岗位报道了。或许有着对大公司的向往,以前都习惯
晚上洗澡、第二天早上顶着“鸡窝头”着便装的上班的我这次特意去买了套工作装,那
天早上早早起床去冲洗,头发梳理得很顺,还化了淡妆,高高兴兴地来到新公司报到。

我的老板很nice,不知道是不是习惯,他一直着便装上班,那天早上把还没能将badge
办好的我领上楼去。到了工作位置后又借用他的笔记本电脑早请各个账号,payroll的
,database access以及其它几个软件的账户申请。

好几个账户申请下来,急需解决的问题也依次解决下来,接下来上班的第一天就显得轻
松。没想到快一个星期不理我的表哥这次给我回了邮件:

\subsection{发送时间:2012-02-27 10:32AM;}
\label{sec-9-49-1}
收件人:me;

Subject: Re: jobs \& visits

XXXXX,

You are not my girlfriend.  I don't want you.

Do not call my cell phone.

Do not send me presents or e-mails.

On 2012-02-22 11:20, XXXXX wrote:

I bought a second hand queen size wood bed last sunday.  \$130 for bed and \$
30 for transportation. I feel like it a lot, 

I don't care about that.  It is not my life.

and hoping you will come to CA to see me soon. when you come, you can help 
me pick a new mattress, or even a ned bed if we want to. 

Absolutely ridiculous.

Do not harass me with these nonsense things.

Live your own life.

--

XXXX

虽然表哥邮件里的内容我并不喜欢,邮件里他没说一句让我高兴的话。而且他说的话我
并不信,因为之前与他相处时他的行为并不是这个意思。想到表哥说要我过自己的日子
,不要管他,我敏感的鼻子心思就嗅出些他似乎觉得他与我之间不平等不平衡,他似乎
觉得与我在一起他会拖累我的生活的味道来。想到我自己并不是这么想的,想要我们之
间还有解释清楚能够继续的余地,我依然很开心。更何况我到新岗位报到上班第一天,
上班一个半小时后发来邮件,这与新年春节联欢晚会上海外学子发来贺电有什么区别?
! 所以收到表哥的回信,相比之前他不回邮件、新年时把去年八月买的两磅好茶和新买
的巧克力寄出去又被打回来,我心里除了开心还是开心!
\section{看不见}
\label{sec-9-50}

去年六月,在我与表哥关系最僵最痛苦的时候,我搬回到了熟悉的前房东家,并养了两
只憨懵小鸡为我作伴。后来不到一个周,那只大的白色小鸡被猫吃掉了,只剩下一只孤
苦伶仃的;后来有机会我又买了六只三个周大的小土鸡,并一再改进养鸡的小笼,不幸
那些小鸡还是被猫吃得只剩下两只;后来看见出生只有两三天的小土鸡非常可爱,便把
别人买剩的四只小红鸡全抱回了家。七只小鸡经历过一次又一次与猫的撕杀,经过我一
次又一次地改进笼子,最后只剩下三只:第一批里两只中的小的小黄,第二批六只里的
一只白色的美女,和最后一批四只里的一只最小的小红。

小黄作为三只里的大姐大,也是历经艰辛才存活下来。第一次的猫鸡大战,她失去了姐
姐;第二次失去了四个妹妹;第三次她从猫的魔瓜下捡回一条命,她的一只翅膀被猫从
笼子外面折断叼走,只剩下那只翅膀的根部剩下了半截,鲜血淋漓。那天下次她几欲昏
厥,我拿剪刀将残骨剪断,用创可贴包扎了伤口,强灌了她些水抱在胳膊上三四个小时
,她才总算回过神来。

后来在我最终改造大一点儿笼子并在四周围上防护纸板后,这群小鸡数量最终稳定下来
,幸运地度过了寒冬,花枝招展地迎来了暖和的春夏。在小公司上班的那两个周,我没
时间照顾他们,便每周做好一大盆棕米饭请房东老板娘帮我照看喂养他们。

后来换到家门口上班后,我便开始能挤白天的时间亲自照顾他们。有一天下班后发现最
小的小红鸡瘫在笼子里不出来吃东西,感觉她像是病了。我便将她抱出来,放在两个姐
姐中间撒有玉米碎的地上,她也能傻愣愣地听着她姐姐们水泥地板上啄食的声音,却不
肯伸下嘴去。我抓一把玉米碎放在手心,摆她眼前,她也看不见,没有任何吃的意向。
我拜开它的嘴巴塞几颗进去,她也还是吃的,只是吃得很费力很不想动。我便不太勉强
她,只又灌了些水给她喝,便把她放回了笼子,心想着第二天我要帮她找些营养水一类
的给她救急。可第二天我下班后已经没有了她的踪影。房东家里有小孩,房东见它已死
去便在我下班之前已经帮我处理了。那总共十二只小鸡现存仅两只,其它所有小鸡都是
被猫吃掉了,只有小红鸡是眼睛看不见饿死的。

那天我打电话回到家关心一下妈妈的病情,也了解到姐姐们正是那天回到老家到爸爸坟
上为爸爸烧了些纸。去年年底爸爸离开的日子(是我上一份工作的最后一天)、爸爸离开
后新年里妈妈生病(是正月里爸爸生日那天发病),以及这只小鸡小红的离去一系列的巧
合让我觉得爸爸似乎在提醒我什么。小鸡是她们四只小红鸡里最小的一个,小鸡是因为
眼睛看不见不能进食死去的。接下来几个星期的时间里,我就有了一块心病,若小鸡的
离去真是爸爸有意想要提醒我什么,那我究竟是什么看不见?
\section{工作}
\label{sec-9-51}

我上班的第一天中午,我老板告诉我说,这个项目是三个月的,不知道我知不知道。我
有些紧张,便打电话到中介公司,agent说得很中肯,一如当初签offer letter之前他
说的,即使这个大公司的项目只有三个月(当初签合同时他说的是六个月),他们还有很
多的clients, 他们也会把我安排到其它公司上的项目上去,让我不用担心。这是一个
比较大的中介公司,我上网查过,每年、至少是去年公司有办多于500个名额的H1B工作
签证。所以agent这么说,我是放心的。

三月份,当我与老板的闲聊再次提到这个问题,我也又开始担心起来,便到写邮件给
agent,并到他们office同他们谈。除了这个agent, 还有公司大概是分部老板吧,三个
人一起谈的。观点同我工作第一天打电话给agent, agent说的一样,即使那种情况发生
,他们会给我其它公司的项目,会给办工作身份,要我不用担心。

那时的我大概也开始有点儿担心了,第二天要agent把前一天讨论的内容写在邮件里给
我发过来。他们并不迟疑,仿佛也不觉得会有什么,我要发他就发,我也就放心了。现
将往返邮件内容抄录如下。

From: me; Sent: Wednesday, March 07, 2012 11:39 AM To: agent;

Subject: employment status

Hi, XXXX, 

After having discussed with my manager, so far he has only about three month
's budget. I had not got this information from you before I reported to the 
client here at XXXX. Though I have called you asking about this information,
I still feel uncertain about my status. So I want to write to you so that I
can better understand my situation. 

\begin{enumerate}
\item in the telephone, you said usually XXXX(my company) will automatically
\end{enumerate}
extend another 3 months. I want to ask if my manager has no extra budget 
even at the end of this three month, what will happen next? Will I be able 
to work for any other clients from YYYY(agent’s company) at YYYY's 
assignment, or will I be dismissed from YYYY?

\begin{enumerate}
\item concerning about my visa status, my OPT will end soon, and I would
\end{enumerate}
request H1B application on April 2nd. will YYYY file my H1B application on 
2nd April, or will the company wait until the end of May?

My questions maybe naive and feel desperate, but XXXX(agent’s name), please
spare me some time to help explain these details to me so that I can either
feel relaxed or be prepared. 

发件人: agent; 发送时间: 2012年3月8日 12:44

收件人: me;

抄送: agent他自己;中介公司分部老板;

主题: RE: employment status

Hi XXXX(my name),

As discussed last evening, here is the plan for filing your H1B:

By 3/19(we will get the client letter initiated)

By 3/23 HR from YYYY(company name) will send H1B docs to you 

Within 2 weeks we get all the paperwork completed

Before 4/15 we will file your H1B.

Thanks.

\section{我的妈妈}
\label{sec-9-52}

妈妈一生顺遂,几乎没有经历任何的波折,除了那年爸爸得出血热生病妈妈说她被吓着
了。我小的时候同妈妈亲,但那时的我还是想不明白一件事,为什么一件事情妈妈可以
拿它同爸爸吵十几年?

外公外婆对妈妈众姊妹的教育大概是:舅舅是家里唯一的儿子,舅舅是一定会被安排留
守家中为他们老人养老的;那妈妈以及姨姨们就该逢年过节多孝敬外公外婆,也要平时
多贴补舅舅家的生活才是。爸爸生病时我还是个四岁半的小P孩,外公都不曾委屈我,
牵头牛来让我坐上费时间慢慢走,也不曾说要我随他走步行七八里路上镇上的医院。所
以我从来都不怀疑妈妈舅舅和姨姨们从外公外婆那里获得了饱满充足的爱,所以她们大
家也理所当然地对外公外婆的话言听计从,待舅舅家非常好,也受外公外婆影响,极其
宠溺表哥。

四姨接了外公的铁饭碗,对舅舅妈妈等众姊妹都非常好,这自不必说;五姨嫁得好,嫁
了有学问有本领的老公,相夫教子享受作太太的轻福不必工作,逢年过节礼尚往来待大
家都好也自不必说;我们家并非妈妈姊妹里过得很好的,但一如那些年二姐放假在家我
们就有去叔叔家吃鱼吃中饭的固定习惯,我上大学后,只要我放假在家,妈妈都会要我
打电话给舅舅,要他来我们家吃中饭,也早早地做晚饭好让舅舅吃罢能趁天黑前走到家
。一年里妈妈大概会为舅舅杀三四只鸡吃吧。而我们家是农村人家,过年前冬腊季节杀
了猪,给舅舅家送些肉去孝敬外公外婆,给城市上的四姨五姨送些去省她们花钱少买几
次,爸爸也是每年都会送的,因为他们也待我们家也好,不因为我们家多几个孩子而礼
轻情薄。大姨离世走得早,妈妈同舅舅四姨五姨们都走得非常亲近密切。中国人讲究“
礼尚往来”,这其中的冷暖寒凉大概的确值得体会吧。

回想当初妈妈退学的原因,我终于明白,以妈妈的自尊心,发生在爸爸身上的事情,也
如当年初中毕业那个暑假我没有把自己的心事告诉妈妈一样,妈妈大概也从来没有把那
件事情告诉过任何人。若是外公外婆或是四姨五姨中任何一个人知道,都有可能开导妈
妈,爸爸并非生性如此只是事出有因,婚姻也是一门理解包容的学问。妈妈就有可能被
劝,但妈妈没有告诉任何人。而从小到大被外公外婆宠溺大的妈妈,都没舍得被外公换
亲到别人家(舅舅比妈妈大两三岁,妈妈比三姨大,三姨后来成为替罪羊),妈妈又如何
受得了爸爸强加的那份委屈?
\section{我的妈妈(2)}
\label{sec-9-53}

妈妈待奶奶其实也不差,只是在奶奶只让我们家两个大的姐姐可以吃东西,两个小的不
可以的时候才会同奶奶顶几句,说些气话。在奶奶面前为我们小姊妹两个出足了气的妈
妈却没意识到,我们小姊妹三个始终认为妈妈偏心,爸爸随妈妈也变得偏心,我们三小
姊妹都盯着大姐不放,二姐会时不时地亲口对妈妈说她偏心大姐,三姐对这个家似乎感
情不太有感情,我虽然嘴巴上不再说,其实心里早已认定爸妈偏心这是铁一般的事实存
在。

我长大后了解到爸妈也时常感到挫败,妈妈也会暗自神伤,会念叨说怎么外公外婆那会
儿就把她和舅舅姨姨们教育得那么好,即使外公外婆走了他们姊妹几个也紧紧地团结在
一起,为什么轮到她的几个孩子就大的盯小的,小的盯大的没有点儿谦让精神,更别说
团结。

那年大姐家的小侄儿上大学了,我刚工作不久,大概为大姐家的孩子考上大学、二姐家
的孩子考上高中礼节性地表示了一下。那时爸妈已改为为三姐种地,他们没有收入,三
姐姐夫只每年交给爸妈一些口粮。我了解到妈妈想要为姐姐家的孩子上大学出两千块钱
,我便早早地做思想工作,劝爸妈说他们现在两个人都老了,做不动农活了,也没有任
何收入了,大可不必为小侄儿考上大学出什么礼钱,自家人也不必多礼,他们只要把钱
花在他们自己身上,好好享受几年晚年,照顾保养好他们的身体就最好不过了。妈妈当
我面电话里答应了,但改天我与侄儿聊天就知道爸妈还是背着我出了两千块,家里所有
姊妹都知道就只有我事后才知道。

二姐上到中专毕业,中专是三年;我虽然读书来到国外,但基本从上大学起我的学费都
算借的,我以后要还的;我的三年高中同二姐的三年中专有什么区别?我的高中还没有
住校,吃住在大姐家,也给爸妈省了不少钱,我连交到好朋友的机会都没有。而为什么
我长大后要自己还学费,要贴补爸妈养老,结果我上大学后爸妈辛劳多年,好不容易晚
年手上攒了三四万块钱却被大姐拿走三万买房子(虽然只说是借的,说是给爸妈买了一
个房间),我寄回去点钱爸妈还要心疼大姐?我又该如何平衡?爸妈劳作一生非常辛苦
,我多么希望至少爸妈挣的钱能够花在他们自己身上!
\section{我的妈妈(3)}
\label{sec-9-54}

只是那年暑假领教了妈妈的哭,我只嘴上不对爸妈说,还是会偶尔会电话里对二姐述一
述我的委屈的。不知道二姐怎么想的,后来我的话就被二姐一股脑地当成她的想法回家
说给妈妈听了。我再打电话回去妈妈就又会向我哭述,我再反过来安慰妈妈。

那年暑假,妈妈哭着对我说,十个指头伸出来还有长短,一娘养九子九子各不同。她同
爸爸又如何能够保证做到完全一碗水端平,也只能是尽力罢了。妈妈也有她的委屈和苦
衷。她说大姐从生下来就体质弱,姐姐还没满月的时候就被爸爸抱出去扎针(不知道是
扎干针,还是注射药水,土话叫“扎针”),满月了还得扎,连扎了四十五天才稍微有
点儿好转,算是勉强捡了条命活了下来。哪个做娘亲的不会对体弱的孩子多点儿心疼呢
?何况姐姐是他们的第一个孩子。妈妈说小时候不只是我同三姐辛苦,大姐二姐小时候
也同样辛苦。那个时候还在农业合作社劳动挣工分,爸爸挣耕田犁田的工分,妈妈挣拉
秧插秧收割水稻小麦的工分。大人们要劳动,我和三姐便是两个姐姐看大的。大姐上小
学的时候妈妈要她把三姐捆在背上背着去上学上课;二姐小时候妈妈怕她乱跑掉池塘水
里了,就把她肚子上系根绳子拴在老土屋后面的枣树上,结头打在背后枣树那一侧,年
岁尚小的姐姐小便大便弄得到处都是。结果姐姐很聪明,慢慢转绳圈,把结头转到面前
来,自个儿解开跑掉了,还好没出什么事儿。

可能小时候那年暑假的我还是觉得妈妈过于心疼大姐吧。比如大姐五千块钱的户口妈妈
也是不打任何折扣一手拿下。而那时的三姐与我都还是农村户口。但儿时的我们是无法
体会到大姐与姐夫一个城市户口一个农村户口的牵扯,只会觉得姐姐的户口来得太不费
吹灰之力了。后来我体会到爸妈待大姐姐夫好对他们老年人来说也是一种精神需要和安
慰,我虽然能够提供一点儿小钱,但我是很难陪伴他们,他们还是只能依赖大姐和姐夫
。后来我学会了自我安慰,也能够把事情看开,爸妈偏不偏心的话我这里就再也没有了
。
\section{我的妈妈(4)}
\label{sec-9-55}

妈妈之前在三姐家里玩过一个星期。姐姐也在县城买了房和家具,日子勉强过得不错。
妈妈电话里对我说,姐夫家的妹妹买房就买在姐姐家斜对门,装修房子也都在姐姐家吃
饭。妈妈说看着家里姐姐一个人上班下班、忙里忙外,洗衣做饭收拾碗筷,姐夫连鞋子
脏了都还得要姐姐帮他擦,妈妈说姐姐过得太辛苦。我心里感慨大概姐姐同我一样都是
小时候没有得到足够爱的孩子,大概姐姐心里同我一样没有安全感吧,只有通过辛苦劳
作和对家庭的付出才能锁定我们想要的幸福。我对妈妈说我们姐妹里现在就三姐过得稍
差点儿,要姐姐们什么事都不要同三姐计较,农村老家里也没什么就一堆破烂,三姐看
中什么就给她什么,绝不计较。

妈妈也问我索要过“特赦令”。那天打电话回去妈妈哭得一把鼻涕一把泪,哭得那叫一
个悲怆! 听她慢慢说才知道舅舅家的表哥问二姐借五千块钱,二姐便借了;事后姐夫同
二姐一点拨,二姐便也真认为表哥借出去的钱是有借无还的,便找个借口追着表哥把钱
又要了回来(替表哥叹一下,当年的初恋情人便是后来逼债的主儿!) 妈妈知道这件事后
气了好几天,气得心里青酸,觉得二姐怎么就有钱了也忘了本一样,舅舅当年待她不好
么,表哥就借了区区五千块钱犯得着他们这么逼债吗?人家表哥又不是还不起她们! 

听妈妈这么说,长大后的我当然懂得如何安慰妈妈,我对妈妈说,让她放心,我绝对不
会像二姐那样子的,舅舅走得早,没花到我半分钱,表哥在我眼里就同三个姐姐亲姊妹
一样,我绝不亏待他半分半毛,表哥从我这里只有赚钱的份,绝不会让他赔出点儿什么
! 现在我的精力在爸妈身上,等爸妈百年归山之后就自然在表哥姐姐们身上,我绝对不
会不认亲不认表哥的。

妈妈电话里哭了半天想听的不就是这话吗?她自然就高兴了。她高兴了我也就免不了要
损她几句,“人家表哥娶了个家里有钱的媳妇,人家丈母娘家里有两三套房子,人家老
婆是家里的独生女,人家表哥到时候过得会比我们姊妹四个哪个差?我没准儿哪天就回
国了,连个男朋友家什么的都没有,过得还不如表哥呢,你也不心疼心疼我,心疼心疼
三姐,成天里把你那宝贝侄儿看得比你亲身骨肉还要亲还要重!”听我这么说,妈妈自
己也忍不住开心极了,前面特赦令不特赦令的,她大概也就早忘了。
\section{爱情故事}
\label{sec-9-56}

我原本就觉得二外公和二舅走后的二外婆与二外公之间会是一段凄美的爱情故事,妈妈
偏说那时的二外婆已经29岁了,生过三个孩子,十几岁的她就嫁过来,想那时再操心个
人家是不容易的。妈妈说得也是现实。

但我对妈妈的说法也有本能的排斥。多少年后的大舅母能够领着一个五六岁的小孩嫁给
大舅,二十九岁的二外婆领着十几岁的大舅和不到十岁的小姨(依稀记得二舅走时十岁
左右),就一定嫁不出去吗?地主婆的身份就一定会阻止个人的幸福吗?那时29岁的二
外婆就不能嫁个家庭条件可能稍差一点儿的贫穷吗?

后来来到美国我同舅舅的聊天时问及这个问题,我才知道,二外婆与二外公之间还真是
一段凄美爱情。舅舅说二外婆不是嫁不出去,而是心里仍有牵挂不愿意随便嫁。当年为
避风头二外公领着舅舅跑了,可留在家里的二外婆何尝不是做梦都想着他们能够早日回
来?二外公走后二外婆与舅舅小姨一日三餐是怎么挣来的我不知道,但我想或许二外婆
也是幸福的。借用琼瑶阿姨<还珠格格>里的一句话,“否则,生命就像一口枯井,了无
生趣!”

舅舅说后来二外婆等到了六七十岁,终于等到有一天她朝思暮想的儿子回到家乡,找到
了她。找到妈妈的舅舅告诉二外婆,他随二外公离开后,流连辗转,最终到了台湾。在
台湾二外公娶了一普通女子安了新家,舅舅下面新添有弟妹。舅舅在台湾上了学,后来
来到美国读书工作,在美国安定下来。舅舅说他把二外婆接来美国时,将台湾安排成必
须经过的中转站,让二外婆与二外公临终前能再见上一面。

那个等了一辈子,盼了一辈子,忍了一辈子,也怨了一辈子的二外婆,见到她那一别将
近四十年的夫君,能否相逢一笑泯恩仇?关于二外婆、关于她与二外公的爱情故事,很
多细节我都不大记得无法考证了,但这个凄美的故事却存在我心里。
\section{慎重考虑这段要不要??????????}
\label{sec-9-57}
发信人: closet (closet), 信区: Dreamer
标  题: Re: 成长的故事 -- 我和舅舅
发信站: BBS 未名空间站 (Wed May 30 20:31:21 2012, 美东)

楼主你知不知道什么叫刻舟求剑啊
你表哥可能是对你动过一点点心,你舅舅可能是对这一点默许过,但是时过境迁,你现
在还发花痴实在是刻舟求剑的典型范例啊。
1,    你难道不知道吗人的感情感觉从来都不是一成不变的。
不要看什么宿命的韩剧了,看看比较严肃现实的国产片吧,中国式的离婚什么的。曾经
最纯真最深遂的爱情,也很容易随着岁月而消逝,就是自由恋爱又结婚多年的夫妻,也
常常有越来越不能忍受对方的缺点,越来越没有激情,因为责任孩子等等而继续凑合的
例子。爱情哪有那么神秘注定,那么保质期长,那么经得起考验和折腾的啊?更何况你
们连爱都没有真正开始过。
表哥对你那一点点动心,就算曾经有过,也早就被你的pushy,冲动,幼稚 ,自我为中
心,以及到后来的,极端自恋,极端自作多情,极端臆想幻想等等冲散得干干净净了。
你还在回忆那些细节,还在幻想绝情得不能再绝情的回信里面隐含什么心意,是因为你
在自己刻舟求剑的梦里面不肯醒来。
那一点点动心,比起很多人很久很深很纯的感情,算得了什么啊,那么多又深又纯的感
情,都未必经得起时过境迁的考验,未必经得起性格不合的折腾,何况表哥对你那一点
点若有若无的动心呢?
2,    你难道体会不了讨厌一个人的感受吗?
有过追求者吗,有没有外在或内在条件让你觉得非常讨厌非常不能接受的?比如说你硕
士时候的老板,你语焉不详,大家也看不出来是怎么回事。大概到后来你感觉受伤然后
很讨厌他的虚伪吧,总之你迷恋过他,对他了解比较真实以后,对他根本上变得不认可
了对吧。那你应该能想象,动心甚至迷恋,都可以因为了解而失去感觉,甚至心生厌恶
,对吧。
如果还没正式开始你已经失去感觉甚至心生厌恶了,对方还非常自恋地以为你很迷恋他
,不停地骚扰你的生活,不请自来,甚至想要跟你结婚,甚至在你三番两次严肃地拒绝
后仍然,自顾自地幻想着跟你结婚,不考虑你的感受,行事说话完全是你瞧不起的做风
,像个牛皮糖一样甩不掉,你会对对方再生渴望之情吗?还是会厌恶地躲他屏蔽他?
3,    他会喜欢你什么?舅舅曾欣赏你什么?
爱情从来不是无缘无故的,你喜欢他的原因很多,他帅,他成熟,他是公民,结婚你就
可以拿绿卡,更重要的是他宠过你一点点,而最重要的是,你迷恋自己被喜欢被宠的感
觉,你骨子里面自卑,你想象他和舅舅舅妈都非常想要你,这个想象让你自信和快乐,
你甚至到后来自欺欺人到了别人拼命躲你赶你你还以为是有心意,被报警赶走后还想象
着结婚的地步。有些人可能奇怪你怎么能很难被那么坚决的拒绝打倒,其实本质上是因
为你根本就是在跟自己的想象谈恋爱的而已。总之,你迷恋他的原因很多,最重要的是
你自恋的需要。
而你想过没有他会喜欢你什么?
你想过没有,他爸爸当年为什么跟你成为忘年交,为什么帮助你,曾经他欣赏过你什么
?你对只见过一面的美国表舅提出让他供你读书的时候,他可能觉得你跟国内很多攀亲
戚的人一样很功利。而你自己申请到奖学金来美国读书的时候,舅舅这样的知识分子会
很欣赏你“靠自己努力”这一点。那时候,你年轻,会读书,在艰苦的农村环境里面靠
自己的努力一步一步走出来,是个肯努力的靠谱的人。另外你是华裔,对舅舅一家来说
,应该也愿意有个华裔的媳妇。表哥对你的好感,最多也就源于此。
但是好感跟喜欢甚至爱的距离,就好比好奇地想去敦煌看看和一门心思真的决定去敦煌
生活一辈子的距离。为什么后来舅舅会对你那么狠,当你很push地要跟表哥交往,当你
有明确的绿卡的因素,当你对礼貌的回绝和躲避视而不见一如既往,当你开始死缠烂打
,投怀送抱;舅舅发现原来看错了人,发现你不是他心目中靠自己努力的自尊,能干,
独立,令人钦佩的孩子,你为了绿卡,为了留在美国,竟然……可以想象舅舅有多失望
。就算曾经乐于接受你跟他儿子在一起,后来也一定非常厌烦你的纠缠。他是不是真的
有一点点世俗,是否曾真的希望你用礼物报答他,我不知道,但是他一定烦透了你的
aggressive的纠缠,讨厌透了为了绿卡结婚的人。
你再想想,你舅舅曾经最欣赏你的一点化为云烟之后,你还有什么值得表哥喜欢和留
恋的?
你说过你不漂亮,你也并不年轻了,一开始有点好感的远房表妹,随着接触了解深入,
表哥发现你脾气不好,对他妈妈很不尊重,(文中看得出来舅妈从头到尾没有刻薄怠慢
过你,你对她从来就没有体谅和尊重过),表哥发现你自我为中心,发现你是个想找个
大叔宠孩子一样宠她的女人,发现你很干扰他的工作和生活却不自知,更重要的是发现
跟你无法交流,你活在自己的世界里面,幻想着别人怎么对你好,鸡毛蒜皮地计较着跟
别人相处时候的一切得失,你根本就不听别人(包括表哥)说什么,你也完全听不懂别
人(包括表哥)说什么,你想象,你臆测,你无法了解别人的感受和需要,你也不在乎
别人的感受和需要,
你冲动,幼稚,你做人的原则就是随心所欲,你一开始犹豫不决又怕失去暧昧的机会,
你暧昧期间又不排除结识别人,你没找不到更好的,想要抓住表哥的时候,你开始投怀
送抱死缠烂打,所有这一切,怎么可能是一个45岁的成熟男人欣赏和喜欢的,何况你并
不漂亮,也不年轻。

那一点点好感什么时候化为乌有的我不知道,很明显写第一封英文据信的时候表哥觉得
你的行为令他无法喜欢,你也不能跟他沟通,他可以把你当个亲戚,作为妹妹,按照亲
戚的礼貌接待你。而你还闹到他家,让他们报了警,伺候你而你还幻想着结婚拿绿卡再
去看病危的父亲,这时候表哥对你,真的是很无语地厌烦了,他一定想,怎么还有这样
的人。就好象你我想,天下怎么还有凤姐这样不自知的人一样。

你再跟他讲你买的床,要他来看你,还有新床什么的,表哥一定觉得怎么有人会这么执
着地“性骚扰”他,他一定庆幸当初没有跟你更进一步,一定纳闷你为什么这么奇怪,
一定觉得你要么是想绿卡想疯了,要么是想男人想疯了,疯到无论怎么拒绝都赶不走要
继续骚扰他的地步了。

很多看客都意识到了,你有心理问题,也许你不是想绿卡想疯了,不是想男人想疯了,
你是活在自己想象的世界里,完全没有能力感知别人的真实感受的人。这也是你为什么
会对硕士导师自作多情,为什么会多次成为舆论的矛头指向,为什么会被说成小三等等
的原因。你需要专业的帮助,你需要面对自己的自卑,然后找回自己真正的自信,而不
是靠想象的生活来掩盖自己的自卑,靠想象和执着的骚扰去等待金石为开。
不管怎么说,就算表哥以前对你有N条动心的理由,现在也已经荡然无存了,你的行为
和思维方式没有一条能让他喜欢的,在他心中你成了非常不懂事的妹妹,继而是近乎疯
狂的偏执骚扰者。醒醒吧,表哥还能喜欢你什么?醒醒吧,舅舅为什么不再欣赏你了?

4,    你知不知道什么样是对别人好?
你写了很多别人对你如何如何的细节,非常敏感,非常计较。你讲了一个你帮助退学小
男生的事情,除此之外,你有没有用你超级敏锐的感觉去观察一下别人的需要,去出于
发自内心的关心去以别人享受的方式对别人好过(礼貌上的面子上的送礼不算)?你扪心
自问,你能不能学会理解别人的立场,能不能学会关心别人的需要(包括你姐姐,表姐
等众人),能不能学会考虑别人的感受,能不能首先不再为了安慰自己而把自己的想象
强加给别人?你说朋友说你善良,但是除了儿时,你真的没什么真的好朋友,你讲的故
事全是“升米恩,斗米仇”的故事,都是各种亲人亲戚为了考虑付出越多,你越觉得委
屈,愤怒,继而翻脸的故事,就是对自己父母,你都仍然是冷血(跟大多数别人比较)
。你觉得舅舅很狠,很冷血,你会对你国内一众亲戚的表侄,表外甥等像你舅舅当年频
繁帮你那样去帮他们,不指望任何回报和感谢吗?

可能这不怪你,你成长的环境有很多宠你的人,你成长的过程也有很多痛苦迷失的经历
,你形成了这种沉迷于自己的世界,无法理解别人的立场,关心别人的需要的性格。我
回这么长的帖子,真的是希望你从这种沉迷中走出来。

如果你真的不是为了绿卡而纠缠表哥,如果你真的喜欢他,欣赏他,你现在最应该做的
是,问问自己,自己怎么做给对方最大的快乐。显然,任何纠缠和骚扰都不会给他任何
快乐(想象你如果被讨厌的人长期骚扰的痛苦)。我希望你能想清楚,你们真的不适合
,舅舅和表哥也确实没有对不起你,就算表哥动心过,舅舅默许过,他们没有做任何伤
害你的事情,是你自己的性格让自己爱上想象中恋爱的感觉而把自己逼成一个失心疯的
花痴。他们不是不主动不拒绝不负责,是了解之后开始礼貌拒绝到决绝拒绝,可你都听
不懂。

如果你从来不真正知道对方要什么,不真正明白对方怎么想,那你其实从来没爱过他。
你只是需要他。学会爱,不容易。

5,    最后,找回你自己。
如果是你的性格,或者心理问题导致你把自己逼成一个失心疯的死缠烂打的花痴,而不
是想绿卡想男人想疯了的失心疯,我希望你早日醒过来,早日学会爱,学会关怀,学会
真正的自信,学会体会别人的感受,学会优雅从容地生活,最关键的是,做回那个靠~
自~己~努力,独立生活学业事业有成的人,那时候你会对这一切疯狂的过去莞尔一笑
,那时候你会学会平等地爱,真诚地关怀,你会有自己的很好的生活。

你真正醒过来想明白的时候,我希望你能给舅舅和表哥写信道歉,为你执迷不悟的无知
对他们带来的困扰道歉,为舅舅当年给你的帮助道谢,让他们看到一个重新自尊自立的
你,成熟了的你。那时候我想他们也会如释重负地想,我们也错怪了她,她也不是心术
不正想绿卡或想男人想疯了的人,她就是当年幼稚书呆子气,我们一时的动心和暧昧多
少害了她,还好,她能走出来,仍然是个优雅从容让人欣赏的亲戚妹妹。

希望我这不是一个肥皂泡一样的愿望。祝福你。你太另类了,我真心希望这对你有帮助
。我写下来的,就是你看不见的东西,也许是你在天国的爸爸希望有人能耐心看完你的
故事,耐心写点对你有帮助的东西吧。
\section{表哥家人}
\label{sec-9-58}

2010年12月那几天与表哥的相处,临走当天与表哥的那场告别,让我清楚地知道我喜欢
表哥。可也正是这打心底里对表哥的认同认定也激起了我对表哥家庭本能的自我保护式
的反感和排斥,因为舅舅的心机和待人的寒凉,舅舅同舅母的察言阅色、见风使舵,以
及舅舅舅母的难得伺候。

我以为我可以躲开这场劫难,却原来也只能是瞎折腾一场(在身边有一个男孩喜欢过一
段时间,但后来喜欢、装不下去了)。心里的那份喜欢和认定是没法改变的,而我即使
装作喜欢别人,我也真是装得了一时,装不了一世。而以我多年来对自己内心的执着,
我又如何能够过其它人的生活,无视自己内心感受?

对表哥个人,我非常满意、没有丝毫怨言,但要我去面对表哥的家人,我总显得犹豫。
我多少次地想过,多么希望我是一武林高手,能够把表哥骗走,隐居深山老林之中,与
表哥的家人再无瓜葛!但那是不可能的。表哥与我,同属于家庭亲人,就算我有本领能
把他人骗走,将他从他家人中隔离出来,我骗得了一时,骗不了一世,他绝不会快乐,
我也不会幸福。舅舅这个媒人的策划,循序渐进让表哥以package的形式在我面前逞现
,这个package包括了舅舅、舅母的养老,还包括了对小表哥的救济。我不接受表哥便
罢,我若接受表哥,便需要接受整个package。舅舅的心机,我能不害怕吗?

舅舅是极善察言阅色的,舅舅也是见风使舵的,舅舅是心机深遂的,舅舅也是冷酷无情
的。我谈恋爱的对象是表哥,我心里没有安全感,感到害怕的却始终是舅舅。

我想过,若最终真能如我所愿,我嫁给了表哥。对于他们家人,我最希望发生的事情便
是我们大家庭、小家庭的日子都只会是越过越好,我希望小表哥那走掉、没了踪影的老
婆能自已回来,我们就当她不曾出走过,那个家也还算完整;若是不能回来,也希望小
表哥能够再娶一个,人还是要有归属感的,有了他们的小家庭,日子才有奔头,我和表
哥便也能过好我们自己的小日子。

可是万一,那曾经的老婆也没回来,小表哥也没能再娶,而如果除了舅舅舅母要从我这
里混口饭吃,他一日三餐还要同我们挤在一起,那我们的日子过得该有多纠心?!

那段时间,我一直在想,我究竟什么看不见。我也知道,我最可能看不见的便是关于表
哥和他的家庭。
\section{我的舅舅}
\label{sec-9-59}

其实仔细去想、去总结舅舅那个糟老头,他也没那么坏。

刚来美国的第一年,大舅母说她告诉舅舅我就在旁边学校,舅舅却没来找我,这让意识
到喜欢表哥的自己纠结了很久,因为那个我很把他当回事儿的舅舅待我居然是那般冷血
无情,这与我家的温暖亲情形成了截然对比。因为那一年我自己生活的痛与怨,我对舅
舅那件事耿耿于怀了很久。

可是仔细想想,我找到舅舅后,舅舅到我的住处看了看,问我有没有什么缺少的东西,
我答没有,一年了该置的东西早就置办好了。舅舅便劝我买辆车。有机会去到舅舅家后
,我觉得舅舅家是“牢”,不知道住在他家的人该如何忍受,我自己看着反正是觉得住
在里面半点儿意思也没有。

舅舅家的大门是用厚的塑料皮纸包裹着的;当初橱房里的四角餐桌也是铺了厚厚的塑料
皮纸,连同座位椅子靠背沙发坐面上全都是,坐上去吱吱呀呀的响;当年表哥的房间是
空着的,电脑桌上也盖了层塑料皮纸;我在舅舅家里转了一圈,就发现我来到的是一个
“塑料皮纸”的世界,家里仅有的几个柜子台面,平铺着被子的床上,洗手间的台面上
全都是!当时的我就禁不住想,生活在这栋房子里的人得有多累啊?

后来想想,不知道之前的舅舅家是不是就是这样;又或许,舅舅正是充分利用了这一年
的时间把这个家、这栋房子打造成如此模样,好让我觉得,他同舅母这一代人把这房子
爱惜得非常好,所以再住一两代人是没有任何问题的。

我会这么想,也是因为后来不早不晚毕业前最后一个学期,我遇见了这个没有结婚的表
哥,从来不曾在舅舅家餐桌上吃过饭的我也被舅舅邀请到家里吃了三餐饭,而在这个表
哥归来之前,舅舅已经说了很多让我着噎、离谱的话。
\section{我的舅舅(2)}
\label{sec-9-60}

舅舅是学校里的教授,我从来都没有怀疑过他会教不好、教不透知识点,因为几年来,
我亲自领教过他对我审时度势、不放过一切机会、“诲人不倦”的“教导”。

来到舅舅家,舅舅先给我介绍他们家的狗,介绍的内容自然会是它是如何如何的忠心,
什么时候什么情况下他们离家多少天,这狗一个人在家里,主人没喂它它既不吃也不拉
,三天过去了,他们家的狗表现得就像仅仅只是一小会儿几个小时,主人拿食物给它吃
它才吃。在我眼里,舅舅把他们家的狗描述得过于空灵。

每当舅舅耍嘴皮、超出客观事实描述某种东西事情的时候,总是激起我本能地反感,一
如我自己对小姨的音容笑貌头脑里仍有印象,舅舅却一定要把她说得多有才,说得天花
乱坠招人厌。

接下来舅舅再循循善诱地说,他会一辈子都留着那条狗,只要他有口饭吃,就绝不会让
那狗饿着时。舅舅说得那么情真意切,“循循善诱”,我就多少听出点儿画外音来。后
来我还是学生时有个还不错的准男朋友前来探望,被房东说我穷亲威、累坠多,还有个
美国舅舅要我养老,那男生被活活吓跑了,这是后话。

舅舅说过:他和舅母同国内的父母不一样,国内的父母会干涉子女私生活,他们是有文
化受过教育的人,决不会干那种事情。舅舅说这话的当时,我在同他聊说希望我将来能
让自己的爸妈过上幸福晚年,舅舅的这话前后语境不搭,让我很噎。

现在我把系列观点罗放到一起。那年我问及小表哥的时候,舅舅很小心地回答说,他们
结婚不久,还没小孩,在我所在的小镇租房子住,只周末的时候回家来吃饭。是否,舅
舅能允许他每个新婚、结婚不久的儿子都在外面租房子住?我心里是有小算盘的,先窃
喜一下。
\section{我的舅舅(3)}
\label{sec-9-61}

08年夏天在表姐家,借我想学包饺子的事,舅舅鼓励我学的原因却是:这样改天我就可
以不用舅母动手,亲自做热气腾腾的饺子让舅母吃。而到那时,我给舅母买过四瓶(也
可能是两瓶,这个真记不清了)保健品,舅母我见都没见过,她又算我什么人,我干嘛
要亲自做给她吃?

09年夏天,我作全职学生时最后一次买礼物送舅舅,那次舅舅说,我那几年还只是学生
,买那点儿礼物不算什么,以后我工作了,更该加倍地买。当时舅舅说那话,很让我触
目惊心。在那暗无天日、备受奴役、看不到希望的学生时代,我几乎没有为自己的亲生
父母孝敬过任何礼物,却因为舅舅建议我买辆车,因为舅舅迟迟不愿意归还车钥匙给我
,我为了表示感谢,待舅舅比待自己的父母都好;既然学生时我那么穷,舅舅都逼我买
礼物了,那为什么我工作后,离开那个地方,舅舅还要“耳提面命”:“以后更该加倍
地买”?!!!

10年12月,我有事回学校,临行前我打电话给舅舅,问及行程及我回来该住哪里,舅舅
要我住他家。在校园里呆了二十多年的我终于工作了,我开心,便真按舅舅的要求给舅
舅买了加了倍的礼物。在表哥office里,我把礼物翻了倍的话对舅舅说出来,糟老头却
一脸的不屑,那种花了钱却撞了一鼻子灰的感觉,连同那次回去舅舅、舅母所表现出来
的谄媚,舅舅的真实意图,想不往表哥头上猜都难!

生活很会同我们开玩笑。当初那个我把他当亲人恩人看待想着可能要为他养老的舅舅,
却因为后来我对他的了解,在我认识到自己喜欢表哥、很认定表哥时,却造成了我极度
纠结、很是担心我与这糟老头处不来,会吵架会大干起来。

而那个满以为以他在我十八岁那年对我的鼓励、送的巧克力,和几年来生活上的一些关
心,就可以诱导我知恩图报,因感恩而嫁给表哥,成全大家美愿的舅舅,却不知道,我
的爱情、婚姻怎么可能因为舅舅的一点恩惠而乱来受委屈?后来我与表哥真正相好,表
哥成为我认定的人,而这个给过我恩惠的舅舅却成为让我纠结、影响我做决定的最大障
碍。
\section{美丽的邂逅}
\label{sec-9-62}

就在我纠结在我究竟是什么看不见,纠结在与表哥及其家人一起生活,我们可能会出现
什么样的问题的时候,我与表哥,相隔在千里之外,却在那几个我有着心病的某个周末
,没有早一步,也没有晚一步,同一时刻相遇在网络空间。现在再回想这次奇遇,仍控
制不住内心的激动,开心极了。

从注意到表哥的profile改成了英文名字中夹着中文的,从自己大公司的工作稍有眉目
,我便常常在表哥的网络空间流连,总想看看表哥那天里还好吗,有没有什么变化。

我的工作很好,我很满意;我的亲人也很好,妈妈身体健康,我同姐姐们也最终做到相
互体谅,没有任何交流障碍;我自己这边的生活也很好。我唯一可能看不见的,只可能
是与表哥的关系,以及嫁给表哥后,与表哥家人相处潜在的看不见和危险。

在对表哥及其家人,舅舅、舅母以及小表哥所进行的一系列的思考里,大概就有那么个
周末,我的脑袋稀里糊涂,又或者是因为表哥至今不肯从语言上真正承认让我作他女朋
友,自顾自的呼呼大睡了两天。

那天凌晨,因为前一天睡得太多,晚上怎么也睡不着,便又看书、上网上到百无聊奈,
便也习惯性地进入到了表哥的空间。咦,表哥的message note居然变了,他目前不接受
邮件和介绍;再看看他的联系人,也已由之前的21个变成了-1个。

虽然平时我进表哥的空间,总期待着有什么变化;可这次真正有点儿什么风吹草动了,
我却又显得不太敢相信自己的眼睛。

表哥没有加我,所以我无法直接进入他的空间;我每次都是搜索表哥的名字加限定语,
在搜索结果里找到那个熟悉的头像再点击进入的,所以也就不存在firefox保存了最后
一次登录网页的可能;我决定刷新再看一遍。于是刷新之前,我把这注意的两点变化又
看了一遍;然后,小心地点击刷新屏幕,结果变了!表哥的message note回复到了往昔
的样子,而他的联系人也由-1个变成了20个。我进他空间的短短几秒钟,表哥居然也
在线,而且更新了profile!  而此时的我也开始后悔,刚刚应该把表哥-1个联系人的
profile存下来!

午夜的房间里万籁俱静,我听得见自己的心砰砰直跳,满脸发热,跳下床来照照镜子,
俨然变成了两个大红苹果。
\section{美丽的邂逅(2)}
\label{sec-9-63}

回想自己这两天的表现,周六我居然忘了上线,周日也是到周一凌晨才上的线,我太疏
忽了。

表哥为我做过很多事,而我因为幼稚因为不太懂得该如何去爱,几乎没有为他做过任何
事。我只劝过表哥一次,希望他不要吃罢晚饭六七点钟七点左右就休息了,要表哥遵循
人的代谢规律,晚上十点钟休息,早上六点起床就很好,晚上七晚休息,凌晨两点起就
不利于身体健康。我想的当然是,到时表哥同我一起生活,我当然希望表哥到时能多陪
我几年,可别早早地丢下我一个人先逃跑了,留下我孤零零的一个在这世上又有什么意
思?

想到这里,我便又赶快从自己账户里登录出来,再重新登录一遍,搜索表哥的空间,点
击进入,如果表哥还在,他就有可能能意识到同一时刻我的存在,意识到我们的网络奇
遇。

我的小脑袋瓜有时也会稍微多想一点儿,比如,当时的我就想到,我刚刚就真的遇见了
表哥吗?表哥是学计算机的,又或者,我没有遇见表哥,我遇见的是一段美丽的程序。
我的登录、我在表哥空间的浏览,引发了基因突变,触发了一段程序的执行。而这段表
哥精心编写的程序执行的结果便是:以上两条结果的改变。这也不是完全没有可能啊?

我对自己有这样的创意想法把自己佩服得五体投地,我简单就是一个genius,不,表哥
简直就是一个genious! 不管我们谁是谁不是genius,还是我们两个都不是genius,我
心里开心极了。若我碰见的是表哥,活该我感动!若我碰见的是一段程序,那表哥不更
是有心人吗?

我知道,我与表哥的激情早已过去,我的右手手背上也早已没有了那一瓣拇指的温热。
但我清楚地知道那时我真真切切地感动过。逝去的感动还刻在心底,新的感动还在继续
。那天晚上,一如当年那激情飞扬的一两个月,我又过电影般重温了一遍与表哥相处的
往昔岁月,良久,带着浅浅的笑容甜蜜入睡。

从那次网络相遇以后,为了延缓表哥的衰老,为了我自己将来的幸福,我都记着早上起
床后或是早上上班后先去表哥空间转一圈;下午下班前也进去逛一圈;通常晚上我也会
上网,那临睡前,我还是会到表哥空间再逛一圈,让他放心。而我心里,同表哥之间的
亲密早已无以言表。
\section{要不要????????????}
\label{sec-9-64}
你敲了这么多字实在不容易。就算我用五笔照抄你的这个贴子,我也得抄半个小时。谢
谢你!

我写自己亲人亲情关系时,因为有多年的沉淀,写得相对轻巧;可是每次写到表哥及其
家人,就写得非常辛苦,怕词不达意,造成不必要的误会。所以,今天也写得很辛苦。

你的回复里,有些观点我赞同,有些我并不认同。我不曾期待你所代表的读者群能够认
同我的观点,能跟我的贴子到现在,所以读到你的回复我很意外。鉴于你很真诚地花时
间、精力列出了以下这么多的观点、期待,很真诚地想要帮助我,我会认真考虑该如何
回复你,明天或最迟后天,一定把我的回复贴上来。

谢谢你的好意!Best wishes!
\section{要不要2 ????????????}
\label{sec-9-65}
不知道你是不是在优渥的环境下长大,我猜想你可能是孔雀女/男吧,你是舅舅派来劝
降的吗,你还真是舅舅古墓派的典型代表呢,世俗,没有灵魂,无法勾通,虽然我不怀
疑你的善良。或许真是因了自己的敏感,也因为你的心理优势,通篇读来,我读到的其
实都是“刺”。挑我刺也好,没有刺,玫瑰或许也不会那么漂亮。“发花痴”不过是我
自嘲自己的一种表达方式,既便你是为了劝我,其实也大可不必劝得这么高高在上,趾
高气昂。

1,    人的感情是会变的,但即使感情已经逝去。我愿意、想要把自己的感情经历写
下来也没什么不可以,更何况,我是在做一件有意义的事。我觉得你世俗,是因为从这
里我看出你没有信仰。不知道你恋爱、结婚没有,找一个工作(阶层)、学历、身高体
重长相,家庭环境、物质条件相当门当户对、看着不讨厌的人嫁了,在婚姻中磕磕碰碰
去磨合,合则幸,不合则为了小孩将就一辈子。这是以你为代表的广大小市民婚姻的典
型模式。或许有一天我也不得不走你这条路,但人只有一辈子,在我还在为自己的爱情
争取出路、还没有沦落到这一步之前,我想为自己好好活,找一个真正相爱的人开心到
老。在你眼里,我们连爱都没有真正开始过,世俗如你又如何能够体会,我们原本就是
相爱着的,只是表哥言语表达欠缺?

2,    你的话说得真重。难道你从小到大就没有讨厌过一个人吗,你又何苦把别人想
得如此不堪?我过去的感情经历、与过去老板之间我是不是自作多情,我想我已经表达
得很清楚了,我不愧对任何人;我是写故事的人,你作为读者是如何理解看待我的经历
,那是你的事,能够认同多少、理解多少、理解到什么层面,也要看你的修行。这个问
题我实在没有兴趣深入探讨。

3,    我认同你,爱情从来不是无缘无故的。表哥喜欢我的特质、我欣赏的表哥的特
质,我心里很清楚。舅舅欣赏的我的优点我也很明白。其实在你试图帮我分析我与舅舅
、表哥的关系的时候,或许因为我的文字表达能力有限,或是该呈现的内容尚未能完整
地表达出来,所以我与表哥之间你看见的便是绿卡,看见的就只有所谓的“不择手段”
。你倒是非常能够理解舅舅和表哥,当一个三十岁的单身女人同一个四十岁的单身男人
相遇,当这个相遇的时间也早被舅舅安排得非常精确,我不知道你从哪里来的自信能够
一口认定我就是那么很aggressive冲着绿卡杀过去的!我从来没有否认我有身份问题,
我也从来不否认表哥是45岁的成熟男人,但你在指出我并不漂亮和年轻的事实的时候,
你思维倾向何尝又不是表哥这个45岁的成熟男人想要找的说到底还是应该是年轻漂亮的
。你不了解事情的真相,就对别人妄加评断,把别人想得不堪入目。从你的描述里你像
是很能理解舅舅和表哥,我把自己的成长经历都写出来,你也丝毫看不出我过往的经历
与我感情生活的联系,你的眼里看到的大概还是只有绿卡,我也只能对你的理解能力心
存怀疑了。你有你的思维定势,你有你的世俗,被你所代表的很多看客已然成为了这个
社会的中坚力量,还不够吗?何苦把我拉入到你们的行列?这个社会上世俗的人如你,
显然不大容易会产生逆势,是的,你很幸运,你拥有世俗不易出错的性格;但是当舆论
的矛头指向我,我就真的错了吗?除了高考那年我说错了一句话,我又真正做错过什么
?你太简单,你太想当然,你太自以为是,你生活在世俗的条条框框、思维定势里,以
至于对一些事情的发生完全没有理解能力。当然,这不怪你,谁知道是怎样的父母、怎
样的生活环境才把你造就成这个样子呢?说得再重一点儿,你这辈子的生活如此,将来
你所选择的结婚对像也会如此;将来你结了婚生了小孩,你小孩子的性格发展还会如此
!我承认我有某个方面的自卑,那是成长过程中造成的尚未摆脱的阴影。但我不缺乏自
信,你也大可不必把我想得过于糟糕,那样只能是侮辱了你自己的心智。

4,    对别人好是我需要学习的功课。从上大学意识到自己不太懂得关心人,我已经
开始在努力改变自己。其实我是能够包容体谅亲人和自己喜欢的人的,比如表哥,我只
是不懂得、不善于去包容、体谅和关心离我没有这么近的人。

小的时候我迷信过,长大也是一个学习挣脱束缚的过程。你并不真正理解我,我们是属
于两个不同世界的人。我并没有迷失自己,我的生活没有你想像的那么糟糕。前面有些
地方我可能无意间有些话也说得比较重,但都无恶意。即使我们无法很好地沟通,我也
相信,你是一个善良的人。无论如何,谢谢你的好意。
\section{这位同学,看你码字这么多,来说两句。}
\label{sec-9-66}
从你和你父母和姐妹的成长和关系来看,你没有处理过太复杂的社会关系,也没机会学
习和观摩你的父辈或者长辈来处理社会关系。你的社交圈子好像大多是以前的朋友和亲
戚。我没有别的意思,看你写到你和父母的感情和很多细节,觉得你不像挖坑也写得比
较真挚,但是又和很多其他读者一样(我也很诧异我读了你写的这系帖子),觉得你对
理解你现在周边的社交圈子好像非常脱节。但又不便做过多的评论,怕伤及你的自尊。
人与人的交往关系不是你带着礼物和茶叶去反复探望就能建立起来的。 你的表哥对你
是完全没有什么感觉的,一开始没有,后来也没有。 看你贴的emails里面,觉得你的
英文水平也有限--客观的说--所以觉得你和美国人的交流是会有相当的隔阂。 你
不应该用你以前周围的人行为举止还推测你现今生活中人们的想法-比如你的表哥是不
是对你有情。 你要学习。 这个是我要说的,你不要把自己关起来遐想了,要学习。

yea,从这个角度讲,你是一直blind的了。
\section{发信人: closet (closet), 信区: Dreamer}
\label{sec-9-67}
标  题: Re: 成长的故事 -- 我和舅舅

发信站: BBS 未名空间站 (Sat Jun  2 00:29:43 2012, 美东)

你对我的猜测,无论年龄,性别,婚否,感情经历,有没有孩子没有一样是猜对的。
果然我大部分都白写了,因为你是看不懂的,或者根本不愿意懂。你表哥越来越直
白的据信甚至愤怒的一而再再而三的警告,你也是一样看不懂,或者不愿意懂。
只有继续把对你指出这些的人都归为世俗和没有灵魂一类,你才能找到些许慰籍。
事实是,

真正对感情有信仰的人起码深知在感情中最重要的是真的能够互相理解,能沟通。
(你表哥跟你根本无法沟通,他说you don't listento me, 到后来你疯掉了似地曲解
像他愤怒据信的“深意”)

真正对感情有信仰的人起码把对方的感受摆在最重要的位置,不会把自己的想象和愿望
强加给对方,不会用自己的行为给对方带来无休止的骚扰和痛苦。

真正对感情有信仰的人起码是感恩的人,总能感受到周围的阳光和温暖,

真正对感情有信仰的人起码是自信的人,不是爱无能的人,给亲人爱人总能带来春风化雨
的慰籍。。。

唉,良药苦口,忠言逆耳,你好自为之吧。

不过既然你能相信我的善意,还是把我的长文保留好吧,什么时候你快醒了的时候,
可能能起个催化剂的作用。(高考前醒过一次)(研究生时候,跟二姐电话,电话
里面二姐骂醒你的关于导师的那次算醒过第二次)希望你有第三次。
\section{写给表哥的邮件}
\label{sec-9-68}

到三月下旬的一天,表哥的20个联系人就变成了21个。我对舆论的感受能力总比表哥和
舅舅显得迟顿。我没有对表哥讲过关于自己的任何事情,但表哥给我的感觉是他对我的
事比我自己知道得都还要清楚。我猜测那两天表哥一定发生了点儿什么,但具体的事情
我不知道。跑到他们学校的网上去搜,学生的记录里已经没有他,那他现在是什么状态
?忍了两天,我终于还是憋不住了,给表哥发了邮件,说了一些我自己心里的想法。

from: me;  to: 表哥;date: Thu, Mar 22, 2012 at 9:52 AM
subject: Re: jobs \& visits mailed-by: gmail.com
Dear Cousin,

I don't believe what you said, because what you have said is different from 
what you behave. I do believe you love me, at least you like me.

Dear cousin, will you graduate this May and be working in XXX after 
graduation? Have you ever thought the future between you and me? I may leave
you bad working professional impression, but the point is, Dear Cousin, I 
love you, and I want to be with you, wherever you are. I had wished you will
move to CA, but if you are going to work in XXX, I can move back to be with
you. 

The past two days have been so hard for me. Dear cousin, as I have told you 
long time ago, you are the most cherished cousin in the world, so please do 
not leave me alone. The life, even only the thoughts about you working in 
XXX and I in CA, the feeling about separation between you and me makes me 
feel so empty. The man I will marry, if it is not you, whoever he is, I will
be living an empty somebody else's life, not mine. Dear cousin, I just want
you know my feelings for you is true and real. If you do not want me, if 
you do not want to keep me, I will be empty for the rest of my life. Please 
do not do that to me. 

I had a dream about you the night before yesterdays. You had two pills and 
then we had sex in my dream and it’s so plain, but when I woke up, I knew 
it was not the truth, just that I missed you so much!

My contract is signed with the staffing company; the first work will be 
three months, which means it will end in the end of May, 2012. I still 
believe the staffing company will sponsor H1B for me now. But if they 
cheated on me, I will need to be prepared in case such bad things happened. 
If I failed, and you do not want me at all, I will go back to China to waste
my life over there whatever it would be. 

But if things go smoothly, if you can tell me your true feelings, I can 
either go back to XXX in the end of May after I finished my first job, or I 
can go back to XXX in Feb 2013, which will be the end of the one year 
contract signed with the staffing company. But all these decisions also 
depend on you. In between, in October, if we have not married, I will go 
back to china to see doctors, it will need super wave B detection and CT 
scan, and potentially an operation on the abdomen once again. After all 
these have been done, when I come back to US in Oct again, I will try to 
exercise so that I can build up strong body and be prepared to be able to 
pregnant later on. And maybe, if it is possible, I will bring my mom to US 
with me for about half a year. My dad passed away from last October's 
motorcycle accident, and I missed him a lot. 

Dear cousin, please take off all your protection colors and treat me true. I
miss you.

在我反复思考着我究竟是什么看不见、有着心病的那几个星期里,在我的邮件发出去不
到半个小时,我就收到了表哥的回信。我刚看了个题目,一看到第三个字,眼泪就掉下
来:You are delusional!
\section{表哥的回信}
\label{sec-9-69}

from: 表哥;to: me; cc: 表哥、舅舅;date: Thu, Mar 22, 2012 at 10:17 AM

subject: You are delusional (Re: jobs \& visits)

On 2012-03-22 09:52, me wrote:

Dear Cousin,

I don't believe what you said, because what you have said is different from 
what you behave. I do believe you love me, at least you like me.

Stop writing.

I do not want you.

I do not want to hear from you.

You are seriously delusional.

You are not a normal person with a normal brain,

who understands things when they are explained to you eight times.

You have been thrown out of XXXX by the police once already.

A normal, thinking person would take that as a hint.

If you come to my office, I will have the police arrest you.

I didn't even read the rest of your e-mail.

\verb,~~~~~~,
   Your behavior falls under "stalking".
I now gather all of your e-mails, and a journal of your phone call attempts,
as evidence of harassment and stalking.

   This is a big hint to you that your behavior is wrong.
The picture inside your head is wrong.
Get out of your private little fantasy world, and smell the reality.

You creep me out.

--
表哥的名字
\section{舅舅的警告}
\label{sec-9-70}

收到表哥邮件的那天早上,因为几个星期前爸爸的暗示,因为表哥邮件里的“
delusional”,我的眼泪默默地掉,鼻涕也稀里哗啦地流。同以往表哥的任何邮件一样
,我总是看不出拒绝,或者我的主观意识无论如何也不肯相信表哥会拒绝我。我们之间
,表哥与我之间还是那么亲切亲密。表哥也不想想,这个人,我明明是喜欢着的啊,我
怎么可能走出这个private little fantasy world,无语。我心里有明确的态度,没有
被表哥的恐吓吓倒,但我却被自己的情绪打败了。我深爱过的爸爸,和现在喜欢着的表
哥,前后竟然表达了相同的意思,表哥怎么可以用我的敏感词汇?

几个小时后,我的情绪终于慢慢平复回来。我又收到了来自舅舅的第二封警告信。
from: 舅舅; to: me; date: Thu, Mar 22, 2012 at 1:43 PM
subject: Stay way! mailed-by: XXX.edu

Ms. XXXX(my family name),
This is another serious WARNING to you!

Stay away from my family. Stop sending any
nonsense email, phone calls or trying to come to
XXXXX to intrude on our lives again.

We have been documenting your continued harassment
against us. It will be used as evidence to prove
that you are mentally ill and DANGEROUS, and get you
DEPORTED from this country.

You better believe I can do it.

Uncle’s full name

舅舅是个很浪漫的人。表哥还没有回来读书之前,那辆蓝色的车,舅舅就早早地为表哥
准备好了,在舅舅家换机油时我看见过的;现在他一大把年级了,居然还知道1343、
1413,想踩着时间点发邮件呢,居然还敲掉了个字母!我早说过舅舅也是个势利的人。
看吧,之前一封邮件就是呼啦啦直呼我名,全名还是,现在我工作了,也开始称呼起“
Ms. XXXX”了。是因为经过一段时间的思考,我正在慢慢放下对舅舅的戒备吗,表哥同
舅舅的两封邮件,我居然都没有反感,反倒觉得舅舅可怜。这个冷血无情的人,除了动
辄使用武力恐吓,其实他没有任何其它办法,或许他也很贫脊吧,思想上精神上,我猜
测!
\section{被拒被警告之后}
\label{sec-9-71}

一天里到两封邮件后的我,心里是有过矛盾挣扎和反复思考的,那几天过得相对沉重。

是的,我与表哥没有恋爱过,至少语言上表哥从来都没有承认过。10年12月我回学校那
次与表哥走在从图书馆回来路上,已经走进了表哥的office大楼。楼道里并排走着的我
问表哥他有没有女朋友,表哥说没有;我再问他,“那你有过女朋友吗?”表哥对我的
话题没有兴趣,答曰“不知道,没见过。”我知道他不想多说什么,便笑说,“那见过
的你都不喜欢,是吧?!”表哥低头看我,我们四目相对,笑而不语。

我回舅舅家总是有事情,一般我也都是要表哥陪我。但我理不应该打扰到他的学习和工
作。所以,虽然表哥可以充当司机陪着我载我往返于office和家之间,但在表哥office
里,为了他的工作,他不喜欢我问乱七八糟的话打搅他,这一条我也基本都是遵命的。
自从那场惊心动魄的告别,我有过心底最深的感动,再后来与表哥的相处,我习惯了在
表哥面前、在表哥的目光里撒娇,忘了该如何正常说话,表哥也从来都是宠着我的。
Office里两台电脑,两个人各干各的,静谧的时光也总显得宁静详和而知足。我从来不
因为表哥拒绝我而放弃,因为我心里始终相信我们这么亲密,眸子里仿佛能看尽对方的
三生三世,我们应该最终能在一起。

恋爱经验不丰富、体验不够深刻的我总是显得乍乍乎乎没心没肺。从表哥身上,我能体
会到的是深沉内敛和相对执着的感情。当舅舅邀请我上他们家吃饭的时候,当舅母送我
表哥从学校试验地摘回来的水果的时候,当表哥在饭桌上小幽默一下试图拉拢距离,舅
母故意问及表哥他某个学生想要刺激我的时候,表哥否定了舅母的试探,但那时候我还
在暗怨表哥对我一点儿情意都不表达;毕业后要到加州临行前表哥送我好不容易心里有
点小小的火花后来却被大表姐浇灭了。那场告别,两天的相处对表哥有明显好感的我对
表哥表达了我的真实想法,我不知道我该选什么样的人,表哥依然很深沉地宠着我。若
我自己不能去感受、体会和发现这份感情,我想,表哥永远不会对我多说一个字。

所以表哥的拒信里的话,我是不相信的。他只是在骗我,想把我赶走,或许希望我物质
上能够生活得好一点儿吧。

我能够理解表哥,但我对表哥也是有怨的。现在,好不容易,我爱上了一个人,他却要
我Get out of the private little fantasy world,难道我爱上一个人的目的仅仅是
因为我需要再去忘记这个人?表哥若有原因,除了是为了我的将来的物质生活考虑之外
,他还有其它顾虑吗?他的顾虑又是什么?

我因为一句话被人肉后我的意识总是跟不上,总是不能足够意识到我周围有多少双华人
眼睛在盯着我。所以那几天当我心里有所郁积上国内校友录到我高中、大学、硕士和这
边学校的班级都看了一遍的时候,由于一些媒体的影响,我与表哥几年来的经营便付之
流水,完全变了样。当意识到这边媒体转向的时候,我后悔及了。

那只小鸡的死就真的是爸爸在向我暗示什么吗?爸爸当初是放心地走的啊。就算爸爸还
活着,与表哥的事情,我会依了爸爸吗,一定不会的,顶多我再多下功夫说服爸爸。所
以,在最终意识到这一次完全是我自己坏了事,这一次若分开可能就是永远的时候,我
无论如何是舍不得表哥的。

从上次八月回去被舅舅打了911到现在也八个多月了,这期间爸爸生病时我打给舅舅的
电话里舅舅说我多回去几次就被直接谴送了,这次邮件舅舅要我“stay way”,有没有
一种可能,舅舅是希望我能回去的?加上现在我自己工作中同中介公司的一些牵扯,这
一次,就算舅舅再次警告了我,为了表哥,天打雷劈,我也要回家!
\section{回家}
\label{sec-9-72}

回到舅舅所在的小镇,同以往一样,我先前往表哥office去找表哥。表哥一定在的,他
打开门走出来,表哥看着我,我对他说我出事了,中介公司怎么怎么样,表哥说了一句
不关他的事便走了。我并不介意,想要打破一个人的伪装不会是件容易的事。

我猜想表哥大概是回家了吧,我便也开车回舅舅家。家里无人。车库侧菜园院子门上镶
了块一米见方的红色木板,在旁边周围涂了浅色系列的车库门、房子中尤其显眼。我知
道舅舅家狗舍的院子也是舅舅用来种菜的,他可能怕我看不见吧,在水泥地旁边又架了
个长方形的苗莆,里面长了两三根葱。这些是我站在房子外能够一眼注意到的变化。

开车也很辛苦,我本想在车里休息的,偶然又想,表哥走了他office的门就锁上了吗?
我干嘛那么傻傻地他前脚走了,我后脚就也走了呢?我还是应该去表哥office里找他。
于是我又折回去了,门是锁着的,敲门无人应。我没有办法,只好再次回到舅舅家来。

这次,没太注意舅舅从哪里回来的。舅舅问我为什么又回来,我很小心谨慎,回答说我
找舅舅有事。舅舅不允许我住他家,不允许我进他家家门,威协说进去了他就打911。
说有事就去他office里谈。在我为自己的不小心疏忽后悔不远千里冲回来了解家里情况
的时候,紧接着舅舅第二句话就是批评我说“你自断后路!”“我怎么就自断后路了?
”我本能地反问舅舅,他却开自己的车飞快地骠跑了,也不顾我记不记得路。开车回舅
舅家的路上我想过,这次回去我要小心小心再小心,尽量为自己避开不必要的麻烦,所
以很多事情要尽量按舅舅的意思办。
\section{谈话}
\label{sec-9-73}

表哥与舅舅的办公楼是同一栋,我背着提着自己的电脑包,还是再次先去找了表哥一次
,敲门无人应,便折回来去找舅舅。我记得舅舅office的门牌号,但从表哥office出发
能找到它并不容易。费了九牛二虎之力,总算找到,敲舅舅office门的时候,我也想起
,来到美国第一次敲这个门是2007年5月吧,时间过得真快。

我是回家来探测家里情况、探测舅舅表哥对我的态度意思的,既然我找不出理由就会被
舅舅打911,那我无论如何一定会能想出理由想出事情的。我便对舅舅坦诚交待了我被
中介公司骗的事。

舅舅把我领到旁边一个比较空的office里,里面只有两三张长桌,舅舅自个儿在桌子上
坐下来,我先是靠着墙对舅舅讲我与中介公司的牵扯,说着说着便蹲下来,再后来就成
了坐在地板上双手枕着膝盖哭。我告诉舅舅我被中介公司骗了三次,签合同之前,上班
工作第一天和三月上旬与公司分部领导会谈时,他们一再告诉我即使我的合同只有三个
月,他们也会把我assign到他们公司的其它client,继续其它项目进而帮我申请工作签
证,但到四月上旬他们就变了卦,要我自己找,而且没有任何损失赔偿。面对舅舅,我
一边说一边哭,哭一阵哭累了歇下来,过一会儿气来了眼泪又扑朔直下,哭了一阵又一
阵。舅舅给我打气说,不要自暴自弃了,哭好了情结发泄完了,就该打起精神来继续好
好找工作。我对舅舅说我一定会尽自己最大努力找的,可我就是气不过他们怎么就可以
为了一点儿区区小利连续骗我了三次,把我最后一次申请H1B工作的机会时间都浪费掉
了,也没有任何赔偿。我就是气不过他们骗人也能骗得这么理真气壮,骗得那么坦然!
说着说着眼泪又掉下来。

我问舅舅,我就不可以把中介公司告上法庭吗?我原本打算工作到一个半月就不干然后
找工作的,但是他们要我签一下放弃上诉权力的合同,我不愿签才留下来继续工作的。
舅舅说一般工作关系我告不倒它。我便说了三月上旬我与他们会谈我有要求他们把谈话
内容在邮件里写下来,而且我存有邮件。舅舅或许为了给我点儿希望吧,但说我只要有
证据就可以啊,我有证据的话请律师给公司发一封律师函,他们作为公司一般都会有律
师出来交涉,尽量争取私了的。我总算是心里找到一点儿安慰,心想着万一我找不到工
作,我可能会告他们吧。
\section{谈话(2)}
\label{sec-9-74}

舅舅大概受不了我哭得那个伤心欲绝,给我寻找出路说,想要在美国留下来也不难,那
么多人都黑下来了,我也可以黑下来当保姆或者做其它工作,美国也会十年一次大赦,
也能最终取得身份。我打断了舅舅的话说,我如果万一找不到工作,我会回中国,我读
了这么多年的书接受教育,我一定不会黑的。

可是我的眼泪还是一阵一阵地往下落,想起什么来就哭一阵儿。我哭的真的只是因为被
中介公司骗吗?我哭的略感不平的还有我的“命”。哪个女孩子没有点儿小心思,想把
自己做到最好,能凭借自己的工作能力、社会地位以、资金积累以及温婉贤淑等赢得婆
家的尊重和好感,我与舅舅一场口舌仗能打一年半,为的不就是嫁到表哥家后尽量少受
气吗?可现在好不容易熬到今天,因为中介公司的骗局,全盘皆输,我的仗也是白打了
!我没有工作嫁过去,舅舅倒还好,舅母能给我好脸色吗?我心里害怕发慌难受。

趁我哭不动了,舅舅又给我出了个主意。舅舅说我一直是有工作、工作还不错的,我可
以通过学校IPO再继续申请OPT延期,并把中介公司的合同和一些证据让老师知道,看能
否OPT再延长几个月或者半年一年的。我想舅舅可能也算是从我们家乡跑出去的枭雄吧
,当时的舅舅气场非常强大,我唯唯诺诺地应了他说我会那么做试试看。但真正回到加
州后我并没有,还是觉得机会面前人人平等,既然大家的OPT都是那么长,是我自己没
把握好被骗的,那我除了从法律上寻求保护外,OPT延期我还是觉得自己不应该过分。

对于舅舅家里说我自断后路的话,一路上开车过来时我也在想着,难道舅舅认为我不该
放弃表哥?可我没有啊,媒体、舆论一转向,我人不就跑回来了吗?况且表哥在我这里
从来都不是后路啊,他可是从来都点据首选重要位置的啊,怎么会是什么后路呢?我便
问了舅舅这话。很意外的是,舅舅指的竟然是两个表姐。两个表姐有像表哥那样进入过
我内心吗,有真正感动过我吗?没有,从来没有。所以,他们在我这里也不存在一条路
,也别说是什么后路了。我如实地告诉了舅舅我的想法,我心里并不觉得她们真正很真
诚地帮过我,所以也不能强求我像当初买礼物感谢舅舅一样对待表姐们,一点小小的恩
惠,我也都买过东西感激过的,物质上我也并不亏欠她们什么。我告诉舅舅他不能对我
要求太高了。
\section{谈话(3)}
\label{sec-9-75}

来office找舅舅时舅舅问我已经去找过表哥了,我答已经找过了。舅舅便批评我以前来
找表哥时一来就朝表哥的小躺椅上一躺成什么体统!我在心里小声反抗“那时的你们也
都没有说什么的呀,到现在怎么就全成我错了?”但想到表哥与我的关系,舅舅的局设
得那么深,我怕自己走不到、做不到舅舅预设的那一步,便将这次与舅舅的见面当作我
有可能回国之前最后一次见舅舅的礼节来处。

想到当年爸爸年轻离婚后、遇见妈妈前的那一段浪子生涯,爸爸的心路历程我再也没有
机会亲口问爸爸了,我便把对舅舅的所有疑问全问出来也算从自己的角度不留遗憾吧。

我问舅舅当初我来美第一年,大舅母告诉他我就在旁边学校,舅舅为什么不来找我?舅
舅说他压根儿就不知道我在旁边学校,大舅母从来都没有告诉他我在旁边学校的话。事
过境迁,我已经从那段阴影中走出来了,至于到底是大舅母还是舅舅在这件事情上撒谎
,对我也已经没有那么重要了。

舅舅08年夏天舅舅把我送到表姐家不久便离开了。舅母后来打电话知道坐greyhound回
家的舅舅把随身携带的小包带丢了。我清楚地记得舅母对我转述这件事情后,眼睛直愣
愣地盯着我问“你相信吗?”我答不信。我便又问了舅舅后来他的小包找到了没有,舅
舅说找到了,丢不了的。舅舅补充说到,他从来都大大裂裂、不拘小节,做事情丢三拉
四是常有的。舅舅的这句补充让我心惊,怕自己在与表哥的关系上一脚跌空闹笑话。我
便对舅舅转述了舅母当时与我的对话。

舅舅接着便说大舅大舅母在他面前从来都是说我的坏话。面对别人的指责,我总是本能
地保护自己。我如实地说大舅大舅母在我面前总是亲口夸我如何如何地好,大学的某一
年暑假还要求我帮大表哥家的孩子补习数学。大舅大舅母如果真在舅舅面前是另一种说
法,那也只能说明他们虚伪。事后我想,这大概是当时舅舅强大气场下逼我得出的结论
吧。我究竟是什么样的人,舅舅自己不清楚吗?
\section{谈话(4)}
\label{sec-9-76}

08年暑假舅舅领我去表姐家前曾经亲口对我说,他把小表姐带出来读书小表姐就从来不
曾向他表示过什么。于是,这次我又问了一遍舅舅,他改口说“我能帮助别人我也很开
心,至于小表姐报不报答我那是她的事。”当时的我就意识到了舅舅说法上的前后变化
,但若舅舅把这当作新闻发布会,那就让它如此发布吧。我接着说听大舅母说小表姐工
作后第一次回国时,躺在大舅舅母的床头说了很多舅母惹小表姐生气的话生气的事。舅
舅继续谦谦君子地说,“我只管我自己没有做过那种事,舅母有没有做过我不管。”我
心里又格凳了一下,难怪我也会觉得舅母嚣张,原本舅舅从来都是惯着她,由她随便来
的!虽说他们家女主人的家庭地位还蛮高的,可还是为自己将来嫁过去可能会有的命运
感到担心。

在我心里表哥还是我正在喜欢的现在时,但与舅舅的这次谈话,语言上我始终把表哥当
作恋爱过去时,因为在我可能还会努力去做什么之前,我需要先探清楚舅舅的态度。想
到表哥,我问舅舅,他清楚地知道、明明白白地向我解释过我与表哥的亲缘关系,为什
么还要说我是表哥的first cousin,当初我问舅舅对这件事的态度时舅舅为什么要说“
既不支持也不反对”?舅舅说他那么说也是量我怎么地表哥也不可能喜欢我!这就是舅
舅的解释,这就是舅舅前后一直这么做的原因?我非常错愕,这冷血的舅舅血冷得也太
离谱了吧?我正要说那表哥还怎么样,舅舅一口否定,表哥没有任何错!

话面上我是把表哥当过去时的,便说也都过去了,无所谓了,表现得海阔天空,云淡风
轻。舅舅却接着说,“你知道吗,你去他office,他完全可以用枪一枪打死你,不用付
任何法律责任!”对于我这个冷血无情的舅舅,每当他说到谁谁可以用枪打死我不用负
法律责任(去年八月我回去时他就说过一次),我就很抓狂,血往头上涌!在舅舅眼里
,难道亲情是用法律衡量的吗?可是仔细想想,舅舅说表哥没有一枪崩了我,不正是想
说表哥待我很好吗?舅舅接着批评我说我都没有读懂表哥的信,“他说你是神经病!”
因为舅舅指出表哥待我很好,因为舅舅回答我提问时表现出的谦和略带谄媚的态度,以
及为保护他自己对大舅大舅母的攻击,舅舅的这话在我脑海中便被直接翻译成了“表哥
给你的信写得有情有义,你居然都读不出这层感情!”
\section{谈话(5)}
\label{sec-9-77}

我对舅舅说有些话我不知道当讲不当讲,舅舅让我说,我就接着把我前段时间比较迷信
的事全说了。爸爸离开的日子、正月里妈妈第一次生病的日子,小鸡死的日子,和小鸡
死的原因。这时的舅舅关起门来,拖了拖桌子坐到窗前,点了一支烟抽起来。这是我第
一次看舅舅抽烟。烟幕中穿着一身破烂的舅舅显得那么贫困潦倒,我淡淡地讲着自己的
故事,仿佛与舅舅相识的十五年时间此时从烟雾指缝间穿过,禁不住内心暗自唏嘘感慨
,那年我十八岁时遇见的舅舅多么年轻伟大,而此时的自己人未老心已老。

“它不知道为什么看不见吃东西就饿死了。”我接着说,“一看到表哥的回信,我眼泪
就掉下来,表哥说我‘de-lusion-al’!”批评了我已经整个一下午的舅舅这时终于笑
开了,而我这次回家的使命到这里也算是最终完成了。我花过一年的时间去恨、去忘记
一个人。舅舅和表哥不惜撒谎说我是first cousin在我看来是打了一场舅舅设计的战略
意义上的舆论仗,却被我一不小心前功尽弃毁于一旦!若舅舅一定需要表哥与我的婚姻
得到世俗的认可他才肯让步,若我一定无法把表哥据为已有,那也是我的命。那在我迫
不得已务必离开回国之前,我不也应该让他知道原因吗?舅舅知道同表哥知道是一样的
。哪天表哥伤心难受了,舅舅还可以转告他的,都是人性。

我的任务完成了,便把去年八月就已经买了的两磅茶拿出来想送给舅舅,说要是我万一
找不到工作,我就回去了。“08年夏天大舅舅母送舅舅去车站回来对我说,人上了年级
就见一面少一面了。我这次来也是想在我回国之前,我想再见舅舅一次!”舅舅回答说
,“你确定你是来见我最后一次?我就怕你临走之前又跑回来闹!”嘿嘿,舅舅前面已
经打好铺垫了,叛离如我,舅舅既然这么说,我怎么可能不再回来一次?我心里一边叹
一边想好,下次回来我要好好地只见表哥,不见舅舅了!

让我意外的是,这次我送茶舅舅不接受了。谈话将尽结束,就要离开了,舅舅追问我说
,我这次回来就这点儿事?我不等到周一同IPO的老师询问好OPT延期的事吗?舅舅说我
如果没钱,他可以帮我订旅馆,但我拒绝了说我有钱。我知道舅舅希望我留下来,也许
是希望我自己找机会同表哥再处一处,处不好闹一场也好。但我还没准备好,每次舅舅
与我谈话,在他强大气场下,我总是本能地反抗保护自己,往往头脑发热词不达意,我
还需要回家后再回味体会好好想想,我怕自己输得太惨,不愿打没有准备的仗。临走时
我又去敲过一遍表哥的office门,没有人应,我便离开了。
\section{回加州}
\label{sec-9-78}

我急着离开,因为周一还要上班,我希望周日能早点儿到家把自己安顿好、收拾好、休
息好。原本我是没打算住旅馆的,但为了不让自己来回奔波过于辛苦劳累,也为自己的
荣貌考虑,开了两个小时后,便在一家韩国人开的旅馆住下来。

回想起来,表哥舅舅对我还是挺包容的,去年二月我回去拖行李被舅舅指责我不择手段
气跑了,自己在附近小镇住了旅馆;等到四月份我自己回去找表哥时,表哥质问过我“
谁是舅母”,但当我说要表哥晚上回家的时候把我带回舅舅舅母家,表哥就同意了。而
现在舅舅不允许我住他家、让我去住旅馆,我竟然还是觉得舅舅不让我住太没面子了,
宁愿再开两个小时的车到一个其它城市住也不愿住在附近小镇。

住进去后意外发现这是一间smoking room,这也是我第一次住这种房间。我的喉部不太
好比较敏感,在国内的时候慢性咽炎从高中开始每过几个月就会发一次,硕士时还因为
喝过多咖啡导致急性咽炎过。到美国这边后只10年春夏发过一次,其它时间都还好。

晚上11点钟左右,迷迷糊糊半睡半醒之间,呼吸着屋内略微呛人的烟草味道,回想这次
回去同舅舅的对话,谈话间舅舅所表现出的态度,以及office里舅舅抽烟时贫困潦倒的
沧桑,蒙胧中我忽然觉得自己悟了,我对舅舅的心机怕了很久,现在终于是可以放下了
。
\section{冷血与亲情}
\label{sec-9-79}

妈妈说舅舅回来探亲时,可能是七十年代末八十年代初吧,给大舅买了个小彩电。那时
的政策大概还不够开放,大舅怕惹事,居然把彩电交公交给了生产队。妈妈说那彩电的
质量非常好,放在生产大队某个办公室里,队里村民进进出出噼里啪啦地拍过那电视好
多次,那彩电居然还在那里放映了好多年才最终坏掉。

妈妈说二外婆命好有福气,大舅舅母最初都没有机会来美国,二外婆还被舅舅带到美国
来探亲玩过,而且来过两次,每次来一年左右。而作为家里唯一“劳动力”的舅舅(我
猜想舅母大概嫁给舅舅后就没怎么工作过吧)不仅把二外婆带过来探亲,尽了孝道;而
且为下一代的发展考虑,虽然家里已有三个孩子读书,却还是把小表姐也带到美国来读
书。

那么舅舅回国探亲找到二外婆大舅之前、之后的大舅家又是什么样子呢?据大表姐说他
们小时候二外婆同舅母三天两头地吵。二外婆是从29岁开始“守活寡”,一个人含辛茹
苦带大大舅和小姨,大舅母是领着当时五六岁的大表哥嫁进那个家的,而大舅母同二外
婆之间竟不能相处融洽。从大舅大舅母后来待我们亲人看大舅在家是对这些家事不闻不
问的,一辈子不为任何亲人求情办事的大舅就只为大表姐在国内上班失业时求过一次朋
友,我猜想大舅可能也冷血吧。

再后来二外婆被舅舅接到美国探亲期间,小姨却因为在铁路上捡煤丧了命。那个时候的
大舅家还会贫困吗?大舅与大舅母都是人民教师,二舅待二外婆会不好吗?亲人亲情在
大舅大舅母家又究竟算什么,何至于小姨还需要去铁路捡煤丧命?

大舅母是什么原因离婚嫁给舅舅的,我不知道;后来大表哥是因为日子过好了在大舅母
的唆使下要换人却是明显的;大表姐嫁的是原本心仪小表姐的人,在这样一个家庭里,
爱情又算是什么?大舅大舅母待我们这边的亲人好坏我自己又不是不知道。
\section{冷血与亲情(2)}
\label{sec-9-80}

我不是总是对舅舅的心机、舅舅的冷血耿耿于怀吗,就我所知道的,舅舅待二外婆、待
大舅大舅母表姐们都不差啊;舅舅鼓励当年十八岁一脸困顿的我这个远亲,不是也没有
任何差池吗?表哥家里除了舅舅舅母希望我们为他们养老,小表哥需要我们接济,舅舅
家里还有其它底牌吗?我总担心舅舅的心机,舅舅的心机加害过我吗?

当纵横几十年的客观历史事实摆在眼前的时候,当我已经走出了那来美国第一年的怨与
痛,舅舅的那一点儿疏忽还真的有那么重要吗?舅舅同大舅母两人之间究竟谁在撒谎还
重要吗?后来舅舅也为我转专业作了经济担保人,能帮我的也都帮我了。

这几年里舅舅也说过做过让我特别感动的事。除了当年“冰调车头”舅舅带我去找雪胎
压惊之外,忘了那年是什么原因,我在邮件里给舅舅回信说“Put some money into my
account makes me feel safe.”后来舅舅见了我说要受过什么苦受过多少磨乱创伤才
会使人变成这样。我虽然并不觉得自己受过很多苦,可舅舅的话让我觉得心里很温暖。
后来我问舅舅借过四千块钱,大表姐都逼着我把两千块钱还了,舅舅为了让我感觉安全
从来不逼我不提还钱的事。

我的舅舅,他又为何会对安全感理解这么深,他自己有安全感吗?舅舅给我讲过他回家
乡找到二外婆时的心理落差(二外婆容貌在他心里的记忆停留在她29岁舅舅当初离开时
的样子),给我讲过他对机械、对专业的爱好,给我讲过二外公在台湾的简单生活,但
舅舅却从来没给我讲过当年他随二外公离家后如何流连辗转去到台湾。为什么舅舅喜欢
吃湖南腊肉,他们在湖南呆了多久,他们如何谋生混饭吃,他们受过怎样的苦,舅舅又
是什么年级重返校园接受教育?那个小孩只有十一二岁便离开了妈妈,二外公懂得既当
爸又当妈地照顾舅舅吗,后来的二外婆待舅舅好不好,这所有的一切我都无从知晓。我
的成长过程父母双全,待我都还不错,我尚且阴差阳错地走过许多弯路,后来修正后与
父母感情不错,与姐姐们就差些;舅舅这个饱尝人世沧桑辛酸的人得到过多少亲情、心
里又有多少亲情呢?为什么之前舅舅对我说他没有亲情我从来都不理解、还对一些不值
一提的小事耿耿于怀那么久?若我再继续迷信一下,之前的我还真是看不见!
\section{我和大表姐}
\label{sec-9-81}

大表姐和小表姐是小时候爸妈为姐姐们和我学习树立的榜样,妈妈说只要我们像大表姐
一样好好学习,考上大学就可以跳出农村农门,端铁饭碗,一辈子吃香的喝辣的不再劳
苦。

08年夏天舅舅把我领到小表姐家去的时候,舅舅并不知道,两个表姐作为我成长过程中
的榜样,我与她们、与她们的家庭感情上却是疏离的,因为爸妈不喜欢他们家,情感上
我也无法做到从心底去喜欢这家人。我上大学后听说过我当年不愿参加高考的大舅母问
过我,为什么我有思想包袱的时候没有去找他们寻求开解,当时的我回来不上来,现在
想来,我大概是觉得他们不是我完整意义上的榜样偶像,我对他们没有信任。这或许也
是当年十八岁的我见到舅舅可以与他分担我的痛苦,却不愿意对大舅、大舅母说什么的
原因吧。

我也对大表姐亲口说过,我很佩服她,觉得她相当有魄力。是因为当年大学里大好年华
的她没能出国而小表姐却被带出来了,她从来不曾获得来美的机会吗,后来大舅母对我
讲说小表姐生第一个小孩的时候,大舅母带着大表姐去签证顺利签过了,便开启了大表
姐的留学之路。

当时的大表姐可能已经三十五六了,英语大概也忘得差不多。但就在这样一个年龄,大
表姐还是毅然决然地在我所在的学校学了一年读言,再花两年时间读了国内专业的硕士
,读书期间将丈夫孩子移民加拿大。毕业后的大表姐运气不好,赶上了经济危机,不好
找工作的年月 里她回到加拿大当大巴司机打labor工干了好几年。经济复苏后表姐又
回到美国这边依靠朋友关系找到第一份工作,从工作中锻炼自己,进而留了下来。大表
姐现有北边小城一套房和加拿大一套房,手中存款若干,儿子已经上了大学,大表姐这
辈子也算是熬出头了。
\section{我和大表姐(2)}
\label{sec-9-82}

拥有俗世聪明的大表姐待我也还算不错,但她们却无法像表哥一样扎根住在我心里。平
时打马虎眼的事根本就入不了我的法眼,而真正到了考验友情、信任的时候,表姐们却
也会为了她们自己的利益和省得麻烦把我抛开,比如我找工作时住表姐家一个周便被赶
了出来,比如大表姐逼我还两千块钱的时候。于是两个表姐在我眼里也就只界于朋友和
陌生人之间,好朋友算不上,没有深刻的友情,亲人更算不上,只是远亲。或许我们都
知道对方的存在,都知道那点儿亲缘关系的存在,比陌生人好一点,比朋友、亲人就差
一些。

10年大表姐逼我找男朋友的时候见我交友那么困难,有给我介绍一个她之前的印度同事
。那天约好了周六晚上吃饭。大表姐是媒人,我想着她可能会请我们三个。但那天快去
吃饭时表姐招待我在家先填了些东西,到饭店后表姐对男方说我们刚吃过东西,我与表
姐可以share一份食物,这样,那餐饭就很不合理地变成了男方请客。表姐也聪明,接
下来便说去看电影,表姐请了她和同事,我坚持买了自己的票。事后我觉得我若是男方
,换位思考一下,即使我对女方有些许好感,我也不会继续了,因为介绍人这个前同事
熟人的做法尚且如此,女方又会又可能好到哪里去呢?我对大表姐的做法心里也多少会
有看法吧。

前边没有讲得很清楚。我最终与大表姐淡下来出现在我把自己的profile改成了与表哥
神似的英文名字中夹着中文名字时。大表姐不信任我,以为我是混不下去了想牵上拉上
表哥帮我解决身份问题。之前一两个周我刚去过表姐家一次打扫卫生,那天我又一如既
往地买了豆制品去找表姐,表姐与姐夫的车都在却没人应门,往常他们都在的,我便打
电话给表姐,但电话表姐也没有接,当天都没有给我回电话。我终于是觉得表姐就是在
故意甩我呢。对我来说在建立信任的关键时刻我总是多次遭到表姐们的遗弃,而我与她
们我努力在做的也只是从物质上不让她们吃亏而已,终于还是觉得这样太累没意思,便
也就淡了下来。

或许舅舅批评我批评得对,我与大表姐之间可能真算我错,我与表姐关键时刻槛迈不过
去时我便是先喊停就此打住的一方。但对我来说拥有一个好朋友就像遭遇一场爱情一样
珍贵,是需要缘份的。既然我无法从心底接纳她们作朋友,我也不愿意伪装自己哈巴狗
般地围着她们转,实在是没意思,这也是无法勉强的。
\section{再说自己}
\label{sec-9-83}

在做一个重大决定之前,我一定需要足够的时间来想清楚,我到底要不要这么去做,值
不值得,以后会不会后悔,我需要学会对自己的选择付责。与舅舅谈话之后,我想清楚
了不少事情,当我理清舅舅一家同大舅一家的关系,当我理清楚我与大表姐的不协调,
最终,我更需要理清楚的还是我自己。 

对我来说,或许重拾父母亲情的过程跨越了太长的时间,花费了太多的代价,以至于到
我十九岁真正重新得到它的时候,与亲姐姐们之间的关系就已经略显淡薄。自己亲姐姐
在我心中的份量尚且如此,何况大舅舅母表姐们呢?

如果我说那年我把自己的留学申请材料一个包裹寄到小表姐家期待大舅舅母能帮我分寄
到几个不同的学校,对于我来说是一场豪赌,大家可能能够猜测到我性格中的“羡慕嫉
妒恨”吧。我不知道这点儿的形成原因起点在哪里,只是觉得生长在农村最底层大环境
下不小心形成这么一个缺点大概也不难吧。长大后的我看得稍开一点儿,每个人都有缺
点,我又怎么可能没有,只是尽量学会去正确客观地value, appreciate别人的价值就
可以了。

当然,当我收到offer letter,知道大舅舅母没有一把火把我的申请材料烧掉的时候(
可能出于那么多年对大舅舅母家的情感疏离吧,我也不知道为什么UPS寄材料的时候会
隐约有这种担心,虽然换位思考我若是大舅舅母,我一定不会做出自己想像出的那种事
),我对大舅舅母多年的积怨、不信任便化解了,他们仍然是我心底最该感激的人,虽
然这份感激并不来自于大舅舅母来小表家临行前在北京请我们吃饭、以及我在这边读书
还是穷学生时舅母悄悄塞进我衣箱的三个红包这种互动。如今的大舅舅母都八十左右了
,我想改天我若有机会回国我一定会去探望他们,明明白白地讲述表达我对他们多年的
误解、不信任和最真挚的感激吧。

在我的情感结构里,亲情最重,重到可以成为我的精神支柱、成为支撑我战胜困难的力
量和勇气;爱情次之,它将会带来自己内心的平静和一生的情感幸福,甚至也间接决定
以后工作事业的发展;友情对我很珍贵,但它也是需要缘份的。一生中能有几个好不错
的好朋友也算知足了。而这三者中我做得最差的便是友情。可能是成长过程中的情感扭
曲吧,学生交友过程中,我能保证物质上的互动,却很难做到情感互动,因为我很冷!

写到这里,大家也该知道我是承认自己冷血的。那么那些年究竟是因为自己的敏感导致
自己与父母姐姐感情疏远进而冷血,还是自己天生冷血才导致了感情疏远,理到这里,
我自己也理不透、理不清楚、理不下去了。
\section{发现自己不可爱的时候就要开始将心比心地学习怎样不冷血,}
\label{sec-9-84}

怎样做个温暖善良可爱的人,骨子里不那么冷血那么自私自利的人,

而不是还一遍一遍为自己找借口,或者借写回忆录,来怪到帮助过你的每个人头上,

这只会令你更加不可爱。

性格决定命运。

如果你选择做个自私自利,锱铢必较,什么都怪别人,从来不能对别人真的好
的人,你必然一辈子得不到亲情爱情和友情,也不配得到亲情爱情友情。

血缘关系是没办法,你对血亲再冷血,他们还是会在某种程度上原谅你,

特别是自己的父母和姐姐,表舅也正是因为一丝远房的血亲关系,

一次又一次容忍了你,自己不善待亲情,指望着挥霍血缘关系给你的庇护和忍让,

不出几年,就会什么都不剩了,最多原谅你,但是绝对是寒心至极的。
\section{我和表哥}
\label{sec-9-85}

是的,表哥从来都没有从语言上真正承认过我是他女朋友,从来没有说过表达过他喜欢
我,除了说他把我当妹妹;可行动上我们每一步都走得很稳很踏实,除了我这个笨蛋会
偶尔添乱。

那时,我明明只是想试探表哥是否把我当妹妹,那个周日傍晚对他说,“表哥,我先回
去,你晚上早点儿回来!”结果那天舅母打了两三个电话催他,我三点钟就回家了,表
哥却因为我这句产生岐义的话,磨啊磨耐啊耐,拖到六七点钟才回家;第二天office里
告别后已经认定对方了临走时表哥说没休息好应该回去睡一下,反而又吓到了我,忙打
掩护说要赶路不休息了。那次回家,那时的我们,包括接下来的来回邮件,已经消除了
把对方当兄妹的可能了。

吃三餐饭时表哥从来不曾表达过任何情意,舅母佯装责备表哥不锻炼时,为了拉近自己
与表哥的距离,我忙替表哥说话羞愧地说,“表哥就这样就挺好的,是我太胖了需要锻
炼!”与表哥那场告别后,兴奋头上的我对表哥与我十三岁年龄差距既不理解又好奇,
便在邮件里试探性地问表哥,“Do you exercise nowadays?”那是我写给表哥的第二
封邮件,他就已经不再理我了。可当我被舅舅电话里骂、二月气冲冲地回去拖我的行李
还钱时,就看见表哥变白变胖了,拖了拖他的胳膊,也明显感觉他好结实的,后来还有
了那天晚上的核桃案。我说过的话,表哥不曾说过什么,但都落实到了行动上。

四月回去,表哥质问过我谁是舅母,我也还算明事理,老老实实地回答是他妈妈,并央
求表哥“你晚上回去时把我带回舅舅舅母家好不好”,表哥便答应了。回想一下,晚上
回去表哥都答应我以后回去不用自带牙刷表哥会帮我准备,行动上表哥已经把我当女朋
友对待了,而我连续开了十几个小时的车,竟然对自己说出来的话付不了责任,还奢求
着表哥在不久的将来会亲口告诉我他喜欢我。有些事情竟因为心底的一点固执没能遂愿
。
\section{我和表哥(2)}
\label{sec-9-86}

表哥可能怕我没想好吧,或许担心他自己步子迈得太大了,便自己往回撤,第二天不但
不起床为我送行,上午还发邮件以表哥的口吻责备我不讲礼貌,私闯他房间。回到加州
后的我想通想明白表哥的缘由,便以“You are the most cherished cousin in the 
world, please don’t drive me crazy. Fine, I will obey your rules”将局势重
新扭回来。

五月底回去前,舅舅电话里的话与表哥邮件里的默许已经产生分歧。找到表哥,表哥先
说我可以不用急着回去,我没能理会偏要回去;一回到家表哥就把我将住的小房间钥匙
亲手交给我。我在表哥面前撒撒娇,要他等我梳洗一下,待我洗完澡穿戴整齐出来,表
哥又让我尝他亲手做的蛋糕。我可真笨,竟然都没有一句表扬;表哥给我的暗示那么多
,晚上回来却还是因为舅舅把我买的礼物摆到门前气跑了。当时的我猜到他们折腾来折
腾去折腾到傍晚,他们是不希望我走的,可我心里没有安全感、要同敌人作斗争、那不
就得同敌人对着干吗?于是我跑,跑得比兔子还要快!却没能站在表哥的立场上替他想
一想,我那一个多月前看似明理实则自己稀里糊涂的话害苦了别人也害苦了自己。

八月再回去,我敲表哥的门,他明明是醒着的,却不应我,要我自己不经允许地推门破
门而入,那种亲密犹在。看着躺在床上穿着背心短裤的表哥,傻傻的我痴痴地讶异了大
半天,却愣是一个字也没吐出来,乖乖地关门躺到与表哥床平行的隔壁舅舅的小床上面
对应变。

我工作后不承认我是女朋友的表哥还是在我新岗位报到第一天发贺电。后来的几封邮件
里,我始终不相信表哥不喜欢我。他只是说不出来。可他为什么说不出来?他能说吗,
他不能说,他一说便是错!
\section{“不择手段”}
\label{sec-9-87}

我的那个同样有着曲折经历的舅舅,他经历过什么样的苦难、事情会把我想得那么“不
择手段”呢?受父母辈感情投射,我把大舅舅母想得很坏过,但最终我还是信任了他们
;而我这个舅舅,若对我没有足够的信任,他是用什么理由把表哥劝回美国来的(改天
有机会要亲口问一下表哥)?若舅舅对我有信任,他又何至于说出那样的话?难不成指
责我“不择手段”也是舅舅空口无凭、指鹿为马用来折腾表哥和我的一记手段?

仔细想想自己可能会留给舅舅的印象,也不能说舅舅就完全没有丝毫根据,舅舅眼中我
的不择手段大概是我眼中自己的进取心吧。

其实舅舅究竟在大舅家过了几个暑假,写前半部分的我想当然地以为至少从97年到大学
某年(大一?大二?大三在学校学习)都在大舅家,实际上我真正见过的也只有97年暑
假和大学某一年的暑假总共两次。大学那年暑假二姐陪我一起上楼到舅舅们的卧室向二
舅索要联系方式,舅舅把他学样的工作邮箱留给了我。

我高考那年发生的事情大舅舅母大概有告诉过二舅吧,所以那时并不知道我那些过去经
历的他们是如何评价我的,我也只能大概地作否定的猜测,难道舅舅会以为我说谎是为
了逼他把我也带到美国去?这也太离谱了吧。而有了二舅电子邮箱地址的我,便真在大
三春夏在考研与考托痛苦选择时向舅舅表达过非分想法,我的那封邮件舅舅没有回。

来到美国后的我又已经经历了八年的历练,已然可以坦然面对高考前后的那段经历,同
舅舅聊天时便对舅舅讲过那年十八岁的我见到舅舅之后所发生过的事情(除了硕士老板
那件之外,except),而之前的,关于爸母吵架及原因,那时的我还没有意识到相关性
、重要性,还不曾想过要告诉任何人。
\section{“不择手段”(2)}
\label{sec-9-88}

舅舅觉得我不择手段,我想可能也与我对一些事情的滋滋以求有关吧。

大表姐为了下一代的幸福可以三四十岁到一个完全陌生的环境重新开始;而我为了自己
和亲人的幸福,为了自己心里的那点儿安全感,总想要努力努力再努力,其实在别人、
在舅舅眼里我大概也就是瞎折腾吧。我的舅舅生性闲淡,对自己的专业有着纯天然的热
爱,之前在两所学校短暂工作过,来到现在的学校,日子便安定下来,一干就是四十多
年,直至退休。

不知年轻人是否都有过一段激情飞扬的黄金年华,反正我有过,国内大学和硕士期间的
前后七八年时间。那时的自己心里可能有喜欢的人,但心理上、精神上始终没觉得我身
边需要有一个男朋友这样的角色。

大学某年的12月7日,武大学生绘画摄影展在我们学样广场上铺开了。因为是学生作品
,没有了大师水平的高深莫测,所以如我等凡夫俗子都很容易欣赏接纳。作品总共可能
有五十副左右吧,我现在还能记得也只剩下几副了。

一副素描取名<收获>(<读书>?),画面上一脸上爬满皱纹的妇人坐在什么地方膝
盖上双手捧着一本揉搓得像破布卷儿般的书,聚精会神,旁边摆着的她盛放破烂的袋子
反衬得她的目光更加地安祥柔和。<小猪赶死队>是三只快乐的小猪耸拉着脑袋一字排
开地往前冲,他们快乐的哼哼声仿佛能从画面里传出来;<水晶之恋>是冰凋,画面晶
莹剔透大概是摄影作品吧。

学生的作品并不复杂,可能是很简单很轻松地契合了当时自己那努力进取、滋滋以求的
心吧,那天中午几十副作品看下来,我的心灵已被洗涤过般滋润清新,我的眼中便也有
了雨过天晴般清晰澄明的世界。往日里下午都昏昏欲睡的我那天上课发言也特别极积正
确。那天那一次的绘画欣赏给我留下了神奇的记忆。后来加州三藩,我刻意去艺术馆再
走过,但没有了对大师作品历史知道的了解,大多我都欣赏不了,还是只对我们国家的
各种陶瓷、青花瓷感觉熟悉一些,后来这块便也不再多探索了。
\section{“不择手段”(3)}
\label{sec-9-89}

仔细想来表哥与我还真挺像的。舅舅说表哥有表哥的理想,表哥没什么物质欲望。后来
工作中的我对小公司前同事的观点有了深刻的体会,终于意识到能有一个可以与自己有
效交流沟通的老板是一件多么惬意、事半功倍的事情,那么年轻时的表哥随他老板远走
韩国几年里发表了二三十篇的文章,我还能把表哥的出走狭隘地归结、假想为表哥去韩
国是为了讨美女吗?

舅舅口中表哥没有物质欲望却被我按到了舅舅头上。08年圣诞节大表姐送我一件短袖衬
衣,我还在心里挑剔非纯棉;去年8月气不过舅舅说要打911把我赶回国,恶狠狠跑回去
要“报仇雪恨”一验真假的我躲到了舅舅床上避难。说是床,实际上就一matrix放在了
地板上;不躺上去不知道,躺上去盖上薄被子的我恨不得把舅舅的被子扔了,居然是化
纤的!我同表哥一样,对房子对车都没有过高要求;可我对被子、睡衣等贴身用品就很
挑剔,表哥的衣服被子我就很喜欢。

舅舅这个贫贪的糟老头也向我索要过幸福,虽然没有明确明明白白地说。我对舅舅说我
从小到大亲眼目睹我爸我妈一辈子辛辛苦苦把我们姐妹四人拉扯大,太辛苦太不容易了
。我希望自己能让他们过上幸福晚年。舅舅说他这一辈子也很辛苦,希望也能够过上幸
福晚年,没有多的要求,只要能吃饱饭就可以了。那时的我心底也不屑过,舅舅说他一
辈子很辛苦,那舅母是做什么的,没给他带来幸福吗?我一辈子的幸福还不知道上哪儿
去找呢!后来真正喜欢了表哥,用过舅舅的被子,便也知道舅舅的标准真的不高。
\section{“不择手段”(4)}
\label{sec-9-90}

看过韩剧爱情剧就知道剧里总有一个女二号,不顾一切、不择手段地争抢所有属于女一
号东西、物品、爱、荣誉以及男朋友。我不是女二号,但我也确实不顾一切地努力追寻
过我想要得到的东西,比如,转专业后争取奖学金。

我是穷学生,出国后我的贷款最高纪录达到向大姐二姐无息各借两万元。换专业时,除
了那个专业我确实非常有兴趣、比较容易survive之外,另一个重要原因是这是学校新
兴发展的小系,所以那时基本上系里的国际学生都可以免外州费。换专业时的舅舅听说
陈述我转新专业的原因后,表达过他希望我转accounting的意思,舅舅说他可以借钱让
我读accounting。但accounting在我眼里涉及到太多语言问题,加上那个专业我争取不
到一分钱的贴补,两三年下来至少两三万美金的学费对我来说是天文数字、压力实在是
太大了。我告诉了舅舅我的想法,他便不再多说什么。作过一年多守财奴的我多少攒下
些钱,加上免外州费,再向亲戚朋友借上一些,最终还是挺过来了。

但是当时转专业的第一个学期,当我有机会可以争取奖学金的时候,我还是不折不扣地
争取过。不对,不折不扣还不足以形容我争的力度,因为我清楚地知道没有奖学金我的
日子会有多难。可能是因为我是转专业半路出家,可能是因为我是女生,不管是因为什
么原因吧,最终我并没有得到奖学金。或许也因为这事与系里某些老师结下梁子了吧,
后来前后两年时间里我都没能有机会进入系assistance center,而那时与我前后入学
的其它所有学生都至少在assistance center工作过,包括语言、学习不如我的其它学
生。我想,那时那件事情对于我来说更多的是一种被强权隔离的孤独。经济危机后后来
系里给我机会,若跟某个老师作毕业论文可以让我延期毕业,但我读书读够了,我想毕
业,再延长一个学期,对我来说也只是有的人活着,他已经死了。

毕业工作后的我从多年的经济压力下解放出来,再重新看待学生时的那些人那些事,觉
得自己很幼稚。我读书是为奖学金读的吗,我是为老师读的书吗?后来意识到的另一点
就是在美国的文化里,desperate是不礼貌的。而当年我努力争取奖学金的那段日子显
然是过于desperate了,会无形中给别人给系里带来压力。工作后的我已经把钱看得比
较开了,算是为自己当年的不当形为表达愧疚歉意,表达对老师们的感激,便在收到捐
款邮件后划了\$1000给系里,让教育培养学生,也让过去的都过去吧。
\section{“不择手段”(5)}
\label{sec-9-91}

回表哥家时有一次舅母说,舅舅在写一本书,我是不大相信的。舅舅这么大年级写什么
书,要写想写早该写好了,不是吗?去年八月当舅舅打了911之后,当社会舆论剥夺了
我幸福生活的环境时,我终于还是大概地写了自己的成长故事,写了关于自己的传记。

那么,舅舅又究竟为什么要我写书、写传记呢?舅舅并不知道我十八岁之前的故事啊,
是什么心理在驱使这个糟老头需要我这么做呢?我还是有点儿线索的。去年二月的时候
表哥给我讲小表哥的爱人走掉,他们家一朝被蛇咬,十年怕蛇毒。别看我那舅舅一天到
晚乐天派,吹虚表哥那时相当于是个小天才,可舅舅还是怕的,他怕我哪天结了婚拿到
了绿卡便跑掉了!唉,亏得舅舅的主意有那么多!

我是认定表哥的,在我付出相当的努力之前,不可能轻易放弃。我一定还会回去找表哥
一次的。只是,我要怎么做才能让自己的胜算多一些呢?

既然舅舅想我写书,那我也就去写书。我还有什么可写吗?有的,我保卫战的故事基本
只讲学习的过程和结果,我的性格形成是空洞不完整的,我还有一堆一系列的亲情爱情
故事可以写,而这个故事也能很好地解释我为什么会喜欢上表哥。而且这个故事也是有
进步意义的。 退一万步说,即使这个故事不能帮我赢得表哥,它也能帮自己洗清舅舅
强按到我头上的“不择手段”的乱帽子。

于是,我鼓足了勇气,从舅舅家回到加州几天后,开始续写自己的故事。我的续集故事
前一天晚上刚写了个序,第二天表哥的联系人就从21个变为26个。我知道,表哥从来都
是支持我的!
\section{虽然这些话说得也重,但已经是我写这个系列故事以来听到的最忠肯、最能听见耳朵的}
\label{sec-9-92}
话了。

来到美国后,在努力去忘记一个人的痛苦里,在异国他乡的孤独里,我也体会到了国内
硕士时那个朋友的珍贵。也是从那时候起,我也学着有意识地结识、帮助身边的人,也
从学校中国人圈子里长期的相处中有了几个比较要好的朋友。

我也一直认为性格决定命运。也正是因为有过这么多年这么多次的弯路,付出过惨重的
代价,近年来感觉自己趋中向良性发展。这次写这部分的故事,让自己总结了很多,也
认清自己、学习了很多。

说指望挥霍血缘关系寻求庇护和忍让对我来说还是说得太重了,我想作为农村长大的学
生,我更多的是不怎么会做人,一些事情上还无法做到设身处地地站在别人立场上替别
人考虑,所以做出来的事情会比较难看,但我自己本身并无霸占之意。

我不觉得自己与父母姐姐们有任何矛盾。可能是因为自己心里没有安全感吧,所以总是
想攒钱存下来,给爸妈寄的钱不多;但关键时候,比如爸爸出意外医疗费还没有保障的
时候,我能站出来寄八千块钱回家给爸治病(出发点并不是为了还两个姐姐钱),我想
我没有做错什么没有不对;现在还了二姐的两万块钱,剩下三万大姐说帮我存着。姐姐
没说要,我也还没说还。我清楚地对姐姐说过我不可能久下姐姐的钱不还,只可能多还
一些充当利息,没有少还的份。这三万也必将用在妈妈生病治病或是直接给大姐因为她
为父母养老了。而我迟迟没说还姐姐钱也是因为我的工作身份尚未定下来。我的出路最
终在哪里,两三个月内很快就会定下来,那时的我应该就会把钱还给姐姐了。

在我眼里,同舅舅表哥要比同大舅舅母和表姐们更亲一些,因为同舅舅表哥偏见少、误
会少,他们也都在比大舅家人更早的时间里给过我最深的感动,舅舅的911会激怒我,
我的很多做法也是需要反醒的。我会努力维护同舅舅表哥以及大舅舅母的关系。但同表
姐们可能还需要更长一些的时间吧。
\section{发信人: Mright (I am right), 信区: Dreamer}
\label{sec-9-93}

标  题: Re: 成长的故事 -- 我和舅舅

发信站: BBS 未名空间站 (Sun Jun 10 09:17:56 2012, 美东)

有点长,没看完

不同人眼中有不同的世界,你眼中的世界似乎跟多数人看到的不太一样,不过这个不重
要,重要的是得明白和尊重多数人看到的那个世界,毕竟我们是社会动物
感觉有些事情你是先得到结论,再去用事实或者自己的感受印证,这个不是很靠谱,一
些重要结论还是谨慎点为好,另外,你在感受周遭的人情世故上有些欠缺吧

比如说下边随便写的几个

1、你决定出国二姐开始不愿意经济上支持你,为什么在你来看就是“要挟”了呢,家
人帮助自需感恩,不愿意帮时坦然受之就好了

2、确定是硕士导师的所谓“小三”么,没听说过只碰过一下手的小三

3、出国读书后小镇子上传言你是“小三”,这事也很奇怪,小镇上的老外和留学生这
么喜欢嚼舌头?还是自己太过敏感了?不客气的说,谁会在乎你是谁啊

4、舅舅说工作后礼物份量应该加重,十有八九就是一句普通的玩笑话吧,不必当真

5、表哥哪儿表现出喜欢你了。。。。
一路走过来不容易,成长环境也许造成了太大影响,想要证明和保护自己,比较敏感,
放大了一些自己的感受
没经验,没办法给建议
\section{写给舅舅的邮件}
\label{sec-9-94}

from: me; to: 舅舅,表哥;
date: Sun, May 6, 2012 at 10:40 PM 
subject: ask for suggestions -- court affair

Dear Uncle, Dear Cousin, 
As the staffing company and I agreed that if they don’t sponsor H1B for me,
at least they should pay me the \$5/hr back, and I will keep my right to put
this company into court. 
So the first attachment “XXXX – Offer letter – Signed.pdf” is the 
contract I signed before I work;
The “Letter for XXXX.pdf” is the one after discussed on Mar 7, which I did
NOT sign;
The other two, one is the thing I have on hand may be able to put the 
company down, the other one is the job description they send, and they had 
never told me it is only 3 months before I checked with them on my first day
of work reported at the client. 

It has only three weeks left now. I am thinking: 
\begin{enumerate}
\item If I do have probability and I do want to put them into court, do I
\end{enumerate}
need to tell them tomorrow that I will leave in two weeks, so that the work 
will end on May xxth, instead of May xx, and I can be on a better stage if I
need to put them that way. I want to know if it is necessary for me to quit
one week earlier before the last day of the contract; 
\begin{enumerate}
\item And which are the necessary steps I should work on to start the
\end{enumerate}
process of protecting myself;
\begin{enumerate}
\item I did not sign the second one, but they added the \$5/hr for me for all
\end{enumerate}
the hours I have worked already. Will this affect anything?

Since this one is urgent for me, and I know nothing about this. I have left 
voice message in my cousin’s phone. I will try to call you tomorrow to 
discuss the problems and concerns about this one. 
I look forward to hearing back from you. 
Thanks,
XXXX
\section{舅舅的回信}
\label{sec-9-95}

from: 舅舅;to: me;
date: Mon, May 7, 2012 at 7:27 AM 
subject: Re: ask for suggestions -- court affair

I AM NOT LEGAL EXPERT, THESE ARE ONLY OPINIONS!

\rule{\linewidth}{0.5pt}
\begin{enumerate}
\item If I do have probability and I do want to put them into court, do I need
\end{enumerate}
to tell them tomorrow that I will leave in two weeks, so that the work will 
end on May XX th , instead of May XX, and I can be on a better stage if I 
need to put them that way. I want to know if it is necessary for me to quit 
one week earlier before the last day of the contract;
\begin{enumerate}
\item And which are the necessary steps I should work on to start the process
\end{enumerate}
of protecting myself;

\rule{\linewidth}{0.5pt}
IF YOU QUIT EARLIER, (opinion:) YOU ARE BREACHING THE CONTRACT YOU SIGNED
BEFORE, WHICH WOULD PROBABLY NULLIFY ANY LEGAL GROUNDS YOU HAVE.

\rule{\linewidth}{0.5pt}
\begin{enumerate}
\item I did not sign the second one, but they added the \$5/hr for me for all
\end{enumerate}
the hours I have worked already. Will this affect anything?     (opinion:) 
YES,IT DOES.

\rule{\linewidth}{0.5pt}
PER THEIR LETTER TO YOU, which has a response date of 4/2/12,
to which you did not FORMALLY reply. Since THEY HAVE ADDED the \$5/hr
YOU REQUESTED, this would make your case VERY WEAK.

AS SAID, these are only opinions. IF YOU WANT TO PURSUE THE MATTER,
GO SEE A LAWYER (for may be \$100-200) AND ASK THEIR EXPERT ADVICES.
GOOD LUCK
\section{表哥表哥}
\label{sec-9-96}

舅舅这次的回信,如同表哥回美国之前我们来回邮件的样子,写得有些长度,应该是用
心用力写的吧,之前的舅舅给我讲车,讲修车、保养车都是写长邮件的。倒是表哥回来
之后,舅舅像是变了个人,话也变少、变得很刻意,变得不可亲近,不可理喻,不可揣
度。

我多么希望时间永远停止在我积极正面的肯定情绪里,但伴随着我的故事写得越来越多
,我的工作也快结束,我若不趁着我还有工作的时候回去找表哥,我怕自己以后永远没
有勇气、没有机会再去找他了。于是,我的故事讲至一半,讲到自己自己初中高中五六
年的痛苦经历,想来我的故事也已经讲得差不多了,我便不顾一切地冲回去找表哥了。

这是我今年第二次回舅舅家,这两次都因为我的车太破了,怕开长途路上出故障便租了
车开回去的。可能人变老了吧,那次我依然周五下午就租好车,傍晚下班后早早地出发
,车里带上毯子和枕头,我还记得路上什么地方我加过油,但我晚上在什么地方、哪一
地带的rest area休息过,此时此刻写故事的我竟然记不起来。难不成我的脑袋里也有
一根筋什么地方在develop一个肿瘤?

路上我已经想好,这次回去我只找表哥,不见舅舅。回到小镇后,我自然就直接去表哥
office里去找他了。
\section{表哥表哥(2)}
\label{sec-9-97}

来到表哥office大概三点钟左右。我敲门,里面传来脚步声,我就知道表哥一定在的!

我的“表哥!”叫得一如既往、迫不及待,可这次表哥却传出哼哼声做足了架势要关门
不许我进。我这次可是专心专意、诚心诚意、全心全意来找表哥的,怎么可能让他把我
关在门外?

推门使蛮力我是干不过门背后的表哥的,我便把左手堵在门上(左手不用写字)。表哥
瓣开我的左手,我就把右手再堵上,反正我一定要有一个手堵在门上!表哥气急了也要
拿门拍我。我是天生的赌徒,我的赌注便是表哥舍不得拍我!当我把手死死地堵在门柱
上的时候,我心里已然作好了四个手指废掉的准备;当然,表哥一定是舍不得不忍心的
,他像征性地压了压门,改变了他的策略。

表哥拉开了门,他不愿让我进门那我当然是得抓住一切机会往里冲,但表哥是堵在门口
的。我平生第一次打架,没有作战经验,用尽了全身所有的力气,却还是被表哥噼里啪
啦三下五除二不知道怎么扭怎么弄了几下就被放倒在了走廊的地上,表哥也气冲冲地走
远了,消失在走廊的尽头。

我再去推门,门已然锁上。表哥已远去,自己守空门。疲困交加的我坐在地板上,双手
枕着膝盖趴着休息。走廊那边的厕所里循序渐进地传来巨大的声响,想来应该是表哥吧
?我半眯着眼睛留意着走廊那边的动静,不多会儿就看见墙角露出躲在另一侧偷看我的
表哥的半个脑袋。然后那个脑袋就又不见了。表哥过一会儿应该还会回来的吧?我继续
趴着休息。 
\section{表哥表哥(3)}
\label{sec-9-98}

蒙胧中半睡半醒之间,有人对我说话。勉强瓣开惺忪睡眼,一个身材魁梧骠憾的
policeman在说什么话。虽然表哥会打911是在预料之中,可等我写完半辈子的故事才真
正来到这里,表哥都没舍得拿门拍我,他为什么要打911?

我不知道该如何说、如何解释自己会来到这里,我喜欢表哥,这早已不再是什么秘密。
我那世俗的舅舅又为什么会想到用911暴力来解觉情感问题。我多希望我还能有机会同
表哥、舅舅再好好心平气和地谈话聊天,在与表哥好好的相处过程中将这场灾难化解掉
!可碰上这个冷血的舅舅,一切都不能够了!我不知道该说能说什么,我便没说话!

走廓的另一头有一个officer在问表哥话呢。表哥说他在墙角看我还在之后跑到舅舅办
公室去找过舅舅,舅舅要求表哥打的911;如此说来,表哥又一次地在我心中洗刷了他
的罪名。可是这个冷血的舅舅,那时的我对他有着无经复加的恨!

被问完话的表哥回到了他办公室里;躺在地板上的我奄奄一息,随他们折腾去吧。走廊
里很快来了医生,做了基本的身体检查后,医生说“你再不说话就要吃点儿苦头了”。
我心里七上八下,能有力气喘气都不错了,哪里还顾得上他?然后我左手无名指上就传
来阵痛,像是一根细线/片/环在指瓣上狠狠地切割下去,我的手指头肯定是皮开肉绽了
!表哥拿门拍我手我都忍了没喊没叫的,可现在我终于是忍不住了,除了高考那年老妈
打过我逼我上考场,我从小到大又什么时候受过这种委屈,忍不住放声大哭起来,仿佛
要把我成长过程中受到的所有委屈都哭个灰飞烟灭!那医生见我这么娇气,说我肯定是
被宠坏了,便也不再为难我。

后来他们把我先后送到医院、jail里,等待我的将是无尽的黑暗。
\section{什么角度分寸,切记:}
\label{sec-9-99}
照实写!
既然你下了这么大的决心写这么长的历程和故事,
一定不要最后因为不敢面对某些结局而篡改或者粉饰事实。所谓的考虑分寸和角度,多
多少少是因为不敢直面真实,
如果最后不照实写,你这前面的一切就没有意义了,
你也无法给自己一个交代。
又回到自欺欺人的高考前谎言中了。

\chapter{人间炼狱}
\label{sec-10}
\section{人间炼狱(1)}
\label{sec-10-1}
晚上不知道什么钟点,我被转送到了county jail。小房间里除了一条板凳宽的木板床
,一个小洗脸水池和马桶之外其它什么也没有。我除了带进去牙膏牙刷、两个塑料杯子
,和自己的隐形眼镜盒和镜片洗液,别的什么都不准,连我的外套因为有拉链都不准带
入。

房间里地板上墨迹斑驳,墙壁上还“流着油”,可能是建房什么时候流下来的痕迹吧,
看着挺寒心的;那水池、马桶也不知道是什么臭男人用过的,也不知道有没有人消毒过
。于是流落到那种破烂地方的我,就只躺在那张木板床上,不吃不喝不洗不涮也不上厕
所。

在那样一个漆黑潦倒的地方,人的思想也会跟着滑坡,我像一个飘荡的灵魂、飞舞的羽
毛也落入了十八层地狱。

某个情境下,我仿佛听大舅还是舅舅说过,大舅一辈子,十九岁和三十三岁两个阶段感
觉人成熟了不少。十九岁时的我自己与自己对话,在亲人的帮助下从无边苦海中解脱出
来;今年,我三十三岁,在这jail里开始了与自己的另一场对话。

我是相信表哥的,他只是想赶我走、离开他而已。911是舅舅要他打的,我有情绪有愤
有闷、要发泄要怪要恨也只能恨到舅舅头上。

那我今天沦落到这样一个地方要舅舅全权负责吗?不对,舅舅不需要对我今天被关到
jail里负责。是我自己开车太辛苦了,没能意识到该对police officer的话给予足够的
重视,我应该从一开始就必须回答他的问题的。既然是自己错了,那要怪就怪自己脑袋
太笨、反应不够灵活好了,不应该归咎别人,表哥和舅舅在我沦落到jail这件事上没有
任何错。
\section{人间炼狱(2)}
\label{sec-10-2}

那我为什么还会对舅舅感到愤闷、心里恨他恨得牙痒痒的呢?我心里始终觉得不平衡,
觉得自己被舅舅骗了,被舅舅家人骗了。在我对这件事的整体感受里,这是舅舅结合舅
母表哥精心设计的一个trap,我社会阅历浅,作为舅舅手中的棋子,虽然我顽强地反抗
过,但终究没能逃脱棋子被操纵操控的命运。我恨那个把我用作棋子的舅舅!

舅舅把你怎么用作棋子了?是舅舅、舅母逼我喜欢表哥的!

舅舅逼问了我毕业的时间,不早不晚地把表哥搬回来的;

10年2月我到加州的时候,是舅舅要表哥第二天早上在车里等我送我的;

10年12月回学校时是舅舅要我住他家,

我按舅舅的要求买了加了倍的礼物他还要不屑一顾,是舅舅逼我试着与表哥相处的;

是舅舅前一天晚上不许表哥与我谈话,逼得我对表哥有话不能说;

是舅母第二天说,早走晚走不在乎那一会儿,诱导我去找表哥告别的;

是舅舅舅母联手造势,说表哥是天才在国际上发表了六十多篇文章,想什么时候毕业就

什么时候毕业,表哥的毕业时间不是问题;

是舅舅说表哥又会讲中文,与我交流沟通没有任何障碍;

是舅母电话里对我说表哥就快订婚了,逼着我赶快行动,要不然表哥就成别人的了;

是舅舅电话里清楚地向我解释与表哥的亲缘关系,我与表哥并不算近亲,

也是舅舅电话里表态如果表哥与我谈恋爱,他们作父母作舅舅舅母的既不支持也不反对;

毕业后临走前是舅舅送我sleep bag,要我没事儿多回去的;

是舅舅在电话里说我坏话,我气不过去年二月回去拖我行李的;

是舅舅既把别人气跑,还想别人住他家里,还要一路跟着车把别人送出去好远;

是舅舅四月份说我人能回去就挺好的,不需要买什么礼物,没有说任何不准再回去的话;

――

所以,舅舅打的911是暴力,表哥打的911是陷阱!我是受了舅舅摆布的棋子!

黑暗的jail里,我无法控制自己的情绪,在想不出出口的极大痛苦面前,掉眼泪是发泄
情绪的唯一方式。于是,我哭,想累了,感到痛了,就哭一阵,36小时不戴眼镜地哭!
\section{人间炼狱(3)}
\label{sec-10-3}

我想问自己,为什么对舅舅、舅母的这些摆布我会有这么痛苦?答案很明显,对舅舅,
我倾注、投入了100\%的信任,我是被舅舅、被我最信任的人背叛了!

小时候,基本上从记事一两年起吧,我就一直把爸爸拒绝在心门之外;虽然我很小的时
候爸妈都很宠我,但感情上我只同妈妈亲,听妈妈讲各种各样的故事,也装了满脑子的
封建迷信;再长大点儿,多少有些聪明敏感的我就明显觉察到了妈妈、爸爸的偏心,我
变得越来越封闭自己。长大过程中我们小姊妹三个都一致认为爸妈偏心大姐,我想辛劳
了一辈子把我们姐妹都赶出农门的爸妈,很成功很伟大地完成了他们的愿望;但也受意
识、觉悟的限制,我和三姐在情感、性格发展上都不顺利。

小学毕业那场灾难我一直不愿意说。如果我承认那时的自己多少有点儿自我放弃、放纵
的成分,亲爱的读者、大家能原谅我吗?可那时的我有什么?爸爸不存在于我的世界,
我也无法再进入妈妈的世界。我的世界那么空,没有力量,没有支撑,我仿佛是天地之
间一个孤独的孩子,念天地之悠悠,独怆然而涕下。我是蒲公英,随风飘散;我是野草
,自生自灭!我说什么、做什么,会给谁带来什么好处,又会给谁带来哪些不利,who 
cares?

但灾难的结果却还是要自己承担的。那年的妈妈若对我有稍微多一点儿的关心,那时的
自己若对亲情能有稍微少一点儿的冷漠,我可能都不至于要一个包袱自己独自承受六年
;六年,中学黄金时代的六年!事隔多年,当回想高一高二的我从来不笑的时候,此时
手指敲打键盘的我终于还是没能忍住自己的眼泪无声地流淌!

后来有一个人给我带来了生活的希望,他就是舅舅,这个为我雪中送炭、拯救我于水火
之中、危难之际的舅舅!虽然舅舅几句简短的鼓励的话并没能解决我所有的问题,但至
少高三的我是快乐的。

十年后的07年,虽然几个月前的圣诞节大舅母在电话里对我表达过舅舅明知道我就在旁
边学校读书却不曾来找过我的不满,但对那个十年前给过我满满的温暖的舅舅,现在我
已经搜到office门牌号的舅舅,我又怎么可能不去找他?我找到了舅舅,于是,新的悲
剧上演了!
\section{人间炼狱(4)}
\label{sec-10-4}

我是满怀着感恩的心去找舅舅的,我的目的若只是感恩,我便不该有任何的失望。那我
的失望来源于哪里,舅舅又是如何背叛我的呢?

舅舅舅母联手造势,把我推到“去喜欢表哥”势能的最高点。就像是他们在bbs上挖了
一个坑,因着我对舅舅的信任,我知道是舅舅挖的坑,便义无反顾的跳了。但跳进去后
才发现,呆在坑里的表哥不对我负任何责任,挖坑的舅舅、舅母也不对我负任何责任。
在我眼里,舅舅打911是摆脱责任,表哥打911是脱离干系,他们是不会为我负任何责任
、做任何事情的。这便应了那句,谁动感情谁先输,在定力如此强大的表哥面前,我显
然输了!

同舅舅我像是在下一盘棋。舅舅带我翻山越岭,一路崎岖地走到今天,我以为我找到工
作了,舅舅便会同意表哥与我交往;却原来工作上我又受骗了;表哥若是路,他一定是
我最想选、最想走的路,舅舅批评我说我“自断退路”,我揣摩顺着舅舅的意思回来找
表哥,却又被舅舅命令再次打911了;现在工作上我出现意外,舅舅会让表哥把我留下
来、留在美国吗?而若舅舅不允许表哥以婚姻的方式把我留下来,我用满腔热血、一片
赤诚诚诚恳恳孝敬对待的舅舅又为什么要这么把我诱进trap里呢?只为显示表明我的不
可救药,只为推脱他作为舅舅以及表哥与我的牵连?

在与舅舅的这盘棋里,舅舅世俗,他赢了,赢得了世俗眼里的一切,物质(礼品)、表
哥的人格、舅舅的人格,对婚姻世俗的遵从;我清灵,我输了,输了世俗眼里的一切,
物质(礼品)、自己的自尊心、人格尊严、叛经离道、“不择手段”!我是用对舅舅满
心的信任、对表哥最真的感情,换来的是世俗眼里的攀亲走势、不择手段、伤痕累累。
对,07年5月是我自己找上门去找舅舅的,而最终我得到的是两次911的报应!我的内心
又究竟该如何平衡?

我的成长过程,同爸妈的亲情关系、同大舅舅母的信任关系,我走得、建立得曲折蜿蜒
,我那为一件事情吵了十几年的爸妈最终也在大哥的劝说下不再争吵。当亲情为我树立
的对人、对世界的看法都是正面的、积极的、肯定的的时候,在异国他乡,我却又经历
了一次跨时间、跨空间幅度的来自最信任的亲人的背叛,三姨被换亲后与外公外婆的关
系一辈子也没能修复,我接下来的下半辈子还能信任谁,我的生活该怎么过?

当我对人、对世界的信念再次被打跨,我真的不知道我还该、还能再去信任谁,我的朋
友、同事人际关系,和找男朋友的恋爱关系可能从此就跨掉了吧。那我将来、我下半辈
子的幸福又存在于哪里?人生真是一趟苦旅。

黑暗里,我控制不住自己的思想飞流直下三千尺,一如我控制不住自己的情绪和眼泪崩
溃诀堤。人活在这个世界上多少不容易! 
\section{人间炼狱(5)}
\label{sec-10-5}

我同表哥的恋爱是一场以一对二的对决。对舅舅的回忆会把我变成恶魔,使我疯狂;对
表哥的回忆却让我回归平静,重闻鸟语花香。

表哥一定是喜欢我的,在与表哥的对望里,这一切都显露无遗。表哥也有一副好心肠,
没有兄弟情谊的他为了给自己的弟弟谋得利益,还是按舅舅的指示去年四月将小表哥介
绍我认识,这次我给他机会可以废掉我的手,他却还是没能忍心。

表哥从来都是支持我的,10年2月清早等在车里送我来加州是,12月我临走时向表哥索
要宠爱和拥抱也是,我的工作、人际关系遇到挫折不顺时是,我续写自己的故事时还是
!看过一个爱情剧的女二号刁钻难缠、自私自利,但有一个男二号采用非常手段、无限
满足女二号的一切愿望,最终感动了她,成为女二号的男朋友。我不是女二号,表哥是
我心中的男主角,可表哥却一直在担当着男二号的角色,成为我最坚强的精神支柱,我
快乐他便快乐!

但时至今日,事情发展成为这个样子,显然这段关系是有问题的。在我如此期待关系更
进一步,期待一个男女关系的commitment的时候,表哥是迟迟做不出这个决定的。当然
,世俗里表哥有他的理由,他也从来没有答应说我是他女朋友啊,虽然他心里很想很想
。或许正是因为受世俗的约束,受恶俗舅舅的教导,表哥才会变成这样明明喜欢了,嘴
上却硬是不能表达!又或许舅舅已经是这样一个没有责任心的人,舅舅教导出来的表哥
无法make commitment也就没有任何奇怪了。

10年12月临走时,明明知道心里喜欢表哥的我,对表哥说,我现在工作还没有定,身份
还没有定,等我找到工作、解决了身份问题才再回来!可能要一年。表哥反问我要一年
吗?我知道他心底的期待,说,半年,大半年!

人是需要自我实现的,作为生活在异国他乡、拿着F1学生签证的我,当然知道工作对自
己的重要性,我又何尝不是从来都把自己的工作、自己的迫切需要摆在第一位的?我又
不是没有努力找过工作。今年二月,我以为自己找到可以为自己解决身份的工作了,还
花了三百块钱改善物质生活犒劳自己,却原来最终结果我被骗了,并没能如愿。

那么表哥,作为舅舅眼中的天才儿子,舅舅对我讲解表哥说,表哥有表哥的理想;作为
我眼中那个盼得戚戚切切的表哥,作为那个与我一样、在年轻时有着激情飞扬黄金年华
的表哥,他又何尝不需要实现自我价值、得到自己和亲人的认可?

我被人肉后,去年四月,当我的个人私生活受到前所未有的猜疑、质疑,当这个世界上
我只剩下心底那个表哥可以寻得安慰时,我追问表哥,他到底喜不喜欢我;表哥的答案
当然是不喜欢,他把我当妹妹,我终于是被表哥气哭了;可回想10年12月从图书馆出来
受到来自美国学生目光的鄙夷时,我等不及表哥为我推门,自顾自地先冲出来了,回到
office后的表哥自顾自地说他还是学生。被表哥气哭的我,一脸不解地追问表哥,他若
是觉得自己还是学生,都还没有毕业,“我等你都行,我等你到毕业!”表哥抚下头来
定定地看了我一眼,速又扭过头去、一言不发、匆匆地收拾他的书桌。那时的自己,一
句等待,说得何其轻巧!
\section{人间炼狱(6)}
\label{sec-10-6}

我的那个舅舅哦,既知他们家儿子、媳妇的婚姻要经得起时间的考验,当初11年一二月
份电话里我追问表哥什么时候才能毕业啊,舅舅又何至于说表哥在国际上发表了六十多
篇文章,想什么时候毕业就什么时候毕业,把我不容质疑地推进表哥这个火坑?

而今,我的工作因为受骗、出现意外,有可能要回国了,舅舅、表哥却没为我想好一个
出路。当初是舅舅亲手把我推入表哥这个火坑,现在他们却不愿意为我提供任何的帮助
,舅舅对我的责任心又在哪里?

08年夏天到加州后,大舅依旧沉默寡言,看他的书作他的画,两耳不闻窗外事,一心只
读圣贤书。大舅母却郑重地讲解给我听,说舅舅帮我转专业考虑不周全,我应该把博士
学位拿到,再读一个现在专业的硕士,这样改心如果不得不回国,一个海外的博士学位
在国内还是能帮到我不少。而舅舅,作为美国大学的教授,却没能为我考虑到这些。

舅舅真的只是没能考虑到这些、疏忽了吗?还是舅舅别有用心就是这么设计的?舅舅这
么设计有什么目的,他又能从这其中得到什么好处呢?就算表哥是舅舅为我设计的他心
目中的“退路”吧,这不,舅舅和表哥都不愿意为我负这个责吗不是?舅舅若真有目的
,我实在是想不出来。我来美国后,自打我买给舅舅豆浆机作为第一份礼物起,舅舅都
是逼着我去揣测他的心理的啊;与表哥相处过程中,舅舅、舅母也都是一步步紧逼着这
段关系向前推进的啊;现在这种境况下的我,内心的冤屈又能向谁述?
\section{人间炼狱(7)}
\label{sec-10-7}

可我是真心喜欢表哥的,当初人们对我有误解,舅舅怀疑我“不择手段”也还算情有可
原,那现在我的成长经历、故事都写得很清楚了,我同表哥的关系却还不能确定,我接
下来又该何去何从?我要换人吗,我就算换人能幸福吗?

后来我想通过一点儿,表哥回来前舅舅暗示我买过几次礼物,我都做到了,表现得不错
,不吝啬、很孝顺、懂感恩。或许舅舅当初的意图并非是贪求我买礼物奢求物质,或许
只是为了考验我的人品吧。后来08年夏天在加州表姐家,舅舅评价说大舅母刀工不好切
不好菜切得又慢,大舅母面前,舅舅对我切菜做菜收拾餐具等做家务能力给予了极大的
佳评。也正是因为这些,我地大舅母前后塞前我衣箱的三个红包始终没有发自内心的感
激,总觉得大舅母那么做是为了讨好舅舅;也正是因为舅舅对我做家务能力的表扬,让
后来表哥回来后的我觉得,舅舅是很希望我做他们家媳妇的,能为他们养老,人品好还
可以接济小表哥,生活、饮食习惯也类似,也没有语言问题方便沟通交流。后来跳了火
坑到现在,才明白,舅舅表哥也还需要考验我的人生价值以及个人的自我实现,虽然我
并不久缺。

对表哥,我有太多的依赖和不舍得。我的成长过程,得到的爱很少。同表哥告别那天,
表哥最开始是不允许我抱他的,只准拉手。于是我拖着他的手蹲在地上哭着说自己舍不
得走不想走;等我趴在表哥背上哭完发泄完自己的情绪,便从后面抱住表哥,告诉他我
觉得接下来的一年好辛苦,要找工作,还要换身份,将脸的两边轮流贴在表哥后背上。
我从表哥这里索要宠爱,表哥思想上、情绪上一定是有变化的,要不然当我转身到表哥
面前,表哥是不会允许我抱的,但表哥宠着我!我说了很多的话,还索要拥抱,“表哥
,就算你作为表哥,就不能抱抱我吗?”表哥没有抱我,但他一直宠着我,顺着我,要
牵手便牵手,要拥抱便让我抱。而且距离表哥那么近,我是可以感觉到表哥身体反应的
。(因为之前我说表哥就这样就挺好的,)一年多不胖不瘦的表哥会按我邮件问的“do
you exercise nowadays?”去锻炼身体,把自己锻炼得壮壮的,要说表哥不喜欢我,
打死我我也不信!表哥是用行动去表达感情,而我幼稚地停留在语言上。

除了那场告别最深的、决定性的感动之外,我这么认定表哥,表哥也是沾了我国内硕士
时朋友的光。国内时的两个朋友,从我的角度来讲,初三时的更为真挚,硕士时我享受
朋友的陪伴更多一些,但为她付出、情感上的共鸣就少很多。来美国的第一年,在自己
有怨有恨的仇肠百结和孤独里,我终于悟出对朋友应该付出的珍惜,而友情不对等下,
那个朋友竟然陪了我四年,让来美第一年的我非常感动;而三年后遇见的与朋友有着相
同属相、相同星座、相同血型表哥,就被我看得尤为珍惜宝贵。加上表哥与我的年龄差
距在我这里是加分的,一举垫定了表哥的地位在我这里是不可替代的。
\section{人间炼狱(8)}
\label{sec-10-8}

其实,我也清楚,表哥在我心目中的位置不可替代;我也清楚,10年12月那天的告别我
已经认定就是这个人了,那天的表哥也同样被我感动了,因为我们的情感需求和个性互
补。

我为什么对舅舅的怨会有这么深,舅舅真有我这一两天想像得这么坏、这么十恶不赦吗
?被我想像得这么极品、不可理喻的舅舅同样也是十五年前真真切切地帮助过我的人,
也是四五年前作为经济担保人帮我转过专业的人,我在得出一个结论之前,颠覆我现有
的人生观世界观之前,我是否应该对舅舅在现有资料和证据下,再作一个更全面、更深
层次的解读?

当那天告别后回到加州的我说出自己心里的话,当我说出以自己对亲情、友情的体验,
在感情在爱情上我决不可能委屈自己,若与表哥不能相爱,那无论如何,没有人可以勉
强我为了感恩一定要嫁给表哥!在舆论一边倒地认为我既抱了表哥就应该喜欢表哥的时
候,我可以冒天下之大不韪撼卫自己的立场,只因为我对表哥的家庭心怀芥蒂。那,经
历了那场告别,真正喜欢表哥之后,接下来所发生的一连串的事情,我还能说我是被舅
舅、舅母逼着去喜欢表哥的吗?当然不能,就像之前我说过的,像我这样一个与舅舅一
样、同样拥有强大自我和主观意志的人怎么可能受舅舅束缚去干、或者不去干某件事情
?我,不是舅舅随随便便就能指挥得动的!我也不能寐着良心把这与表哥的恋爱关系所
有的错都归结到舅舅头上。舅舅左右不了我们的关系,我真正喜欢表哥了,舅舅用十匹
马也别想把我拉回来!舅舅在我们的关系上没犯什么多大的错,他所做的也只是“兴风
作浪”、顺水推舟而已。若说舅舅真正有错,那他最大的错就在于把表哥教得很世俗,
以世俗和911法制手段强行禁止不允许表哥与我恋爱!这才是我最最最该怪罪舅舅的地
方。

那么在那场告别之前,舅舅又做了什么?舅舅让我住他家了;舅舅前一天晚上不允许我
同表哥多聊天,舅母暗示我当天去同表哥告别。其它我之前怪罪过舅舅、舅母的话言行
全都发生在告别之后。可前一天晚上表哥都躺下了,我还是第一次地硬闯了表哥房间,
依依不舍地告诉他,“我明天去学校办完事就要走了!”表哥劝我早点儿休息,当天晚
上都没能同表哥说上点儿什么的我,早前二月份走的时候表哥还特意送过我,而这次是
表哥学期末的考试周,我又怎么可能不去找表哥?舅舅前一天晚上干不干扰、舅母第二
天暗不暗示,对我要去做某件事情会有丝毫影响吗?

所以,告别那天我就认定了表哥,与舅舅舅母无关。而回到加州后的我,什么问舅舅我
与表哥的亲缘关系呢,什么舅舅对表哥与我谈恋爱的态度呢,等等也只是我天生的保护
自己的本能;而后来从舅舅、舅母那里获得的鼓励、支持或是暗示也只是增加了我自己
的安全感、加速了我认清自己、明白自己心里真实想法的过程。

我这个人啊,说明白吧心里也明白,说不明白吧还真不明白。想当初,我心里明明是喜
欢表哥的,可对舅舅心机的戒备啊,打了那么久的持久战,愁死人了!当初我不还南辕
北辙,心里明明喜欢表哥,却故意同朋友跑出去玩吃东西,舅舅打911那阵儿想不明白
,回来心里还气得要命,那谁“鸡犬相闻,老死不相往来”不都被我搬出来用过?这次
被表哥打911,打死我也再说不出那样的话了。
\section{人间炼狱(9)}
\label{sec-10-9}

好吧,在表哥与我的恋爱关系上,我可以不怪舅舅舅母任何事情。可为什么这么喜欢、
欣赏我的舅舅没有为我一步踏空、在美国万一找不到工作准备一条出路呢?舅舅既用批
评我“自断退路”和邮件“stay way”把我诱惑回去了,为什么又会要表哥打911呢?
表哥真打了911,那表哥至少这次一定不是我的什么退路,舅舅给我准备的退路在哪里
,舅舅真没为我准备任何退路?

在转专业这件事上,大舅舅母无疑为我想得更周全一些,但舅舅作经济担保人帮我转专
业就错了吗?那时,我已经是三十岁左右的人了,我转不转专业、转什么专业是应该舅
舅帮我作决定的吗?再说了,就当时的心境和学习状态,就算舅舅要我把ph.D读完,我
坚持得了吗?所以,博士学位、转专业这件事上,大舅舅母作为经历过岁月的长辈,为
我将来的发展考虑,花费了心思和精力,我应该感谢他们;而我也应该同样感谢舅舅的
乐观、对我的信任,不因为我当时有语言上的担心、学习表现不够好而不帮我,以及对
我学习兴趣的尊重,也没有强迫我读accounting. 他们都是真心为我考虑、帮助过我、
我应该感激的人。这里的博士学位,至少那时那个以前专业的博士学位不可能成为我的
选择。

那么舅舅就真甘心、甘愿这么眼睁睁地看着我卷铺盖卷回国?舅舅对我这个他喜欢、欣
赏的准儿媳妇就没有半点儿的留恋?舅舅既制造了表哥与我的相遇,那现在可能的分开
便断送的是两个人的幸福!这个问题,大概会连同我对表哥的爱情,以及表哥到底是不
是喜欢我等等一同接受时间的检验了。

正是因为我对舅舅有着诚挚的感激和解读,我就不该忽略忘记另一条信息。那时大概还
是早前同舅舅聊天的日子吧,舅舅讲到他在台湾有half brothers and sisters,我听
大舅母说舅母把她的几个亲人都弄到美国来了,我便问舅舅,他就没有把比他小那么多
的弟弟妹妹什么的弄到美国来?我印象里隐约记得,舅舅说人的发展看环境也要看天份
,他不觉得他的弟弟或是妹妹有过人的才华或者能力适合来美国发展。回想起了这一点
,那么现在凭良心说话,舅舅对学生、亲人晚辈的教育大概还是以人为本吧。那若我真
没办法在美国生存下来,舅舅当然会舍得眼睁睁地看我走啊!那我当然也不能怨舅舅什
么。我想若那天到来,我会尊重舅舅的做法,就像我会尊重我对表哥的爱情一样,这是
人性。
\section{人间炼狱(10)}
\label{sec-10-10}

被表哥打911的48个小时之后,周一下午三点,我被带到了court,同我一起的还有一个
美国男孩。那男孩对法官非常尊重,很有礼貌,勇敢、积极地承认了他的错误,惹得厅
里笑声阵阵。而我,数日来的疲乏劳累困顿,嘴唇干裂皮飞肉绽,脸上紧崩崩的像是顶
了两块大锅巴。我对法官表达了我的不满,表哥不是我的first cousin,舅舅故意骗人
了,表哥同我是可以结婚的!法官说,表哥同我是不是first cousin没有关系,有关系
的是以后我不可以再去找表哥了!听法官这么说,我便只好早早回家。我的那个自称“
大大裂裂”的舅舅哦,就这样折腾、断送了两个人的幸福!我那一年多前对表哥轻巧承
诺的等待,这时便变成了沉重的命运。

坐在室内长椅上等车的我不停地撕扯着嘴唇上干裂的死皮,嘴唇便被我不小心弄出血来
。穿着外套重新走进阳光里的我感觉很温暖,眼睛受光时稍微有些刺痛,一两天来我对
舅舅上天入地的怨与恨,如同jail里的无边黑暗永远地退后消失了,但愿我永远不再经
历那种恶魔般的疯狂和仇恨。阳光的温暖和光明很快让我变得快乐起来,也情不自禁地
回想起表哥的目光,那会让我感觉安全和放松的目光,很快又可以见到表哥了吧!

Policeman对我说以后不可以再去找表哥了,我答应了,便从车里取出洗涮用品,快速
奔向表哥office走廓另一厕的洗手间,洗掉“锅巴”,把自己收拾干净。我从容地走向
表哥的office,这应该是接下来或许一两年里或许一辈子最后一次见表哥了吧,我想我
会小心,不让911重演,我也会淡定,不增加对方的痛苦。我把自己的心情、情绪都调
整准备好了,可惜,敲了门,表哥不在没人应。而我,也还需要赶回公司上班。

公司里,我的小老板试探性地借话题打趣我,“哎,这得多丢人啊!”我心里暗自后悔
叫屈,谁让自己当时没能反应过来呢,一失足成万人笑,爱笑不笑!便也自己开玩笑幽
默一下缓和周围的气氛,一切仿佛都恢复如往常。
\chapter{结束语 Sun Jun 17 01:00:47 2012}
\label{sec-11}
发信人: deepwaterooo (梦魇), 信区: Dreamer

标  题: Re: 成长的故事 -- 我和舅舅

发信站: BBS 未名空间站 (Sun Jun 17 01:00:47 2012, 美东)

亲爱的读者,我的故事讲到这里就该结束了。正如大家从这个故事里看到的,我的成长
过程充满了曲折,虽然自己、周围的亲人都没有一个十足的坏人。我从来不觉得自己是
坏人、有什么恶意恶念去伤害任何人,但我始终没能想明白我有这么一个坚强潇洒、深
沉内敛、拿得起放得下的爸爸,而我心目的男朋友形象却竟然是小混混,直到这次我写
这个故事,意识到我把叔叔家的堂妹误称作了表妹时。爸妈为我们姊妹在学习上树立了
大表姐小表姐两个活生生实实在在的榜样,他们却没能意识到,我们受妈妈爸爸的感情
投射,情感发展性格形成时,我们有着另外一个榜样――表哥。回想一下幼时自己最想
成为的人,当然是表哥啊,集外公外婆、舅舅舅母、妈妈爸爸、大姨四姨五姨姨父等万
千宠爱于一身,还受我们家四姊妹、四姨家的表妹、五姨家的小表妹等众多姊妹的亲睐
。而那时的自己心中没有爸爸,我能不喜欢“暗恋”表哥吗?

在这个庞大的亲情故事里,外公外婆算是成功的,他们在妈妈六姊妹的心中播种了爱,
后来外公外婆离世后,妈妈的众多姊妹之间都还一直非常团结;但他们也是失败的,没
能突破那个时代固有的局限。他们重男轻女、对表哥的过度宠爱不仅使得表哥后来的生
活备受折磨,连我也间接地被表哥带坏了;他们让每个子女都得到了爱,却没能培养他
们承受挫折的能力。妈妈如此,一件应当被原谅的事情被她拿来同爸爸吵了十多年;三
姨也如此,到外公外婆临终可能都没能从心底原谅他们。

而外公外婆,他们又有着怎样的故事?斗地主年代大外公二外公成为批斗的对象,为什
么我的外公又没被批,那个大家庭有着怎样的故事?外公同父同母四弟兄,他们四弟兄
之间有没有什么矛盾,那个尚未成年结婚的三外公早早离世的原因是什么?我的外公是
真心爱好拉二胡,还是把他同父母、弟兄之间情谊感悟转化成了二胡的哀婉凄凉?这些
,或许将来我还有机会向大舅讨教,或许也就成为上上一代人的故事失传永远没有答案
了。

我这边的家簇亲情爱情故事如此,表哥那边呢?舅母究竟是什么性格的人,我的舅舅那
些年的颠簸流离又究竟发生了什么,改变了什么,是什么把舅舅成就为一个乱世枭雄?
大表姐说苦难会改变人,我的记忆经历里,劳动是创造美的,情感教育、性格发展会改
变人。以我对舅舅的了解,舅舅观察力敏锐,很浪漫,却不懂得无法体会别人的心思想
法。而舅舅舅母又是以怎样的教育方式抚养了三个表哥,他们成就了表哥哪些方面的性
格,又失败在哪里?这些问题的答案就只能等到我与表哥真正走到一起的时候才能揭晓
了。

我的情感归宿自然是在表哥这里,我的命运也只能是等待。等待是沉重痛苦的,等待的
日子却也可以让自己过得充实快乐。

妈妈这次脑溢血生病,吓怕了我们众姊妹。先前一两个月独自在农村老家住的妈妈也被
大姐姐夫接到了市里随姐姐们生活。之前提到的同大哥电话里的唯一一次顶嘴也早已成
为过眼云烟,我们众姊妹之间的感情还是很好,紧紧团结在妈妈大姐周围。我会尽可能
地从经济、精神上去照顾好妈妈的晚年生活,而日常起居里真真切切照顾妈妈的重任就
落在了姐姐们的头上。

而我也需要去充实沉淀自己,希望自己能够生活得更踏实。等我有机会回国,我想我会
去医院,去做一次彻底的身体检察,如同那封给表哥的邮件里说的,之后也会锻炼身体
,希望能把自己准备好,结婚后可以尽快怀孕生小孩。自己亲身经历过痛苦,我希望将
来我能有幸作一个成功的妈妈,把爸爸的精神精髓传承下去,也会重视对孩子的情感培
养、成长教育,希望我能把下一代培养成有能力获得幸福的快乐的人。

如果我发扬表哥当初那把桔子长回去的精神,这一次我一定会同中介公司磕个头破血流
,但我原本没有多少胜算,我也不打算那么去做。成长是挣脱封建迷信等束缚的过程,
也是培养完善自己,成为更好的自己,培养自己从心所欲不愈距的过程。

我把自己的成长故事写出来,贴在这里,以亲身经历的血泪教训告诫大家去关注小孩的
培养和情感教育,也希望华语读者和朋友不再犯我同中介公司之间类似的错误,同时也
希望大家能够不再纠缠我的过去,能体谅包容我同表哥违背世俗的爱情,希望能如舅舅
所愿,在社会世俗的包容下,我同表哥能早日走到一起。

谢谢大家花时间阅读这个故事!

\chapter{解释与短暂交待几件大事}
\label{sec-12}
\section{最后一次解释}
\label{sec-12-1}
发信人: deepwaterooo (梦魇), 信区: Dreamer

标  题: Re: 成长的故事 -- 我和舅舅

发信站: BBS 未名空间站 (Sun Oct 28 19:55:36 2012, 美东)

最后一次解释

没有想到我还需要再最后一次站出来为自己澄清申明一次。挣扎了很久要不要再站出来
说话,最终还是希望能站出来把可以说清楚的事情说清楚。在以往的经历里,因为与国
内硕士导师之间的事情,我后来在阴影里生活了很久。当不幸的事情再次发生在自己身
上,我希望今天的自己能够学会保护自己,能站出来把可以说清楚的事情说清楚。其实
所有的答案都在以往的故事和人肉结果里。

先交待重点。我与那位专业里的前辈没有那层关系。作为一个从农村走出来的学生,我
没能避免农村学生的普通缺点:个性比较耿直,人际关系僵硬,不够聪明灵活,略显木
纳。正如之前的人肉结果所显示证实的,男女之间,我心里一直没有明确的界限。现在
想来,那位前辈说过还比较有印象的事情是有一件。就写几件小事吧。

我的第一份工作找得比较顺利,没有体验太多痛苦。所以11年2月当我需要找工作的时
候,我不紧不慢地找着。那段时间,与他之间,是他先以邮件的方式与我联系的。因为
那时的自己没有经历过太多找工作的痛苦,当他以中介公司的身份与我联系的时候,在
我的理解里,他和之前的中介公司一样,他帮我找工作,他自己的公司也会从中获利,
中介公司与我这样的求职者之前应该是一种相互合作、互惠互利的关系。他打来电话说
他出差来加州,只有那个周末那个中午有时间可以同我好好谈一下。我想别人中介公司
帮你找工作,你当然得迁就顺从别人的时间,于是去了。这是大家都能看到的。

大概是一家意大利餐厅,各人点了各自的东西,他另点了一份蛋糕用公勺共享。反正乱
七八糟的问题,他问了我都按自己觉得最好的方式、答案回答了。那份蛋糕很好吃。我
农村孩子的习惯还是没有改变,吃完饭后我把自己没有吃完的那份打包带走了。但我没
有带蛋糕,因为我注意到后来不知他是不是忘了,用了他自己的勺子。蛋糕虽然好吃,
但想到他用过他自己的勺子,我终于还是介意,没有再吃和带那他吃剩下的蛋糕。

回到他公司后,他问了一些专业上的问题,答过了,我就回家了。那天中午房东也知道
我拿着餐盒回来的。后来因为工作签合同的关系我又到过他的公司大概两次。每次去他
都是在外面同女同事(朋友?)谈事情。而我也只能同他办公室里的唯一一个男小秘聊天
。我心里也打过鼓,这人为什么愿意帮我找工作,但我保护自己的方式是去了解他,从
他小秘身上去了解他。于是他们在外面谈事情,我就在里面同那男孩热热闹闹地聊天。
我大不了他几岁,他大概也是狮子座的吧,聊天的过程我们都不觉得生份。我问他,这
前辈说是能帮我推荐一份工作,他会帮我推荐吗?他之前推荐的人多不多,你知不知道
一些案例?那男孩没有给我醒目的回答,我也就没往心里去。

有一次大概是工作已谈好了吧,我就要离开他公司回家,走出公司楼的路上他拿了张印
有LV包包图片的纸问我周末有没有时间能不能陪他去选个包包?我瞅了瞅图片心里第一
反应是觉得特别搞笑,呵呵,我一农村长大土得掉渣的女孩子,用过最贵的包包不到\$
50,居然有人要我陪他去选LV包包,我就直接说,我没买过我不会选。他说他会选。在
我面前,心理上,他是帮我推荐了一份工作的人,他身上还有专业前辈、公司老板以及
年长异性等多重光环,而作为一个心里没有明确男女界限的人,我没能分辩出话外话。
我想,在11年3月底4月初的舆论暴发前,如果他真打电话要我去陪他选包包,也许我真
的就马大哈地去了吧。

签过合同之后,我再也没有见过这个人,只因为工作上的关系来往过几次电话。在11年
3月底4月初的舆论暴发前,我没想过这个人可能会有的企图。所以新的工作开始后,有
一次回南湾打球,当一个姐姐问及我的工作,我大大方方、坦诚有一位专业上的前辈帮
忙推荐了我,没有任何遮掩。

舆论暴发后,我明白了当时的话外话。虽然后来与表哥、舅舅的关系完全在我的意料之
外,便当时心理上我是有表哥的。退一万步,我也还有表哥,我不可能走上那条路,便
在心理上坚绝与这个人划开了界线。五月底回去看表哥,与表哥、舅舅间的意外事件导
致我可能会丢工作时,对工作上的直接上司、他,我没有多哪怕是一句的请求,工作完
一天,当天解扉,当天回家。

这便是我与那位前辈间所有的过往。大家有时间搜、愿意搜愿意暴随便搜随便暴,我就
不信没有发生过的事情能被说成发生过。而我的缺点,大家早人肉过了,吃一乾长一智
,以后在这方面警觉些便是了。

另一件:专业上的那男同学是我朋友,我们之间只有朋友情,没有男女情。我早说过来
到美国后经历过痛苦,也精心培养有了比较好的朋友。他是我来美国后最好的朋友。但
他有他选女朋友的标准,我有自己的情感需要,我们不适合作男女朋友,但是是很好的
朋友。
\section{朋友}
\label{sec-12-2}

发信人: deepwaterooo (梦魇), 信区: Dreamer

标  题: Re: 成长的故事 -- 我和舅舅

发信站: BBS 未名空间站 (Sun Feb 10 18:28:40 2013, 美东)

朋友 

生活有时候真的让人觉得很无奈。 很久之前看过一个电影,讲一个单身母亲领着独女(
因为小镇文化氛围的原因)在不同的小镇之间游离奔走,最终遭遇爱情,在一个小镇安
定下来。之前 (在11年11月打算写这个系列故事的时候)那种周围所有人与你的敌对,
会让你觉得如果不把故事写出来,如果不把事情的来扰去脉解释清楚,呵呵,在美国,
你就完全没有credit,别想再混下去了! 现在我的故事写了一遍又一遍,在这个小镇,
我前后已经生活了四年了,故事也写了,可明明说与系里的那个男生只是朋友,却还要
被大家颠三倒四,反复怀疑,把账都算到我头上,看来多年前那算命先生的话一点不假
,我的逆势还真不是一般的多!

有的人对暧昧极恶如仇,比如我(我的情感向来很明朗,喜欢就是喜欢,不喜欢就没有
办法了),有的人却趋之若骛,享受至极。这个朋友,在我眼里,暧昧是他枯燥学习生
活的调味口,暧昧在他是一种本能。

再来写这个,真没意思。我所能做的,也只能把曾经说过的话再说一遍:我与他没有男
女之间的感情,我们会有不同的感情归宿。上一次澄清的时候,关于他的部分,打电话
与他联系过,他不希望曝光他的生活,所以我说得极简单。而现在,当大家把所有的账
都算到我头上的时候,为不辜负大家的关心(与敌意),我就不得不把一切都给解释清楚
了。

2007年8月我与他们几个新来的男生一起组了family plan,他是成员之一。那时的自己
还没有经历过表哥的爱情,凭外在处朋友,理直气壮。他比我小,但在这个学校里已经
算是与我年龄很相仿的了。作为plan里的“大姐”,他们大的小的一众四个男生在我家
里吃过很多次饭。我邻居的小老乡美女嘴甜,提醒大家说“看他们俩个像不像学生家长
”,心里都还瞒开心的。

后来大家聊多了了解得多了,就发现每次一提到同他一起坐飞机来的富二代美女,他就
一脸甜笑。正好那时(08年)我的同学是那美女的好朋友,于是我们朋友们摄合,我请大
家到我们家吃火锅,给他们创造点儿直接接触的机会。人家美女有情有义,玩牌的时候
还是留下了挺明显的暗示,可是他老人家得到了享受了别人的暗示,却迟迟没有行动。
后来同学告诉我说美女等不下去早就对他心里起火厌弃。人家美女家里父亲有自己的公
司业务,她是独女,脸蛋漂亮身材又好,追她的人无数,都不动心。他还真不知道天高
地厚,我也只能恨铁不成钢。

09年family plan里的一个男生学习上受到挫折,一个人住在旅馆里。我打电话给这个
朋友,问他这个男孩能否住到他家里(他当时一个人住one bed room apartment),他比
较世俗,对对这个男生有些避之不及的感觉,很不愿意帮忙。电话里我给他分析男孩的
困境现状以及可以有的结果,最终他答应了。男孩对我说他有十几本英文他们专业的教
材,太重了,不要了,说要留给我。我说我不喜欢买卖东西折腾,让他把课本留给朋友
,没准这朋友能帮他在网上处理了。男孩走的时候两件还不错的大衣外套,床单枕头等
都留给了朋友。不知道男孩走的时候课本是如何处理的,朋友从来没有对我提起过教材
的事情。

08年春天学统计后,我们有好几门课是一起上的。我与他除了是相距几个小时公汽车程
的老乡外,我们是华中农大的校友,他比我低一级,后来我们先后到北京读完硕士再呆
一年,最终在这个狭小的学校里认识,组到一个family plan里。真正一起上课,我们
就变成了真正意义上的同学,与他一起同学同桌过,讨论问题过。

09年春夏,我失恋了。走到分手那一步,该看清楚的已经看得很清楚了,分手是当时应
该做的最理智的事情。只是该如何面对以后的生活,当时我觉得自己以后再也没什么希
望了,我这辈子完了。之前我们有过不少暧昧,我也知道那段时间他其实正与另一女生
比较暧昧,但或许是在巨大的痛苦面前无法面对,或许是急于为自己寻求努力生活下去
的一线希望,我打电话给他说我要过去找他。我跑去跟他表白了。

到他家时我具体说了什么也全都忘了。他拒绝了我我似乎也不意外,最感到意外的却是
拒绝的理由。他拒绝的理由很直接很明白:我不够漂亮,他对女生的外表有很高的要求
。他说我们可以作一辈子的朋友,好朋友,但我不符合他选女朋友的标准。意外吗,意
外,嫌我不够漂亮,从来没觉得我能达到他选女朋友的标准,却还暧昧了那么久(因为
有饭吃?)! 也不意外,富二代美女都没能让他真正行动起来,我又算哪来的狗尾草皮
皮虾?

09年夏秋,与他暧昧过的女孩要去其它的州读博了。她与我当时的roommate比较熟,临
走前来roommate这里作客,我也找机会与她聊了聊天。或许我这人耿直吧,我把我对朋
友的了解全分享给女孩了。我对女孩说,以我的了解,他是一辈子只谈一次恋爱的人,
除非女孩子让他非常动心,否则他绝不肯轻易迈出那一步。他的感情只有可以猜测到了
一两种结局:工作后从国内搬一年轻小美女,或者工作后找一个长相很不错的在这边读
书或是工作的人。他是感观动物,曾与他roommate出巨资合买一个大电视只为周末偶尔
可以看看电影。我给女孩讲了富二代美女的往事,虽然她比我年轻几岁,还是劝她到新
的学校后尽快忘了这个人,有合适的机会多为她自己考虑。我没有任何的恶意,女孩也
给我讲了他们之间的暧昧,从来没有一句明确的话,最接近的一次也只是他带她去他老
板家参加了一个party吃了顿饭,但那之后,心怀鹿撞的女孩终于再也没能等到下文。

女孩告诉我的他带她去老板家吃饭的事情让我对这个人耿耿于怀了很久。是什么原因让
他带女孩去他老板家,这意味着什么他明白吗?他只是为了迷补他上女孩家吃过太多次
饭的心理亏欠所以找个机会带她吃饭,还是他是真正明白的,想要迈出爱情这一步的,
但最终还是犹豫了?我在第一个答案前停留了很久,最终还是选择相信第二个答案,相
信他有选择障碍、选择恐惧,把他一辈子可能只谈一次恋爱与他选择的困难关联起来。
后者在高考志愿、出国申请学校上体现出来,他自己也分享过。

大家也看见了,之前的清葱岁月里,我理直气壮过,努力过,追过,但真正意识到我们
观念上的不同,人生没有交差,或许哀莫大于心死吧,我从此就对他死了心,绝了缘,
再不来电。女孩分享过她的经历,我心里发过誓绝不作第二个被他带到老板家吃饭的人
。他可以将暧昧进行到底,那是他的本能,但我了然于胸,根本就不动心。后来秋天学
期开始时,在其它朋友家的party上,他继续暧昧着,那是他的事;我受另一朋友的鼓
舞,最终从失恋的痛苦中走出来。他还是以前的他,但因为对他多一层的了解,他的暧
昧对我无能为力,真正能帮到我,真正帮到了我的人,只能是其它人。

09年秋天,我向舅舅借了四千块钱,是借了两次每次两千,第二次借的时候,舅舅或许
稍嫌我花钱太快吧,略感不适,后来就问朋友借了五百。朋友的基础是从这个时候打下
的。一般人,又有谁肯借钱给谁?当我向舅舅借钱略感不适,能从朋友这里解决困难,
我又为什么不可以依靠朋友?再说,我们想法观念不同,没有人强求什么,朋友还是我
们关系的一条很好的出路。从那时候他肯借钱给我,我就还是把他当朋友了。10年2月
我去加州前,手上钱不上多,我再次向他借钱。他取好现金拿到我家给我,数时说数好
的五百怎么只有$495,我会意,赶快说,“没有关系,等我工作后,第一时间给你转$
1000到你账户上。后来工作后,我确实这么做了,先还了借他的钱,舅舅的四千相对比
较多,是后来才还的。

10年12月见过表哥,也请大家一起吃饭,他向我们show他有多少张信用卡,我直接打趣
他,说有多少张信用卡关系不大,没什么难度没什么意思,他要是真想show, 就赶快找
个女朋友show给我们看看吧! 回到加州后,他打电话给我问那个我说买的滤花椒八角或
者茶叶用的小球我帮他买了没有?我想他大概从后续的舆论里嗅我与表哥之间的爱情味
道,想要与我确认一下吧,虽然我买好了在表哥家当时忘了给,虽然我想到我还可以把
表哥的电话号码给他让他自己上表哥那儿去取,但猜测到他可能会有的真正意图,我直
接说我买了在表哥车里,当时吃饭后忘了拿给他们,就没再多说什么了。

时间飞转到12年,在回国还是回学校读书的决定之间,我打过电话给他征求意见。回国
,我不甘心,舍不得表哥;读书,一方面学费不够,另一方面可能还会有学习上的困难
。他说国内现在发展很快,他劝我回国;或许同别人交流的过程也是帮助自己认请自己
内心真实想法的过程吧,我最终决定留下来,但我以后的学费就可能会有困难。他说他
如果能帮我他一定会帮我的。

去年八月,我刚回来,还临时住在以前房东家出租楼上的某个房间里,做东西吃东西都
不方便。这么多年来他们时不时地在我家吃饭,我也请他在外面吃过,现在开车回来两
天吃得简单,住在这里吃得还是极其简单,就打电话给他,直接说想去他家蹭顿饭吃。
他说downtown有家咖啡厅,一起去喝咖啡,他把XX(他的朋友,后文就戏称小秘吧)也叫
上。我转专业了,没有什么基础,以后学习上可能都有困难,我需要朋友在我困难的时
候能够帮得上忙。于是去了,他和我都只点了一杯咖啡,“小秘”比较厚黑一点儿,听
说我买单,要了一块cheese case还是蛋糕之类的一盘子里装了一片,一瓶can装的饮料
,还是一份什么不记得了反正他点了三样,最后总共十六七块钱是我结的。稍聊了一会
儿也就散了。那天心里的期望与现实的落差一定是有的,但自己是那个学习上需要帮助
的人,以后可能还需要向朋友借钱,对他的缺点可能也只能稍微包容一下吧。

开学后他和小秘打电话邀大家一起出去吃餐饭。我车很破,这次从加州回来都是以60的
速度慢慢爬回来的,我是一定不开车的,但去银行换好现金带上。后来我们三人(朋友
,我,和小秘)坐另一个女孩子的车去的。我还是很能吃东西的,朋友和小秘似乎略显
不快,但也没什么关系,本来就是出来吃东西,又没有说男生多付女生少付,我心安理
得。他们是男生,嘴巴很甜,讨好讨好女生就过去了,我是女生,坐了别人的车来回,
我就把自己的份连同小费一起给了女生,这样她可以刷卡付我们两人的份。

还在一件我现在用的电脑是从他那里买来的。他是感官动物,他对一些前沿的东西科技
比较有嗅觉。这个电脑是08年感恩节还是圣诞节据说联想官网出问题时\$500块抢下来的
,后来他升内存到8G,换了固态硬盘。他找到deal买到另一笔记本要处理这个旧的,正
好我自己的电脑本键盘不好用,便打算买下来。我对价钱没有概念,随机说200。小秘
说200不卖,朋友光升级就花了多少钱。我随口再说230。朋友后来230给我,说他还有
点儿愧疚。我安慰他说,大可不必,不管小秘怎么说,朋友如何认为,230在我这里还
是一个出手给朋友的合理的价格。因为我的理论很简单,人性本善。若真心认为卖给我
卖贵了,那以后我有困难他就会愿意帮我。后来他拿笔记本给我时说屏幕有点儿花了,
这大概是他这个感官动物最不能容忍的缺点了吧。Linux系统开机时鼠标有点儿拖,没
别的大问题。

后来大家的学习都很忙,真正能够帮得上忙的不多。这次给他留言请他帮我从jail里
bait出来,四点多给他留言的,八点多钟他带上小秘一起过来的。出去后,小秘拿东西
给我吃,问我吃晚饭了吗?我说没有(其实我可以在里面吃,只是不愿意,知道他们会
希望一起出去吃的),于是我顺便问,你们呢?小秘说吃了,朋友说没有。“那我们一
起出去吃吧!”朋友对面条是没有兴趣的,他决定的地方,带我们到了上次10年12月有
表哥一起去的那家餐厅。朋友说现在在外面吃的次数多了,跟本记不得上次我请他们是
要什么地方了。于是我把上次表哥与我,还有他和其它人一起坐的坐位指给他看。这家
餐厅在local算比较贵的,最贵的菜\$15一份,最便宜的可能也要至少\$11.5吧,但味道
据说确实好。他们两人点了份鱼、回锅肉,我也顺他们的意,加了份他们喜欢的炒肥肠
,后来加小费四十多块钱是别人帮我,我应该付的。

写到这里,把今天写下的这乱七八糟的发到网上,我已经做了背叛他的事情了,因为至
少去年上次解释的时候他还是不希望曝光他的生活的(而今天我写没有再征求他的意见
,因为我固执地认为今天的局面部分原因也是由他的暧昧造成的)。而且我将来还是有
求于他的,因为将来我学费不够的时候需要向他借。现在我已经做了背叛的事情,因为
我有自己想要保护的关系。我与表哥已经不能再联系了,但我心里还有一分期待,我不
希望不必要的误会多心把我们分开。(想瘦,也是因为表哥。心理上会觉得表哥久我一
个拥抱,而我亏欠他实在太多!)

今天我已经不计后果地写到这里,写到这种程度,如果这样还有人说我拿他作备胎什么
的,大可以找块豆腐撞死算了。

再说说车。我的车是有硬伤的,从去年1月一个零件就坏了,修车师傅说就是换旧的也
得几百块钱,车太老,修了不划算,于是不换,只是帮我把车调快了些。以60迈的速度
从加州爬回来的。回来后油箱坏了是老问题,但local别人不给修,只能换旧了,花了
二百六七十块。没开几个月,手刹坏过,到上月底(1月31日),又哪里露油了(把
Goodwill前留了一滩油)。这车96年的,07年十月我买,也已经开了五年零四个月了,
而且前前车主时有过车祸。或许它真的老了吧。对于我这个车无能,每次面对修车都是
一次心理折磨,远没有不要车来得自在。而且现在只是学生,没有车确实能过。所以不
如放手给懂车的人吧。

车拖走那天我收拾车里的东西,居然有个红色箱子,打开看是两条雪链,看上去是全新
的。这个东西不是我买的,舅舅当初帮我修车时有要我把前前主的东西还给别人,舅舅
帮我把它们全放在一个袋子里的。那这雪链应该是舅舅帮我准备的了。自己粗心大意,
10年12月回去下雨路滑,一路开得很辛苦,回家心里还怨舅舅连句关心问候的话都没有
,别人早早帮你准备好的东西你自己不发现,能怪谁了,只能把感激留在心里了。
\section{误会}
\label{sec-12-3}

发信人: Dreamer (不要问我从哪里来), 信区: Dreamer

标  题: Re: 成长的故事 -- 我和舅舅

发信站: BBS 未名空间站 (Sat Mar  9 03:22:39 2013, 美东)

误会

好吧,我妥协,既然树大招风,人红事非多,前段时间风口浪尖之下想过是否该写点儿
什么,但因为自己忙很多作业要写,后来便不了了之。现在,好不容易学校放spring 
break了,与表哥的案子在刚过去的周四也算最终了结,加上希望自己做事情有始有终
,也是对自己感情的一份交待,特意抽时间来作必要的解释,将这个故事阶段性地写完。

再次强调,因为学习功课很忙,时间有限,我不能再像之前那样长年累月的写了,就只
能捡重点有必要的写了。先解释先前的一些误会吧。

先说刚刚发生过的食堂里的事情。周六晚餐改成那种集中一个section serve的style已
经有一段时间了。最开始几次大老板一般都在,周六晚的工作人员一般也都固定了,我
同另一个美国女生负责从warmer或cooler里向外搬食物,以保证外面餐盘的食物不间断
,以及准备餐盘、刀叉等,间或也会有当天的学生manager或者其它不太忙的学生工作
人员帮忙。那天一开始也没觉得有什么异样,后来感觉只有那个女生与我还在前台,再
后来,那个女生已经站在甜点后fix不动了,整个前台就剩下我一个人跑来跑去地忙!
于是我时不时地扔手套嘟嘴表达着自己的不满。后来等到另一个女manager出来的时候
我便同她讲,我累了我想要take我的break,她同意了,我将手套远远地扔进垃圾桶,
便找到一个座位开始吃东西,这才开始意识到偌大的食堂大家都异常地小声,回想一下
刚刚那么长的时间里只有我一个人忙的时候,我拿食物出来,都有学生等在那份食物的
旁边,是真的他们喜欢那些份食物吗,还是只是同情我?那为什么大家都躲起来不帮我
呢,也是他们的工作啊,只是因为大老板不在他们懒吗?反省自己的原因,我有像食堂
里吃饭的大家想像的那么难相处吗?

食堂里那份异样的小声安静让我感到很挫败。回想我与同事们的表现我并没有做错什么
,自己付出了辛苦劳动却遭到同事的冷漠回避和食堂里学生的误解,心里很委屈、悲愤
和孤独,不争气的泪水开始在眼睛里打转,只能努力地控制自己不要当着大家的面掉下
来,却被前来同坐的女manager撞见,她将餐盘放下,借着打电话友善地避开了。我及
时地从食堂大厅折走,到了洗手间,眼泪终于还是掉下来。如果表哥在身边,肯定会抱
着表哥大哭一场吧,可惜不能够,而我所有能做的也只能是作一个更好的自己,而此时
外面我还需要继续工作,等自己情绪稍微稳定,便故作坚强地出去了。只是,内心里,
能清楚地感觉到那个从十九岁便开始遭受舆论伤害的孩子终于是在反人类的道路上越走
越远了。后来食堂快关门的时候,manager叫住我安排工作说,“Could you please do
pots and pans for me?”我这人很直,回想刚刚他让收账的同学take break,他就坐
在收银台看我跑来跑去地忙,不调配劳力 不说,还要我去拿什么东西,搞得大家都误
会我,便直接反问他“Why would I do it for you?”他继续以一种乞讨的方式说“
Could you please do it for me?”我很严肃也很直接,“No,  I am not  going to 
do it for you, I am doing it for myself!”便离开干活去了。

那份受了委屈的苦涩是无以言说的。忙完回到系里写作业、去洗手间的时候眼泪会掉下
来,晚上躺在床上眼泪还是会默默地流出来。反省自己,我也有很多做得不足的地方,
比如话不多,不是必须得同学生打招呼,主动同别人打招呼就比较少。难道我真的只能
做一个反人类的恐怖主义分子吗?这时大学时听到的一个故事又涌现在了记忆里。

故事的主角是一个大龄师兄。师兄在比较孤僻的环境下长大,大学时女朋友却被杀了,
后来调查结论是情杀,是师兄女朋友的前男朋友干的。不知道师兄是始终没能从痛失女
朋友的悲痛中走出来,还是女朋友与前男朋友的关系(藕断丝连也罢)让他感觉到背叛
,他不信任人,也不能接受来自任何人的批评。师兄读了两个地方的博士,都因为导师
们的一句批评彻底放弃放位,而过段时间后又再考取另一位导师的博士研究生,听说那
个故事的时候师兄已经是第三次读博了。

可能大学时自己的精神世界还比较狭小吧。这位师兄身上有太多自己的影子,在孤僻的
环境下长大,接受别人的批评很困难。第一次听说那个故事后,那个关于人的局限让我
郁闷了好几天,最终希望自己不要落入那样的境地。可是现在发生在自己身上的事情又
能说明什么呢?难道我这十几年是白活了?为什么到现在我还在遭遇如此尴尬?我小的
时候被水淹过,很长时间一直怕水,现在我已经学会了游泳,虽然只是在浅水区扑通几
下。我不要再作那个不爱说话的自己,我可以改,我一定可以改的!

第二天周日我有工作,前一天工作的manager也在,见到他我也只是淡淡的,没什么好
说,也没什么意思。在工作台我意识到,有很多善解人意的男孩子戴着耳机来到我所在
的salad bar拿食物,鬼都知道我内心里脆弱得要命,可表面上还若无其事、故作轻松
地与大家一一打招呼。可到了无人的角落,每走一步都能感觉到心里的丝丝阵痛。另一
个女孩子拿了盘甜点在我旁边吃,这是平时工作不允许的。当时我还没什么勇气,可等
她走后,想到大家觉得我工作太认真了(我能想到的可能认为我不太好合作的理由),便
东施效颦也罢顾不了那么多了,大摇大摆地走出去拿把勺子,回来再取了盘蓝霉甜点倚
着吧台大口大口毫不避讳地吃起来。美味的食物总是让人开心的,可心里还是有着丝丝
苦涩的,或许这就是生活吧,苦中作乐笑中带泪。好吧,我承认本姑娘今天就是受刺激
了,仅此而已!

后来那天真正让我彻底缓过来的是一位小留朋友,年龄很小,见识却很不一般。她来到
这里拿食物,我们聊了可能有十分钟的样子。能在如此困境下得到这样一位朋友的支持
我很感激也很知足。以至于后来再想,什么屁大一点儿事,会有什么大不了?后来前面
吃甜点的女孩过来找我聊天,试探性地问我说听说我一直在salad bar 干会感觉紧张?
我说是啊,这边琐碎事情多,又只有一个人,常常会忙不过来,shut down 的时候活更
紧更多,所以就只能早早地开始写标签,精神上自然会早早地就开始紧张起来。她又试
探地问听说我向manager抱怨什么什么,我听不下去了,很没意思地说,这里的事情对
我来说就只是一份工作,我干一个小时的活挣一下小时的钱而已,犯不着向谁抱怨什么
,很没意思地走开了,言下之意,请不要用这样的想法让我难堪,女孩子便也打住了。
其实我心里明白,我被安排在这个salad bar 的频率太高了。这里已经是美国最底层的
小镇,如果这个学校的学生不能排除心里的偏见,无法接纳我心目中那个对表哥的选择
,他们不喜欢我,那我受他们的冷嘲热讽、呆在这个大家不太愿意干的salad bar 受点
儿苦受点儿累都没什么,只要最终我能得到我想要的就好了。

晚上刚上班的时候,昨天的manager 还在甜点的地方没活找活儿干,大概想讨好我吧,
我只是淡淡的,没什么意思;我take break的时候,同一个美国女孩子坐在一桌吃东西
,他挤过来坐下同女孩聊天,我问他“Don’t you have a seat over there?”女孩帮
我说“Leave us alone!”看他离开我说“I  am kidding.”他说“I  am serious.”
”OK, take it seriously. ” 后来他再来我的工作台找我聊天的时候,自己心情已经
慢慢好转起来,毕竟也不是什么大不了的事,早晚都会过去的,同事的关系还是要好好
处下去的,再加上自己也有错,忙不过来的时候就该同manager反应啊,便从心底原谅
了也,好好同他聊天也算是这事就过去了。

离聊天刚刚过去半个小时不到,七点半shut down的时候,那manager又跑过来拿着他的
手机给我看批评我说现在是七点二十八分,你提早了两分钟shut down!我那天正好忘带
手机,shut down之前是看过厅里的钟了的,便也没什么意思地说,你批评我之前先把
厅里所有的钟先校对好了再来找我说吧!他看了看钟躲一边儿去了。后来晚上躺在床上
想,是我太包子了吗?有这么干的manager吗?即使我提前两分钟,离我原谅他还半个
小时不到,他至于这么让我难堪吗?我想以后再有manager这么挑我刺,我就该直接吵
回去,给他难堪,省得别人把你当柿子捏!
\section{解释另一件事}
\label{sec-12-4}
发信人: Dreamer (不要问我从哪里来), 信区: Dreamer

标  题: Re: 成长的故事 -- 我和舅舅

发信站: BBS 未名空间站 (Sat Mar  9 04:15:58 2013, 美东)

另一件关于还在加州时的一次令广大WSN耿耿于怀的一次饭钱。那天是要去参加一次活
动的,忘了什么原因可能要掉队了,不能同大家一起在预设的地点集合。也忘了是如何
联系上的了,我还是很想参加那次活动的,他也想,可是他没有车,也不太愿意坐公交
车,而且他住在三藩,他想要我开车去三藩在某个地方接他,然后一起参加活动。那时
的自己还是傻吧,现在想想都还觉得不安全,可能当时我还是很想参加那次活动吧,没
办法就只好按他的建议开车从San Jose去三藩接他了,晚饭钱他付是他先前说好的,我
的那份也的确是他付的,只是没有想到那天大家选的餐馆那桌会算到每人二十。桌上就
两三个女生,我便也打包了一份剩菜。因为油费怎么也算不到二十,原本可以退给他十
块八块的,可那天所有的着急赶一块儿了,急着快出发便忘了检查零钱。那包作为女生
打包的剩菜车上说好他带回去的,可下车时他忘了拿,我到家也是第二天才想起。回想
一下,从见面后的经过来看,他可能是对我有点儿意思吧,只可惜我们缘份不够, 我
心里早就有了别人。我开个破车开车技术又不高加上三藩那边非常难开,来回也没少受
罪。晚上我自己从三藩回来的时候不小心闯了红灯,心里很难受。就为了参加一次活动
,受这样的罪,何苦来?后来家里没有收到罚单还好,可能没装摄像头吧。人都是吃一
堑长一智的,从那以后,我再也没有单独同陌生男生一起去赶什么活动。
\section{表哥}
\label{sec-12-5}
发信人: Dreamer (不要问我从哪里来), 信区: Dreamer

标  题: Re: 成长的故事 -- 我和舅舅

发信站: BBS 未名空间站 (Sun Mar 10 07:19:22 2013, 美东)

表哥

去年五月,因为没能处理好与中介公司的关系,没能解决自己的身份问题,加上早有回
学校读计算机的想法,便及时地联系学校,打算回来读硕士。六月底七月初被学校录取
后我就没有了身份上的担忧,表哥就在旁边学校,近水楼台,于是八月份还没开学我就
跑去找表哥了。第一次去那天是周四下午,表哥不在office,便先回家了。接下来的周
末,我再去找他,远远的还没进楼,表哥就提着个粉红色的包包从楼里出来了,远远的
看见我便加快了脚步,我也赶紧凑到表哥车旁边,他已经发动了车要走,我拦住,表哥
摇下车窗来说,“你为什么不把你自己的事情做好?”表哥口中我自己的事情应该是指
我的学习,那我就回家好好学习吧。

大家似乎对我的感情上的事情还比较关心,总能感觉有环境压力想要摄合我与一位朋友
。我还是会常常上表哥的网页上去“看看”他的,某天便惊喜地发现,就像先前有人说
我一个搞统计数据分析的,干嘛要同一个学计算机(表哥)的过不去,表哥更新了他的
网页,他也喜欢C++,喜欢pointers,喜欢Emacs, lisp,他用templates还有筣得心应
手,他还作过好多门课的TA,当年也自学过一门语言,先前他show给我看的望远镜也成
了他的爱好之一!注意到表哥网页上的变化,我心潮澎湃,就好像表哥突然一切都变得
同我一样,我们俩个怎么就这么像呢?我的惯用伎俩就是跑去找他,于是呼我又故技重
演跑去找他。那个周Policeman打电话给我说邻居反应我车漏油要我修车,因为漏油,
不允许我再开那辆车,只准用拖车拖,于是周四周五我就打电话给保险公司把修车拖车
的事情联系好,周六一早便随同拖车来到表哥所在的小镇。表哥不在他办公室里,幸好
我带了自己的自行车,于是我便骑着自行车从他office去舅舅家找他,看见表哥的蓝色
的车停要门口,我知道表哥在的!

我按了门铃,没人开门,我便在后院里转幽,看看表哥家的狗在不在,看看新种的两三
棵果树是什么树以及菜圻里老迈的蔬菜藤蔓。不知道什么时候狗叫起来,舅母也出来开
门,我便进去了。舅母没多说什么,赶紧跑表哥洗手间门口问表哥穿衣服没有,告诉表
哥说XX来了。还是觉得同表哥亲密吧,进到房间里时表哥只穿了短裤和线衫,可我并不
觉得尴尬和不适。后来的情节是大家可以想像得到的,先打拉扯,我把表哥 线衫给抓
没了,他上身便只剩件白色T恤内衣,看着很fit,再打911,我们俩个隔着表哥房间的
门板僵持着,门是半开的,我们分别卡在两边。表哥打着电话,感觉不知道从什么时候
开始表哥已经对法律的条条款款了如指掌,我管不了那么多,有机会这么近距离地看他
,便一次看个够。他的T恤穿得很合身,看着很舒服。我的一只脚轻跺在表哥卡住门的
脚上,一只胳膊试着去抓表哥的胳膊,把他惹火了又把我“打”一顿,这么僵持 着直
到警察到来。意识到 表哥只穿了很少的衣服,一个女policeman 便一直站在外面院里。

一个policeman问我话的那种对我备加怀疑的语气让我很恼火,回答问题时简直想要同
他吵架。谁会是随随便便就可以撒谎、谎话连天的人?谁愿意被怀疑是否诚实?就因为
我waive掉了\$50的罚单(12年五月回去时policeman说他们会take care 我的车,让我不
用担心。但等我开车时,车上一张\$50 parking ticket)没有出这\$50我就该被怀疑?后
来自己回答问题让他没有任何可怀疑的,他便说让我在表哥床上坐一会儿(开始我是蜷
缩在表哥床前地毯上的),他去同表哥问一个什么问题。

再后来,第一学期结束后我学习还可以过得去,我便去表哥学校找过表哥,表哥在办公
室,只是比以前更清瘦了,穿衬衣的他销骨显得很突出,看着很心疼。可以想像到的情
节重演了。我已经走出了表哥的办公楼,到了自己车的旁边,就差坐进车里开车回家,
警察还是拦住了我。那天记忆最深的是policeman说以后我不能再来这个学校了,接下
来两天内我会收到(禁令)材料,我很不解,当时就同他说,我是一名国际学生,我在这
里努力学习,最终还是需要凭自己的学业找工作的。这所学校离我自己的学校这么近,
我还是需要近水楼台从这里获取有益于自己的学习资源的,所以请尽量不要对我施加这
样的禁令,至少与我专业相关的统计系、计算机专业以及图书馆除外。我是真诚的,我
也是有道理的,大概是我的真诚打动了他吧,还是他们另有考虑,最终回到家后我一直
没有收到相关的材料。

今年春天舆论的力量发现表哥一定是喜欢我的,我就被施压应该去找表哥。可是我也是
矛盾的,我自然是希望能够早日同表哥真正在一起的,可表哥老是打911总不是个事儿
。于是周末我去找表哥表哥再打911的时候我就直接跑回家了。可舆论的力量也够强大
的,我也遭受了更多的鄙夷和嘲笑,而我内心里是有着表哥的,既然舆论的力量已然如
此,我又为何不能再去找表哥一次呢?于是紧接着的周三上完自己的课我便开车过去了
。不用去表哥办公室,楼梯里已经撞着个正着,表哥严厉警告我他要去上课,我不许进
去!可这个时候,箭在弦上整装待发,表哥的警告无疑成为一种更大的诱惑和启示,表
哥这堂课我必然是要去听的,我只是尽量不要去打扰他的学生而已。表哥把我推撞在楼
梯上,我忍着痛还是进教室了;他出去打911了,我先躲起来,等他再上课,我就从后
门悄悄进教室了,认真听讲努力学习的学生应该注意不到我的存在。表哥讲电子电路晶
体管什么的,我没多少基础不是很懂,但能感觉到他的幻灯片准备得挺充足的,用来讲
解被绝缘体隔绝的电子流动方向等手画的图也很透彻显得思路尤为清淅,他惯用的幽默
装饰也让课堂生动并不沉闷,表哥的英语讲得这么好除了一句“Good night!”居然从
来没对我多讲过,倒是他的普通话进步不少!一堂课的时间显然很短,就下课了又见
policeman,所谓做贼心虚,我赶快躲啊\textasciitilde{}~可我是躲不掉了,受了一番警告后,我还是
开车去表哥家找表哥了。因为在我心里,到这个时候, 这个关系无论如何是要有所进
展的,不管以什么样的方式如何进展。

舅母开的门,她在忙网上一个什么东西,我很累头痛,心里暗自盘算早就打定主意
跑到表哥床上去打个滚儿(这已经是间接的与表哥最亲密的接触了!)。三分钟还没躺到
,舅母说男生的床你一个女孩子怎么可以随便躺,要躺躺她床上去,被舅母骗起来表哥
舅舅就已经回家来了。没什么好说的,911照旧。同以往一样,我离开了。我已经坐进
车里开出去很远,可policeman还是路上拦住我,检查一番证件后,要求我下车、手背
后,告诉我我被arrested了。
\section{官司}
\label{sec-12-6}
发信人: Dreamer (不要问我从哪里来), 信区: Dreamer

标  题: Re: 成长的故事 -- 我和舅舅

发信站: BBS 未名空间站 (Sun Mar 10 16:21:43 2013, 美东)

官司

真正被arrest后我很平静。Policeman说是舅舅要求他们arrest我。从前年春夏舅舅用
暴力威协我再回去就打911的时候,我恨恨地跑回去一句话不说,他也没舍得把我怎么
样。那件事情后,我与舅舅之间便也有了基本的信任。 这次这事电话是表哥打的,主
意嘛,我猜 测可能是舅舅出的。表哥虽然读那些个法律的条条款款没什么问题,可他
也同我一样在校园生活多年性格比较单纯,后来还去韩国生活了那么久,他对美国社会
的认识和把握未必能够如此世故老道。那既然舅舅要这么做他就一定有他的理由。晚上
我被转送到county jail时里面的前台说交\$500可以把我bail出去,问我想不想那么做
,我问清楚如果呆在里面第二天就可以见法官,问题可能第二天就解决了,对于交\$500
块出去后可能有的变数我不太有信心便坚持留了下来。我第二天只有一堂课,如果问题
能够在很短的时间内解决掉也算省掉麻烦省得以后分心。和上一次的极度不平静不一样
,这一次,对表哥肚子里的几根弯弯肠肠子也算是很了解了,这只是万里长征过程中必
经的一步而已。于是就当我到了一个更为艰苦一点儿的环境休息而已。平时不是学习忙
经常头痛吗,现在正好借机会好好休息放松一下。

第二天下午四点开庭了,法官、我、两个秘书、一个看护和电话里的一个翻译。法官向
我解释了我所面对的困境,提出我可能的两个选择:遵守他的规定、放弃一些权利(比
如不准再过表哥学校),可以被释放;若我觉得不服,我可以请律师打官司,若我没有
工作,他可以帮我指定一个律师。但他并没有让我选择。若让我选择我可能还是会人在
屋檐下不得不低头、迫不得已地选择前者吧。但当我问及一些问题,比如这些个条款我
要遵守到什么时候,总有个合理合法的起始日期吧,再者,之前上学期结束的时候我向
policeman提过的请求我也再合情合理地向法官请示了一遍,但最终法官单方面认为我
有非法企图,直接指派 了一个律师给我,bail out的钱一度升至\$5000,一听到法官说
出这个数字,这次从前一天被arrest到现在一直好精神好休息、没掉一滴眼泪的我终于
情绪失控,眼泪扑簌而下,我全身上下所有银行卡加起来也没有\$5000块钱啊,后来法
官把它降到\$2500,一周后再上庭。court结束的时候一个作记录的秘书对另外一个旁听
实习般的女秘书说,“It was just the atmosphere!”至此,我终于是滔滔大哭,全
线崩溃,意识到我这颗统计、计算机专业、凡事喜欢把它弄确定确信的典型的理工科脑
袋终于是给自已惹麻烦了。

一周后才会再开庭,我还是学生有自已的学业要完成,是敬酒不吃吃罚酒吗,现在就算
是bail out的钱升至\$2500,我也还得先出去才能不耽误学业呀。于是给朋友打电话,
说清楚自已的状况以及钱的事情,请他帮忙带钱带卡过来帮我bail出去。见到律师后我
第一时间告诉他我的专业、我所受过的教育,把事情前因后果前后原委、让他弄清楚明
白这只是我一惯的思维方式。与表哥的关系走到今天,我自已知道轻重缓急,我并没有
什么不良企图。一周后开庭庭上,当我意识到我被压着往下走走上一条不归路的时候,
当场我对律师讲我不要再进行下去了,上一次庭上我只是问了几个问题而已,法官并没
有给我选择的机会,我们把误会讲清楚就结束吧。律师说这个时候我要结束可以会得交
\$5000罚金外加一年in jail,他说按照程式走下去他会帮我减缓到\$1000以下,2-3 天
in jail. 意识到自已已经踏上了一条不归路,便只好随律师一路走下去。这样,社会
舆论、个人行为、法律法制和司法监督间就产生了一系列的矛盾冲突冲撞与相互制约,
最终只罚交了\$50的走正规程序费用,当然前提是我不得再打扰表哥的生活,期限一年
。
--

※ 修改:·mitbbs.com 於 Mar 10 16:23:15 2013 修改本文·[FROM: 匿名天使的家]
\section{表哥}
\label{sec-12-7}

与表哥的关系走今天,其实我早就走上了一条不归路,这是2010年12月那场与表哥的告
别便早已注定的结局。我没什么好埋怨的。从同学的口中无意间听到一个bad 
influence的表达法,想想与表哥的关系,因为要最终得到社会舆论的认可,我也动用
了舆论的力量来保护自已,显然,从我开始写<成长的故事>亲情部分,我就俨然成为这
个社会的bad influence了。所以等 待也是应该的,对我来说自然也是值得的。

在我与表哥的这段关系里,舅舅错过,好在良心发现,后来并没有再做错什么;表哥并
没有什么错,他自始自终至今为止还从来没有真正亲口承认过要我作他的女朋友;舅母
一个妇道人家,爱丈夫爱儿子,对侄女平和,错过什么没做错什么也都显得不那么重要
了。

而我自已,从一点点动心,到大表姐掐死火星,从正常回学校 一趟办事,到发现表哥
的惊天秘密、真真喜欢上表哥,这之间也已经历时很久,转眼我们认识已经第四年了,
喜欢上表哥也两三年第三个年头了。

回望自已的感情路,磕磕碰碰一路艰辛。偶这个极品,年少无知的时候也还是有几句轻
狂话的。一如侄儿(大姐家的孩子)小时候对姐夫说“爸爸你用木棍把火车拨下来给我玩
好不好”我会记得,我自已也有那么几个时刻的话、事儿是印在脑海里的。

还上大学时的我热情满怀地对二姐说“我觉得这世界上,只有我自已的爱情才能称得上
真正意义上的爱情,其它所有人的爱情都好世俗、庸俗”姐姐当场接话浇了我一桶凉水
:世界上所有的人都觉得只有他们自已的爱情才是真诚浪漫美好的,是弥足珍贵的! 清
楚地记得自已当时仿佛被电到,不得动弹,脑子还在飞速急转,噎了半天,总算领悟过
来“是哦!”原来那时的我只是生活在自已的小世界里。

上硕士时的我其实还不是很懂感情,听说谁谁要结婚了,听说谁谁哪个师姐(所里的一
个师姐)已经结婚了,心里都会生出无限遗憾和惋惜,唉,好端端一个姑娘就这么随随
便便地嫁了!恨不得自已能替她们作选择,能替她们好好活下去!

爱过、恨过、原谅过、解脱过。一个女孩子的花样年华又能有几年?在国内出来前的七
八年里算是自已初步体验感情的阶段,来到美国后努力地找过,为自已创造机会过,可
六七年的日子流水般哗哗逝去,能感觉到的也只能是越来越难。一个人的一生又能有多
少次的恋爱?星座上说我的第二个结婚幸运阶段是在三十岁上下,第一个在二十岁左右
,按照那种说法我是早就错过了我自已的黄金年华了。

我很感激在我最想要感情安定的时候遇到表哥,他很有执行力,这场遇见仿佛给自已吃
了颗定心丸,从此安心。在美国走过这六七年的漂泊岁月,遇见表哥后慢慢静心,也是
在去年爸爸离世的打击下,我终于是开始浪子回头,认清自已该认认真真努力的方向。
岁月会印证这个选择。
\section{往事}
\label{sec-12-8}

在过往的记录里,有一件事我只记录了一半的客观事实,为伤及无辜并没有记录另一部
分。现在这件事已然成为往事,所有的当事人都已经走出了当年的困境,我想,这时再
来完整的记录追忆这件事应该不至于伤害太多人。

2006年夏天我拿着全奖来这边读Plant Science Ph.D,我的老板是一个和蔼可亲的基督
教徒美国人,对我也很好。一个学期后,老板一家要move 到非洲去帮助那边发展科技
。临走时老板说我可以从一个作大田作物的Associate Profesor和一个作病毒分子的
Assistant Professor里选一个作老板继续自已的Ph.D 学习。我因为内心里那点从硕士
时就开始有的对分子的热爱,俨然不顾老板只是Assistant,毅然决然地选择了后者。

怎么说来,从后来转专业到统计后的表现来看,那来美后的五六年里,我不得不承认自
其实是心浮气躁的,没有目标没有方向,似乎并不很明白自已到底该如何生活。作学生
时那黄金般的年华我居然浪费了那么多的时间睡懒觉、跑去钓过那么多次鱼,现在回想
还觉得很可惜。

后来因为自已的懵懂混世,也因为学习上语言上的一些障碍,以及自已总想选感兴趣的
课程实则眼高手低等(07年秋季居然选了门医学病理基础课,差点儿没被词汇折磨死)
,加上找到舅舅萌生了转专业的想法,,到07年12月期末考试我的平均分在3.0以下,
系里给发了我被put on probation的邮件,我的导师也是知道的。

那时我的导师还是AP,在这个学校里他已经通过合作有了一些funding,应该还是很有希
望可以拿到tenure的。导师大概同我committee里其它member已经通过气了,于是那个
学期结束时我的Ph.D开题报告会议里,所有member一致要求我春天停课,只在实验室里
做实验,若实验做得好,秋季再选课。那个学期我已经选了一门统计的课,我便对他们
说我的数学基础比较好,我能否只上一门统计的课不会占用太多时间,也能成为我单调
实验室里做实验生活的一种调剂,老师们不允许。那个开题报告里受到老师们那样的要
求,一向感性的我自然是免不了落泪了。

之前的一年里跑了些病毒的序列胶,构建了几个病毒载体转华到大肠杆菌,并接种到植
物上等待发病。以两天为周期的提病毒蛋白颗粒实验重复了N次(N>5)老板还是比较
满意的,他让我把这些颗粒存起来以备得到抗体后再作进一步实验打算。从过往的实验
里,我基本的分析能力还是有的,实验也还做得认真还过得去,大概久缺的就是一把
push的力量吧。

整个寒假我一直在想这个问题。到春天开学的前一天下午,老板快要走时我去他办公室
找到了他。因为第二天就要开学选课交学费了,我想同老板谈一下。我对老板说我可以
考虑接受春季不选课,认认真真呆在实验室一心一意做实验,希望在课题上能有些重大
进展。不过,我既然要呆在实验室里全心全意地做实验,那老板能否帮我把这学期的学
费(我是指三千多块的州内学费)给交了。老板大概对我选了一门统计课起了戒心。我
便对老板讲我只是一个穷学生,我国内的家庭状况是什么样的(我来这边第一年连手机
都没有配),这些钱对我真的很重要。老板始终无动于衷,根本没有可能。

可能还是觉得受了委屈心里不平吧。找到舅舅后原本就有些想转专业的想法,舅舅说很
多中国学生来到这边后读完第一学期就转专业,在系里影响非常不好,舅舅要我好好先
给老板干两年活。原本我还在犹豫要不要拿Ph.D,要不要按舅舅的最高指示执行下去(
先前对舅舅只有感恩和服从,从不曾反抗过什么),可是在受了委屈的心里,Ph.D显得
遥不可及,舅舅的理论在这时也显然遭受到了质疑。给老板干两年活真的就是最好的出
路吗?那样做就没有浪费老板的科研经费吗?自已的学习兴趣发生改变时到底该如何处
理就在心里翻江倒海起来。想来想去,终于还是觉得不愿太委屈自已,打电话给舅舅,
向舅舅的理论质疑,舅舅也终于是觉得我那时的心态已经不适合再拖下去了,同意
sponsor我转专业。

于是联系统计系,他们是愿意接受我的,只是虽然以后的外州费可以免,但第一学期我
还处在probation阶段,这学期的外州费我还是得自已交的。但我原Ph.D program所在
的农学系为我waive掉了这笔巨额外州费(当时是5000多美金)。农学系为自已曾经的
、受了委屈的学生所表现出的公正、宽容的名校、大系风采一直让我心存感激、深为动
容。

后来不知道两个系的头目们有没有私下交流什么,统计系始终不曾给我奖学金,系主任
的理由大概是为平衡两系之间的关系。当时能免去外州费我便也基本能够生存,就不多
想。系主任老是跑到农学系去开会作报告以证实他作为领导者、决策者的清白立场,我
因为不再计较,便也邀朋友陪我一起到农学系去听过他们一次报告以表心迹。再后来我
工作后为统计系捐过\$1000也算是间接地为母校建设微尽薄绢之力吧。

现在事情的全部原委都已知晓。其实作为自已传记的记录者,我心里有转专业的念头和
想法这件事情天知、地知、舅舅知、我知,我大可不必承认自已的初衷,我大可把所有
的过错矛头全都指向博士学生时的老板,但人在做、天地看,天地良心,人性本善。转
专业这原本是矛盾下的结局,缺少了矛盾的一方,我便也愧对前老板,愧对农学系对待
自已的公允。我那亲爱的舅舅,当他听说这件事的时候,他是会骂侄女傻呢,还是会收
回成命,收回他当初认为我“不择手段”的话和想法?我想这件事也算建位信任的一步
努力吧。

这次回来读计算机硕士时,我找过之前的老板,他已经拿到tenure,成为Associate 
Professor,手下有很多学生为他干活。老板说他自已也在想办法多拿些funding,但目
前不缺人手。其实想来也很可以理解,当初在他最需要我干活的时候我走了,他又何苦
要与我分享他的胜利果实?他顺利拿到tenure成为Associate Professor我倒是挺开心
的。
\section{一份作业}
\label{sec-12-9}

从加州回到学校后,我转了专业,开始读计算机的硕士。虽然我所受到的教育、对数字
和编程的偏好让我有勇气重返校园从头开始学习计算机这个陌生的专业,但这并不是说
我就真的完全无所畏惧。我心里还是 很害怕的,尤其是在第一学期第一二三个星期的
时候。每门课老师一布置作业我就开始心里发慌,也不是说我一定写不出来,但心里就
是害怕万一我要是写不出来可怎么办?于是这出现了这么一幕,某周五下午,校园就快
空了,我跑到系里的help center去慢慢悟慢慢想某门课的一个现在看来很简单的作业
到底该怎么写。系里一位老师正好下来取件什么东西,看见我还呆在那里,便提醒在那
里工作的同学说,当学生的问题很简单的时候,不要直接告诉他们答案,稍微提醒点拔
一下他们就可以了。回想一下,C++的作业前两三次我都是从网上搜到code然后改编成
我的作业交上去了,我所做的只是把网上那些不work的code让它们能编译能运行满足老
师的要求便交上去了。后来上了快半个学期的时候才慢慢学会了思考,不管是作业还是
实验,基本能够按照自己的想法把它实现出来了,算是最终摆脱了恐惧心理和对网络的
依赖。

既然转了这个专业,已经第二个学期了,贴份作业与时俱进吧。这是上学期<
Programming Language>课程Lisp语言的一份作业。要求如下:

Homework \#2: Lisp Homework 

Due: Monday September 17, start of class (9:30) -- Extended to Tuesday, 
September 18, at 9:30am.

Turnin: using the cscheckin command: 
cscheckin -f hw2.l -c cs210

Use Common Lisp to write the following program. It evaluates exactly one 
move of a three-dimensional 4x4x4 game of "tic-tac-toe". You should google 
this game if you are unfamiliar with it. It is a two-player game. One player
marks spaces with "x" and the other marks with "o". They take turns. The 
object is to obtain a straight line of marks without the other player 
blocking them. Your Lisp program should require two command line arguments (
read in clisp from pre-defined symbol \textbf{args}). The first argument will be an
'x' or an 'o' indicating which side moves next. The second argument will be
the name of an input file (example) that looks like: 

?|?|?|?   ?|?|?|?   ?|?|?|?   ?|?|?|?

\rule{\linewidth}{0.5pt}
?|?|?|?   ?|?|?|?   ?|?|?|?   ?|?|?|?

\rule{\linewidth}{0.5pt}
?|?|?|?   ?|?|?|?   ?|?|?|?   ?|?|?|?

\rule{\linewidth}{0.5pt}
?|?|?|?   ?|?|?|?   ?|?|?|?   ?|?|?|?

where ? is either an 'x', an 'o', or a ' '. Your program should consider all
possible placements of this piece, and write to standard output a board 
configuration (in the text format of the example above) with the "best" move
for your argument. 
In order to do this, you are required to read the file and use it to build a
list of four lists of four lists of four integers. It is the internal list-
of-lists-of-lists representation that should make it easy to consider 
potential board positions. You should assess the potential board positions 
using the following heuristics: 

•    +1000 points for your side attaining 4 in a row (win) 
•    -100 points for failing to block an opponent's three in a row 
•    +10 points for obtaining a 3 in a row 
•    -5 points for failing to block an opponent's two in a row 
•    +1 points for obtaining a 2 in a row 

Note that each of these heuristics can occur multiple times for a particular
move; if a potential move will create three 3-in-a-row situations that is +
300 not just +100. 
In the case of a tie (multiple board positions with the same \# of points) 
you should print out each of the potential positions with a blank line 
between them. 

还是简单翻译下:大致意思是说用Common Lisp实现4x4x4 Tic-Tac-Toe游戏的一步走法
,棋盘布局由外部.txt文件、以及player的一方(是X还是O)都通过common line 
argument输入,通过heuristic function选择最佳position,并将实现这步走法后的棋
盘在standard output输出。输入棋盘只考虑空盘、不满盘和不䊨盘(输入的棋
盘不会有一方已䊨)。
\section{lisp tic-tac-toe 作业代码}
\label{sec-12-10}

\lstset{language=Lisp,label= ,caption= ,numbers=none}
\begin{lstlisting}
#!/usr/local/bin/clisp
;; require two arguments
(if (not (= 2 (length *args*))) (progn
    (print "ERROR: we take only two arguments: player and input board file 
name.")
    (exit)
    )
  ; else
;  (progn
;    (print "OK, we got 2 arguments")
;   (print "arg1 is ")
;   (print (nth 0 *args*))
;   (print "arg2 is ")
;    (print (nth 1 *args*))
;  )
)

;; check if the first arg is "x" or "o" only
(let ((arg1 (car *args*)))
  (if (not (or (equal arg1 "x") (equal arg1 "o")))
      (progn (print "Player can take two values: x or o.")) ))
;(exit) )) ; want to continue, so NOT exit yet!

;;; set current player value
(setf player 
  (let ((arg1 (car *args*)))
    ;(if (= (car *args*) "x") 1 2))
      (if (equal arg1 "x") 1 2)))

;;; set current player "x" or "o" mark for final vector specific position 
update
(setf playermark 
  (cond ((= player 1) #\x)
    ((= player 2) #\o)))

;; read lines
;(let ((in (open (nth 1 *args*) :if-does-not-exist nil)))
;  (when in
;    (loop for line = (read-line in nil)
;     while line do (format t "~a~%" line))
;   (close in)))
;(close in)))

;;; read input board line-by-line
(defparameter *L* (open (nth 1 *args*)))
(setf b1 (subseq (read-line *L*) 0 37))
(setf b2 (subseq (read-line *L*) 0 37))
(setf b3 (subseq (read-line *L*) 0 37))
(setf b4 (subseq (read-line *L*) 0 37))
(setf b5 (subseq (read-line *L*) 0 37))
(setf b6 (subseq (read-line *L*) 0 37))
(setf b7 (subseq (read-line *L*) 0 37))

; cleaing b1 to be 16-char long string vector: b1
(setf b01 (remove #\| b1))
(setf b11 (subseq b01 0 4))
(setf b12 (subseq b01 7 11))
(setf b13 (subseq b01 14 18))
(setf b14 (subseq b01 21 25))
(setf b1 (concatenate 'string b11 b12 b13 b14))

; cleaing b3 to be 16-char long string vector: b2
(setf b02 (remove #\| b3))
(setf b21 (subseq b02 0 4))
(setf b22 (subseq b02 7 11))
(setf b23 (subseq b02 14 18))
(setf b24 (subseq b02 21 25))
(setf b2 (concatenate 'string b21 b22 b23 b24))

; cleaing b5 to be 16-char long string vector: b3
(setf b03 (remove #\| b5))
(setf b31 (subseq b03 0 4))
(setf b32 (subseq b03 7 11))
(setf b33 (subseq b03 14 18))
(setf b34 (subseq b03 21 25))
(setf b3 (concatenate 'string b31 b32 b33 b34))

; cleaing b7 to be 16-char long string vector: b4
(setf b04 (remove #\| b7))
(setf b41 (subseq b04 0 4))
(setf b42 (subseq b04 7 11))
(setf b43 (subseq b04 14 18))
(setf b44 (subseq b04 21 25))
(setf b4 (concatenate 'string b41 b42 b43 b44))

; concatenate into one string: bc (120)
(setf bbb (concatenate 'string b1 b2 b3 b4))
(setf ba (substitute #\1 #\x bbb))
(setf bb (substitute #\2 #\o ba))
(setf b (substitute #\0 #\  bb))

(setf board (make-array '(4 4 4)))

;;; try to read b11
(loop for i from 0 to 3
     do (setf 
     (aref board 0 0 i) (subseq b i (+ i 1))))

;;; read whole b1
(loop for j from 0 to 3
  collect (loop for i from 0 to 3
    do (setf 
         (aref board 0 j i) (subseq b (+ (* 4 j) i) (+ (+ (* 4 j) i) 1)))))

;;; read whole b into "board"
(loop for x from 0 to 3
  collect
    (loop for j from 0 to 3
      collect (loop for i from 0 to 3
     do (setf (aref board x j i) 
       (parse-integer (subseq b (+ (* 4 j) (* 16 x) i) (+ (+ (* 4 j) (* 16 x
) i) 1)))))))


;;; check for empty board: set the first element to be player
(setf x (coerce "
0000000000000000000000000000000000000000000000000000000000000000" 'string))
(if (equal b x) 
    (progn
      (defparameter *L* (open (car (cdr *args*))))
      (setf b1 (subseq (read-line *L*) 0 37))
      (setf b2 (subseq (read-line *L*) 0 37))
      (setf b3 (subseq (read-line *L*) 0 37))
      (setf b4 (subseq (read-line *L*) 0 37))
      (setf b5 (subseq (read-line *L*) 0 37))
      (setf b6 (subseq (read-line *L*) 0 37))
      (setf b7 (subseq (read-line *L*) 0 37))
      (format t "~%~A" "The original input board:")
      (format t "~%~A" "")
      (let ((in (open (car (cdr *args*)) :if-does-not-exist nil)))
    (when in
      (loop for line = (read-line in nil)
         while line do (format t "~a~%" line))
      (close in))
    (close in))
      (setf (aref b1 0) playermark)
      (format t "~%~A" "The Updated Final board is:")
      (format t "~%~A" b1)
      (format t "~%~A" b2)
      (format t "~%~A" b3)
      (format t "~%~A" b4)
      (format t "~%~A" b5)
      (format t "~%~A" b6)
      (format t "~%~A" b7) 
      (exit)
      ) )
;else

;;; define opponent player
(defun opponent (player)
  "Return the opponent of PLAYER."
  (if (eql player 1) 2 1))


;;; ------------------------------------------------------------------------
;;; Part 1: Check for input board has winned condition or now
;;; ------------------------------------------------------------------------

;;; "A player has won if either a complete row, a complete column or a complete diagonal is finished."

;;; process results for define winner:
;;; check winner: for b11
(defun check_winner (player board)
  (if (= (loop for i from 0 to 3 sum (aref board 0 0 i))
      (* player 4)) 1 0))

;;; check winner: for b1
(defun check_winner (player board)
  (loop for j from 0 to 3 
     sum (+ 0 
    (if (= (loop for i from 0 to 3 sum (aref board 0 j i))
           (* player 4)) 1 0)) ))


;;; check winner 1: for board, 16 situations towards y axis (i direction)
(defun check_winner_i (player board)
  (loop for x from 0 to 3
     sum (+ 0
       (loop for j from 0 to 3 
     sum (+ 0 
       (if (and (= (loop for i from 0 to 3 sum (* (aref board x j i) (aref 
board x j i)))
               (* player player 4))
            (= (aref board x j 3) player))
           1 0)) )) ))

;;; check winner 2: for board, 16 situations towards z axis (x direction)
(defun check_winner_x (player board)
  (loop for j from 0 to 3
    sum (+ 0
      (loop for i from 0 to 3 
     sum (+ 0 
      (if (and (= (aref board 3 j i) player)
           (= (loop for x from 0 to 3 sum (* (aref board x j i) (aref board 
x j i)))
         (* player player 4)))
           1 0)) )) ))

;;; check winner 3: for board, 16 situations towards x axis (j direction)
(defun check_winner_j (player board)
  (loop for x from 0 to 3
    sum (+ 0
      (loop for i from 0 to 3 
     sum (+ 0 
      (if (and (= (aref board x 3 i) player)
           (= (loop for j from 0 to 3 sum (* (aref board x j i) (aref board 
x j i)))
         (* player player 4)))
           1 0)) )) ))

;;; check winner 4: for board, 8 crosss situations towards yz axis (j 
direction)
(defun check_winner_cj (player board) 
  (+ 0
     (loop for j from 0 to 3 
    sum (+ 0 
     (if (and (= (aref board 3 j 3) player)
          (= (loop for i from 0 to 3 sum (* (aref board i j i) (aref board i
j i))) 
        (* player player 4)))
         1 0)))
     (loop for j from 0 to 3
    sum (+ 0
     (if (and (= (aref board 3 j 0) player)
          (= (loop for i from 0 to 3 sum (* (aref board i j (- 3 i)) (aref 
board i j (- 3 i))))
        (* player player 4)))
         1 0) )) ))

;;; check winner 5: for board, 8 cross situations towards xz axis (i 
direction)
(defun check_winner_ci (player board)
  (+ 0
     (loop for i from 0 to 3 
    sum (+ 0 
     (if (and (= (aref board 3 3 i) player)
          (= (loop for x from 0 to 3 sum (* (aref board x x i) (aref board x
x i))) (* player player 4)))
         1 0)))
     (loop for i from 0 to 3
    sum (+ 0
     (if (and (= (aref board 3 0 i) player)
          (= (loop for x from 0 to 3 sum (* (aref board x (- 3 x) i) (aref 
board x (- 3 x) i)))
        (* player player 4)))
         1 0) ))))

;;; check winner 6: for board, 8 cross situations towards xy axis (x 
direction)
(defun check_winner_cx (player board)
  (+ 0
     (loop for x from 0 to 3 
    sum (+ 0 
     (if (and (= (aref board x 3 3) player)
          (= (loop for j from 0 to 3 sum (* (aref board x j j) (aref board x
j j))) (* player player 4)))
         1 0)))
     (loop for x from 0 to 3
    sum (+ 0
     (if (and (= (aref board x 3 0) player)
          (= (loop for j from 0 to 3 sum (* (aref board x j (- 3 j)) (aref 
board x j (- 3 j))))
        (* player player 4)))
         1 0) ))))

;;; check winner 7: 4 cross corner-to-corner situations
(defun check_winner_cc (player board)
  (+ 0
     (if (and (= (aref board 2 2 2) player)
          (= (loop for i from 0 to 3 sum (* (aref board i i i) (aref board i
i i))) (* player player 4)) )
     1 0)
     (if (and (= (aref board 2 2 1) player)
          (= (loop for i from 0 to 3 sum (* (aref board i i (- 3 i)) (aref 
board i i (- 3 i))))
        (* player player 4))) 1 0)
     (if (and (= (aref board 2 1 2) player)
          (= (loop for i from 0 to 3 sum (* (aref board i (- 3 i) i) (aref 
board i (- 3 i) i)))
        (* player player 4))) 1 0)
     (if (and (= (aref board 2 1 1) player)
          (= (loop for i from 0 to 3 sum (* (aref board i (- 3 i) (- 3 i)) (
aref board i (- 3 i) (- 3 i))))
        (* player player 4))) 1 0) ))
     
;;; final: win, combining 1 to 7
(defun win_check (player board)
  (+ (check_winner_i player board)
        (check_winner_x player board)
        (check_winner_j player board)
        (check_winner_cj player board)
        (check_winner_ci player board)
        (check_winner_cx player board)
        (check_winner_cc player board)))
;;; if win return greater than 0, the player has winned already!

;;; quote: I copied this function from http://stackoverflow.com/questions/7912232/how-do-you-copy-an-array-in-common-lisp website
(defun copy-array (array &key
                   (element-type (array-element-type array))
                   (fill-pointer (and (array-has-fill-pointer-p array)
                                      (fill-pointer array)))
                   (adjustable (adjustable-array-p array)))
  "Returns an undisplaced copy of ARRAY, with same fill-pointer and
adjustability (if any) as the original, unless overridden by the keyword
arguments."
  (let* ((dimensions (array-dimensions array))
         (new-array (make-array dimensions
                                :element-type element-type
                                :adjustable adjustable
                                :fill-pointer fill-pointer)))
    (dotimes (i (array-total-size array))
      (setf (row-major-aref new-array i)
            (row-major-aref array i)))
    new-array))

;;; define a function to change vector index to 4x4x4 array index
(defun array-index (board pos)
  (if (or (< pos 0) (> pos 63)) 0
      (aref board (/ (- pos (mod pos 16)) 16) 
        (/ (- (mod pos 16) (mod (mod pos 16) 4)) 4) 
        (mod (mod pos 16) 4))))

(defun pos_index (pos)
  (setf x (/ (- pos (mod pos 16)) 16)) 
  (setf y (/ (- (mod pos 16) (mod (mod pos 16) 4)) 4))
  (setf z (mod (mod pos 16) 4)) )


;;; get potential move positions
(defun potent (board)
(loop for i from 0 to 63
    collect (if (= (array-index board i) 0) i 0)))  ;;; return only one 
value here

(defun potent2 (L)
  (if (null L) nil
      (if (= (car L) 0) (potent2 (cdr L)) (cons (car L) (potent2 (cdr L))))))

(defun potential (board)
  (if (= (array-index board 0) 0) (cons 0 (potent2 (potent board))) (potent2
(potent board))))


;;; check potential 4 in a row: potential win check
(defun win_check_pp (board pos player)
  (setf *c* (copy-array board))
  (setf (aref *c* (/ (- pos (mod pos 16)) 16) 
          (/ (- (mod pos 16) (mod (mod pos 16) 4)) 4) 
          (mod (mod pos 16) 4)) player)
  (win_check player *c*))


;;; update potential list positions for check pp_3ina_row
(defun potent3l (L player)
  (if (null L) nil
      (if (> (win_check_pp board (car L) player) 0) (potent3l (cdr L) player)
      (cons (car L) (potent3l (cdr L) player)))))


;;;; Auxiliary Functions 
;;;; Quote:http://aima.cs.berkeley.edu/lisp/search/domains/ttt.lisp
(defun check-k-in-a-row (board x y z k dx dy dz player)
  "Does player have k in a row, through (x y z) in direction (+/-dx +/-dy +/
-dz)?"
  (if (>= (+ (count-pieces-in-direction board x y z (- dx) (- dy) (- dz) 
player) 
         (count-pieces-in-direction board x y z dx dy dz player) -1) 
      k) 1 0))

(defun count-pieces-in-direction (board x y z dx dy dz player)
  "Count player's pieces starting at (x y z) going in direction (dx dy dz)."
  (if (and (< -1 x 4) (< -1 y 4) (< -1 z 4) (eq (aref board x y z) player))
      (+ 1 (count-pieces-in-direction board (+ x dx) (+ y dy) (+ z dz) dx dy
dz player))
    0))


;;; check potential 3 or 2 in a row    
(defun pp_k_ina_row (board pos k player)
  (setf *c* (copy-array board))
  (setf (aref *c* (/ (- pos (mod pos 16)) 16) 
          (/ (- (mod pos 16) (mod (mod pos 16) 4)) 4) 
          (mod (mod pos 16) 4)) player)
  (pos_index pos)
(+ (check-k-in-a-row *c* x y z k 0 1 0 player)
    (check-k-in-a-row *c* x y z k 0 0 1 player)
    (check-k-in-a-row *c* x y z k 1 0 0 player)
    (check-k-in-a-row *c* x y z k 1 1 0 player)
    (check-k-in-a-row *c* x y z k -1 1 0 player)
    (check-k-in-a-row *c* x y z k 1 0 1 player)
    (check-k-in-a-row *c* x y z k -1 0 1 player)
    (check-k-in-a-row *c* x y z k 0 1 1 player)
    (check-k-in-a-row *c* x y z k 0 1 -1 player)
    (check-k-in-a-row *c* x y z k -1 1 1 player)
    (check-k-in-a-row *c* x y z k 1 1 1 player)
    (check-k-in-a-row *c* x y z k -1 1 -1 player)
    (check-k-in-a-row *c* x y z k 1 1 -1 player)
))

;;; update potential list positions for check pp_2ina_row
(defun potent2l (L player)
  (if (null L) nil
      (if (> (pp_3ina_row board (car L) player) 0) (potent2l (cdr L) player)
      (cons (car L) (potent2l (cdr L) player)) )))

;;; check potential 2 in a row
;;; share pp_k_ina_row with 3-in-a-row except need to update position list

;;; develop evaluation function
(defun pos_eval (board pos player)
  (+ 0
     (if (> (win_check_pp board pos player) 0) (* 1000 (win_check_pp board 
pos player))
     (if (> (win_check_pp board pos (opponent player)) 0) (* -100 (win_check
_pp board pos (opponent player)))
         (if (> (pp_k_ina_row board pos 3 player) 0) (* 10 (pp_k_ina_row 
board pos 3 player))
         (if (> (pp_k_ina_row board pos 3 (opponent player)) 0) 
             (* -5 (pp_k_ina_row board pos 3 (opponent player)))
             (if (> (pp_k_ina_row board pos 2 player) 0) (pp_k_ina_row board
pos 2 player) 0
             )
             )
         )))))

;;; loop for return the best value for one move
(defun bestm_pos_val (board player)
  (setf potL (potential board))
  (loop for i in potL maximize (pos_eval board i player)))
    
;;; a bug within this function, alternate with followed three helper 
functions  
;(defun bestm_pos (board player)
;  (setf potL (potential board))
;  (loop for i in potL 
;       for y = (pos_eval board i player)
;     finally return (list i (= y
;   (loop for i in potL maximize (pos_eval board i player))))))
;finally return (list (loop for i in potL maximize (pos_eval board i player)
) i) ))

(defun bestm_pos2 (board player)
  (setf potL (potential board))
    (loop for i in potL collect 
         (if (= (pos_eval board i player) (bestm_pos_val board player)) i 0)
))

(defun bestm_pos3 (L)
  (if (null L) nil
      (if (= (car L) 0) (bestm_pos3 (cdr L)) 
      (cons (car L) (bestm_pos3 (cdr L))) )))

(defun bestm_pos (board player)
  (if (= (pos_eval board 0 player) (bestm_pos_val board player)) 0
     (bestm_pos3 (bestm_pos2 board player)) ))

(defun final_board (board player)
  (setf pos (car (bestm_pos board player)))
  (setf *c* (copy-array board))
  (setf (aref *c* (/ (- pos (mod pos 16)) 16)
          (/ (- (mod pos 16) (mod (mod pos 16) 4)) 4) 
          (mod (mod pos 16) 4)) player))


;;; write to standard output

(format t "~%~A" "The original input board:")
(format t "~%~A" "")
;;; read lines, and write to standard output at the same time:
(let ((in (open (car (cdr *args*)) :if-does-not-exist nil)))
  (when in
    (loop for line = (read-line in nil)
     while line do (format t "~a~%" line))
    (close in))
(close in))
(defparameter *L* (open (car (cdr *args*))))
(setf b1 (subseq (read-line *L*) 0 37))
(setf b2 (subseq (read-line *L*) 0 37))
(setf b3 (subseq (read-line *L*) 0 37))
(setf b4 (subseq (read-line *L*) 0 37))
(setf b5 (subseq (read-line *L*) 0 37))
(setf b6 (subseq (read-line *L*) 0 37))
(setf b7 (subseq (read-line *L*) 0 37))

(defun update-result (board player)
  (setf pos (car (bestm_pos board player)))
      (cond ((= (/ (- pos (mod pos 16)) 16) 0)
        (setf (aref b1
             (* (/ (- pos (mod pos 4)) 4) 10) (* 2 (mod pos 4))) playermark))

        ((= (/ (- pos (mod pos 16)) 16) 1) 
        (setf (aref b3 
            (+ (* (/ (- (mod pos 16) (mod (mod pos 16) 4)) 4) 10) (* 2 
                              (mod (mod pos 16) 4)))) playermark))

        ((= (/ (- pos (mod pos 16)) 16) 2) 
        (setf (aref b5
            (+ (* (/ (- (mod pos 16) (mod (mod pos 16) 4)) 4) 10) (* 2 
                              (mod (mod pos 16) 4)))) playermark))

        ((= (/ (- pos (mod pos 16)) 16) 3) 
        (setf (aref b7 
            (+ (* (/ (- (mod pos 16) (mod (mod pos 16) 4)) 4) 10) (* 2 
                   (mod (mod pos 16) 4)))) playermark)) ))
(update-result board player)

;;; write to *standard-output*
(format t "~%~A" "The Updated Final board is:")
(format t "~%~A" b1)
(format t "~%~A" b2)
(format t "~%~A" b3)
(format t "~%~A" b4)
(format t "~%~A" b5)
(format t "~%~A" b6)
(format t "~%~A" b7)
\end{lstlisting}
\section{怪胎}
\label{sec-12-11}

这份作业其实就是个怪胎,头大脚小、头重脚轻,主次颠倒。连空格行加注释约500行
的code,开始的读入数据就占了30\%。又费了九牛二虎之力把4-in-a-row的76种可能性
按所在的平面特征用7个function一一检查了一遍。这部分code又占了25\%。而后面真正
的重点很快就结束了。而到后来等自已真正发现了绝炒的好方法,这25\%的code已经可
以完全删除不必要了。只是俺弊帚自珍,对自已辛辛苦苦写出来的code舍不得扔而已。

但我还是很愿意坦诚我是爱极了这份作业的,因为它是个怪胎,因为它充分体现了偶打
不死的小强精神。Lisp的第一份HW是几个recursion function,我对string function
就显得有些陌生,而这次当我不得不、一定得处理string的时候,我就充分发扬笨鸟先
飞、舍得折腾的精神一行一行地读入数据,一个字节一个字节地折分每一行,最终把输
入文件转化成了4x4x4的integer距阵,方便以后的操作。我甚至不惜把76种可能性归为
7大类一一检查一遍。但交作业前一天当绝大部分的同学们都写不出来,上课的时候老
师便稍微提醒点化了一下“directionality”,点化启发只给有准备的头脑,我立即明
白了这些4-in-a-row,3-in-a-row,2-in-a-row之间的相关性,并成功把从\url{http://aima.cs.berkeley.edu/lisp/search/domains/ttt.lisp网站启示来的二维Auxiliary} Functions应用到了三维空间,从而彻底解决了问题。或者说,我前面一步一步哼哧哼哧地写了那么多的code,实际上已经很好地锻炼了自已,所以越到后来code写得也就越得心应手,加上从老师的点化和网站上的启示对方法已经胸有成足,自然不用像最开始什么也不懂时那样走弯路、九拐十八弯地瞎折腾了。

后来老师给了班上一两个同学满分,而也有同学因为各种原因这次作业之后便withdraw
了这门课。在牛人倍出、在网络上广大的专业前辈面前,我无意说自已有多优秀(相反
,我还只处在这个专业的入门阶段),我只想说这份作业就像历史上那场篮球赛对于我
考研、表哥反问我为什么不把自已的事情做好一样,是给了自已极大鼓舞的。因为学习
这门偏僻的语言,大家基本是在同一起跑线上,那它也就极大地帮自已建立了专业上的
、至少编程上的自信心。
\section{告别 Mar 11 15:17:56 2013}
\label{sec-12-12}
发信人: deepwaterooo (梦魇), 信区: Dreamer

标  题: Re: 成长的故事 -- 我和舅舅

发信站: BBS 未名空间站 (Mon Mar 11 15:17:56 2013, 美东)

告别

亲爱的读者,写到这里,我已经完成了这次站出来写的任务了,清楚地解释了先前所有
的误会,也为自已与表哥的感情作了完整的交待。

走地六七年没有目标、不能静心的恋爱懵懂年华,现在的我还只是学生,还需要时间去
沉淀提升自已,这不是在网上写一两个贴子就能升华得到的。想要得到表哥,等得起只
是前提,也还得有能力在这个国度立足生存下来方可。我必须为此付出努力。

被人肉出来时,我原本只是普通得不能再普通的小老百姓,临走时我挥一挥衣袖,还自
已一个清白,也该还这个网络一片宁静。我原本是喜欢自由的人,这次离去,我应该不
会再上来了(现实生活早已把年少轻狂、桀骜不驯的我打磨成溜光圆滑的鹅卵石,以后
的生活不出意外应该不会再有重大挫折了。)也请大家转移目光,给我网络世界之外的
轻松自由。你们想要看到的(表哥与我真正走到一起)一时半会儿也看不到,而现实生
活中给予我过多的关注对所有的人都是不公平的。接下来的年月里,我相信表哥会很好
,我也会好好的,也忠心地祝愿大家各就各位,各自精彩吧!
\chapter{SOS}
\label{sec-13}
\section{潜水员冒泡兼征版友意见}
\label{sec-13-1}
发信人: deepwaterooo (梦魇), 信区: WebRadio

标  题: 潜水员冒泡兼征版友意见

发信站: BBS 未名空间站 (Tue Apr 29 16:09:48 2014, 美东)

一如大家可以搜到,我是那个过去写过 <成长的故事> 的书写者。

我小时候爸爸作过木匠、砌匠,那时我一个小丫头片子帮爸爸弹墨线盒, 搬半砖头,在这样的环境下耳濡目染, 我从小偏爱数学。虽然从小到大我所受到的都是理工科教育,但我却有一颗崇尚文艺的心。我出生在一个普通的农民家庭,小时候唯一的课外读物就是自己和姐姐们的 <语文> 课本。高中的时候渐或明白一些诗词。

我想我还需要再最后一次地记录自己的生活,这两年的精修我的生活已然发生了本质生猛的变化,个中艰辛一言难尽。作为曾经的依赖者寄生虫,作为过去两年生活被关注过的小人物,也希望能够最后一次用自己的生活经历探讨、谱写普世价值, 尽自己的最大努力最后一次发挥正能量(偶确信小苗苗默默地从小长大长到今天,已然完成了人生一半的修行),我希望分享自己的经历,为海外留学生的生活作下自己的记录和借鉴。

多年来我一直跟随你们这帮我从不曾露面的朋友,笑看大家嘻笑耍骂,默默修行,也很感谢你们大家带给我的快乐。选择来贵版续写这个故事,是因为这里有铅笔头老师,狐帅,瓜老板,可爱的小碧妹妹等等等等, 窃以为,贵版是mitbbs的一汪清泉,一缕清风, 这里有着诸多各种样的原创和自由自在的灵魂,贵版版友的灌水风格,相互间的包容,和自己的曾经写故事的痛苦经历都让偶觉得, 能在贵版续写,自己作为写作者的情绪能够较好地不受外界环境影响, 却也能受各位老师启迪精进。加上版上诸多老师朋友都有自己的连载故事,我也很真诚地希望能够向版上的各位大牛学习写作,希望能把自己的故事写得更精彩。但作为潜水员的我也弱有担心,不速之客的到来会否打扰大家的清幽闲适,会否给贵版带来不便,贵版的版友们会否介意?所以特来先行报到\textasciitilde{}~(现在还不会,计划在一两个月后等眼下的生活安定时来继写)。如若这里的版友们并不支持,我应该会选择滚回自己的dreamer老家,在哪里孤独地继续?
\section{轻量版剧透}
\label{sec-13-2}
发信人: deepwaterooo (梦魇), 信区: WebRadio

标  题: Re: 潜水员冒泡兼征版友意见

发信站: BBS 未名空间站 (Tue Apr 29 23:10:52 2014, 美东)

轻量版剧透

2012年夏天,在OPT到期后,被以前上学的小学校录取。邮件往来里导师只允许我第一学期选CS150和CS121 共计7个本科生学分的课。邮件里我直接说我有经济上的困难,一万多元的学费只能选7个不计入毕业课程的学分让我想要放弃,但导师允许我可以跳过cs121后抱着对爱情的一线希望将自己留了下来, 却成了苦难的开始。

来到学校后,第一个学期,告诉导师我老了,不能呆太长时间,愿用生命学习,争取早日毕业;第一学期同 <Programming Language> 的老师达成不成文默契:系里允许我两年毕业,但不可以争系里的奖学金,因为我又要毕业又要奖学金的话势必对其它学生不公平。看到系里学生的现状,我亦不忍心自已残忍,便默认配合老师被系里雪藏,低调,沉默,上课不回答问题,默默地自己学习。甚至老师一再给B,都没计较,毕竟学校允许我两年毕业。

2013年四月之后,导师问我找工作怎么样,我告诉他系里大牛说我学得好学得快的话当年12月就可以毕业。我喜欢现在这个专业,所以想要拿到自己的硕士学位再走。导师说即便现在,我还可以找学校等地方所非工业界H1B离开这里,但我想着自己近在咫尺的学位,坚绝没找,想要拿到自己的学位。

春天的学期里我很为自己秋天的学费操心,没有经济来源。我旁听算法的导师课堂上他暗示若能作两年研究,他或系里可能能有有限的经费可以帮助我。但考虑到还有暑假,我可以找实习,既然自己实在没有攻克三年的耐力,依靠自己去找经济来源会更好些,至少需要自己先偿试一下,便示弱装笨默拒了别人的好意。

2013年秋天,我暑假挣到钱便都交了秋天的学费(到这时累计三学期我已然交了多过\$33000的学费交尽了自己挣的每一滴血汗,毕业在望)。因为没有春天的学费,我找系里的大牛求情,想要毕业,他允许我春天part-time注册两个research学分到八月毕业,并建议我换了导师。我感激老师能为学生辛辛苦苦两年的人力物力时间投资的生路考虑,毕竟每个选择留下来的人心里都有着希望。并承诺大牛,若能在这个国度生存下来,必将为学校donate些钱好让系里老师们多点儿科研经费。

2013年秋天那时选编译课,课堂上若有感觉代课老师想要黑我,课堂气氛显得诡异,便不再沉默,整个学期正当防卫性回答问题四次,但最终还是没能逃脱得C的命运。

面对这个C的结果,因为之前的课程老师全给B, GPA 3.0以下,老师稍微一个不小心或是小动作,小兵都将死于无形,小兵便开始了刀锋上讨生活的殊死挣扎,像刺猥一样竖起全身的寒毛,破斧沉舟,殊死防卫!

秋天里后来系里小秘说有一个老师愿意支助我春天给我奖学金。因为转专业毕竟我只学了不到两年,多学一些课程还是有帮助的。加上自己始终有回报学校的心, 我便坦然接受了。拿系里的奖学金,便理所当然地选导师给推荐的课,导师让选什么就选什么,我只需要1个Directed Study学分导师希望我选2个省得他麻烦,小兵便听了导师的。我问过系里的小秘是哪位老师帮忙支助了我,但小秘不愿意,没有告诉我是谁。系里别的同学tutor十小时改一门作业十小时,他们安排小兵tutor十五小时,改作业五小时,小兵也默默承受,没有任何抱怨。

可春天里,一如去年秋天,系里舆论上仍不但施加压力希望我能拖一年,可姐老了,拖不起啊。包括最近EC的代课老师说“No Free Lunch Theory”。小兵怕代课老师不知道,便去他办公室找他告诉他说了先前我同大牛承诺过的话, 因为小兵并不愿意从来没有想要白白点用系里的资源, 一如第一学期同代课老师达成的默契。

同系里的这一切都说得这么清楚了,可舆论上系里释放出的信号似乎还是不放人。小兵实在不知道该怎么办了,不知该如何破现在的状况。

而这么多年来,小兵的身体状况日益恶化。2001年7月29日小兵在武汉市第一外科医院作了阑尾炎切除手术。因为那次发病当时正处生理期间,医院非常正规地作了B超检查,发现右侧卵巢囊肿,便在作手术时特请妇科医生在场,一并作了右侧巧克力卵巢囊肿穿刺手术。手术后伤口感染,到十一期间伤口彻底长好。来美国后2009年春夏最先感觉身体不适。但因还不严重加上经济原因,便一直拖着。2010到2012年8月OPT期间因为自己身体还过得去,这期间从来不曾买医疗保险便也一直没看医生。这次回学校读书,身体的不适一次次侵袭,看过几次小病,而这期间小兵也真正意识到作为女性她的生理周期已经紊乱很长一段时间了,而且生理期痛日益严重。小兵严重怀疑其右侧囊肿发作。虽然一拖再拖,可读了再多书如果健康都不能保证,活这一辈子又是为什么,虽然一直交学费房租,向朋友已然借了不少钱,但想到自己的身体状况,还是最终觉得健康更重要,借债也得看病!今年三月在病痛的折磨下最终鼓足勇气去看了医生,作了ultrasound 检查,检查结果子宫增生(参数是4.3 X 5.1 X 10.1cm)。小兵因为这两年在系里小镇上受到了孤立心理上略为扭曲,对医生超声波检查的结果也稍有怀疑,因为很多病征都与至少右侧卵巢囊肿相符(极小可能性手术伤口感染造成右侧脏器粘连),但在作ultrasound 检查时即使小兵当时就说了右侧更严重,希望检查员检查右边,但检查员的超声波探头依然始终只放在左边。多年来(至少从2009年开始)小兵右半侧生殖器官麻木,现又确诊子宫增生,外加怀疑右侧囊肿严重加重(至少同2001年完全没感觉比较起来非常严重了)严重的话变可能会需要切除, 小兵为自己的身体状况极为担心。

这个13岁遭遇性侵,接下来五六年生活在梦魇里,一直担心自己将来不能生小孩,甚至间接导致高考失败的小兵,在短暂地遗忘这件事几年后,近年来却又再次成为心头大患。后来诊断增生后看的妇科医生认为不容易确诊是什么原因导致的,只有三种选择的药物止痛让小兵对这边的医疗略显绝望。八年来小兵不曾回家,期间爸爸车祸离世,妈妈劳累成疾,病倒并被诊患脑动脉瘤却并未作肿瘤手术。小兵希望能尽早毕业,拥有一份工作,尽早回国看妈妈,也对自己的身体作彻底的检查。小兵希望并呼吁社会舆论监督和帮助,希望能尽早参加工作,尽最大可能保存将来作母亲的权利,希望还能够拥有一个健全、完整的人生。

小兵1979年润6月22日出生,阳历8月14日,狮子座,AB血型,今年35岁。热爱工作,是个工作狂。小兵希望能有一个工作岗位能让自己尽早治病,根除心头大患。而小兵也愿意在工作上发挥自己旁若无人,挑战极限的个性,愿意前往接受任何有挑战的工作(只要有相对不错的mentor或manager指导),加班加点在所不辞获得工作上的满足感,也以这种方式回报公司。小兵从来也不是贪心的人,在国内献过三次血共计600 CC,在美国为现在的学校donate过\$1000, 并将自己的第一辆旧车在无意修理时捐献给了社会慈善机构。小兵愿意接受社会舆论的监督,将来回报学校和社会,作一个对社会有贡献的人。也很希望知道现在的困境缰局能有什么好破法。
% Emacs 24.3.1 (Org mode 8.2.7c)
\end{document}