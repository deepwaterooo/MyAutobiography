% Created 2015-12-17 Thu 00:53
\documentclass[12pt]{book}
\usepackage{graphicx}
\usepackage{xcolor}
\usepackage{xeCJK}
\setCJKmainfont{SimSun}
\usepackage{longtable}
\usepackage{float}
\usepackage{textcomp}
\usepackage{geometry}
\geometry{left=1.5cm,right=1.5cm,top=2cm,bottom=1.5cm}
\usepackage{multirow}
\usepackage{multicol}
\usepackage{listings}
\usepackage{algorithm}
\usepackage{algorithmic}
\usepackage{latexsym}
\usepackage{natbib}
\usepackage{fancyhdr}
\usepackage[xetex,colorlinks=true,CJKbookmarks=true,linkcolor=blue,urlcolor=blue,menucolor=blue]{hyperref}


\lstset{language=Java,numbers=left,numberstyle=\tiny,basicstyle=\ttfamily\small,tabsize=4,frame=none,escapeinside=``,extendedchars=false,keywordstyle=\color{blue!70},commentstyle=\color{red!55!green!55!blue!55!},rulesepcolor=\color{red!20!green!20!blue!20!}}
\author{deepwaterooo}
\date{\today}
\title{The Autobiography of deepwaterooo\textasciitilde{} \linebreak Part 2: CS辛酸史(3)}
\hypersetup{
  pdfkeywords={},
  pdfsubject={},
  pdfcreator={Emacs 24.3.1 (Org mode 8.2.7c)}}
\begin{document}

\maketitle
\tableofcontents


\chapter{2/13/2015 前序}
\label{sec-1}

re, 1/5打电话问到新的court day是1/14;

1/14那天,他们说他们还没有收到来自警方的材料,也不知道表哥那方对这事是什么处理  态度,他们一时半会儿也不知道该怎么办,所以给我分了个律师,下一下出庭日是2/13,期间间隔一个月仅差一天;

期间2/5日去见了律师,将对12/27日警官面前讲过的话重复了一遍,我仅在1/17/2013一两天后的一个晚上,有两个警官来到我家,给我一份termporatory restriction order的复印件,并要我1/31/2013出庭,那天表哥没去,所以没有下文.我明确对律师说我知道自己没有被任何警官serve permanant order,但为给自己减少麻烦,我希望这件事尽早结束.律师说她会向警官要serve order的proof,再根据情况今天对我说下一步该如何做.当时问过她,她说警官对她说2/5她应该能收到来自警官的反馈,但她还是会等到今天见了我再与我计论下一步,中间就不再见了.

2/13今天她说还没有收到反馈,所以拖到一下次的出庭日为2/27/2015.

他们可以无限拖,但我拖不起.明天会写个序,这两天好好理一下思路,还是打算把这一年的经历写下来.会尽量以第三者旁观者视角写得客观.我是理科生,完全不会写文章,所以若有建议请尽量提.先谢了\ldots{}

\chapter{序}
\label{sec-2}

三年前,在表哥、舅舅等亲人都不发表意见、要我作自己的决定时,因着导师和系里在节骨眼上的宽容,允许我选课,而不是之前同导师Email里交换的意见我第一学期只能选cs121与cs150 7个学分,因着自己对爱情的一点小小坚持,因着导师、系里都允许选课,怀揣着两年读完MS. Computer Science的梦想蓝图,在33岁高龄留了下来,花尽自己所有的积储,来读一个计算机硕士。。。

就像四岁那年坐在外公牵着缰绳的牛背上努力想像着远方的医院和献血会是什么样子的懵懂孩童,就像十八岁那年被舅舅科普地球的另一端有一个神奇国度的懵懂想往,2012年7月,我怀揣着对爱情的一丝眷恋和执着,又一次地开车走在了通往以前学校的5号公路上,懵懂依旧,却心想着这一刻将是梦想起跳的地方。

时间过得真快,三年过去了,转眼我们都老了。这三年前一年半的时间我都过得很麻木,心甘情愿在臣服在各种委屈之下;后随着雕刻时光、风雨飘摇、争锋相对的渐变(很有量变引起质变的感觉),这第三年的两学期就演变成了手段与诡计共存,作秀与暗黑齐飞,各种出乎意料层出不穷,要想出头永无天日。。。而我现在的情况是,通过人为设置障碍,我现在已然爬到了生死边缘。。。

虽然早在2008年春转统计是因为当时老板拼tenure的欲望与压力,要我干一学期的活,但不允许我选哪怕一门功课,那时感觉自己稍微受了点儿欺压;读统计时因系主任的偏心,系主任把奖学金任凭给一对夫妇也没有给我,但好歹那几个学期我的外州费是被免掉了的。常抱一颗感恩的心,直至统计硕士读完,我对这个学校的总体认识都还是正面的、肯定的。可经历了这最后一个三年,我会时不时的恍惚,这还是我记忆中我认识生活过的那个学校吗?

受尽了各种委屈,为爱情执着到今天,被种种手段逼迫到生死边缘,我又该如何存在?19岁时我长那么大第一次质问自己,人活着是为了什么?没有生的欲望、死的勇气,我选择攒钱准备离家出走;今天困境中,我一次又一次地问着自己同样的问题,爱情的答案是明确的,可这个社会不需要爱情。。。

之前写自己的经历的时候,曾无意中提到过这个学校的外州费比较便宜(虽然后来学费也涨了不少),直接导致了这个学校小留数量激增。荐于自己这三年的过往,我也希望用自己的经历让那些缺少信息的家长和朋友更真实客观地认识这个学校。人选择可以有很多,十八九岁来读语言,七八年的时间把语言、本科和硕士一条龙全读下来只是其中的一种,逃得过逃不过博士再说。

我想此刻我还没有清楚完整的思绪,十多天的时间我大概也还说不清楚这一年里发生的事情。但我会尽自己最大努力还原这一年里事情的真相。

\chapter{QQ群惊悸见大神记}
\label{sec-3}
\section{QQ群惊悸见大神记(1)}
\label{sec-3-1}

前面有提到自己那时进入到一个QQ群,这是我第一次在QQ群里呆了那么久的时间,前后可能有数月,但也是最后一次。我对群主之前有过很多的赞誉,但后来当他及其团队将专业里的前辈群友们几年来无私奉献出来的资料不再分享的时候,也因为个人发展的需要,我最终还是离开了那里,选择了一个更适合自己的平台。 

去年暑假续写故事的时候,还远无法预见接下来一年可能发生的事情,写得蜻蜓点水、轻松随意、实在是玩乐的心态居多。 

大概是EC课上代课老师的阻拦,和这门课只得了B的事实,最终激起了自己最本能的反抗吧,我开始意识到呆在这个池塘里,我将永远只是一条小鱼,终究会被大鱼吃掉的。我必须走出来,同广袤的世界建立联系,才可能会有出路。于是那个夏天,那种纯粹的貂丝心态,我还真有想认识几个同专业里的朋友,想见见大神的心态。

那时我还积极地请过群里的一位号称“公子”的群友帮我mock interview。我想见大神的心态如此迫切,于是有一天,群里有名的交际花女神约了她自已、另一位大神(下文简称“此神”)和我(女神一提我就答应了),我们三个一起出去吃饭。我的车很破,此神即将上任成为G的一名正式员工,可谓春风得意,举手投足间已然挥洒成为风景。我住SJ小镇中心,那天中午他开着他租来的新车先来pick up我,我们再去接女神。

开车前往SF接女神的45分钟车程里,我们一路聊着天。聊天专业领域里中英文教材的区别,他是英文原版教材的坚定拥护者,读书必得读原著,否则谬误多矣\textasciitilde{}我褪不去小吊丝的本质,如果我在理解概念上本身就有一定的困难,我又为什么要打肿脸充胖子,读自己的母语写成的书,轻松愉快地获得我需要得到的透彻又何乐而不为?聊算法题,我们一致认定总结是非常重要的。缘自于冬天过一遍cc150对程序员编程序所拥有的种种控制力的体会,缘自于春天见识了别人说EC课难难于上青天,小马过河了一次,也没见难到哪里去嘛,于是我也很是大言不渐地说,等我刷完一遍LeetCode,我要在前人的基础上总结出一本最好的pdf,就像那个冬天我曾累死累活用latex干一两天体力活一个字母一个字母地敲出、一个章节一个文件地编出一本自己的Ctci pdf电子书一样。再次感叹一下2014年2月被朋友介绍org-mode后普通劳动人民的生活多么美好,再也不用为那些latex的标记发愁了。。。

后记一下,一年后的这个冬天,当我真正准备编出一本这样的pdf时,不断technical上org-mode关于latex的很多工具细节我解决起来还要花些功力,比如怎么打印出一个美观的数学上大花括号,花括号里包括几个分项,真正算法层面的总结与LeetCode官网上的电子书相比,更是差之甚远,更别提差之毫厘,谬以千里了。。。认识到这一点,便对LeetCode官网上的大牛景仰之情如滔滔江水了,那几个美刀花得实在是太值了。 

\section{QQ群惊悸见大神记(2)}
\label{sec-3-2}

到了SF见到了久仰大名的女神,长得靓丽娇小,有着白皙的肌肤,精致玲珑,非常可爱。吃饭聊天的时候我们也聊到恐怖片,我或许还是太心慈面善,我第一时间申明我绝不单独一个人看恐怖片,大有这个话题我绝不参与就此打住之意。女神附和地说过些什么,此神略有坚持地说如果看多了恐怖片,总结出他们的套路,他们其实是披着儿狼皮的爱情人生或心理伦理片的温情糕羊了,所以就没什么好怕的了。看来我不得不对恐怖重新下定义了。要怎么说来,就打个比方来说吧,我解释着,某电影情节里地上长着一根垂直向上的尖锥,画面上眼见着某人就要不小心倒在它上面了,顺着剧情,大家显然能够猜到接下来会发生什么,可我心有余而眼力不足,不忍直视,只好把脑袋转开,这一刻一定不看。。。等过半分钟,听声音,情节过去了,再接着看。。。至此我们终于转移了话题。

中间吃饭的过程我大概是吃得太专注了,很多细节都不入记忆,唯有此神某刻看我时的冷峻目光就像刚刚聊过的恐怖片情节画面一样惊悸到了我内心深处,不忍记忆。是谁说过的,对待敌人要像秋风扫落叶般冷酷无情,这是我的词库里能搜索到的描写无情最无情的句子了,可用这句来形容那一刻某人的目光,尚不为过。此神不是敌人,他是一位百尽杆头更进一步的大神,为何他此刻看我的目光让我无所适从、找不着北?哪里读到的某位女性作家、心灵写手提到过,判断自己是不是喜欢一个人,问下自己能否在这个人面前作真实的自己便知道了。虽然那时的心境或有些许叛逆,但内心里,我依然留恋着某个人的目光、留恋生活在那份目光中的恩宠,至少没人会愿意生活在恐怖片里吧。

虽然我从来号称自己是理科生不喜欢文科,但与两位大神的聊天里,尤其此神时时分享着的话语里,那天我还是被迫被洗脑。此神或正统或呆板的纯理科思维方式时刻提醒我,我是有着大二还是大三寒假奇葩经历的半个艺术生或文科生,当感性遇到了理性,我的被洗脑就被洗得彻底而憋屈。所以接下来我们也还随大家的意去唐人街逛了逛,或许因为中午的饭钱我们都是分滩的,人家女神出行一场都没受到什么照顾,女神说我们喝点儿东西吧。虽然吊丝的我成功劝退此神倒出昂贵的停车场,改成街泊,但毕竟吃饭时的停车费还是此神付的。作为QQ群里弱弱小字辈,作为那天不幸沦落成的被洗脑的对象,对女神我也还有着诸多感激,唐人街的饮品也贵不到哪里去,于是就成了我请大家喝东西。女神180度转向的冷幽默思维方式印象非常深刻。

回来的路上,大概我提到某位群里另一位富二代女神名字稍多,而那女神那时略为中意上心的男神那时在群里对我稍有照顾鼓励,此神表达不平说,那女神比男神年龄大得太多呀\textasciitilde{}我很诧意,据我所知他们仿佛是同龄啊,倒是我比大家要大上几岁,便问此神他比男神年龄大还是小(他的意不平点在哪里?),此神答曰:小\textasciitilde{} 至此,冷峻目光后我所有残存的见大神结识朋友的愿望热气都在心里瞬间凝固成冰雹迫降砸落刺痛着我自己。麻雀虽小,五脏俱全,此刻我比麻雀卑微,我不再讲话。此神春风得意马蹄轻,转眼轻舟已过万重山。

后来,为感激公子帮我mock interview,女神帮我约了公子、鼓励过我的男神、以及她帮自己和他们约的此神和公子的室友及好朋友们一起出去吃过一次饭,我好得以请公子吃饭以表谢意。男神、此神和我共一辆车,但我与他们之间的缘份并没能在这一段车程中增加多少。女神选定的餐馆也是山也遥遥路也长长,中午我吃得有限的食物吐了好几次全吐在了路上。至此,我与男神、此神的缘份便随着这段山路结束,与他们有限的交集也始终不入记忆。倒是女神,和富二代女神那在群里的温婉随和的水瓶天性影响着我去努力看轻别人的批评,若不看得那么重,伤害就没有那么深。后来我还不幸被此神黑过几次,但我知道不去搭理这种人就是对自己最大的尊重。后来我在群里的时间越来越短,直至后来鲜有联系。

\chapter{想念的滋味}
\label{sec-4}

在SJ小镇中心租的房子我只租到了两个月,到七月底,这正处在学生又一新学年的交界,不管我多么还想再住一个月,因别人要签新一年的lease已经是不可能了。所以我要回学校,明天就回。

这个暑假我跑出来,收获了什么?

我与表哥的爱情,现在到什么程度了?

10年底刚见表哥那阵儿,只能算是激情吗,曾经停留在手背上很长一段时间的一辨拇指的温热,早已消褪找寻不着;曾经在见了表哥接下来一两个月的时间,每天晚上都会在脑海里过电影,想念一个人,与他相处的每一个画面、每一个动作,回忆起来也是美的,总被感动,在回忆中甜蜜入睡。这好几年过去了,那温热早已褪去,相爱一场,带来无尽的回忆。

当表哥极其宠我,在他的呵护里,我想要怎样就怎样的时候,那教科书里在一个人面前,可以作真实的自己的感觉,我就已经体验过找到了。经历过真爱过就知道,那种感觉非常强烈,真遇到了对的人、真正合适的,当事人一定知晓!这是人性的本能吗,还是人类成长、恋爱历练的必然?

那剩下的,来自网友的"你当坚信"部分呢?你当坚信,他永远是最适合你的那一位,至死不渝的相爱下去,直到死亡将你们分开(当然网友写得比我写得好看很多,直到有一天,一个人在迷留之际的时候,另一方还能对他说一句,"有你真好\textasciitilde{}")。从感情上说,表哥一定是我最好的选择,没有人会比表哥待我更好。别人说什么门当户对、考虑经济条件等等在我这里统统不管用,所有可能存在的标准里,受限于成长的经历,我其实从来都是把感情看得最重。山遥遥水遥遥地走到老大不小,现在又四五年过去了,我曾经遇到过什么样的人,在这样的年岁我又还可能遇到怎样的人,繁华沧桑历尽,我对森林已经没有任何向往,工作上或许还可以自己再折腾折腾,但情感早该安定下来,能在一棵树上吊死,已然是最大的幸福。

\chapter{梦一场}
\label{sec-5}

见到表哥,我总是离他很近,恨不得贴着他,10年同穿某个门框的时候是,11年某天中午证求他意见我想先洗完澡再同他去学校说话的时候也是,大概只能怪眼中的风景太美。

这一次也是,我正面几乎贴着他,抬头凝望着他,感觉自己瘦而高挑、眼神很深邃,他的眼神同样炽热。我闭上了眼睛、眠着嘴巴,静静地等待着某个时刻的到来。

就像那场与表哥的告别里一样,轻着抚他的后背,我的手向着更广的地方探去。

他的前胸贴着我的后背,当他的右手触到我的腰腹部,却分明的顿了一下。原来是被我的手术疤痕给吓着了。没过多久,他的手指便顺着盐碱地游走,像是在识别它的形状。我之前想过千百回,如果有一天我们在一起,第一次在一起,会是什么况状,我以为我会让他千等万等,等到几乎就要放弃,再泪眼婆娑地对他说,我有一个疤痕,可原来真正发生时,这些都不是事儿!

我看进他的眼睛,娇纵地对他说,"表哥,我要和你在一起!"我们并排向前走,脸贴着脸,享受着耳鬃厮磨的亲昵。表哥搂着我的右肩,我左手勾着他的腰,我们紧紧依偎着在桃源里穿行,同表哥如此心心相印,心中无限甜蜜,我感觉自己是这个世界上最幸福的人。

\chapter{梦醒时分}
\label{sec-6}

醒来就知道我又做梦了,又一次地梦见了表哥,虽然之前关于表哥的梦,我早已做过好多个,手指头脚指头也该数不过来了。同以往一样,只要做得是关于表哥的梦,醒来时刚刚过去的梦境我也都一定记得的。

这些个梦里,这次应该算是感觉最亲密的了,在梦里居然已经感受到了巨大的幸福。知道在梦里爱情很甜蜜,已经醒来,当然一定要去回忆每一个细节的。呵呵,等着表哥吻我,那是什么感觉,可我的吻呢?

我做过的梦总是奇奇怪怪,不仅情节断断续续不连贯,情节场景也是张冠李戴,这梦境中的桃源明明就是我本科时系里的果园嘛,我们怎么会出现在那里?梦里的故事有时也会匪夷所思,居然还作过在水边洗自己的脱下来像是充气气球一样的自己的皮,把自己的皮都脱下来了,那清洗者本身是怎样一种存在,空气抑或是灵魂,一缕轻烟?

如果日有所思夜有所梦真有一定的科学道理,那这个梦则帮忙道出了自已的心声,我一直是在等待一个拥抱一个吻的呀!梦里千等万等是真的,我真实这么想过,不是事儿却也是自己相信的。只可惜,现实里没发生的事,就连在梦里也直接故事镜头切换了;同样无奈的还有身体上的爱抚,和那桃源中穿行时我是这个世界上最幸福的人那种内心获得极大满足的感觉。它们就像我手背上的那一辨拇指的温热,它们曾经存在过,但一段时间之后,我就只有它曾经客观美丽存在过的镌刻在脑海里的记忆,却无法再现再体会那份温热。

或许这个梦帮道出了我的心声,但梦却也没能超越我现有的经历与感悟。悲催之余也禁不住问自己:如果我所有渴望想要的亲密都只能到梦里去找,还未必找得到,就不委屈吗?

大表姐这个假期早些时候曾对我说,她一个人受尽千辛万苦,苦日子都熬过来了,还一定要去维持家庭的完整;而她之前吃过的苦,受过的难,她说就当是坚强了我们自已。到这时,梦醒时分,表姐的话我终于算是有所体会了,因为我舍不得,无法割舍这段感情、那委屈就委屈,梦不到,那就还是老老实实地期待它们在现实中发生吧。 

\chapter{新学期选课}
\label{sec-7}
七月底八月初回到了学校所在的小镇。那时只有相对两个合适的房子,一个是与我们见过面打过招呼的台湾女生合租两室一厅每月网费电费之外纯房租大概\$261,而且是这个价钱范围内离学校最近的房子了(谁说姐不爱学习的,姐可是每天白天晚上都在系里学习不怎么挨家的。。。),这是那时能找到的最合适的,我也非常希望能促成这事,回学校前就一直在打电话同她联系。但那个女生给我的理由是她原室友的导师一直还在折腾她室友,所以她们的房子可能一时半会儿还定不下来。当时的自己后悔可能我没能给那女生留下的比较好的印象,后来,经历了硕三,我开始愿意去相信她说的都是真的。

没有了别的选择,就只能住这个university professor的房子了,\$300一月杂费全包,室友与我同样的房租,她住的房间比我的要大多了,不过这些无大碍,我也不上心。只是因为没什么租house的经验,没想到不包网,后来\$45的网费与室友分,想想还是觉得这房子对我来说住得太贵了点儿。。。

回到学校,看一下class schedule就知道老师们在控制选课呢,比去年秋天更甚,去年秋天尚且还有与统计相关结合的课,只是导师动用了些小trick没让自己选成;这个秋天,是那些Neural Network/ data mining的课压根儿就不开!在另一分校的Parallel programming与本校本科生的senior design之间,犹豫了很久,最终还是本能地希望自己能有些团队合作和Software engineering的项目决定选了本校生的senior design。大牛的computer forensics这课的概念太多,加上全是生单词,连单词意思都还得google translate,根本没法选。Fault-Tolerant Systems这个课概念多,但是必须得选的才能够9学分的full time;还有一门课是自已跃跃欲试,想要去选的,就是先前给自己代programming language、达成两年毕业协议、自己甘被雪藏的老师开的android app programming。就像刚过去的春天会选EC课一样,不管之前与这个代课老师之间有怎样的过往,在自己该要去学的知识面前绝不退缩,正面面对以求解决问题。这们课的注册要求得到代课老师的同意,于是我就跑去他办公室找这个老师了。

\chapter{Android App Programming}
\label{sec-8}
老师的办公室们是开着的。我敲门应允进去在一个他对面的位置上坐下, 我对老师说这学期我想选他的这门课。或许老师虽然也早有预料,就像我偶尔也会抱一些不切实际的幻想,我说这句话的语音刚落,老师的脸瞬间就变黑拉长了。春天EC课上代课老师已然作出那样的事情了,从这学期系里的课程安排上看,他们这学期很有可能会为难我,那现在生死关头,这个学期我选课实在太困顿了,那这课选也得选,不选也得选,哪里还能顾及得到代课老师高兴与否?

老师机械地描述了这门课的基本要求,他应该还会有其它学生,但仍将会是小班代课,学生不会多于五个。我们将或组一个Team project或者work individually写出一个类似于windows paint的app。App会包括几个button, 画笔的粗细,不同颜色的选择、画出矩形、三角形、圆形等不同的形状etc。但detail的要求会根据我们的进展再作具体安排。会使用java语言。

老师的描述很机械,却也似乎透着有一股恐吓味道,用你不熟悉的语言学习编写全新的领域层次,具体要求可以根据情况随时更改。你若一意孤行,我便让你死得好看\textasciitilde{}虽然我对java还不够熟练,但老师给出的这些要求描述还都算很合理,于是便向老师坚定地表决心,我对这门课非常感兴趣,希望您能允许我选课。我选这样一门课的原因是我希望借助一个这样的外界环境,能够帮助自己形成一股environmental force,好静下心来学点儿新东西。老师的脸色从我进来就没好看过,但他还是说很显然他没有任何理由可以拒绝我来注册这样一门课。

这门课是directed study,注册学分飘呼不定,那我再问到我应该注册几个学分呢?鉴于这学期空气中飘荡着"系里课程安排来者不善"的味道,我自己首先申明我需要的是这样一股外界环境的力量促使自己学习新知识,对我来说,1个学分也是够的,我剩下的学分我可以用别的课程来填充,但我想知道老师您如何建议?老师说我是研究生了,那研究生当然就得注册3个学分了。3学分就3学分,要么在沉默中暴发,要么在沉默中消亡,I am completely into it now…

\chapter{奖学金}
\label{sec-9}
不记得是十月还是十一月的某个时间,我注意到春季学期学校的class schedule系统的课尚没有upload server,这在往年基本上是不可能的,接下来一年的课程早就安排好网上可以查询了。

回到学校,安顿下来,当然得去找系里问一下关于奖学金的事。没有奖学金,我当然也就没法在这个系活下去。其实早在2013年秋天,夏天挣来到手的一万多块钱全用来交学费了,大龄女生接近\$1400的医疗保险从那时开始起就是问朋友借的,包括部分学费都是朋友帮我交的,我从朋友那里借了三四千块钱,后来零星了还了几百的零头,到来年春天返税之后再还朋友一千,自那年秋天起,我经济最好的时候只欠朋友\$2000(信用卡里的欠债暂且不计),13年秋天我还在学校食堂找工,但因为是全职学生,我每周打十五六个小时,挣得的工资基本就只能保住自己的生活费用。来年2014的春天,我倒是拿了系里免外州外\$4000的奖学金,但因与食堂two week notice period冲突,系里我第一周不入职,扣掉了\$220多块,而那个学期我的内州学费交了\$3900多块,也就是说,我早就在花信用卡里的钱,而这个学期,我是一分钱都没得进;再加上13年夏天买的车gas pump出问题,估价要\$850左右,因为没钱我几个月的时间都是把它停在停车场,一直到后来车店降零件价允许我去auto zone买part才以五六百块的价钱把车修好了。这一年多来一直花信用卡,我两个银行的额度都快被我用完,这学期我必须得去找学校。

能同谁说呢,当然同自己的导师讲了。我告诉他这所有的实情,请注意任何时候我对自己、对别人都是相当坦诚的,没有丝毫的欺瞒。导师说会帮我想想,他可能也需要同系里再好好沟通一下商量对策的吧。过了一个星期再去找导师,他说他的c++课缺实验课TA,他可以安排我去作他课的TA。

夏天的那次站出来写也带来了系里的一些改变,比如这学期,系里的TA就不再是之前的\$4000交完内州费什么也不剩下了,现在改变成哪怕是改作业的人,也有\$4750,代实验的话一般两个section 实验课共占去4小时,CSAC每section三小时答疑共计6小时,剩下10小时改作业,可以得到\$5500; 我对老师讲我回来找房子时没有选择,只能住到了一个教授的房子,房租每月我都要付\$300,我代两个sec还是没法活,我也知道导师的课总共有三个sec,我想征求老师的同意,我可不可以代三个sec,并且给出老师理由说,因为实验课的材料我是需要一起准备的,代一个、两个还是三个sec在准备材料上花费的时间基本上是一样的,不同的是增加的需要批改作业的数量。老师带我去系里,这过去的一个周系里显然他们是作过仔细分析的,系里小秘说她不放心把这些学生交到我手上,与导师我们三个人讨论的结果是,我带2个sec,4小时代课时间加6小时CSAC答疑共计10小时;再给我每周10小时CSAC辅导学生时间,但两个sec学生作业我可以在这10小时里改,这样同样理论上每周20小时,我一个学期可以拿到\$7500。这对我来说已经是非常体谅合理的安排了,我有什么理由去拒绝呢?于是就有了后来CSAC 16小时每周的安排,就是后话。

\chapter{编程实验课}
\label{sec-10}
说起来这个学期也够怪的,大学本科生仅仅是注册cs120 c++编程的就有6个sec约150人之多。我现在去查一下春季学期的schedule也不过五个sec嘛,而且所有人都知道,开学前十天是可以 何改课的,上课一两个星期之后,还是会有不少学生会drop掉这门课,这样学生就一般会在预计的范围之内。

后来确定真正代实验之后知道,那个被微软裁回来的我们学校的本土硕士,代了EC课老师3个sec的TA,他应该也是拿了\$7500。有一次我们聊天聊起他表达了对微软的不满说,微软给他的裁员理由是他说他一年之内能够拿到硕士学位,但是一年过去了,他还没有拿到,所以把他赶出来,他非常愤愤不平,觉得自己工作还是干得非常出色的,他以后找工作会去找其它公司。

我导师的另一个sec给了一个新来的印度学生,不过出乎我意料之外的是与我同上CS121的那对男女朋友(这学期前刚过去的暑假已经结婚,下文再详说),AI课(artificial intelligence)上过没有我不清楚,春季学期两人也都没上EC课,因为是本科生,这个秋天将两人一起上编译课。女方,后文都简称E吧,本科生也作了第七个sec的TA,尽管课程只有6个sec,而且她的班上只有几个学生,不足十个,后来好像也只有七个学生。。。这样E成了几个TA里唯一的本科生, 根据系里之前发出的邮件,她这学期好像拿半个代课TA(\$5500/2)。

有一次E同我聊天,她说她非常喜欢之前哪个哪个谁谁谁的带实验的方式,就是每周的实验之前都会有一个quiz,实验课前几分钟会把上周课程、实验进展情况总结一下,然后在大致提示、演示一个新一周实验的相关要求事项等。E想听我的意见,我总是太坦率,直接说,我上过的所有的课,quiz最多的也不过是与她与她那时称男朋友的一起上的cs121共10个,每周一个quiz太多太繁琐了。。。

这门课两个代课老师,四个TA,几个人会在周五下午学生散去后聚上半个小时一个小时,聊聊天,聊一聊课上的情况,下周的实验安排等。从开学的第二周正式开始,每周实验都有quiz, 每周一个homework,每周一个两小时的实验,quiz在原纸上改, homework和lab 另一个代课老师都给的有rubric,上面哪项值几分,相当于是一个grade sheet,别看这是CS120 introduction to C++课, 当好一个TA需要的可也是专业素养。。。  

\chapter{大家都结婚了}
\label{sec-11}

\section{大家都结婚了(1)}
\label{sec-11-1}
夏天刚开始的时候,系里的女博士结婚了,老公是外系的硕士还是博士,只听说当初追了她很久。或许那时她还有意中人吧,还主办过一场当身party不知是否想给谁暗
示。这是典型的女强男弱组合,女方博士,男方农经系小硕吧可能是,女方专业上跟了大牛,大树底下好乘凉,男方家里蹲待就业,但也还是很有上进心,在家里作些网上的生意吧。后来在一起这么多年下来,也终于到了扯证结婚的这天。之后朋友对我描述说,女方没哭,一滴眼泪也没有;男方哭得稀里哗啦,泪流成河,感天动地。朋友问我为什么没去,我就知道会有这样的场面,我对朋友解释说,我也怕自己哭得太难看。。。

这一桩之后,中国留学生圈里好像接连前后还有两桩女生的婚事,但因为我与他们都不太熟,就真仅限于听说而已了。一个据说是找了个已经工作了的男的,另一个男方也是在校学生,也算是同舟共济、有福同享、有难同当的吧。

当然我记忆里这个校园最美的爱情故事男女朋友也有那么一两对。一对是女方先到一年,男方后到结果申到另一所学校里去了,因怕女朋友被别人追跑了,在那边过了一个学期就转学过来同女朋友团聚了,后来因双方都是家里独生子女,两人都硕士还是博士毕业后一起回国了,结了婚,现在可能已经有小孩了;另一对是男方环境硕士,才子一枚;女方计算机硕士,据说又会学又会玩,喜欢多项体育运动。人也好,09年秋天我初试cs120 c++,有不懂的问题女孩还主动借书给我。后来男方申了名样的博士,女方出去工作,后来我没有了他俩的消息,但关于他们的传说始终停留在我的记忆里。

\section{大家都结婚了(2)}
\label{sec-11-2}
那个14年2月帮忙介绍org-mode给我的小伙伴,是我在去年夏天站出来写之前最要好的小伙伴了,后文还是简称小伙好了,同先前夏天实习时自己的mentor很像,同为白羊座,热情大方,为人真诚,是team里的利他主义者,对cutting-edge knowledge有敏锐的直觉和洞见。其实我已经不记得是如何认识这样一个同学的了,大概emacs还是帮了很多的忙吧,我是只会emacs,他是用了vi后再转emacs的生力军。虽然我.emacs configure文件的bug总是时不时地出现,几乎是满天飞,就像现在敲字的前几天,midiUI readme怎么古老的org-mode cn-article居然不能再帮我生成pdf了,function argument void stringp nil,后来才知道总结一本算法时,source code block \#+begin\textunderscore src我把它hard code成 java了,但文件里有一处用到这个src block先前我忘加c++ /java了,就像这样as far as I can fix the bug, 我仍然痛并快乐着,从来没有任何丝毫要放弃Emacs的心思。。。。或许因着emacs我们有许多共同的话题,他用得比我晚,但用得比我更好,我对他有极深的敬佩和尊敬,他是系里性格极好最优秀的两个学生之一,另一个也是白羊座的春天已经毕业离去。

秋季学期之前我们的关系还很要好。五月份夏天还没有开始的时候,小伙对我说,再过多少天,到了哪一天,他就满21岁了。他说每次出去,他女朋友和朋友们喝酒,他就不能喝酒,特没劲,可再过不了多少天,他就可以了,他盼这一天盼了好久。。。回想21岁时的自己,那朝气蓬勃、年轻敏感的心,几幅学生画作都能让自己在艺术的海洋畅游一番达到"洗心革面"雨过天晴般的清新。这时再看年轻自己十几岁的同学,受感染于他的年青,只能笑笑,心里感慨年轻真好,我已经老了。。。

秋季学期开始之后,很长一段时间我见到过他,但我们一直没有机会说话。有一天真聊上天了,CSAC里,几个学生也都在,他告诉我们两个好消息,他结婚了,紧接着的周末他会带他老婆到镇上哪家店里去选钻戒;夏天他在微软实习,微软给他offer了,所以他工作的问题早就已经解决了。我说怎么这学期他没有像春天学期一样作TA呢,原来他是没有任何经济上的顾虑的,而且他的态度还显得对系里极其呈服,没拿到TA没有丝毫怨念。。。虽然我时常被他们的年轻感染,可一时间我还是不太能接受姐弟恋男女朋友一年左右,在一起一个暑假刚满21岁就结婚的事实,这也太考验老太太的心理承受能力了,心里暗想,我已经out了。。。

再后来,当我听说同我一起上CS121她和她那时的男朋友就坐在我前排的这对老美男女朋友也结婚了的时候,到这时这个结婚的小概率事件集中集体性发生的时候,就轮着我惊诧不已了。。。这是另一对的姐弟恋,女方二十七八快三十了,不过长得还不错,在我们这样奇葩专业的女生里是长得非常漂亮的了,男方要比她小上好几岁。那天,E在CSAC里大概是要打算打印一个什么东西,她旁边坐了一圈五六个本科男孩陪着她聊天,简直就是一个现实版的<乱世佳人>里关于思嘉丽的一个画面。。。后来她作结论说,"I knocked him down…"

\chapter{CSAC值班表 (Fall 2014)}
\label{sec-12}

新学期开始之后,系里的小秘换了,之前是前系主任的老婆。前系主任好像是意外离世后,他的老婆一直在系里作所谓的"行征"工作,不明就里的国际学生,比如之前的男闺密和另一个中国学生一直说系里的奖学金是她拥有绝对支配权,是她决定的。但明眼人一看就知道这就是个傀儡掌权者,就像系里会有硕士生讲师专干牛牛们不方便干的脏活一样(虽然大牛偶尔偶尔判断失误的时候自己也干过脏活丑活),所有这些错综复杂的存在却共同创建维持了一个看似系里大牛小牛几位导师间尖端对立、矛盾重重、实则困难面前高度统一的带fault tolerant性质的强大存在。

也多说几句吧。按说自己前导师是被系里孤立的对象,在industry工作了15年,难说不是大牛,但因为一种绝对的自负而常常耐心不足、行百里者半于六七十,还喜欢招惹小姑娘,从学生的选课也可以看出系里的分化进化,尤其喜欢编程的人可能会选他的课,倾向于大树底下好乘凉的人却一般不会。都快进土的人了还因特殊原因招惹小他几十岁的人,自负如此,但我却不止一次地看见这样一个人卑微地耐在别人的办公室里作别人的思想工作,春季我选高级操作系统时找过我导师,找过EC课的代课老师、春季甚至找到系里新来的大牛手下的讲师,到这个秋天也找过代我android课的代课老师。这样一个人此种作法,目的何在?

扯远了。小秘或许还是心太软,非常时期需要非常手段。从CSAC里值班的几个学生第一次开会,已然可以明显感觉到,此人个头不高,一切外在身高长相看起来都那么平凡甚至四十岁左右的年龄还微微发胖腹间搭有几层塔,但却时时处处透出一股心机和城府的味道。看到她,我总是怀念大学室友总结的刚从家回到学校时自己那特有的纯相清新的家的味道。这学期与往年很大的不同是,以前同一时间段可能有两个三个学生同时值班,但这学期这里工作的总人数减少了,每个时间点最多两个人,偶尔时间段因为学生选课等原因会只有一个人。因为奖学金数额增加,总人数减少,不难想像,安排一个合理的时间表也还是考验人的琐碎的。那天因着不够熟悉和了解的谨慎,也因扑面而来的心机城府味道,大家都不敢作声,异常安静,她在我们第一次拿到这张时间表,她介绍这时间表时,特意大大地转了一个圈,双手握着拳头举到肩头又放下才镇静下来说,她已经排了三遍了,她再也不想再动这个时间表了\textasciitilde{}!如此销魂的舞姿,大家却笑得好谨慎。

第一次看到这张时间表,感到触目惊心的当然是自己。因为博士生、其它几乎所有人的时间安排(一般十小时左右)最多工作三四次,在两三天的工作日排满,唯有自己有16个小时的工作时间(这个前文已经解释清楚),而且唯有自己的时间被分在了10个不同的时间段,而且每段最多一个半小时;而且一个周五天,我被安排成了有四个早上必须8:00值班!刚过去的春天我的作息非常不理想,八点钟,冬天会下雪的,但工作就是工作,一个学期下来,事实证明,反而是这个八点钟必须值班、必须早起的安排使得自己这个学期作息异常的好,每晚都休息得很踏实、过得很充实!

那时我还没有决定选本科生的senior design而是选过一个晚上7:00到十点的课,等我看见他们因选课等原因还是去找新小秘让她第四次地改了值班表的时候,我就也去找小秘了,等我把这门本科生的课选上,先前相要保留下来好去帮朋友(我先前的房东他们周六在farmer's market卖饺子包子)的周五下午的时间也不得不被占用。最终稳定下来的时间表我就被排成了集中在最早上和周五,相对于周二到周四问问题的学生会稍多点儿,加上自己被掐成那么多段,心里也会稍有不舒服,但我又能说什么呢,把自己的工作做好就是最重要的了。

那时的自己还是懵懂,没太意识到这后来11个时间段与谁共事的问题。这张时间表就像当年美国扔给日本的两颗原子弹一样埋了两颗不定时炸弹,最终在春天的时候系里拿他们大作文章特作文章,毁人清白,炸起舆论一片天昏地暗。他们以为给一个作得还不错的学生微软的return offer暗示他在满21岁刚过几个月的时间就结婚,就可以帮我树立他们美国学生21、22岁谈一场恋爱就可以结婚的信念,以便我及时热情投入到一场恋爱,但就像这个学校导师都可以对女学生乱来、就像这个小镇从来就鲜有文明一样,他们鄙视(无partner)单身、没有对爱情人生的感悟与体验,所以后来再回过头来回想这样一个小自己13岁的糖衣炸弹的奇葩设置与存在,是猪都会笑掉牙的,好吧?

虽然经历了岁月,当又一个的前导师诱惑摆在自己面前的时候,不像22、23岁不曾踏入社会晚熟的自己,到此刻自己已然有了足够的阅历与勇气去分辨和说不,但并不是任何时候的任何诱惑身在此山中的人们都能及时分辨和拒绝,前导师毕竟是比自己大很多、甚至是比表哥都还要再大出一个轮回的人,我从来都是把那美国小伙当作这个校园里最好的同学和朋友的美国人的呀,谁又何曾往恋爱结婚上去想过?若时间倒回去一年,我一定像在自己现在打工的食堂一样,对那些过于粘乎、拖泥带水、纠缠不清的人横眉冷对、站到尽量远、冷眼静看这个学校和食堂还能自导自演出多少出闹剧。。。。

\section{Fall 2014 CSAC Schedule}
\label{sec-12-1}
\begin{itemize}
\item it was posted \url{https://github.com/deepwaterooo/cs120ta/blob/master/lab1/f2014.org}
\item for convenience, one copy is listed as followed:
\end{itemize}
\begin{center}
\begin{tabular}{llllll}
\hline
Time & Monday & Tuesday & Wednesday & Thursday & Friday\\
CSAC Hours & 1.5 +1+1 =3.5 & 0 +1.5+1 =2.5 & 1.5 +1.5 =3 & 1 & 1.5 +2.5+2 =6\\
\hline
 & CSAC Matt &  & CSAC Alex & CSAC John & CSAC\\
08:00-09:30 am & CSAC Matt &  & CSAC Alex & CSAC John & CSAC Brandon\\
 & CSAC Matt &  & CSAC Alex &  & CSAC Brandon\\
\hline
09:30-10:30 am & CS449 &  & CS449 &  & CS449\\
 & JEB025 &  & JEB025 &  & JEB025\\
\hline
10:30-11:00 am & CSAC Xin &  &  &  & CSAC Brandon\\
11:00-11:30 pm & CSAC Xin & CSAC Matt &  &  & CSAC Brandon\\
\hline
11:30-12:00 pm &  & CSAC Matt &  &  & CSAC John\\
12:00-12:30 pm &  & CSAC Matt &  &  & CSAC John\\
\hline
 &  &  &  &  & CSAC Anup\\
12:30-1:30 pm &  & CS120 JEB 321 &  & CS120 JEB 321 & \\
\hline
01:30-2:30 pm & CSAC Anup & CS120 JEB 321 & CS480 & CS120 JEB 321 & \\
 & CSAC Anup &  & OH &  & \\
\hline
02:30-3:30 pm & CS499 & CSAC John &  & CS499 & \\
 & OH & CSAC John &  & OH & CSAC Anup\\
\hline
 &  &  & CSAC Matt &  & CSAC Anup\\
03:30-5:00 pm &  & CS480 AS106 & CSAC Matt & CS480 AS106 & CSAC Emeth\\
 &  &  & CSAC Matt &  & CSAC Emeth\\
\hline
 &  &  &  &  & \\
07:00-9:45 pm &  &  & CS511 &  & \\
 &  &  &  &  & \\
 &  &  &  &  & \\
\hline
\end{tabular}
\end{center}
\begin{itemize}
\item \textbf{Notes:} 
\begin{itemize}
\item Whenever the grid includes "CSAC" it means I was on duty and I need to be there in CSAC on time.
\item I didn't like the schedule that one week five days I have to be there four days in the morning at 8:00am. So eventually I changed Thursday morning one hour with M to be on Thursday between 11:30am and 12:30pm.
\item Nobody else would want to or dare to change schedule with me except M, either because he had special duty assigned by the department, or his kindness or fairness as a human being.
\end{itemize}
\end{itemize}

\chapter{Senior Design序幕}
\label{sec-13}
\section{Senior Design序幕 (1)}
\label{sec-13-1}
这学期在所有统计相关的课程都不开、对我来说没有任何其它课程可选的情况下,我选了他们本科生的Senior Design。我具备了注册这样一门课的所有要求。选课前我还是去同系里的代课导师打了声招呼,他同时也是我cs121 c++编程的代课老师。他问我选这门课需要下学期选cs481,我解释说没有任何问题,我之前几乎没有任何team project的经历,所以很想感觉一下大家一起作一个项目会是什么样子的。下学期选cs481?当然选呀。于是不在话下。

选了这门课才知道这门课的上课和效果很在自己的预料之外。这是Interdisciplinary Capstone Design,整个工程学院的学生坐在一个巨大的扇形教室里上课,只有小的几次review课分系小班上,邮件通知上课项目报告等情况。前两周的时间我们都在扇形教室里,这个学期所有的项目那两周都有相关报告人在台上作简单的介绍,以及需要哪个专业的人才。听项目的时候听的时候还是听得瞒激动人心的。假期在微软实习拿到回头offer的小伙也被协商在上面作了一个关于linux core的项目清简介绍,这小伙是我春天里在CSAC认识他来最要好的小伙伴了,借由emacs的帮助,在我眼里,我们有着非常好的互动,从春天学期最开始的时候他就要过加了我的facebook。也因大家春天一起上EC课,有时候有什么学习上、作业上没想通或是理解稍有难度的地方,包括我configure org-mode中英混编时遇到过的问题也都有在facebook上发消息请教过他。那里,我们传过显当多的消息。虽然我自己自打新专业一开始,自打开始用emacs,一直就生活在linux的屋檐下,可自我感觉学得有限,就像我有motivation很想学学android学习一些新版块的知识,我也很希望借助这样一个机会、借由与这个学校里对自己来说最好的小伙伴的合作机会,能好好再系统地学习一下linux. 小伙上台作项目报告让我觉得非常欣慰,这是别人工作上的项目,却也可以是对自己的来说绝好的学习机会。他上台作报告的时候我就清楚地意识到,这是我最想要参与的项目。

后来,那年夏天实习回学校来后倍感孤独的自己与他和印度人三人共同share一个办公室一段时间的那个之前真正流浪过的与我同一个导师的博士也上台作报告了。因后文还会出现,以后称此人为"流浪者"好了。这里也稍微提一下,那时13年秋天原本他也和我一样注册了编译课,但我导师自然知道那个特殊学期编译课的风向,所以以对他课题项目进行施压的方式逼迫他drop掉了那门编译课。过段时间,为告慰他肯听话有功,导师摆一台新买的apple显示器(台式机一体的?)没有用,他向导师征求意见想用这电脑,导师就一口答应了。他,可能也算是我导师的得意门生吧?刚过去的春天我们二十多号人一起上导师课directed study的时候,导师也给他机会要他帮我们上ZigBee无线网络部分。他也还算敬业,还给我们每个人刻了一张光盘,里面装了些需要的软件和文档之类的。现在此刻,作为小伙之外系里的唯二学生站在奖台上介绍我导师的项目,他得到了与小伙相提并论的殊荣,大概像之前上课讲课一样,我导师又分了一两个小项目让他带一带玩一玩好让他充分享受一下作为系里大哥大的开心吧。
\section{Senior Design序幕  (2)}
\label{sec-13-2}
项目介绍完之后,我们每个人是可以选择四个项目的,但最终会被分配到什么的项目,是由学校里、系里决定的。在被分配到项目之前,我们会交上去一个表格,上面包括有每个人想要参与的四个项目,按最想参与到最不想参与的顺序,附上最新简历,以及一些相关问答,比如有没有什么即使不选这门课、不作项目也不愿意与他合作的人?以及同专业里你有哪些好朋友?前面一个我填没有,后一个我一定是第一个列了小伙、还加上了后来这门课项目manager那对新婚美国男女夫妇的e和她老公,以及春天EC课前了解到spring break期间千里迢迢赶回家的m以及一个中国学生等(后面的顺序我不记得了)几个我认识、留下普通印象也都还不错、并且也都在选这门课的本科生们。敲字敲到此刻,回想这里写到的至少两个我认为也还算、够得上算是朋友的名字,后来却发现就像是命运的安排,我眼中的朋友在利益立场面前站到了我的对立面,最终大家形同陌路,当然这是后话。

可真正让我们选的时候却还是感觉受到各方面的限制,比如什么知识产权一类的合同,看得相当的头大,我也就大致地问了一下其它同学选校内与校外项目的不同。最终我选的项目第一个是小伙的Linux core,第二个大概还是EC课老师的AI吧,第三个如果不是android app课老师的多线程编译就应该是我导师的什么项目了。大概最后一个才是导师现在的这个tower light。之所以把导师的这一两个项目列在最后,大概也难逃因着熟悉引起的心理取舍吧。之前在哪里什么场合我说过,其实我对自己的老师或是mentor是有一定的要求的,我需要一个严格的mentor,这样我可以按时保质保量地完成自己应该做的事情,心肠太软太好说话的人做不了;我也需要一个有洞见inspiring的mentor,一个老师讲问题讲到第四遍自己才听懂,又如何祈求别人能够听懂?

正如大家所知道的,第一个项目我是一定参与不进去的。小伙在幼小的年龄结了婚,拿了别人家的回头offer,作了别人家的项目,这样的项目参与机会又岂会是我想要参与就能参与得到的?其它有什么人参与到那个项目我不知道,但系里那个五短身材的男生参与进去了(可能150-155cm吧),他是大牛和系里大牛新招讲师的跟屁虫,一有什么风吹草动赶紧向系里打小报告,去年春天这轻量版剧透一贴就是他向系里打的小报告。这是系里我最不愿望招惹的唯一学生。经由这个项目的人选与走向,到这时我慢慢观察就开始发现,小伙已经越来越刻意地远离我了,(后来的日子不下十次地证明他只在有限的时间在系里相关老师的诱导下出现在我面前,而其它时间大家是不接触的,当然这是后话。)这样看来,春天里与这位小伙的一切互动都有了种雾里看花水中看月的感觉?那时我是知道他有女朋友的,他也好多几次地提起过,但我一直以来从来都是把小伙当好同学好朋友看待的,如果他因为结婚了就要疏远我,那也只能说明我们友谊的天秤并不平衡。如果作为同学和朋友我不能提供足够有益的价值,那不如我默默地走开。。。。
\section{Senior Design序幕  (3)}
\label{sec-13-3}
正如大家所知,后来就就被分到了现在的这个项目tower light。小组里有五个人,就像代课老师之前问我我的,这是一个一年的项目。组里五个人,新婚美国夫妇里的妻e, EC课上spring break千里迢迢回家的m, 后来知道早在software engineering课上就在一个队里的r,冬天就将本科毕业的p和我。

再多说几句e 吧。12年秋天e 和她老公那时还是男女朋友,让课的时候他们大概坐在靠门口侧第3排,我就坐在他们后面。我们这个专业的女生不多,可能e自觉自己足够漂亮或是气质出众,很有意向在课堂上表现,不是说好好回答问题,而是常与那个硕士代课老师对些脑袋急转弯一样的话,或许她真的只是想平衡一下男女学生在课堂上的表现。不过,因为与这对男女接触不多,人性本善,尽管这个女生的第二次考试考得很糟糕,但我从来都觉得他们两人应该也还都不错。所以上篇提到,写有哪些朋友的时候,我就把他们加进去了。

关于m.这个春天之前我不知道系里有这样一个人存在。借着春天第一次来到系里CSAC辅导学生写作业,认识了几个这里工作的学生,当然这个春天他也同时出现在EC课上。话说回来,EC课上的男女,除了几个CSAC里接触多一点儿的学生,其它又何尝不是停留在上课回答几个问题的份上?或许还是离开家太久,当听说有坐我后排的这个男孩千里迢迢地也回家过,我开始对天下能回家的人都羡慕不已!这个同学略高傲、强势,学生问他问题会倾向于用强制性命令往下一步探计,属于把你领到正确的方向上去了,你自己再去好好琢磨琢磨吧。CSAC里我们辅导学生的学生之间也会常常探讨一些问题,我也有问到过他一些关于c++和EC课的问题,他比前导师更耐心不足,常常前半部分回答得还算和气,后半部分就冷场了。我只是促进邻里友好系问个问题而已,若你觉得我烦人,那我就离远点儿,没什么大不了的。所以接下来某次我被又一次冷场后,下一次当我与他必须得同时在CSAC值班,我就逃到屋子里去坐了,而不是像正常建议的那样坐在前厅。或许他也意识到他嘘我过了点吧,后来稍微好点儿了。后来我就还是回去正常坐了,但我慢慢很小心在他面前问问题不要超过三句,免得他没耐性了会发火的。

对r 了解不多,P是这个冬天就要毕业了。组里前几次开会都只是完成一份所谓的team contract。作为自打cs121以来一直带有的哗众取宠的性格,组里男生们常常会逗她开心。比如她搞了个什么Princess taste like chicken blah blah之类的邮箱,他们就赶快去youtube上搜她发的视频之类的。新学期的项目也算是在平淡开心中开始了。

\chapter{这学期状态}
\label{sec-14}

正如前面值班表里提到过,到这学期,自己的作息已经非常有规律了,学习效果也还不错。只是我这人也总是活在自己的世界里,对外界的信息很是后知后觉。 

八月初回到学校的时候,虽然说家里是包床包家具的,但那个装饰用的方形table自然是作不了学习用的课桌的。网上有张长方形木桌子看起来很不错\$20,在一堆桌子里这个价钱也还算合理,我一看见就发邮件过去了,但是人家说是下午五点已经有人约好要来看了,他说到时若不成交再与我联系。后来五点过不到一刻钟,贴子删了。后来我继续找桌子,在效外一家仓库里找到一张以刚才那张比较像,桌面大,方便放两三个显示器,只是表面刻损得厉害,找到这样一张桌子已属不易,桌面就再想办法吧。后来因为意识不够我把学校自己办公桌上的玻璃拿回家去放在上面用了一段时间,办公室里,因为是自己桌子上的玻璃,不用就是了。后来据说系里的大牛在他的课上把我暗讽损得非常狼狈,听说来心理很不舒服。

这学期的fault tolerance课,原本是讲计算机的fault tolerance,可这课却很多时候若有若无地感觉一阵阵地老师的讲课话里带话,一句一句的关于lie,问题是谁lie了呢?就像春天学期的EC课,在那个快学期末的时候代课老师一定要去讲那个什么选择,当代课老师们用他们授课的内容来传递什么话题,我当然悲愤,但我光悲愤能起什么作用呢?虽然学得也还算扎实,但心里却也无比苍茫。后来想想,自己对环境也实在是没有任何的掌控去,不如放空心思去,把自己能做的事情做好,不再去想他们想要借助授课内容去传递什么话题,争取好好过好自己的每一天吧。于是我淡淡地在CSAC里工作着,淡淡地写自己的作业,时空在在我这里停止,因为所有的希望都被这无比压抑的环境阉割了。直到有一天,我因偶尔前一天没能休息好,在原本应该3:30pm老师不在之后回家休息去了,但老师贴到网上带着黑我性质的另一个学生的进度惊醒了自己。不是因为我的进度或是另一个学生的进度,而是能够感觉到的老师的黑,他也不是第一次这么黑我了。但这一次却让我觉得关闭到自己所有的触角与掩耳盗铃有什么区别,而且我只会死得更快更难看,拼实力拼体力,无论拼什么我都要坚持到最后!

\section{Tower Animator - Team Contract}
\label{sec-14-1}
e (e@xxxx.uxxxx.edu)
Sent:        Thursday, October 02, 2014 4:01 PM
To:        
代课老师 (代课老师@uxxxx.edu)
Cc:        
(me\textasciitilde{}) ((me\textasciitilde{})@xxxx.uxxxx.edu); m (m@xxxx.uxxxx.edu); e (e@xxxx.uxxxx.edu); r (r@xxxx.uxxxx.edu); p (p@xxxx.uxxxx.edu)
Attachments:        
TeamContract\textunderscore v2.pdf? (106 KB?)[Preview on web]
?Hello Prof. 代课老师

A digital version of our team contract is attached. Our team will complete the Client Transcript assignment next week.

I would like to organize a meeting with you next Thursday, October 9,2014 at 3:30 pm. The purpose of this meeting would be to review our team and individual progress by inspecting our logbooks and portfolio. If there is another procedure for completing this assignment then please let us know.  

Thank you,
e
\&
Tower iLLuminati

\section{Re: Tower Animator - Team Contract}
\label{sec-14-2}
e (e@xxxx.uxxxx.edu)
You replied on 10/10/2014 4:52 PM.
Sent:        Friday, October 03, 2014 1:29 PM
To:        
代课老师 (代课老师@uxxxx.edu)
Cc:        
(me\textasciitilde{}) ((me\textasciitilde{})@xxxx.uxxxx.edu); m (m@xxxx.uxxxx.edu); e (e@xxxx.uxxxx.edu); r (r@xxxx.uxxxx.edu); p (p@xxxx.uxxxx.edu)
Attachments:        
TeamContract\textunderscore v2.tex? (11 KB?)
?I have attached the \LaTeX{} version of our team contract. I changed the font size to 12pt and also altered some copy/paste errors.

e
\&
Tower iLLuminait

From: 代课老师 (代课老师@uxxxx.edu)
Sent: Thursday, October 02, 2014 6:46 PM
To: e (e@xxxx.uxxxx.edu)
Subject: Re: Tower Animator - Team Contract

?
I have your group on my schedule for next Thursday.

代课老师

P.S.  I thought you were going to send me the \LaTeX{} file\ldots{}

\section{From: (me\textasciitilde{}) ((me\textasciitilde{})@xxxx.uxxxx.edu)}
\label{sec-14-3}
Sent: Friday, October 10, 2014 4:52 PM
To: e (e@xxxx.uxxxx.edu); m; e; r; p
Subject: RE: Tower Animator \textunderscore GUI Interface

I want to get access to the interface the other day Dr. 我导师 has showed us when we have our client meeting. I know some of you have that interface, I tried in CSAC to load that .vcproj file, it built but failed to get the GUI interface.

Anybody can help give simple instruction how to access and get the interface, or at least help attach an interface snapshot?

Feels like we still get lots of things to do for the Snapshot day. Let me know if we need to gather together to prepare something materials.

thanks,
(me\textasciitilde{})

\section{RE: Tower Animator \textunderscore Simple GUI Interface}
\label{sec-14-4}
(me\textasciitilde{}) ((me\textasciitilde{})@xxxx.uxxxx.edu)
Sent:        Friday, October 10, 2014 8:06 PM
To:        
e (e@xxxx.uxxxx.edu); m [m@xxxx.uxxxx.edu]; e [e@xxxx.uxxxx.edu]; r [r@xxxx.uxxxx.edu]; p [p@xxxx.uxxxx.edu]
Attachments:        
Screenshot from 2014-10-10\textasciitilde{}1.png? (17 KB?)
hi guys, 

now i believe m is correct, Qt thing is really not that hard. I have worked two hours on it today and got a simply menu bar interface. 
I attached it in case anybody else except me think maybe we could go ahead and create a simply interface for our Snapshot day. 

Let me know any ideas you have about interface, and according to your response, let's see if we need to target make any progress this weekend. 

thanks,
(me\textasciitilde{})

\section{CS499 class today and html Grade display}
\label{sec-14-5}
(me\textasciitilde{}) ((me\textasciitilde{})@xxxx.uxxxx.edu)
You forwarded this message on 10/13/2014 11:53 PM.
Sent:        Monday, October 13, 2014 11:50 PM
To:        
Android代课老师 (Android代课老师@uxxxx.edu);android同学 [android同学@xxxx.uxxxx.edu]
Dear Dr. Android代课老师, 

Sorry that I missed today's class, I thought today's class was cancelled or something. I will go to Thursday's 2:30-3:30pm your office hour to discuss about this class and today's missed class contents, or I will try to schedule a time to meet you if I am not allowed to occupy your other class' office hour time. 

And one more thing I want to mention is that, I don't think it is an appropriate way/manner to put our performance online, even we have only two classmates. I know most of the instructors won't share student's grades with other classmates, and I don't remember I have any instructor who have ever put student's performance online. So I am wondering if you really think this is reasonable or just put there by mistake. 

I look forward to hearing from you. 

Sincerely,
-(me\textasciitilde{})
\section{FW: CS499 class today and html Grade display}
\label{sec-14-6}
(me\textasciitilde{}) ((me\textasciitilde{})@xxxx.uxxxx.edu)
This message was sent with High importance.
Sent:        Monday, October 13, 2014 11:53 PM
To:        
Android代课老师 (Android代课老师@uxxxx.edu)

From: (me\textasciitilde{}) ((me\textasciitilde{})@xxxx.uxxxx.edu)
Sent: Monday, October 13, 2014 11:50 PM
To: Android代课老师 (Android代课老师@uxxxx.edu);android同学
Subject: CS499 class today and html Grade display

Dear Dr. Android代课老师,
\section{Re: CS499 class today and html Grade display}
\label{sec-14-7}
Android代课老师 (Android代课老师@uxxxx.edu)
You replied on 10/14/2014 1:01 AM.
Sent:        Tuesday, October 14, 2014 12:45 AM
To:        
(me\textasciitilde{}) ((me\textasciitilde{})@xxxx.uxxxx.edu)
(Me\textasciitilde{}),

I am not aware of having put performance online.  If you mean my record of what we are doing, OK then I will make that file private. It did not contain grades, but it did contain a summary of some of the things we have done so far.

Thursday office hours is a share item and probably not the best time if you want a whole hour, but it might be fine for stopping my for a few minutes and discussing where you are at.

Dr. J

\section{Re: CS499 class today and html Grade display}
\label{sec-14-8}
(me\textasciitilde{}) ((me\textasciitilde{})@xxxx.uxxxx.edu)
Sent:        Tuesday, October 14, 2014 1:01 AM
To:        
Android代课老师 (Android代课老师@uxxxx.edu)
Yes, please make it private or simply send email to us so we know among us three. Thanks.

> 在 2014年10月14日,0:45,"Android代课老师 (Android代课老师@uxxxx.edu)" <Android代课老师@uxxxx.edu> 写道:

\chapter{M}
\label{sec-15}
\section{M(1)}
\label{sec-15-1}

既然校与食堂造谣已经造到空上份上了,那就先写关于M(M in SCAC Schedule)的部分吧。 

从哪里说起呢,值班表。对自己的工作,对系里的安排,我还是总是显得很小心。比如当小秘以夸张的舞姿宣称她再不想改第四次值班表的时候,没有看见别人去改我就不敢动;比如第四次修改后新更换出来的值班表没有别人在上面涂改exchange更换时间段,我也还是不敢轻举妄动。大家已经知道我要说什么了,所以看见那个新表上已经有人更换了时间段,我也就开始打我周四早上仅有的8:00-9:00那一个小时的主意。我问过好几个人,比如新来的同我一起代导师编程实验的印度人,他那么精,怎么可能会与我换?问过两三个美国人,没有愿意与我换。而我那天早上接下来也就是9:30-1:30还是12:30一直是m值班。于是不记得是哪天与他一起值班的时候,还有另一个美国人也在,我就向他提出了能否与他交换一个小时的提议。他倒还好,就真同意了,反正那天早上他也得早起,就再早起一个小时,把我换到了11:30-12:30. 在系里没有别人的任何人敢、愿意与我交换一个小时的时候,好歹M这么做了,还是很感激他的。我周二早上可以稍微多休息会儿,之前周中有时还是太累的时候会在周三中午时间相对充裕的情况下回去补个午觉,现在周四早上也可以稍晚点儿起床了,感觉这种一周五天甚至周末也在学校学习的有规律的作息,这种灵身心合一的状态,不烦恼不焦燥,非常有益身心健康。后来不在系里工作被系里没收了办公室之后,身体立即微微发胖,作息再度失调,当然这是后话。

Senior design课上作那份合同的时候是需要我们所有人的github以及邮箱的相关信息的,M的邮箱是带年号的,就像是blahblah92@gmail.com。我们兴奋完了美女奇葩的princess taste like chichen blah blah之后,不记得是谁问起,M解释说他是92年出生的,按中国的属相他就是属猴了,比自己要小13岁,年轻真好呀。后来他们自己男生自己比较,得出了他是组里最小的一个的结论。生活在中小城市,我的三个姐姐结婚都很早,她们三个都在各自20岁就结了婚,大姐更是结婚没多久就怀孕了,所以在92年我13岁的年龄,姐姐姐夫一家就在初夏时节给我新添了一个属猴的侄儿。开学前我亲自在大人的指点下带过小侄儿好几个月。尤其记得某天他生病了,身体不舒服,姐姐让我抱他出去玩儿,可能还是恋妈妈,在我刚走出家门的路上,他一小人意识到这点儿,两小粉拳头乒乒乓乓地往我肩上砸(这份记忆回忆起来真是温馨)。现在一个小我13岁的M,在远离家乡思家心切的我眼中就像那个刚入社会的侄儿,希望他一切顺利,心中也有些许疼惜,以至于第一次snapshot他像小孩子一样一边表达对e的不满,你又用指甲抠我了!("you poked me\textasciitilde{}!!")一边show他的小胳膊时,我情不自禁地对e 说,"he behaviors like a baby now\textasciitilde{}~"男人都是视觉动物,他们大家都很喜欢逗e开心,在senior design扇形教室上课时,喜欢哗众取宠的e 自然是会微偶露那么一两声的,M甚至在同一课堂上前后间隔不到三分钟紧接着E自告奋勇回答一个什么问题,这是追爱的节奏?

新学期开始后,知道那个春天因为EC课上我一度过于兴奋以至于让系里得以大打提问题回答问题不make point的我说最世俗的老美因与M一起上information security课也常常来CSAC陪M。因为长得比较帅,后文还会简单出现,就暂且叫他帅哥吧。帅哥这学期通报我们的消息是他有女朋友了,是个韩国女生,可能也是本科生吧。他也带那女生到系里一次,帅哥人个头中等偏上,女朋友个头偏矮,但性格可能也都还好吧。只是经历了夏天几个月的岁月,或许也因有了女朋友,从帅哥身体上的变化也不得不感慨岁月是把杀猪刀。帅哥喜欢聊天,有些幽默感,所以有三个人的时候常常我们就忍不住聊天。有一天,他们说起他们information security的作业,M提到他原本是怎么做的,后来"then Amy called", 帅哥怎么没再接话,我望向M,小样儿,一脸的兴奋与陶醉,不难猜测,他应该是爱上了一个人,至少是喜欢上一个人吧。

\section{M(2)}
\label{sec-15-2}
当人们沉浸在自己的世界里为各自的理想努力的时候,至少对我个人来说,我就不太能顾及得到网上、谣言流言在传些什么说些什么。13年夏天实习热传关于我mentor与我的部分、14年春天我与小伙在facebook上传过N多消息,mentor与我的关系又被回锅肉般地炒了一回、以及这个秋天学校加大马力地想尽各种办法为我刻意安排制造绯闻。可真正现实中的人们呀,该笑岔气的时候还笑岔气、该传消息还在传消息,该干嘛还在干嘛呢,哪里有精力管得了这么多?网站上要做广告要流量可以理解,但关键时刻该合作应该也还是得合作的吧。

这一季秋天,系里总导演安排下伙休假享受生活去了,所有在CSAC里一起工作的认识的学生里,就只有与M还算是了解时间最长,从春天EC课春天CSAC就认识,只是那时因为要用emacs的缘故,CSAC里我与小伙互动更多。春天与我们一起上EC课的印度人也去做课题项目了,不在CSAC里,之前编译课上认识的一帮美国学生在大多在春天本科毕业离去,这所学校,也就只剩下个M成了这个系里我认识最久、最亲密的小伙伴了。加上我沉浸在自己的世界里,对外界的舆论判断就更迟顿了。但关键时刻我还是有心思的,比如秋天的career fair,我都还是穿了之前13年春天从goodwill捡的几件浅黄色衣服里较浅的一件去了(免得太扎眼睛)。校园里传M与我怎么样怎么样的时候,既使我们在CSAC里同一时间段上班,我们相互都已经前厅里屋地分开坐了,话都少说,怎么还在传?后来也不知道什么原因,可能买买题的风刮得大了点儿猛了点儿吧,校园里也是沸沸扬扬,我已经开始束手无策,不得不考虑要不要再冒泡了。。。。

十月底,有个南瓜节。系里邮件说是切南瓜,定在了CSAC里周三下午4:30pm. 周三下午我3:30开始上班,M那天下午从2:30pm就开始值班。之前我不知道,当3:30pm我到了之后,见前厅空着,M应该是坐进里屋去了,我就在前厅坐下,开始忙自己的活。后来四点多钟,很意外的,小伙来CSAC了,我看见了小伙?!真不敢相信自己的眼睛,这是我新学期第二次见到小伙,第一次他告诉了我那两则结婚和工作的新闻。 眼见着我导师拖着一平板车有十一一二十个形状各异的南瓜,后面跟着两三个小跟班也来到了CSAC里。不进里屋不知道, M右边并排坐着一五观精致玲珑的美国小美女,虽然我早就知道M喜欢一个叫Amy的女孩(为敲字方便,虽然出现有限还是简称A吧),但及至今天真正见到,第一反应仍是意外,随后才是为他开心,哦\textasciitilde{}~\textasciitilde{}我说今天CSAC怎么这么安静,原来是有天仙驾到。。。。他俩都在用笔记本写着各自的作业程序,偶尔交谈一两句。

\section{M(3)}
\label{sec-15-3}
不知道到底该如何描述那个学期前半学期的心境,就是听fault tolerance课老师讲课常常会觉得老师的授课内容话中带话,但对这种现状自己又无能为力。想接地气地像蚂蚁像麻雀一样的生活,一方面对android代课老师把学生的进度表现贴到网上不满就直接开门邮山地表达出来了,并发了一个high importance,一方面又对系里抱着幻想,试图好好沟通。这不,自发地想要帮导师把tower light项目作好(10/17号的事,等M部分写完再补),导师带人来切南瓜,小伙招呼我切南瓜,我也就去切个南瓜吧。

后来我知道小伙今天会出现在这里也是因为我导师。他切了个lambda,切了个42, 我好歹算是个走心的人,切得也足够奇葩,相对于他们选矮小敦实的胖胖的南瓜,我挑了个高挑傲然挺立的直立行走型的把实习公司的名字七个字母花了一个小时的时间写在上面。期间我突然意识到我要拿那个字母"A"怎么办?白羊与狮子多少还是有些互相欣赏的吧。小伙依然热情如故地教我说怎么样怎么办,于是我就照办了。

在这样一个懒洋洋的节日师生情、友情、爱情氛围下,我突然八卦心起,小声问小伙,"Do you know if they are boyfriend and girlfriend?"小伙没有清听我说了些什么,在我对面切瓜的他居然把头伸过来,直接把耳朵递给了我。各位看管,这里请允许我小小地醉一下。。。小伙的举动太让我意外了,实在没有想到他居然还这么温情;也大言不渐地宣称一下,感激这份来自小伙的举手之劳般的温情不仅仅只是人类的本能,是所有哺乳动物的本能。。。于是,我把刚问过的话再小声问一遍,小伙也没觉得有什么,只是说他也不知道。

接着小伙也说了句让自己摸不着头脑的话,他说他听说流浪者感冒了,病得厉害。。。然后他回头去问我导师,我导师也肯定的说,是啊是啊,一边说居然还在一边看我。。。我心中暗自不爽,心想流浪者病不病,病得有多厉害,与我又有什么关系?不过他们说过了也就不大在意。

后来导师与我稍微讨论了一下tower light那晚失误可能解决方案。至此这个南瓜节在这个下午就算基本过完了。但这短短的一下午、一个小时的时间,却像一个缩影,浓缩出各自心中的想往与惆怅;又像一剂中药催化剂,把百足之虫错综复杂的矛盾摆上台面来。这个节过完了,导师,我,小伙,大家都悻悻的。。。

\section{M(4)}
\label{sec-15-4}
后来第二天周四早上,我还是有碰见M,便笑笑地八卦问他,昨天那女孩是不是他女朋友?他说我说的是Amy呀,他是Amy是朋友,但不是他女朋友,我还不忘赞扬他朋友说"I think she is very pretty."言下之意你应该考虑 一下让她作你朋友。M是稍微有点儿胖,但是总是收拾 得干净利落,不像流浪者总是显得拉塌。女孩儿虽然个头不高,但身上没有一点儿多余的肉,很精致,就外在条件考虑配M我觉得应该是绰绰有余了。

其实我忘记了也从来没有去确认一个情节。春天上EC课的时修M的左边坐着一个短头发的清瘦女孩,因为上课基本上从来不发言,所以我很少注意到她。 但是能够感觉到EC代课老师对这样一个美国女生的偏爱。或许也是受这个影响吧,M对他左边的这个女生极其要好,而对我就常常带理不理的。因为M也就我一同学,春天EC课上他左边的女生现在回想起来可 能就是Amy呢,但我对这些不太进脑子。

后来在CSAC里又见到过Amy几次。只要Amy来到这里,当然就是坐在M旁边的,而M与我也很久以来不曾再同时出现在前厅了。但一次我linux下连接的CSAC打印机连不上,打不出来,我就直接用里屋靠前厅的最近的一台电脑准备登录打印,可意外的发现被M lock了,就是他登录一段时间后没有动静,电脑就自动lock他的登录了。他也在里屋,屋里也还有好几个学生,我便大大方方地问M是不是lock电脑了(回头发现A也在),他让cancel出来; 我不知道从哪里cancel,CSAC里学生那时还算正常自在,于是他便继续他一贯的风格坐在A旁边遥控指挥从哪里点一个什么;后来我试了,弄成一个花花绿绿大大小小好几个窗口关不掉。知道我出不来,他走过来了,右手用鼠标自己在弄,我已经站起身来离开座位,不能让人家A吃干醋呀。

后来似乎有学生对于我问MA是不是他女朋友的事情耿耿于怀。我从来就是大大列列,没有去想那么多,这个强按在我头上也实在说不过去呀。后来感恩节之前某天我们senior design会后他急急冲冲地要回家,原因 是他弟弟有一场重要的体育比赛,还有用他自己的话说"my own shit",不知道是关于春天选课还是关于读不读研究生,还是关于有一个意向中的女朋友A,他没明说,E没问我也就不再多问什么。

后来某天CSAC里M与我同值班的时间,系里小秘来找M问他感恩节期间有没有时间?很意外的是M居然有看我一下,让我感觉极其诡异。而且系里小秘发关于感恩节期间的一封邮件,也让自己很是摸不着头脑。于是再后来感恩节前一周开会大家再问感恩节期间打算M说depends.

于是感恩节前一周某天,我注意到M办公室的记事板上神奇地出现了 一副素扫画,说是素扫,其实是几根浅条勾勒出来的人形,记事板了一身形酷似M的小人儿穿条短裤,头上顶着一句话:"M IS COOL"那画儿还是有些水准的,至少我画不出来。看见这个后接下来的CSAC值班,帅哥很久以来再次出现在那里,但我已然顾不上他们的聊天,忙着写自己fault tolerance第二次考试呢。

M在E指责会做出GUI是做不了不该做的事后帮我挡过一次,非常感激。后来,我不再到CSAC值班,从senior design被代课老师要求退出来,至此我与M之间就再也没有任何联系了。后来春天我导师作为senior design的client和导师对我meeting时间的控制,试图把我们绕在一起,制造了一次舆论;第一次见到导师后接下来他要求我做play/stop Qsound这是导师要求的,我不可能不做;之前senior design分工时原本是M是lead programmer,我也想作这个lead programmer, team里同意后就改成了我;我离开后,可能就还是M了吧。秋天的学期里我有见好几次M向大牛亲自汇报什么东西进展,我不知道m有选大牛什么课,还是senior design的project,但至于M为什么要跟着我一起update github 进展,我不表楚,也与我无关,我顶多也就学习参考一下他的做得有价值的方法而已。在我眼里就像我对自己的侄儿会满怀希望,我觉得在这乌殃殃没有任何别的人肯与我交换一个小时的环境里,不管他是出于他被安排到周四我之后意外系里的意思,还是出于助人为乐的真心,M是像金子一样闪光的人,只是因为年轻,厉练不够,将来是可以成为有大作为的人。但这些与我无关了,感激这个人金子般的心灵就好。

\chapter{Towerlights Tonight! 10/17}
\label{sec-16}
\section{Towerlights Tonight! 10/17 (1)}
\label{sec-16-1}
我的导师是个大忙人,也是个音乐爱好者,平时喜欢与学校里其它相关人士、音乐、舞踏学院一起组织开个什么活动呀之类的。这个导师手下的好几个项目senior design的,硕士生的博士生的都是他为满足他的音乐上的需求想出来的。 我们senior design的这个Tower lights就是。又是一年一度home coming季,我的导师又计划用这个项目来安慰抚慰那些归来的人们了。。。。

那天周五下午TA的会开完,就感觉导师一直在试图拉着自己参与进来,比如其实我只是问了一个极简单的什么,他居然说要把test用的样品板连接起来在测试板上demo给我们几个学生看那个楼上大窗户的Tower lights是什么样子的。一方面感动导师是从什么居然开始真的这么重视自己了,一方面又为导师这想要做的简单的室内demo居然也显得很费周折。到五六点钟我都该要回家吃饭去了还没有弄好。而同时,小伙(就知道前面我记错了,这个是第二次见,南瓜节第三次见)、senior design zigbee项目的负责人(这里也可以看见关键项目还是始终由那个真正懂得的人才守住,要不然不知道学生们的这些个项目会做成什么样子的呢\textasciitilde{})以及其它几个与导师做过相关项目的学生都来CSAC开始要试图室内测试了。因为导师的重视,我临走时与导师打好招呼说我吃完饭就回来,看到时大家调好没有。 

这个时候导师的几个得意门生才算是真正都在了,却感觉还不太我头绪的样子,这晚饭我哪里敢拖累,冲冲扒了两口就回学校了,他们居然真的还在调试。还是那几个人只多了一个女孩子陪坐在小伙的旁边。我简单地问了 一下,似乎所有的工作都结束,仅只小伙还在攻坚,他在努力用他的Mac试图连接pulse audio但连接上好像有问题,在google各种方案。我对小伙 说现在所有的压力都在你一个人身上,便拖把椅子坐到了小伙的另一边以示舞励。 看他搜什么样的关键词,他是怎么用terminal的,从一个新翻开的网页他是如何滚动去挑选他想要的相关信息的。后来小伙左边的女生接到电话必须得去办一件什么事情,剩我一个人陪着小伙"攻坚"。

不知道导师是估计时间差不多了,还是显得稍微古董看不惯我这么坐在小伙旁边,其实刚才的女孩子不是他老婆不也是就这么这种距离地坐在他左边吗,而且我们的距离并不是特别近。后来导师一把把我叫到了他那里去,问我我不是有linux系统吗?我答是啊,我的是linux mint 17,怎么啦 ?导师就让我用自己的linux电脑测试,连接的时候差一两个软件包,但安装起来非常快,很快就测试通过了。更大的困难还在field测zigbee无线连接,他们需要赶过去。谁知道他们一屋子的人在这里就缺一个linux系统呢?既然导师今天这么铁定地把我拴在这个项目上,电脑也被用上了,少不了晚上就花这个上面了,便与他们去了tower下准备测试。 
\subsection{Fw: Towerlights Tonight!}
\label{sec-16-1-1}
代课老师 (代课老师@uxxxx.edu)
Sent:        Friday, October 17, 2014 1:30 PM
To:        
r (r@xxxx.uxxxx.edu); p (p@xxxx.uxxxx.edu); m (m@xxxx.uxxxx.edu); (me\textasciitilde{}) ((me\textasciitilde{})@xxxx.uxxxx.edu); e (e@xxxx.uxxxx.edu)
Attachments:        
Towerlights.jpg‎ (12 MB‎)

I was going to send the group a note about this.  You should want to see what you
are working on/towards\ldots{}

代课老师

\subsection{From: 工程学院@uxxxx.edu <工程学院@uxxxx.edu> on behalf of Espinoza, xxxx (xxxx.edu) <xxxx.edu>}
\label{sec-16-1-2}
Sent: Friday, October 17, 2014 9:42 AM
To: Mailman - engineering@uxxxx.edu
Subject: [Engineering] FW: Towerlights Tonight!

\subsection{From: 系里小秘 (系里小秘@uxxxx.edu)}
\label{sec-16-1-3}
Sent: Friday, October 17, 2014 9:39 AM
To: xxxx (xxxx.edu)
Subject: Towerlights Tonight!

Please Join us tonight on the Tower lawn for the Towerlights show! Show starts at 9:45pm after the Homecoming fireworks.

系里小秘
Student Coordinator
University of XXXX
Computer Science Department

\section{Towerlights Tonight! 10/17 (2)}
\label{sec-16-2}
来到大楼下,把我们需要用的设置连接好,便开始测试无线网络。到这时七点钟左右,可能大多跟导师作过项目的学生也都到得差不多了。Senior design项目的负责人我们一起作过 一个春天的项目,是个瘦高个的老美,为后文称呼方便就叫他瘦高个吧。 我们测了两个小时似乎还没有进展,9:45就该开始了,这是我第一次被导师捆着参加一个这样的demo,他们可能多多少少也参加过至少一次了,我不明白为什么其实人的承受力会比我好,所有的人里似乎只有我在着急。时间在分分秒秒地过去,我心里在冒汗,不知道是导师估计时间差不多了,还是是某个人的意外,把问题的源头找到了,好像导师还是谁把demo测试用的网络接口接错了?\textasciitilde{}~我还不曾真正参加到这样一个项目里来,因为zigbee的版块还不够熟悉,但也还是觉得这个问题的源头实在太狗血(测试的过程也未免太没有逻辑了吧)。。。。

后来9:45pm demo正式开始后,我享受了那么一两首歌的乐趣,主持人拿起话筒作介绍声音里透露出的那种范儿还是很专业的。看见好多人都来了,EC代课老师也来了,还找我聊天说他常常不太敢这里看demo,因为据说zigbee不稳定,受天气等因素影很大,不经意就容易fail掉,虽然刚刚经历了两三个小时的狗血测试,但听EC老师这么说还是说得自己一愣一愣的。。。两着歌之后就真如EC老师所说的那样了,无线网络连不上了。。。这个时候,实在是太尴尬、无地自容。于是我们再跑进去测,过了差不多十多分钟吧,才又连好,接着进行。也有几个小问题是在所难免的,比如有一个窗口的网络接口坏掉了,所以那个窗口永远是黑的;同一楼层间窗口的网络接口被接返了,就像原本展示字母p的就成了q,还是会有loose package等。

这样一个晚上我过得很是心惊胆颤。去想原因的话,还是那句老话,大树底下好乘凉,很多学生都跟大牛,大牛新招的讲师去作项目了,像导师这样几乎没钱的导师偶尔能有那么一两个学生真正有兴趣肯钻进去一点儿已经是凤毛麟角难能可贵了,这个系里谁又能真正做到攻坚把zigbee的这个loose package的难题攻克下来呢?那时我还没有经历接下来的事情,对导师安排自己作TA还心怀感激,所以亲身经历了这样一个低效挫败的晚上,我与瘦高个和另一个EE的同学说,要不感恩节期间我们把这个弄透了吧(当然后来因为各种矛盾的迸发终没能成)。结束后,导师带瘦高人和那个EE的学生去酒吧喝酒,已经很晚了十点多快十一点儿,我没有任何兴趣,瘦高人说到时他帮我喝一杯。。。

\chapter{又一学期TA季}
\label{sec-17}
\section{TA Appointment for Spring}
\label{sec-17-1}
系里小秘 (系里小秘@uxxxx.edu)
You replied on 11/5/2014 4:12 PM.
Sent:        Wednesday, November 05, 2014 3:27 PM
Good Afternoon,

As you know the fall semester is coming to an end and it is time to start thinking about Spring Semester.

I need to know if you would like to continue your current TA position with the department. It is important that you express our interest quickly, so If I do not hear back from you by 5pm tomorrow November 6, I will assume that you are not interested in continuing your appointment next semester and will award it to another student.

Thank you.

系里小秘
Student Coordinator
University of XXXX
Computer Science Department

\section{RE: TA Appointment for Spring}
\label{sec-17-2}
(me\textasciitilde{}) ((me\textasciitilde{})@xxxx.uxxxx.edu)
Sent:        Wednesday, November 05, 2014 4:12 PM
To:        
系里小秘 (系里小秘@uxxxx.edu)
Good Afternoon, 系里小秘, 

I know that we have lots of students that are interested in the TA position in our department, and I may lose it eventually, but to think about it positively, I do need to express in this important email that, I AM interested in the spring semester's TA position to support my tuition fees and living expense, and my medical bills and potential medical expenses as well (I know I have important health problems and I have not start to work on it yet\textasciitilde{}). 

We all don't know about the results yet, but by sending out this email, please help consider me as a spring semester TA candidate as well. 

thanks,
-(me\textasciitilde{})

\chapter{Senior Design}
\label{sec-18}
\section{Senior Design (1)}
\label{sec-18-1}
\subsection{[Senior Design] capstone class today (for 1st \& 2nd semester teams) - Ag Sci 106 at 3:30}
\label{sec-18-1-1}
代课老师 (代课老师@uxxxx.edu)
Sent:        Tuesday, October 21, 2014 9:44 AM
To:        
代课老师 (代课老师@uxxxx.edu)

From: Beyerlein, Steven (sbeyer@uxxxx.edu)
Sent: Tuesday, October 21, 2014 4:56 AM

Subject: engineering capstone class today (for 1st \& 2nd semester teams) - Ag Sci 106 at 3:30

Capstone Teams, (代课老师 please forward this message to your teams as well)

Both first and second semester capstone teams should come to class today in Ag Sci 106 at 3:30.  Our agenda will be\ldots{}

1- VIEW opportunities this/next semester linked with senior design (George Tanner will be leading this)
2- Snapshot Recap (bring your logbooks with reflective entries about last week's event) - led by Dan Cordon
3- Schedule \& Deliverable Update (for both 1st \& 2nd semester teams) - led by Dan Cordon
=> review of capstone items on course webpage (for both 1st and 2nd semester)
=> Jon Teske will be having wiki consulting hours in GJ 114 next Tuesday if you need them
=> We will be sending out team invitations to specific 15 minute slots for reviewing project wiki pages on Nov 4th
=> Feng Li will be setting up a PCB design workshop for Nov 11th
4- Project Management Tools (for use in Instructor/Team Meetings) - led by m Riley
5- Preparation for and Assessment of Design Review w/clients (to be completed by Nov 21st) - led by Tao Xing
6- Team Member Citizenship Assessment (to be completed by 1st \& 2nd semester teams by Nov 21st) - led by Dan Cordon

Second semester teams are excused after item \#3 which should be \textasciitilde{}30 minutes into the class period.  However, teams are welcome to stay and share insights about later topics with first semester teams.  The course webpage mentions Design Reports for 2nd semester teams.  Make sure that you review the item posted under 'project guides' on the course webpage and discuss this topic in your next instructor/team meeting so that everyone is on the same page with your outline/plan for this important piece of project documentation.

\begin{itemize}
\item 一代课老师
\end{itemize}

\subsection{Fw: capstone wiki page review/consultation tomorrow (Tuesday, Nov 4th) - Ag Sci 106 at your assigned slot}
\label{sec-18-1-2}
代课老师 (代课老师@uxxxx.edu)
Sent:        Monday, November 03, 2014 10:04 AM
To:        
代课老师 (代课老师@uxxxx.edu)
?All:  Please note

Hope to have a room reserved for next week's Design Reviews.  I will email
you when I have details\ldots{}

代课老师

From: 一代课老师 (一代课老师@uxxxx.edu)
Sent: Monday, November 03, 2014 5:24 AM
To: engr-WALT@uxxxx.edu; engr-chillin@uxxxx.edu; engr-visionquest@uxxxx.edu; engr-mbc@uxxxx.edu; engr-toppick@uxxxx.edu; engr-hotrods@uxxxx.edu; engr-liquidgold@uxxxx.edu; engr-gps@uxxxx.edu; engr-taps@uxxxx.edu; engr-threephase@uxxxx.edu; engr-teampowerteam@uxxxx.edu; engr-rabbotix@uxxxx.edu; engr-rainmen@uxxxx.edu; engr-rocket@uxxxx.edu; engr-robosub; engr-teamrehab@uxxxx.edu; engr-theyachtclub; engr-tcoils@uxxxx.edu; engr-turboflakes@uxxxx.edu
Cc: 二代课老师 (二代课老师@uxxxx.edu); 三代课老师 (三代课老师@uxxxx.edu); 四代课老师 (四代课老师@uxxxx.edu); iew; 五代课老师 (五代课老师@uxxxx.edu); 代课老师 (代课老师@uxxxx.edu); 六代课老师 (六代课老师@uxxxx.edu)
Subject: capstone wiki page review/consultation tomorrow (Tuesday, Nov 4th) - Ag Sci 106 at your assigned slot

Capstone Teams (1st and 2nd semester),

We've designated Nov 4th as a time to assemble AT LEAST TWO REPRESENTATIVES FROM EACH TEAM (at your assigned time below) to review, share progress, and provide consultation on your capstone wiki pages.  First semester teams, please post your problem definition, project learning, and team information in preparation for this session.  Second semester teams, please update these first semester items and post information about your final design and its evaluation.  You are encouraged to bring along burning questions about content and wiki implementation.  DON'T MAKE PREPARING FOR THIS SESSION BECOME A BLACKHOLE, bring what you've got.  The session will have a shot/tell/Q\&A format.  We have grouped teams by lead instructor so that they can be part of this dialogue too.  You can use your wiki pages to college key graphics and talking points for your upcoming design reviews.  Each review cycle ran for about 20 minutes so you will have plenty of time to meet outside of class on other action items.  Teams are welcome to send more than two representatives, but this is not required.  Please arrive on time and, if necessary, wait at the back of the room until the group before you is dismissed.  We look forward to reviewing your progress and providing feedback that will help you advance your projects.

TIME SLOTS FOR WIKI PAGE REVIEW
3:30-3:50 teams with Fend or m as lead instructor
3:50-4:10 teams with Tao or Joel as lead instructor
4:10-4:30 teams with 代课老师 as lead instructor (please forward this email to your teams)
4:30-4:50 teams with Dan or Steve as lead instructor

\section{Senior Design (2): 一次会议一场lie}
\label{sec-18-2}
\subsection{Senior Design Meeting}
\label{sec-18-2-1}
e (e@xxxx.uxxxx.edu)
You replied on 11/4/2014 8:10 PM.
Sent:        Tuesday, November 04, 2014 6:10 PM
To:        
(me\textasciitilde{}) ((me\textasciitilde{})@xxxx.uxxxx.edu)
Cc:        
m (m@xxxx.uxxxx.edu); r (r@xxxx.uxxxx.edu); p (p@xxxx.uxxxx.edu)
Hello (me\textasciitilde{}),

Today for senior design we had a short meeting where we cleared up some confusion about the GUI design. The grid which is the main work area will be static. i.e. it will not move and all squares in the grid will always show. The frame, which represents the light shows display area will be a movable object. In Paul's GUI design the grid is the gray squares and the frame is the orange squares. When the project is complete, the user will be able to fill in the grid however they want and they will use the frame to capture sections of the grid as frames in the light show.  

Also, we all need to write a short bio for ourselves to put on our project's wiki page:

\url{http://mindworks.shoutwiki.com/wiki/Tower_Lights_Animator}

If you would like to edit your bio yourself then you should make an account at the wiki and edit the table listed under the  Team Members section. If you would rather me edit it then please reply to this email with the bio you would like to include. 

e
\subsection{RE: Senior Design Meeting 这封上次贴忘记了}
\label{sec-18-2-2}
(me\textasciitilde{}) ((me\textasciitilde{})@xxxx.uxxxx.edu)
Sent:        Tuesday, November 04, 2014 8:10 PM
To:        
e (e@xxxx.uxxxx.edu)
Cc:        
m [m@xxxx.uxxxx.edu]; r [r@xxxx.uxxxx.edu]; p [p@xxxx.uxxxx.edu]
Hello e, 

I am sorry to say that I missed today's meeting without notification. This is my first time missed a meeting. As an "environmental-friendly" team, I hope later on if such things happen, a short one sentence email reminder could help, and hoping that later on whoever is missing or late, we could do the same thing. 

Though you declared it's a short meeting, the statement and attitude about the meeting contents still surprised me a little bit. I am currently working on the GUI Qt Creator Interface as we have assigned about a week ago, and I will meet p to discuss and finish our subteam's GUI design and documentation. Since p has added the GUI image into wiki already, I simply added my little bio-:)

By the way, are we suppose to meet this Thursday for our team member meeting, or it's always consistent on Tuesday afternoon at 2:30 (and I need to stay in CSAC from 2:30-3:30 on Tuesday, but available on Thursday 2:30-3:30, as you already knows)? And later on if we do use the 2:30-3:30 Tuesday one hour, I hope we can do it in CSAC, or we find some other hour so that I can join. 

thanks,
(me\textasciitilde{})

\subsection{Team Meeting 3:30}
\label{sec-18-2-3}
e (e@xxxx.uxxxx.edu)
You replied on 11/6/2014 12:25 AM.
Sent:        Wednesday, November 05, 2014 3:44 PM
To:        
(me\textasciitilde{}) ((me\textasciitilde{})@xxxx.uxxxx.edu); m (m@xxxx.uxxxx.edu); e (e@xxxx.uxxxx.edu); r (r@xxxx.uxxxx.edu); p (p@xxxx.uxxxx.edu)

Hello Tower iLLuminati,

Just a reminder we have a meeting tomorrow at 3:30. This meeting will be in m's office unless it is unavailable. During the meeting we will prepare a plan for the information we want to share during the design review. Though 代课老师 said we should just talk about the "big-picture" features, i think we still try to match our presentation to the grade sheet requirements. r will be our presenter for the design review but we should all agree upon the content since its all of our grades.

If someone is unable to make it to the meeting then please reply to all on this email. 

Thank you,
?e

\subsection{RE: Team Meeting 3:30}
\label{sec-18-2-4}
(me\textasciitilde{}) ((me\textasciitilde{})@xxxx.uxxxx.edu)
Sent:        Thursday, November 06, 2014 12:25 AM
To:        
(me\textasciitilde{}) [(me\textasciitilde{})@xxxx.uxxxx.edu]; m [m@xxxx.uxxxx.edu]; e [e@xxxx.uxxxx.edu]; r [r@xxxx.uxxxx.edu]; p [p@xxxx.uxxxx.edu]
Hello e, 

When I received your email yesterday, once I finished my wiki page bio, I replied to you as soon as possible within 2 hours. I didn't hear back from you until today around 2:30pm after your compiler class, you stopped by the office and hurry to grab your water bottle and we chatted for about 5 minutes. You declared it was just a short meeting lasted around 15 minutes and I didn't feel too bad about it at that time. You refused to use Google calender because you felt r is the one won't check it; And you insisted to send us email reminders because you need to remind r any way. Then you asked me questions about my hearing, and which sounds I have difficulties, I answered overall in general I am OK, but subtle difference like th \emph{f} \emph{tf} I have some difficulty to differentiate and pronounce correctly. 

I can't feel as comfortable any more when I talked to m in CSAC and received your email today. I can't imagine, as a american student, how could you possibly confuse 2.5 hours (from 2:30 to around 5:00pm according to m) to be 15 minutes? Just as I feel surprised yesterday receiving your email, your attitude in that email, you avoid reply to me through email and talked to me instead, your today's email completely ignoring yesterday's matter, and emphasize the "grades" and "reply to all on email" thing, all together does make me feel bad now. It feels like I haven't done anything and I don't deserve a good grade, and maybe I should be ignored or dismissed from this team. 

Let's be frank and face it. You mentioned that during your software engineer class Paul seemed to be the one didn't do anything, I answered that as a sub-team Paul and I are OK, we don't have such a problem. 
Since you lied on 2.5 hours, if you are actually talking about me, at least when you asked me to type the team contract when no other members have any assignments, as the only international in the team who may have difficulties on language part later on, I did it without any word; 
on 10/23 team meeting, you declared to work on wiki page, Paul and I declared for GUI; GUI is due on 11/11, wiki was yesterday, and I showed you from my laptop on 10/31 Friday in CSAC when you entered that I was working on Qt creator GUI interface, I showed you the form I was working on with Josh and other students together in CSAC; But on yesterday's wiki due day, you made such a thing. 
On 10/2 you and I spent 1 hour to review and make necessary contents and format updates on team contract in our office using my laptop and Emacs; on 10/3 you sent out email declared you changed document font size from 10pt to 12pt, and I ended up change it back to 11pt without sending emails; 
As you talked to me having seen how efficient I used Emacs and org-mode for latex, you asked me to be the person responsible for the design documents. In the followed team meeting, I specified that I haven't taken software engineer in regular class, not that confident about document contents, and we as a team agreed that, we all are responsible for the contents, and I will be the person to integrate and organize them to be good latex documents, with review like you and I did  already for team contract, because by that time all you four will be berried with heavy duty compiler homework. 
And I am ok that I will be dragged behind for project implementations when all you four have heavy compiler duties, and we plan to get only the GUI interface implementations ready without linking the button functionality. Assigned to be the secondary Lead Programmer in the contract, with the primary one r will focus on his compiler, I will still be the primary responsible person for GUI interface implementations. 

I may have listed too much already. Above all, I am the only person in CSAC who needs to start 3 mornings from 8:00am, and m was the only person among those I asked who was willing to exchange one hours with me so that on Thursday mornings I don't have to wake up early, and he and I don't have any say sorry issue. You lied or not, p and r could still be the judge. Since when I talked to m today he mentioned "We don't like stay here in CSAC", I think I have 2 main concern/issue here:

\begin{enumerate}
\item On tomorrow's meeting, we need settle down this "Tuesday meeting time and reminder" thing. I don't like working in CSAC while having meeting neither. So we need discuss about this;

\item For e only:
\end{enumerate}
I should have graduated this May already because Dr. Android代课老师 and I have 2 year oral graduation agreement but I was blocked by the department for one more year, together with my medical problems, just so that it will make other students (like international ph.d students) feel better because my cs master's degree without a CS undergraduate processed too well under this 4.5 to 5 years on average bachelor's degree environment. For the passed more than two years, I am mature enough to know clearly my goals, and I worked hard and can bear whatever situation the department have possibly cast on me. I appreciate that the department gives me the opportunity for TA for this fall semester. I expressed my expectation for next semester, but I wouldn't demand it, and it's up to the department to decide who they select for TAs. And to my understanding, you got this semester's one special section TA was only because your husband was willing to accompany you to stay on campus for one more year, and you may not qualify for the TA yet because you haven't even finish compiler yet. And it was you not your husband was because you are the superficial one and can be induced to behave the way they expect. 
Unless you are especially cultivated as a "spy" by the department to dismiss discrimination on international students signal, if that's the case, simply ignore my email, you will have your destination; Otherwise, you need pay attention to your personality. We have every reason to prevent yesterday's thing to happen, but it happened (we skipped one meeting during your compiler exam week; you all didn't appear in CSAC yesterday as regularly did; you all didn't offer email or phone call reminder; and you lied); as the project manager, I replied to your email yesterday, but you didn't reply; And you lied to me today with later on your email mentioned nothing about Tuesday's meeting. 
The only possible reason I can possibly think that you behavior so wired recently (say things on last Friday's meeting, make such an attitude yesterday, and lie to me today) is that you are very aggressive, and worked too hard just for a TA opportunity. I wondered why you can't figure out the questions your two female students have on Lab7 or Assignment 7, and needed your husband to help you out in CSAC the other day. Now I think I have the answer. Keep sending emails out for senior design course and registering Dr. Soule's Video Games and Evolution course doesn't necessarily guarantee an A nor your TA position, but by correcting your personality and stopping being superficial and practice more can help you go a long way. I just had a sorry from you recently, and I don't need that word any more. I state this out sincerely just for your own good. 

I never want to make anything a big deal, but your 2.5 hours lie to be 15 minutes did challenge my bottom line. Since Dr. EC代课老师 treats you especially well, I struggled whole evening ever since I read the email to think if I should state it out, should I or shouldn't I send out this email? If I did become very sensitive and did make it a too much big deal by the environmental force recently, like suddenly on this Tuesday's lab, there were several students trying so hard to make me behave ugly, if that's the case, please ignore the email and forgive me. 

Sincerely, 
(Me\textasciitilde{})

\chapter{You Are Mine\textasciitilde{}~!!}
\label{sec-19}
\section{You Are Mine\textasciitilde{}~!! You\textasciitilde{}}
\label{sec-19-1}
是的,昨天去court的结果 是他们并没有serve我permanent protection order,所以并没有那么严重的guilty。说没有那么严重,是因为他们是按照之前13、1、17的case给我结案的,虽然2013、3、7的就2013、1、17的case结案律师给我的口头有效日期是一年。所以昨天即使我因为2013、1、31 temporatory order上庭时对方并没有出现,而且我没有收到任何相关接下来的follow,我还是被罚了\$150元,被当庭serve了permanent order,并被设置了两年(有效期到2017、3、21)的trial,这两年期间我将不被允许有crime,否刚这个case会被自动activate为有效。两这两年期间有crime,昨天的律师告诉我我得请律师帮我打官司。昨天的女律师也帮解释了crime,超速不算,但drive without valid driver's licence算。而心惊肉跳、被吓得魂飞魄散一场,虽然结果也很冤(前律师2013、3、7结案的那个,告诉我的有效期是一年,我过了21个月才去找的表哥,为什么要我犯案?),这个结果已经算是好的了。我并不认为任何留美学业快要结束的留学生都能享受到这样的待遇,感谢自己的经历和故事,也提醒过我类似经历的人一定要注意法律文件的有效日期。
\section{You Are Mine\textasciitilde{}~!! 你值得等待}
\label{sec-19-2}
之前看见过买买提上的名媛youaremine(你是我的)写过一系列贴近生活的原创,比如<老张留学记>等等。受她ID的启发,想起了表哥,心中有所顿悟,表哥,you are mine\textasciitilde{}~!!

2009年8月我们首次相遇,从2010年12月再相遇(被网友戏说相亲,原因我想找的人在自家后院。。。),四年过去了。这四年里,发生了很多事,我也是走得深一脚浅一脚,但自始自终,我都把自己的心弄丢了,确切地说,丢到了你这里。

我是什么时候把自己的心弄丢的?答案是非常清楚的,2010年12月的那场告别,那场感动了自己、让我感觉被表哥你深深宠爱的告别。当你遇到一个人,在她面前,你可以最本能地作最快乐的自己,那份本能、那巨大的快乐就像从骨子里开出的花,这种感觉得如此强烈,你一定知道,因为你遇见了对的人;而我,就在那场告别里遇见了对的你\textasciitilde{}~

如果人世间每一份真心相爱的爱情都能不参杂外在财富物质的考量、不参杂世俗、伦理、道德的束缚,那表哥我们现在是不是已经在一起,人世间是否早已经变成了最美好的明天?好久以来(正如星座上所说,26岁以前我都过着随意的生活,从27岁开始)我一直都在寻找一份真心相爱的爱情,特殊的经历把自己练就成一个走心的人,但正因为走心,正因为从那场告别里体会到了无比的愉悦,我想逃跑,我怕被你的家人捆在永无天日的黑暗里。。。

我努力地逃跑,我去见自己的朋友,同他们打球吃饭,作出一切背离这份感情的事情,但关键时刻,对于没有任何深刻体验的与任何人的往来,我都不secure,行百里者半于五十;原来我的心还是深深地留在了你这里。

我知道你不能说,一说便是错。。。但我都懂的,都能体会得到。你想我离开,希望我去寻找自己的幸福,可我想要的幸福只有你能给。即便我是个冷血动物,我也还是有着满腔热情,即便我没有几次像样的恋爱经历,我也好歹多多少少有过几次感情经历,大学同学、国内硕士时导师等等,我确信那个对的人,表哥,是你,我已经遇到了。这一年,我31岁。

去年12、27日表哥我再次见到你,一两年不曾见面,岁月在你我身上都刻下了深深烙印,以至于再次见到了你,我有些心碎。遥记得2009年时舅舅给我讲过一个男女朋友七年的故事。那时与前男友刚处上,给我讲这个故事之前,舅舅认为那男生极有可能是骗我的, 我反驳舅舅说人家刚处上,热情正高,你要人家分手,多痛苦呀。。。真正与那人相处发现不对之后,与前男友分手要多快有多快,才不要去想舅舅那什么七年的故事,觉得那些都是猥琐男专门用来骗女生的。五年过去,那天见了表哥,看着表哥清瘦的身形和灰白头发,我终于开始明白这份等待,不仅仅只是一份爱情,更多的是一对有情人几年的青春年华,更多的是一份责任。意识到这一点儿,我突然很开心,原来,岁月悄无声息中,时间也深深磨砺着我自己。这份感情、我这一路的顽强反抗也把自己从一个涉世不深对感情理解很肤浅的人变成了一个有责任心、有担当的人。

记得那天我们从你地下室的办公室往外走,我想牵起你手时,刚好就遇到你等在那里的手;及至走道上遇见其它的人,我竟然还是会觉得有点儿不好意思,松开的表哥你的手。当然是给表哥牵着手走的好,我又把自己的手交给了表哥。接下来就这样我们一直手牵着手表哥你把我送出去。表哥,请原谅那期间我些许的叛逆与对尘世的顾及,这次把自己的手交给表哥,我就绝不再放手。

我很感激他们没有在2013年3月的时候就告诉我这个permanent order,因为若是那样,那时四年的等待我不知道自己会不会觉得漫长,会不会在期间遇到变故的时候没有足够的勇气能够等待下去;而现在,阴差阳错,刚刚好,对来我说,一切都是确信的,两年的等待已经过去,只剩下两年的时间,而且我有着坚定信念和责任,冬天已经到来,春天还会远吗?

\chapter{cs120 TA}
\label{sec-20}
\section{cs120 TA (1)}
\label{sec-20-1}
正如前面有所提及,这个秋季学期的120注册显得稍微多了 些。另外本科生E加入作为一个特殊小班的TA多多少少有些怪异。因为E的班上只有少有的几个学生,而且系里甚至没有时间段和适合的教室来开她这个小班,最终是把他们几个人调到了外系的一个教室。而且我是学生已经知道,系里的老师们当然应该开学十天后会有很多学生drop掉不适合他们的课。根据往年的经验,他们应该能够很清楚地判断一个学期入多少本科新生属于正常。总之,现在看来,E的小班在现在看来怎么看怎么怪异,但当时的自己还是麻木的。

TA课的印度人不知道为什么开第二周上第一个实验课他故意逃课,原本我与他说清楚的事情(我带SEC 4和6他带5)他又把导师给绕进去,与导师找借口说他没有准备那天不能带课。于是周二我带了4 和5 两个班。后来偏巧那周四我生理期,身体发冷很不舒服说话也吃力。后来他也还是主动帮我带了同四的SEC 6那次课。我一直没有去想过为什么他会第一周不愿意带课,后来经历了501一系列seminar之后,凡是作research的我都避开了,而与工作相关与自己工作意愿相吻合的我都参加了。后来他加入到了fault tolerant代课老师的一个研究项目中来作硕士论文,系里主动帮他cover掉一些短板,我才明天作为初来乍到的印度学生,他不会用EMACS,不会用VIM,只用nano,好像美国学生们不大满意,也许他只是想避开第一周代课学生们考验评价代课TA的关键时期吧。

我对E的了解 一直停留在CS121她与她那时男朋友坐我前排的记忆上,感觉就是一长得漂亮的普普通通的美国本科生。有时候在CSAC学生问问题,涉及到C++ 类pointer的问题,是第6次作业还是第7次作业,她实验课上的两个女生问的,不知道是不是那天她状态不够好,她反复问过她能想到的所有问题帮她理思路后, 她还是没有答案,就出去找她老公来帮她答了。然后我就意识到绝大部分时候CSAC她值班的时间(她有三个小时还是两个小时不记得了),她老公都是陪着她的。她老公男生的话一般情况下还是应该学得还可以的吧,虽然CS121第二次其中考试他俩考得不理想,期末考试最难那个关于binary search tree的题我写得非常顺,早早地交卷了,他们绝大部分人也都还在教室里耗着。一次CSAC学生问我一个关于计算器要测乘法,为什么terminal里输入*后结果不是他想要的,偏巧E的老公也在CSAC并且正好坐在那问问题学生的左边,就像我理解为什么M被留成了under第五年,我理解他多留下来一个学期是因为想与E同时毕业,尽管那个简单题目我俩同时想到了答案,那会儿我在CSAC里值班是我的责任来解释,但我把解释的荣兴和机会谦虚地留给了E的老公。

\section{cs120 TA (2): 老大}
\label{sec-20-2}
TA课最开始我并不知道如何放松自己后来慢慢地就真习惯了,也会问他们些如何优化代码的简单小问题来带动他们思考。如果说这四个TA谁讲得好,当然是从微软回来的EC代课老师三个SEC的那 个TA了,起个方便记忆的名字叫他"老大"吧。老大和E两个土生土长的美国人当然会在周五下午的例会里讲话稍多些,E是纯粹找话调节气氛的那种,老大才是真正工作过知道自己需要什么回校园里来补的这种,所以是我们四个人最有思路、表达最能为大家所接受、独步江湖的一个,相信他在学生中也同样会有威慑力。

老大是有威慑力的,老大偶尔也会是嚣张飞扬跋扈的。偶尔他有一些的确比较好的想法后,他是会在两代课老师面前坐桌子、拍桌子砸椅子的,我知道curses window出来后他就拍过好几次。其实老大最初也选了fault tolerant的课的,但是他上课从来都带自己的电脑不知道在忙什么。这门课的作业往往简单。但有一次的作业老大大概是因为上课没有真正听讲,作不出来,结果课前他与我在CSAC讨论后才做再交上去。还是一次是需要用一个软件来写程序的,但这个软件程序只在几周前代课老师有效时间开放过link(好像是24小时吧)。所以那天他没能交作业。代课老师给面子让作业推迟了。但系里大概是绝不允许他这样的老大要与我讨论问题后才能做得出来的状况发生吧。所以后来,不知道是真正出于系里的压力还是他自己自愿,他把这们课drop掉了。

E对自己的工作也有热情,但她的热情常因为理解不透或使用不当而使错了方向。比较讲class,她在例会对大家讲她甚至为她班上的学生举了一个inheritance的例子,醉倒了我们一大片人,所以她常常惹得大家又怜又笑的;有一次实验6还是几例会上我说我班上今天第一次有人问我segmentation fault是怎么回事了。并且开始解释在什么情况下发生这种状况(是因为他分配数组大小的时候他想用入输入大小),我是怎么解释的。等我解释完,还没有上过编译课的E脸又红了。我知道自己又拥了马蜂窝,这种情况下,代课老师不主动出面安慰一下E老大也会出面的。果然没多久老大就开始说不提倡从标准输入读入数据的作法了。

实验七还是八当我讲得比较顺的时候,我就想向他们学生极力推广一下emacs,就像ipad改善了人类的生活,emacs极大地方便丰富了我这个单调码工的生活。于是某天我就给他们demo了一下那几个最常用的程序包的方便之处,demo了一下.org文档自动生成latex和pdf的过程,希望他们学习的兴趣能再高点儿。但我万万没有想到后来这个举动使得自己招到了围杀和排挤,以至于下一个SEC我就不再demo,介绍几个程序包了事。 但事情并不会如此简单结束,作为老大,作为系里的镇宅利器,EC代课老师与老大合力(EC代课老师按排实验和作业内容)推出了curses library的一个wrapper libray,在terminal里用起来比较方便。我不明白为什么这个curses window为什么会受到大家那么极致的推崇,甚至工程学院一度发生关于patent考核的邮件,让我觉得这个小学校,或老师严格点为儿说小系对本土学生的地方保护主义有点儿过了头。在E的坚持下这个curses window的子弹在这学期的本科生中多飞了一个星期,同样的作业和实验延了一个周。而在这样一个周五,E甚至开始公然点名鄙视我认为我讲课学生听不懂。E这么说我虽然当时我脸也热辣辣的,但我一定大度,不往心里去,我还想看热闹呢。后来E给我写了封邮件,感觉那时大家系里对我还处在调和磨和阶段。

当我的不以为然遭到大家鄙视的时候,反抗是最本能的形为。于是从网上搜curses library里有哪些处理颜色的函数,在多飞一周的SEC里为大家demo的了一下实验、作业作颜色处理后可能得到的效果,并且本能地普通了一个这个curses library 的本质,在这个神秘面纱下是怎样的过程处理等等。

再后来又是一学期TA季(来年春天)。有次周五例会后他向EC代课老师抱怨说下学期如果EC代课老师还带CS120,能不能不布置那么多作业,或者是分配一个专业的人改作业,这样讲课的只管讲课,改作业的只管改作业。并且他对老师承认所因为要改作业,他把那们fault tolerance课都drop了。其实他们大多数人还是没有改实验一之后的实验和作业的,因为我导师说系里的linux server出问题了,暂时还没想到好的办法把学生们的作业拷出来分班分SEC分实验,但我要求他们交纸版的,并且改了前面至少一半实验的hard copy,但对自己的工作我绝不报怨。在这样的事实面前,EC代课老师说也许可以考虑少带一个实验,这当然不是老大想要的答案,他也就不再说话了。

\chapter{Senior Design现状}
\label{sec-21}
\section{Senior Design现状(1)}
\label{sec-21-1}
这门课的总体感觉就是慢慢慢,慢得不能再慢。虽然我很理解组里其它四个人都在上编译课,这是传说中系里最难的话,但我极尽所能地去理解他们,这个项目的慢还不是一个慢字可以形容的。 

说说学期大半学期过去了几个人干的几件事儿吧。首先,在嬉笑怒骂寻找各种high点时,我被team manager E,这个"team manager"的头衔是在那嬉笑怒骂中已经由几个男生定了下来,被她点名安排type了这个team contract. 这个合同里,github repository被安排为P 和R负责,lead programmer 是R 和我,project documentation是所有成员,Project Portfolio是M和E。manager是E。M负责team minutes.

被E这么一阵嬉笑怒骂中点名还是很意外的,但大家一定知道的,我绝无二话地把它给type好了。我们的项目在一次小的与系里硕士生代课老师的会议上被代课老师要求为用Qt creator和C++ 编程。后来的一次CSAC组会上,M帮把Qt creator 的安装弄好了,那天P说他花了一百美元弄了个MAC WIN 双系统,"just for you guys"; 而我买男闺密的二手电脑他之前更换的SSD只有128G,根本就没有windows系统,所以用linux mint 17安装linux版本硬上,倒也基本上没有任何问题。我是组里唯一一个在linux系统下安装软件工作的人,我的安装倒是没有任何问题,跑M的基本test,就是一个form界件里有个File里嵌一个Exit的命令,都运行得好好的。当然那时候的自己也很傻,我喜欢Emacs,极少用IDE类的界面,连学习第一年的一的那个夏天实习的python都是在emacs下运行的,所以到这时我也是半傻不傻地不知道如何编译运行,M帮指挥了一下我该点哪些键,算是顺利。但是P说他的双系统就不知道是什么原因,怎么也弄不好。结果那天CSAC里一个小时的会议,M主要帮P把他的双系 统给弄好了。

而这期间,因为我早早地弄好了自己的linux下,我便加入 到了E与R的聊天队伍。E说她家里电脑的VS C++都快run 不动了,也不知道能不能装Qt Creator的module package,于是我们三个忙忙碌碌在M的帮助下弄好自己的工作环境的时候唯有E一动不动连电脑都没有打开地在同R聊天。R倒是team好队友(至少该算是E的好队友吧),一马当先地说他会帮E看看到底是怎么回事,该如何安装的。这样这次会议下来,三个人的环境好了,E和R他们打算自行处理。E在白板上画了一个工作任务deadline的流程图,说我们只要在这样的截止日期前把这几件事做好就可以了,所以我们进度还是很不错的。E接下来安排R得负责读上一年从ACM来的C\#代码版本的代码。M问我们三个四个接下来的任务是什么,E说只要能从Qt创建一个button,能从那个button退出来就可以了,没有再接下来的任务了。M接着说"I already did that."但是E不再说话,被问及接下来与代课老师的会议需要准备什么,E也说不需要准备什么,就是个普通的会议。

后来接下来一次与代课老师的会议E在台上大致讲了一下几个载止日期前的任务。并没有cover多少有实质性的内容,但E的态度却显得非常正式。之前第一次会议在一个小会议室长条形圆桌上进行的(没有任何人上讲台),这次是在一间教室里,一时间空矿的教室里气氛显得有些异样。E反问R他没有什么要讲的吗?R说没有,E再问你的C#代码review没什么要讲的吗?见状,R就上台去讲那个版块了。或许对刚刚异样气氛的反馈,台下我们其余三个大家都积极地参与进来,提问R一些C#处理的某些必要细节,提出一些他可能没有读懂或是读漏掉的地方,讨论我们目前版本与这个可能有的不同等。大概到这时这种情况下这会议就远不能算R一个人上台讲的功劳了。会议结束后,大家倒也都开开心心的。
\section{Senior Design现状(2)}
\label{sec-21-2}
记得M帮我们configure好Qt creator upload到github上那次组会P和我分别把我们的Windows和Linux系统弄好,虽说自己半傻不傻,人家把test case写好了,我连怎么compile都不会,需要在别人的指点和帮助下才能运行程序,多多少少,我的内心还是受了些冲击的,觉得小自己13岁的M把这一切都作好了,我自己竟然连点儿自已该早早地准备这些的意识都没有,所以心里也还是难免暗自敦促自己要努力赶上进度的。

接下来记忆里的事就是第一次snapshot day了。记得snapshot day前一周的周五,大概我中午还是回家稍微休息了一下的,没有睡意状态还不错,于是下午CS120 TA例会前我就问E有没有我们项目前一年版本的GUI界面我想自已试试看,E说她没有,我很想要,她拿我没办法,只好建议说,要不你问下他们吧。于是五点CSAC值结束后五点左右我就给TEAM发了邮件要GUI图片,但是他们没有回。我学生办公室有个租的流浪者的笔记本是有WINDOWS的,只是习惯了LINUX后极少再用它了。于是我爬回WINDOWS去就自己找到了张图片。Qt C++编程,与python GUI相比,似乎还是有那么些不同啊,我就在那里试呀试的,试了两个小时,也不过是用代码编了菜单栏的几个简单命令而已(这些个简单命令用FORM实现会更快,只是我更钟情于代码实现,不喜欢用鼠标点来点去麻烦)。后来又试了试中间的layout,那时还是没概念,不知道中间的那些个小东西到底该怎么排列,QT kept telling me layers on layers blah blah,八点钟左右做得也有点儿烦了,便给他们发邮件,贴了自己目前进展的snapshot图片,征求team里的意见,看大家在接下来周二snapshot day之前有什么事情要做,要不,我们这个周末一起做个真正的GUI出来也好啊。我的热情还是瞒高的,但是热情只是属于自己的,我发出去的邮件他们都没有人回。

接下来周二是第一次snapshot day,周一在CSAC里我与M有一个半小时一起值班时间我就问过M这件事,不记得M是怎么回答的我不记得了,意思大概是说等manager E安排吧。后来周二上课前P总是show 他的arm说一切都准备好,just be there.后来知道在周一他们的编译课结束之后E就只叫了P,两人一起打印了几张大字号的纲要,P用图像工具生成了一个根据上一年的GUI版本和上次与代课老师的会议讨论的结果生成的图片版的GUI界面,就算作完了。第一次snapshot day,代课老师甚至把我们这个项目弄忘记了,E说她直接找了个角落把我们的几张纸的poster板放在一个角落里。等我看了我们team准备的这个poster,难免心生失望,我心想难怪E要找个角落躲起来。。。但E却带领他们表现得信心满满的样子总结说,只是因为老师们把我们忘了,我们被迫坐落在一个角落里,我们的poster才会显得无人问津的。。。。那天他们照了张team里五个人的合照,我在边中笑得奇丑无比,其它人都还正常吧,凑在E的身边看了看相机,禁不住感慨,"too fat…"
\section{Senior Design现状(3)}
\label{sec-21-3}
写到这里大家也看出来了,我们这个项目其实都是在赶这门课一个又一个的deadline而已。第一次snapshot day之后,我内心有所不满却不曾表露;到11、11的design review,中间的过程我基本就没什么印象的。只记得期中考试他们四人得考compiler那周大家都没心来开什么组会,所以那周没有会议;为准备11、11的design review,小组的分工情况如下:M负责state graph, E负责wiki page, P和我负责clickable GUI,R和E负责class diagram。Wiki page的截止日期好像还比11、11要早,大概是11、4号吧。

M是早早地完成了state diagram,根据他们senior design的项目经验,state diagram, class diagram这些都是必须的,而且好像他们那门课上都有过系统的训练。虽然M早早地完成了他的部分,但一直没有人没有任何在这个点上的review;wiki page的deadline较早,尽管E负责,E周一还打电话给M(他们几个年轻人之间好像相互留了电话号码的,我不知道忙什么去了没留,我与他们主要还是邮件沟通)试图寻求team帮助,不过可能最后电话里她还是自己决定自己完成吧。后来知道三张图占了绝大部分(M的state diagram,P的第一次snapshot day的图片版GUI,和后来11号左右完成补上的class diagram),另外开头有三段极短的话来描述项目(后来才又进行了修改)。等E的wiki page完成后,他们开会跳过了我。平时是怎么开会的呢,周二课前我是CSAC on duty,所以在CSAC进行,周四课前一小时我们在M的公办室里进行,那里桌子多椅子多白板多,足够我们在相对安静的环境下坐下讨论问题。但是这天周二他们直接在M的办公室里开会而没有叫上我。

后来知道,就在这个他们开会跳过我的会议上,P和M两人一起设计了几个QT creator 的header file,关于项目的设计,我是多么地想要参与,因为过往的经历,我没人任何像这样大项目的设计经验,而这样的机会也被他们直接跳过了。。。。会议后E给我发了封邮件,声称这只是一个short meeting,但这封邮件惊着自己了。这是我与E接触两三年来(这是我在校的第五个学期,与E一起上课的第三个学期,春天与她和她老公及其它一些同学一起上过我导师的高级操作系统课)第一次体会到E对自己有这么冷,读邮件的自己牙齿打颤,心惊肉跳。。。。我想不明白E的态度为什么会是这样,只能陪尽小心,在邮件里小心翼翼地试图缓和气氛。

\section{Senior Design现状(4):话中带话,万事具备,只欠东风}
\label{sec-21-4}
我周二晚上回给E的邮件她没有回,直到周三下午2:30左右E从她的编译课下课回来step by our office。虽然我们share同一个办公室,但不知道什么原因E很少出现在办公室里,只是偶尔会在那里用微波炉吃些午饭点心。所以E来后,我们就讨论到一起邮件里提起的问题。

E先说那只是一个很短的会议,about 15 minutes.见E说这个会议有这么短,我也就不往心里去。只想着以后不要再发生这样的事情才好。E说以后每次会议她都会写邮件提醒大家。我只是因为被skip掉了一个重要项目设计的会议,心有余悸,但E要每次会议都提醒的话,我本能地觉得还是太过了,便对E说"you don't have to do it this way. "只要我们大家都知道就可以了,但E坚持她的做法给出的理由是我从来不曾接过的信息,E说组里R是那个不读邮件不写邮件、不喜欢读不喜欢写的人,说R更习惯倾向于电话或是别人口头提醒。我心想我是组里的国际学生,我因为语言的障碍才是那个不喜欢读paper不喜欢读英文教材,不喜欢写paper不喜欢写article的人,难道美国土生土长的R也这样?便对E说"I think in that case you should call and remind your friend instead of send emails to all of us all the time. But if you really want to do it this way, you are the manager."于是这个话题不在话下。

E接着问我,根据她往年与R和P上software engineering的经验,P好像是个在什么组合与任何人合作状态下都基本不干什么事儿的人。E接着我问我我与P合作GUI有没有这种状况?到这时,我已经禁不住感慨这天E step by her own office与我的short chat真是奇怪。First snapshot day明明是E只拉了P一个人与她准备的,是P用图像处理工具生成了一个图片版的GUI,E也就只打印了几张大字号的纲要,缺少了P的那张图片第一次snapshot day根本就没法看,什么叫P什么事情都不做,倒是我自己,作为team manager,E也就只给了我type一个team contract的机会,心想E你这是在说谁呢?便直接了当一口否绝E说"No, P and I are perfectly fine, Idon't have such a problem working with P.

M E没什么好报怨的,别人什么事情都早早地安时完成了,E也没有带领大家review; M甚至还帮助R和E,在跳过我的周二的会议上,与R一起设计了好几个QT creator的header文件。只是在这次M帮助他们以后,没有任何人再动过这个设计,到11、11deadline 时E更是直接拿他俩那天的设计的header缺胳膊断腿地(不该加的也加着,缺的依旧缺着)连了几个class diagram. 至此,对team member的话题到此结束。

接着E问了她Im sorry邮件我回复她的问题,我把自己耳朵听力有些问题的原因再亲口对她解释了一遍。她特意地问了我到底是什么样的发音我分辨和自己讲起来会有问难,于是我就在我们公办室的黑板上画了几个音标,告诉她这几个音我会有些困难。心里也难免捉磨,E这是要以她的aggressive来激发我去拼去争春季的TA呢,还是在制造舆论彻底否绝我春天的TA呢?

后来我们再无话可说,E也接到她老公的电话便离开了。及至那天(周三)下午3:30 到了CSAC问了M关于那天会议的情况,M说那天他们是呆到接近五点钟的。到这一刻,我终于是风中凌乱了。。。。晚上回去给E写了封长长的邮件,捉磨着要不要发,最终过了午夜,还是把邮件发了出去。

\chapter{Senior Design (2): 一次会议一场lie (overlap, consider if need it or not)}
\label{sec-22}
\section{Re: Team Meeting 3:30}
\label{sec-22-1}
e (e@xxxx.uxxxx.edu)
You replied on 11/6/2014 7:25 AM.
Sent:        Thursday, November 06, 2014 2:44 AM
To:        
(me\textasciitilde{}) ((me\textasciitilde{})@xxxx.uxxxx.edu); m (m@xxxx.uxxxx.edu); r (r@xxxx.uxxxx.edu); p (p@xxxx.uxxxx.edu)

Well everybody, we got stuff to talk about tomorrow. 

For the record:

\begin{enumerate}
\item I did say that r wouldn't check a posted schedule if we had one. My experience with r has been that he prefers verbal interaction over emails and documents. If we are going to devote time and resources to a making a calendar then we need to be sure that it will get used. If it is not used then it just becomes more busy work that nobody has time for.
\item I did say that p did not do very much work for our Software Engineering project and p will probably agree. I was telling (Me\textasciitilde{}) this because I want everyone to contribute to this project. We are all very busy and I don't want anyone to end up having to do more than their fair share.
\item I did say the meeting was about 15 minutes. This was an exaggeration because the group was mostly unproductive for the first hour. Pretty much the only members who got anything done was m and r and they were the ones who worked late. p and I were there till about 3:50 (im not sure the exact time we left). 15 minutes is literally a lie that was meant to reflect our productivity. From now on i will be more specific so that there isn't any confusion.
\item I do have an aggressive and superficial personality. r refers to me as "cold-hearted" and this does not bother me. If there are specific things I do that are aggressive and cause disruption, then I want to hear about this because I'm probably not aware that I'm doing it.
\end{enumerate}

What is important is that we resolve problems where the group is unable to work together in a respectful manner. I think we have had some misunderstandings, miscommunications, and a lack of organization. This has resulted in (Me\textasciitilde{}) feeling disrespected and we should all do what is reasonable to resolve this. What is considered reasonable can be discussed tomorrow. If anyone else feels that another member of the group has been disrespectful then tomorrow is the day to talk about it. 

e

\section{RE: Team Meeting 3:30}
\label{sec-22-2}
(me\textasciitilde{}) ((me\textasciitilde{})@xxxx.uxxxx.edu)
Sent:        Thursday, November 06, 2014 7:25 AM
To:        
e (e@xxxx.uxxxx.edu); m (m@xxxx.uxxxx.edu); r (r@xxxx.uxxxx.edu); p (p@xxxx.uxxxx.edu)

Hello e,

I had bad personal experience towards lies, and I have absolutely zero tolerance towards lies ever since. And I hate to be in the situation being suspected that I lied when actually I didn't. And that was the reason your lie triggered all my sensitivity and bad feeling towards you yesterday. And from your reply, at least, you and I didn't run into the corner that without email record, each of us deny on everything, which is good news for us. At least as this point, I can calm down.

It was your fault on Tuesday as the project manager didn't notify the meeting and didn't remind me about the meeting through email, and the team member didn't offer any kind reminder neither. And your email and coldness on Tuesday did surprise me, and make me feel disrespected, and you seemed not realizing this fact yet, but we can discuss about it all together today. 

e, please realize that as the only undergraduate TA in the CS department, you have deep and wide influence on students here. In the student's eyes, as the only undergraduate TA, you are the representative and influence of the department, please behave properly so that the department's reputation won't get damaged by you. 

I have no experience lied by any other classmates here in US so far yet rather than you. I never know that people, students, project managers can lie this way yet until yesterday. And that's the reason yesterday I couldn't calm down at all. Let's prepare and expect an effective and productive meeting today then to solve the problems. 

(Me\textasciitilde{})

\section{Re: Team Meeting 3:30}
\label{sec-22-3}
e (e@xxxx.uxxxx.edu)
Sent:        Thursday, November 06, 2014 10:15 AM
To:        
(me\textasciitilde{}) ((me\textasciitilde{})@xxxx.uxxxx.edu); m (m@xxxx.uxxxx.edu); r (r@xxxx.uxxxx.edu); p (p@xxxx.uxxxx.edu)
(Me\textasciitilde{}),

Stop personally attacking me through email. Our senior design group has nothing to do with my TA position and your opinion of my personality or how it reflects on the CS department. If you want to discuss these things then you need to email me and not the whole group. I mentioned that if there are actions that I do that disrupt our ability to work together then tell me what those are. I can change how I interact with the group to enable a better working environment. Obviously, lying is something you have zero tolerance for and I will do my best to ensure that my communications are understood not thought of as lies. 

I did mention that you feel disrespected in my past email and I understand why. You are upset because you feel lied to, but the problem isn't that I lied to be malicious and make you feel left out. It was just a poor word choice. I said "the meeting was about 15 minutes" what I should have said and what I really meant was "we worked for about 15 minutes". r and m worked later than everyone else and 2.5 hours only applies to them not the whole meeting. 

Again, please restrict your communication to the group to topics that relate to the project not to me personally.

e

\section{Re: Team Meeting 3:30 他们应该是早上见了老师,已然讨论过了}
\label{sec-22-4}
p (p@xxxx.uxxxx.edu)
You replied on 11/6/2014 11:00 AM.
Sent:        Thursday, November 06, 2014 10:22 AM
To:        
e (e@xxxx.uxxxx.edu); (me\textasciitilde{}) ((me\textasciitilde{})@xxxx.uxxxx.edu); m (m@xxxx.uxxxx.edu); r (r@xxxx.uxxxx.edu)
(Me\textasciitilde{}) and e,

Relax, we are a team and we are all on track to meet our goals that have been outlined. There has obviously been some miscommunication among some members, which is understandable considering we have a larger team whose members have dramatically different lives and schedules. 

On a side note: Since this topic was mentioned below\ldots{} In terms of our Software Engineering class a few semesters back, I do not agree that I did "very little work". However, I will agree that when comparing my contribution to that particular project during that semester, there were many other students who invested a much larger portion of their time into the project. So in comparison, I did less work than some others, but in no way did I do "very little work". I did do my part in that project, and I completed everything that I was tasked with for that project,  just as I am doing my part in this project. I am sure there is some point that e was trying to make in bringing that up, and that is fine. I am not offended by it, I just wanted to clarify that portion since this has turned into a public document. 

e and (Me\textasciitilde{}), you are both valuable members to the team. I really think that this has been a big misunderstanding. I am sure we can work out our differences and continue to make progress on this project. 

p.

\section{RE: Team Meeting 3:30 考虑一下,需要不需要这个?}
\label{sec-22-5}
(me\textasciitilde{}) ((me\textasciitilde{})@xxxx.uxxxx.edu)
Sent:        Thursday, November 06, 2014 11:00 AM
To:        
p (p@xxxx.uxxxx.edu); e (e@xxxx.uxxxx.edu); m (m@xxxx.uxxxx.edu); r (r@xxxx.uxxxx.edu)

Hello Tower iLLuminati, 

I am sorry for what I have brought to confused us during last night's email and this mornings' one concerning about e's TA part. I apologize for that. I never mean to attack e at all, if she feels I was attacking, it is not my intention. 

I can be open-minded to consider this time e's lie to be hers "trying to be considerate" good manner. But at the same time, I did still feel very bad about Tuesday's meeting and team environment, and I consider it was mainly e's fault -- who is the team manager and who supposed to be able to handle this case well. 

I don't know if all of you guys consider this about the same way, but just as I mentioned to some friend earlier, in our culture, we believe there is a perfect god's mind live in every alive human beings, and I believe what you declared. So what happened during the 2.5 hours, I don't even try to think and be judgmental at all. But at the same time, please also try to understand my consideration for e. I was kind enough to remind her about her influence because Tuesday's meeting exposed her weakness, and I want her to perform good and our department to be good as well if anything matters and affected. 

At this point, I can't consider e to be a good team leader, but try not to make this too much a big deal, and also for my last year's sake, I can tolerant it. I will try my best to cooperate with all of us together, and I deserve all other team members to treat us in appropriate manner as well. 

I thought my mornings' email should be the last one, didn't realize there came two more. I don't think I will reply any more before meeting. Let's hope we make a good meeting today.

thanks,

(me\textasciitilde{})

\chapter{几封贴丢的邮件}
\label{sec-23}
\section{Im sorry}
\label{sec-23-1}
E (E@xxxx.uxxxx.edu)
You replied on 11/2/2014 12:20 PM.
Sent:        Saturday, November 01, 2014 6:09 PM
To:        
(me\textasciitilde{}) ((me\textasciitilde{})@xxxx.uxxxx.edu)

Hello (me\textasciitilde{}),

I'm very sorry that yesterday I mentioned that one of your students was very upset about your accent during the meeting. I know that you are capable of teaching and I feel bad that I ever said anything. Not being able to be understood is very frustrating and I know you are trying very hard. I wish I could take back what I said during the meeting because I know it made you uncomfortable and that was not my intention. I was trying to make a point that the assignment might have been too long for one week and students are getting very stressed out about it.  I hope that you can forgive me and we can continue to work together without any bad feelings.

Sincerly,

E

\section{RE: Im sorry}
\label{sec-23-2}
(me\textasciitilde{}) ((me\textasciitilde{})@xxxx.uxxxx.edu)
Sent:        Sunday, November 02, 2014 12:20 PM
To:        
E (E@xxxx.uxxxx.edu)

Hello E, 

I went to friend's invitation for dinner last night and came back late and tired. Sorry for the late response. 

I think you thought too much and made it such a big deal at it. To be honest, I barely remember anything you said during the meeting on Friday. And I always believe that as far as a person doesn't lie, the person is the judge for himself to decide if he needs to say sorry or not. And if you don't feel you did anything bad or it was not your attention, you don't have to be forced to say sorry to me by the environment force or whatever reason. In western culture, I know many people believe in God; In our country, we don't have that many people believe in God, but our culture does believe that a person is born to have all kinds of good inborn personality/nature deep inside the mind. 

And if I stand even further away, sincerely,  you could potentially hurt your partner, but how could you hurt me? So no worries, please. 

It is the fact the I have some accent, and unfortunately that's because I have very bad hearing. When I was a kid, I cried too much too frequently that my ears were sick for years. It was until last fall Dr. (fault tolerantw代课老师,data communication课代课老师) Data Communication course that I realize I had bad hearing. On average my classmates can hear up to 15/16 kHz frequency sounds, but I can hear only up to 7 or 8 kHz. And that makes it even difficult to pronounce correct with identifying pronouncing mouse shapes. It is a fact for me, and that's the reason I try to be considerate and always take my time to prepare slides before lab for the students in case they have difficulty, and at least they could still take a look and read the slides.


We have take cs121 data structure course by Dr. 系里硕士生代课老师 back in Fall 2012 together with your partner, cs543 advance operating system this spring by Dr. 我导师, and cs480 senior design for this fall semester. As an American native girl, with all the language benefits you have, you always brought laughter to the classroom. And just as I said to you the first day when I realize that we are going to share the same office this semester, I like you as a friend and please do not think too much and made it too serious. 

Sincerely, 

(me\textasciitilde{})  

\section{RE: Senior Design Meeting 回复E最冻人邮件}
\label{sec-23-3}
(me\textasciitilde{}) ((me\textasciitilde{})@xxxx.uxxxx.edu)
Sent:        Tuesday, November 04, 2014 8:10 PM
To:        
e (e@xxxx.uxxxx.edu)
Cc:        
m [m@xxxx.uxxxx.edu]; r [r@xxxx.uxxxx.edu]; p [p@xxxx.uxxxx.edu]

Hello e, 

I am sorry to say that I missed today's meeting without notification. This is my first time missed a meeting. As an "environmental-friendly" team, I hope later on if such things happen, a short one sentence email reminder could help, and hoping that later on whoever is missing or late, we could do the same thing. 

Though you declared it's a short meeting, the statement and attitude about the meeting contents still surprised me a little bit. I am currently working on the GUI Qt Creator Interface as we have assigned about a week ago, and I will meet p to discuss and finish our subteam's GUI design and documentation. Since p has added the GUI image into wiki already, I simply added my little bio-:)

By the way, are we suppose to meet this Thursday for our team member meeting, or it's always consistent on Tuesday afternoon at 2:30 (and I need to stay in CSAC from 2:30-3:30 on Tuesday, but available on Thursday 2:30-3:30, as you already knows)? And later on if we do use the 2:30-3:30 Tuesday one hour, I hope we can do it in CSAC, or we find some other hour so that I can join. 

thanks,

(me\textasciitilde{})

\chapter{流水账}
\label{sec-24}
\section{流水账(1)}
\label{sec-24-1}
想不到我居然真的写了那么长的邮件,现在读起来依然可以回忆起当时发生的几件主要的事儿。 

10、31周五,作为系里交际花般的存在,E提着个小桶里面装着糖来到了CSAC。我极少吃糖(糖可以促使肿瘤生长更快)所以自然没要。因这天早上之前我的sub-team member P刚教会我怎么创建form-based interface (早前安装Qt-creator时M的test case也是form-based一个按键,但一直以来,我厌倦了鼠标,自己更喜欢code-based interface而已),于是跃跃欲试地用form复制了很多个save button。到E提糖到来时,我把自已的最新进展form-based interface就从自己的笔记本上展示给E这个team manager 看一下了,看过了,她也什么话都没有说。 

有一次CSAC里,是周二还是周四下午我一个section结束的那个小时里吧,我在那里等自己SEC里的学生抓紧最后的时间交上作业来,但E和老大等在那里。他们也并不是(至少并非两人都是)在这个时间段值班,但当他两人同时出现,而且以极大的声音和极高的热情帮忙回答我seC里的学生的问题的时候,就造就了一种奇怪的景观,我心里难免难受。更有甚者,R碰巧来到 CSAC,R压根就不是CSAC里的值班人,居然也唯E马首是瞻地加入了他两人的行列中去。后来senior design课前几分钟,R居然还再试图讨好我开心,让我教他说中文,他的确这学期在选一门中文课,有正规的课本。不管当时我心里有多难受,做做样子也需要的,于是帮他念一段中文。

一次与代课老师的会议上(大概就是上次R作了C\# REVIEW的那次吧)E问谁愿意跟她一起作一个具体什么幻灯片版GUI(好像有点儿记错了,具体是做一个什么不记得了),他们男生都想跟她一起做。会后我问E我能否回入到他们一起,E就说她想好了一件更适合我的事让我做,她说上次与我一起REVIEW team contract,她看见我用emacs latex org-mode太纯熟了,她安排我说我应该把文档整理到一起。我当时就很犹豫,我是一个国际学生,对阅读文章和读写文章等的障碍使得这文档的活儿是天下我最不愿意干的事情之一。她坚持说让我先考虑一下。后来接下来一个周的组会上,我明确地说,文档是分配给所有人一起干的活,我一个人干不来,因为我不知道到底该包括什么内容不包括什么内容。但若你们真看重我的latex技能(这五人team里只有M和我用latex,他们三个不会,M只是偶尔会用到它,senior design课上他尚没机会用到),你们把版块内容规划好,把内容材料准备好,我再作作体力活把他们连起来是可以的。否则我一个人干不来。话说到这份上,E也只好说这事不急,那这个就等等再说吧。当然,这个天底下我最不愿意干的活儿,supposed to be 所有team member一起干的活,最终还是不折不扣地落到了我一个人的头上,当然,这是后话。
\section{流水账(2)}
\label{sec-24-2}
其实我原本是想把一些邮件去除掉姓名和邮箱地址后都贴出来的,但无奈买买掉不让,只好忍痛割邮件了,那些个英文版的东西就只好不再出来吓大家了。

不难想像收到邮件后的E的态度,尽管她极力保持着镇静,但言语间的着急分辩也就成了不争的事实。可以想像,接下来的几封过往邮件,也不过是E分辨她的立场,解释为什么会说P的表现会是那样,P的邮件会公开反对E的说法,因为这对P已然造成了伤害。对于E的前言不搭后话,系里既要安慰P所受的伤害,又要维护E的绝对江山统治地位,于是,偏巧,就在那两天的某天,P拿到了他所在实习公司的permanent position。这个学期太多的小概率事件集中爆发,这次P的permanent终于让我想明白,为什么大家会在暑假集体结婚,为什么E会在暑假结婚,为什么E作为基础并不扎实的本科生会拿到系里的TA,为什么小伙会在刚满21岁周岁多几个月就结婚。。。进而,后来为什么M会在感恩节前回家一次,而感恩节却曾另作安排,为什么M没有选大牛的课却老去见大牛,为什么M会像粘在我鞋底的口香糖一样踢除不掉却一再协助系里confuse大家,说到底,他们不都是在像那大牛的跟屁虫一样(前面交待过,senior design安排在了小伙的linux core项目里)在争取系里的信任好得到一个好工作吗?

那段时间的android app,我早已调整了自已的进度,开始集中精力写代课老师课程所要求的app,因一次不小心的无心之失,表露出对android graphics的一丝兴趣,马上代课老师的新邮件就发到学生那里了,呵呵,还真是响应人民群众的呼声,发得真是及时,那我先前想选的统计相关的课程为什么这年头一门都不开?同样这个代课老师的game programming是每隔一年的春天开一次,上次板砖还是板块(上次写给他起的名字我自己也不记得了,同是系里的中国学生,是男闺密的亲密伙伴)选时我基础不够,为什么此刻这个春季学期已然是隔了一年,这门课就还是没有开呢?当然就这想要强迫别人读博士的风波终究还是会过去的。

Dear CS Students,
Thanks to an administrative feat of no small merit, a section of CS 328, Introduction to Computer Game Development, has been added for Spring 2015. The course will be T/Th at 12:30. Since this is a late addition I thought I had better let you all know.
I note that CS 324, Computer Graphics, is also being offered this Spring, and I highly recommend it.  Computer Graphics is not a prerequisite for CS 328, but most computer games are graphical and the two courses go very well together. CS 328 will be easier and more fun if you have done some computer graphics programming before, or are taking CS 324 concurrently. CS 324 and CS 328 are both prerequisites for the CS 428/528 Multi-User Games and Virtual Environments course, if you may want to take that down the road.
I am happy to answer questions by e-mail.
Warm Regards,
Dr. J
\section{流水账(3)}
\label{sec-24-3}
没有了邮件的跟踪与track,我自己也快要写混乱了。写到哪里了呢,我的GUI,确切地说,组会上分配给P与我这个sub-team的clickable GUI interface.遥记得那次分配任务的组会上,M反问我你知道GUI是什么吗?就是一个可以点击的,晚些时候我们会把各个主要function连上去的图形界面。话说为什么一个组里五个人偏偏就又是我选了GUI呢?记得当时任务分配出来时,M似乎对新事情新方法总是很感兴趣,所以到那天分配任务的组会,他已然已经找好了创建state graph的软件! state graph就这样被他抢走了;E承认她暑假干过wiki html那些活儿,所以她抢wiki,class diagram也是他们之间software engineering 课上训练过的,所以他们练习过的、在他们脑海中留下过记忆的活,他们全抢走了,就算下一个全新的GUI,P迫不得已与我一起干,还将在E的诱导下,至少这次与我合作GUI,最好什么活也不干,反正有的是人出面帮他担着。

那这两三个星期前早就分配好、千呼万唤、众里寻他千百度的GUI,我究竟又干得如何呢?不好意思,非常悲催,11、11 design review,到10、31号E提糖进CSAC的时候,她看到的还是不堪入目的form-based的几个button而已(虽然在team environment下我也是一再尝试用不同的方法作出自己的努力);到11、6周四的组会,有且仅有m的state graph和E的wiki page 11/4 due完成了,但是都还没有review. M在周二的会议上已经帮E同R一起设计了几个qt creator的header文件,但E和R的class diagram也还没有完成值到R后来不得不做幻灯片的最后一秒。

R这个学期的这门senior design课,他除了稍微动了动嘴皮子,几乎什么也没有做。 11、6周四的组会,E继续推荐R来作报告,我心里显然不服。不是说我自己想要怎么怎么样,而是我心里有更好的人选。不得不说P的图像处理和video taping以及授课教课的技巧还是很不错的,他在youtube上创建了很多教别人学数学、学programming等等的视频,据他说因为这些个视频,他现在每月还是相当可观的收入。我心中更合适的人选自然是P了。但这想法我憋在心里没有说。这次组会上R说 ,要是11、11 review那天我们能有clickable GUI demo就好了。然后大家聊起review那天R拿什么来作底板参考,R说什么也不用直接说就好了,这次我当然说要用幻灯片,这种review怎么可能不用幻灯片呢?但E说用R用wiki page就好了,wiki page那时两张图片,三段介绍项目的话,四五个bio,我心里难免呵呵;E接着说要是review那天我们team没有什么好demo的,她可以安排P作一个短视频模拟一下我们最主要的功能(呵呵,终于明白为什么学生合作的项目居然还需要弄出一个拥有绝对权力的manager了)。我能说什么呢,我的GUI还没有写出来呀。。。

那天组会结束的时候M问我GUI写得怎么样了,我答"My gui hurts people's eyes."他便作罢不曾看见。组会结束后,我的sub-team member P与我前往CSAC(之前组会是在M办公室里开的),我想问他form-based interface怎么往一个button里嵌入图片,比如打开文件夹怎么在那个按扭上嵌一个传统的黄色打开文件夹的图片,code-based的我知道怎以弄,但form-based我查不到。于是在CSAC我们并排坐着用两台笔记本分别又搜了半个小时,都很挫败。现在周四,下周二review,已然到了最后的时刻,要怎么办呢?同样很挫败,P便打电话给E。E很给我们台阶下,P转告我E电话里的内容说,我们只需要再创建一个设计上比第一次snapshot day有所改进的图片版GUI就可以,怎么改进,比如说,上次GUI只有四个箭头只能往四个方面移动,组会里我们讨论过了,可以改成八个方向就行。 P甚至用他的MAC系统软件教我怎么用powerpoint画一个箭头。因为我写GUI进展缓慢,在我眼里第一次snapshot day P好歹已经有所表现了,理该我自己做些事情才好,于是我主动承担下这个任务,对P说这个周末我就作这个了,到周一下午你编译课后如果我还有什么问题我再打他电话聚一下看进展如何在一起完成所剩无几的我不那么confortable的部分。P给我留了他的电话号码。至此,蓝天白云,云淡风轻,是夜无话。

\section{流水账(4)}
\label{sec-24-4}
人的意识真是个奇怪的东西。E在那天开会的邮件里刀光剑影,寒气逼人,并借用step by 办公室的机会借用他人的假想全面塑造了我很弱什么也不会的立体形象。所谓等待我GUI的结果,也不过是万事惧备只欠东风。不难想像,东风与了E,后果不堪设想。只可惜,亲爱的读者,对,你没有猜错,这个GUI最终还是一定是会写出来的啊。接正式时间顺序叙事体吧。

那时的自己当然还是相当的懵懂,全然没有意识到周遭到底在发生着什么。话说那天周四的晚上,回想自己在qt creator上只花了三个小时的时间折腾折腾了menu bar,并没有花太多的时间。虽然E很给自己台阶下,但我真的就该顺着台阶下吗?遥想"公谨"当年,曾经做出的项目建立的自信都停留在心中,第一个学期第一个月写出500行lisp是班上仅有能写出来的两三个同学之一,那学期唯一一个写出decision tree项目的AI学生,更有甚者,暑假实习最后一天午饭归来仅剩三个小时强弩之末状态极差的自己毅然决然地在咖啡的强制药效作用下把剩下没有头绪的部分写完直到fully-functional demo给team manager看.所有的个人历史经验都告诉自己,我一定要把它写出来。更何况,这整个周末,我就做这事,之前之所以layout出不来,可我也只花了三个小时的时间而且写的是code-based,我有什么理由做一件事情只花三个小时遇到一点儿困难就立即放弃?那天周四的晚上躺在床上,我终于是鼓足勇气下定决心,明天,我才不要去做什么powerpoint based改进版升级版图片版GUI,我要做当然就是做真正的GUI。

当一个人自己没有放弃的时候,就没有事情会自动坍塌,所以,正如大家所期待的,我做出来了。从周五周上一到CSAC,我就开始google写那些个连button 的代码,step by step,到下午开TA例会的时候,我已经连得不亦乐乎,因 为到这时,我已经拥有足够的自信,我知道,到周二review前我一定写得出来的,而且我正在快马加鞭地赶进度,只可能写得更快,只可能写得更好,没有完不成的道理。

待我真正完成这样一个GUI,奇怪的问题出现了,E不允许我与R一起上台讲,不允许我上台讲就不允许吧,我自己又不是不知道自己做出来得稍微晚了点儿,要不然要是能早点儿做出来也好有个讨价还价的资本。当然非常悲愤的是E不仅不让自己上台demo,而且还把理由说得冠冕堂皇,还表现得像她是多好的leader一样,就好像是说这个E又 做了婊子不允许我上台demo还要立牌坊她是多么好一个leader,我当然免不了悲愤不已,怒火中烧,狠狠地讽她一回她这样几近完美的manager原来也是打击别人的积极性不留余地,我之前从来不曾遇到过这样的manager,感谢她,遇见她,这也算是让我长见识了。。。。

后来,E还是很不满别人把GUI写出来吧,还是对别人能写出来她行动失败恶气心中留,我upload到google docs上好几张GUI snapshot图片,她挑了张最早的长得最丑的放到了她的wiki page;11/11 demo那天,十一月的天气E故意穿了professional的职业正装裙子去上课,课前还故意造势问其它同学说,你有没有遇见过什么不懂得如何communicate professionally 的人(暗讽我不懂得如何交流)?课堂上R不去讲真正design review的class diagram,而是专挑我GUI上没有完成的部分讲,比如那两个显示时间的框,人家也没有说没做完,而就是专就这两个框反来复去地讲,直到听众自己心里明白,哦,原来这个东西都还没有 做完呀。如果说之前我心存幻想,到今天这次review的课上E和R的联手表现(难怪E要R上台讲,原来 是要科普歧视教育呀,看来他们software engineering也不曾白合作一场,终究是达成了某些共识的,可怜这个跌跌撞撞误入senior design的自己),终于使我明白,生活在别人的地盘上,受欺凌是时时处处存在的,不服不行。那段时间我不小心把一些相关邮件遗留在github上,及至很多人读到邮件,都一致认为E太刻薄,实在是冷得不能再冷。但就像2、14、2013我写出那个实时操作系统keypad的configuration,生成square wave非常高效时受到我的同学最纯真的尊敬,那是第一次也是最后一次,觉得E待我冷那也只是人们最初最本质最本真的反映罢了,再之后紧接着就是别人爱国教育的结果,全方位协同系里协同学校乃至后来协同食堂一致对外了。

\chapter{来自cs120例会的风向与脏水}
\label{sec-25}

大概这段时期的某个周五的例会,他们发明了一种新玩法,兴起了一股热潮。记得提到E时前面我好像有提到过,E和她的老公,不知道他们两个有没有选EC代课老师的AI课,但刚过去的春季学期,只有M与我一起选了EC课,E与他老公与我一起选了我导师的高级操作系统课。这门高级操作系统课上得那叫一个难受,代课老师的programming一块是相当弱的,一次一个同学提出的问题,代课老师误把string说成了int对他自己的口误意识到却无力纠正,让大家的心理都失望了很久。那时我对自己的导师在2014年1月在编译课老师强给我一个C后我滔滔江水般泪水泉涌对系里代课老师的公正性产生严重怀疑时导师给予我的鼓励还心存感激,所以后来在导师讲pthread的时候装弱借一个老师清楚的小编程问题提问请教老师使他开心。到后来,及至我的毕业课题一拖再拖拖到每周见面例会我只能对牛谈琴般讲出自己实现的部分而导师不再提出任何建设性建议时,我终于意识到一直以来我究竟生活在多少绝望的环境里。

呵呵,终于是又说偏了。所以E和她老公春天连EC课都没有选,是迎合EC代课老师这个秋天因她心领神会EC代课老师的意思暑假就结婚了而给予她这个本科生的秋天半个TA的机会,原来那天大家的热情在EC代课老师春季开的一门新课上,那天例会E一脸热情讨好地对EC代课老师说,她知道自己很多课没有选,她能不能春季选EC代课老师的robot什么课。呵呵,我很麻木。天知道E这话是说给谁听的,刚过去春天的EC课代课老师刚拦了别人的胡,傻瓜才会来年春天再接着受同样的罪好吧。

这次TA例会后,来自于导师印度TA的脏水就立即泼了过来。是开学的时候四个TA导师都给了他们一份打印出来的教材,但导师没有给我。后来说是给我补一直没有。所以期间就真有一次我借了印度TA的教材,但事后按照他的要求放在了系main office他的main box里。我借过一次之后就敦促导师问他要了教材了不曾再借,而且之后我们的实验还有用到教材,此印度学生早不说晚不说这时便把脏水泼向我,谁心里都明白是怎么回事。所以礼节性地回了他一封邮件之后就不再理他了。

I need the book (professor Terry's book for CS120) that I gave to you back to prepare myself for tomorrow's lab. Could you please drop it in my mail box ASAP?

I remembered that I borrowed the book from you once, and I put it back into your mailbox before my Tuesday's section already, and I don't have any CS120 book that belongs to you. Could you please try to remember if you can memorize where you put it?

I have not collected any book from my mail box. I never got one there. May be u misplaced it. Could you please double check on your side?

后来感恩节期间,我被自己男闺密一拨在系里男博士的疏远影响下孤立了,却被女博士一拨邀去potluck参加了感恩节聚餐。他们是联络比较多,而且惯例是去邻州一个免税的城市抢够的。但我没钱不去购物的原则还是不难坚持的。倒是印度TA泼墨有攻,系里他们印度人一拨被邀去了E家作客。至此系里的几拨几派依稀可见。 

\section{Lab 11}
\label{sec-25-1}
Soule, Terence (tsoule@uxxxx.edu)
Sent:        Monday, November 10, 2014 12:31 PM
To:        
e (e@xxxx.uxxxx.edu); (me\textasciitilde{}) ((me\textasciitilde{})@xxxx.uxxxx.edu); 我导师, Robert (rinker@uxxxx.edu); Chitrakar, Anup (chit8942@xxxx.uxxxx.edu); Rubini, Joshua (rubi4714@xxxx.uxxxx.edu)
Hi,

Lab 11 is now posted at \url{http://www2.cs.uxxxx.edu/~cs120/f14/soule/lab11f14.html}.
The lab itself is pretty simple, enter the code from the text, make some fairly simple modifications.
However, the concept behind the code is tricky and may be worth taking some time to explain at the beginning of lab.  The program creates a 2D array of pointers to robot objects.  This array represents the "world".  Most of the pointers are null, except when a robot is in a given cell in the world.  When a robot moves it is "handed' from pointer to pointer.

Let me know if there are any questions/concerns.

Thanks,
Terry


\section{CS120 Book}
\label{sec-25-2}
Chitrakar, Anup (chit8942@xxxx.uxxxx.edu)
You replied on 11/10/2014 4:59 PM.
Sent:        Monday, November 10, 2014 4:57 PM
To:        
(me\textasciitilde{}) ((me\textasciitilde{})@xxxx.uxxxx.edu)
Hello (Me\textasciitilde{}),

I need the book (professor Terry's book for CS120) that I gave to you back to prepare myself for tomorrow's lab. Could you please drop it in my mail box ASAP?

Thanks,
-Anup
\section{From: (me\textasciitilde{}) ((me\textasciitilde{})@xxxx.uxxxx.edu)}
\label{sec-25-3}
Sent: Monday, November 10, 2014 4:59 PM
To: Chitrakar, Anup (chit8942@xxxx.uxxxx.edu)
Subject: RE: CS120 Book

Hello Anup,

I remembered that I borrowed the book from you once, and I put it back into your mailbox before my Tuesday's section already, and I don't have any CS120 book that belongs to you. Could you please try to remember if you can memorize where you put it?

thanks,
(me\textasciitilde{})
\section{Re: CS120 Book}
\label{sec-25-4}
Chitrakar, Anup (chit8942@xxxx.uxxxx.edu)
Sent:        Monday, November 10, 2014 5:02 PM
To:        
(me\textasciitilde{}) ((me\textasciitilde{})@xxxx.uxxxx.edu)
I have not collected any book from my mail box. I never got one there. May be u misplaced it. Could you please double check on your side?

Regards,
-Anup

\chapter{脏水源源不断}
\label{sec-26}
实事求是地说,在E wiki page due后,在他们开一场设计会避开我时,我还很迟顿,并没有意识到在发生着什么。但当周五我有确信的自信我能写出来,周六周日在系里遇见大牛的跟屁虫以及E的强烈不合情境的反对我上台demo,以及周一team mate看见我的GUI后的第一反应,让我开始意识到我曾经处在怎样的危险中。亲爱的读者是否还记得那年暑假实习的倒数第二周发生了什么?8、16我对manager说我的项目刚做上热乎劲就要结束了,能否延长一个周?接下来的周一8、 18我已经征得导师同意推迟开学延期一周的签名,但因为那时的不成熟,在一些程序的处理上出了点儿小差错,那个周公司这边HR这边在干些什么呢?去年夏天我再次站出来写,或许这也是那次我站出来写给学校的一大启迪吧,因为到这时学校这边显然已经开始了同样手段的暗黑处理。而到这一刻,我终于意识到,如果我按照他们上compiler的进度,如果刚过去的几天我顺着E给的台阶激流直下,如果我没有写出这个GUI,E早就在对我的邮件里发起"冷"攻,更有周三那席看似不经意chat周密的旁敲侧击的风向布置,局面到现在此刻我都不愿去多想。可以这么说,若是我写不出这个GUI,那么那年暑假的python GUI的功劳成绩将被清零; 就像我与那个小自己13岁的M本来就是什么关系也没有,还要在这个学期被系里诱导利用大刮特刮舆论歪风,若是那个周末我的GUI出不来,那实习期间以及接下来的春天因与美国同学小伙在facebook上的互动而被再次热炒了一遍的我实习期间的mentor与我的所谓"暖昧"关系,保不定还要被学校系里炒成什么样呢。。。。

好在,好歹我还算是把它写出来了,即使是在完全不给力,甚至是在E故意制造种种障碍拖别人后腿的情况下。倒是后来,越来越多地意识到在真正做一个项目时,c++资源管理方面的种种问题,到我对这个项目理解越来越多越来越深的时候,回过头来,再看自己写出这么一个东西的这一步,却多少有些为自己曾经在写出这样一个半残残次产品时的所谓自信感觉很差羞愧,太不直一提了。而系里,竟然会能设置这样一个情境来考我,也只能是反应了系里不曾真正诚心教我这个学生而建立所得的自信和系里整体教学水平低下的客观事实,俱往矣。。。。

所以,当时,意识到这种所谓的压迫,欣喜自己真正的拿出了曾经邮件里表露出的热情和坚持,用一个周末的时间把这个东西写出来,这之前不曾经历过的新经历还是多多少少给当时前途暗淡的自已增添了些许光亮和自信。不是不让我demo吗,github是属于我自己的领地,我可以在那里记下培养成长自己的心路历程,于是我就真把这些记录下来了。

记得从小到大的政治课上我一直在背"一颗红心,两手准备",对于系里EC代课老师春天里故意拦了别人的胡,系里决意拿我杀鸡警猴,极有可能是故意策划制造了这样一起方案,岂有没做好两手准备的道理?所以,我在github 上创建一个新的repository senior design后,系里小秘的邮件就发出来了。。。

我也算傻,当时真的是没有看出来这邮件与我有什么关系,甚至还因为她说感恩节其间的什么时间表CSAC值不值班的问题, 怕因为自己没理解透误事,特意跑去系里找小秘问过,没有什么异样。之前我一周三个早上要到系里值班,正如之前我已然提及,这个学期因着这样的值班时间安排过得非常充实,我并不曾怎么迟到,倒是一个叫Alex的人常常与我一起值班的时间不来,我语言上警告过他一次以后,他也不敢再大意了,至少在与我一起值班的时间按时出现在CSAC里。

这天收到小秘的邮件,感觉到她似乎在黑自己实施高压,而我也没有什么 好惧怕的,便在收到邮件的半小时内把实习期间我最后一个项目的python GUI用手机拍的两张照片放在了github上,并稍微解释了一下那个项目我都做了些什么,具有哪些个功能。后来就印度TA不值班(与我同时值班的时间段里他不出现)的问题我问过小秘,小秘回答得非常精巧:管好自己的事情。及至后来系里我导师解除我CS120 TA的事实证明,系里印度人脏水的小打小闹正结束,而系里学校里惩治一个异端的手段做法才刚刚拉开序幕而已。。。。

\section{Upcoming CSAC Changes}
\label{sec-26-1}
系里小秘 (系里小秘@uxxxx.edu)
You replied on 11/12/2014 4:30 PM.
Sent:        Wednesday, November 12, 2014 2:57 PM
Cc:        
Donohoe, Gregory (gdonohoe@uxxxx.edu)
Good afternoon all,

As you know fall break and Thanksgiving are coming up. At your earliest convenience we will need to know what your intentions are for that week and which shifts you are available for.

We would also like to inform you that after the week of Thanksgiving we will have a new sign in policy in place in the CSAC. Beginning Monday December 1st you will need to come to the front office at the beginning and ending of each of your shifts to sign in and out respectively.
If you have any questions of concerns please feel free to contact me.


系里小秘
Student Coordinator
University of XXXX
Computer Science Department


\section{RE: Upcoming CSAC Changes}
\label{sec-26-2}
(me\textasciitilde{}) ((me\textasciitilde{})@xxxx.uxxxx.edu)
Sent:        Wednesday, November 12, 2014 4:30 PM
To:        
系里小秘 (系里小秘@uxxxx.edu)
Hi 系里小秘, 

I am writing to say that I will stay on campus for the thanks-giving week. So my schedules can keep unchanged. But if you need me to cover any shift or something, please don't hesitate to let me know. 

I understand and will obey the sign-in sign-out rule from December 1st. 

thanks,
(me\textasciitilde{})

\chapter{怎么做都是错}
\label{sec-27}
(加段题外话。因为这个学期在学校食堂打工,根据每次打工食堂开设窗口的不同,有时回来会非常累,有时晚上回来又可以写一会儿,但无论如何我还是会坚持把它写完的。请不要因为我哪天贴晚了还去揣测我可能有的心思变化什么的,对于我,感情的归宿是确定的,我已经写清楚了,我的生活也很平静。我只是根据自己身体生理的疲乏程度来决定打工回来的晚上要不要写和更新。但我也会尽量给出预估的更新时间。这个周结束后下周是spring break。希望最迟到spring break那周加班加点也把它写完。)

能攻心则反侧自消,从古知兵非好战;不审势即宽严皆误,后来治蜀要深思。

在fault tolerant课上,代课老师前段时间还在一直强调似乎暗示影设我lie,可真正是谁是lie呢,是谁在转移lie的bottom line呢?后来与E的邮件事情之后(从邮件里可以清楚地看得出,那个skip掉我的设计组会的那周三晚我发出去的邮件周四一早他们都有去见那个硕士生senior design代课老师),所以系里也是很清楚地在把握着这些脉动。之后fault tolerant的代课老师不再强调lie了,强调global agreement。我心里也在暗自捉磨,也许有一天,我也需要设一个与系里的global agreement,看谁再lie\textasciitilde{} 

我按时写出了一个看起来还不错的GUI后,后来,正如大家所预料的那样,作为一个fault tolerant的建校125周年的系,风向无声无息中瞬间就变了。11、11号那个周我们计算机专业的项目分两次课review的,我们的项目是周二的最后一个review, E的邮件拒绝我demo的理由是说我们每个项目只有10分钟,没时间demo,但真正review时,代课老师特意给E和R长威风给足了R台上20分钟,而这段时间R讲了那两个起始的时间框至少讲了10分钟。。。。

及到周四的review课,有一个另外项目的同学台上就直接说谁到去读别人1000多行的代码(我的GUI因为是用code实现的,而不是像M的test那样form-based,所以几个文件的代码加起来差不多1000行左右吧),意即这是design review,压根儿就还没有到implement 一个GUI的时刻。至此,一两个月来种种规犯加在我头上的一场生死考验瞬间就变成了我们这个项目的全组成员不给力,miss 掉了design review的point。可这个point早在我写GUI的周末写给大家的邮件里我早就指出来了,我们的point是design review,why bother 要P去制作一个视频?但E装聋作哑不听见不回音。所以这个周四review的最后一个项目,我直接提间那个台上的人同学,请他把skip掉的那张class diagram幻灯片设计里有向个类,类与类之间是什么样的关系讲清楚。

如果说周四的review还只是风向初转的端倪,那接下来的周末11、16 真正design review组会上E才算是真正发作,把罪行强按到我头上,那天会上E指责这是我一个人的错,是我miss 年他们的point,没有人要我implement一个GUI。。。。回想那次组会分配任务他们挑剩下我就只有一个GUI可选时M的好心提醒,回想11、11 design review前一个周四组会上R说要是我们有个GUI可以demo就好了,回想E说要是我们实在没有什么可demo的,她可以要求P制作一个视频,那个周末临时加演的design review组会上,真真切切亲耳听到E对我的蛮横指责,我疯了,小伙伴们也凌乱了。。。。

\chapter{11/16 design review}
\label{sec-28}
前面已经点到,今天具体写design review的过程的几个重点事件吧。

那是一个周日中午11点钟,等我到时M和E已经在那里了,白板上M甚至已经开始画起了GUI界面menu bar的几个总栏。E的小笔记本已经打开,已然是做好了架势继续做她一贯的方案工作。可怜的P和R是本科生(奇怪,E和M也是本科生啊,按理本科生没有周末出入系里大楼的权限才对,这个细节当时我没太注意所以我也就不知道了),居然是进不了大楼的门,于是M还出去给他们留了一下门的。P打电话说他在学校哪里的coffee shop,问我们要不要什么东西,于是几个人点,我点了杯普通咖啡。P是我们圈里的新贵,上周design review周周三还是周四刚拿到permanent position,非常开心,已然不需要在这样的科目上再如何出力,告诉我们说让我们先开始,他随后就来。后来那天的会,他也是扮演了和平使者,没有从项目上出力,但在调和组里的气氛上还是有些功劳的。

及至P来到我们所在的thinktank会议大厅,M已经领导大家基本完成了"file"下拉菜单。在M的领导下我们努力想着这个项目这个GUI可能用到的所有的button命令,他们谁可能私下有见我们项目的client也就是我导师,说是GUI里用到的所有的button命令必须在菜单里出现。在M的领导下R似乎稍显郁闷,偶尔发言但话不多。我也在努力地增加着自己的份量,那些个什么play pause stop 快进、快退,最前最后等视频播放 器里我有印象的相关的我也都说了。P坐在E的旁边当了E的小跟班。

M的几个下接菜单花的时间并不是很长。等M基本快要讲完他引导的菜单,这时作了好一阵子不曾说话的E走上讲台要发话了。E的表情和观点都很颠覆,E说为什么我们要抄前一年他们ACM的GUI设计,为什么我们不能有我们自己的设计,为什么我们要把大方格操作主界面放在左边而不是把它放在中间。和我们大家感受到的诡异一样,E在讲台上继续讲着她自己脸也红了。等E说完了我是第一个站出来反对这种观点的人,原因是E的新观点设计基本完全否定了前一个版本的设计和第一次snapshot day P和E的功劳,以及我已经实现了的GUI,而且现在的monitor都是宽频设计,返回到竖长条的设计是南辕北辙。所以我反对。我的反对是有理有据的,我的语气也是平和的。大概E是等待这样一个阐述和表明她观点的机会等了太久,所以她显得那么地迫不及待。他接过我的话头火急火燎地就指责我说,"Nobody asked you to implement a GUI. It was your fault had implemented it."

到这一刻,我感觉自己五雷轰顶、脑袋短路,除了苦笑,想不出自己该说什么,无言以对,甚至眼框有些湿润。那个那次组会我领这个GUI任务时提醒过我什么是一个clickable GUI、任何时候都能早早地按时保质保量地完成自己的任务的M这时发话了:"If it weren't you, we should have done this and finished the design review one month ago. You are the one have been blocking the whole team from processing well as the team manager." M大概也是个火暴脾气的人,作为任何工作事情总是早早地完成任务的人,而他的state graph也一直不曾review,或许也消耗了他的耐心吧。到这时我被雷昏的脑袋有所恢复,被E折磨了太久,经历了太多,我的语气中透着愤怒和坚定,镇静地对E说,这是她的错,作为team manager,"you lead us do wrong things at all the times. You know what, I am going to write an email to you set as a mark listing all the wrong things you have done so that you know you can realize you made all these mistakes."听我说这话,E非常不以为然,说你写呀。。。。

后来在P的调停作用下,大家慢慢地calm down。至此,E要逆转风向、否决有人要我implement一个GUI的陈词显然成功传达给了组时的其它成员,E 的所谓新设计不过是要否定我已有的GUI implementation多一些从而fault tolerant掩盖系里曾经给我出过这样一个难题的历史。到这时,组里这次会议的气氛早就已经变了样,要否绝一个方案也已经不能只凭嘴上功夫了。E和P一起出去买Pizza了,R用M的笔记本在建E的新提案form-based GUI。不知道什么时候本科生的大牛的跟屁虫居然也来到了会议室(这个本科生也有通往系里的钥匙?现在回想起来居然有这么多疑问,但当时是不曾多想的),同他们打过招呼之后,居然是同M最亲聊得最多最久,还同M一起吃了pizza,吃完两人还在聊。。。。回想一下早前在CSAC里有好几次此跟屁虫也是去到CSAC里与M热聊过了。是从什么时候起,M与跟屁虫的关系变得这么好的?我心里不禁起疑。

后来吃过中饭pizza后,因为我们想要我们GUI的每个小窗口是与实际我导师用来tower light show的tower的实际窗口成比例的,根据R的form-based E提案GUI,E的提案的确不符合现代人的宽屏视觉习惯,而且必须得向下拉滚动窗口,非常不方便。R是与E一起经历过software design想来这种配合磨合过很久的"狼狈为奸"的小伙伴,及至R亲口否绝了E的提案,E也就没什么好说的了。R把这天中午他借用M的笔记本制作 的这个GUI存了下来,E和我都希望R到时能够发给我们,R口头答应了,但后来我从来不曾收到来自R的这个form-based E提案GUI,倒是后来我自己整理docs需要这样一个东西时我自己已建了一个,也不麻烦,几分钟搞定的那种。

那天吃完中饭,稍微讨论了一下两个方案了比较之后,四点钟左右,M声称他有事要走人。这样这个前后持续了五个小时的design review组会成为了又一次的蜻蜓点水、隔靴瘙痒。及至后来我终于意识到,自始自终我都从来不曾参与到任何与项目设计相关的部分(后来我被要求对项目完成自己的设计),以及后来被代课老师把我从这个项目隔开之后M/大家对我的冷淡态度,我开始明白M与他们也是一样的,只是一种方案的不同走法而已。我对M的戒备在增加,只是我的麻木使得我没能想到,我也已经走到了与这个项目、与大家说再见的时候了。

\chapter{footprint \& global agreement}
\label{sec-29}
\section{footprint \& global agreement}
\label{sec-29-1}
(me\textasciitilde{}) ((me\textasciitilde{})@xxxx.uxxxx.edu)
Sent:        Tuesday, November 18, 2014 6:42 PM
To:        
e [e@xxxx.uxxxx.edu]; 系里小秘 (系里小秘@uxxxx.edu); 代课老师 (代课老师@uxxxx.edu); Alves-Foss, James (jimaf@uxxxx.edu)
Cc:        
m [m@xxxx.uxxxx.edu]; p [p@xxxx.uxxxx.edu]; r [r@xxxx.uxxxx.edu]

Hello e, and the related department, 这部分不要

As I pointed out that it was all your fault which leaded to these whole confusion, and have stated in the meeting on Saturday that, I was going to write a email as a footprint to prevent you from future suffering, and help myself clear my life here in U of I. 

Facts:

\begin{itemize}
\item On team meeting \textbf{10/7/2014, Tuesday}, always being active and fast towards projects, m suggested us to install Qt Creator, downloaded Qt Creator sample codes into github. He helped p installed the software on Windows, and I did mine in Linux. m's sample code created and menu bar "File" command already, and "Exit" from submenu of "File".
\item On team meeting \textbf{10/7/2014}, the same Tuesday meeting, you excused yourself having trouble with MS c++ IDE, and would do rather the IDE package than the independent Qt Creator, and r spontaneously offered helped to you that he was going to help you look into MS c++ IDE if you needs any help.

\item On team meeting \textbf{10/7/2014}, the same Tuesday meeting, m, p and me got our Qt environment ready, while r and you chatted and talked about your compilers, and draw the "Design Flowchart" on whiteboard and listed, "C\# prototypes organized uml diagrams, prototypes/language training Diagrams", and assigned to r dated "by coming Tuesday r", deadline 10/21/2014; "Design/Docs/Exact Class Diagrams" deadline 11/11/2014;

\item On team meeting \textbf{10/7/2014}, the same Tuesday meeting. In the end of the meeting, as you were the respected one with full communications with course instructors, and could draw exact deadlines like listed above, when asked about tasks before next meeting for Qt Creator coding training part, you clearly stated as far as we could create one button and exit from there, we are good. m insisted that "I already did that", but you blocked the whole team from processing by giving no followed up tasks/assignments on Qt Creator.

\item On Team meeting \textbf{10/9/2014}, the meeting with instructor Professor 代课老师, you covered part of the project agenda; and you \textbf{begged*/kind of *forced} r to stand into front and speak of something. And r did follow your call and covered parts of the c\# codes on color Pale, color wheel, Grid, Main form. I specifically asked if r could draw a clear class diagram for C\# design, he looked into Professor 代课老师, and the professor gave enough tolerance to r that we didn't necessarily need any class diagrams even you listed it on whiteboard two days ago.

\item On \textbf{Friday 10/10/2014}, in CSAC between 4:00-5:00pm, I asked if you have base version tower light interface, you said no, and we agreed I sent email to ask help from team members; The same evening I sent two emails out, even call for gathering and preparing for coming Tuesday's snapshot day, as the team manager, you completely ignored my email and leave me no chance to prepare for snapshot day by having asked p only (without letting other team member knew about it) to work with you to prepare the snapshot day. And what a probability that our team was completely forgotten by the course instructor as well, not shown at all in the email list the instructors sent out.

\item On week \textbf{10/12/2014-10/18/2014}, because all of you have compilers, we didn't meet. 

\begin{itemize}
\item On Team meeting \textbf{10/23/2014, Thursday}, after having checked with Professor 代课老师, you listed us on board in m's office, that we have assignments need to be done, including state diagram, wiki page, class diagrams, GUI design and description. And e you wanted to work on Wiki page, m would work on state diagram, I wanted to work on GUI design, r and you would work on class diagram, and p jointed me to work on GUI. You set very high standards on GUI doc name conventions and even suggested that we need to consult some other department, but my sub-team member p argued it back that we don't need to.

\item On Team meeting \textbf{10/23/2014, Thursday}, In the same meeting when I picked this GUI design assignment, as an always considerable teammate m is, he argued me, "do you understand what a GUI design is? -- To create a clickable interface that we as a whole team could link functionalities on the buttons later on.  As far as I understood, I knew clearly what I was supposed to do, as clear as any other team member knew. And nobody ever rejected any of m's reminder opinion on the GUI or mine on this GUI design at all.
\end{itemize}

\item On Team meeting \textbf{10/28/2014, Tuesday}, I didn't really remember when m checked in and uploaded his state diagram onto Google drive, but it should be sometime during this week. We chatted and discussed about some details about merging two paths/pattern. r did draw some pic on the board. In the end, you asked r to send this photo pic to you, and asked from r about the same, but I never received any pic from r yet about that meeting's draw on board. 

\begin{itemize}
\item On Team meeting \textbf{10/30/2014, Thursday}, by this time, m probably has checked in his state diagram already, and we emphasized on e, you work on wiki page by coming Tuesday; and r and e work on class diagram, and p and I work on GUI and GUI docs. As m has checked in already, and been emphasized and deadline was coming, I set my mind to work on it.

\item On \textbf{Friday 10/31/2014}, on the mornings somewhere, my sub-team mate p taught me how to create form-based Qt Creator interface. I had fun building some buttons. And between 11:30-1:00pm when you showed up in CSAC, I was working on the GUI, and I showed my currently working GUI form-based interface to you. You seemed to feel exact but good or bad, you didn't say anything about it.
\end{itemize}

\item On \textbf{11/4/2014, Tuesday}, for the first time, you skipped me from the team meeting with no excuse at all. This wasn't the first time that we skip a meeting. When it was you guys' compiler exam week, we skipped team meeting once. And This was the first time you all moved downstairs to have team meeting in m's office instead of in CSAC on Tuesdays, and as the team manager, you leaded the team without notify me about the team meeting. And your coldness in your email freeze my mind, it was like finger pointing and clearly saying that me, as the only one who ever missed the meeting, was the one who didn't do anything at all as you descried p to me the followed day. And I was so considerate and tolerate to you by just saying that "your email surprised me a little bit".

\item On \textbf{11/5/2014, Wednesday}, you didn't reply to my Tuesday evening's email and chose to step by our office, and chatted to me for several minutes. During those several minutes, the points you made includes:
\item Checked me if my sub-team mate p was doing anything at all. Almost as motivated as I am, as everybody knew p has at least prepared the snapshot day's image design already, so did I, I explicitly emphasized that to me, p didn't have such a problem at all, and I always has my full confident on him.
\item When you insisted that later on you will send email to remind us about meetings, I wanted to make it clear that you didn't necessarily need to send us email at all, but you made a point that r is a person prefer oral works, and doesn't like any reading and writing at all. Yet still you insisted you would send email out so that you could remind r because you need to anyway.
\item Combined your coldness on 11/4 email and the scene you prepared for me, if later on I wouldn't be able to produce the clickable GUI, I would be basted to death. Could the scene/environment be any colder on me? Why do I need to suffer this?

\item On \textbf{11/6/2014, Thursday}, on the mornings I received 代课老师's email and he clearly stated that he will have access to our emails. And I agreed I have no problem with that at all with all my reasonable guess what happened during the mornings.

\item On \textbf{11/6/2014, Thursday}, since coming Tuesday will be the design review day already, and nobody ever brought up the design review thing ever, nobody! We were mainly on the r's performance thing, and you asked r to try to mimic the presentation in m's office without any writing material. And r clearly asked if we could have our GUI interface ready that would be something great. I also wished that I could finish have my interface so that we would be able to demo it, which according to me, would greatly improve our grades as well. As that time, e, you tried to be a considerate manager without giving me any pressure by, simply completely ignore the fact that I have been working on it, and have at least showed it you once, simply clearly stating that if r really think we should have something to demo, e you as the team manager, would simply ask p to create a short video to demo the functionalities, which was what happened during the coming weekend.

\item On \textbf{11/6/2014, Thursday}, after our team meeting, which ended at around 4:30pm, my sub-team mate p and I went to CSAC and I asked him about the difficulties that I have, like how to add an image using form-based interface. Together with me, p and I used two laptops searched for some time. As frustrated as I was, p called you discussed about the difficulties that we encountered, and as the team manager, still you didn't say anything about the clickable GUI design related issues, rather than relax us by stating that we could still simply create an image based interface just like what p did already for our snapshot day, as far as it's slightly better, like we agreed that our moving direction should have 8 directions instead of 4. It was 5:00 already, I felt I contributed to the team too limited, and p and I agreed that I would work on the image and docs during the weekend, and our sub-team would meet on Monday at 2:30pm to clear work out. It was during weekend when I really worked hard on the GUI, in the email I moved the GUI docs works to p so that in sub-team he would be able to contribute as well.

\item Before Tuesday's meeting, you have your excuses like you have TA lab section on Tuesday mornings, but you never took your imitative to offer a time like sometime on Monday or weekend to review the slides of r's presentation or offer any guide.

\item On \textbf{11/11/2014, Tuesday}, Design Review with CS related teams gathered together. Before class, I noticed that e you were asking classmates if they have met any person who didn't know the correct way to communicate in a business manner. And during the day's 5 team presentation, Professor 代课老师 gave r too long time on our review with r spent half of the time description on feature that I didn't finish on the time Editor. I did feel bad about you guys by behaving like this. Professor 代课老师, you and I talked after that days' design review class in the same classroom, but I got sick of the environment already. Professor 代课老师 said that he didn't read all the emails, I wondered if he "did", could the situation be any better? And at around 4:30, I received email from human resources, whatever the contents are in the email, I felt pressed.

\item The followed several days communications were completely a mess, and personally I don't even want to recall all these things.

\item On \textbf{11/13/2014, Thursday}, you casted a great anger toward the whole team by blaming us that we missed the point of design review, which I have pointed it out on last Sunday's email already but you paid no attention to it, let alone correct us at all. But as far as you recognized in the end of the meeting, you blamed yourself you didn't know what the fxxxxxx you were talking about in the end before leaving the room, I tolerant your anger, forgiven you and didn't make any sound.

\item On \textbf{11/15/2014, Saturday}, you applied your advantages by taking team meeting notes. You didn't offer any suggestions except that you explicitly specifically blamed that it was my fault misunderstood you. At this point, I cannot tolerant you anymore, and stated to you clearly that it was all your fault by misleading the whole team do wrong things at the wrong time, and I would write to you make this clear to prevent you from repeating such situations. I must be too anger to blame it was you, but actually it could be the department as well, cause you have been the only person keep close relationship with the department, and you were entitled absolute admire by an undergraduate TA.
\end{itemize}

e, please understand that I never take any initiative to hurt any other person. Whenever I need to stand out to say something, I stand out only when I got very hurt and cannot bear the situation any more. It was you, who was extremely unreasonable, that forced me to stand out, clear everything clean, so that later on I won't suffer any more from these afterwards. You were the source produced all these troubles. 


First of all, I appreciate the TA I got for this fall semester. But please all help to realize the facts that I listed. About the TA: as clear as i demonstrated, I appreciate if I could get it, but I won't be any aggressive to fight for it at all. Maybe I am writing trying to reach global agreement between the department and me. 

\begin{enumerate}
\item Early on 10/23/2014, cs120 lab8 section 6, I covered email and demoed emacs presentation slides generation using org-mode, just wanted them get encouraged and learn the command-based editor. When I realize the afterwards influence, with my current advisor fully supported me to demo by design the lab8 to be one easy function only, the coming Tuesday 10/28 (the emacs slides were still left there, but I didn't cover that much and didn't demo org-mode slides generation any more at all), I hide my shining avoid any emacs demo just asked the section 4 to add one line of global-line-number configuration for their convenience. I just wanted to keep low-profile life.
\item After the unfortunate Emacs thing, Dr. Soule fully supported Josh, the other TA's idea about curses window for assignments. For me, it was just a wrapper class like any other project that I can do. I have my priorities for works to do, I didn't pay enough attention to that, but it was followed by e's aggressiveness by picking my accent. I am perfectly ok with my accent.
\item Dr. Soule fully supported towards e and Josh, and the assignments extended one week, the students looked down on me, and I have even received "patent" emails during those days. How the environment could possibly appeared to be that way? I was driven sad by the environment and studies a little bit on <curses.h> library, and demoed the class on Thursday's lab to protect myself from any further suffering. But still, it was just a wrapper of a source library, patent?
\item Compared with a TA for spring semester, I would rather prioritize my time so that I could make good preparation for my job searching, and get graduated smoothly from this department. And that's the reason from very beginning from emacs demo week, unlike e as aggressive as she behaves; I have been completely outside this TA aggressive war. I will get graduated.
\end{enumerate}


I got pretty much all Bs from my previous courses, like cs210 programming languages, cs570 AI, EC, which ones that I liked too much, though I did feel very unfair about some of them, like cs210. I wrote a 500 line of Elisp code to make one tic-tac-toe move within my first month here in U of xxxx. But still with limited homework scores, I got B still. With all these grades, I must be very stupid to progress well at all. And on 11/17/2014, fault tolerance, Dr. 代课老师 was saying some minds don't seem to turn around at all. According to all the facts happened during the pass more than two years, it must be especially stupid me that was not able to study well, nothing to do with the instructors, nothing to do with this university at all\textasciitilde{}

I have been brave to solve my technical problems, but now when I looked back to review this process working with you, and what has happened during the past two years, especially during this year, I felt frightened. I am a girl grown up from the countryside in a developing country with all the suffering during my childhood. I had all my out-of-state tuition fees waved for my Statistics master's degree. How could I ever imagine that I was going to suffer all these discrimination on me, good or bad, all Bs for my courses, separation from the rest majority of classmates and so on? Tortured by personalities like yours, I always suffered from feeling unsafe, and this unsafe feeling dragged my life miserable. If I know any of this detailed information, if we could rewind the time back, I would never want to step into this department ever again! And in the future, if I have any choice, I would not want to come back to this department environment again! 

Up to this point, I wonder how many professors in our department now are still dreaming that I would continue with a Ph.D in this department. We have two Chinese Ph.Ds, Xin and Jia, both of whom chose to stay here for Ph.D without even trying to work in industry by applying OPT. It's their personal choices to stay in UI for Ph.D without OPT after masters, but as clear at this point as we shall all reach and agree, it will never happen on me! I AM GRADUATING in Aug 2015\textasciitilde{}! No regret, no come back. 

I learned that all the instructors, Professor 代课老师, who is the only master's degree instructors here in our department; Dr. Soule, who doesn't necessarily qualify for a Assisted Professor here in UI before, was all completely brought up to be an associate armed with best students during the tenure tracking process, which is similarly happening on another assist professor right now as well . Since all those persons who gave me hard time (Dr. Soule blocked me from having an industry offer which company I have worked for earlier in spring semester by covering cooperative convolution in his EC class during critical decision time) are all appreciate to Dr. Alves-Foss, now I do begin to think if it is the truth that actually it has been Dr. Alves-Foss who designed and directed the whole process of blocking me here in U of xxxx. And also he will be "on war" for coming spring semester for cs210, I will stay tuned, sharp my eyes and ears to see what's Dr. Alves-Foss's opinion on this thing. Will he work hard to block me from graduation, from finding a job, or he may even help me find a job, or could simple just let me go? I look forward to see his response towards this whole thing. 

-- (me\textasciitilde{})

\chapter{跟进一下}
\label{sec-30}

前晚实在是太累了,不到十二点已经迷迷糊糊快睡着了,五六点钟午夜梦回大概稀里糊涂想了一个小时左右的心事吧,后来又睡着了,及到下午1:30醒来,居然是赶着时间踩着点儿去打工的,连早中饭都来不及吃,看来我真的是老了。。。。昨天晚上精神还比较好,跟进了一下站上的反馈,不看不打紧,大致看完,已经被不少网友的发散思维惊呆了。。。。

或许我根据亲身经历写出来的故事太支离破碎了,可这都什么时候了,还换男主角,开什么国际玩笑?相信到这时,也有不少网友也在同时跟进github上我项目的汇报进展了。正如学校食堂和LinkedIn上学校已经开始盯人跟着我同步更新,我不是一所学校,不是一个组织,我只是一个有点儿想法的苍白个体,光脚的不怕穿鞋的,就让暴风雨来得更猛列些吧。。。。项目更新的网页上是有写"Gosh, I missed the mentor so much, who as the technical lead knows programs and projects great, inspired and taught me sincerely and effectively during the internship season…."可根据上下文语境不难看出我是在什么语境下说出这样的话吧。。。。而且在我这里,早说过了,之前过得很飘呼,停浮在表面;12年重回学校读书对我来说是给自己的一次机会,这三年,自己真的是改变了很多。早说过愣是买买提风雨交加狂风大作"喜欢一个人看她在他面前笑得多开心就知道了",可现实里2013年8月14日生日那天,遇上与男闺密同样好性格的team member,冷不防蹦出一个冷幽默,我这傻大妞的性子就是不想笑也忍不住啊,平凡如你我的默默努力的人们又有多少能有精力关注到网上了动静了(除了现在我站出来写,已经要十天内写完时?)在我,我当然留恋那个夏天蓝天白云云淡风轻,一切回忆起来都是那么美好。。。。留恋那份真切的专业上的引导和启迪。可我内心里最深切最激情的感动来自于表哥,那场分别里我所感受到的唤醒生命的短暂恩宠让我坚信,我已经在对的时间里遇到了对的人,不管受多少委屈,我都愿意等。买买提上说研究称34岁是人生最幸福年度。从2010年十二月到现在,我从31岁到现在36岁,我已经等过了一个女人最美好最幸福的年度。有过经历的人该能懂得34, 35岁对一个女人意味着什么。

再回头说M。呵呵,不同于流浪者,不同于实习时的mentor,不同于我身边接触过的任何其它人,M在我眼里是美国身边一个"小侄儿"的存在,我甚至对他抱有自己亲侄儿般的满心期待。除了上初中那些年寒暑假放假有机会照顾一下大姐家自己的小侄儿,后来他长大后我并没有机会与他多接触。M与我我也只有去年秋天不到三个月与他共同在CSAC值班的时间以及senior design项目其它成员都在的组会时间,我对他的了解也并不多。我能肯定去年感恩节前在系里的诱导下他有试图给我些暗示(门上白板上的卡通漫画和接下来的很长时间不曾出现的帅哥与他在CSAC里聊天),但我要写自己的fault tolerant考试没有加入他们的聊天装傻直接秒过了。我对M不够了解,事实证明关键时刻M也并没有顺从系里真正全身心地搅和进去嘛。看来经后M的角色就只能现在进行时地写了,或者以后直接省去不再写。就算M有一万份真心,亲爱的读者,把你换到我的位置,你又会怎么做? 
毕竟假设的场景更多地是假设在自己的立场上,那我也只能换位思考表哥与自己的小侄儿。和M解释他92年出生时说的情况一样,小侄儿到如今也还没有谈过恋爱。假如有一天小侄儿领着一个大他13岁的女生回家了,我想家里一定炸开了锅。。。。

谁没有年轻过,过来人谁又没有初恋过,大学里的室友兼情敌以宿舍里对我动之以情,晓之以理,说暗恋与我现在不平等,将来不平等,两人的地位永远不平等,但那时的我何尝不是与她一样内心里真真切切地喜欢着同一个人。。。。最后一次做我工作试探后,她知道我这边绝不会让,她只好在接下来的课堂上一次又一次地将自已的长发盘起。。。。现在回忆起来真是忧伤,若把我今天的成熟按放回当时的肉身,我想一定会让,因为他俩是同喜欢体育更协调般配的一对,可我那时的真心又错了什么?

若那个场景真的发生,我想一定会在所有亲人里第一个站出来反对,我会以自己27前生活都过得很随意与小侄儿交换对爱情人生的看法。参考个体有不同,若小侄儿一定要坚持,退一万步说,我也会采取表哥的做法,拖上几年再说。。。。

几年前看过一遍网文,是一位网名hocc写的乱评大话西游。好文章常常有,能打动人心的有几篇?而这篇不偏不倚地砸中了我。。。。感动于作者白富美一介文臣对小人物恻隐之心和淡淡迷散的家国天下的情怀,这种真正走心的小文,比某些所谓的版主用文字谋生的穿凿文强的何止境界情怀(读到这篇我不曾冒泡,却对那篇作者感激之至,她的小文引导了我去认清自己)。昨晚看网友们的反馈,真是要把人看崩溃了。一口气写了这么多,想来我也是醉了。。。。

\chapter{系里大牛的回复}
\label{sec-31}
那封邮件,因为牵扯到系里,因为在我眼里与大牛有不少关第,所以我也CC给了大牛。周二发出那封邮件后,周三早上我收到了系里大牛的回复。内容如下:

\section{RE: footprint \& global agreement}
\label{sec-31-1}
系里大牛 (系里大牛@uxxxx.edu)
Sent:        Wednesday, November 19, 2014 9:16 AM
To:        
(me\textasciitilde{}) ((me\textasciitilde{})@xxxx.uxxxx.edu)
(Me\textasciitilde{}),

I am sorry you are having problems with your senior design team. However, your should not have sent that whole email to the group of people you sent it to. We are not involved in that.

Your other comments, some of which are hurtful, are not appropriate in this email.  

Personally I have not orchestrated anything against you and am bothered that you think this. I do not discuss your grades or performance with other faculty, I am not your major professor. I do not ever remember talking to an employer about you and do not plan to talk to any. I have not tried to block you from a TA position or done anything against you. Personally, I don't think about you except when I see you. 

Your studies, your thesis work, your graduation plan are between you and your major professor. Talk to him about your concerns and get guidance from him on how you can graduate.  Your ability to get good grades in classes, to finish your MS degree is up to you.

-系里大牛

相信到这个时候,任何明哲保身的人都会划清自己的界线,大牛并不举外,甚至大牛为大家列出一个腹黑神回复的模板。

也回忆一下大牛的几件经典事件吧。

2012年8月开学前当前导师把我领去找大牛征求选课计划时,大牛为我列的第一学期同样是cs150和cs121两门课共计7个学分;当我明确表求这样的选课计划自己压力太大,更想放弃后,他与我的导师一起松口,认为我是工作过的人,对自己的学习目标更为明确,可以多选课。前导师用他的手机把大牛白板上第一学期7个学分的两学年选课方案拍了下来。

2013年春天,我选了大牛的几门课小课,比如software engineer,算法课。与之前数学系的某位现在已毕业的中国女博士一起上课,我的两门课与博士一起一周见一次大牛,简要绘报做习题过程中遇到的困难,一次不到50分钟。某次我借机会问起毕业计划时,大牛说快的话13年秋天12月我就可以毕业。这个春天一门大牛的大课大牛想要整人似乎已经等不及,某次考试前临考试布置作业,后来以课堂上他向学生道歉结束。

2013年那年夏秋我大哭了三场,为许是为段时间看不见希望的暗无天日苦苦挣扎。秋天当学校一次又一次发起想要我读博士的运动后,我躲过哭过,但在我眼里,我亲眼见过别系的人去找大牛请教管理治理方案计划,亲眼看见系里小秘去征求大牛意见关于系里的奖学金,我相信大牛才是系里真正的大内总管,其它都是傀儡。于是我去找了大牛,询问来年春天的计划。大牛说我春季学期可以注册part time,并建议我换导师。当时我对换导师不明所以,但前导师那时已经碰过我的手,心理或许会对前导师稍微有些别扭,而且这是大牛建议的,也就同意了。大牛帮发给了我换导师的pdf表。

\section{Github Sample Demo codes}
\label{sec-31-2}
(me\textasciitilde{}) ((me\textasciitilde{})@xxxx.uxxxx.edu)
You replied on 11/20/2014 4:56 PM.
Sent:        Wednesday, November 19, 2014 10:50 AM
To:        
m (m@xxxx.uxxxx.edu); p (p@xxxx.uxxxx.edu); r [r@xxxx.uxxxx.edu]; e [e@xxxx.uxxxx.edu]
Hello Tower iLLuminati, 

I uploaded into github account about one hour ago the GUI clickable codes according to our Saturday's design review. I consider this one low priority task for me, so it isn't a COMPLETE one yet, but have main "clickable" functionalities. The update I made from last time includes: 

\begin{itemize}
\item Removed self-defined buttons.h and buttons.cpp file, used default mainwindow.h and mainwindow.cpp file instead because MainWindow inherited from QMainWindow satisfies all our requirements, the self-defined buttons inherited from QWidget can NOT satisfy all the requirements, like menubar and central layout;
\item Removed uncessary verticall*.h vertical*.cpp files;
\item Added menus according to our design on just passed Saturday;
\item Self added "close file" toolBar option, and just got accepted during today's meeting;
\item Removed .tan file open line menu from main layout;
\item Redesigned First/Play/Pause/Stop/Last commnad line (need slightly more work to combine start/pause though);
\end{itemize}

It was just a sample code, we could draw dramatically away from here. I just wonder if m would like to demo it to Dr. 我导师 in our client's meeting, but please don't feel pushed to at all. I sample tested in my machine, let me know if you have any question or concerns. 

I will try my best to make noon's meeting with Dr. 我导师. 

thanks,
(Me\textasciitilde{})
\chapter{系里的处置}
\label{sec-32}
那封邮件发出去后,所有的人里只有大牛给出了回复。如果说周三的时候大家----学校和系里还在商讨对策,周三中午不强制性的与senior design client的会议我努力去参加以获得最新信息,那个会上导师也不过是对team里的关系稍作建议警醒大家弄得不要太难看,那么经过这么一天的讨论他们基本也已经达成了一个有效对策。

周三晚上我收到学校IPO负责人打给我的电话和留言,留言说她听说我学习上有问题,非常关心我,希望我有什么想法可以今晚给她打电话或是周四早上去她办公室里去找她。我能有什么想法,但我总是很傻很天真,本能地去想IPO会不会有什么好的建议。于是周四一早就兴致高昂地去IPO了。相比于前一天晚上听到留言获得的感动,这份在IPO的谈话更像是一记警告,警告我我可能被学校开除,警告我春天you may not be here on campus。我的崛脾气到这时也上来了,警告就警告,我就不信学校会这么光天化日之下以这种理由借口把一个学生开除。

从IPO来到学校里,及至我到CSAC值班时,我已经收到导师的邮件,要我去代这天的实验之前先去找他。我立即就去导师办公室找导师,但他办公室的门是锁着的。而我再来到CSAC,是系里的小秘来到CSAC要带我去见楼下thinktank旁边一个心理医生。人世间最低谷最难受的莫过于从小长到大的第一次经历打击考验,而在亲人的陪伴搀扶下我走过了高考,我也已然把自己磨炼成一个经得起考验有韧性的人,我不需要飞越疯人院,我自认为自已足够正常理智,我能handle我的生活,我拒绝花费自己宝贵的时间去见任何心理医生。

程序走过了一道又一道,接下来会发生什么呢?经历了这个早上的奇变,我心理也难免犯疑。谁告诉我要去senior design课找代课老师的?上次不小心把邮件upload到github上后来系里好像也是知道了,是哪一方心里有鬼,从此开始拒绝使用邮件了?没关系,所以我去找senior design代课老师,系里的那个唯一的硕士生代课老师。老师我说被这个项目开除了。。。。我他妈的到底又做错了什么?呵呵,到这时终于明白这个学期这个senior design课的项目为什么会有诡异的team manager 和所谓的team contract了,欲加之罪, 何患无辞?老师说所以这门课接下来我需要独自一个人完成,我的任务是需要document出自己的设计,与之前其它team member不再有讨论来往。

回到CSAC里中国男博士已经告诉了我导师要他去代今天的实验,他要我把实验材料交给他。既然是要他代,那我就只需要把前一个周五例会EC代课老师发给TA们的邮件转给他就开以了。所以也大致知道接下来与导师会有什么谈话内容了。临近这天的实验前十分钟,代课老师来CSAC找到了我,去到他办公室里,同时系里小秘也跟进来了,像是一个第三方见证人坐在一把椅子上。

那天是11、20(下周感恩节学校放假一周),导师对我说我接下来的所有实验课会停下,但系里还是会把TA的钱给我。马上学期就要结束了,也就剩下最多只有两个实验,就算是任何其它的人什么人带,都不是什么大不了的事。我对导师坚持我可以自己完成TA lab的教课和改作业任务,我完全没有问题,但导师不准,说这是系里的决定。并对我说,系里正在考虑对我春天学期的处置,现在还没有结果,但结果出来后会告诉我。我追问一个有没有什么预期结果什么时候能出来,导师说他也不知道。小秘同时接着说,因为我不再是系里的TA (就是这么别你们拿走的?),所以我需要这周结束前归还系里我所占用一个座位的办公室钥匙。
\chapter{最后一次借钱的经历}
\label{sec-33}
\section{最后一次借钱的经历(1)}
\label{sec-33-1}
昨天状态不好,写得深一脚浅一脚的,今天不写悲愤题材,写篇相对平和自然无关大雅的吧。

到寒假我所有的TA奖学金发给我 之后,因为寒假12、27日傍晚发生的意外,\$500也进去了。去年年末秋天的时候导师说春天我只注册一个学分,学费大概是\$1100(后来实际交了\$1300,这是到了开学第一天才知道的。) 1月份的房租我告诉房东"I am broken",把箱子里仅有的\$150现金先拿给她,请她给我宽限几天,等我周转过来了再给她。

到这时我手上几乎就一分钱不剩了,银行里checking账户已经清零,我连吃饭的钱也没有了,一度把箱子里桌子上包包里能找到的硬币全拿去买菜。有一次手上大概有\$8块多的硬币,我估算着买菜,但毕竟是生活需要一不小心就还是买多了,于是结账知道自己钱不够的时候再慢慢退,退去一条\$1的面包,退到自己能买的额度。后来用塑料袋装东西时结算员告诉我有人帮我付了这个面包的钱,我抬头望去,后面跟的是一对中国人男女朋友。当时没好意思说什么,从店里出来后眼泪就情不自禁落下来。

那些都是小事,等春天开学打工,忍忍也就过去了,可我开学第一天的学费该怎么办?我要向姐姐借钱。

电话打给一向与自己最亲的二姐, 那个我来美一年后得知我还生活地过去、电话里狂轰乱炸把自己从痛苦中骂醒的二姐。

电话打到姐姐那边,简要地说了一下自己目前的状况,姐姐先问我从这边朋友那儿借不到吗?借不到。然后姐姐接着批评我说,"你在那边生存不下去,是你能力不足;是你一直太不现实,一直奈在校园里;你不需要再为自己找借口,在那边生活不下去,那就回来;中国十三亿人都生活得了,就你一个人回来就过不了了?"

"呵呵,二姐,你还是那么语不惊人死不休哦?我大学的时候你反击我说天下所有恋爱中的人都觉得他们两个才是这个世界上最般配最幸福的人;现在你又成功概括了古今中外英雄成败的永远不会错的原因。那这个世界上还要亲情作什么?"

"这也是立场问题。你在国内的时候考托福GRE没钱,我们说你还小,实在没钱作姐姐的能借你点就再借给你点儿;现在,你已经是一个36岁的成人,我们都有自己的小家庭,这么多年来我们没有计较你没有要求你给爸妈养老,我们不愿意借钱给你的时候,你有什么资格强迫我们借钱给你?"

"是,我当然没有资格强迫任何人做任何事。只是从今以后,我只对妈有赡养的义务,对你这样的姐姐,我不需要再认了,以后我们也不需要再来往了。。。。"

"不用再说了,我这里忙,挂了。"随即姐姐就挂断了电话。
\section{最后一次借钱的经历(2)}
\label{sec-33-2}
这哪里是我认识的二姐嘛,是从什么时候起,二姐开始变成这样的?上次13年夏秋我大哭三场 的时候二姐不还说即使卖老家的房子,也给帮我完成学业的吗,虽然她不是大姐作不了卖老家爸妈房子的主?

二姐也几个姐姐里精神上与自己最亲近的人,我也把电话打给大姐三姐过,大姐说她在市里买房没多久,还装修,背了一身的债,一分多的钱也没有可以借给我的;三姐说现在大家都缺钱,连舅舅家我们的亲表哥都在一圈地借钱。三姐反问我说我该知道她一向是几姊妹里最穷的,实在没有能拿得出手可以借给我的钱。

这么多年我不曾回过家,我与家里与亲人们之间的隔核已经这么深了?我不死心,过了一两天又打电话给二姐。

电话里,鲜有的一次,姐姐叫了一遍我的学名说,"XXX,你还真是冷血,一句话抛出来能把人给噎死。我们三个姐姐都不到20岁就参加工作了,20岁都结婚了,这么多年来我们照顾父母,各自的小家庭的日子也都过得好好的。我们就想不明白,你怎么能一拖都拖到36岁了还没有结婚?说你一个人单身应该能攒点儿钱过得还不错吧,到头来快40岁的人了还在问姐姐们借钱,你说你怎么就混成了现在这个样子呢?你看你如果三年前就回来了,你自己工作过攒得还有点儿钱,那时回来也还能再年轻几岁,谈个恋爱成个家,现在不是也该过得好好的?我们都想不出来,三年过去了,你还在问姐姐们借钱。我们不借钱给你吧,你说姐姐狠心只得认钱;我们把钱借给你吧,我们实在是在放纵你骄纵任性,放纵里认不清现实,借钱给你实在是在把你往火炕里推害了你。采取对你负责任的态度,我们所有亲人没有人愿意借给你一分钱。"。。。。

同样的电话我也有打给远在加州的远亲大表姐(四代远亲)。可以想像,同样类似我怎么混成这样的话也从大表姐的口中说出。虽然之前大表姐圣诞节度假后从加拿大回加州时曾说过借我\$1000块钱,不要我还,只是这是最后一次帮我,也算是对我仁至义尽,要我把银行账号发信息给她,表姐当时在转机机场,说是一到加州就打给我。但我把我银行账户信息发给她了,开学前我查账户迟迟没钱到账。等我再把电话打给表姐时,大表姐说她没有可以(应该是不愿意)借给我的钱,要我学费、生活费呀什么的问朋友和自己的亲姐姐们借;只是若到最后迫不得已,如果这个夏天我不得不回国的时候,我可以打电话给她,表姐电话里说她可以给我出一张回国的机票钱。所以到目前为止,我也还是没能从大表姐那里借到一分钱。

那次给自己亲二姐的电话我好歹算是明白了姐姐的苦心,我也一向相信至少二姐,与自己精神上最相通的姐姐,绝不是只认钱的人。那次的电话里还是忍不住哭了很久。姐姐说我一个人如果一定不会过日子,连吃饭的钱也没有,那当姐姐的又怎么真的可能看着我挨饿受冻不管我呢?只是我要立足现实,不能再抱任何幻想,要踏踏实实本本份份地过日子。

如今,从2012年爸爸离去不久妈妈脑溢血发病,到今年正月一直在三个姐姐那里住的妈妈已经独自一人回到了农村老家,我也尽量在每次告诉她我下次什么时候打电话稍早的日子打电话给她,好让她知道我心里总是想着她的,让独自留在家里的老妈妈不用挂念我。

二姐说的又有什么错呢?生活的选择让我一步一步走到今天,作为一个36岁的成人,我又有什么资格向有各自小家庭的姐姐们借钱?别人肯借钱给我是恩情,别人不肯借钱也是别人的本分,我实在没有任何资格伸手要钱。如果说三年前看不清这一切,我还傻到想回学校去读个计算机硕士,那么现在,我一定不会再去读什么狗屁屁挨着地。作为穷人没有任何更好选择的时候,我愿意选择生活最本真的路,正像电话里我给姐姐保证过的,爱情里生离死别也罢,在这边没有留下的理由,我便回去,不跳楼不跳桥,不缺胳膊不少腿儿地带着自己生病的身体平平安安地回去。。。。

投胎真是个技术活,不得不承认买买堤上的坑王们概括得还真是经典,叹为观止,不服不行。。。。
\chapter{后来}
\label{sec-34}
\section{后来(1)}
\label{sec-34-1}
周四的白天接下来发生了什么我已经不大记得了。从2012年8月到现在,用同一间office的同一座位已经两年多了,这里堆放了我各种的作业教材,流浪者的笔记本、显示器、打印机、微波炉、咖啡水杯等,我需要这个晚上把车开过来,把自己两年来积攒下来的所有这些学习材料和杂物拖回去。

可能那天晚上早早地就用过晚餐了吧,我把车开来walmart里,找来找去也没有找到有卖玻璃的地方,在电子品区前正好看见有两个工作人员在聊天,就稍等,想打听一下这家店到底有没有卖的。他俩一个机巧些,说这个店里没有卖玻璃的,倒是到旁边小镇可能有课桌大小差不多一百多块钱吧,一边说还一边像是给同伴使眼色。都这份上了,XXXX的,就算是值15 吨黄金,我也把它还回去!后来我自己再转悠,找到一种塑料作为替代,便回到系里。

来到系里我还是没有概念木木的。在那里学习了两三个小时,差不多十点多钟,预计着等收拾好也该是我平常晚上回家的时候了,便去车里搬我曾经借用过的玻璃。可是等到我把玻璃拿到系门口,我的门口却刷不进门,这时我才意识到,我应该是被系里block了,我周末晚上不再能刷开进系里了。好在那会儿随身带了办公室的钥匙,回家还是有钥匙的。

这样办公室里的灯都还亮着,所有书包里的随身用品钱包手机等那晚都留在了办公室里。等到第二天一早,七点多钟,我再次开车来到系里,把曾经因为自己意识不够、借用系里自己办公桌上的玻璃再毫发无损、完壁归赵地还回去,把自己办公室里所以私人物品全拿回去。

周五上午,等我把系里办公室里原配老旧台式机里我的写过的作业全拷出来,该归还给流浪者的两条网线都还给他。我有问过系里小秘系里的办公室座位我常常用,晚上和周末的时间都在这里学习,我可不可以保留办公室的座位,小秘不同意,小秘也确认了即使作为研究生,不是系里的TA,我周本也不可能有access去系里。并且小秘说要我尽快还系里的钥匙,那我再去看了一遍办公室我没有落下什么东西就还给她了。我问系里IT小胖,他能否在CSAC里摆一个大monitor,这样我们学生可以连自己的电脑。后来小胖与系里讨论后CSAC里我个儿矮想用的矮桌子全放系里的电脑了,剩下一排的高桌子摆在墙的一侧;我想用的monitor小胖后来告诉我说系里暂时没有,可能得到明年夏秋天才能有。呵呵,不说我也知道,系里没收我办公室就是要剥夺我的有效学习环境呢! 

那个周五回家的路上比较悲怆,我猛力地摇头,想要摇醒自己,回想一下,senior design为什么会给我安排GUI的活儿?我的毕业项目同样是使用Qtcreatorc++,早早地就在开始做,为什么我一直不曾做出来呢?这个学期senior design的项目更像是一个就地正法的刑场生死考验,我为什么就做出来了呢?我知道自己受益于自已专业里第一次享有的真正意义上的team environment(与之前已毕业数学博士生的两外的team不算),它让人的自尊自信自强全方面能力都得以激发。而就是这样一个处处受刑的team environment也还是被系里无情地剥夺走了。我像一个走投无路的孩子,一个被剥夺得一无所有的人,还想从他身上挣钱,去死去作梦吧。。。。走到今天的境地,起来也不禁悲怆,安慰自己说,就像剑法里练剑到一定的程度就能有悟性,经历了这个秋天senior design的炼狱,我已经拥有了自学能力和悟性,以后我就天不怕地不怕了。。。。
\section{后来(2)}
\label{sec-34-2}
(昨天写的时候写忘了一小部分。) 周四上午,得知自己自己被senior design项目代课老师从team里移开,我还是洋洋洒洒地写了封邮件给组里的人,与他们正式告别。别人回不回是别人的事,我写出去是我自己与他们共做一个项目一段时间的一份情义。邮件如下:
\subsection{farewell}
\label{sec-34-2-1}
(me\textasciitilde{}) ((me\textasciitilde{})@xxxx.uxxxx.edu)
Sent:        Thursday, November 20, 2014 4:56 PM
To:        
m (m@xxxx.uxxxx.edu); p (p@xxxx.uxxxx.edu); r [r@xxxx.uxxxx.edu]; e [e@xxxx.uxxxx.edu]; (me\textasciitilde{}).huang2010@gmail.com
Hello Tower iLLuminati, 

Every party has to come to an end, and here it comes. 

I have been much elder than you, and I have had way more experiences than you in industry, and that may be some reason that towrads my work and priject, sometimes I was way too restricted, though I tried my best to understand that you guys have compilers. 

Since I doesn't fit the team and your culture that well, right now I was asked to design the project independent from you guys, and as far as the department is concerned, I am completely ok with it. During this period of time, I am not allowed to talk to our client than expected, I am not allowed to ask more detailed questions with you guys, unless I am going to ask questions when I have difficulties as I am a CS majored students, and I have the right to ask questions towards you guys whoever works in CSAC.

I would still say that I have learned a lot from you guys, like try to communicate in a business manner. I appreciate the opportunity that I have worked with you all together for a while. It will be memorable days for my life. 

thanks and take care, 
(me\textasciitilde{}) 

周三周四两天,出出进进大牛办公室的人很多,大概他们地商讨对策吧。哦,大牛的那封划清界线的邮件我没有回。什么时候是大牛早已手痒痒想要亲自动手办人的时候,到底是谁决定系里的奖学金,是谁是系里的真正大内主管,以我的观察和见识,我一直知道是系里的大牛。所以,当周四看到情势变得超呼我的预想的时候,我便要找机会见到大牛,因着大牛的时间比较紧,又在节骨眼上,大牛就他的时间与我短暂聊了几句。

聊天里,大牛说我写邮件显得正式了点儿。又 转口说这是系里的事,与他无关。但看我着急,又 建议我去找college of engineering的学生办公室还是校学生办公室,反正大牛给我指的是学校行政楼主楼的(极有可能是)校学生工作办公室。

大牛是系里真正的大内总管,按照大牛建议的方法去做当然是最有效的解决办法。可怎么找校学生办公室呢?当然还是老办法,先写封信过去约个时间吧。因为写封邮件过去,我有更多的主动性,我可以把发生过的事情条理理清楚,写进邮件里,这样省去我作为一个国际学生因为语言的障碍临场表达时可能会有的更多遗露的部分。

Senior design的代课老师早在11、11我们项目design review那天就说了他会access我们的邮件,我嘴上答着,"I have no problem with that at all",心里想着,你是今天才开始有access  to our emails的吗,还是我们的邮件你早看过上百封了? 

我周五早上一来到CSAC就找了个最矮的桌子在那里写呀写,好几个小时过去了,小胖到CSAC来了又走了,来了一遍又一遍。。。。想来,这是一封长长的邮件,或许还经过艺术加工,若真正发出去,若再一次地upload到github上去,势必会给秘里造成一定的伤害\textasciitilde{}?所以周五下午,我的正在写的、打算写给学校学生办公室的邮件还没有来得及真正发出去,我的导师就来CSAC里找到了我,告诉我我春季学期我可以注册part time student,注册一个他的课学分来做自己尚未完成的课题项目。

春季,自己的最后一个学期,我没有任何拿或是去争TA的欲望愿望,我只想找份工作,安安静静地离开这个地方。想来,这应该也就是自己一直想要的答案、想要的global agreement了。回到家后便把这个确证信息upload到了github上。
\chapter{半决赛}
\label{sec-35}
\section{半决赛(1)}
\label{sec-35-1}
虽然senior design项目我被系里和代课老师人为给我填加设置障碍,从组里移开了,但是既然代课老师要我去展开自己的设计,去独自完成原本为整个组负责的项目design review documentation,那么即便是作为个人,我也还是想要遵守整个课程对项目的规定的。12月5日周五上午是第二次snapshot day,虽然我可能不会去参加参观他们的项目展,但我一定会github upload上我个人对项目的设计和相关进展的,倒是desing review documentation没有要求那天交(这个最终的deadline后来被代课老师设的是这个学期的最后一天)。而早在11、20周四当代课老师要求我独自进行的时候,那时为所有的课程所要求、google drive上我们项目所拥有的文档我都早就备份到github上了,这样所有的人都可以监督项目进展,是非对错我相信人们心中自有评价。

那周四被划出来单飞时,我对代课老师强调过,项目design documentation是整个组负责的内容,而且E早就试探过我的意思,这是作为国际学生的我在这个项目里在组里时我申明过我最不愿意干的活儿,现在安排我一个人设计自已的design,并且完成这suppose应该全组成员共同完成的review documentation,你们系里和作为代课老师的你,对我要求是否太高了点儿?代课老师说,没有办法,"think about it."我继续反问代课老师说,既然是think about it,那我不愿意这样的选择,我更愿意继续以前的team environment去fight for my grades,可以吗?代课老师说,虽然他说的是think about it,但实际上我没有任何其它路可选,我必须走这唯一一条(系里、代课老师他们)已然为我安排好的路\textasciitilde{}~??!!

哈哈,歪楼一下,借用网友们的话说,XXX教导我们,"马上、枕上、厕上";总理教导我们,工作来不及干的话,带到"厕所"继续干! 学校学生办公室里的学习座位没有了,学校里系楼晚上周末的access被deny了,回到家里还是需要继续学习的。人或许真有第六感吧,鬼才知道这个八月自己刚从加州回来时哪根神经搭错了,居然花了25美元大刀从小镇郊外的一个农场里先前养鸡的窖楼里刨出来一张大桌面的课桌,我如实告诉卖主我从网上看了那个我喜欢的桌子他要\$20只可惜被别人先我一抢走了,那农场主也很好说话地说那就收我\$25吧,因为桌面稍大我的车装不下,他会用他的大卡帮我送回来,所以多收\$5汽油费。这是几年里我买过的最贵的桌子了。现在这个桌子居然就得要真真切切派上用场、发挥作用为我所用了。用了几天的(从walmat买回来的)塑料表面,可每当放热咖啡杯的时候我就开始担心会把塑料给烫坏了,还是十分怀念、想要一块玻璃的。从网上搜了搜,网上卖的最便宜不带运费42"x30"要\$42,但可以想像得出玻璃作为易碎品运费非常贵,并且那时搜local玻璃店,一不小心顺着link又点进了facebook。。。。从什么时候起为什么facebook成为了我的禁忌?后来从家隔壁不远的二手店刨出两片稍有磕碰碎边角的橱柜里的挡风玻璃,正合我意,花了不到\$5。我就像那想像中结婚多年不曾生育的一位养母抱着一个刚刚领养到的新生孩子,傍晚走在结冰的路面上,小心翼翼地把它们抱回家,平铺到桌面上,尺寸大概是44"x29",非常不错了。这是这个感恩节给自己找到的最好的礼物了。这时我老朽得常常不开翘的耳朵开始翁翁作响,山不在高,有仙则名。水不在深,有龙则灵。斯是陋室,惟吾德馨。无丝竹之声乱耳。。。。
\section{半决赛(2)}
\label{sec-35-2}
虽然家里的网速不能和在办公室里时的比,但现在我已经有了非常好的学习环境了。感恩节前senior design的代课老师说是让我完成了一个documentation的初稿,感恩节后拿给他review一遍。代课老师这么要求我的时候,就没有再给我完成我自己个人的design的时间?这个学期的经验告诉自己,但凡E、R和代课老师并不真正愿意希望我去触碰的,我就更应该跑得更快,飞得更高,把那一块完成得更好。

所以这个感恩节期间,我便一直呆在家里试着去适应几年来不曾努力过的在家学习。把之前11、16M领导的menu bar的design部分全部完成,把11、19我们与client也就是我导师meet时要求的popup window也未完成;11、16时组里讨论过,说我左侧的工作窗口太小,不该只是12*6,应该改成至少14*8,但大家也都更偏向于再多大些以方便我们的快捷键操作,不用老是需要用鼠标点来点去的,于是我也把它们改成20*12, 这样8个方向的移动会更有操作潜能。至于命令行、窗口的layout等,都是些小改,这些对这时早已完成了主GUI的我来说,实在是a piece of cake,不在话下。

我似乎还是不太急,慢幽幽地做着自己的活儿。等感恩节一过,接着开学的周五就是第二次snapshot day,我总觉得自己时间够用,但真正到这一天来临,还是有着诸多的工作、诸我的愿望我是想要加在这个项目里的,可似乎时间已经有些来不及了。

大致看了一下大家的反馈,大致意思是走位飘乎,带着浓郁的初生婴儿奶水味,流着一滩的口水。。。。希望自己能在把握好大主题的前题下,行文方面能尽量能接近大家的品味。
\section{半决赛(3)}
\label{sec-35-3}
感恩节后,出于对家的环境还不太适应,也希望在学校CSAC里有同学环挠的环境里学习效率能高点儿,那个秋天,直到这个学期的最后一周的最后几天,我才不再到系里出没。

感恩节后第一周的某天,周二还是周三的下午,我坐在CSAC里一个座位上,连了那个座位的monitor键盘等在写senior design的项目,流浪者不知道从哪里冒出来了。那时距感恩节前周四周五劫后逃生尚不到两周,心有余悸,我都不大敢跟多跟CSAC里的学生们多说话,流浪者居然主动走到我座位前与我打招呼。

我懒得理他,也一向直率,就直接对他说,"I dare NOT talk to you any more."前向他解释说,"When you asked my interest about your office, I suffered by having lost my office already\textasciitilde{}!"换句话说,流浪者就像那乌鸦嘴,与他聊天的关于我的内容,在接下来的日子我保准倒霉\textasciitilde{}!呵呵,是考验人的极限承受能力吗?流浪者接着对我说,"I don't know if you will be here on campus in the coming spring semester, when do you plan to return my laptop and monitor and other minor related staff?"我直接告诉流浪者说,春天我会在这里选一个学分,不会离开校园,但我毕业之前一定把所以借用他的东西还给他,并会按照之前租用的惯例给他相应的租金的。他这才离开。

在这场秋天的suffering里,买买堤与自己基本是共进退的,甚至当TA老大整出个curses window时,买买提一度提醒自己防犯意思要加强。流浪者身份特殊,这场与流浪者的谈话之后,买买提刮大风说留得青山在,不怕没柴烧,大致要我避开锋芒,再作打算。但我了解自己,把时间看得很珍贵。他们旁敲侧击地作了一两年的工作想要我读屁挨着地,我不曾动摇。没有退让,这注定是一场你死我活的殊死较量。我绝不会退,所以周四傍晚便按原计划把一部分的材料upload到github上。但upload之后,看见专业上前辈们的反馈就知道,作为国际学生,似乎我做得还远远不够,我明白我还需要再加一把火,打算当天晚上再好好做几个小时增加些内容。
\section{半决赛(4)}
\label{sec-35-4}
人与环境的互动,尤其是在与系里进行了两三年艰苦卓绝的读与不读屁挨着地的斗争只剩这最后一段的时间,我个人的历史已然是步入系里国际学生历史上前无古人,后无来者,进一步万丈深渊,退一步万劫不复的境地。殊不知,作为一个小小个人,这正是凤凰涅磐的过程呢。没有这样艰苦卓绝的生死考验,如我般一直逃避现实的人又怎么可能去面对现实?没有这样一个过程还去妄谈什么成长,岂不笑话?

作为浩瀚宇宙星辰中的一粒尘埃,我对天发誓,我会越挫越勇;而系里也大可不必过于嚣张,谁又不懂得,再深的感情也会有消失殆尽的一天,为什么我不忍心回答前面狐帅的回复,谁又不能感觉到,离别的日子即将到来躲不开?就像买买提可以成就一个人的名声,可以翻手为云,也必然可以覆手为雨,将之毁之于一旦,易如反掌;这所建校125周年的学校想要拍死一个学生何等容易。强权之下高压抵制是一种力量,是非对错人们心中也必将自有一杆天平。这样一部反抗压制的小人物的成长史,也终将在人们心底留下感动和涟漪,人们心中的史诗也必将开出花来,这也终将间接影响到一代人的去留选择。面对系里一再崩出的恐吓、倒乱,内心里总有一个声音告诉自己:你是卑微的,你也是勇敢的,不用怕!

到这个snapshot day前一天的晚上, documentation已经准备了相当一部分。到这次真正看了这门课对design review documentation的具体要求才知道,这是极具Uxxxx特色的一门课,整个engineering学院的学生一起大教室上课(借以更好的传播这个国度深入骨髓的不平等文化?),一门课有五六个代课老师伦流上台讲,对课堂上学生的整体把握有着严重的掌控。这个design review docs也是极具Uxxxx特色的design review documentation的要求,绝对不同于传统design review的材料要求,要求的内容更多地反映了这个学校不平等文化的传播途径。

Docs不是这个晚上的重点,我心里一直在思考着一个问题,我到底要怎么做才能让自己的项目设计得更合理、科学、周全呢?这种大项目的设计是我目前为止的专业"成长史"上还不曾真正经历过的,而这个晚上我得过了这一鬼门关。
\section{半决赛(5):鬼门关}
\label{sec-35-5}
那年暑假实习时作过的项目,与现在作的比起来,也还都是一个一个的小项目。那么相比于现在GUI我所实现的部分,我又到底还需要哪些内容才能让自己的项目真正丰满起来呢?

Design report outline/components列举了这个学期docs要求的所有内容,开头部分的cover page, table of contents, executive summary,report body中的background, problem definition, project plan,以及结尾部分的future work, appendices贴些图表啊什么的都还算正常;可是,这个学期这门大课的要求里,重大版块的结构不完整,只包括了要求两页纸的system architecture,占总共10页纸的20\%,而没有包括传统意义上的data design 和component design等;部分版块的要求轻重失衡,比如report body中concept considered 居然要求占用3页,这至少对国际学生的我会是很大的考验;而要求居然也鬼使神差地多出个concept selection版块来,感觉这是给11、16提出另一种方案的E加分的。

后来我自己的docs里填加了data design和component design两个大的版块。System architecture 里填加了high level model项目几大重要功能版块的折分连接图,也填加了decomposition description每个功能版块的重要组成部分等的图和文字描述。Data design里加了data dictionary,component design顾名思义,简要介绍了那些个最基本元素的components。Human interface design添加了自己在github上update了无数次的GUI主界面和popup window的界面。而背后的功能设计,class diagram,是自已这个晚上要死要活的重中之重,因为,linux下我还不知道该怎么画图。

所以想来想去,从网上搜如何装dia -- 一个linux下的图形工具软件.装的难度是最小的,问题是我还不知道怎么用。。。。找到class diagram到底该怎么画了吧,这个项目背后的几大重要功能到底该如何连起来,这时我才意识到我从来不曾真正参与到这个项目的设计中来,11、4最重要的项目设计组会他们跳过我了;11、16的design review五个小时我们讨论的仍旧是皮毛,只占据我全部实现的GUI的一个menu bar,而全部menu bar的内容,11、16那天E所记载的所有文字在我的docs里也只占半页纸不到的一个table,背后的功能, class diagram是11、4日组会上M帮R设计的,是R和E负责的版块,后来E直接把那天的几个header file组成了一个class diagram,我们,作为一个完整的组,从来就不曾真正讨论过这背后的功能衔接,而11、16日那天我们该讨论的时候是M领导讨论了menu bar,连吃中饭休息带玩,五个小时后,是M有事要走人的,而现在我得设计这个。。。。这,是造化弄人?

不怕不怕,临时抱佛脚我也要整个有模有样的出来!于是用新装新学的dia, 根据自己实现GUI已经用到的类,和M和R先前有的半残品header file,开始有模有样的设计起画起自己的class diagram来。当然,因为是第一次用dia,一时没闹明白到底该怎么画相互关联线,就是一根直线两端长两个箭头,我把距离线都整上去了(就是两端都有箭头,但中间的直线上带着个2.02 cm的距离标示),如此庞然大物,我后来因为没时间也把它们upload到github上去了,很是搞笑。

等这个基本做完,再画两个form-based GUI(11、16号R用M画了一个,可R从来不曾把他答应发送的任何图表内容发送给我过)作为concept selection的填充内容,好给11、16号那天的review给E加几分;等这我能想到的所有的一切一切作完,感觉自己作出来的还就是一个普普通通,放眼望过去,与别人与组里原来的成员也好不了多少嘛。。。。

几个小时过去了,那会儿应该已经凌晨两三点钟,可看着自己的设计、完成的内容,仍旧是感觉失望大过希望,我一定还需要一些"constructive"的提升,一定需要。。。。

想来想去,我想到用软件的人,既然它是用来设计tower light时的一个一个frame,而这些个frame是与音乐同步的,那每个frame的开始时间、持续时间一定是要参照music文件的。那如果要我们的项目软件真正达到client所说的好用的要求,其实我们是需要整合部分music file里(比如.wav file)的信息的,比如seekslider timestamp,比如背影音乐的音高音频时间图。这才是作为单飞的自己的创意,我应该把这个写进我的文档里,并把seekslider timestamp和这个背影音乐的音高音频时间图用google搜到的一张图替代,虽然很费了些周折google到底如何在mainwindow插入一张大尺寸的图片,但最终还是把它成功放入到了我的主GUI里一并放入文档。

到此时,我对自己还是比较满意的。只是一看时间已经是早上八点,这门课上的同学们应该已经在指定的地方布置他们的展板了。我赶快写自己senior design的readme,把这个忙了一夜的所有的进展upload 到github上去(到这次,基本是这个项目我的最后一次更新了,只是后来期末最后一周我还是再更新了最后一次,大概是放最终的docs终极版22页纸的pdf文档吧),并去他们的项目稍微简短地转了一圈。我看得不太仔细,没有看见我们项目里的人,在一两个自己比较有兴趣的项目上稍停下,我也就回家休息了。根据下午CSAC里同学们的眼光反应,在这所建校125周年,排名200左右的破烂学校,我做得还是相当不错的。后来周一课上fault tolerant的代课老师为本校学生说话给同学们洗脑,找机会插话说,一个项目保留最少的内容,达到能够function的目的就是最好的,其它任何看似好用的信息都是累赘,我也只好在心里"呵呵"了。

\chapter{一个单词引发的血案}
\label{sec-36}
\section{一个单词引发的血案(1)}
\label{sec-36-1}
正如前面介绍过了,这个学期我选了一门先前programming language代课老师教的android app programming。想来代课老师教这门课也是想经弘扬一下这综合排合200的破烂学校好歹也还是有一两门能跟得上时代的课的。可教学质量和上课进度我也就只能呵呵了。

这门课开学的前两三周都没有开课,后来才确认只有两个学生。与senior design一样,大家都有了用github的意念和项目要求。好在,就在这个刚过去的暑假早些时候,我也早就申请了一个账户,磨手霍霍,跃跃欲试。

这门前半学期上得平淡无奇,上课的内容虽然是围绕着仿造一个windows下paint的app,但上课cover的内容更是东一榔头,西一棒槌,来无影去无踪般浅尝辄止,所学所得之物聊聊。。。。更加上我这门课的同学是一个大三本科生,别人确实生来年轻,可代课老师经意不经意留露的偏袒也让人心生反感。如果说这个时候的自己还是麻木的,那一次因为自己状态不佳 没有上课引发的意外,终于引起了自己的思考,我不需要这么被动地被老师、同学牵伴,我完全可以自己主导课程、项目的进展,于是思虑一两周后,我开始动手做课程所要求的项目,严格围绕既定目标坚持下去。其实从上面为一个项目、为给一个项目增加点儿自己的创意,可以在午夜两三点钟的时候再继续熬个通宵也要把自己那点儿创意给真正作出来,已然可以猜测,没看出来,我其实还算是一个走心的人呀,同大表姐一样,在节骨眼儿上,是不会退缩,一定会坚持到底的。所有这门课,后半学期的表现,在github上上线upload后,在买买提上前辈的鼓舞下,就比前半学期风声水起多了。

正如一贯的惯例,当我把最后的老师要求写的一个小docs的反馈也拿出来后,便也一如既往地upload到github上去了。

这天傍晚,我们系里的学生收到了系里main office小秘的邮件,邮件内容如下:

\subsection{[Csgrads] Employment Opportunity}
\label{sec-36-1-1}
csgrads-bounces@uxxxx.edu [csgrads-bounces@uxxxx.edu] on behalf of 系里小秘 (系里小秘@uxxxx.edu) [系里小秘@uxxxx.edu]
You replied on 12/12/2014 10:35 PM.
Sent:        Friday, December 12, 2014 4:18 PM
To:        
Mailman - csugrads@uxxxx.edu; Mailman - csgrads@uxxxx.edu
Cc:        
Donohoe, Gregory (gdonohoe@uxxxx.edu)
Attachments:        
ATT00001.txt‎ (450 B‎)

An exciting employment opportunity right here at the University of XXXX!

Are you a current student in the Computer Science Department? 

Looking for a great way to boost your resume?

Are you looking for a part time job for next semester? One that won't distract from your studies? 

The Computer Science Department is currently seeking students to fill Teaching Assistant positions for Spring 2015. 

These are PAID positions available to qualified students and range up to 20 hours per week. 

Interested? 

Please email the Computer Science Department's Student Coordinator, 系里小秘, with you resume and stated Interest. 
\subsection{RE: Employment Opportunity}
\label{sec-36-1-2}
(me\textasciitilde{}) ((me\textasciitilde{})@xxxx.uxxxx.edu)
Sent:        Friday, December 12, 2014 10:35 PM
To:        
系里小秘 (系里小秘@uxxxx.edu)
Attachments:        
resume\textunderscore Jenny Huang.pdf‎ (20 KB‎)[Preview on web]
Dear 系里小秘, 

I am so glad that the department still has position available for the spring semester. I guess I missed the opportunity on the email you sent out on about 11/5/2014. But as far as we still have opening, I am writing to state that I am interested and ask for putting me into consideration as well. 

I am a graduate student and my adviser suggested me to register 1 credit for spring semester. But I am planning to talk to him on Monday. I would appreciate this opportunity a lot if I can get it. 

So far please put me into consideration as well. I will talk to my adviser - Dr. 我导师 on Monday, and I will talk to you as well on Monday afterwards. 

This is really good news for us, and thank you so much!

(Me\textasciitilde{})
\subsection{From: 系里小秘 (系里小秘@uxxxx.edu)}
\label{sec-36-1-3}
Sent: Saturday, December 13, 2014 1:04 PM
To: (me\textasciitilde{}) ((me\textasciitilde{})@xxxx.uxxxx.edu)
Cc: 我导师, Robert (rinker@uxxxx.edu); Donohoe, Gregory (gdonohoe@uxxxx.edu)
Subject: RE: Employment Opportunity

(Me\textasciitilde{}),

I apologize for the confusion. This was a generic email that was sent out to all current CS students. As we discussed in Dr. 我导师's office, we are not able to offer you a position next semester. I wish you the best of luck in your studies.

系里小秘
Student Coordinator
University of XXXX
Computer Sciences Department

Sent from mobile
\subsection{RE: Employment Opportunity}
\label{sec-36-1-4}
(me\textasciitilde{}) ((me\textasciitilde{})@xxxx.uxxxx.edu)
Sent:        Saturday, December 13, 2014 1:10 PM
To:        
系里小秘 (系里小秘@uxxxx.edu)
Hi 系里小秘, 

I completely understand this result. Thanks for replying fast and let me know this. thanks for your wishes. 

Thanks,
(Me\textasciitilde{})

\subsection{[Csgrads] TA Positions Still Available!}
\label{sec-36-1-5}
csgrads-bounces@uxxxx.edu [csgrads-bounces@uxxxx.edu] on behalf of 系里小秘 (系里小秘@uxxxx.edu) [系里小秘@uxxxx.edu]
Sent:        Tuesday, January 13, 2015 8:21 AM
To:        
Mailman - csugrads@uxxxx.edu; Mailman - csgrads@uxxxx.edu
Cc:        
Branting, Susan (sueb@uxxxx.edu)
Attachments:        
ATT00001.txt‎ (450 B‎)
It’s not too late! An exciting employment opportunity right here at the University of XXXX!

Are you a current student in the Computer Science Department? 

Looking for a great way to boost your resume?

Are you looking for a part time job for next semester? One that won't distract from your studies? 

The Computer Science Department is currently seeking students to fill Teaching Assistant positions for Spring 2015. 

These are PAID positions available to qualified students and range up to 20 hours per week. 

Interested? 

Please email the Computer Science Department's Student Coordinator, 系里小秘, with your resume and stated Interest by January 19!

系里小秘
Student Coordinator
University of XXXX
Computer Science Department

系里小秘@uxxxx.edu
\section{一个单词引发的血案(2)}
\label{sec-36-2}
如果说,因为12、12这个周五upload到github上去的review docs上一个词特别刺眼,那一定是批在review第二页纸下方的这个"practice"了。

记得那个周五中午,我见代课老师的办公室门是开着的,便敲门应允进去了。老师说他正在改我的review。说到最后,他指给我看我review的最后几段的"practise"说,他知道英国等一些国家名字也用"practise",但在美国,作动词用时是practise, 但作名词用的时候还是用"practice",并把这名词规规矩矩、端端正正地写在了这页纸的下方。我不以为意。

及至这天下午紧接着系里发出的邮件,我也以为天上掉馅饼,系里有多的奖学金的话,不是说我春天也可以拿到TA了吗?LinkedIn上当天也刮这种风,虽然自己当时并不没有回,但晚上临睡前还是发信回复给系里,问一下到底是什么情况。但系里小秘第二天中午用她的周末时间给我的回复是,那是发给全系的邮件,即使系里有奖学金,我是没有资格的,因为先前导师与我已经讲好了春天我只能注册一个学分。我友好地回复系里小秘的邮件。不给奖学金不是她个人的错,但她发出的邮件一定要这么穷凶极恶,作了恶人(不给奖学金)还装大款(学期结束和春季开学第一天TA邮件)的穷酸迂腐气息就容不得别人不客气了。我随即更新了自己的github内容,对"practice"作出详尽解释,并对这门课上自己的表现作了详细的陈述。
\subsection{practice vs practise}
\label{sec-36-2-1}
When I knocked the door of my instructor’s office yesterday (12/12/2014, now I am modifying on 12/13/2014) at noon, the instructor said he was right on working on my report, and if I could wait several minutes, he would be able to finish. So I waited in his office, and he searched internet on propose that the practise in my last section should be writen as practice, and that was the reason he wrote practice on my second page.

And after he searched the word using google, he emphasized by talking to me that in bratish english, people may spell practice as practise, but in America, they use to write as practice. He wrote the word down on my report without circling the original wrong-spelled word.

\subsection{Personal Conclusion}
\label{sec-36-2-2}
\begin{itemize}
\item I had minor difficulties setting up my Android environment at the beginning of the semester, like my window’s SDK manager never worked; And at the beginning of the semester my Linux Mint 17 Eclipse kept crashing… It was from time to time, I searched and googled, and get my Linux version stable; And I did have some help from the other classmate as well;
\item The first half semester kind of, the course contents were slightly distributed, and I felt I didn’t really know what to focus, and I don’t like that half of semester;
\item The rest about half semester I worked on this DrawingFun project, and I am confident that I did pretty good job, comparatively spearking, compared with the other classmate.
\begin{itemize}
\item I applied Color-Picker functionality, while he applied mine;
\item I applied ListView for drawing primitives, while he applied the same original setting-brush-size methods - a popup dialog with button choices included;
\item I initiated to include the undo/redo button in our app ideas orginally came from Picasso app motivation; We independently implemented undo/redo functionalities while mine is fully functioning and his some primitives cannot conduct undo/redo yet;
\item I found and debugged my setBrushSize() function to remember last applied color, the other classmate didn’t seem to be able to notice this, or he hadn’t have time to look into it yet;
\item I spent some time on the onSizeChange() function tried to make my program work then I hold it horizontally, while the other student directly set his App to be applied vertically FIXED so that he didn’t have any onSizeChange() issues at all;
\item I used bitmap and reimplemented my erase function, while he kept my default first version method. And his undo/redo/erase design supports only erasing smoothline (one of my six primitives, he had four or five), while mine supports earsing all my six primitive styles, and I liked this implementation;
\item I googled and applied FloodFill function in my App on bitmap level, and I tried two implementations, one ASyncTask idea (which was slow and sometimes my App main UI froze), and one Thread implementation (and the UI never freeze when I floodfills my bitmap); while the other classmate floodfilled the whole bitmap background with one color while loosing all other App functionalities, and I guess he didn’t really understand the difference between canvas and bitmap because otherwise he should have loaded a background bitmap which is adapted for floodfilling somewhere;
\end{itemize}
\item I completely understand that the other classmate could have his other priorities, and he is undergrad while I am the graduate level, just like cs480 Senior Design the course instructor didn’t require me to do any further design, but it is my first time to be able to handle such big and interesting project, and I want to dive into it and get myself well-practised, and I will insist this idea by implementing at least one of project great (senior design project, or midi Controller project, most probably the latter) so that I learn and understand.
\item As the course instructor has claimed at the beginning of semester too, he was also trying to learn Android for his Unicon graphics implementation later on, we were NOT a great nor efficient team yet, but it forms a great learning experience, at least for me. And later on, I should set higher standards on myself than now, and I will practice more for my own good.
\item This became another experience for me that Self-adaption, self-motivation and passion are very important for projects and career success.
\end{itemize}

\section{一个单词引发的血案(3)}
\label{sec-36-3}
就像编译课代课老师使用小动作、故意不及时反馈学生几次作业的情况,一定要再累加一次期中考试后把别人活活打倒成C;这次小秘又是在别人无心把代课老师这个别有用心的"practice"贴到github上后,她才立即故意穷凶极恶地卖富说系里有奖学金;到春季开学时更是在自己周一已经现身学校食堂、跟学校食堂说清楚了打工的事后紧接着的周二(间隔不到24小时)一早立即再来一个TA邮件,及至春天,我终于是不再去理会这邮件这人了。。。。生命有意义的事正多,何必浪费在这种人身上。只是这一个关于"practice"的机关与对战,终于还是开始拉开了系里、学校食堂里对我个人种种诬陷的序幕,种种手段粉至踏来,当然这是后话。

Senior design的docs因为材料已经准备得相齐全了,所以把这所有的材料内容组装成了一个pdf对于我这样的latex中等水平使用者、emacs org-mode的狂热爱好者来说实在是labor work。感恩节后的那周代课老师帮review一遍,倒数第二周周四12、11交上去update稿后,没想到代课老师最后一周周二才给回反馈,于是最后几天又交上去一份终稿。

导师先前对我说学校IPO可能能为我提供些经济上的帮助。学期快结束时,我便去IPO找相关负责人,问清楚春天这个学期我一个学分的学费情况等能否获得补助。IPO的老师明确地说,他们IPO是还有些经济,但帮的是有困难学生的最后一个学期。虽然我也是最后一个学期,但他们IPO帮也只能帮wave掉外州费,因为我一个学分不存在外州费要交,所有没有办法帮我。呵呵,要怎么说来,几年前我读统计时的一个中国女同学就告诉过我,她最后一个学期的学费就是被IPO帮免去了\$2000,而不是只能免全部外州费。说来也只能说环境要与你作对起来,要格外小心罢 了。

这个学期几门课要在github上上线,所幸,我没有在任何一门课上倒下。而我的秋季学期便在种种突入其来的惊悸感觉中结束了。 

\chapter{探访表哥 (12/27/2014, Saturday)}
\label{sec-37}
\section{探访表哥(一)}
\label{sec-37-1}
遥想起秋季学期最后一周在CSAC的时候,流浪者还去找过呆在那里作"监督"的E,态度诚恳地想请E帮他做一个项目的什么版块,这样看来,别人的寒假至少是有项目的、是过得充实的。倒是E什么时候与流浪者之间开始有了如此真诚的友谊与交集,还是很出乎我意料的。反观自己,整个寒假,像个没事儿的和尚,每天网上到处逛悠,每每心中对自己多出几分憎恶。

这段时间,我已经是全天候挂在LinkedIn上。新年将至,Linkedin上也常常出现一些标语,诸如"Only one month left, make it meaningful."之类的话语。有什么事情对自己是meaningful的呢,那一定是表哥了。从2013年1月到现在,已经接近两年的时间我们不曾见面了。是的,你猜对了,新年将至,年前,我打算去找表哥。 

这是一个下着雪的周六(12、27)。我用家里的扫帚和手套把车窗上的雪铲干净,边想这破小车今天千万得给力点儿,可别关键时候掉链子,这当口的再出个什么问题,可把我的美好计划全打乱了,边发动了车,费了九牛二虎之力,可算是把这破车给倒出去了,只要能发动能开,基本算是这车至少一时半会儿应该不会再出什么大问题吧。

雪正大,路上的车不多。窗外鹅毛大雪纷飞,车内开着暖气暖暖的,地方台的广播放着音乐,能够寻得这样一个日子出行,也好不惬意。一路上畅想着见到表哥的种种可能,不知不觉间,已然来到距离自己所在小镇11迈的表哥所在的小城镇。表哥计算机系的楼在一个坡上,开车上去下来停车都不方便,我打算把车远远地停在downtown街道旁边,然后再寻条小径迂回地走上去,省得车不笨的时候人笨,人脑子不笨的时候车会笨。。。。
\section{探访表哥(二)}
\label{sec-37-2}
如此这般想来,便把车径直停在了中央大街的右侧顺手边,不偏不倚停得妥妥的。雪已停,只是雪刚停下的水泥人行道铲雪车都还没能来得及铲。我拉上羽绒服的拉链挡风,肩上挂着跟随自己多年的深褐色K氏小包,脚踏虎皮长靴,轻轻爽爽地走在了前往坡上表哥计算机楼的纯天然自然景观小道上。表哥计算机系的楼,远远地,就伫立在我眼前。呵呵,就要见到你了。。。。

差不多走了半迈多吧,登上山坡,左手侧就是表哥所在的"eecs"的楼了。我先走2010年12月那场彻底感动过自己的告别里表哥带我走错过的那个门,远远地看见三两个车稀稀疏疏地泊在门前,车身上堆满了雪。及至走近,待我去拉门,却拉不动,门是锁着的。我这才意识到,作为系里硕士研究生的自己,周末和晚上通往系楼的access尚且被deny了,作为外校学生一一非本校本系学生,我有什么资格可以在这样的周六畅通无阻地进到他们的研究楼去?

感到些许有些意外,但我并不死心,与这个后门同一走廊的半腰处有一个侧身、装饰更为精美,门上标注有作息时间的前面带有楼梯的门,我可以再去那里试一下。原路返回,顺着山坡路往下走,再迂回到楼梯门前,门也是锁着的;不怕,就我所知道的,至少还有两个门呢,一个正门,门上架着一座天桥,另一个表哥先前所在的basement办公室走廊一侧的门,费了这么大的劲儿,人都来了,至少要把能走的门都走过一遍吧。

依着顺时针方向的顺序,所以这第三个门就是距离这里最近的basement走廊那侧的门了,呵呵,我就知道,这所有的门里,总有一扇是为我开着的\textasciitilde{}~ 走过basement长长的走廊,清楚地记得表哥先前的办公室在哪里,我在走廊的哪个位置被上刑过。。。。寻找到表哥先前领我走过的某个特定的楼梯,我能绕到顶上架天桥的正门,再从正门走过一条长长的走廊,在走廊的某个位置的教室里,表哥在这里代过课,我听过他讲的一堂课。。。。到长长走廊的南侧,从那里乘电梯到顶楼。电梯坐落在东走向侧腰走廊的中间,不幸的是,表哥办公室在走廊东侧,而在电梯视力范围的走廊西侧就有舅舅的办公室,我在心里祈祷,最好不要被舅舅发现,如果他偏巧这个时候就在办公室的话。。。。待我被载至顶楼,就着依稀可见的昏暗的灯光,我像是到达一处人迹罕至、荒芜不烟的境地,还好,舅舅办公室的门是关着的。我轻手轻脚、蹑手蹑脚地向走廊东侧拐角90度后的第一个门一一表哥的办公室探过去。。。。
\section{探访表哥(三)}
\label{sec-37-3}
从南侧侧腰走廊左转,就转到了东侧长走廊,这个长长的走廊里,灯光明显明亮起来,而这个距离自己最近的第一个门,就是表哥的新办公室了。我轻手轻脚、蹑手蹑脚地挪到表哥办公室前,见两年前来找表哥时我在表哥的办公室门上画过的小太阳还在,小太阳里面长着像向日葵一样的一个娃娃笑脸,这个向日葵娃娃笑脸(此刻敲下这段,我怎么会忽然想起高一报名前,初到大姐大哥家时,几姊妹里大哥买给我一个人吃的娃娃脸?),后来在我的android app programming中bitmap level的floodfill功能案例里也如此设计和画过,图片还在github上放着呢。两年前我用表哥门上的笔和白板画这个向日葵娃娃笑脸的时候表哥就在屋子里坐着呢,他当然知道这是我画的。回想起这些,我心中不免多了几分感动和满足。

我悄悄地拧了拧门把手,门是锁着的,那我就先静静地在门外呆会儿吧。楼里的暖气很足,我把外套羽绒服脱掉,折叠挂在左侧手臂上,再搭右手、双手搂抱在胸前,就像2010年12月表哥陪我去他们学校图书馆,我在那里去用洗手间出来时看见的,表哥抱着我的外套、双手搂抱在胸前等待我的样子。。。。

我还需要在向日葵娃娃笑脸上再添加些多的创意么?来不及细想,听见脚步声,表哥已经出现在了走廊的另一头,手上端着少量餐具,应该是从洗手间或是什么地方洗干净回来。而最先注意到的,却是表哥的一头灰白头发。刹那间,我禁不住面孔扭曲,浩大的眼眶湿润了,眼神几许空洞几许迷离。。。。不到两年的时间,我们都变了,我已不再拥有少女的身姿,身体臃肿发福了很多,不曾想,表哥你也早生华发。。。。想来,枯燥的科研生活几许清苦,又岂是人人都干得了的?

我不开翘的耳朵又听见了久违的表哥的话语音,只可惜,表哥重复着同样的话,叫你不要来这个地方,"I am going to call the police."表哥打开门,把餐具放下,拿起手机,打起电话来。

相见时难别亦难。不见如何,见了又能如何?泪水盈满了眼框,一个转身,不觉已是泪水决堤,扑朔而下。
\section{探访表哥(四)}
\label{sec-37-4}
车停在主街上,我无意走正门,便依旧原路返回,从basement表哥先前的办公室前走廊一侧的门返回来。或许这就是我的命吧,年轻的不爱,年老的看不上;浅薄肤浅的,浅尝辄止,过目即忘,就像别人总结过的"恋爱"换对象换朋友的速度比翻书还要快。。。。刻骨铭心的,却注定要经历如此这般孤独纠缠。。。。

风干了泪水的眼眼枯涩晦暗,拉上外套的拉链,一个人怏怏地走在去往下坡的山坡上,好不寥落。一辆警车从标有"road closed"大雪封住的坡面上开了下来,停在我背后的坡上。一个policeman探出头了,像似在叫我,对于policeman,我又岂敢不应。回过头去,人问你是XXX吗?我是。。。。重新回到山坡阶梯交汇处,警官也下车来问我话。

自2011年以来,因着与表哥的关系,我已经无数次地经受来自表哥所在的学校和表哥所在小镇上警官们出乎意料的"垂询"洗礼,但人一辈子值得记忆的事儿正多,不相干的又干麻要记得?所以这些个表哥学校里的、小镇上的警官们,我还是傻傻分不清楚。

但是,显然,这个招呼我留下,并主动走下车来问我话的警官看上去很和蔼。警官也收到来自他队友的呼叫,于是我在一边儿枯站了会儿。

警官说,"I see you look sad walking down the hill."唉,这真是哪壶不开揭哪壶,干枯晦涩的眼睛再度瞬间噙满了泪水,无语已先凝噎。既然警官问话,一定大意不得,我把自己13年3月7日的最终决案先对警官讲清楚,然后说这转眼已经快两年了,而且也快过年了,非常想念表哥,所以自己主动过来想看看他。

警官问,"X said you called him the other day, have you called him recently?""Yes, I did call him about one week ago. Let me check my phone to get the detailed information."一边拿出自己的手机,翻出电话最近通话,告诉警官,"I did call this person on 12/18/14 at noon for less than one minute, and in the call he told me not to contact him anymore."

警官问,"Do you know X has a permanent protection against you?""No, I don't have any of permanent protection order information. I know only temporary protection order。"

"Do you have any memory about some policeman serve you some documents in 2013?"这个我记得的,就是13年1月去找表哥回到家后的第二个还是第三个晚上,两个警官来到我家,给我拿了一份来自表哥的temporary protection order, 上面有要求我1、31到表哥所在的小镇就表哥的这份temporary protection order出庭,我记得出庭那天出门之前我还精心打扮了一番的。"Yes, I did remember the fact that two policemen arrived at my apartment on the evening about two days after I arrived home in Jan 2013. But the document they delivered to me was the copy of temporary protection order, and the doc asked me to attend the court on 1/31/2013 for the temporary protection order. But on that day 1/31/2013, my cousin didn't show up and I didn't received any information about the temporary protection order afterwards."

"Do you have any memory about permanent protection order?" "Permanent protection order is something that I only heard from you this afternoon within the last hour."

"Do you remember any expiration date?" "In 2013 court season, 3/7/2013 was the last date I remembered that I attended the court. And from this court result, from the lawyer, I learned that I should NOT contact my cousin for one year. So I supposed the expiration date should be 3/7/2014, and that was the reason I arrived here today quite relaxed and want to see my cousin. "

"Do you have any memory about the date of 3/21/2017?" "No, I don't have any memory about this date and today is the first day I hear about this date from you if it has any meaning special."

或许审问信息的过程过于枯燥无味,警官问完了我所有他想要问的信息后,便非常友好地陪我聊天。他用脚像小孩子般顽皮地铲着他脚底下的雪,并问我来美国多久了,喜不喜欢这里的天气和风土人情等。提到一个什么话题时,他竟然也如我曾经情不自禁对E评M说"he behaves like a baby now"般提起什么像baby行为,让我感觉很放松很温暖。我便也问他是土生土长的美国人么,他说他也不是,他是从加拿大过来的。或许与那堂听完表哥的课后遇到的那个警官对表哥的代课和研究生涯有着极大的肯定、诚意和尊敬的警官一样,他们同属走心爱情派的吧。阶梯路上一对男女朋友小情人经过,看见我们聊天,这位警官也同他们聊了几句。

这位警官与我不紧不慢地聊着天,差不多过了一个小时左右吧(我从家出门的时候是两点左右,这时大概四点多钟?),又开来一辆警车,紧挨着前一辆停下。车里下来一个更为年轻的警官,一对"八"字眉,满脸横肉,整个造就了我眼前这幅器质性猥琐男形象。年轻的警官要先前的大叔长者警官去复印一份什么材料,先前那警官便开车离开了。过了大约十来分钟,先前警官问过我的审问的话被这位警官再次问了一遍,同样的答案我于是又重复回答了一遍。

待得去复印的先前警官回来,气氛突然显得异样,大叔警官也站在离我不远处,年轻警官再度发起一股问话时,语气已然显得生硬凝重。"When was your last date you attended court in 2013?""It was 3/7/2013 I got final result from court and I was warned that I should NOT contact this person for one year." "You last court date was 3/21/2013, and now, you are under arrest!" 

在买买提的首页上,我至少看见过一次哪个学校的学生被陷害,他的同学或是亲人在买买提上发文章请求好心人帮忙petition他是无辜的,难道自己在这只剩下最后一个学期的时候,此刻自己也在经历着类似的陷害?这突入其来、完全没有心理准备的待铺,便得自己像被寒流侵袭,又像是被雷劈,我浑身一阵痉挛,不禁打了个寒颤。我被要求手背后,手腕上也被戴上了手铐。

"You have the right to keep silent, "…..他们快速地机关枪般打完了一通长长的话,问我听懂了吗?"NO, I didn't fully understand."我本能地变得非常警醒,提醒自己,这种时候一定要保持镇静,全面自卫,没听懂绝不装懂。他们拿出一个小卡用手电筒照在上面让我读,我一个单词一个单词地看着,"Are you done?"上下两段话我才只读完了第一段呢,"No, I just finished reading the first paragraph."于是他们耐着性子等我读完。

被要求两脚分开与肩同宽,一个警官执行搜身也搜去了我挎在肩头的包包,然后他们把我押上车,车发动开在了驶向临镇country jail的路上。
\section{探访表哥(五)}
\label{sec-37-5}
那个county jail我是知道的,我已经不是第一次去光顾那里了。只是这一次他们没有先把我运到表哥所在小城的police department去作中转。驶往county jail上高速前的一趴平地,警官把车停下来,后面紧跟着的一警车也停下来,紧接着就有一位从后车下来的警官上前来坐上我所在警车的副驾,他们两个小声嘀嘀咕咕地一直在嘟噜着,我耳朵不好使听不大清楚。过了大约半个小时,像是最终完成了一道司法程序,副架上的警官退出去,司机警官一辆车一个人载我开上了高速。

傍晚高速旁的黯淡的土地几近荒芜,只有来往车流的前后灯还在提醒着夜幕中的人们这里曾经有过梦想。半小时后,警车抵达目的地,呼叫,报号,车库电门打开,我被带到高墙深院内的前厅。

填表、换拖鞋,这些都是自己曾经经历过的程序。除了前厅里那个头不高,典型美国人宽厚面庞长相的170cm左右的看守,还有一个年轻约二十多岁的精干小打杂。走完自己的正常前厅办理手续,司机警官问看守,他能否用一下这里的厕所,对于警官来说,这当然不是问题。而我,这时也被提醒、想到了自己的问题。

是受到惊吓、恐吓后的正常情绪性生理反应么,来到这里的高速路上,我意识到身体下体正在不正常地排泄一些分泌物。这么多年来,对自己微痒的身体反应已经摸索得非常熟悉,每年不超过六次的不规则生理周期,说它不规则是因为我先前追溯过一年左右的各周期起止时间,知道不规则,想记起来特费脑子,而且生理来潮初期的嗜睡、右后侧腰肩盘酸痛都会告诉自己是怎么回事,也就不用记了。不规则的生理周期间间或着更为不规则的排卵期不规则出血。排卵期时只是嗜睡,不会腰痛,分泌物是以巧克力色老死坏血打头,接着才是新鲜血液,量少,而生理周期以鲜血打头,量稍多点儿,老死坏血收尾。这两三天我的精神很好,没有嗜睡症状,感觉有一定量的分泌物还是很奇怪的。

我也接着问了我能否也用一下这时的洗手间,看守同意了。我想过是否等我去趟洗手间确认后再问,但多年来我对自己的身体反应还是有把握的,便也没法害羞,直接问那男性看守说,你们这里有没有卫生巾,能否借我用一片?那个确切的英文单词我并不知道怎么说,换用几种不同的描述,他便递给了我我所需要的护理用品。待到厕所确认,知道自己是猜对了,却多出几分惆怅来。2001年7月29日,武汉市第一人民医院为我施阑尾切除兼右侧卵巢巧克力囊肿穿刺手术。那个当时不到22岁的自己得了这样的怪病,是因由9年前的那场意外,还是接下来五六年stressful的岁月,抑或两者兼而有之?就像手术时那位妇科女医生对当时躺在手术台上的自己极为惋惜和同情,此时回望这段岁月,竟是对当年的自己多出几分恻隐同情来。

等我用完洗手间出来,司机警官还在等我交待一件事,他说permanent protection order 的复印件他帮我带了一份,与我的包包衣物放在一起,要我回家后仔细读一遍读清条款。然后我就被领到了前厅里的小jail里等候进一步的招呼办理手续。警官司机还没有走,高墙深院内的自己,尽管出于警醒,我努力捕捉着周遭的蛛丝马迹纤微波动,但我已然清不听他们在说着什么。几分钟后,司机警官也就从前厅离开了。
\section{探访表哥(六)}
\label{sec-37-6}
呵呵,原来前厅小jail里是有简易马桶的,我竟是忘了。看着墙上的挂钟,六点还差几分钟。前厅的看守和小打杂还在忙忙碌碌地几台电脑上处理着各种文件,一时半会儿应该忙不到我头上了。这次他们也没有给我条毯子什么的,无心睡眠,便倚在墙头,透过小窗厚厚的玻璃望向前厅想着心事。

七点多钟,忙完了一阵子,我以为小打杂要来我门前将我领出,却领出了我隔壁的中年大妈,原来这里并不只有我一个人的。对他们外国人我一向猜不准年龄,想来四十多岁接近五十?身体臃肿肥胖,还是就称呼她大妈好了。看守带着大妈填表格,按指印,进行着一道又一道的程序。大妈或许身体也不好吧,看守带她走程序的间隙,她还需要尽量地坐在椅子上,双手轻轻捶打着膝盖帮助放松。待这些个步骤走完,应该就是bail out这一步了。

我在这个小镇上的朋友们,男闺密和大块头(我已经不记得去年夏天写时叫得他板砖还是板块了,反正他是几个男生里块头最大的就是了,下文还是叫他大块头好了,方便好记,他是男闺密的"好基友"。)已经毕业有时,这天此时此刻应该已经开车走在去往加州的高速上了。若说13年时还有闺密和大块头来帮忙接我出去,今天最好的朋友们离开后的自己,竟是没有再能可以真心诚意愿意帮助自己的人。

所以刚才我就特意问过看守,看守说花\$250我就可以bail出去;我也特意问过他,先前都是别人帮忙把我bail出去,现在我找不到可以帮我的人,那我能否把自己给bail出去?看守回答说是可以的。我知道虽然自己几张卡里剩余的钱都不多,但是花\$250块钱把自己先弄出去、脱离危急关头、不至于把大好青春浪费在高墙深院的钱我还是有的,所以当下也稍为安心。

只见大妈拿走墙上的电话,按照看守报给她的数字播着号。每个电话三五分钟不等,待大妈播过三个电话,大概这已经是所有available的public帮助电话了。大妈的眼底涌上了泪水、晶滢剔透,情绪似乎一再落寂。

这时小打杂来到我的门前,我也被带了出去,看了看钟快八点钟。走着大妈刚刚走过的一道道程序,填表填表再填表,大妈的呻吟声在加剧,小打杂帮大妈从内屋打了杯水递给她说,"Are you OK?""I'm NOT ok."像是在竭力想哭出来。看守在处理电脑里的文档,小打杂接着帮我审查处理表格。稍过一两分钟,大妈的哭声终于响彻大厅,我麻木地站着,望向此刻那全身已经已躲在地上的大妈一阵阵抽搐着,回想当年那带着十分忤逆横躺在表哥先前办公室走廊里的自己,几分相似几许同情。若是没有这\$250块钱,我还能有此刻的心安么?一场\$250块引发的心理防线的崩溃,不知能否起到火中送炭的效果,还是像自己所经历过的,\$416的医疗费用看似wave掉了,却在2013年3月后收到\$832的救护车费用和其50\%的另\$416 AAI 中介processing 费用的收费通知,让自己原本拮据的处境雪上加霜?同对待当年的自己相似,看守呼叫了救护车,一大队人马很快前来,几近麻木冷血地把大妈架在担架上,剪开胳膊的袖子量了血压。。。。是怕我逃跑,还是怕我把这一切看透(我又怎能轻易看得透?),骚乱中看守把我领回了前厅刚才所在的 jail,待大妈等一大队人马离去后才再领我出来继续填表。填完表后,小打杂早已不知去向,看守说他现在有事要忙,我需要重新回去再呆会儿,等他一忙完就来叫我。看看钟,晚上八点二十。 

我注意着前厅里的变化,看守到底在忙什么呢?他时而出现在前厅,时而离去,把杂物等运往过洗衣房,也从洗衣房取出过洗过了的衣服,我猜想可能大概也还给狱里被看管的人们送过晚饭吧。

等待总是灼人心。到九点钟,看守来领我出去说我可以把自己bail出去了,正当我拿了自己衣物箱中的钱包,抽出卡来跃跃欲试急欲离开的时候,看守无形中已经把防线升级了。看守说他刚才忘记了今天是一个什么特殊的日子(2013年12月27日周六会是什么特殊日子呢?),所以bail out的钱升到\$500了。好你个家伙,感觉别人要整人时,自己比窦娥还冤,欲哭无泪。"Though I am sure I have \$250, but I am NOT sure I have \$500 with me."看守是要打探我的经济情报么,他对我抱以了极大的耐心说,没有关系多想想,几许几张卡并起来可以凑够的。于是我一边对看守讲我的银行账户分布情况,一边以借机帮助自己理清头绪。我告诉看守,我只有两个银行账户W和C,两张debit卡两张信用卡。一段时间没有查账户了,但我记得W和C的信用卡都远远没有\$500块钱,W的借记卡里有四百多块,但具体四百多少我就不记得了;C的借记卡时余额为零,W的信用卡余额几乎为零。看守说若是用信用卡付这五百块,需要我自己多付\$25块,并且到时五百应该会返还我,但这信用卡的处理费用\$25是不会返还给我的。看守建议说他可以试着划一下两张信用卡,先确认里面分别不足五百;"I have a question though, when you try the \$500 on one credit card, if it deny, will I still need to pay the \$25 bucks?""That's a good question; let me confirm this one with some professional help."于是看守就把电话打进里屋,一阵嘀嘀咕咕后,看守回答我说,如果信用卡deny了,那我是不会被charge \$25的。那我就同意他两张都划过了试过了,都没有五百;折成\$420+\$8?,先试信用卡,但C的信用卡还是deny了;又试\$450+\$53,当两张卡都通过时,我终于是仰天长啸,苍天啊,你终于没有把我一个人遗忘在这个角落。。。。呵呵,我夸张了,长出一口气,这个晚上终于不用再费周折耗在这里的感叹还是有的。

我签名,穿戴好外套靴子,收拾好账单的收据,准备离去。但我意识到我还缺一样东西,便问看守,刚刚送我来的司机警官说他有给我拿一份permanent protection order 的复印件,刚刚那复印件也是放在与我衣物一起的敞口箱子里,可是我的复印件为什么不见了呢?

看守故作记起状,拉开他文件柜的一个抽屉,寻找抽出那一份轻薄的文件,看了看再转手递给我。我扫了一眼最底行的"permanent protection order"这样就顺从看守的引导穿过一扇扇铁门来到后厅,打算从这里打算离开了。看守叮嘱说,"If you need/want to wait here for a while, feel free to stay in the lounge. It is especially cold outside."我礼貌地回答着,好的谢谢。

看守走后,我打开文件,寻找着某个日期标注,赫然看见一行日期,2017。3。21。虽然我早就该有了足够的心理准备,却还是像见了鬼般逃也似地逃串到了downtown主街上。。。。在这午夜的街头(晚上九点钟),我仿佛刚从活死人墓里爬出来,惊魂未定的自己,看着往来的车流,终于寻得一丝慰藉。。。。
\section{探访表哥(七)}
\label{sec-37-7}
逃串到大街上,丝丝凉意感觉很清醒, 望着往来的车流,我确定自己站在小镇的main street,却不知道返回我车所在、表哥所在的小镇该走往哪个方向?远远地看见一个二十来岁的年轻人行走在大街上,我便凑上前去询问,年轻人给我指了方向,并强调说,"It's 17 miles away from here. Make sure you are NOT going to walk towards there."谢过他我知道的,便顺着年轻人指的方向走过去。

在这远近小镇上,坐公交车受制于时间,非常不方便,我已经习惯了在上下高速的路口随手拦车。早在12年的时候有一次回加州,从一上X7I号高速就载了一个露宿街头的牧师,那时我赶路回加州上班,高速上好几个小时开了几百迈把他从CA Redding放下,他不曾想要表示点儿什么,我也不往心里去。但在这小镇上,我是为自己方便才在路口路边拦车的,所以总是会准备几块零钱放在口袋里,麻烦别人停车帮助载我一场,好歹汽油费也还是该出一点儿的吧。有次是在13年自己第一辆三菱二手车破旧得不行捐出去后出行不方便,有次搭公交车到county court,回来时,差不多是自己此刻站着的位置,拦了辆车到表哥所在的小镇,给司机女主人了\$4油费。她知道我接下来还是需要拦车回自己所在的小镇,她和她老公晚上正好也需要到我所在的小镇,就把我载到她家去,呆上半小时,她准备好晚上potluck party的食物,又接着把我载回学校所在的小镇,我又给了她\$4。她不愿意接,其它帮助我的人不愿接的时候,我都这样说,"If you don't accept the money, next time when I need help, I won't have enough courage to stand roadside and wave my hand ask for help any more."诚恳到这个份上,大家都还是会接受的。

此刻,我的牛仔裤口袋里刚刚装了钱包里正好还有的三块零钱。远远地感觉身后有灯光照过来,更挥手把车拦下。等车近了一看,是辆警车,真是阴魂不散。我非常抱歉地对警官解释说,我只是想拦辆车把我载到临镇,不曾想是辆警车,不好意思,打扰您工作了。。。。警官Make sure I am NOT going to walk towards there后开车离开了。我一边留意着从身后照过来的灯光,一边踏着雪后雨后的泥泞火急火燎地往前走。中间大概拦过一两辆不曾停下的。到再撞拦下另一辆的时候,又是一辆警车,警车今晚难道是在专门跟踪我的吗?警官要求我walk no further than前面一杆昏黄的路灯,而且提议说,他现在还在这里on duty,但是如果10:30 pm他从这里下班时如果我还没回去,他可以帮忙把我载到表哥所在的小镇。我回答他说,我会尽量拦一辆车尽早回去,但万一我拦不到车,十点半时我应该就站在前边那路灯下,到时再请他帮我载回去,他同意了,我对他表示谢意。。。。墓里的活人哪里会是我想要沾惹的?我想回家,越快越好!我就已经站到一警官先前标记过的路灯下了,继续拦,我都还没有招手,怎么又是一辆不同的警车,我很生气。。。。警官说walk no further than here,我告诉他,是的,我就站在这里,我不会再往前走了。。。。

再后来,又拦了一两辆车,终于有一辆passenger car停下来,是一对加州工作的夫妇回来过圣诞节的。他们可能对我有所耳闻,司机主动与我聊统计计算机等相关专业他做过的一些个项目,聊他们的圣诞节等。我坐在后座用手臂支撑着身体前倾着与他们热热闹闹地聊天,仿佛唯有热闹、叽叽喳喳说个不停的话语才能驱散刚刚那过去不久的惊魂不定。车后座上的狗狗侧身倚在我手臂上打瞌睡,散发着丝丝温热。

快到达时我如愿把汽油费给了坐我前方的妇人,自从下午两点种左右发动机车到现在,不过几个小时的时间,我泰极否来,仿佛从远古走来,经历了千年,穿越了纷飞战火,拖着千疮百孔、疲惫交加的身体坐进了自己车里,仅过了一个下午,不到十个小时,却感觉自己老了十年。。。。及至二十多分钟后真正走进自己租住的房子,才由衷地感慨自己租住的破窝竟是从未有过地温暖。。。。

\chapter{经济崩溃}
\label{sec-38}
先前我导师对我说,据他所知,春季我只注册一个学分我只需要交\$1100。后来12、27因为要把自己先弄出来,\$500就进去了;我以为秋季学期的奖学金早发完了,后来又发过一次,这样我几张卡并起来一共大概可以有\$800。我还差\$300。前面因为自己状态不稳的时候写的轻缓版的借钱就是发生在这个时候。

后来大表姐答应就是打到我卡上的\$1000迟迟没能到账,事后打电话大表姐否认了,她不愿意这种时候借钱给我(她只愿意在最终这个夏天我一定走投无路、不得不回国的时候,第一次也是最后从经济上帮我,电话里大表姐说,她愿意帮我买一张回国的单程机票);而真正要问自己的亲姐姐借,从国内弄借过来我还不知道到底该怎么弄,而且从来不曾寄过,也不知道需要多长时间才能到账,也只能作罢。

美国,这里,还有谁那里我也许可以借点儿钱呢?想来想去,还是想到了男闺密。男闺密是这一圈所有的中国人里唯一一位借钱给我的朋友。前面说过到2014税退回来后我有还给男闺密一些零散的借款,到那时,是从男闺密那里借得最少的时候,还欠他\$2000。后来2014年8月,我手来相当拮据,连租房子交押金的钱也没有了,男闺密说,再借给我\$500最后一次帮我度过危机,要我以后钱的事不要再找他,这样从他那里就借到了\$2500。可是现在,到自己举目无亲、走投无路的时候,我还是只能再次转向男闺密。

我怎么才能说服别人愿意帮我呢?我告诉他春季我只有一个学分,而且因为前面2014年1月与学校食堂two week notice period交接工作做到位了,所以这个学期我会无疑问在学校食堂打工(呵呵,说的是毫无疑问,事实上我又何尝不是被拖延开工、随时担心被栽赃走人、甚至提前释放?),且不受20小时全职学生半工时间的限制,所以我告诉男闺密,这\$300我一定有有限的收入来源可以到这个学期结束的时候还给他,而且我会尽自己最大努力把那\$500也还他好凑个整数。

呵呵,答案是大家猜到的,男闺密从来都是好人品的,他一定会帮的,这也是这么多年来他成为我在美国最好的朋友的原因。2009年初夏当我品尝失恋的痛苦,觉得自己这辈子完了,跑去找他时,男闺密说过,我们成为一辈子的好朋友都可以。从2007年认识这样一位朋友,这么多年过去,到此刻自己举目无亲、草目皆兵时,我无法不真心感概,这样的好朋友,一生不要遇到太多,让我这样一个千疮百孔、支离破碎的中老年人对友情抱以最真诚的期许和虔诚的希冀就刚刚好。

只是真正到了开学第一天,学费是\$1290时,我还是欠了\$190没能交上。而且由于学校食堂拖延我开工时间,到开学后第三四周的时候才能真正开工,\$190就再次成为我心头大患,到处找硬币买食物以求温饱就是这个阶段发生的。只是,出乎意料,当我不好意思再问男闺密借这\$190,转向系里流浪者看他能否借钱给我让我能稍微周转一下时,终于不倚不靠地触发了一场来源于系里、为我私人订制的浩劫。

\chapter{流浪者}
\label{sec-39}
\section{流浪者(1)}
\label{sec-39-1}
我想流浪者在漫长的流浪岁月中一定是练就了某种超能力一一神奇的预言能力,以至于他对我说过的话、做过的事,还没有发生的,就必将发生,比如我很不幸地丢掉了自己的学生办公室一个座位。

感恩节后的某天我在CSAC写作业,不知道什么机缘巧合下流浪者过来问我说,现在我该对他的办公室很有兴趣了吧?我心说,从什么时候起,你竟然可以操控起计算机系来,要别人对你的办公室感兴趣就必得使用如此手段么?就算是我有十颗心,我也早就被你活活气死了!但小人物的悲哀便是对家里每月交付\$46.34的网速耿耿于怀、忍无可忍,于是我还是跟他到basement他办公室里先去看看再说。

现在想来,basement里也还是有着一位肥版的老师办公室进驻作眼线的?我知道这个肥版老师好像是系里的老师,只是我不知道他的名字,从来也不曾看见他去代课什么的,只见他每天嘻嘻哈哈乐呵呵的,或许是作研究的吧,辅导几个年轻本科生学生作作项目课题什么的应该是有的。我之前的学生办公室与这老师的同侧,先前从他办公室门口经过时就常见一个漂亮的美国小妞儿在他办公室作项目时展汇报。后来漂亮小妞本科毕业了,这位肥版的老师似乎顿时木有了招惹年轻女孩子的魔力?(这段就只是调笑吧。。。)

肥版老师的办公室与流浪者的门对门,那天他也在,穿着蓝色衬衣坐在电脑前网上冲浪吧。流浪者打开门,我就坚持把门全开着。如果说我先前办公室是能摆两张办公桌的方屋,那这一间没有窗子,但即便两屋同宽,也足够有先前的两个大,两宽边和一侧长边都是长条桌,加塞一把椅子绝不是问题,2013年秋季学期伊期,在结束暑期实习、离开一个优良工作团队、别离一位优秀的mentor后,男闺密和大块头已如我春季期般与学校不着边际,那种高效、激发潜能的工作环境已不复存在,倍感孤独的自己这学期第一个月曾在这里短暂借住过。那时还有一位印度男生,我们常常三人在这里共同学习过。

其实想要借用这样一个办公室,对我来说,我最看重两点:学校有线网的网速、和流浪者多如牛毛的宽频monitor,而这些是自己占有一个座位的办公室被没收后系里无论如何不肯轻易给予、或至少付出沉重代价的,就像正在发生的此刻与流浪者的对话。

这是学校的公有财产,不能私自配钥匙,所以我要想进系里(系里办公楼晚上和周末的时间需要刷卡才可以进入)、或想要进到这个办公室,我必须得先与他取得联系。流浪者说他有手机了,不再用固定电话了,他把电话号码给我,方便我想要学习时可以打电话给他。此刻如此诡异的电话号码问题让自己几分警觉,便必须得摆明立场。

我对流浪者说,在我决定share这个办公室之前,有几点事情我们必须讲清楚。第一便是,"Whenever there are two person in the office, we should always keep the door open. ALWAYS!"当流浪者问及原因时,我就解释说了,我是一个女生,印度人和他,或许还有一位中国博士生,你们都是男生,这个办公室没有窗子,甚至嵌在门上一片狭长的玻璃也被白纸封严,我不希望我们简单、单纯的同学或朋友关系被说得扯得不清不楚。我与流浪者站在敞开着的门口讨论这番话,肥版老师的神情显得不可理喻,好像可以听见他的叹息。我不明白这为什么这个前提会对流浪者造成困惑和干扰,他作罢,我也就随即作罢,家里网速慢多过段时间适应了就好了,返回上面CSAC去接着写自己的作业。
\section{流浪者(2)}
\label{sec-39-2}
去看过流浪者的办公室之后,与此人再无瓜葛。很长一段时间以来,都是流浪者主动来找我,我没有任何想要去找他、打搅他的意思。只是注意到秋季学期快要结束的时候,在CSAC里,他主动去找E,非常谦虚地想要请E在寒假假期里帮他一个什么项目。

必须得说,在CSAC里亲眼目睹耳闻这一幕的时候,我还是实实在在地被惊到,惊异于流浪者去找E护身(就像是光天白日之下用事实说话,呵呵,你不想与我沾点儿关系呀,我喜欢的可都是混血美女超女级别的,我还不想与你有丝毫瓜葛呢!)惊异于E的自信与漠然。E说等她把手头的几个项目几份任务弄完、假期开始后再找去他再去帮忙。

秋季10月17日home coming我导师要作tower light show,而我被导师用一个linux系统网罗锁定到tower前的demo现场。当demo进行了几分钟就出现重大技术故障的时候,这对于我这样一个第一次被抓到现场就不得不经历这一幕的幼小心灵来说,实在是太残忍了。。。。

来来来,回忆一下那些惊心夺魂的记忆瞬间吧:当年仅几岁的自己耳朵一阵呜咽,就什么也不知道了的时候;当躺在姐姐怀里,知道自己几分钟前差点儿小命儿就木有了、夺命亡魂般哀嚎不已的时候;当第一次认识到爸爸错了,生命从此失去支撑的时候;当经历了几个月的挣扎、静养,父母姊妹亲人都不再给我施加任何压力,我最终主动对爸爸说如果今年我没能考上我想再复读一年爸爸眼中泛泪说出一个好字的时候;当"小学生年级"弱弱级学生非法操作便得电脑系统崩溃的时候;当亲身经历法律的"栽脏"与无情,从活死人墓里爬出来的时候。。。。当我们苦等"长今姐姐"做菜,却得知长今姐姐身体不适大出血的时候;当我们期待又一次的精神之旅心灵启迪的时候,看见的却是老将们对loly满腔热情猥琐男练成记的时候。。。。

而E,不知道选过AI没有,一定没有选过EC课,senior design项目里不管是使用github账户、还是Qt Creator的安装,她都是拖大家后腿的那一个,代TA课class类会给入门级CS120学生讲inherence,提到segmentation fault正在上compiler课的她也还是会情不自禁地脸红,除了暑假结婚顺了系里学校里的小镇文化号召、顺着EC代课老师的意思讨好TA老大诛伐异族,极尽讨好谄媚之能事,E来能干什么?对于计算机课,如果能取得成功,一般基本编程能力都不会差。编程不强的E自已都说,她最喜欢干的是类似文秘类的计算机相关工作,那E的自信又来自于哪里?E的态度、表情就是在说,我早已从XX那里听说这事了,这三下五去二的活儿,待我忙完手头,便去帮你,不在话下。。。。

眼下自己实在是举目无亲,再没有别的可帮自己的人。我想到了流浪者,而我从学校办公室搬回家的所有东西里,我还是租着流浪者的笔记本和一台宽频Samsung显示器的。我心里也曾经犹豫过,虽然那天在他办公室里记下了他的手机号码、也还是没能谈妥,似乎他对我有某种意思,虽然后来有了E的保护。无论后来的事如何吧,我还是相信人间自有真情在。09年前(07,08两年主要是)我不是也始终对男闺密暧昧不清、抱有幻想么,但09年一事终于使得一切明了、再无瓜葛,成就一段友谊。我需要别人帮忙的时候,还是要请人帮忙的,那天下午,犹豫之下便还是播通了流浪者的电话。

简要解释了我还久学校\$190没交,学校食堂还没有开工暂时没有收入来源,但只需要借点儿钱转把手,等我一拿到工资便还他的事实意愿。电话里,流浪者没有丝毫兴奋,却来得十分潇洒自信坦然,"We could figure something out."流浪者说来,还在"We"字后长长地顿了一下,过了这个寒假,过了一个多月,我们是在"开关门"立场问题上产生的分歧,"我们"就只是最简单普通的同学朋友,流浪者每说一个"We",我这头就已是听得触目惊心。我不知道流浪者是否在CSAC里上班,还是在他的办公室里,这些个"We",他到底是想要说给谁听的,那通电话里,他说了不下三遍,他哪儿来的胆子敢这么说一字一顿、一遍又一遍地说"We"?就过往平时我对他的了解,再借他十个胆看他敢不敢平和坦然地说出一个"We"字来?

是的,有过太多的暗淡忧伤,如前面提到男闺密和大块头最后一个学期与学校不着边际,学校对于我已经只是似有似无的存在,这个学期几乎还不曾去到学校过(若是去过,仅是短暂找导师商量过weekly meeting时间等工作时间表定下后再确定)。晓松老师说,走进大城市,我们追逐梦想,是一个同化和迷失自我的过程;选择小城镇,是把人性看透和认识自己、坚持自已的过程,深以为然。在我看来,事态的发展不由控制,我打算如电话里与他说好的,去学校仔细与他说这事说清楚。而在我,心里多少期待一场与09年与男闺密对话同样的效果,解铃还需系铃人,心结以解不易结。

我如约而至,只是我呆了不到五分钟。流浪者先说他秋季学期也不会呆在学校里。我反问他,那你博士读完了,拿到学位了,这么快?他说不,他拿不到暂时也不拿学位,只是读累了厌了出去工作一段时间再说。当这个悠哉乐哉经年累月被储藏在basement的计算机系、学校级一级保护动物,我很意外。他或许真的变了?我的观察变得考量,当自己亲眼目睹猥琐男之猥琐现状,他不愿意借任何钱给我,反而想要与我有更多的联系,比如典当我的物品,一台老旧电脑之类的事情。意识到他被笼罩在剖析和考究下,他说借我钱也可以,要我把手机押在他那里一个晚上!呵呵。。。。回想电话里那一个个生生爆米花般崩出的"We",回想这几年来学校里系里对流浪者要TA给TA要钱给钱要项目给项目,这哪里还是作为人的最朴实的情感,心里暗骂自己,呵呵,你太天真了!我不愿意在某个问题上再有任何纠葛。若我"解"不清,我还可以了结。问题是,流浪者的问题也早已不需要我解或结,因为E,从E那里获得了暗黑属性的流浪者,早就有了来自系里和学校的强大支撑与倒向。倒是我自己,该滚回去好好想想,该如何交学费吧。

回来的路上,回想那些与我有着或多或少交集的同学们,除了系里想要促成与我男女朋友的男闺密(拿到学校某老师的一万多块作项目的钱),哪个逃脱了恶运?tictactoe与我同写出来作业的那同学的爸爸是剑桥毕业的,这同学实时操作系统课时第一次期中考试被老师打了67分,drop掉了这门课,到EC再与我同学时已经时时公开表明与我对立立场;编译课上同桌的"你"的小帅哥最后一学期已经不敢再选编程课。。。。

回到家后,见山不是山,见水不是水,看见学校主面上当天下午立即摆出了"thank you"照片,意思仿佛在说,你谈恋爱了,读完早点儿滚蛋吧。。。。自此,学校主页和linkedin上频频更新的event成了我的心咒,这个春季学期来自学校的浩浩荡荡的"暧昧"隐晦黑把我这样一个幼小心灵生生折磨成了略带神经质。。。。

\chapter{官司之混账体流水账}
\label{sec-40}
\section{官司之混账体流水账(1)}
\label{sec-40-1}
在国内时,我很年轻,一路走来,至少那时还走得风调雨顺。小混混的内心包裹在平时成绩可能不好,但关键时刻使点儿劲努把力也都能取得不错成绩的躯壳里,比如考研考托考GRE。

不得不说,对于我这样在国内大环境风调雨顺的三好学生来说,拿着F1学生签证在这样一个曾经的梦想中的国度生存着的自己,有着晓松老师总结出的北美华人最本质的特征一一低调,什么事情能包容就尽自己最大努力包容了,绝不惹事生非。所以对于招惹、触碰美国法律,我也有着最本能的低调和忌讳。而现在,自己却不折不扣地滩上大事了。

12、27在表哥系楼旁两警官谈话的时候,有一个八字眉的警官后来说是有file permanent protection order的,所以我被逮捕了。而我真正放进卡里所有钱的\$500把自己bond出来时,看守对我交待的是,现在(他说话那时)的系统里是查不到这个permanent protection order的,所以要求我1月5日的时候打电话到表哥所在的小镇上的court,询问这个permanent protection order到底file没有,以及接下来的court安排事宜,如果需要出庭的话。

先前在买买提的首页上我有看到过一个学生出事了,极有可能也是最后一个学期,然后他的同学或是亲人在网站上发贴有助,希望大家能够帮忙petition他是无辜的。我从前天不怕地不怕,因为我心中有着小小信念,法律应该是永远公正的,法律同样也应该有酌情作轻重处理的人文关怀。

我怕,是因为事前我没能获得必要的信息。这同样是自已在这里学习快要结束的时候,事态发展严重的话,被遣返都是有可能的,可是谁愿意无辜接受被迫害、被遣返的后果?我怕,所以我要自救,我采取了和先前我所看见被迫害学生亲友的作法,跑到网上发贴去了。

1月5日,我如约打电话到表哥所在小镇的court,被告知我应该打到county court,于是打回county court,接线员说,是的,这个case现在已经被file了,出庭日等相关信息已经被寄往家中。因为我先前所缺少的permanent protection order极有可能被别人赖账说我收邮件丢了啊什么的,我已经有了这样的心理准备,所以电话里立即与接线员确认她寄出去的地址,她果然把材料寄到我前一学年所住的校园家中了。我告诉接线员我现在的地址,请她电话里先亲口告诉我接下来的出庭日期(她回答说我接下来的出庭期是1月14日早上9:00在county court,就是那个11+17迈远的地方county jail所在的小镇上),再请她把先前寄过的材料再寄一份到我现在地址。

2013年春季court季我住off-campus,与一孟加拉国女孩住share two bedroom apartment。2012年8月我搬进去时就去邮局取过新换过锁的邮箱钥匙,孟加拉国室友也不懂为什么这里邮箱的钥匙需要一年一换,但我们的邮箱的确是有钥匙,不可能丢邮件的。但是我现在与室友合租的地方邮箱是没有钥匙的。住在这里已经一个学期,我也网购过一些小东西,比如ebay上从国内寄过来的精美便宜的小鼠标垫等,从来不曾丢过东西。但那两三个周,我的邮箱材料一直在丢。

C银行的信用卡大概去年11月底的时候过期的,当时我不知道。等我意识到的时候打电话,银行说因为我没有及时更换那个银行的家庭住址,所以寄到我前一学年所住的校园地址去了。于是他们会把那张卡废掉,再帮我重新寄一张到我现在的地址。我的电话是28、29号左右打出去的,我期待1月10号前一定能收到卡。从4号起,我每天去查,每天都收不到。平时暴满的邮箱那两三个周也显得特别地空。10与我再打电话给银行,别人说这张新订的卡1月6日就确认deliver到我的新地址了,那就是说我一定丢失了这张新订的信用卡。没办法,我只好电话里请她废掉最新的一张卡,重新帮我再一次地寄出一份新的。到1月14日出庭前,我也不曾收到纠正了地址的新的来自court的邮件,我只得前一天再打了一次电话确认了一下我的出庭日期和时间,第三天早起上庭去。

1月14日我上庭的时候,法官出面前都是processer帮忙先安排一些律师啊定夺一下案子什么的。所以那天我没有等到法官出现,就被告知说,就我这个permanent protection order的事,他们目前还没有收到来自警官的材料,也没收到来自表哥方的材料,所以他们也不知道该怎么处理,有什么别的好办法,就只能帮我安排了一个律师,要我与律师联系,等到下一个出庭日再作定夺。13年春的court季,我两个出庭日之间从来不曾间隔一个月过,这次,下个出庭日被安排在了2月13日,一个月差一天。LinkedIn上学校装模作样的Events纷飞,这个拖延时候的出庭日还真是拖得够有效的。那天返回家前我从前台窗口要了先前我不曾收到的所谓的出庭材料一一半张小张片,上面有个出庭日期。

后来1月14日我上庭后,重新第二份订的C家信用卡到1月20日左右的时候终于到手。而也就在这20号前后两三天我终于收到等到了邮箱里早该收到的出庭材料。事后的我有时也会禁不住去想,那两个周的突发丢失邮件,是纯粹的意外,还是机缘巧合?
\section{官司之混账体流水账(2)}
\label{sec-40-2}
先前13年春court季分配给我的律师是个有些年级和经验的男的,2013、3、7日被我问及告诉我court结果那个文件执行效率的有效期是一年。这次1月14日他们分配给我的律师是个中年女的,稍微一点儿胖,比我壮两圈儿吧。那天走之前与她打过招呼,她说因为要等从警官那里拿到些材料,我与她约见面的时间太时,她极有可能还没有收到材料,所以没法真正了解我的case,她建议我将我们的见面时间约在二月份的前一个周。后来回家后,我们约到2月5日下午2:00pm. 律师有提我可以直接打电话给她,但是鉴于我是国际学生,语言沟通方面可能潜在存在的问题等,我自己一口否绝,坚持说我会去她离county court不远的她办公室见面面谈。 

所以那天,我去后她刚结束一个小时的电话,小秘让我给她五到十分钟休息时间,然后我坐到她办公室里。律师说她收到了来自警官的部分材料,但是关键材料一一就是有警官serve我permanent protection order的证据材料还没有拿到。所以以她的经验,我接下来可能有的走法就有四种:

一若是能收到来自警官的证明材料,若是他们没有serve我,那当然是最好的结果,我无罪释放;

二若我不等来自警官的材料,而且直接大事化小、小事化了地承认我有罪,我会被罚\$500(还是\$350,具体数字我不记得了),这个会像驾照超速会有纪录、会影响insurance等一样会保有纪录;并且我必须保证终生时间不得再见此人,她会帮再set一个lifetime permanent protection order之类的东西!

三若我不等来自警官的材料,而且我坚持不承认我有罪,那我也还可以申请走陪审团的路子,就是有五六个人一起审案,这个周期律师说至少要等一个月,那么这个案子的结案时间会需要再往后拖至少一个月;陪审团审的结果,同样没事万事大吉,若是有罪必重罚;

四若是能收到来自警官的证明材料,材料能够充分证明,警官有serve我permanent protection order,那我的后果将非常严重,绝对会被重罚,多到\$5000,和多到一年监狱时间;

我对律师提出的这些统统不感冒,我就知道我是一;二的终生不得见此人,我不能明说我一定还会再见到此人的,但要保护自己,我必须为自己找别的借口,比如一生的时间不得见,而且还要set到法律的效力上,让自己感觉终生为奴不得释放,我不愿意背负这样沉重的心理障碍过后半生!三的周期还在拖长,对于那时接下来人生路的几种选择之间让自己很是挣扎,到底是该早了解此案,越早越好,还是争取对自己最好的结果?那时我又如何能作出所谓真正正确的选择?四我认为是最不可能的结果。

前面提到,特定的那两个周我一直在丢邮件,事后的几个周我该收到的材料一份也没有少过,比如两次从信用卡里兑出来的各\$50、网络账单、医疗账单等。与律师的谈话也有了几许这种味道。

我当时想,等来自警官的材料是一定的,我不能随便让自己背上不属于我的罪名。但律师面前我也忍不住对律师发泄我的不满和鄙夷不屑说,"I know here in U.S.A, pretty much everything needs a signature. When the policeman serves a file, doesn't it need a signature at all before it becomes effective?"律师给我的答案也够狠,她说在美国,有些人是别人押着他签名他也不签的,没办法的时候,警官就会直接代替他签名也是同样具有完整法律效力的。我心里忍不住骂这坑够黑的,嘴巴上却也只能说,呵呵,"I know the policeman didn't serve me the file. "因为只有一次那两警官到我家,2013年1月17日后的第二还是第三天的晚上,两个警官去过我家,但是拿给我的材料是表哥复印件的temporary protection order的11条罪状,两三页的材料要求我2013年1月31日去表哥所在的小镇去上庭。那天我精心打扮过,想要再见到表哥,但是表哥那天没有出现。

我追问过律师,警官有没有说到底什么时候你可以收到来自他们的serve我permanent protection order的证明材料?律师回答说按照他们之前给予她的知会,她是期待这天(2月5日)就能收到材料的,但是在她见到我的这个时间点上,她so far还没有收到。所以我们必须得等。律师说因为也就下个周的周五就开庭了,所以这期间不用再见她了,有什么结果和情况,她会在2、13上庭的那天告诉我的,也就不在话下。

后来2月13日的时候不等见到法官,他们指派给我的律师继续说,她来没有收到来自警官的材料,所以我们还是必须得等,并且她帮我把我下一次的出庭日期设置成了这个月的月底2月27日。
这期间,我心里也是在七上八下不停打鼓的。看见网上有案子说有被罚两三千的,就像开车没有有效驾照或是开车不买保险,我们这里动辄就是\$500,这是我在2013年3月7日的庭是亲眼见到法官处理了几个case的(我那时的律师把我安排在他们之后,大概就是让法官体谅我这个仅被罚\$50的case吧,否则我会被罚少于\$500吗?)。只是这两个星期里,我在买买提首页上再次读到有经验人士对梁警官案的深度剖析,我开始认识到陪审团的作用可以最大限度地避免地方法官作弊,所以随着时间的推移,对周期至少一个月的陪审团处理不再存心理障碍,心想着到时如果警官出具的材料对我不利,那我就选陪审团吧。

后来2月27日,警官拿不出材料证明我已经被serve了permanent protection order,所以仅就触犯permanent protection order这一条,法官假定我姑且无罪;因为2013年春季时我的前律师错误地告诉我的有效期是一年的,我还是被罚了\$150。那天在court里,法官的小秘帮serve了我permanent protection order,这份permanent protection order的有效期至2017年3月21日,从2月27日那个时刻起具备法律效力。姑且无罪是这么假定的,载至2017年3月21日,如果我不曾触犯任何法律,不上庭,至2017年3月21日,这个案子自动归为无罪;但若是我有任何事情上庭,这个案子会被自动触发激活,律师要我到时一定请律师来处理,所以我的脑袋架在脖子上还是悬着的。。。。
\chapter{食堂心经}
\label{sec-41}
\section{食堂心经(1)}
\label{sec-41-1}
我想我还是太幼稚太没有经验了,前面写官司的时间虽然纪录了不同时间段发生的事情,但段与段交接的时间点我却忘记没能点清楚。这篇食堂相对独立开的希望能写清楚,并力所能及地补充官司里涉及的必要时间点。

因为我导师说我或许可以从IPO那里得到部分经济补助,但秋季学期最后一周等我真正去问及此事,IPO国际学生负责人告诉我IPO没有任何能够帮助我的,所以我春季就只能在食堂打工。所以春季学期来临,估算着他们最迟周日应该就开门了,我去找他们(1月11日),原来1月10日周六他们就已经开门开始准备开张了。这个与负责招聩的总负责人,是个个头非常矮的很干练的女负责人,说过之后,基本就能够确定下来,他们是需要用人的,所以要求我1月12日周一下午带上I-20,护照和SSN五点去参加safety培训。一个半小时的填各种相关表格(电话联系方式等)、看vedio后便是各种复印证件、和打电话索取一个什么工作号,这差不多也忙了一个小时。复印时我的两本护照(2011年过期时在旧金山更新过一本,但是旧本上有I-94表,所以任何时候需要用护照我都抱两本)。培训后负责人要我们周五(1月16日)打电话给她确认第一天的开工时间。后来接着第二天周二,计算机系的小系就如同秋季结束时发出的TA邮件一样,再次穷凶极恶、跳梁小丑般地发出怪异的TA邮件,我也就再不去理会这些了。待我那天周五早上打电话过去,负责人才说我的I-94表过期了!这是什么借口,我就只有这么一个I-94表,当年申请硕士实习都用这个,从来没人抱怨过期过,怎么到这里会过期?于是跑到学校同负责人解释,它就是这个样子的,我不需要申请或更新或是作任何的工作;然后负责人再推说什么什么负责payroll的人出差了,所以没办法process我的入注文件,所以必须得等。。。。别人说必须得等,我还能有什么办法呢?除了陪尽礼貌和小心,尽量不去得罪人!后来经历了与流浪者的那场实属意外却又是他们设计之中的见面后,我大致猜测,他们是要等学校主面和LinkedIn能够打出这场舆论之后才允许我工作,所以最终,我开工的时间就变成了再一个周后的1月23日才正式开始在食堂里打工。

虽然12月底因为自己滩上事儿了我爬到网上去求助,他们还是(临时,或非临时,又或者2013年已经非正式地?)file了我的case,但我的求助还不曾大面积铺开,所以我一直被攻心一一邮箱里丢失的邮件、律师给我问题的答案等。但自2月14日我从网上全面拉开自我防卫,那case没法再做下了,因为我把这所有的写出来,这个地方法案已无颜面见天下,所以才有了我后来2月27日的头架在脖子上走接下来两年路的处理,这在他们,已然是对自己付出了最大的仁慈。同样,学校食堂里就是另一种的处理方案。开工第一个周,我似乎还是这个食堂里的香勃勃,但之后,我便就成这里舆论的火山发源地,和这座食堂里的苦力。
\section{食堂心经(2)}
\label{sec-41-2}
中间有TA的这一年时间我不曾在这里工作,这次回来,就能明显感觉到这座食堂变了。这学期在这里打工的女生,或面相姣好但身高极矮,或正常身高但肥头大耳,或者个子很高却也生得五大山粗,加上食堂负责伙食的负责人帮我领了XS号的黑色上衣和M号的裤子,相对于另几个国际学生女生中大号的上衣来说,行走在食堂的大道上,我的肤色、长相和身材相对比较起来就变得有模有样起来!再看看这座食堂里打工的男生,或高或矮或肥或瘦或帅或丑或有个性,就像我曾经呆过见到过广西的山,俊俏婀娜,千姿百态起来。我记得13年秋季时我主要被安排在classics,就是平时学生打饭的主要滩位,但极简单,有左右两侧,但一般情况下只有一侧serve食物,而且一侧只容放三个大餐盘,相当于米饭、肉和菜的所在地。人基本可以站在那里不动,所以7:30pm关门后收拾打扫卫生也极其简单和方便。那时我干完活,一般都会帮拖中央大厅的地。这学期,我只有前一两天被分配在这里,那时候,我在这里为学生打饭,负责人还过来帮忙,greeting的话"How are you today?"负责人也都是讲的,所以我也就很热情地做好自己的本职工作。可接下来的一个晚上,当我看见他们安排一个美国小女生站在那里打饭,而且全程不与学生说话时,那天晚上学生自动不愿去那个她所在的classics领取食物,和接下来一天下午正常的四点多钟的会议时,manager强调这里是工作,不能flirting时,我终于明白别人的旗帜是在如何摇。倒是这天,这个完全受精英领导的社会、学校和食堂里的学生已经领会了上层指示,积极主动去那女生的滩位取起食物来。自我明白这一点,这个学期我尽自己最大可能避开与任何异性的目光接触,路上走着遇见manager都要主动把头扭开。

而接下来,我就像被钉了钉子一样钉在了位于食堂中央的salad bar里。这里salad serving的样品多活多,classics有两个大餐盘位,这里有四个。正常情况下这里需要放生菜、波菜叶,6种小菜、再6种小菜、胡萝卜,dressing,strawberry topping等3个,和10样yogurt和灌装水果等。关门 打扫卫生时更是也还需要搬下面的冰等,是所以滩位里人们最不愿意干的活。一般情况下,打扫卫生收拾负责人都会安排另一个人帮忙,但他们极少安排人来帮我。即便是安排人,也一定是男生,试图制造与我的诽闻,所以我一般都会避免与他们有任何近距离接触。

但即便我做到了这样,禁不住别人会主动往你身上推和炒诽闻。

当LinkedIn上post出男女两个异性一起工作的一张照片的时候,他们就在那天晚上把我安排到了包面包的deli这个滩位,因为来要面包的学生众多,而且速度慢,他们就安排一个高个子的男生帮我,而在整个晚上帮我的过程中此男生都尽最大努力往我身边靠,非常烦人,但我也都在不失礼貌的前提下最大限度地避开此人。

当我在LinkedIn开始添加新联系人时,我注意到我的mentor也加了一个新联系人,而这居然也就被网站大作文章,即便是在秋季学期在极其恶劣E领导的team environment条件下,我已经把自己该写的GUI写出来,还作了signal与slot的联接已经可以上色,因着网站上之前的炒作,mentor与我还是被怀疑认为我们之间的关系不单纯,进而去推断我实习所得的劳动成果含有水份。他们继续了工作环境下的flirting这一论调,诋毁我们毁坏公司的企业文化。我的mentor是一个working professional,可以合理猜测,他自然也能够感觉到这股风暴,进而作出相对回应,所以这段怀疑和风暴但最终也就不了了之。

食堂里也主动发动过一场event,大家一起都晚上almost 7:30pm (具体时间大概是晚上7:15pm),我的座位旁边就被强行安排坐了一位"King"一一戴着一顶特殊的帽子。因为到了最后的时刻,所有之前没能take break的几个人就一起坐到了一张大桌子上吃饭。他们只有15分钟break的两三个人走后,也就只剩两三个人了,国王还在。我忙着吃自己的东西,另一个小男生说他反正接下来要干的事很少,不如在这里多坐会儿,可能是怕我们只剩两个人尴尬吧。国王想要赶走另一位不明所以的小男生说,他可以去干什么什么活。小男生稍微离开,只可惜在我这里,这样在沙位巴工作了一个晚上的我,食堂里任何简单剩余的饭菜都来得比任何一个被人为刻意安排出来坐到我旁边的king对我来说有吸引力多了。于是他们15分钟break的人返工走之后,尽管只剩下king坐在我旁边,但我与禽劳勇敢的美国小镇人民的情谊那时还远没能激发我与邻为善、友好交流的觉悟,于是傻乎乎的我就只顾着自己傻乎乎、聚精会神地专攻美食了。走后的小男生大概是看不下去如此这般,重新返回来同国王聊天。塞翁失马,焉知祸福?第二天我意识到傻人还是有傻福的。。。。

这个学期,这个食堂,这里成为了就像国内火车站一般的流动人口的天堂,闲职人员众多。就像我们出国来的学子考完GRE后申请这边的学校有一个想要与这边教授建立联系的所谓"套磁"、"碰磁"的过程,这学期的食堂里,我也体会着类似被碰磁的过程。当我take 30 min break时,他们试着安排不同的男生,很多时候这个男生一般都会(或故意)不负责任,外面食物没有了也不帮添加东西。到头来一个男manager还假惺惺地帮我,对他们这个学期我学会了冷脸,manager自己故意挑事端把他的手套扔到垃圾桶里扔尽了他所有的力气,我不明所以、极为同情地看着他的背景离去。后来后面我就直接对manager说,我take break时帮cover我的学生不负,这是他在制造问题,如果他负责真正帮忙了,我不需要你来帮我我也能干得好好的,只差明说你的手套扔得莫名其妙,是能力不足的表现。他的事端没能挑到多大,也只能安排了那男生来向我道歉。没关系,这些都不是事儿。至于说其是留动人口,是因为很多个套磁失败的对象就再不曾再来这里工作过,包括几个套磁失败的manager。

这个春季LinkedIn上的贴子转载和like也是千奇百怪,打M与我的舆论时的唱歌声、小孩的哭声,声声入耳!party照片、baby照片,食堂里的面包照片,片片攻心!前几天甚至还出现了面包师傅?这三个人的小team又与我有什么关系?

前面提到在官司里有丢邮件和与律师对话的攻心,在食堂里又何尝会少了这些?

我被安排到沙位巴后,因为要补满一些东西,常常看见装东西的大餐盘脏了,我就会换一个新的装,把用脏了的拿到厨房去洗。后来接下来一天负责人让一个中国女生解释给我对我说厨房要洗的东西太多了,要我把拿进去待洗的餐盘先自己用水龙头冲一遍。卫生是很重要的,冲一遍就冲一遍,没有什么大不了的,不在话下。后来一次meeting上讲到contaminating,回想自己自两年多前在食堂里工作以来用保鲜膜封装食物时都喜欢用手不带手套,既然这样,就改过来吧,以后晚上收拾东西用保鲜膜封装时都戴上就是了;然后一天也有grill的人把辣椒粉洒到了直板热锅上,食堂里曾经咳嗽一片。一天中午去工作,一个肥男(他名字以J打头,就叫J吧)安排我说要我去帮他倒一个积满水的桶,因为已经要溢出来了,我不高兴,勉强答应,但也问他知不知道谁中午在这里工作,垃圾桶里的垃圾已经很重了也没有倒,他承认说是他中午在这里工作。然后他说他要教我如何倒垃圾。等来到食堂外,他话说得很重说,如果我连这点儿垃圾也倒不了,那也许我做不了这里的工作;这时我的气就彻底出来了,板着脸加重语气直接对他说,我能不能胜任这里的工作不是你说了算,不需要你来评价。这里他服软了,说我绝对能够胜任这里的工作,他对我的工作能力没有任何怀疑,我也就作罢。但进去后,J继续跑去对先前转告我冲餐盘的中国女生说我误会他了,害得中国女生来找我。我对女生解释说,我知道J与我是什么意思,我们已经说得很清楚了,不是事儿,让她不用小题大作。

再后来就是manager亲自直接施压了。农历新年前后,当他们关掉grill,关掉noodle bar,salad bar就会不堪重负,非常的忙,因为离食堂后面储备间很远,一趟趟地跑来跑去(走来走去,不准跑)是真会跑断腿的!而学生们也早就已经学聪明了,这种情况下,他们也会明白领导是什么意思,所以他们会拿比平时更多的量,而且掉到餐台上地上到处都是。里面我已经忙得顾不上来,外面这个site最大的官一一那个14年春天负责打电话给系里三两句话帮我解决two week notice period的司令,就拿着扫帚扫,司令一遍一遍地扫,学生们就一点一点一再一再地洒,司令与学生天衣无缝地配合,就成就了自己七上八上、坐立不安的农历新年。这种salad bar特忙的况状一直到后来买买提打舆论帮打出力度时才有所改观,作法就是增设别的人窗口来分流。

攻心不成,接下来的便是各种小动作,比如指责我这里那里作得不好啦。。。。或者把东西到处乱放。我来上班时一大瓶dressing被放到小冰箱上几小时过,也有一回一小瓶备用dressing被放外面几小时到我上班时不得不扔掉;上个周某天我两点去上班,餐台下cooler的开关居然会被停掉(又或者整个早上不曾打开?)到我7:30pm收拾时70\%-80\%的冰块都化成水了。当然也还有找渣的,比如别人会提着个纸箱来找你,说什么必须得把纸箱一有就扔掉啦,学生找渣的会说这个是什么名字啊,有没有什么成分啊,有没有哪个特殊的东西啦。。。。想要在这里打工,必须得有强大的内心才能经得起考验、不受累。

但无形中,我最最受不了的还是当买买提随学校舆论而动,打不出任何力度的时候。明知道我被淹没在学校设置的种种诽闻中,买买提这边似乎已经不再出力,那难解的心魔,在自己一次次地想要维持自己坚持等待表哥的立场的举动中,在学生们一次次受精英领导在食堂里成双结队大秀恩爱时,无形中,我变得越来越神经质。。。。

我后来想过,为什么他们把我订在这个沙位巴,仅仅只是让国际学生干更多更累的活吗?也是因为他们要盯紧我。而若是我下到地下室去取一些食材,都会有manager从另一个楼梯下去跟踪我。我的工作、脸上表情纤毫差别都会被他们拿来作文章。。。。 

\chapter{读博士?见鬼去吧! (Ph.D? Go to hell\textasciitilde{}!)}
\label{sec-42}
从开学第一天起,我就告诉过导师,我岁数大了,我读不了博士,只想能早点儿毕业早点儿去工业界工作。但回望这读硕士的两三年时间里,这个计算机系、这个计算机系的代课老师们,他们一直试图使用种种手段逼迫我读博。13年春天我前导师想要给我奖学金却还要装模作样,任何学校每年决定一次的RA名额,到了这所学校这里就变成了RA需要保有两年期效?13年夏天我有机会找到实习了吧,秋天一回到学校就被前导师用种种不公正手段打倒粉碎为那门课得C,将夏天受启发得到的一丝成就感与自信淹没在前导师强憾和一手遮天的海浪里!14年初夏,别人好不容易有机会可以出去工作,却还是被EC代课老师用授课内容阻拦拦截,一所学校,他何至于要这样对待他的学生?14年秋天关于选EC代课老师的robot课不还是一种赤裸裸的逼迫吗?就连这个春天,他们也丝毫不曾放弃过!

再来看看除读博外其它我在这所学校所受的种种待遇吧。2012年9月tic-tac-toe有着诸多要求的programming language第二次作业,我分明是班上仅有的一两个写出来的同学之一(另一个写出来的同学,他爸爸是剑桥毕业的,后来在系里代课老师的一再施压下站到 我的对立面),而我的前两次作业老师只给我4/27, 8/33分,并把他那纪录学生成绩的excel表格展示给同学看,让同学们中谣言四起,说我学习不好,而我还是那课堂上抢着回答问题而后配合他雪藏的人!后来14年秋天fault tolerant代课老师说我lie,但事实的结果是这位programming language的代课老师一直在lie,打压别人成绩、使用欺骗手段蹭别人没有上课的时间,把Android App Programming同学的表现加强,并又一次地造谣把不属实的课堂performance表现发到网上,邮件证据前面已经贴出来了。当Android App Programming后半截我把所有的进展都发到网上,同班同学被我甩掉很远,代课老师的偏心、对我的打压不是再一次地得到了见证了吗?在证据面前,读者不免再来看看,这到底是谁在lie?

编译课代课老师故意捣毁我的形象、故意拖延作业和考试的返馈情况,尽管自己被雪藏了一年多,还是必然地揭杆而起,为自已的成绩战,这学期这门课上四次主动发言回答问题,但我何尝逃脱了这门课得C的命运?

我承认,与自己的同学们一样,我们都很喜欢EC代课老师很直觉、比较有启发性的授课。可就算AI课我写完第一个项目代课老师因为偏见或他自己出于系里统战政策的执意,课堂上目光顾左右而跳过我、怀疑过鄙视过我,认为我写不出项目代码;可这门课的最后一个项目,作为班上唯一一个(顶多两个,如果另一个修过统计课的老美TA也写了这个项目的话)写decision tree项目的人,而且项目写得结果是正确预期的,代课老师会为他曾经的偏见向我道歉吗,代课老师会给我A吗?这门课作为我最喜欢的几门编程课之一,我不是还是只得了B?同样接下来春季学期的EC课,cs121硕士生讲师和cs210 programming language代课老师都不曾讲OO相关,这是系里的统战政策故意如此设计好让我迎接EC课最险峻的挑战?当我把EC课一个一个的知识点踢爆,关于函数指针、关于make、关于个体群体OO design,当我不折不扣、风声水起地能够保质保量地完成所有的项目,迎接所有项目挑战,EC代课老师却用一场考得课堂上学生们蛙声一片的一次考卷把我这门课的成绩再次强行降为B?

14年秋天的这段senior design的公案,真的仅仅是意外吗?我的课题项目midi controller也是被导师要求写一个GUI, 但我迟迟没能写出来,而是在midi读写上作文章,稍微明白了些原理。那时的自己何尝知道,我所缺的也不过是一个Qt Creator的大环境而已,又或者我所缺的仅仅是一个team environment的大环境而已!当我现导师以为我写不出来GUI,当我前导师作为唯一一个看过那年夏天我实习时所写GUI图片的人,他们联合系里策划了这段公案,结果呢?我的项目能力得到公认,但却也没能逃脱别强按到我头上的"无法合作"的罪名,被踢除到项目之外。此时,欲加之罪,何患无辞?

后来当我失去办公室里一个座位的时候,我真的曾经后悔过自己把那封长信发出去。但再后来,及至经历了这大半个春天,我终于明白,即便我不发出去那封邮件,我不会被项目踢出去吗?我不被踢出去,春天他们又要找怎样的理由和借口阻止我选cs481 senior design II这个原本就是一年的项目课,他们,可能会给我春天的TA吗?我春天的遭遇又怎么会有不同?

当我在这所计算机系里的成绩被一再打压,当我在这所学校这个计算机系里的成长时时受限,谁能告诉我,帮我找一个一位36岁的单身女人还需要呆在这里、在这样的破烂野鸡学校计算机系里继续读博士的理由?我找不到,这近几年来我从来就不曾找到过!

\chapter{11条罪状}
\label{sec-43}
\section{11条罪状(1)}
\label{sec-43-1}

\begin{itemize}
\item Ms. H (hereafter "H") has repeatedly visited me from California throughout
\end{itemize}
2011-2012, coming unannounced, uninvited, and later on, deliberately 
ignoring our demands that she cease and desist. She has encountered Pullman 
PD 3 times, and XU police 3-5 times. Despite this, here attitude remains, "I
don't believe you!"  

\begin{enumerate}
\item There were several prior incidents in 2011, but my records are at home.
\end{enumerate}
During this time, H lived and worked in California. Here behavior pattern 
was to sneak out of town without informing here boss, drive non-stop to X, 
visit/intrude for a day or so, then drive non-stop back to CA.

\begin{enumerate}
\item 2011/?? (late summer?) - (a) H snuck into my office in XU office (and
\end{enumerate}
took a nap on a cot). I snuck out without waking here, and went home (to 
dfhdfjkjk). H walked to our hourse and locked herself in my dad's bedroom to
sleep. My dad returned home and called X PD, who escorted here out of our 
home. At this time, I told here verbally face-to-face that I was not 
interested in her, and didn't want her. X PD coordinated with X police to 
return here to campus to retrieve her backpack (which she had locked inside 
my office). X PD trespassed here from my office. No further contact. 

\begin{enumerate}
\item Sat 2012/04/14: H appeared at X office, interrupting my work. I listened
\end{enumerate}
to here for 0.5 seconds, then walked away (down the hall). No further 
contact.

\begin{enumerate}
\item Sat 2014/05/12: H appeared at X office, interrupting my work. She lunged
\end{enumerate}
into the doorway 9which I was blocking, and grappled with me in an attempt 
to get inside, but I stepped into the hallway floor next to muy door. i 
called X 911, and X PD found here unresponsice and uncooperative. Ofc. Y 
placed here under mandatory arrest for 4th degree assault, and summoned a 
medical team 9with stretcher + ambulance. She was held in J county jail 
until here court appearance on Mon 05/14 (which I did not atend). No further
contact.

\begin{enumerate}
\item Thr 2012/08/09: A colleague in my new office location (X dkfdj) infromed
\end{enumerate}
me that a female matching H's description had knocked and asked for me. (I 
was in "my\textasciitilde{}me\textasciitilde{}" all day.)

\begin{enumerate}
\item Sun 2012/08/12: As I left office and walked toward the dfkljdf parking
\end{enumerate}
lot, I saw H appearing on the sidewalk up the steep hill (dfkld Ave.) I sped
up to avoid her, but she chased me down and pulled at my car door handle. I
drove off without her, and went toward my home. As I turned right onto 
dfkdjk just below dkfldj, she car-chased me briefly up dfkdjfkdjk, down 
dklfjdf, and part-way down dfkjdfj, where I eluded her. I drove directly to 
X SRC (gym) instead. Later, my parents recounted that she did go back to our
hourse, but they sent her away. 

\begin{enumerate}
\item Sat 2012/10/06: I returned home to dfkljdf to find H in my bedroom. My
\end{enumerate}
mother (retured) recounted that H had used a "doorbell ambush" to gain entry
, where she rings the doorbell while hiding out of sight from the interior 
windows, then pushes here way in once the door is open. I attempted to 
forcibly remove H from my room, but she physically resisted. I called X 911,
and Ofc. dfjdhjfh responded. We did not demand an arrest at that time, so 
they trespassed here from our hourse, and escorted her out. By this time, H 
may have moved from CA to dlfkjdfj. 

\begin{enumerate}
\item 2012/mid-Dec? H appeared at X office. I had forgotten my cell phone that
\end{enumerate}
day, and I have no landline office phone. I borrowed a cell phone from kfdjj
, and called X 911. H eventually fled, but X PD apprehended her outside. Ofc
. dfljdf trespassed her from X for the day. 

\begin{enumerate}
\item Sat 2013/01/13: H appeared at X dfjdfjk. I pushed here away from my door,
\end{enumerate}
then walked down the hallway while calling X 911. She fled before I turned 
around again, but X PD found here outside. 

\begin{enumerate}
\item Wed 2013/01/16 @16:10: As I left my office to give a class lecture (I am
\end{enumerate}
instructor for dlfjdfjk), I met H at the stairwell doors. I explicitly 
forbade here to disturb my class, but she followed me downstairs and sat in 
the classroom. I delayed class by 10 minutes to call X 911. At end of class 
@17:05, a X PD officer met me by the front door, while H fled via the back 
door. I think he chased her down. 

\begin{enumerate}
\item (ct'd from above) Wed 01/16: When my dad and I returned home \textasciitilde{}17:35, H
\end{enumerate}
emerged from my bedroom. My mother said she had used another "doorbell 
ambush" to enter the house. My dad sternly ordered her out etc. She dragged 
her feet and stalled for 7-9 minutes, during which I called X 911. I watched
out my bedroom window and described her driving slowly back down Sunset 
toward dfkldjf. Ofc. dflkjd stopped her there, and arrested here for 
criminal trespass. I don't know if she posted bond. 

dated Thu 1/17/2013

\section{11条罪状:因为爱情}
\label{sec-43-2}
首先要感谢2013年1月17日后第二还是第三天傍晚到我家去的两位警官,没有他们帮我拿这份源自表哥列述有我这11条罪状的纪录,也许我永远也没有机会明白表哥的一些心迹。

其实这所有的罪状前面的几次写都已经写得非常清楚了,但时间过了这么久,上次2014年夏天打算写出来,但因为这份材料没有被我带到加州去,终究没能成愿。现在,它一一作为爱情的一种见证,终于还是来和大家见面了。

下面,就分条再简要阐述一遍每条罪状的内容吧。

\begin{itemize}
\item Ms. H (hereafter "HH") has repeatedly visited me from California throughout 2011-2012, coming unannounced, uninvited, and later on, deliberately ignoring our demands that she cease and desist. She has encountered Pullman PD 3 times, and XU police 3-5 times. Despite this, here attitude remains, "I don't believe you!"
\end{itemize}

就当是一条总述吧,亲爱的读者,当大自己13岁的表哥一承认这份恋情便有违伦理道德的时候,您,能期待表哥会说出什么样的话来呢?

\begin{enumerate}
\item There were several prior incidents in 2011, but my records are at home. During this time, H lived and worked in California. Here behavior pattern was to sneak out of town without informing her boss, drive non-stop to X, visit/intrude for a day or so, then drive non-stop back to CA.
\end{enumerate}

2011年之前发生的事情,那就是垫定、铁定了感情基础的那场2010年12月的著名告别了。是的,在那场告诉里,我试图从表哥那里获取一些温暖,但让我千真万确感到意外的是,表哥竟然可以如此宠我!我从来不曾被任何人如此宠爱过,于是粘在别人的怀抱里,天真地希望他承认他也喜欢我,天真地希望表哥能够抱我,"表哥,就算是作为表哥,你就不能抱抱我吗?"我粘在别人的怀抱里舍不得放开,最后是表哥导师闯入办公室,我们才不得不将牵着的手分开。表哥送我出来时,他同样舍不得我走,把我送到了一错的门口,再次把我送出去时,表哥与我的手就没有谁先谁后、分不清谁分谁后地牵在了一起。虽然中间因为自己看见有同学害羞松开过一次,但再牵住表哥的手之后再见到任何人,我也不舍得把手分开。

说我回学校或来表哥这里不告诉老板,应该就是表哥故意表现不喜欢我的一种方式了。我2010年12月回学校办STEM延期时是同老板请假的,后来11年回去也都是同老板请过假的。就像前面提到,作为国际学生,对于法律和自己的工作,我还从来都是抱极其谨慎和认真的态度的,绝不敢大意。

non-stop back to CA是真的。因为开长途太累,很容易就使自己变老,我后来发誓过,不到迫不得已,我绝不再开长途。我付出如此努力,表哥会真的就没有一丝一毫的感动吗?

\begin{enumerate}
\item 2011/?? (late summer?) - (a) HH snuck into my office in XU office走廊尽头有个门的地下室里办公室 (and took a nap on a cot). I snuck out without waking here, and went home (to dfhdfjkjk). H walked to our house and locked herself in my dad's bedroom to sleep. My dad returned home and called X PD, who escorted her out of our home. At this time, I told her verbally face-to-face that I was not interested in her, and didn't want her. X PD coordinated with X police to return here to campus to retrieve her backpack (which she had locked inside my office). X PD trespassed here from my office. No further contact.
\end{enumerate}

在这之前几次到表哥家的过往就被表哥省略了,比如二月份因为电话里与舅舅一句话不合,我第一次主动回去拖自己之前留在舅舅家的箱子。表哥知道我要回去,我给表哥写邮件说,I have something huge I need to let you know.那次回去我对表哥表白过,表哥听我表白的眼睛发亮,但他还是拒绝了我。在超高里,表哥进门时领了一枚superbowl的戒指,随手拿给了我,那枚戒指直到现在我还精心保存着。超市里表哥带我找核桃,表哥拈碎两枚核桃后,大家都笑了。可是那天晚上回去后,舅舅要找我谈谈,我终于还是被舅舅的话气饱气跑了。

那一次在舅舅家,表哥的确有当着警官的面告诉我,他不喜欢我,但是,他说的这话又该让我如何相信?而且来到加州后,表哥的LinkedIn联系人增加了好几个,而且早在2010年的时候舅母就说过他们作父母的希望表哥就在家门口的工厂里能找到个工作就可以了,不希望他年级老大不小的远走他乡。所以我知道表哥建一个LinkedIn的网页完全是为我,在这之前我就搜到过表哥的主页,而且表哥到目前为止的五十多个联系人都完全是为鼓励我而增加的。所以即便表哥当着警官的面说过那样的话,我也就是从来不信!而且10年12月我走后因把自己的小长片洗脸毛巾落在了他们家,我写邮件请表哥帮我寄过来,舅母就帮我寄过来了;那封邮件里我不经意地问表哥,"Do you exercise nowadays? "等我下次回去拖表哥的胳膊,一拖就知道表哥变壮了,他的胳膊很有力,我拖不动。

那天在学校里时因我去洗手间洗涮时我的背包书包被锁到了表哥的办公室里,我不得不提着自己一个小小的化妆包一路踏过上百户人家找到走到表哥家里去,我敲表哥的房门,他不应,但我把稍微推开,就看见表哥短背平角短裤躺在床上,就像我给表哥写过的邮件里说过,"You are the most cherished cousin in the world, please don't drive me crazy."我与表哥之间有着特殊的亲密,他给我权利闯他的房门,他不需要我进他房门前敲什么门。

那之后我再续写自己的故事的时候,我提到过这天这会儿见表哥我非常遗憾,心里其实很想走进表哥的房门去,他穿着白色背心和平角短裤,长胳膊长腿都露在外面,我非常想走上前去,抚摸一下他的胳膊,但最终还是抑制住了自己的冲动,退出去躺到舅舅简易的床垫上与表哥隔着一堵房墙并排躺下去休息了。

\begin{enumerate}
\item Sat 2012/04/14: H appeared at X office, interrupting my work. I listened to here for 0.5 seconds, then walked away (down the hall). No further contact.

\item Sat 2014/05/12: H appeared at X office, interrupting my work. She lunged into the doorway 9which I was blocking, and grappled with me in an attempt to get inside, but I stepped into the hallway floor next to my door. i called X 911, and X PD found here unresponsive and uncooperative. Ofc. Y placed here under mandatory arrest for 4th degree assault, and summoned a medical team (with stretcher) + ambulance. She was held in J county jail until here court appearance on Mon 05/14 (which I did not attend). No further contact.
\end{enumerate}

那是自已内心里非常不平衡,小混混的恁性内心发作的一次。事后我为自己这次的不明理、不与警官配合付出了沉得的代价。2013年3月7日的court结束后,我收到\$832的救护车费用,和其50\%也就是\$416的AAI中介公司collection fee。而且因为自己在加州实习的两年多时间里一直不曾买过保险,所以这所有的钱每一分都是自己掏的。这,好歹也算是为地方政府作贡献了吧,毕竟,有着昂贵费用负担的救护车又怎么可能是人人随便都会坐的?

\begin{enumerate}
\item Thr 2012/08/09: A colleague in my new office location (X dkfdj) informed me that a female matching H's description had knocked and asked for me. (I was in "my\textasciitilde{}me\textasciitilde{}" city all day.)
\end{enumerate}

真巧,这一天里我们都分别跑到了对方所有的城市。我找到了表哥的楼上三楼新办公室里,他的室友帮答的门,我问了一下表哥在不在,他说不在我就只好走了回自己学校了。而表哥,同一天里一整天都呆在我所有的小镇。这一天是什么日子呢?正如大家所猜测的,2009年这一天我受舅舅邀请到他家去吃饭,这天我与表哥第一次见面。喜欢表哥却不大敢看他,妒嫉表哥舅舅待表哥这么好,待我就没有待表哥这么好,所以那天也就只是傻傻地吃了很多东西。到这一天,我们算是认识有三年了。多么遗憾,当初没有主动去多认识、了解表哥!

\begin{enumerate}
\item Sun 2012/08/12: As I left office and walked toward the dfkljdf学校里表哥计算机系楼旁的 parking lot, I saw H appearing on the sidewalk up the steep hill (dfkld Ave.) I sped up to avoid her, but she chased me down and pulled at my car door handle. I drove off without her, and went toward my home. As I turned right onto dfkdjk just below dkfldj比较靠近表哥家的街道了, she car-chased me briefly up dfkdjfkdjk, down dklfjdf, and part-way down dfkjdfj, where I eluded her. I drove directly to X SRC (gym) instead. Later, my parents recounted that she did go back to our house, but they sent her away.
\end{enumerate}

这里,表哥说了,他开车过猛,把我丢一边儿去之后,他又直接去学校锻炼身体去了。表哥经常去锻炼身体,我心里自然是欢喜的。

\begin{enumerate}
\item Sat 2012/10/06: I returned home to dfkljdf to find H in my bedroom. My mother (returned) recounted that H had used a "doorbell ambush" to gain entry, where she rings the doorbell while hiding out of sight from the interior windows, then pushes her way in once the door is open. I attempted to forcibly remove H from my room, but she physically resisted. I called X 911, and Ofc. dfjdhjfh responded. We did not demand an arrest at that time, so they trespassed here from our house, and escorted her out. By this time, H may have moved from CA to dlfkjdfj.
\end{enumerate}

这里说明一下,表哥的描述是有些过的。我开门时,舅母站在门后,也没有要立即关门的意思。而且,就像是满足自己的心愿,我想在表哥的床上躺会儿,我是有说给舅母听的,但是舅母在弄她的一个什么账单,所以相当于是默许了自己去躲。当然还没有躺到三分钟,表哥和舅舅就从学校回来了,但我的心愿是满足了的,表哥的床非常的软,而且注意到舅母给表哥床上铺的是蓝色床单和红色被面红蓝结合的,也是我很喜欢的搭配。

\begin{enumerate}
\item 2012/mid-Dec? H appeared at X office. I had forgotten my cell phone that day, and I have no landline office phone. I borrowed a cell phone from kfdjj, and called X 911. H eventually fled, but X PD apprehended her outside. Ofc. dfljdf trespassed her from X for the day.
\end{enumerate}

思念总是无所不在,所以我还是去找了表哥。这天,在走廊一头的与表哥的打斗挣扎中,当表哥抓住我一只胳膊的时候,我终于如愿用另一只手抓住了表哥的胳膊,满足了自己长久以来想要抚摸一下他胳膊的愿望。

\begin{enumerate}
\item Sat 2013/01/13: H appeared at X dfjdfjk. I pushed her away from my door, then walked down the hallway while calling X 911. She fled before I turned around again, but X PD found here outside.
\end{enumerate}

我应该是这一天藏在表哥办公室门外画了那个小太阳向日葵笑脸。最开始我以为表哥不在,画得也很轻,后来听到办公室内传出的声响,我就画得更小心了。画完后在外面站着等着表哥出来。后来见表哥又打了911,我就真走了。后来接近两年后的2014年12月27日,当我再去找表哥时,画在表哥门上挂着的白板右上角的这副自己的"大作"一两年来还被表哥精心地保存着,让自己感觉心里非常温暖。

\begin{enumerate}
\item Wed 2013/01/16 @16:10: As I left my office to give a class lecture (I am instructor for dlfjdfjk), I met H at the stairwell doors. I explicitly forbade here to disturb my class, but she followed me downstairs and sat in the classroom. I delayed class by 10 minutes to call X 911. At end of class @17:05, a X PD officer met me by the front door, while H fled via the back door. I think he chased her down.
\end{enumerate}

这天去听表哥讲课了,也是这天表哥去讲 课前我们清楚地知道大家都是爱着对方的,表哥把我一把推到了楼梯的阶梯上几天后回家照镜子看时磕了一条青梗。但那天接下来自己走路有些异样,还是去听表哥讲课了。我坐在后排,很安静地坐在那里,并不为打搅表哥上课,我只是很想他。

那天课堂上表哥说他也理喜欢CS专业,而不是EE专业,这话说着是讨好我吗?表哥后来下课时也不舍得下课,时间到了,学生们开始捡书包了,表哥还站在讲台上说着什么,直接警官到讲台上去找表哥,我赶紧溜走。

但我一定溜不走的。追上来的警官应该爱过吧,非常能够体会我的心思,对表哥表达出极为尊敬礼貌的态度让自已感觉很舒服。这才是自己真心喜欢的人嘛!

\begin{enumerate}
\item (ct'd from above) Wed 01/16: When my dad and I returned home \textasciitilde{}17:35, H emerged from my bedroom. My mother said she had used another "doorbell ambush" to enter the house. My dad sternly ordered her out etc. She dragged her feet and stalled for 7-9 minutes, during which I called X 911. I watched out my bedroom window and described her driving slowly back down Sunset toward dfkldjf. Ofc. dflkjd stopped her there, and arrested here for criminal trespass. I don't know if she posted bond.
\end{enumerate}

如果说前一段时间的联系,我们还能够努力增加对彼此的好感,那么到这一次,到现在,我们已经积累了足够的了解,我清楚地知道,表哥一定是喜欢我的,而我正有着同样的心思。那天傍晚,当我读到表哥这个复印件的". I watched out my bedroom window and described her driving slowly back down Sunset toward"时,终于是禁不住眼泪滚滚落下,禁不住问自己,为什么相爱的人必须经历这些,要爱得如此痛苦?
\chapter{是谁破坏了我们的爱情}
\label{sec-44}
那么与表哥的感情走到今天,究竟又是什么破坏了我们的感情?是社会制度,是美国作为一个发达国家其冰泠的法制国家机器破坏了我们的爱情。

从我这边来说,从去年秋天起,这个学校就发动各种舆论,试图把我葬身在诽闻的火海,系里刻意安排了M和流浪者两个人,而春季食堂整个就变成了拥有为我安排相亲第二功能的会场。我能不遭受这些舆论么?我能,只可惜我明白得太晚,至少秋季学期还不够明白,那时的自己还不是很懂得拒绝,至少拒绝得还不够彻底吧。可春天呢,自己够明白、拒绝得够彻底,自己都变成了别人眼中的异类、自已几乎都快变得神经质了,可效果呢?自己想要坚持的立场和意愿不是照样在这样与别人的冲突中被碾磨得连粉尘都快看不见了,自己想要坚持的立场和意愿能得到丝毫的声张么?

先从表哥这一侧作分析。从与表哥这所有的过往,大家不难看出,即使是在2013年1月表哥弄了一个temporary protection order,这也都有一定的加深递进顺序,也就是在temporary protection order与permanent protection order之间,我没有作任何事情,为什么表哥会直接从temporary protection order过渡到permanent protection order?可以合理猜测,这一定不是表哥的本意。而那个春季,我被两警官拿去了表哥深情款款的temporary protection order的复印件,独自参加了1、31、2013的court,而表哥则是独自去参加了3、21、2013的court(这个是2015年2月27日法官小秘serve我的permanent protection order上写着的,只有我这一方不曾出席3、21、2013的court),这,难道仅仅只是巧合?这台帝国主义的国家机器不仅从法制上严格地把相爱的人们分开,而且也还从舆论上、从递与我让我知晓的列有表哥11条罪状(这是type进去的,不是手写的)的temporary(permanent上就没有这些内容)、从1、16、2013听完表哥课的警官对于相爱的人心态的体察和对表哥的充分敬意与肯定,所以这一切让自己等待的暗示就是为了让自己无期限地待待、就是为人作贱别人的人生?

而就在2014年12月27日,最终得知他们还给我加了个permanent protection order的自己,何尝不是对这个国度这台法制的国家机器悲愤不已?我是一个人,一个有血性有感情的人,当别人加一个permanent protection order在自己身上的时候,作为一个人,一个在这个国家有着合法居留权的个体,我又为什么不应该拥有知情权?是谁在浪费别人的生命?是草菅人命的学校和帝国主义国家对国际留学生的赤裸裸剥削!

就像这个学期度算机系里还在大张旗鼓地想要招data mining相关的AP,问题是你若真有心思让别人学这个版块的知识,早作什么去了,2013年秋天我的导师不是故意使手段让自己没选成与计算机相结合的统计相关的课程么?这后来,2014年秋天不是这些自己真正想上该上的课都被系里强行不开了么?这个时候还有作这种文章整出这种打法,原谅我承受力太低、其惺惺作态只能让我恶心想吐,愿离开这里越早越好,那些这所计算机系里试图还想要劝我读博士的人,见鬼去吧!

他们这所作的一切不就是为了最终实现逼迫我读博士吗?可是,我有什么理由来读这个博士?你他妈的三年前早告诉我我被录取为计算机硕士学生的前提是我必须得读博士,你看我要不要读这个硕士?我的三年青春就是被这样不负责的前导师、代课老师、和这所为虎作伥的计算机系刻意人为摧残的?既把别人强行留在这里读了硕士还剥夺了别人的工作机会,我TMD倒也真想看看这种狗屁野鸡学校到底还能猖狂多久!
\chapter{我到底想要什么}
\label{sec-45}
人之所以成为人,便是他有了各种各样的情感。我已经背负着爸爸生病时没能照顾、没能出钱出力的巨大愧疚和遗憾。这种遗憾每当想起爸爸都还会时时刻刻折磨着我。记得爸爸刚刚离去的那一两年时间里,只要任何人提起与爸爸相关、可能与爸爸相关或是有类似经历的人或故事,我都免不了泪崩。在QQ摄像头视频里,妈妈对我讲起爸爸临走前的光景,妈妈怕我伤心轻描淡写地说,我侧倚在折好的方块被子上任凭泪流成河,后来很长一段时间里,妈妈再也不对我提起爸爸。直到最近,伯伯也离世后,妈妈轻描淡写地告诉我说,伯伯新离世,今年伯伯家你堂哥和堂姐也给你爸上坟了。

妈妈现在已经65岁,而且身体不好,2012年春天爸爸离世的不久已经确诊患有脑血管肿瘤,若是我还不出息,既没工作也不成个家,就像二姐在电话里直接警告过我说过的,"不只爸爸现在是你心中的遗憾,将来等待你、折磨你的遗憾还有很多!"

那么眼下36岁的自己,究竟想要过什么样的生活呢?三年前打算回来读这个计算机的硕士前我有接近三万的存款,但历经三年读完这样一个硕士,我花尽了自己所有的积蓄,还背负着两张信用卡分别\$2500和\$5000的信用卡债务,以及从男闺密那里借得的\$2500(我会这学期还他\$300),借贷共计一万美元。知道妈妈身体有病,知道自己身体有病也一直不曾治疗,我又如何还能如此在36岁的高龄在原本属于年轻人的青葱校园继续混下去?

姐姐问得对,我得好好反醒一下,为什么在这样的年龄,我还需要问亲人们借钱,为什么在这样的年龄,我似乎还没有独立生活、打理好自已生活的自理能力?我觉得星座上说得很对,我是一只羊,一只生于现在、活在此时此刻的羊,没有远见的缺点使得他没有作为,而一旦目标成为现买,他就让它随机而过,他无法长久地系于任何事物一一我太善于忘却未来了。是这种生于此时此刻的羊属性让自己能够锲而不舍地等表哥等到现在,而同样也是这羊属性导致了自己今天走投无路的某种必然。若我是一个跟着舆论走的人,或许我也早就结婚成家,享受着世俗里的种种幸福,只可惜,没人告诉我关于表哥permanent protection order的事,我便不知道,舆论于我便如过耳东风,吹过即散,我尚不能记起,这风到底有没有吹过?

将心比心,我若是表哥,我若做到表哥的份上,在43岁的时候还能回来读博士,那我36岁的年龄应该也义无反顾地去读博士,但,我做不到!我没有表哥的科研钻研精神去读博士,读博士于我更像是加在被压死骆驼身上的最后一跟稻草。我不喜欢读写paper,不喜欢作长时间跨越几年的大项目,我更喜欢享受一天的工作一天完成的轻松愉悦,和工作中会给自己带来挑战、需要自己去一步一步解决短时间内(一天,几天,一个周,一个月,几个月?)能解决问题的快感。读博士对于生于现在、活在此时此刻的羊来说,实在是天山路漫漫,我会读抑郁死掉的!

眼下,我究竟想要过什么样的日子呢?多少次电话里妈妈问起,我总是一再保证,我一找到工作立即解决男朋友问题,甚至最近一年多里妈妈逼得紧了,我直接告诉妈妈哄她说我让她放心,我今年一定结婚!

我从来都不是贪心的人,我也不过是想要找到一份工作,解决自己的温饱问题,还完信用卡里的借贷,以及所欠朋友的钱。能有一点儿积蓄,那么妈妈生病或是自己生病的时候,至少大家都不至于慌乱。

当与表哥的这段爱情在帝国主义国家机器的碾压下走投无路时,当作为国际学生持F1签证在这里无法生存下去的时候,我也就不得不选择主动放弃。我又能怎样呢?就像电影里放的,一个穷小子爱上了一个姑娘,可当姑娘病重就剩一口气,穷小子除了把姑娘抱放在富家子弟的门口让他们去救她,自己大雨淋漓中暴走暴跑来缓解对自己的痛恨,穷小子又还能做什么呢,真要眼睁睁地看着她死么?此时的自己,与电影里穷小子的境界又有什么不同?!

看过大话西游的人都知道,留在他心里的一滴泪。若你享受过别人的苦恋,自己却绝尘而去,那么过后你想起来,必不会是窃喜,而是一生心痛。因为那个人必已在你心里留下一颗泪珠,如砂在眼,如梗在喉。你不动心则已,一动心,这颗泪珠就会时时折磨着你,让你难熬疼痛。世人有语:"宁可人负我,不愿我负人。" 就是怕有这样一颗泪珠吧。我是一个在缺少爱与关怀的环境中长大的孩子,那种在四五岁时便认定爸爸错了,从此从心理上情感上疏远爸爸,直接十几年后的98年高考,经历一番重创的自己重拾亲情,体会爸爸对自己的爱;随着97年舅舅对另一个国度的介绍,九年后的06年我终于坐上飞往这个国度的飞机,再三年后遇上一个表哥,一场离别,他仿佛给了今生想要的所有宠爱,在表哥给予我的爱与关心里,我仿佛重新拾回爸爸曾经给予过我的深邃如山的爱,在表哥的爱里,我一直是幸福的!从2010年到现在,我心甘情愿地等了表哥这好几年,等到2017年3月21日也绝不会是问题。但还是请大家去相信,表哥不曾在我心里留下一滴泪吧。。。。

只是,当自己在这个国度无法生存下去的时候,爱情难免过于奢侈,出于来自家庭和亲人的压力,我也终究还会抵不住他们的劝导,迫于压力,匆匆嫁人,休谈爱情。所有正畅快淋漓享受爱情的人们啊,也请你知道,这世上还有表哥与我这样一类人,他们的内心其实也拥有着最美丽美好的爱情。只是,命运已经为他们做了选择:他们这一生,注定要这样以最卑微最平凡的方式度过。花儿自会绽开,在最深的夜里,在独处的时候。请您努力去相信,我TM足够强大,进可取之,退得全身。别问我是否会在梦中的草场哭泣,别问我午夜梦回,会想起谁。我是一只鱼,我只有七秒记忆!

既然2015年3月31日是申请H1B工作签证的最后一天,就把午夜凌晨24:00为这份旷日持久的爱划一个终点吧。我没有合适的工作机会不得不放弃这段感情,是我自己主动放弃选择了退出,痛苦当然属于自己。愿风沙把这消息带给表哥,愿他在接下来的岁月里能把我忘掉,记得一定要过得比我幸福!至于我,如今的社会,谁还会去关注人的内心?我过得幸福与否,与其它人有什么关系!

那一夜,请别为我哭泣!
\chapter{梦见爸爸}
\label{sec-46}
在白皑皑的房间里,一个人帮我用一个脸盆接在床沿下方,我使完全身所有的力气强撑着身体,一阵又一阵涌上来的呕吐,让自己几乎喘不过气来。

我抬头望过去,我以为是表哥在照顾我,原来是爸爸!爸爸帮我拿来温水,我漱过口后,便重新瘫倒回病床上。

我头枕在爸爸腿上,时睡时醒,意识时有时无,朦胧中记起,那年临出国时二姐和姐夫以及小侄女一家把我送往北京送上远去的飞机。在开往北京的列车上,那时11岁的小侄女一路上与姐夫父女间嘻笑怒骂、亲密无间,那一路上看得自己心里酸涩,对小侄女拥有这样的父女情很是羡慕嫉妒恨!却原来,此刻,爸爸也陪在自己身边,禁不住心头一暖。

"我初三毕业即便我已经知道自己取得了全镇第一的好成绩,爸爸你和妈妈都还愿意顺了我的意,请我初三的代课老师们到我们老家去吃顿饭。我一直像你和妈教我的那样,心存善良,从来都没有变过,在国内时献过三次血,在美国这边也曾捐过一部旧车for the blind,以后也不会变,可为什么在这样一个我曾经一直以后那么向往的国度里,我却会遭受到来自代课老师们无数不公正的打压呢?我甚至还一度被他们雪藏了两年!"

"我就想不明白,为什么我明明无罪,明明是前律师告诉了自己错误的有效日期,自己才去找表哥的;自己明明无罪,因为那之前还不曾被serve permanent protection order,却还是要被判成姑且无罪?没罪就是没有罪,还姑且无罪,这就是架在国际学生脖子上的一把特殊的钢刀吗?还是这种"死缓"方便他们有机会将来再多身上剥削出更多的金钱?"

"是自己学校的学生,想要得到校外的帮助,别人实习公司都愿意帮忙了,但被却学校阻止了。被学校阻止在这里多呆一年的自己,学校也没有真正教别人更多的知识,而是在自己专业能力已经比较强的前提下,最后一个学期还是变成了part time,更多的时间被迫用来作苦力打工了,而没给我作TA更多的从专业上去发展自己的机会,这样的学校真正替学生考虑过吗?他们什么时候又真正替我考虑过?学校计算机系里大牛已经说了,没有人阻拦我拿工作机会,那实习公司呢,他们曾经是真心帮我吗,还是也与学校把我骗到这里读硕士一样,惺惺作态?"

"这个学校和计算机系简直如跳梁小丑般的丑陋,让人忍无可忍。这又一年三年的学习,终于是给了自己一个对这个计算机系和学校恨的理由!Dad, to be honest with you, now I felt ashamed that I have ever donated 1000 dollar to this current University!"爸爸像我儿时一样抚摸着我的头说,"孩子,你受苦了!几年前,你看不清;现在,都看清楚了?回去看看你妈吧,他想你!"

说罢,爸爸把我放平到床上,起身就要转身离开,我忍不住使出全身的气力喊出声来,爸,你带我走吧!我已经无法和表哥在一起了。你不要把我一个人孤零零地留在这世上!当爸爸烟一般瞬间迷散,我终于再一次地忍不住崩溃大哭起来。

睡梦中哭醒了自己,生病了一直在流鼻涕是真的,凌晨一点再吃一次药胃不舒服翻滚着躺到三点终于是呕吐把胃吐空也是真的,只是爸爸那个年代是学讲俄语的,我不该对爸爸讲英语才对,没想到这刚刚躺过去的两个小时里爸爸来过。回想梦中的点点滴滴,终于是觉得这人世间如此的苍凉,与其醒来,不如睡梦中还有爸爸陪着自己!

\chapter{details that I forgot to write before}
\label{sec-47}
解释serve:就是正式的人员(警官、法官、法官小秘、律师等)以一种正式的方式或场合拿一份文件给你,文件未必需要你签名,但只要别人把这个文件真正给了你,那这份文件就开始产生法律效力。
解释bond \$480,因为我有收据,他们不应该只还给我那么些钱。最终凭着自已的收据要回了自己的五百元。
% Emacs 24.3.1 (Org mode 8.2.7c)
\end{document}