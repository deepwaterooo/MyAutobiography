% Created 2015-12-12 Sat 15:42
\documentclass[12pt]{book}
\usepackage{graphicx}
\usepackage{xcolor}
\usepackage{xeCJK}
\setCJKmainfont{SimSun}
\usepackage{longtable}
\usepackage{float}
\usepackage{textcomp}
\usepackage{geometry}
\geometry{left=1.5cm,right=1.5cm,top=2cm,bottom=1.5cm}
\usepackage{multirow}
\usepackage{multicol}
\usepackage{listings}
\usepackage{algorithm}
\usepackage{algorithmic}
\usepackage{latexsym}
\usepackage{natbib}
\usepackage{fancyhdr}
\usepackage[xetex,colorlinks=true,CJKbookmarks=true,linkcolor=blue,urlcolor=blue,menucolor=blue]{hyperref}


\lstset{language=Java,numbers=left,numberstyle=\tiny,basicstyle=\ttfamily\small,tabsize=4,frame=none,escapeinside=``,extendedchars=false,keywordstyle=\color{blue!70},commentstyle=\color{red!55!green!55!blue!55!},rulesepcolor=\color{red!20!green!20!blue!20!}}
\author{deepwaterooo}
\date{\today}
\title{The Autobiography of deepwaterooo\textasciitilde{} \linebreak Part 2: CS辛酸史(1)}
\hypersetup{
  pdfkeywords={},
  pdfsubject={},
  pdfcreator={Emacs 24.3.1 (Org mode 8.2.7c)}}
\begin{document}

\maketitle
\tableofcontents


\chapter{2012归去}
\label{sec-1}

我小的时候,爸爸待我和姐姐更为慈爱,尤其记得冬日腊月的早上寒风呼啸,我和姐姐都懒懒地躲在被窝里舍不得起床,爸爸总是就着一块老式的手表每天早上早早起床,为我和姐姐煎油饼用纸包了好让我们走在上学的路上边走边吃。

人总是很奇怪,在那最温暖最幸福的日子里,我却因为心里的那一丝别扭,总是对爸爸冷冷的,有着一种若有若无的距离;可慢慢长大在经历了高考的洗礼之后,在远离家乡的大学生活里,在离开大陆来到海外的这些年孤独漂零的日子里,爸爸对我的教育,给过我的温暖和感动才真正转化为血脉亲情和精神支柱支撑着我去战胜生活中一次又一次的困难和考验。

爸爸出了事情后,虽然姐姐、姐夫,妈妈和我,亲友我们大家都有心理准备,但爸爸的离去还是给了我不小的打击。想到以后的日子里再也听不到爸爸慈祥亲切的声音,听不到他略带文艺的唠叨和叹喟,我的生活已然变成无根的浮萍,对爸爸的思念和悲痛就像狂风过境的野草般倒伏辽阔。

妈妈这次发病来势很急,当邻居对姐姐说妈妈好像有点儿迷糊,认不得人时,姐姐便随即播打了120,并要离家最近的姐姐迅速赶回家,120先把妈妈送到了县里,很快转又送到了市里的医院,检查出来说妈妈是有脑肿瘤的,手术有风险,医药费大概需要至少十万左右。爸爸的离去已经是让我很愧疚,没能为爸爸做什么,可次妈妈病成这样,我怎么忍心不给妈妈好好治病?电话里我把自己当前的状况以及眼下可能会需要回学校读书,可能回国的打算都对姐姐说清楚了,当时的电话里我们姐妹达成一致,为妈妈治病,爸爸的意外来得太突然,希望妈妈能够有机会安好好享晚年。但第二天再打电话回去问妈妈手术情况时,姐姐说亲友们还是认为不要太勉强,等妈妈这次病好,她自己能够调整精神状态会比手术来得更有效。我心里难受,在妈妈的劫难面前,我又一次地作了逃兵,而心里也很感激姐姐,不给妈妈治病的话我是无论如何也说不出口的,但她看到了我的困难,帮我做出她认为对大家比较好的决定。而之后姐姐还在家默默无言地照顾着妈妈。

春夏的那份工作之后,我心里还有着自己恋恋不忘的人,从大学学计算机VB编程开始,对这个专业,我始终都还有着敬畏和向往。尽管舅舅和表哥都需要我惧重考虑作自己的决定,但为了爱情,为了给自己一个机会,我还是申请了这个专业的硕士,准备可能回学校读书。

申请自己以前的学校,因为研究生院的申请材料早就有了一份,所以申请和录取的过程都非常顺利,只是在后来同导师谈论秋季入学选课的问题时被撞了一下腰,无论如何,我不可能每学期交上万美元的学费而只能选7个不计入毕业学分的课,我已经33岁了,如果这样一个学位一定要等那么久,那我还是应该接受现实,准备回国。好在后来老师还是允许我选课的,连很重要的cs121 c++编程课都可以让我跳过,便最终决定留下,拿着原本该为妈妈治病的钱留下来读书。

于是, 就像四岁那年坐在外公牵着缰绳的牛背上努力想像着远方的医院和献血会是什么样子的懵懂孩童,就像十八岁那年被舅舅科普地球的另一端有一个神奇国度的懵懂想往,现在我怀揣着对爱情的一丝眷恋和执着,又一次地开车走在了通往以前学校的5 号公路上,懵懂依旧,却心想着这一刻将是梦想起跳的地方。

\chapter{选课邮件}
\label{sec-2}

\section{6/26}
\label{sec-2-1}
I am a new cs-majored graduate student with a statistics master’s degree. I got your contact information from Karen today. It was my lucky to get introduced to you through email, and I do need you to help assist me in selecting classes  for coming fall semester.

From our department’s course information, since I have a MS statistics degree, I guess I don’t need to select math176, math170 and math175. And I have taken CS120 in fall 2009 and got a “B” as the grade. So I am thinking I should take the following courses for fall 2012:
\begin{center}
\begin{tabular}{lllrl}
\hline
CS121 Computer Science II & MWF & 10:30AM - 11:20AM & 4 & credits  3 ON WAIT LIST\\
CS150 Computer Organization and Architecture & MWF & 2:30PM - 3:20PM & 3 & credits\\
CS210 Computing Languages & MWF & 9:30AM - 10:20AM & 3 & credits\\
CS240 Computer Operating Systems & MWF & 2:30PM - 3:20PM & 3 & credits\\
CS270 System Software & MWF & 11:30AM - 12:20PM & 3 & credits\\
\hline
\end{tabular}
\end{center}

I am worried that I may not be able to select CS240 because the time is completely interrupted with CS150. I know I am pursuing a graduate program but taking undergraduate courses for this semester, but I am not sure how many course credits I can take at most. So far without cs240 I have 13 credits. I need your suggestion if I should take some more CS courses for this semester. I am quite interested with CS383 (Software Engineering) as well, dreaming if I can take this one in fall semester, I may be able to try my best to find some intern for summer 2013, but I also need your suggestion on this one.

By the way, I have took several credits for Visual Basic programming and one class for database theory. I am thinking if I can take one or more course for this semester, I may want to select some courses taught using VB programming or taught about database.

I don’t have any valid account for selecting classes until the beginning of the fall semester. So please take your time and offer me some valid suggestions. 

\section{6/27}
\label{sec-2-2}
Great to have you as a new student. 

I have not seen the paperwork yet from the graduate committee to see their recommendations.   I have read your letters of recommendation and your statement of interest.   I also see that you are not in town so we cannot meet face to face, so let's try email, although it will be more difficult that way.

So let's start with a few questions. 

What are your goals?   This is an important question for not just me but for you. Are you interested in getting a job in a particular area going on to do research? Do you have a passion for a particular topic in computer science? It is not necessary that you have that defined right now but if you do it could help us select the best courses for you.

Your choice of study may also be limited by funding that is available depending on how much you need a research job.

Programming is the important skill that you need to harness to do your work. Please, tell me about the programming experience you have had.

My guess is that you are making the right decision to "get back to basics" in your classes.   If you can afford the time and money then you can take the full breadth you suggest below.  But it is possible to focus your classwork to hone the skills you need for your area of interest.   

Some of the courses you have listed below have prerequisites for the reason that you need to know the material from one course for the follow on course.   For example, CS121 is a good background for CS210.   You can get in a lot of trouble in CS210 without the info from 121.   Again it depends on how much programming you have had.   Starting off with some successes may build a solid base and confidence so again 121 might be a good choice.   

I look forward to your responses to my questions and patience to try to do this by email.

OK.   So you aren't in town.
Then let's try to do this by email.
\begin{enumerate}
\item what programming experience do you have?
\item what area of computer science are you wanting to study?
\end{enumerate}


\section{Hi, Dr. XYZ,  6/28}
\label{sec-2-3}

No, I don’t have a clearly defined particular topic of interest right now. I have a MS Biology/Agriculture from China, a MS Statistics from U of I, and a MS Computer Science years later, and after two and half years’ professional work, I feel I have programming and computation interests, together with some business intuition. I would like to combine my background and personal characters together, but still, I like to challenge myself in job market in marketing. So I guess I will spend some time and do some research to figure this one out.

“Your choice of study may also be limited by funding that is available depending on how much you need a research job.” I feel confused about this one. If you can help explain a little bit, that will be great.

The programming experience I have had including a 3 credit undergraduate course “Computer program design” from Huazhong Agriculture University (Wuhan, China) teaching Visual Basic programming, a 4 credit undergraduate course CS120 from U of I using the C++ as teaching programming language, and statistical SAS programming. I have got four SAS certifications, base, advanced, predictive modeler and clinical trials. And I do think I have some interest in programming, coding and computation.

I will check with the graduate school later after I have got formally accepted to see how many credits I can take at most for one semester. I agree with you that for the first semester, I will focus on my course work to build a solid base and confidence for later development. I will take CS121, CS150, CS270, and put CS210 and some other courses into consideration. I will also need to ask some Chinese friends’ experiences and suggestions on CS121 and CS210 relationship and difficulty to make my final decisions.


\section{Hi me,    6/28}
\label{sec-2-4}

Have you thought about bioinformatics?   A strong stat and biology background combined with computer expertise. If you like stat and biology this might be ideal.   There is a strong market for these skills.  If you are coming in the Fall you might want to take the BCB intro computing courses. This will get you involved with that end of computing and sharpen your computing skills at a very basic level.

Coding is a strong component of a successful student.   This is coding more like C++ than SAS or Visual Basic.  If you only have "some interest" in programming are you more interested in theory or hardware than programming?   Most successful students love programming.

You can take 20 credits or more if you like but there are limits on how many you \textbf{should} take.   And there are limits on your credits if you are working as a TA or RA.

You won't be able to take 210 and 270 without having taken 121 and 150 first.    
If you really need to take 121 then you are not ready for 210.    
270 will require some solid coding skills. If you feel you need to take 121 then 270 may be a problem as well.

It is always good idea to ask your friends about their experiences to get a good understanding of what is ahead.


\section{Hi, Dr. XYZ,   7/1}
\label{sec-2-5}

I have taken one bioinformatics course back in China from graduate study in Chinese Academy of Agricultural Sciences, but since I have only experienced limited computer science skills, I am not quite sure bioinformatics will be my favorite/interest. But since this area may also be a window for me, I can try one BCB into computing course for this semester.

I guess it won’t work if I skip CS121 and select CS150, CS210 and CS270 for this semester. I agree with you that this time for the new computer science major, and I do want to build a solid base for later development. I have checked with my friends that CS210, CS240 and CS270 are also open in spring semester, so that will make me feel better if I have to select CS121 for this semester. I hope I can select slightly more computer science courses for this semester. I just feel that I pay more than \$8000 tuition fees and select only 7 credits computer science courses  will make me feel slightly pressed. I hope we can discuss and negotiate a better choice for this semester.  

By saying I have some interest in programming, coding and computation, I mean my most interests lie in programming, coding and computation. And I will try to have more coding (like C++) practice and experience.


\section{Hi me,   7/5}
\label{sec-2-6}

But you can't get a master's degree without computer skills. So you will have to have computer skills which then makes bioinformatics a possible choice.

I recommended CS121 because you got a B in CS120. I would expect a someone preparing to go into a CS graduate program to be able to get an A in CS120 without trying hard. So I find this a concern. I therefore recommended a solid base in programming.   

Second the 200 level courses assume proficiency in the 100 level courses. If you believe there were extenuating circumstances in getting the B in CS120 such as difficulty with the language and believe that you understand the material from CS121 then I won't stop you from skipping CS121. But with the information I have, it seems like CS121 is a good idea.


\section{Hi, Dr. XYZ,  7/15}
\label{sec-2-7}

I understand your concerns. The reason I got B for CS120 was because I was focused on my two SAS Certification tests in that semester (I selected 12 credits courses, and passed two SAS certifications in four months). I did not devote enough time and energy on that course, but I do have confidence on myself. I feel slightly hesitate and my main concern is on financial side.


\section{Hi me,   7/15}
\label{sec-2-8}

If you feel confident in your coding abilities and want to try to move directly to higher classes, you can do that since you are a master's student.  In the same way that a medical doctor must rely somewhat on what symptoms the patient reports, I as your advisor must rely to some degree on your evaluation of your skills.   The B in CS120 is definitely a warning flag to me.   But I also believe students to some degree make their own decisions.   Skipping 121 could be a big mistake or it could work out just fine.


\section{Hi, Dr. XYZ,   7/16}
\label{sec-2-9}

Thank you so much for the consideration and understanding. I know you are being responsible for me as the advisor.
I will work hard and try my best this semester to make it work out just fine.

\chapter{第一学期}
\label{sec-3}
\section{第一学期(一)}
\label{sec-3-1}
回到学校后,我就写邮件同导师约好了见面时间,想讨论一下我秋天究竟该如何选课。好玩儿的是,老师在自己帮我列选课选项之前,带我去见了系里的大牛。后来知道系里的大牛是安全领域里的专家,每年都有诸多的会议,天南海北地跑,经常都很忙。不过那天他正好在,在大牛的办公室里,根据我的需要(我希望能够两年拿到自己的硕士学位),列了一个两年计划的列表,但第一学期仍然只有CS121 和CS150两门课,后来一年的课程明显要重很多,我感觉这样选课第二年压力会非常大。我对大牛的选课建议是不满意的,于是导师帮用他的手机拍了大牛白板上选课的照片,回到导师的办公室里,我们再讨论出另一个选课计划。

选的结果是,虽然导师同意了我可以跳过CS121编程课,但我自己坚持要选那门课,因为我清楚地记得09处秋天时我linked list弄得还不是很透,要在programming上取得胜利,CS121带实验4个学分坚决不能跳,我同导师商量跳过了CS150,选了CS210 language programming, CS270 system software,CS385 Theory of Computation和为下学期计入学分准备的CS336 Introduction to information assurance共记16个学分。当时我的课还是在系里main office小秘帮选的,注册课程的时候小秘还意外。不过看看陪我一起来这里、站在我旁边的导师,还是默默地将我的课注册好了。

CS270是我的导师带课。开学的前两个星期试上课的时间里,对于作业,上课进度等等在我这样的一个新生的眼里这代课的老头已然有了近乎变态的严厉和疯狂。这还只是我转专业的第一个学期,我还有着自己新转专业的心理障碍,我还有自己对作业的恐惧,我不接受,不应该,也不需要在自己还没有完全准备好的情况下来接受过于严格的挑战来挫败自己。我选几门自己认为合适,自己能够弄得来的课程就好了。于是在可以改课的前两周deadline之前,我就同导师商量,将cs270改成了cs240。这是一个外国老师教的课,那个老师为人很好,没有现在的老师这么严格,学起来可以稍微放松一点儿。

是的,对于新专业的作业,我还是有着本能的恐惧, C++的实验因为09年秋天上过4个学分的CS120,所以还一如既往地顺利;CS210 第一次作业要求用lisp写9个recursion函数,字符串的三个我没写出来,我只写出了6个;CS240的作业我就被吓得早早地开始写,星期五的下午还在CSAC里写,但被系里的老师提醒tutor, 系里的学生问的问题如果简单不要直接给答案,让他们自己先想想。 

大概是自己选课选得太多了吧,开学不久就得到了老师的一记不知所以的打击。那是我相对进展顺利的C++课堂上。前几周还在上CS270的时候老师建议我们用Linux系统,于是我跑回去装系统,自己的台式机大电脑XP + Ubuntu 12.05顺利地装好了,可自己的笔记本win7 + Ubuntu 12.05 却被自己整成了grub failure,几个星期没能用成笔记本。于是有天晚上当CS121有一个实验自己的程序因为double 调用destructor出现个什么类似看起来很像的东西时,原本脆弱的小心脏就又被吓得魂飞魄散,立即写邮件向老师求救。我的邮件老师没有回,第二天一早我就去了CSAC,那里的tutor帮我找出了原因。上课前从老师的讲台经过时我高高兴兴地告诉老师我的问题解决了,却没想到因为一种效应,那堂课成了我的灾难。

其实当时老师说了什么我已经不大记得了,大致是说有些人笨,都没想到怎么选了现在的这专业吧。课堂上的同学们那堂课会意老师的暗示,很多学生发出奇怪的声响。而那时我的Tic-Tac-Toe一步move早就已经写了出来,而面对来自老师的突入其来的变故还是让我没时间反应。在刚转专业的脆弱心理下,在同学们奇怪的声响里,我摘下了扎着马尾巴的像皮筋,眼泪在头发的遮掩下无声的流淌。同学们的声响还要继续,而这堂课我早就已经上不下去。后来,后来,再后来,我身边的同学终于开始提问老师,眼看就要成为同学们注意的焦点,我趴下了,不愿意让别人看见此时自己的面孔。记忆里那天我原本因为解决了问题很开心,可是那堂C++课特别的长,完全不知道老师究竟说了什么,而如我般脆弱的灵魂在这个系里也从此充满了自我保护和正当防卫。

\section{第一学期(二)}
\label{sec-3-2}
给我们代CS210的老师也是某门语言领域里的一个大牛吧,就是常常觉得上课时因为老师的话是在变着法儿地重复,就不知道老师真正想说的是什么。上Lisp刚开始讲recursion时,老师讲如何写一个myLength(list L)函数计算list的长短,讲了一步后,老师问接下来该怎么办,老师像是想听同学来回答问题的,我跃跃欲试,当老师示意,我便大声响亮地回答了这个问题,"plus one myLength cdr L"带着自信。就这么点儿小基础,同学们居然还稍感意外。老师没表扬我什么,倒是鼓励近乎要求地让课堂上另一个同学上黑板去写一个什么答案,并表扬那个同学说"That's very brave of you"。之前C++课上受到过的无名的打击、同学们对我选课的反应,以及这堂课上老师的行为终于让我明白老师是不鼓励我课堂回答问题的,或许我应该低调一点儿。我是这里少有的大龄学生,国际学生里应该就只有我一个了,最可怕的是,这是一个对自己有着要求和期望的学生。如果他们允许让我两年毕业,如果我能勉勉强强用自己的能力在经济上生存下来,如果我每个学期选这么些课堂一定让系里如此为难的话,那我低调一点儿又有什么关系?接下来这门课上课前我就去找了这个代课老师,告诉他我以后会低调一点儿,只要系里允许我两年毕业我还是很感激的。毕竟年龄大了,一直耗下去实在是耗不起的,太痛苦。

再后来,CS121 C++课有一次不知道为什么,代课老师讲起故事来,说是一般在社会上工作过的学生再回到学校里,他们往往比那些一直呆在学校里读书的学生学的效果好,因为经历过了在社会上的那几年的工作,他们能够更清楚地认识道他们想要的是什么。"Ooh, by the way, his mom passed away when he was working. " 听到老师的这句话,像是一种生物本能,我下意识地转过头去,背离人群,自觉面问部表情僵硬扭曲,眼眶里噙满了泪水。老师讲那个故事的时候差不正式授课内容已经结束了,好在接下来应该就是下课了吧。接下来一个小时的CS385课我迟到了十多分钟。

那学期最大的收获是开学不到一个月的时候CS210用lisp 写了 Tic-Tac-Toe 的一步move,班上只有一两个同学写出来,为当时新转专业、对各种作业心怀恐惧的我多多少少建立了一些专业上的自信心。这个之前已经写过不再冗述。

那学期最大的意外来自于网上,不知道为什么就开始传说我什么什么乱七八糟的,那时的自己心理上还是受到了不小的影响,后来我的C++两三次实验课都稍有延迟。再后来为了减轻对自己的伤害,还是爬上网去稍微澄清了一下。

后来期末出成绩时,听同学们说C++特别严,好多学生得了C。我登录进自己的账户,也得了C,这个C对于我来是完全不能接受的,因为之前老师公布的成绩里我没能得到A,但是是稳拿B的,而我的期末考得也很好,最后一个Binary Searching Tree的程序很难,平时的实验里老师没有要求写过,但我想出了思路都写出来了,不可能得C。于是我给老师发邮件,想查成绩。老师回信说,网上的成绩不知道是怎么更新的,我的成绩应该是B。

CS121的问题解决了,但CS336的问题就相对棘手很多。当初为什么从Plant Science Ph.D作了逃兵,不就是因为不喜欢读paper,理解不了概念么?现在同样的问题出现了这门课上,除了考上课的内容,班上十几个同学的项目,每个项目也有3个题目。而一起上课的那个美国女孩说实际上考前复习时因为他们在Think Tank复习,他们是有八九个学生到那里去,把他们自己项目的题目都说过一遍的,但那个女孩子没有告诉我任何他们的题目。也是同那个女孩聊天我才知道系里的代课老师里只有代我CS121的老师是硕士,其实所有老师都有博士学位。显然,这门CS336课如果要argue学分的话,我就没有了CS121的那份理直气壮。如果这个老师一定要那么严格,那我得C也没什么好说的了。想想这个领域对这个呆在异国他乡的自己来说,即使钻得再深,也未必有很大的用武之地,自己毕竟是外国人,便也没什么太多的遗憾和可惜,只想着以后好好学习,把GPA补上来就可以了。

那时的自己在学校食堂里打工挣生活费用。认识了一个同在那里打工的大陆女孩子,也是我们学校里的学生,说话做事待人处世都比我成熟很多,便同她比较要好。圣诞节去她家玩儿的时候,还有另一对年轻的中国夫妇。那时她刚从教会组织的加州旅游回去,给我们看她在那边玩时拍的照片。我在加州生活过两年半的时间,仔细看过<世界日报>的人应该都会知道某一种XXXX的可能性。我对她稍有怀疑。她学的专业并不是很好找工作,而这两三年的学习也是全自费,或许她想要的只是生活的一种出路吧。只是我清楚地知道自己在网上太出名了,这样可能对大家都不好,那天聚会看过那些照片后我们就下意识地减少了联系。我从心底承认她仍是待人不错的朋友,只是我们道不同,不相为谋。

她对我的影响和改变来自于那天聚会的一句玩笑。 她对那对年轻夫妇介绍我时略带调坎地说我拥有十个硕士学位。我不为她的调坎甚至嘲讽而伤心,但这"十个"(实刚三个) 硕士学位的事实震撼了我自己。是什么让我们在这里漂泊,我究竟为什么要流浪,一个人要走多少路才能够真正成熟?而我这第三个硕士学位的出路又究竟在哪里?我看不到黎明晨曦的方向。圣诞节的午夜,因着朋友那一句话的震撼,就着1个C、4个B的无奈,一回到家,终于是禁不住自己眼泪往外滚,坐在床头趴在膝盖上,任凭泪水无声流淌,哭了半个小时,告诉自己,tomorrow is another day,一定会有办法的!

\chapter{第二学期}
\label{sec-4}
\section{第二学期(1)}
\label{sec-4-1}
路漫漫其修远兮,吾将上下而求索。因为看不到方向,那就只能努力,唯有努力才能改写自己的命运。只是,交完了这学期的学费,我就要开始为下学期的学费发愁了。

这学期我选了CS570 Artificial Intelligence,CS552 Real Time Operating System,CS504 Software Security,CS502 Algorithms,CS502 software engineering和CS501 Seminar 共计14个学分的课程。大牛说他愿意帮助我们,给我和一个外系的中国女同学开小课,选了他的2个学分的算法课和2个学分的软件工程课。但大牛很忙,他是没有时间系统地帮我们上课的。我们两个人好几门课(她还有200 level的课程),一周见一次一个小时计论一下我们看书有什么不会的地方就下课。那真正能够学到的知道又有多少呢?鉴于算法课那么重要,而这个学期我的导师正好也开算法课,我便去旁听了导师的课。

AI课的课程要求完成五个项目的作业,外加一个研究生的文献综述报告。第一个项目path searching的几种算法项目交上去后的课堂上,老师上课环顾左右而不看坐在偏中位置的我,让同学们和我都感觉到我受到了歧视,但可以想象,老师这么做的原因是他觉得我写不出来code,他认为我的作业至少code的部分是抄的,我猜他那么做是想要警告我或者要小伙伴们孤立我吧。原本项目的报告只写报告就行,是不需要源程序了,但因为老师课堂上已经这样了,后来的Min-Max Connect 4 game human being vs. Computer AI,Neural Network,Fuzzy logic control system等等,我就开始把源程序代码也交了,这样的怀疑便不再有。

这些年来我一直开一辆破车,这车是我来美第二年的时候找到舅舅,他建议我买的。买的时候因为有过车祸纪录,油箱也被撞破了,有修补过的痕迹。这么多年来小毛病一直不断,因为漏油,在加州的时候就被过两三次油箱(每次花一百左右),而回到小镇,因为小镇上的人不帮忙补修,只能换,哪怕是换旧的也得换,于是换掉了我三百大刀。问题是换掉后,原本破旧的车不知道怎么回事同样的力度踩油门,车子像是发疯了似的跑得时快时慢,很是吓人,然后还时不时地死机死跷跷,这已然超出了我的承受范围,最终在换油箱后几个月后将车捐给了Heritage for the Blind社会慈善组织。

\section{第二学期(2)}
\label{sec-4-2}

因为我上大牛的课,大牛是系里最有钱的老师,我理所当然应该先去求他。大概是因为上学期的选课我没有按照他的建议来选吧,大牛对于我的困难是视而不见的,显然不愿意提供任何的帮忙。表哥只是在异国他乡作过几年科研的小研究员,回到现在的学校尚且欣赏提携woman in science,难道我就真的那么差么,根本进入不了大牛的法眼,丝毫不能引起他的注意?于是小弱弱的幼稚心思就有了幼稚的解决办法。那天去上大牛的课,我便从头到尾上课都不拿眼睛看大牛,我上课是坐前排的,一堂课下来,大牛无论如何也该看得见我了吧?

可是再后来的事情就是让自己诧异的了,接下来旁听自己导师的算法课,却能明显感觉到他反常地高兴,高兴得过分,一如前一堂大牛课堂上来自于我的意外。人在困难的时候难免会做出一起过分的举动,但那也并不是说,作为当事人的我就会以此为傲,她何尝不是被逼无奈呢?因为别人如此行为便特兴奋,这算是什么?作为导师,系里老师之间的矛盾是真的么?作为导师,你又何至于开怀至此?当然老师自己的解释是那天是一个什么walawala day 还是什么hula-hula day。是他们的什么day我不管,那堂课我的脸是木的,动弹不得。

大牛显然没有帮我的意愿,于是我去找实习,找自己以前工作过的公司,当自已以前实习过的一个小公司也没有示好帮助的意愿之后,迎来了自己导师的一点儿启发。那堂算法课的意思是说,因为我的programming比较好,对hash table、各种树结构理解得都还比较透,这会在生物方面有很多很大的应用 。春季的学期已经快要结束,虽然现在我还没能找到学费来继续秋天的学业,但是我还有夏天可以去捕食。那堂课意识到导师的旨意,下堂课便装傻,表示还没有好到可以在那个领域有所用武之地的份上,所以就让我还是依靠自己去找食物吧。

\section{第二学期(3)}
\label{sec-4-3}
那时的自己怎么说来,经历了十个硕士的震撼和第一学期成绩的无奈,有着卑微到尘埃里去的朴实,默默地付出着自己的努力。因为选的课多,还要在学校食堂每周打十五个小时的工,所以写作业必须要快。

二月份的时候实时操作系统的老师布置作业用ATMega32A + 16键的数字键盘制作square wave 音乐1 \textasciitilde{} 2 \textasciitilde{} 3 \textasciitilde{} 4 \textasciitilde{} 5 \textasciitilde{} 6 \textasciitilde{} 7 \textasciitilde{} 1 (记错了,music是下一次的作业) ! 并设一位四位数字的密码,用LED灯indicate密码的正误。因为数字键盘有限,我们所有的学生只能在实验室里写作业共享有限的几套配置。那天下午的时候同学们跃跃欲试,他们年青也很聪明,我就让了,自己在旁边查找一些理论基础,等傍晚一个同学走后,我也不知道怎么回事,稀里糊涂地就就着一个网页把键盘给设好了,连原始的代码都还没有完全看懂,引来同在实验室做作业的其它同学一片诧异,觉得我太有效率了,居然在那么短的时间里就给弄出来了。那次大概是最后一次同学们发自内心的自由表达对我的欣赏吧。后来环境的变化,这样的状况就极少了。当有同学问我怎么做时,我就把自己的思路讲了一下,把自己参考的网页发给他们,很快他们也做出来了。那时的自己,心里应该还是有着某个想念的人吧。我清楚地记得写出作业的那天晚上是阳历2月14日,因为那次作业我设置的四位密码是“2514”。

看到同学们的反应,大牛在他的课堂上讲,有些学生,网上搜索能力还不错,现在网上的资源已经很多了,但要真正学到知识,还是要清楚地知道每一行代码的原理是什么。因为我的部分代码来自于网上(根据我们自己的键盘连线设置作了必要的修改),对于大牛这样的评价,我是很不以为然的,他就是妒嫉、看不惯我写得太快进展太快,到大牛这么说我心里真的开始、很是有点儿洋洋得意的。

\section{第二学期(4)}
\label{sec-4-4}

这个时候我注意到表哥的linkedin加了一个人,我便把自己的也加了一个人,就当它是一种鼓舞吧。

额,上次写得简单,其实在二三月份被表哥自己打911后policeman有亲手拿给我一份表哥亲手写的什么隔离一类的申请告状,上面有type出来的表哥细数我的 11 条罪状和他的签名复印件。但这次明知道自己应该会继续写,但最终那份复印件状子还是被自己遗忘在了storage unit里(如果以后有必要再回来补吧)。可以猜想,那些条罪状表达的应该是爱意吧。

后来三月份court结果出来后,我收到了另一笔的罚单,来自AAI Collection公司的Ambulance费用和Collection费用,外加接近一百刀的延迟罚款。Ambulance费用是\$832.20,Collection费用是\$416.10,延迟罚款有九十多刀。而court 的结果是我一年内不得再上court。电话打过去,查到延迟罚款的原因是他们把罚单寄到了我来美后的第一个address,这让我很是哭笑不得。最近这几年任何一次的超速罚单、任何一次的911纪录都有我的更新后的地址,他们何至于把这样的邮件寄到我来美的第一个地址?先前说过,2012年5月左右回去的时候policeman开的\$50的罚单我wave掉了,后来想想数量不大,我还是应该交了。而这次的这些费用数字看起来这么熟悉,原来\$416.10也是Ambulance把我送到医院后进行简要检查的医疗费用的数字,2012年秋天我收到后因为全自费读硕士,我申请把它wave掉了,而Ambulance费用正好是它的两倍多。这Collection费来得不明不白,我却没法再上庭! 算了吧,让它吧,这就样吧,这同样是一个小稹,小镇的建议同样需要各种资源。虽然那两年多的时间里我没有买医疗保险,我不能报任何的费用,但是该自己出的就自己出,没什么好报怨的。4月份收到报税return后我邮寄了一张\$832.20支票过去,秋季开学后邮寄了另一张\$416.10的,只把那原本不该归因于我的延迟罚款wave掉了。而别人帮我wave掉延迟罚款我还得感激别人,别人帮我wave掉医疗费用我也得感激别人。有时想想这个世界真搞笑。这真是一场昂贵的恋爱,来往油费、超速罚单、court程序费、医疗费用、Collection费用、Ambulance费用,还有更重要的,青春年华,回想起来都心惊胆颤、后怕不已。

\section{第二学期(5)}
\label{sec-4-5}

后来的AI最后一个项目老师给出了三个选择:可以写Prolog knowledge-based rules,或者是local search,这个我们前面的项目里还没有涉及到,或者关于一个100个observation的 mobile robotics根据环境变量作出决定的decision tree。因为自己有统计背景,我想写最后一个。只是根据老师的要求还需要将数据分成Training data和 test data,有几个不同划分结果的sets要做。我的同学们绝大部分都选择了相对简单的Prolog项目,但我因为想要自己试着combine statistics背景,便一心想写decision tree。那是第一次使用vector容器来存数据,参考网上能搜到的例子来写,后来Debug debug稀里糊途就写出来了,不敢相信是这是真的,跑去Excel用pivot table生成数据和图形来对比,原来我真的写对了!实在是不敢相信我居然真的写出来了,也不甘心写到这里就算完了,反正就着老师给的这份数据的这些个自变量,100个observation是远远不够的,于是不同的data segmentation就可能会造成完全不同的答案,test结果会非常 sensitive to 数据的划分,所以偶就借着写出来的一丝兴奋,狂喜中的我硬生生把这份数据穷凶极恶地给划分了almost 99次,凡造成不同答案的拐点都划分测试一次汇成一个完整的图,都被我写进了最终的报告里。最后一个项目是考试的时候交,我交的时候告诉老师我把decision tree写好写完了,但我的研究生报告还没写完,等我写完再拿给他。最后一周考度的课堂上,小伙伴们听说我写出了decision tree,回想第一个项目交上去后老师对我的不信任,他们开始大声说话兴奋起来,教室里很快炸开了锅。如果另外一个有数学背景的TA没有写decision tree的话,那我应该就是班上唯一一个写这个项目的人了,而我之前的中国学生里,甚至上一两界的课程里也未必有这个项目,或者未必有人写这个项目,说我抄我都无从抄起啊。这个项目并不是很难,但是能够把自己跃跃欲试、心里想往的事情做好,这种感觉真的很好。而这也正是我这样一个转专业的二愣子成长过程必经的一步、这个项目也成为了自己成长的一个里程碑。

\section{第二学期(6)}
\label{sec-4-6}

幼稚的心思不止如我般小弱弱独有,大牛偶尔也会犯小弱弱的低级错误。大概是我的court结果出来后,还是期中或是期末考试(更可能是期末考试)前一次课他在last minute时为我们布置作业引来同学们哗然一片,当然这样一个乌龙最终以他在课堂上向我们学生道歉结束。据说大牛一般都很爱惜自己的羽毛,他的CS504我得的是B。大牛帮上的两门课算法pass,但软件工程的项目我们没能做完,incomplete到了下学期。seminar一个学分也是pass。

但CS552 RTOS的课最终在自己建的小Linux系统里遇到了困难。不知道为什么我的Context switch就是转换不过去,用ATMega32A很快我的程序就会crash掉。当我AI Connect 4 Min-Max计算机player不够聪明的时候我的老师还到我student office里帮我找出问题,提出建设性的解决方案,而带这门课的老师理论知识很强,却在code方面相对弱很多,他没法帮我找出我的问题,而我的程序也陷入了绝境。我对老师说我想再调几天,万一调不出来就交上去,老师说我应该可以拿B,但老师希望我incomplete再自己试试看,我便顺了老师的意思,但最终暑假的实习,秋天繁重的学业,我也没能再花多的时间在这个上面,最终也还是拿了B,只是再后来自己最终脑袋开窃后才将这个程序给调出来。

AI的课对编程的要求比较高,所有的项目都要求编程,我都还进展得很不错。但最终的成绩里,老师并没有给我A,还是一个B。老师把我的期中考试成绩压得很低,可能期末他也压得很低吧。我没有去argue我的成绩,但是为自己争credit的欲望却在增强。 

\section{第二学期(7)}
\label{sec-4-7}

虽然我上大牛的算法课,但因为完全是自学,我学的效果并不是很好,导师的算法课我去旁听了,并且参加了他们的期中考试。上课的内容一般都不是很难,拿笔记本 Linux + Emacs + Python + typing + 转转脑子,跟上导师的课就基本没什么问题了。我还参加过他们的期中考试,老师给列了详细的提纲,考前的课堂上还事无巨细地讲,连我这个旁听生都听很明白了,他还在讲,当然也有很 sweet 的学习好的学生也在假装哪个步骤不明白举手提问请教老师,坐在我前一排的小伙伴会意,接着问了第二个很 sweet 的贴心问题。这个前排小伙伴后来成为了我下学期(秋季学期)编译课的同桌,这是后话,以后再表。

第一学期上CS385的课还没什么特别的感觉,这学期上了AI的课,开始明显地感觉到这位老师应该是我们系最好的代课老师了,他的授课有着分析的逻辑在里面,可以真正讲到透彻,讲到让学生能够真正听懂、明白原理,对于这样一位和自己一样有着本能的problem solving skills的老师,如果自己的项目出现什么问题,他应该能够给出及时的、建议性的建议和解决方案。最后一堂课的期末课程总结课上我已经向老师明示问题示好,表达了我跟他所做科研的兴趣,期望以后能有机会跟他一起作课题。

但与此同时,因为前面好几次的作业都写得还很不错,RTOS的老师虽然code较弱,但他是爱好音乐的人,他对MIDI有多年的兴趣,一直在试图将不同的MIDI产品应用到学校的各种Band表演上去。暑假前他帮买了一款最新的 25键 QuNexus keyboard让我暑假拿去玩儿,可惜我对这方面好像还不是很入门。在两个导师两个方向的选择中,接下来的那个暑假我也该好好考虑,我到底该选哪个好呢?

\section{第二学期(8)}
\label{sec-4-8}

想到夏天选课、实习学分的可能性,我去找了导师。可笑的事情在这里又发生了。导师说他没有想到我是要在这里拿一个硕士学位的,他说如果我想找,即使不能找industry 的工作,还是可以找学校里的能够sponsor工作签证的职位。而先前不远的时候大牛刚对我说过如果我一切课程进展顺利,快的话今年 12 月份就可以毕业。这是我到目前为目学得最努力也最喜欢的专业,眼睁睁地看着自己就要拿到手的学位,我怎么可能这个时候放弃?便暂金截铁地告诉导师:我要拿自己的硕士学位! 

导师接着说,那我暑假可以找小镇上的工作,比如什么什么统计小公司,比如旁边的小镇的一家工厂。我非常疑惑导师为什么要我留在小镇上,而我过去的生活经历,过去的实习、过去的暑假不是从来都在加州度过么?难道加州不是我在这个国度的故乡么?心里疑惑,但我还是答应导师我会去试试的。但是曾经的生活经验告诉自己,留在这个小镇是没有出路的,我应该去到大城市,走进更为广阔的世界,到那里去寻找生路。随着期末考试的结束,我便开始找车,最终找到一辆自己能够承受、性能也还勉强过得去的车,找小镇上的好心人帮换了轮胎,便开车走在了返回加州的 5 号大道上。

\section{第二学期(9): RTOS作业-password-2514}
\label{sec-4-9}

The purpose of this assignment is to give you more experience using the AVR ports, and to add some I/O devices that might become useful later for our RTOS.

You are to use one of the provided keypads to implement a “digital lock.” Your program should allow the user to enter a four digit code from the keypad, and if the code that is input matches the one included in your program, the “lock” should open. In this case, the lock opening will be represented by the lighting of an LED.

As the user enters numbers, consecutive LEDs should light, showing the user that a number has been entered and received, and to show how many digits of the code have been entered. In other words, after entering thefirst number, one LED should be lit, after two numbers, two LEDs should be lit, etc. This behavior should occur even if incorrect numbers are entered, so the user can tell how many numbers have been entered, but can’t tell if the correct numbers have been entered until the end.

The pinout of the keypad is not intuitive - it is shown below. As before, in addition to writing the code, determine the size of the code in your program.

My answer could also be found at \url{https://github.com/deepwaterooo/csMajorCourses-Projects/blob/master/cs552RTOS_hw3_2514/hw3final.c}

\lstset{language=java,label= ,caption= ,numbers=none}
\begin{lstlisting}
// CS552 Assignment 3
// me~me~me~~!!!   Feb 14, 2013

// Size of code: 000616
// quote: the original codes are adapted from blog.jeffmurry.com
// blog.jeffmurry.com/2008/11/01/how-to-interface-with-4x4-keypad.aspx
// But I have modified them to cater to our keypad design

#define F_CPU 1000000UL 
#include <avr/io.h>
#include <util/delay.h>   // used for _delay_ms()
//#include <math.h>

#define BUILD_VERSION 4   // Update version in comments + display on LCD startup for a few seconds if using

// Keypad constants
#define keyport PORTB           //Keypad Port
#define keyportddr DDRB         //Data Direction Register
#define keyportpin PINB         //Keypad Port Pins
#define col1 PB2                //Column1 PortB.2
#define col2 PB4                //Column2 PortB.4
#define col3 PB5                //Column3 PortB.5
#define col4 PB6                //Column4 PortB.6

//#define LEDON PORTD &= ~(1<<5)
#define BLINK_SPEED 250;
unsigned int KeyLock[4] = {0xde, 0xac, 0xe8, 0xb0}; // ****~~:)

uint8_t i;
void delay(unsigned int dly) {
    for(i = dly; i != 0; i--);
}

void key_init(void) {
    keyportddr = 0x8b;   // 0111 0100
    keyport = 0x74;      // 1000 1011
}

unsigned int get_key(void){
    unsigned int key= 0;
    // Make rows low one by one (check for press, wait for release, return key)
    // First Row
    PORTB = 0b01111111;
    if (!bit_is_set(PINB, 2)) {while(!bit_is_set(PINB, 2)); key = 1;}
    if (!bit_is_set(PINB, 4)) {while(!bit_is_set(PINB, 4)); key = 2;}
    if (!bit_is_set(PINB, 5)) {while(!bit_is_set(PINB, 5)); key = 3;}
    if (!bit_is_set(PINB, 6)) {while(!bit_is_set(PINB, 6)); key = 10;} //A
    // Second Row
    PORTB = 0b11111110;
    if (!bit_is_set(PINB, 2)) {while(!bit_is_set(PINB, 2)); key = 4;}
    if (!bit_is_set(PINB, 4)) {while(!bit_is_set(PINB, 4)); key = 5;}
    if (!bit_is_set(PINB, 5)) {while(!bit_is_set(PINB, 5)); key = 6;}
    if (!bit_is_set(PINB, 6)) {while(!bit_is_set(PINB, 6)); key = 11;} //B
    // Third Row
    PORTB = 0b11111101;
    if (!bit_is_set(PINB, 2)) {while(!bit_is_set(PINB, 2)); key = 7;}
    if (!bit_is_set(PINB, 4)) {while(!bit_is_set(PINB, 4)); key = 8;}
    if (!bit_is_set(PINB, 5)) {while(!bit_is_set(PINB, 5)); key = 9;}
    if (!bit_is_set(PINB, 6)) {while(!bit_is_set(PINB, 6)); key = 12;} //C
    // Forth Row
    PORTB = 0b11110111;
    if (!bit_is_set(PINB, 2)) {while(!bit_is_set(PINB, 2)); key = 14;} // Column 1  *
    if (!bit_is_set(PINB, 4)) {while(!bit_is_set(PINB, 4)); key = 16;} // Column 2  0
    if (!bit_is_set(PINB, 5)) {while(!bit_is_set(PINB, 5)); key = 15;} // Column 3  #
    if (!bit_is_set(PINB, 6)) {while(!bit_is_set(PINB, 6)); key = 13;} // Column 4  D
    // Reset key ports
    key_init();
    return key;
}

int compareKey( unsigned int value[]) {
    int y = 0; 
    for(; y<4; y++) {
        if (KeyLock[y] != value[y])
            return 0;
    }
    return 1;
}

int main (void) {
    DDRD = 0XFF;
    PORTD = 0X00;
    unsigned int keyval;
    key_init();
    keyval = 0;       // Start with key value = 0 (ie.. not pressed)

    unsigned int cnt; // indication of number of keys typed
    unsigned int count[4] = {0xfe,0xfc,0xf8,0xf0};
    unsigned int Type[4] = {0};
    int checkflag = 0;
    uint8_t x = 0;

    while(1) {
        keyval = get_key();
        if (keyval != 0) {
            PORTD = ~(keyval<<4|(~count[x]));
            Type[x] = ~( keyval<<4 | (~count[x]) );
            x++;
            delay(65000U);

            if( x == 4) {
                x = 0;
                delay(65000U);
                if ( compareKey(Type) == 1) {
                    delay(65000U);
                    PORTD = 0Xfc; // password right: two LED will be on for correct keys
                } else                
                    PORTD = 0xf7; // password wrong: one LED
            }
        }
    }
    return 0;    
}
\end{lstlisting}

\section{第二学期(10): 报告:Decision Tree}
\label{sec-4-10}

My answer for this project could be found at

\url{https://github.com/deepwaterooo/csMajorCourses-Projects/tree/master/cs570AI_Project4_DecisionTrees}

// CS570                                05/08/2013

// me\textasciitilde{}me\textasciitilde{}me\textasciitilde{}~!!!     Project 4 Decision Trees 

// Decision Prediction for Robot Decisions

\subsection{Abstract}
\label{sec-4-10-1}

In this project, a decision tree algorithm is developed using the ID3 information gain algorithm, and it was implemented using c++ language. The algorithm selects decision split attributes based on information gain, and selects the variable with highest information gain from the available variable attribute open list. The algorithm will repeat this step until either run out of attributes, or run out of training data set. And later on, based on the tree rules we have built already using training data, we will evaluate the performance of the decision tree model by test the model on the testing data set. The model performance will be evaluated by misclassification/prediction rate. And based on the available 100 observations, I have built a pretty good decision model with mis-prediction rate is almost 0. 

\subsection{Data Structure}
\label{sec-4-10-2}

Data structure is very important for this project. I need to figure the data structure out before I can develop any codes for this project. The most fundamental tree node is easy, just includes the attribute, arrived\textunderscore value, isLeaf label and node pointer vector stores the node pointers from parent to children (for this project since every node has only two children, it should be much easier to simply use left and right pointers, but anyway I learned how to use vectors). The input data was read into a 2D array, and there are several vectors and vector of vectors to store the attribute strings for all available and remaining exploratory variables, the attribute value strings for each attribute, and total training and remaining training observations. The whole C++ code for this project is attached in the back for convenience of checking. 

\subsection{Learning Algorithm}
\label{sec-4-10-3}

For this project implementation, B(q) is defined as the entropy of a Boolean random variable that is true with probability q:

B(q) = -( q*log2(q) + (1-q)*log2(q) );

A randomly chosen example for the training set has the kth value of rth attribute with probability (pk+nk)/(p+n). So the expected entropy remaining after testing attribute A is : 

Remainder(A) = sum(k=1 to d)((pk+nk)/(p+n))B(pk/(pk+nk));

And information gain from the attribute test on A is the expected reduction in entropy: 

Gain(A) = B( p/(p+n) ) - Remainder(A);

Pseudo code for the algorithm: 
\lstset{language=java,label= ,caption= ,numbers=none}
\begin{lstlisting}
function Decision-Tree-Learning (examples, attributes) returns a tree
	 if examples is empty then return default
	 else if all the examples have the same classification then return the classification
	 else if attribute is empty then return major decision
	 else
		for each remaining attribute
		    calculate information gain
		create a new node using the best attribute
		separate the examples based on the best attribute
		subtree nodes = Decision-Tree-Learning (examples, remaining attributes)
	return tree
\end{lstlisting}

\subsection{Result}
\label{sec-4-10-4}

According to the project instructions, I have applied the 2, 5, 10, 20 and 50 observations as the training dataset to build decision trees respectively, and the produced tree were tested using the remaining observations from original 100 observation data set. All the models are pretty good and really look alike. And the models are performing better and better as the training data set increasing the observations, which means the decision tree models are well-trained and performs great job for predictions. 
train\textunderscore obs\textunderscore cnt mis\textunderscore predict\textunderscore test
\lstset{language=java,label= ,caption= ,numbers=none}
\begin{lstlisting}
2 60
5 60
9 20
10 26
13 48
15 19
20 19
25 21
30 10
35 10
40 10
43 10
44 3
50 3
\end{lstlisting}

The final tree model that predicts best is listed as below: 
the decision tree is: 

\lstset{language=java,label= ,caption= ,numbers=none}
\begin{lstlisting}
SafSit
	1
		SafCriDec
		1
			FamSit
			1
				AskdBef
				0
					No	 cnt:3
				1
					Confident
					1
						Yes	 cnt:3
					0
						No	 cnt:2
			0
				Yes	 cnt:4
		0
			AskdBef
			0
				No	 cnt:9
			1
				FamSit
				1
					Confident
					1
						Yes	 cnt:1
					0
						No	 cnt:1
				0
					No	 cnt:3
	0
		FamSit
		1
			Confident
			1
				No	 cnt:1
			0
				Yes	 cnt:3
		0
			Yes	 cnt:14
tree_size: 21
the total observation count is: 44.
Total number of misclasification for TRAINING data is: 0.
Total number of misclasification for TESTING data is: 3.
\end{lstlisting}

\subsection{Discussion}
\label{sec-4-10-5}
As we can see from the graph above, between trains observation of 9 and 10, there was a dramatic reduce of misclassification rate for test data, and as I have detected, the observation number 10 created a bunch of new tree nodes and completely changed the tree structure, and which actually is not as good as predicting using the 9 observation tree before. These specific two trees are pasted below for reviewing convenience. Since these models are not mature ones yet, the trees are just presented as the indication of tree structure changes. 
the training = 9, testing = 91, training data built decision tree is print out as followed: 
the decision tree is: 

\lstset{language=java,label= ,caption= ,numbers=none}
\begin{lstlisting}
SafSit
	1
		No	 cnt:7
	0
		Yes	 cnt:2
tree_size: 3
the total observation count is: 9.
Total number of misclasification for TRAINING data is: 0.
Total number of misclasification for TESTING data is: 20.
\end{lstlisting}

the training = 10, testing = 90, training data built decision tree is print out as followed: 
the decision tree is: 
\lstset{language=java,label= ,caption= ,numbers=none}
\begin{lstlisting}
SafSit
	1
		LongDel
		0
			AskdBef
			0
				No	 cnt:2
			1
				Yes	 cnt:1
		1
			No	 cnt:5
	0
		Yes	 cnt:2

tree_size: 7
the total observation count is: 10.
Total number of misclasification for TRAINING data is: 0.
Total number of misclasification for TESTING data is: 26.
\end{lstlisting}

Our data set has 8 variables and has only 100 observations. Theoretically we would expect at least 28= 256 observations to be an even representative training data set. Comparatively, our training dataset doesn’t have enough observation, and may produce bias. 
Because I used the information gain algorithm to build the tree, the built tree will largely screwed by the number of observations in the node class and the spilt attribute will also be dependent on the number of observations fall into arrived\textunderscore value node. So decision tree is very sensitive to unrepresentative data. 

Besides, since the tree depends on split attributes as well as attribute-values to build node, it is also very sensitive to missing values, and cannot perform predictions on those observations unless we take good care of the data cleaning and missing value modification. 
For this project, I separated training and test data simply by using the first N numbers of observation as training data, and all the rest obs. as the test data, and misclassification rate as the fitness function, which is a just-so-so method to get project done. To extract representative training data set, a better approach would be m-fold training data, and using cross-validation will sure perform much better. 

\section{第二学期( 11 )}
\label{sec-4-11}
因为看不清前路的方向,这是唯一一个我不曾哭过的学期,除了因为担心秋天的学费还是流过不少辛酸的鼻涕。

早在回学校的第一个学期,大概十月还是十一月吧,生理期间的一天早上,不到六点,远不到我平时正常的起床时间,我被右侧腹肚部位的疼痛痛醒了。我以为自己要上大号,去了洗手间但不是;胸口着呕很想吐,却又吐不出来;躺下,依然痛得厉害;于是我躺一会儿,去一次洗手间,躺下还伴随着连微的呻吟。那个早上出了一身汗,折腾了一两个小时后,排出来几个大血块,才总算消停了,又倒头睡了一两个小时。终于是忍住没有去看医生。

后来十二月身体又起了怪病,在加州两年多的时间里都不曾买医疗保险的我,这次却得了个怪病,感觉身体敏感部痒起来。早先我已为是因为自己是了公用的洗衣机和烘干机,自己的身体比较敏感,所以不小心染上了什么乱七八糟的怪东西。为了保护自己起见,便买了个小型的甩水机,从那以后自己的内衣内裤等贴身衣服全都自己在家里洗。后来随着病情加重,已经没办法忽视,只能去看医生。期末考完的周一( 12/16/2012 )我去看了医生。医生听完我的描述说80\%的成年女性都会有这个问题,他希望我用一种外用皮肤膏一类的东西先试试,如果不行再回来作身体检查。我知道自己从出现症状到现在两三个星期呈严重加重状态,便坚持请医生直接作了身体检查。后来花了90块钱(half pay,保险付了另50\%  \textasciitilde{} \$90)付医生的检查费,另花了大概\$8的药费,开了 14 片的一种什么药,但这些药一个星期吃完,真的就非常有效、完全好了,这一两年里都再没有这种困扰。

今年四月( 04/08/2013 )当生理期身体的不适再次困扰自己的时候,我还是最终去了校医院看病。这次接待自己的是一位女医生,见医生的过程觉得她非常礼貌客气,似乎我说的每句话每个字她都想记录下来,写在一个note便条上。聊起来,这也是老问题了,早在2009年春夏的时候也有一次我是就这个类似的病情看过医生,这些都是在自已的病历里写着的。那时也早就已经承认01年手术时查出过右侧卵巢囊肿,但她问我的问题是"Are you sexually active? How long have you been sexually inactive?"这是医生,她问的问题我得回答,但却总有一种被冒犯的感觉,很想骂回去一句,我有没有与你有什么关系?但只是从自己心里骂了一遍。这次看医生,后来查了尿样、两个血样(一个是 CA 125 ,另一个不记得了),做了一项身体检查。身体检查我付\%50 \textasciitilde{} \$90,看医生的费用完全我自己出$115,两个血样检查一个$ 58 ,另一个$82,这样$ 350 花出去,我除了从 CA 125 的结果知道自己没有得癌症,其它所有的费用也都对自己的现有的病情没有任何帮助。


\chapter{暑假实习}
\label{sec-5}
\section{暑假实习(0)}
\label{sec-5-1}
来到加州我就开始找实习。去面了一个Mountain View的小公司,面我的是清华出来的一位中国人学长,问我如何reverse string里的 words(刷过 lc 的都知道这是所有字符串里最简单的一个了)。我用了最原始的方法一个字母一个字母换回来的(汗!)后来被他提醒,想起stringstream的捷径,但具体的函数当时我还是没想出来,只能挂了。不过学长还很负责任地(很无奈地?)帮我点明了他问我这个问题他自己想要考查的思维和逻辑。

我来加州后表姐也帮我把我的简历传给了他们公司,因为表姐的强大(或许也部分受益于公司的文化?),我就被安排了面试。因为后来这个暑假我的确就在这个公司实习,因为这次出来写也即将写到自己实习的部分,也希望职场新人的幼稚表现能够得到牛牛们的批评指正,所以人物的称谓我还是用简称吧。面试安排在 05/30/2013 。面我的第一个人是我后来的第一个mentor,称之为B吧,中国人,看年龄,应该是senior了吧,倒并不为难我,直接用汉语问我的问题,关于操作系统,关于软件工程,code上只问了最最简单的 command line arguments pass in variables;第二个人A是我后来的第二个 mentor ,问了些简历上的小问题,考了 OOD ,设计汽车;考了最简单的算法, 100 层楼,从楼底升气球(还是从楼顶扔瓶子,不记得了,大概前者吧)在某层楼气球会爆掉,怎样在最快的时间内找到是在哪层楼爆掉的。都是简单的题目,基本也都还回答上来了,如果表达还算清楚的话;第三个人C是我后来的 manager ,也同样是问了简历上的问题和 OOD 。整个面试表现得并不是特别好,但主要还是他们给我机会吧,表姐让我给他们写感谢信,第二天我就给三个面试我的人都写了。后来这个公司给了我实习的机会。

\section{三封感谢信}
\label{sec-5-2}
\subsection{Dear Ms. B,}
\label{sec-5-2-1}

It was very enjoyable to speak with you yesterday about the test development intern position at XXXXX.

The job seems to be an excellent match for my skills and interests. The very kind team environment and strong technical skills that you demonstrated during the interview inspired my strong desire to work together with you in the team.

In addition to my enthusiasm, I will bring to the position strong code developing skills, and the ability to encourage others to work cooperatively with the department.

I appreciate the time you took to interview me. I am very interested in working for you and look forward to hearing from you regarding this position and potential responsibilities.

\subsection{Dear Mr. A,}
\label{sec-5-2-2}

It was very enjoyable to speak with you yesterday about the test development intern position at XXXXX.

The job seems to be an excellent match for my skills and interests. During the half an hour interview, you impressed me deeply with your clear ideas and logical thinking. And I was motivated then and have been taken every efforts and trying my best to get the chance to work for you. Like you are so confident today, I have been taking my steps gradually to achieve career goals to be professionals with enough skills and confidence just like you. Your today's performance is my goal for tomorrow for my career, and I will sure get great practices and learn a lot from you if I have the chance.

On the other side, in addition to my enthusiasm, I will bring to the position strong self-motivation, self-learning, courage, coding skills, and the ability to encourage others to work independently as well as cooperatively when necessary.

I appreciate the time you took to interview me. I am very interested in working for you and look forward to hearing from you regarding this position and responsibilities.

\subsection{Dear Ms. C,}
\label{sec-5-2-3}

It was very enjoyable to speak with you yesterday about the test development intern position at XXXXX.

The job seems to be an excellent match for my skills and interests. The available and potential tasks that the position requires, together with your sharp technical skills and impressive management skills inspired my strong desire to work for you.

As an impressive classmate or team member with quite a few classic projects, I have always been motivated, delight, and being very clear about my goals. In addition to my enthusiasm, I will bring to the position strong self-motivation, self-learning, courage, coding skills, and the ability to encourage others to work independently as well as cooperatively when necessary.

A friend in need is a friend indeed. Please let me express my sincere thanks for the opportunity and time you offered and spent on me. I am very interested in working for you, rewarding back my appreciation for you through the position tasks in action, and I look forward to hearing from you regarding this position.

\section{第一顿午餐}
\label{sec-5-3}

去实习的第一天,偏巧,VP正好来总结报告,他们一大team的人热热闹闹地去开会,我一个小实习生也混在里面他们又有谁知道,开完会去吃饭,开了大概三四大长桌,每桌十多个人。表姐同A,B,C和我都在同一个team, 外加一个和自己学校男闺密性格很像的年轻人D,他们都是正式员工,只有我和第二天来的另一个加州学校的E是队里的两个实习生,E的cube在我对面,A和B在我同一走道斜对面,D和表姐离我比较远,C只能更远。

吃饭的时候,我们女生坐了一大桌,我坐在了表姐同我的mentor B之间,B,C,表姐和我原本是坐同一辆车表姐开车过来的,但C来座位比较晚,她去了我们后排VP所在的餐桌,另外一部分男士坐了一桌还是两桌他们的桌,只有一位女士坐进了他们男士桌。我们这桌的小兵小将也落得个逍遥自在,可以随便吃随便聊。但意外却在我们这桌发生了。

点菜的时候很多人、包括我自己并不知道哪个好吃,哪个实惠,waiter说lunch special比Simon更好吃,表姐原本要点那个Simon的但改成lunch special了,但表姐说我们两个可以点不一样的,于是我还是点了Simon。后来陆陆续续上餐盘的时候,同桌上两个会点的女孩子点了铁板烧一类的东西,两个热锅贴端上来还在噼里啪啦冒着热烟和声响。这时就有人不淡定了,我旁边的B叫了waiter点了两个还是三个冰淇淋,意思是帮我们周围一圈人点的,不知道她点单的时候有没有意识到我们周围大家佯装的附和的不自在。

后来冰淇淋端上来的时候大家也是在吃,却吃得并不怎么有滋味。好在后面VP桌有容乃大,竟然也有冰淇淋端去他们的桌,小兵偶吃东西的压力都减小了好多,有没有?!!!VP桌的冰淇淋也端上来了,我们桌的也吃了不少,对面的女士们就要去洗衣间洗手。B是坐在右边最角落里的,我左边的表姐也去了洗手间。B之前已经自己去了一次洗手间我也很想随大家走,但如果连我也走了,B就要彻底被孤立了,而她还将会是我接下来两个半月的mentor,我不能走啊,于是我陪着她枯坐着,没话找话地聊着天。整个大长桌,右小半截empty,最右边孤零零滴坐着相依为命的我们俩个。

女士们回来后,表姐同对面的一个女华人manager站在餐桌旁边聊起天来。那位是另一个组的,工作上这两个team或许也有联系的吧,再加上餐桌上也可以聊工作之外的事情的吧,还有啊,那位女士刚VP开会的时候还在底下窃窃私语对我们manager C说表姐是她在这个公司最好的朋友来着。聊天本也并不意外,只是在这样的场合,是否,任何出列出格的举动行为都会给大家留下不好的印象呢?表姐大概只是因为能把我弄进去她很开心吧。后来表姐落座后,我站了起来,想去洗手间,但后桌VP桌没有给我这样的机会,大家都站起来一起撤离了。

就像刚刚过去的小小选择,就像B和表姐两个同样有性格的人的较量,在接下来的一两个月的时间里,在B同表姐的关系里,我同mentor B的关系也是三进三退,像一个漩涡,越搅越大,却最终不得不放弃;而我同表姐的 \verb,~ 亲情关系,像一缕青烟,经受过风吹雨打,却依然飘飘苗苗、颤威威地升了起来 ~, :!

\section{实习(一)}
\label{sec-5-4}

实习的第一天,C带我认了组里的同事,还有相关组的senior,C把我领到B的cube里,告诉我她将是我这个假期实习的mentor,跟着她做就好了,但这个过程我没有见到A,大概他来晚了吧。
我在自己的cube里坐不久,A来了,来到我的座位同我打招呼,A的前来我很意外,这是onsite三个面试者里感觉面我面得最 professional 的一位,很年轻,有着颀长的身材,帅气的面孔,也有着这个年龄段人里少有的礼貌、客气与和善。他来到这里大概一两个月,负责用python搭建各team test suites的 test automation framework 。后来他给E和我发了python IDE的链接和安装指南。看着他这么好,coding方面实实在在的能力也这么强,就真希望以后能有机会作他的实习生,让他来 mentor 我。第一个周有一次也们去会议室开会,E正好有问题问A,A打算向他演示一下E的问题到底是怎么 work 的,经过我的cube时,A看了我一眼,我便也凑过去了。后来中午还是傍晚,B问过我跑哪里去了(B同A的座位相邻隔壁),我便直接对B说,大家都在一个组,谁说我就不能从A那儿学点儿东西呢?

我同B的第一份任条是写MSTK的test case。是啊,最开始,专业里的第一份工作,我也并不知道该如何去写这些东西啊,那该怎么办,去翻文档,总有别人写过的吧,找几个例子出来先。于是我看例子,琢磨着第一个测SSD  temperature的该怎么写。而这时我也不得不承认,B大概早已经从C那里会意,在我就要下笔开写的时候,B已经写完了两三个其它SSD的test case可供我参考,而我要写的这个case的思路我也是先去同B讨论过的,所以具备了一切先决条件,下笔如有神 \textasciitilde{} 开敲。不到两个小时就弄好了,剩下的就是同B交流征得她的看法了。

我程序里用到的parameter都是从文档里调出来的,比如SSD芯片的上限温度就是55F,而B却坚持认为应该是55C。更好玩的是,我拿文档指给她看说就写在那儿呢,她却要坚持去问另外一个team的一位华人女士。B的衣着是每天都会换新的款式,不算很formal但也不很随意的工作服装,在这样的一个technical公司,在我这样一个新人眼里是多少有些略为刺眼的,同表姐聊天时表姐倒是说B有她自己的穿衣风格又有什么不好。我同B走过长长的走道,穿过一个又一个的cube,我感觉我们走的不是走道,是T台,我们走的是模特步!到了后那位女士也很不懈地说,是华氏度,“你用手摸一下,是烫手的,不就知道了?”于是我们再从长长的T台走回去。每天只走一次的话我还是可以忍受的,要是走多,每天超过两次,我想我会先崩溃掉的。

这时我已经深刻地感觉到,实习生,考验我们的也许不是coding,不是干活能力,而是如何同别人交流沟通,因为真正开始干活前的讨论准备工作和干活后的汇报交流工作会花去真正用来干活几倍多的时间和精力,重点还不够突出么?

我们每个周四向C发邮件汇报这个周都干了些什么?我的第一个test case两个小时写完,但真正clean up这份工作已经拖到了几个周之后。这期间,因为写不同的test case,读各种各样的pdf,有些几份资料之间不match的地方,是真正错了,还是只是新旧version的不同,我都把它们弄清楚了,而且我也把不match的地方列出来汇报给了C。

\section{实习(2)}
\label{sec-5-5}
E是坐我对面的实习生,比我晚来一天。我们site公司里周二到周四中午都是有饭吃的,我们两实习生中午也都去蹭公司的午饭吃(其实我们site可能有至少三个实习生的,因为我在总部第一天见到的一个小美男实习生也在这里吃中饭),所以有机会坐下来七长八短的闲聊。他是UC系统的本科生,第二年就可以毕业,对于他毕业后的去向,有时候他说想读研究生,有时候他会说想工作,具体他什么想法也许他只是不想让我知道吧。

我报到的第一天先去headquarter领门牌卡,那里加我一共有6个实习生我们同一天报到;我问了E他说他那天加他有4个,这所有的实习生应该都是内部系统的人refer过来的。他是他自己声称为Family friend的(据大家说是他舅舅)board里面的人refer过来的。比我小很多岁,却少年老成,比组里最为 professional 的A都还要显老,显得心机沉稳。

有一次周一还是周五中午我们俩个在食堂吃东西,另一个负责晚上订饭的正式员工也加了进来。后来知道他离婚了,有一个正处在恋爱季节的女儿。他头不大肚子相对比较大,为了以后叙述方便,暂且称之为尖人吧。印象里尖人同我的舅母(表哥的妈妈)比较像,极善察言观色,据说尖人也是表姐refer进公司得到现在职位的。

那天我们三个先前也不记得都聊了些什么,聊着聊着,尖人说他压根就不相信现在的年轻人能够一两年的时间里弄懂弄透computer science,当时的自己麻木不仁,丝毫没有明白这话里的意思,还笑着说,他也该知道,现在的电脑、笔记本满天飞,网络遍布Mississippi南北,这个时代已经远远不是他成长长大的那个时代了,生在这个时代的孩子远比他的同龄人要幸运很多,现在的孩子理解、明白computer science比他这个老古董当初学习、明白得快些也就不足为奇了\textasciitilde{}~我的英语结结巴巴,但他听完我的吐槽也拿我没办法,只能翻白眼。之后的实习经历才知道,这时的尖人话里有话,只时当时的自己没能听出来罢了。

\section{实习(3)}
\label{sec-5-6}

这期间,因为工作,B、C和我之间常常也有些邮件往来,一些B给C的邮件会cc给我,但明显感觉这个人把我干的活也都向C汇报成B自己做的了,这多多少少也会引起我的不满。我试着在每周四给C的汇报里侧面表达了自己对B的不满,但显然C对我这些隐诲的表达是视而不见的。

这期间,因为读文献,读别人的code,也无意中看到了B的bug。是程序员都会出bug,没什么大不了的。B也就是把hex转化为integer给弄成了每 4 bit乘以16。拿到这个bug我并不知道该怎么做,因为B的各种争抢豪夺已经激恼了我。

接下来的一次team开会我便竖起耳朵、拿了 120 倍放大镜看C与组里的人如何应答,C对B的偏坦态度让我很妒嫉又无能为力,而C对表姐的刻骨冷也让我难受。C是欣赏A的,对D也都大面上过得去。而我们实习生在manager的眼里就是一阵风,刮过了就没有了,而这风有没有刮过,manager或许也都不知道。似乎有一场战争,蓄势待发。

为了给自己争credit,在开会结束后B的座位上,我就给她讲了这个bug,她似乎有点儿反应不过来,不明白这个bug是什么意思。事后表姐批评了我,“人家是senior,你一个实习生,你怎么可以这么没大没小地当着这么多人的面指出别人的错误呢,这让别人多没面子多难为情?”

后来C大概是为B出气撑腰吧,B schedule了一个meeting,大概是讲她的medusa项目的分析实施方案。这个项目实际上也是A需要automation的一个test suite,到B这会儿要讲时,其实A已经完成了大部分的文案备案工作就在subversion里放着呢。当然那天办公室里的气氛已经很怪异,不知道来源在哪里。感觉上多多少少有点儿针对我吧,因为B schedule meeting时说话的语气英俊潇洒、荡气回肠的样子,仿佛我成了一切麻烦的制造者。那天在洗手间里我被公司里的气氛给逼疯了,躲在厕所里哭了。

但因为我心里对B有意见,向C汇报工作时跟我抢活儿抢功劳把我给抢烦了,而作为mentor,她也缺少了可以真正征服我的知识实力和人格魅力。即使在她最熟悉的领域里,她对待工作所表现出的一切都只是试-试、而不读文档、不去搞清楚原理的态度是我不愿接受的。所以她的meeting,我最好的办法就是无视。后来知道表姐因为一直呆在实验室里忙,也没有去;坐我对面的E少年老成察言观色也没有去,这样C带的一个team 6个员工就只去了一半,一个表演者,两个观众,三个没去的。感觉manager的领导很不给力啊?

B的会结束后C 1:1 meet team里的每一个人。后来最终的解决办法是:E和我,我们两个实习生也有了checkin subversion的权利,这就意味着我们可以公开、公平地竟争credit,对我当然是再好不过的消息了。

\section{实习(4)}
\label{sec-5-7}

周末的时候如同往年夏天我提了豆制品去表姐家玩。表姐问我干得怎么样?问B的关系好么?我忍不住咬牙切齿道“笨死了,实在受不了她了!”表姐赶快问我怎么回事,怎么到这种地步了?我便说了事实。周四一早就嚷嚷着说周五要用lab 里的station,她是正式员工,我只是小实习生,她有活儿,我当然是赶快给她让地儿好让她把活儿干完。周四中午就把自已的活清完,告诉她我用完了,station归她用了。她老先生从周四中午就开始测,测到周五下午,就测一个SSD power on hours,连夜跑测了二三十个小时,测不出来,还不知道问题出在哪里。这个Power on hours不就测试前记录一个current power on小时数,再power on几个小时,再去拿一次数据读一下这时power on小时数有没有更新么,更新的对不对?就这两个步骤,做一下加减法,能难到哪里去?去帮她看code,人家抄别人test case,power on几个小时后再去拿读一次数据的这一行code压根就抄丢了!两分钟就把问题解决了,就这别人还霸占实验台占了二三十个小时!人太笨了不服不行,我对表姐感概着。表姐打击说,又过了三天两早上,尾巴又开始跷起来了!我明白表姐要我不要骄傲,便不再说话,听表姐教训好了。

一回生二回熟,写完了第一个test case以后,我再写就多少test case都会感觉味同嚼蜡,兴趣和毅力都少了很多。后来根据不同的company的需要,根据不同的firmware release,写过上百个test case checkin到subversion里。但在我眼里,那些东西都不能再叫test case,我只是一个Emacs长工做一天和尚撞一天钟,做着修修改改、缝缝补补的活儿,拿一份工资而已。想要找到点儿自信心、成就感,那是万万没有的。

\section{实习(5)}
\label{sec-5-8}

那天我们拥有checkin subversion的权利是C在我的cube里亲自告诉我的。那次1:1里我对B不满的方面C后来的确是在B的cube里向她讲过,我亲耳听见的。只是讲话的声音比较小,不知道B有没有真正听进去,还是会会意成别的东西。那天C还说会帮忙清理我们的MSTK test case code,亲自管理B与我的任务分配,不会再让抢功劳、任务分配混淆不清的事情发生。

后来B和我的test case累积了很多之后,我受不了了,所有的任务完成之后从来不结,何时才能真正了结?于是建议B我们把第一第二版块的任务checkin吧。于是那天傍晚快要下班的时间她拿着自己的笔记本到我的cube里,我们并排坐着她一步一步教我如何checkin到subversion。等我checkin所有的步骤都完成之后,她说,我的这份code这里不好,那份code那里不好,所以说,应该把所有的code完全清理好、清理干净再checkin!

这时的我头已经10个大,100个大,1000个大,内心的怒火也无限放大\textasciitilde{}~C说她要帮我们check clean code,但她毕竟是manager,不会有那么多的时间来帮我们做这些事情;你是mentor,你至少如此教我么,为了报复,出你自己心头的怨气?

我心里起火,不是因为要清理程序,而是觉得你既然对我的程序不满,你早说嘛,干嘛等到别人真正交上去了才说?清理程序的工作量对于我这样一个emacs爱好者来说根本就不是问题,真正的问题是交任务的步骤不合理嘛!可mentor把你摆到这个位置了,改也得改,不改也得改,于是自己subversion checkin有第一份checkin就这么出师未捷身先死了(有了不少modified mark)。

\section{实习(6)}
\label{sec-5-9}

后来我把写过的各个任务都陆陆续续checkin之后,到那时应该有不下五个任务了吧,数test case的话多少也得有 50 个了。于是接下来的某天就迎来的E的第一个项目的checkin,因为注意到A帮他review得很仔细,E接下来需要修改的几个方面、细节都说得很清楚。

他们正式员工忙他们的,我们小兵玩我们的。我就跑到E的cube里,问他能不能帮我demo一下他要交的项目,我很想了解一下啊,我的test case我已经写得手脚发麻、写得山无棱天地合,急需更换新鲜血液啊。他见我这么热情洋溢、心潮澎湃,就向我演示了一遍。很明显地,我写再多的test case,我没能从mentor那里学到多少东西,心是空的;而他做的是一个类似一个小app之类的,需要、能够学到稍多一些的software engineering新思维,这是我从B那里学不到的。

对E的项目,我是心生羡慕的,便睁着诺大的眼睛热切地问他,”Do you feel you have learned a lot working on this project ?”他点头;我再问”Do you feel satisfied (when the project is done) ?”他已经控制不住自己发自内心的甜美微笑了,是的,学到了很多,他边点头边回答也边回问我,”How about yours, do you feel the same way ?”被回问到这个问题,我终于是难掩失望与落寂,热情的肥皂泡破灭了,脸上严肃起来,苍凉地说,我的工作只是重复一个个的test case, “I feel nothing.  I have no feel !” 病殃殃地回到自己的cube,一屁股跌坐在椅子里,心里难受极了。

\section{实习(7)}
\label{sec-5-10}

没有想到的是,伴随着我的苍凉,office里也变得异样的安静。这是一个欲求不满的实习生的诉求,那种对知识的渴求、那种想要学习的欲望,或许这在每个真正经历过的人心里都是相通的吧。

只是这事不久的一天中午吃饭,board里的一位长者那天坐进了我们的餐桌,据说这位就是那帮忙把E refer过来的人了,就叫他老K吧。我们几个年表人,A、D、E和我,还有D的一个小伙伴,我们五个年青人是常常混在一起玩的,一起吃中饭,一起周五的中午出去淘宝、打捞好吃的,吃完中饭一起出去到block里走两圈,只是E是美国士生士长的,语言思维都很活跃,他不喜欢跟我们走路。

Board里的老K都坐进我们桌上来了,那自有趋炎附势者同坐进来,饭桌上也就绝不会只有我们五个。那天中午,老K给我们讲了两个他早年在中国人餐馆吃饭的故事。一个是说,一天他们从老师还是谁家的party出来,他们一大帮人就去了一家中国人餐馆。进去后他们点了单。后来服务员给他们端了一盘不是他们点的,那他们就赶快吃,后来服务员要收那盘菜的钱,他们不满。桌上A是abc,D、他的小伙伴和我都算华人,只是D和他的小伙伴是印尼人大概。讲完第一个故事后,我心里已然不快,只是没有表现出来;老K的第二个故事我不记得了,但是是类似的情境。等他第二个故事讲完,我可能难掩不快吧,也都没有说什么,大家也都站起来直接拿腿走人了。他是board里的人,我只是一个小小实习生而已,我惹不起他这样的大人物,我躲还不行么?以后的日子里,虽然偶尔K也会坐进我们小伙伴的饭桌,但我基本上同他没有任何正面直接接触。

额,后来看subversion history,看见E 第一个项目的checkin没有加label。因为是app,他checkin了大概1000多个文件,里面应该有很多文件是属于driver本身自带的。在诺大的history表格里,只有他这一行没有标题,不知道他是想要引起大家的注意,还是真的只是忘了。后来他的checkin都加了标题,这是后话。

\section{实习(8)}
\label{sec-5-11}

表姐帮我弄进去实习,已经是帮了我很大的忙。从开始上班后的周末,我还是老样子一如既往地提豆制品周末上表姐家去。与此同时,受E和他的 mentor 的影响,他们都极其热爱锻炼身体,我也为自己烈了一个锻炼计划,开始了自己的运动项目。我买了一个呼啦圈,每天晚上500圈一个方向,我轮流转上个 5、6 次,周末一般都跑去hiking了,效果还不错。

B会议前后的周末,我把小表姐也拉上,我们三姊妹一起去hiking了。就工作上的事情聊天,但聊得不是很愉快。大概意思应该就是表姐帮我弄进来,我应该从物质上感谢表姐。那是一定的,可此时的我并没有太多的资源可以感激她啊。如果我照着每个周去一次豆腐店的节奏下去,我应该可以花掉\$200吧。我现在拿的工资是需要秋天用来交学费的,交完了学费,我就什么也没有了,连生活费也没有。而同表姐,来日方长。就像几年前某个寒假表姐动了买房的心思时,小表姐搅混水说她一同事租别人house里一个房间每月\$800。若表姐真买了房子,只要表姐不原本\$500的房租收我\$1000,我又有什么理由不同表姐一起住呢?这么多年来表姐因为照顾我也花了不少时间精力吧,若我们两姐妹同住,表姐不喜欢做家务又有什么关系,我顺便多做一个人的饭菜又有什么关系?

但或许是我真的做得不到位,表姐显然多少是有点儿气的,至少在那段日子里,我是这么认为的。

\section{实习(9)}
\label{sec-5-12}

那天没有去开B的会议的周末,B、表姐和我都在公司加班(我在那里学习)。表姐气冲冲地把我拉去了Costco为B买花,买了两瓶保健品,买了晚餐的食物和饮品,回到公司表姐要我向B道歉,也因本质上我并不觉得自己错了什么,错了多少,我说不出来,就始终没说。只是表姐这架势吓着了我,表姐又何至于要这样。就当是看热闹围观,也想看一下就接下来的戏该如何上演。

我们三个在大会议室里坐着吃东西聊天。B是表姐refer进来的,我也是表姐refer进来的,但我们俩个却合不来。看表姐把自己的姿态拉得这么低,估计B心底即便有多少的怨气也早就消散了吧。

B很有自信,给我们讲她的故事。忆往昔,B也是位一等一的人物。我猜想B年轻的时候应该是长得还不错、很有手段的抢手女。嫁给她老公随他出国,先去欧洲,辗转反侧,来到这个国家。同那个年代的大家一样,三十多高龄转专业学计算机。 忙的时候也同时做过两份工作,挣得比老公多,白天忙得累得要死,晚上回家继续同老公吵架;不同老公商量直接打卡给自己爸妈买房。她老公也是个人物,晚些时候回国内发展事业,灯红酒绿,几年都不愿回美发展。他们的婚姻,似离非离,离了没离,外人又如何看得清楚?

\section{实习(10)}
\label{sec-5-13}

So far,工作上表姐还没怎么真正在我最需要的时候帮助过我,除了帮我检查过一次邮件,说我写的句子太长了,谁有时间读你那么长的句子?既然表姐待我冷冷的,既然表姐逼我向B道歉,我又为何不能敞开心扉、真正再给大家相处一次机会呢?B的cube里还摆放着她往昔的奖杯,小兵我也很想看到她的神奇之处啊。于是那天,当被告知她第二天要因公出去办一件什么事情的时候,我便略带讨好地打趣她说,“ 那你明天要不要带上我去给你当小秘呀? ”这时我就发现了一个惊天小秘密:那个从来都声称自己不会讲,听不懂汉语的A居然在一两分钟之内跑去表姐的cub e向她请教问题安慰我的表姐去了,小概率事件 0.0001 \%! 看来表姐同B不合的传说渊源已久,扎根很深啊。

而到这时,subversion里,除了我隔三差五地checkin MSTK test cases,B也做这个项目,她是我们组里负责这个项目的responsible person,至少supposed to be。除了我的第一次checkin修改过code,其它就都是一次checkin了。但因为C分配给我的工作她会抢,她抢到我前面的她自己就会直接checkin,这中间也有我们的任务总是晚很久才交上去的原因,她自己也常常会心血来潮想到哪做哪儿,还有就是各company的要求不同,firmware release之间version版本的不同,结果subversion里MSTK test suite我们两个checkin的程序就像拧在一起的两股麻绳,已然傻傻分不清楚 (一定要去细分也应该还是能够分得清楚的,就是怎么说,缺少了条理性,即便想去分,也很烦啊),office里异样的话语声响,昭示着大家正看我俩笑话呢 !

\section{实习(11)}
\label{sec-5-14}
那天manager C是出差还是什么,反正是人不在。B和我决定都还再好好想一下。我坐在自己的座位上、抓破头皮地想要把一切都回想清楚,也不知道B有没有什么好的头绪。这时听见A在E的cube里问E是不是overwrite了A在Linux系统下的A的主文件夹? “You should set the path point to your own main folder. ”

是的,这是同C一起竞争这个 manager 职位的 candidate ,只是在C胜出后屈居在C之下做着 test automation framework 的工作。在C不在的时候,尽着自己的努力帮助自己的同事,平衡一下team里的关系。这种能力后来在QQ群里偶一见钟情的一位大神身上也有见到,这是后话,以后再表。

中午吃饭的时候,我们还没吃完,尖人早早地坐到了我们的餐桌。我们那天的话题自始自终就是 about God 。因为那天同B的混乱,我蔫不呐叽的,尖人信不信教,我不是很清楚,但他是同E一样,一起向A提问argue的。我吃自己的东西,默默地听他们聊天。那天,E鲜有地加入了我们饭后散步的行列,但聊天 的对象始终是A,而聊天的话题依然是 about God 。大概通过谈话、聊天和发问,可以彰显、证明逻辑思维的发散和精深吧。我对自己的小伙伴们感慨,” When the philosopher meets the believer,  the conversation is never ending\textasciitilde{}~ ”

\section{实习(12)}
\label{sec-5-15}

后来一天周四还是周五下午,略为轻松,B带我在lab里debug一个test case。尖人看见了,问我们在做什么,B便笑说我们在debug程序呀。因为B已经带我、在lab里教了我该如何debug,所以接下来在office里,当C过来说她可能下周要出差,问起B的各种bug,B说她还有好几个bug没有fix的时候,我已经全然领教过了B在coding方面的光荣事迹,便为她减轻负担、自告奋勇地对C说,” If B can help figure out and organize the ideas, I can implement them and finish the coding part.  ”说这话时我们站在B cube的走道里,B对面的senior也在,C和A都在,所有的人都认为我这实习生、毛头小子又翘尾巴说大话了,却没想到最终是把B自己给陷了进去。所以接下来一个周C出差,正像她之间说过的,她写邮件给B和我,要我debug B之前项目遗留下来的五个test case bugs “under B’s direction”。

五个test case,一般的、简单的、有思路的,对我这样的Emacs coder来说根本就不是事儿啊,因为MSTK那么多的test case摆在那里可以参考的啊,于是所有其它的test case 一两天的时间里基本就全做完了,除了一个难的。那个我觉得难的B也有给我讲过她的思路,但我读过文档,她那方法行不通,所以周二白天的时间我压根儿就没去code她的方法,而是自己去读文档,希望能找到合适的方法。犹记得那天坐在lab里,B当时也在,尖人像是从我们旁边路过,连问都没问我们在做什么,就直接说,”You guys were debugging on Friday, why are you still debugging ?” 那会儿我已经读过部分文档,最怕在自己真正遇到困难的时候遇上猪一样的队友,急火功心,开口便维护自己道,”We are debugging, why are you laughing at us ?” 尖人问话的表情是取笑的,我的表情是严肃的,坐B cube对面的senior正好从lab出来,看了尖人的好戏也笑开快笑死了,而B却去给尖人陪笑脸说了什么blah blah两三句,然后尖人自己识趣地走开了。尖人走后,B警告我说我这样会把人际关系搞垮掉的,要我说话做事注意一点儿。

那个test case需要先修改什么地方(firmware, log page?这个不记得了,不是重点)的一个bit,再测试修改这一bit 后某些读或写行为的正确性或一致性,但我们没有找到有效的修改那一个bit的方法。我知道要用logSense或者logCommand中的一个来修改,但这些命令的参数是怎么传进去的,我找不到现有的例子,仅凭文档里的理论,我只学了一年,脑子就有点儿稀里糊途的。那天所有的压力都是属于我一个人的,一时间喘不过气的时候还自己独自一个跑出去在block里转了一圈。


\section{实习(13)}
\label{sec-5-16}

那个周二lab、office里的氛围是我这个实习生很笨,做不出suppose应该完成的东西,还眼高手低、自以为了不起地说了大话,那天的自己真是压力山大!连平时我们一块儿吃饭、一块儿玩的小伙伴A都因为压力在我头上顶着,他都不好意思中午再同我们(主要是不想再同我吧)一起吃饭,自已跑出去在外面餐馆吃的,那天饭后的走路自然也是没有的事。

B显然没能领略到我们的困难。她同任何人一样也觉得我笨,甚至对我说话的语气、腔调都变了,意思是我只要把她的思路实现出来不就可以了吗?我不服啊,直接告诉她、拿文档指给她看她的思路是行不通的。但她也不服,因为她这人不怎么讲理论。于是那天晚傍晚、人群散后,我们就在实验室里把她很不服的方法用code实现了一遍,当然是得不到我们想要的结果的。

于是我们agree,第二天早上来上班后就请教别的team的人。这是她一贯的行为方式,但凡我们不会的时候,不是学习、试着自己去解决问题,而是去找别的team的人帮忙,全然忘记了这个是她自已的项目,她才是那个应该全权负责这个项目的人,她supposed to be这个项目的专家才对啊。

于是第二天早上,我们两个惨不兮兮地跑去别的team问,那个印度女孩子也轻描淡写,略为不快,可能因为前一天办公室里的氛围已经那样了,耻于告诉我们答案吧。B自己还想走开,我已经用中文对她说了两次,“你不要走”,而她还在继续坚持打主意想要逃跑时(试着挪开脚步远离火灾现场),我也火了,直接用英语对她说,“C asked me to fix the bug under your direction, you ARE leaving it to me !”B没走成,但印度女孩子也有会议要走开了。


\section{实习(14)}
\label{sec-5-17}

后来那天lab、office的氛围终于又变成了,不是我实习生不会做,是她mentor自己都不会做。表姐有着抑制不住的开心,这大概是她俩不合的渊源吧,我所体会到的,B既喜欢抢别人的功劳,至少从表姐手里也抢过,又喜欢抢项目,至少从表姐手里也抢过吧,从别的team里有没有抢过就不知道了。真正能力强、能者多劳,别人也没话可说,但她抢到手的项目却又都大多行百里者,半于六十,一遇到艰难险阻,总需要别人为她收拾乱滩子。B还喜欢抢机器、霸占机器,实验室里一个station不够,还要自己cube里backup一个,也听说过她因为霸占实验台太久,又不出活被别人当脸骂占着茅坑不拉屎。B作我mentor最开始的时候,有时她必须得用lab实验台,我就坐她座位用她cube里的station,后来同她的关系越来越糟糕,她的座位我就再也不坐了,我们只需要有必要的工作上的往来就可以了。

于是那次等C出差回来,我们fix了几乎所有她临走前布置的bugs,除了那一个攻不破的没有完成。这也是B几个月时间累积下来没能及时完成的任务。有时候想想也会觉得很奇怪,这么简单的事情为什么还要拖几个月,这点儿coding对她来说就真有那么难吗?

后来暑假快结束同表姐聊天聊到这样事情的时候,表姐问我这件事后来是怎么了结的?我告诉她,我是实习生,我经验不够、做不出来情有可原;她是mentor ,又是她自己的项目,她也不会,我能有什么办法?那是office里第一次浩浩荡荡的同仇敌忾大氛围,别的team不帮她,她有什么办法?后来是我实习第一餐午餐同表姐聊天的女华人manager(她是firmware的manager,负责那些firmware release和相关的bugs)会意上层旨意后给她面子,上B的座位去找她,哈哈哈哈大笑说,那个东西原本就做不出来,做出来才奇怪呢(座位与B相隔不远,坐在座位上的小兵偶听得满头黑线,那哈哈哈哈大笑声,偶这个耳朵带点儿聋的都听得犹为刺耳,好不好?)!

\section{实习(15)}
\label{sec-5-18}

当然,这次的buggy事件,也让我真正领教看到了B的神奇之处,那就是虽然她对项目困难、钻研不透、无法深入,但她很乐观。在第二天周三早上,当我把自己前一天所承受的所有压力都归还给她之后,她还始终为自己制造机会、给自己找借口借话说,“我有很多priority,这只是我很多项目中的不重要的一个而已!”这也是一位很有热情的实干家,coding方面确实很弱,但凡是command-based, interface-based 项目她都可以拿出她 120 分的好奇和热情尝试下去,直到有一天厌倦为止。

这件事情结束之后,有天傍晚快下班的时候,我上尖人的座位上去找他聊天,跟他开玩笑、劝他说,他以后要学会正当防卫,解释说那天我正高压运转,如果当时我手边有石头或者剪刀什么之类的话,我可能会做出对他很不利的举动来。他笑说他相信我不会的。然后我们聊土豆,各种各样的土豆、各式各样的做法,土豆丝、土豆片、土豆泥,炸的、煎的、煮的、炒的、凉伴的。他说他喜欢吃土豆,希望下次我们见面时我能帮带giant potato给他,我答应了他,好的 \textasciitilde{}~ !

与尖人,我们几乎没能在任何事情上达成一致,除了一次晚饭饭桌上的聊天。那天我们一大帮人坐成了一个大圆桌,聊到他的女儿,他强调说,父亲在女儿心目中的形象非常重要,会直接影响到女儿将来的择偶标准。显然我被这个话题吸引住了。他接着说,为了防止他女儿变得stupid,他想做一个好父亲,要为他女儿树立一个好傍样,希望他女儿将来能够找到一个像他那样稳重如山的好男人。 回想自己成长过程中同父亲的关系,回想自己心目中最初的以表哥为模板的小混混男朋友形象,突然就眼眶有些湿润,点头冲他说,”  I can’t agree with you more\textasciitilde{}~  ”。

\section{实习(16)}
\label{sec-5-19}

日子还在一天一天地过,若真像manager C出差那周一样任务分明,B不抢,我或许也就没有了那么多的烦恼,至少没人跟我抢啊。而E也是慢腾腾不知道在折腾什么东西,也没见checkin任何任务。而更主要的好消息是,我接下来的小项目是用python做B medusa项目的一个plot。B对我讲时,她的目的是希望我最终能做出一个类似user interface的界面,这样从这个界面里可以直接选择多个文件,随便挑选两个变量来将多个文件的plot图形叠加在一起呈现在一个plot图形里。

前面提过了,这个medusa是A test automation的一个test suite,而A是用python的专家。这至少已经意味着我可以同A有更多的接触,能够有机会从A那里学到更多的知识。

其实,老老实实承认的话,我还从来没有真正自已主动地写过python code。因为虽然上学期我导师的算法课用的是python编程,但因为我只是旁听,我没有写课后作业、要上交的作业,期中考试也没考编程,所以我的python也只停留在旁听课程时抄大屏幕上导师的笔记而已。但挑战也是机遇,不是吗?这就真正意味着我的生命开始鲜活起来,我可以开始真正学习接收新知识了!

但我没能想到的是,因为B几个月累积下来的bugs在C出差的那前几天被我们全给做了,她在office里的地位、影响都受到威协,而她同我抢活干,抢credit的心这时却有增无减。

\section{实习(17)}
\label{sec-5-20}

盛夏的花期正好,而我学习用python编程的干劲正足,花不醉人人自醉。最开始熟悉这样一门语言,学着debug一些小毛病还是稍微费了一点时间的,最开始完全不开窃的时候,D和A都有帮助过我,B自己不会python,她也还想学来着。D帮我解决问题时给我更多的是鱼,而A则是授人以渔。

很快单图的,就是一个图只画同一个文件里的两个变量的图的代码就写好了,要不了几个小时,但当我从多个文件来读同样两个变量,再把这多个文件的图形放到一个里面的时候,脑袋就有点儿转不过来了。这个可就不是简简单单的程序语言的bug了,这就涉及到理论基础了。前面讲到过,我的CS120、CS121分别是一个TA学生和硕士学历的讲师讲的,讲得并不深入,而面试时虽然我近似于背答案答出了问题,但在这 code里,我的object与instance之间就变成了一片混乱。

因为我请教了A,那天傍晚A坐在我的cube里,在我的笔记本上帮我写了个例子。几个不同的 object,几个不同的member function,几个不同的instance,通过一些不同的操作,A帮我稍微讲解了几句之后,要我回答哪些instance的特定函数调用会返回什么值。毕竟还是理解过,虽然是当初没能理解透,但经A这么一语点醒梦中人,瞬间就明白了这些不同概念之间的相互关系了\textasciitilde{}~ 所有他列在例子里的问题就都答对了。

就在A在我的cube里帮我理清这些概念的时候,site里其它组的一个中国人还到我的cube里来找A,站在那里问A某个test相关的问题。补充一句,A比我早到一两个月,我去实习期间,早期有听过他两三次例行月(半月)报,文档、幻灯、coding和归总,都做到有声有色。到此时已经automate多个test suite,已然成为了这个site的不可或缺的核心人物。这几乎是我最后一个月(加上同B overlap的时间,不到五个周吧)实习时间里A花时间在我的项目上最多的一次,但这也架不住有人会故意放大这一点儿。那个中国人离开后,A帮我点通我理解后,他也就离开了。但那也是让自己最觉寒碜、最为愧疚、自惭形秽自己基础知识薄弱的一次,后来同A所有的项目,我基本上都是用后天的努力弥补了先天的不足。

\section{实习(18)}
\label{sec-5-21}

那次Bug风波后的组会上,C依然是对B百般呵护。B迟到在lab里忙,C要我去找她。我去把她请回来来到会议室里,B的任何话语、任何建议都弥足珍贵,C没有否定对B的任何语句;而C对表姐的态度却不明所以地冷。E和我在C那里应该就是空气,她呼吸、看得见,但她视之为无物。也不得不承认,A是游刃有余的,百炼成钢,还具备了柔韧性,C的问题、组里状况百出,他则随机应变。

那天周五快要下班时,大概五点钟左右吧,可能是C在下班之前,发了邮件给我,cc B要我implement B在这次会议上提到的关于subversion上MSTK test suite的一个什么flag类的东西,感觉更偏重于subversion,而不是MSTK或是test case, coding一类的,我没有头绪该如何做。收到邮件,我第一时间光速跑到lab里找到B,周五的下午lab里还坐了不少其它人在做项目。我对B简要描述了我的新任务,她说,那个应该不会难,但她今天有事需要早走,过会儿就走。她说,等她周一再指导我该怎么做。

是的,对于这个没能从她身上学到多少新知识新思想的mentor我一直还都是百般依赖的,怕她有什么好东西藏着掖着,不教给我;也怕她老先生像我第一次subversion checkin那样背后使拌子。而后来跟着mentor A,反倒是每个项目都是在自己独立完成之后,再请A帮review idea和提高提升的,这是后话。

我信以为真地等待着周一到来好work on new task,但第二天――这个周末发生的事情却直接导致了一周的风起云涌,有些心境注定是回不去了。

\section{实习(19)}
\label{sec-5-22}

接着这周五的周日,中午偏下午我去公司学习,看见了B周六还是周日早些时候发出去了邮件,不用说,大家猜得到,她把周五下午C分配给我的活抢着已经干完了,邮件都已经发出去了,倒是没忘CC给我。

回想我周五下午同她打招呼时,她信誓旦旦说过的话,我那个火大呀,无以言表。C也有给B分配任务,是关于一些什么bug的,cc了我。既然B怕困难,不去干原本属于她该干的活,偏要来抢我的任务,那我不怕困难,我也抢不赢她,我去干她不想的脏活累活好了。周日下午,气头上,我给C回邮件,cc B,告诉C,我的任务B又抢了,因为她抢了我的事情让我没事可干,那我去干她的好了,”Even though I am not sure I can fix those bugs yet, I want to try. ” 临了,还说,”By the way, I need the Bugzilla history in order to fix those bugs. ” 我没有Bugzilla的账户,而要去fix B的这些bug,我需要这些bug的history,哪怕是text的也好,好歹那些纪录可以帮助我追踪朔源,找到问题的所在。

因为邮件不止回给C,cc B,给C回邮件时,C原邮件里原本有多少账户,cc了多少账户,我也就回了多少账户,cc了多少账户,所以周一早上,office里的气氛早就奇怪开了,因为我给C邮件里的语气火气可能是难以控制的大,应该是大到C不应该再忽视的程度了吧。

\section{实习(20)}
\label{sec-5-23}

因为我已经开始在写python程序了,因为我同A的联系也相对之前来讲也多了起来。而且仔细去想一下,B自己都不会python,为什么她会安排我这样一个她并不擅长的项目呢?怕我把从她那里学到的东西教给表姐?表姐已经是不少项目的灵魂人物,她压根就不需要B的那么些七里八里的东西呀?我敏感的嗅学就嗅出某种可能性来,很神往啊。

于是就等到某天半下午一两点种,E跑到A的cube去问问题,我原本是还在做一个什么别的项目的,见状就即刻停止了手上的工作,赶紧去翻自己最后这个python项目,看看还有什么问题,还有哪里不懂的,打算抓住机会紧挨着E去问A。 

E也很给力呀,下午黄金时间请教问题也请教了二十分钟左右,这样就给了自己足够的时间可以准备好要问的问题。万事俱备,只久东风,我等得花儿都快谢了\textasciitilde{}~ A的cube在我同一走道的斜对面,E的问题一问完,E一从A的cube里撤出来,我就凑上去了,跑到A的cube去问他我刚刚准备好的python项目的问题。当然早说过了,A也是游刃有余、十分聪明的,明知道我这个无奈瞅准机会、算着时间走进去的,当然会以自己忙没有时间,建议我去请教B的回答把我给挡回去了。这缺的就是一道过继手续嘛\textasciitilde{}~ 

\section{实习(21)}
\label{sec-5-24}

给C发邮件后的第二天周一我在lab里忙,C先找到我,单独问我周五快下班时那个任务到底是怎么回事?我便说了是B周日抢着做了,并且把我原本正确的MSTK test case 给改错了。C说是这样吗,她要我把test的结果准备好后给她看。那天我是在lab里度过的。C同我谈话后不到一个小时,当C再从lab路过的时候,我就把B改成错误的test的结果show给她看了,error message什么的在lab station的台式机上显示得清清楚楚。

必须承认,那场事故,对我俩打击、伤害都挺大的。当时的我们、至少是我,是没能意识到B的身后有着宠大的支撑体系的;至于B自己,我不是她肠子里的蛔虫,她的想法我也不是很清楚。B是当时就有点儿懵了,不敢从subversion check out一个版本来测试;我是胆子大、理直气壮,虽然对我的打击也是后序的、深远的,后来的自己源源不断地承受着这件事情的余震伤害。

在当时的自己看来,这些都是老问题了,就是因为你抢,你忘了,你的meeting别人都没有去参加?meeting后1:1上我向C反应过,C在B的cube里对B讲希望B以后能注意一下,我坐在自己的cube里亲耳听见的C对B讲过。后来C给我分配任务就真的像她自己说过的很明确了,该是谁的就是谁的,为什么你还要抢呢?

这样那个周,前半周我气场强大,在这明抢面前,她理亏;但后半周却不知不觉间、无声无息间B就变得更顺畅了,因为大老板出来给她撑腰,那个site里最大的manager出来同office里的人谈话,就在office里谈,谈话的对象还不是任何manager,而是平时office里比较会察言观色的人(意味着鼓励他们搅混水),来调动office里的氛围。这氛围为什么就这么像是针对我的呢?这时我的实习已经差不多过了一个半月,还有三四个星期的时间学校就要开学了,为什么这一个半月的时间我就没觉得自己真正学到什么东西呢?不服啊不服,心不甘情不愿地,我跑去老虎机用一个quarter买了袋甜薯片回到cube里,那site里最大的manager就在office cube里同别人谈他们的,我坐在自己的cube里吃我自己的,嚼着薯片必要时发出人类咀嚼它应有的声响,这心理实在是不平衡啊,实习生在公司里就是注定要这么被无视、蹂孽的么?

\section{实习(22)}
\label{sec-5-25}

后来C应该是看出了我的不满,我嚼薯片的声响所表达的不满site里最大的manager当时应该也不难看出来吧。周四的晚上,白天里那个这整个site里官衔最大的manager把我叫到他的office里去问话了。

他问我是不是出什么事情了,我原本还有点儿不太想说话,可当他再接着问我是不是在这里做得不开心时,白天自己在office里的那份不满不服就瞬间化作滚滚泪水、噼里啪啦、大珠小珠落玉盘了。Manager安慰我说,不要害怕犯错误,实习生在这里 ” suppose to make any kind of mistake they want ”。我有点儿云里雾里、不明所以,便直接对manager告状说,我在学校里拿不到credit,我想通过工作来检测自己,但是B同我抢活儿干、抢credit,两个人像拧在一起的麻绳一样,我讨厌别人跟我抢、纠缠不清,虽然后来C帮忙勒令分开,但B还是那样抢,而B的工作方式、不求钻研、遇到困难的行为方式都是我不能认同的,从她那里我能够学到的知识、idea和能力都很有限,我对我和B这样的关系很失望痛苦,我有点儿同她干不下去了。

见我已经这样了,manager竟然是在第一时间立即先自己认错,说没能先配得力的mentor给我是他们的错,让我很是惭愧,心有戚戚焉,一时间我不知道该说什么话、该如何表达才好。manager紧接着就问我,那我想同谁干,team里还有A、D和表姐,他帮我把这些组员的名字都列了一遍,见他问我问得很真诚,我也就很直接地表达说,” If I really have the chance to choose the mentor, I DO prefer working for A. ” 听我说出这话、作出了选择后,manager立刻就同意了,说他会同C讲,明天就换!

\section{实习(23)}
\label{sec-5-26}

第二天上午,C 1:1我、B和A。C对 我讲,换了mentor,要我不要给B太大的压力,说换mentor在实习生的实习过程中也是很正常的事,大概是说希望不要打乱正常的工作氛围吧。到中午吃饭时间(到12点零几分了应该是),C应该把我们三个人的会议都开完了(A是最后一个,我第一个,B在中间)。只是那天已经是周五,我同A以及B都没有表现出同以往比较有任何的异样,真正A mentor我的启动应该是从下个周才开始吧。

那天中午在lab里,其实那时B同我走T台去找问55℃的女士确实有问过我,说那天她们组开会,老板帮她们订了三明治,问我要不要同她们一起吃。我对来自这位女士的这样一个问题感觉很奇怪,但还是直接回答说,她们组的人我都不熟悉,我们小伙伴也已经约好去headquarter总部吃汉堡呢。后来D因为等不及我们开会,也早已经同他朋友约好了时间,D便先去总部会他朋友了,等A也开完会( 12 点多钟),我便搭他的车同他一起去总部大家会合了。额,想起来了,D的小伙伴那天在那位女士team里开会,所以不能和我们一起;那天早上E没有来上班,半下午可能离下班时间两三个小时来的吧,那周前半周的什么时候他有问过我这种情况需不需要向manager请假,我很意外,那种假有什么理由可以不用请的么?当然,我并不知道他最终有没有请假。那次是E唯一一次没有加入我们周五的午餐。

那天换了mentor(这还直接导致了一两周后我生日时一件很开心的事,以后再说),原本是一件非常开心的事情,但接下来的下午在实验室lab station里我独自走完的一步却深深地震撼了自己。

\section{实习(24)}
\label{sec-5-27}

那个周五的下午,因为换了mentor了,以后同B共用的那个实验台我应该就不会再用了。我在lab里整理自己的实验台、文件夹,不必要的文件、过程文件也需要删除掉了。

那天下午,当我用公用账号(至少同B share,别的team的好像也有人有这个账户,比如那个55℃的女士,但总共有多少人用这个账号我就不知道了,这也是我所拥有的这个station的唯一账号)login我平时用的lab station的时候,机器上显示的界面是Visual Studio C++ MSTK test case的测试结果的界面,上面有error message,这个test case正是这个周一我显示MSTK subversion checkout B修改后的结果给C看的那个,出于好奇,我去追踪了一下这是谁的test case,居然是我自己的!试着去想一下可能的原因,脑海里就会马上想起一个事实:就像Google chrome Settings有一个选项是On Startup,  Continue where you left off,这个station Visual Studio C++的default选项是Continue where上一次test case停留的地方。而当我把build/rebuild project/solution生成的.exe删除掉后再重头开始build,我就得到了我自己想要的结果。我是一个Linux/Emacs person,在此之前,基本上所有的作业、项目我都是在Linux下用Emacs code,用terminal g++/gcc command line debug的(假如要print out什么变量,Emacs macro用起来也很方便),所以当这样一个看似station公共账户造成的B与我始终拧在一起纠缠不清的假像,实则暴露了自己对IDE环境不够熟悉、更甚一点儿,对程序语言的compile、运行原理没理解透的事实。

那么究竟是从什么时候自己开始忘记这一点的呢?我努力地回想,上次debug B的五个test case我都还是记得的,应该只是最后一两个星期的事情。于是那天下午我一直坐在lab里,处在新旧mentor更换的交接棒上,我把自己最后两三个星期run过的test case都再run一遍,检查一下我还有没有其它test case在测试的时候被自己遗忘了这重要的一步。当然测试的结果正如自己所回想起的,也只有最后这次B修改过的这个。

\section{实习(25)}
\label{sec-5-28}

刚刚过去的同小伙伴们一起出去打捞美食的欢声笑语犹在脑海里,这份摆在自己眼前的错误对我来说来得实在是太意外,意外到毫于防备、吐槽无力。那天明知道自己从下周就要换成A是mentor了,但认识到理论上、知识上,自己还是错了,我独自一个人在前半截的lab里连续坐了几个小时,坐到下午五六点钟,心里说不出的难受。

这次事件再次证明了,B和我就像两股拧在一起的麻绳,浑然一体,傻傻分不清楚。她可能遇到一件她比较拿手的任务,兴致一来,她就动手把它给做完了,虽然她忽略了那任务是分派给我的事实;而且那份任务的出土建议也原本就是她自己在先前的会议上提出的,她或许真的只是在她熟悉的领域、任务上有太大太多的热情吧,我应该试着去理解她的。

这个错误 错得如此戏剧、离谱,让我也禁不住去问自己:究竟是什么原因让自己一开始就那么坚定地一口认定一定是B错了,而对自己就不曾有丝毫的怀疑呢?答案是明显的,是人际关系的失败! 在我们数不清的T台来来回回里,在发现B的bug里,在二三十个小时的power on hours里,在(她cube里)同她一起work on  high priority command-line based task她读题目要求粗心大意里,在fix她遗留下来的bug里,在她完成了发送一封社交礼仪邮件仅仅告诉其它相关人员station is available and ready for fix俨然完成了一件壮举里,我对她从来都没能真正建立起平等、尊重与信任! 与此同时,没有了表姐的唠叨和叮咛,我的心在膨胀,我的骄傲也在膨胀\textasciitilde{}~

那一次那个最终摆在自己面前的错误是整个夏天最让自己振聋发聩的教训,一年后的今天当我再回忆这段经历的时候,依然感觉自己心在滴血,地上的血流正汩汨地流淌着\textasciitilde{}~ 它让我认识到新转专业的自己基础知识是多么的不足,也认识到自己那该死的骄傲。在浩瀚的Computer science领域里,一个人第一年里能够学到的知识是多么有限,个体的能力、作用在历史的沧河中又显得多么渺小。

\section{实习(26)}
\label{sec-5-29}

(回忆起来的时间、事件顺序可能稍有错乱。) 到接下来的周一,我的python项目的第一个overlap小任务一一从一个文件里画两个变量的图就早已完成,等待A帮我review了。大概是周一还是周二的中午,吃罢中饭(走罢那两三圈的路大概),A就帮我扫了两眼那个程序,一个学生的第一份python code能好到哪里去,其乱无比倒是可以想像。A就稍微帮我点评了一下乱在哪里,可以从哪些方面改进一下。比如,这份程序目前只能应用于这样一个小任务,怎样才能将这样一份程序变换成一个可以将来被再利用利用的module?我应该绝大多数时间还是比较清楚自己在做什么的吧,所以明白他的建议后,便轻描淡写、去留无意地说,就是把现有的那个程序改成为一个module嘛,然后再将两个变量的名字以string的形式传进去就可以了啊,好说,”It should not be that difficult.  I think I should be able to finish it within about one hour !” 说出那样的话并不是因为自己还在骄傲,而是有着好不容易、费了这么大的周折、费了九牛二虎之力,终于A也成为了我的mentor的欣喜若狂,哈哈\textasciitilde{}~,多想,能有机会从他那里多学一点儿!我步伐轻快,追着,跑着\textasciitilde{}~

说那话时是中午两点钟左右。我想着尽快把这点儿活儿给干完了结,好让A再帮我布置新任务、新项目,但显然,我严重低估了它的难度,改的过程中出现了各种各样的问题。到三四点钟的时候,平时还比较有毅力的自己就有点儿沉不住气了,跑到lab里去找A,当时,B对面的senior也在lab里,” I cannot do it myself!  It’s beyond my ability. ” 我向A抱怨着,或许真的是给做烦了,senior抬头看了眼我,” You cannot do it, can you ?” A稍微严肃地看着我说,语调升起来,语气略带节制,说等他忙完帮我去看看,便再也没有了多余的话,去看他的屏幕忙他的事了。我凑在他旁边呆了会儿,终于无趣地跑回去继续自己苦难的历程了。就好像说,遇到问题,整了几下,感受到压力,我会先喊“偶先疯了\textasciitilde{}~\textasciitilde{}~”,但喊完疯完减压之后,终究还将继续前行\textasciitilde{}~

\section{实习(27)}
\label{sec-5-30}

那天傍晚A下班走得稍晚,平常他五点多钟就可以下班,那天快到六点钟他必须该下班的时候,他来到了我的cube问我之前问他的问题有没有什么新的想法和进展。不知道那个时候的senior走了没有,B还没有走(刚才还听见她说话来着),D一般都在,表姐应该也还在的。因为我们那个site实际上晚上是有简便晚餐的,有不少员工会因为晚上还有活儿想再多干一会儿也是常有的事。

A来问了我,我便抑制不住解决了问题的兴奋,开开心心地告诉他,我所有的问题都解决了,module改好了,该传的变量也传进去了,该画的图也都画出来了。我告诉他,” Once I point my test file to the right place, it works!  ” 他看了一眼我那一行point to the right place的代码,便也被我的快乐感染得笑了,说我的问题解决了就好,他今天有事要早回家,明天再帮我review。

那天下午后来,虽然我回到座位后不久、早就把自己的问题解决了,但我心里还是很感激A替我考虑、给我足够多的时间让我试着自己去解决问题。A是非常优秀的,除了语言好、文案工作做得好,编程方面硬实力也特别强大。当这样一个mentor对我说话略有严肃和节制的时候,我能意识到自己的不争气,没有人愿意教不努力的学生,也没有人想做孺子不可教的学生,何况我遇到的是高人大神。那是我第一次也是最后一次对A说 ” I cannot 。。。” \verb,~ 印象里E开始时有走到过A的cube里问他某个什么步骤应该如何做,不知道问的是不是我今天这个问题,但是我明显地感觉到,这合作之初的第一步小小考验坚实地奠定了我们合作时信任和相互理解的基础。 后来也发生了很多事情,我的脾气也不太好,但我们之间的信任和理解从来都不曾真正被打破过~, :)

\section{实习(28)}
\label{sec-5-31}

过了这么些天,我之前心施神往想A mentor我时A的拒绝,加上昨天A也帮稍微点评了我的程序了,到这时大家也就大致地猜测到我真的换mentor换成A了。office里那么多懂得察言观色的人,加上B一贯的表现和前段时间刚发生的mentor自己也不会的舆论压力,虽然site里manager、各大头目竭尽全力地帮她平息那件事情的后续影响,但毕竟那次影响也太大了,现在自己又果真换了mentor,B这几天的日子大概并不好过吧。于是第二天A帮 review我的项目就成了我当时想像之外的另一翻模样了。

我的时间不多,我只有三四个周的时间学校就要开学了,我多想回学校之前能够多做几个项目,真正从A那里积累一点儿工作经验。但A总是很慢,这不,一个等待review的小任务一等就又等到周三周四了。A批评我说,你不应该总是想着、急着把事情做完,而是应该享受做事的过程、在这个过程中真正学到知识、提高能力。如果你每一行程序都是由自己originally想出来,那要比你去参考网上的例子、抄别人的程序来得快的多、有效得多,你也能理解得更透、学到更多。

A帮忙给我review,原本该是开开心心的事情,但真正review过程中我却感觉A帮我review时总想找理由批评我。虽然我这人对待别人的批评天生敏感,感觉特别刺耳、难以忍受,但是对于A,我有太多的欣赏、羡慕和敬佩,我努力地去相信A对我说的每一句话都是对的、是对我好的,于是我便自愿分享了与他的一致意见说,是的,其实昨天如果我不参考你的程序,不把你程序里的 ” @staticmethod ” 也抄进我的module的话,我就不用多花很长时间来debug这一步了。A见我神回复、居然还有这么一段插曲,很开心地笑我问我说,现在明白static method是怎么回事了吧?我狂点头呀,是啊,花了好长时间才找到、明白这个是源头,再不明白,我可以去找南墙撞死算了啊\textasciitilde{}~

\section{实习(29)}
\label{sec-5-32}

那天下午A帮我review批评我之后,B是非常开心的!  A帮我review完后我就捡起书包、离开公司了,那天晚上我没在公司吃晚饭,跑到外面吃了越南面安慰自己,人群背后我也哭过。辗转再回到公司,尖人看见我了,A那天居然还没有走,当时应该已经是七八点钟、快 8 : 00 了吧,看我像霜打的茄子、A便把我领到会议室开导、安慰了我一会儿,才离开公司回家。开导是开导,安慰归安慰,可能是个性使然,A的那第一句话我当时听得似懂非懂,到我真正做出这样的改变、意识到自己真正有了这样的改变已经是好几个月之后的事了。

第二天我依旧九点钟来到公司坐定;九点半钟,大家到得差不多了,尖人并不常与我们打招呼,但那天尖人的问候如期而至,” When does your school start ? ” 我猜想过他会敲我两下,准备好的答案还不会说? “It’s weeks away.  I enjoy working here! ” 是啊,从什么时候开始,工作开始变得有乐趣?从什么时候起,A的批评也开始变得有价值?这些体验,为什么这前一个半月的时间,我就没有真正体会过?B这一天蓝天白云般、清清白白地非常开心。对她,我真的开始茫然不知所措、无语了。

检讨自己,我可能真的是在公司里太安静了,所以赋予了自己不该有的吸引力, 这些是没必要的。于是,那天快是午,当A到我对面E的座位去问他一个什么想法的时候,我不再去听他们在讲什么,而是就坐在自己的座位上打电话问自己那次04/08/2013 看医生账单是怎么回事,收费收了一次又一次,把人都搅晕了。那天中午,A又一次没有在公司吃午饭,我就奇怪了。当人们开始怀疑我俩因为昨天的批评心生嫌隙的时候,我已经开始满世界地寻找这个人了,开什么玩笑,怎么可能 ?\textasciitilde{}~

\section{实习(30)}
\label{sec-5-33}

当然那个 “point to the right place”只是A考验我的前奏,后来用了正规建module的方法(在文件最开始的地方from … import …)。那个overlap的从一个文件传两个变量plot的小任务完成之后,A开始练习我读code。先前他的项目进展会议上他demo他的test automation framework的时候,我问过他我们MSTK test suite automation的一个问题,现在,他留了MSTK一小块给我做,就是那Visual Studio C++ run MSTK会生成log文件,但在他的framework里,这一块因为某些原因他并没有整合进来,所以我需要读他的code,在必要的地方作修改,把log文件插入进来。

E从实习第一天起就跟着A,这时已经check in一两个相对比较大的项目了;而我,学习python这门语言才刚刚开始。Office里一个有点儿bully的老美manager那天下午甚至时不时地清理喉咙发出声响 ( 第二天有了环境反馈他才没再那样 ) 。那天接到任务后我就开始读code了,大部分的code很快基本就懂了,但在A run MSTK test suite的结果的最后有一行小字引起了我的兴趣,大意是说,detailed log file can be find somewhere(接的是像C:Users\ldots{}这样的路径)。我就想这个东西一定在A的framework的somewhere写着呢,可写在哪里呢,那天吃过中饭,我就去翻他的code。二三十个必要的文件都读过了,但踏遍铁鞋无觅处。两个小时后,我开始猜测应该不是我找不到,而是他的code里可能压根就没有,正好A从lab走到大厅里,我便上前问了他一下,他说,” There is nowhere I have written that sentence in my framework ! ” 我可是硬生生花了两个小时高效运转去找这一行程序,很冤枉啊,” I want my two hours back ! ” 感叹着自己浪费过的时间,可我还是想多了,那manager的声音还在继续,A会不会是安慰我才这么说的呢?于是乎,我又跑去把他的程序用了大半个小时左右的时间又翻了一遍,才去想自己新项目的思路。第二天同样的问题在lab里找到他又问了一遍,这时A快要起火了,我向他正式道歉,彻底打消了这个念头,可我心里就是不服啊\textasciitilde{}~ 带着自己的唯一的疑问,所有其它的思路我都想好想清楚了,幸福开敲\textasciitilde{}~ 

\section{实习(31)}
\label{sec-5-34}

那个夏天,严格来讲,从我开始锻炼身体以来,我早上来公司一小时之内一般不敲键盘。很长一段时间以来,人们一直以为我人来到公司、实际上早上没睡醒呢,还需要在office里懵一阵子。直到某天B早上来到公司后,找我说一起去喝咖啡,我告诉她我早上脑袋很清醒的、不需要啊,我只有下午和晚上的时候才会需要它们帮忙,早上的黄金时间我一般都在读文档呀。

那个周四傍晚五六点钟快下班了,我问了A一个整个夏天最弱智的问题,我的某个单个module文件我写完了,独立测试也通过了,可不知道为什么当我把它同我其它的module串起来之后,再去测试就过不了。过了十分钟等A把手头的工作忙我,帮我看一眼,他说他相信我能自己fix。这种问题我一定是相信A的,他说我能fix,那我就两分钟之内把它fix掉了,因为我把某路径字符串写了两遍。后来到真正写完这个项目后我才最终明白,先前那一行代码完全可以是从MSTK自带过来的,与A的framework无关。而我却受环境影响想多了,走了更多的弯路。

那个项目我也是从周四快中饭时间(大概 11 : 00 左右吧)开始敲,到周五下午写完 11 个module,轻松写意没压力。再写一个小的test文件一个小时内测试完毕,周五下午五点刚过一点儿,我告诉A我的初级独立版本的,我已经测试完了,他说下个周再帮我review。

\section{实习(32)}
\label{sec-5-35}

其实虽然换了mentor,实习生也会面临同样的问题。B让我陪她走了很多次T台,A没有这样的过分举动,但对A的项目和指导,我的任务总是完成得更快,对A总是等得迫不及待,就像是个催命鬼,催促着A帮我review,催促着他帮我想新的项目。如果说一周五天一个小项目,我大概需要想半天左右,code一两天,而剩下的就是等待review和review后的修改了提交了。A也很忙,整个site的test suite有事情都找他,他还要带两个实习生,他还有自己每天的工作要做。学会等待、善待等待,也是这个实习的一门重要功课。

后来在A的帮助下我refractory了module之间的联接,把module文件数压缩到 8 个。当我把这些文件check in到subversion,我的标题清清楚楚地写着,补充了 ” STEL Automation Framework ” MSTK test suite log section,后来那天,我已经能够从周围人身上感觉到我们师徙的工作都做得好、相互成就的局面,那些眼看着我落后一个半月、想看我笑话的同事大概难免要失望了吧。

那个项目唯一的遗憾是我补充进去的log同A的test suite系统环境不match,就是说他的framework环境是每行有时间等作行头,我有能力想去生成与他的一样的行头,但A拒绝了,他说我不需要,我心里感恋能从A那里真正学到知识、每天都做得开开心心、感觉充实,我也就很知足了,不需要就不需要吧。


\section{实习(33)}
\label{sec-5-36}

是的,我还是很着急的。每次我的新项目一等我作出自己的第一个版本来,我就禁不住去催A,我这个初级版本都已经出来了,就只等你review了,你有时间就赶快再帮我想想下一个项目吧! 

等上一个项目review完、check in,我的下一个项目就拿到手了。Gradually,补充一个log版块后,我的读写code的能力都增强了,那我接下来就该自己写一个test suite插入到A的framework里去,test suite是VDBench,A告诉我,这个软件表姐用过(要我有困难可以去找表姐帮忙),需要考虑Windows和Linux两种操作系统。

其实我读code的能力原本就不差的,从最初网上搜读Lisp Tic-Tac-Toe 4-in-a-line, 3-in-a-line, 2-in-a-line;到第二学期RTOS四位密码,网上code还没读懂,作业就已经写出来了;到AI Decision Tree第一次使用vector container存储切分的training/testing数据;到一周前的翻遍二三十个相关文件去找一行code,读code对我来说从来都不是问题呀;顺藤摸瓜,找到A的framework里test suite的工作原理、架构、具体suite的driver (app)名称、路径、command命令的传入读入、两个系统的设置,etc ,这些都能从A的framework 上百份code里很容易地找到读到的,这种分析能力对我来说,也从来都不是问题呀! 我的问题是,转专业的学生安装不好VDBench,两个系统windows下居然java环境配置不对,VDBench同学在我的台式机上压根就不work ! 连这最基础的一步都玩不转,要肿么去写test suite ?

当然,有问题就有解决问题的办法。那天傍晚晚饭前一个小时,那个鬼玩意儿装了5遍,删了5遍,最终在一位华人同事的帮助下早早地将这个问题解决了。


\section{实习(34)}
\label{sec-5-37}

忘了说,那天周三,我过生日。我们小伙伴一起吃中午的时候,我打了个头,问他们有没有想到过自杀。我先坦诚高三高考之前我第一次仔细地想过这个问题,虽然那时候我舍不得就那么着了结了自己的性命,哪怕是流浪出走作叫化子也好,我也还是想再多看一看这人世间的美好啊\textasciitilde{}~ 我问完,有人说没想过,有说试过,D接了一句,” Did you want to kill yourself when they assigned A to be your mentor ?” 这个过于离奇、脑筋急转弯似的回答弄得我差点儿笑岔了气\textasciitilde{}~ 坐我旁边的A也正从他专注的电视节目里发现了笑点,开开心心地笑了。D的那一记幽默成了我生日那天最开心的事。

因为刚过去的周一也正好是D的小伙伴的生日,他人非常聪明,又有领导精神、大师风范,时不时地还幽默一把,人缘极好。于是周五就有人帮他呦喝着,我们一大帮人就早早地跑出去庆祝生日了。当然也能感觉到这有一点儿公司领导风向所致的意思,因为firmware的女华人manager、其它一两个manager以及一些原本不太熟悉的人也都来了,可能是给D小伙伴面子吧。他们选定的餐馆很有一番希腊风情,到了之后大家热热闹闹地聊天,我第一次发现,工作之外的A也真会说,同一桌上几乎所有的人都喜欢听他说话! 只有E用他土生土长的混血儿文化从小说、电影等文艺作品方面稍微“压制、抹杀、平衡”着A的光茫。

后来坐我对面的A讲起他的强迫症怪僻。他可能没有geek到来office必须把钱包放在固定的位置、车钥匙放在固定的位置、手机放在固定的位置、笔记本放在固定的位置,但他却对自己电脑屏幕的清洁程度极为敏感和苛刻(这点儿和我学校里的男闺密像有点儿像),我亲眼见到过A用昂贵的专用monitor湿纸巾来反复擦拭他的屏幕!然后A就讲到说对我几乎不能容忍,因为我在他cube里问问题的时候,我的手指总是会不小心触碰到他的屏幕。见他今天当着这么多人的面提出来,我穷凶极恶地开玩笑说,哈哈,这下可抓住你的弱点了,以后我可是要先把自己的手指头都染上颜色,然后再去向你请教问题,然后再“一不小心”碰到你的屏幕了!

A的魅力实在是太强大了,那天后来结账,大家帮把D的正式员工小伙伴的账单给出了,把我这个实习生的账单也给出了?!!\textasciitilde{}~


\section{实习(35)}
\label{sec-5-38}

周五的午餐对于那些个manager来说显然不太厚道公平呀。正如我周五猜到的,周一Site里就有官衔不小的manager站出来吆喝,于是紧接着的周一,我们又出去吃了一次,这次是大老板带上那几个manager,为D的小伙伴又过了一次生日。明知道周一是去过生日的,E当然不会去,而我则坚持一定去!

那天中午去吃火锅,A因为生病周一没来上班,D、他的小伙伴和我因为晚到同一位senior坐在一个小桌,其它人来得略早些,在大老板的带领下坐了相当于我们两个桌子长的大长桌。虽然A没能来略有遗憾,但能有机会同D、他的极其聪明的小伙伴坐在一起吃饭、近距离聊天也是很惬意的事。D的小伙伴问我,同新mentor关系处得怎么样呀,我笑啊,” We tried so hard to confuse each other. ” 想到我们合作之初的种种误会,他们也会心地笑了。

插一句,我们小伙伴们平常周五出去打捞食物时,好多次E因为没带现金都请D帮他先出,于是上个周五,E就帮把D的份子钱出了。不知道的人会觉得E人还不错,帮D出了这么大一份,而我却真心感慨D人品真好,在D为E的那么多次帮助里,E想必会欠D更多的钱,因为我们周五出去吃东西,E个儿高点儿(不过没有A高),从来都是E点得最多(我们的账单是平均分的)。上周五那天,同样人高马大还生得壮的D就只点了我们 12 个人中最普通、经济的食物,而E是唯一一个点了两份的人;清楚地知道这些,周一我们又同坐一桌,我便向D学习,那天我们两个都点的是最普通、经济实惠的lunch special, $10.98,他是男生,我后来还留了盘子里一大片untouched羊肉给他。那天正好也是我们桌上senior的生日,减去2份,最终我们人均滩了$ 22 。真正做到了礼尚往来,我是非常开心的\textasciitilde{}~


\section{实习(36)}
\label{sec-5-39}

上周五庆祝完自己的生日,过完下个周学校就要开学了。我是希望开学后能够多呆一个周的,但如果学校和公司任何一方谈不下来,我接下来的周末就应该会开车返回学校了。于是庆祝完生日的这个周末,我就理短了头发、采购菌类干粮、(因为多年来慢性咽炎的底子)还特意地买了一瓶川贝枇杷止咳糖浆,来美后发病同国内比少了很多,但发过几次后也已经喝完了一瓶,要为返校作好充足的准备。

那时我们中午走路,小伙伴们谈论最多的就是奖金问题了,民以食为天,这就是大家的食物啊。而且据说奖金的比例很高,大家也都显得格外兴奋。理论上最高可得年薪25\%的份额作为奖金,他们说拿到手、最高的有高达正常工资24\%的,我猜想应该会非常多。A和D因为刚来公司不久,这一次工作年限不够还拿不到,但想拿的决心和热情都很高;D的小伙伴因为极其聪明,所以拿了很不错的奖金;D说他问过B,B的反应很冷淡,感觉她拿到的奖金不会高。D说表姐会拿很不错的奖金,因为表姐是重要项目的核心人物、攻艰主力、也是大神般的后卫。一段时间以来A从方方面面帮助我,我们小伙伴一起走路的时候,他也有意识地帮我纠正发音,倒是有时候我自己害羞、很难为情,该练习的发音说不出来。

那时一两个星期里,正式员工的表姐和B正忙着他们业界的一个什么大型测试宣传发布会一样的考核,大概到我们生日那周周四傍晚才算实习公司里的所有的测试都通过。最开始同表姐聊天可能聊到一些关于B的负面信息,很长一段时间以来,表姐和我都不再聊天以保证正面导向,或许A也是希望我同表姐多联系的吧。先前可能是周三下午,我有见表姐从会场返回公司,便跟着她,确实想问表姐两句话,哪怕只是传达A的意向也好。也看得出来,表姐一回来就在lab里忙,大概需要copy一份firmware on call更新的new version。我叫了表姐一句,她说现在忙,要我周四下午再同她联系。


\section{实习(37)}
\label{sec-5-40}

有次尖人正从我的座位边走向lab,那天我大概正在大厅里问完A那个我转不过弯来的一行代码(MSTK log file location)问题,见到尖人,A还没走远,我居然也没能禁住向尖人报怨说,” My manager does not give me enough projects for me to do. ” 我报怨完了,我high了,留下A向尖人解释了些什么、怎么解释的,我早就不care跑掉去干自己的活儿了。

表姐早前已经说了那样的话,于是周四下午,我打电话过去的时候,表姐说她终于忙完,在家休息,要我上表姐家去找她。于是我五点多钟一放下电话就立即出发,到了后表姐把晚饭已经做好了,还买了卤鸭子,小表姐也正好同我先后脚到家,我们三姊妹先吃饭。

小表姐问表姐,为什么大表姐夫居然直接打电话给她,电话也还直接打到国内去找舅母了;表姐说他求表姐一个同学帮办事;大表姐感慨,“这么多年来,这个人也终于学会求人了?”我们已经在吃饭,先前三姐妹共聚一桌的热闹在我这里就变成了悄无声息地啃鸭子\textasciitilde{}~ 后来,我对表姐说A让我有什么问题就找她。小表姐问我同大家的关系都处得怎么样呀,大表姐接话说,“这不,又给打发到我这里来了! ”表姐的话更让我挫败,便赶紧纠正她说,“我们三个性格比较合得来,兔一一猪一一羊,魔羯一一白羊一一狮子,A的STEL Automation Framework第一个test suite是表姐你的NuMV,我的最后一个项目如果能把大家联结起来,也算是对缘份的一种表达\textasciitilde{}~ ”见我解释的有理,表姐也讲了她的理由,她说她应该是这个公司里the last person I should seek for help。表姐说如果她帮我了,别人会说是她表姐帮她解决问题的,而我如果向任何其它的人寻求帮助,就不会有那种疑问,表姐的话我听得直点头。


\section{实习(38)}
\label{sec-5-41}

其实上周五庆祝生日的下午下班之前,我就同C讲了,A是非常优秀的mentor,而我跟着A到下周结束前我也才只能做完两个小小的任务,感觉刚刚才热身就不得不离去,也挺遗憾的。我就问A,她能不能帮我想一下,帮我看看我有没有可能再多工作一个周,跟着A再多作一个项目。

那天在表姐家吃完饭,我们便到附近的公园去散步聊天。表姐同我是有段距离的,但我心里却也是认可表姐的,经过多年的钻研和积累,这个人的内心是有着执着、坚持和抱负的,只是自己经过艰苦卓绝的努力才得到的东西,便多少有点儿难以认可他人。我听得进表姐的苦口良言,就再也不曾问除( except )A以外任何人问题了。而我同时也劝表姐说,D同舅舅、舅母、我的同学都很像,极善观察,我问暑假最笨的问题时D也过来问我问题安慰我,D是组里、甚至整个site里同任何人关系都极好的人,你们还就坐在隔壁,怎么就不见你们说几句话?

我给表姐讲,那次B mentor自己不会后,对B影响非常不好。除了firmware的manager主动给她面子外,site里的大官们便打出了另一套方案,就是人为设置difficult bugs,然后分配新员工D去啃硬骨头。D因为初来乍到,怎么不懂不会都情有可原,加上人缘好,向任何人请教问题大家都心生欢喜,借以提倡培养一种team间相互帮助的culture,也好帮助平衡掉B样样不会、处处求人的尴尬局面。

我也问了表姐我的疑问,他们有项目在外面开会测试时,明明是B的MSTK相关的测试出了问题,为什么回来office里copy取更新的firmware的人却是表姐,而不是B?我问表姐,B的MSTK项目出问题了还经常要表姐你去帮她收场,你会觉得委屈么?表姐答说,“不委屈,老板让干什么就干什么! 老板不说的时候才会讲人情去帮她。”听完表姐的话,对她这种大神般的后卫更加敬佩有加。


\section{实习(39)}
\label{sec-5-42}

周一我们再去庆祝生日那天,A生病没来。但作为A的小“跟屁虫”,我一直都有着相当的独立性。我的  code总是写得很快,周一过半,我的windows系统的sweet程序就已经写完,需要测试一下就可以去补充Linux版本了。既然作test的A今天没来,那我应该可以用lab里的机器来测试吧。在lab里转了一圈,同B对面的senior更熟一点儿,之前问过他一个小问题。 Senior说他还不知道有没有机器可以给我用,但他会帮我确认一下。

那天下午后来,office里, site里好几个人都有high priority tests,一两个小时后,senior来我座位上告诉我,今天大家都比较忙,lab里没有我可以用来测试的机器。没有就没有吧,其实等到senior来告诉我答案之前,我的framework已经在自己的台式机上测试通过了。我的Linux版本最初不太懂得driver到底怎么工作的,但就像Windows版本我会先将软件装在自己机器上,跑几次看看工作的效果是什么样的,借着run软件的方式去理解它的工作原理,Linux下我自然会作同样的尝试,而它的bash文件也不难找到,所以Linux系统并没为自己制造任何的麻烦。

因为lab里总是很忙,到要测试时,我没有机器可测试。我自已的台式机安装有Linux系统,但不是dual boot,我是没办法共享Windows下的文件的。后来周二的傍晚,A允许我用他cube里的Linux Server进行测试。E同我,我们两个实习生都用的是小屏幕。当server的屏幕比client的屏幕还大的时候鼠标拖来拖去就很麻烦,于是那天A下班走之前,我问他,“我可以坐你cube里用你的Linux station么?”A是team里的难得的利他主义者,他当然会同意,我也很开心,其它人就有人禁不住去回想B同我的关系了。


\section{实习(40)}
\label{sec-5-43}

记得第一次测MSTK log版块的时候,A可能也正大量debugging或是测试他自己的test suites,lab里非常的忙,忙到A在几个station上移动、轮流看守。为了能快速有效地完成自己的任务,我先在自己的台式机上先都测好。然后等到有天下午A说某个station我大测可以用一个小时的时间。station是A安排的,没有选择;测试的时候,我原本打算从A几十条现在的command里随机抽查一两个命令(测试某些specific版块的test cases,比如log page related, SSD temperature related, etc)测试看能不能通过,但因为我原本就只需要检查MSTK的log能否在STEL Automation Framework里正确运调,所以,最终A帮我选择了一条执行时间最短的命令,run十分钟测完后,我告诉A所有测试cases,一切都顺利通过! 来我所在的station瞟了一眼,见我测得这么顺利,A也合不拢嘴巴。那天下午我测试时正好B对面的senior也全程都在实验室里,他是我summer实习自始自终的见证者,我还是有那么点儿不一样的。后来,二十分钟后,我就将station还给A去忙了,自己撤回office里。

就像之前提到过的,A也是很忙的,所以我一般不去打搅他,我的问题原本也不多,便利用自己地理位置之便,总是在他从lab里出来的时时机去问他,以免其它时间找他可能会打扰他工作。周四快中午 11 : 00 左右当他早上从lab里忙完出来,我便问了他一个关于 VDBench 我新发现的小问题,就是虽然我的 test suite 得到从framework里来的“PASS”,但从 sweet framework 里得到的结果与直接run软件得到的结果还是稍有不同,我把不同的地方在自己位置里指给A看,并对他讲,“ It seems I cannot and shouldn’t trust this ‘PASS’ yet! ”虽然过去同A,当我完成几乎任何任务,A都会同我一起感觉很开心,但这次、这会儿A难得地说了一句不一样的话,“ Most probably, you know this software much more than I do ! ”这该是在这个夏天、专业领域里,我听到过的最好听的话了! 为听这句话,我等了好久,瞬间接一句“Thank you \verb,~ ”脱口而出、情真意切、志得意满~,


\section{实习(41)}
\label{sec-5-44}

那个周一,庆祝生日,PALO ALTO吃完火锅,D,他的小伙伴和我,饭后我们三个小伙伴在风景风情都还很不错的主街上散步闲适。小伙伴发感概说,"D很帅,A很聪明!"。我很不识相地问,"那我呢?" 

"混血儿很聪明的!!!" D不明觉历地接起话来。

"那又怎样?"回想起来前段时间,因为学校要开学了,我同那个中国女孩的软件工程课,老师要我们写一个网上预约系统(预约见professor)的项目还没有完成。对于我们两个转专业那时又都没有多少JAVA编程经验的人来说,这个项目对我们有些偏难。因为这些D都学过,所有需要的语言和设计思想他都非常清楚。但大概因为我问过他这些问题,他可能会觉得我的基础有比较弱吧。

"我能把自己的工作、项目做好,遇到困难时能自己去解决问题,能让自己每天生活得开开心心,我就挺知足的了。"我补充得云淡风轻、波澜不惊!

是的,这也是我一贯的观点。别人是高富帅、白富美;别人是亿万富翁,别人是伟大的文学家、思想家、诗人、作家、艺术家,别人是QQ群的群主,别人是天才\textasciitilde{}~ 如果这些人与自己的生活没有任何交集,那他们又与自己有什么关系?我又不能过两辈子,我依然只有短暂的生命,只有一个属于自己的有限人生。

"我们学校(或者至少我们系我们专业吧)正常本科生平均毕业时间4.5年,硕士研究生平均毕业时间2.5\textasciitilde{}3年!我几乎所有的课都拿B,因为我入学的时候年龄已经很大了,系里老师能允许我两年毕业我就挺开心的了。得B就得B,GPA低一点儿就低一点儿吧!我虽然也很不情愿,但还是可以理解他们的做法的。"


\section{实习(42)}
\label{sec-5-45}

周四中午 12 : 05 ,吃饭前我告诉A我的任务已经完成了,他答应下午下班前帮我review。这个项目期间也看见一个关于error和warning message数量的小bug。A一定是事情太多、不屑于花时间才留下这么个小东西,只是几行代码的小问题,后来便同自己的更新一起交了上去。

早前就已经同A讨论过了,这个项目的目的是读懂他的frame,并在读懂的基础上插入一个新的test suite。至于我有多少条命令,哪怕我只有一个命令,哪怕我只会弹一首钢琴曲,只要我能把这个test sweet 写出来、run出来,那我只嵌入一条命令,又有什么关系?所以下午的REVIEW,一如他带我后的第一个REVIEW,还是能够清楚地感觉到他想"找我的茬"的。但时间转眼又过去两三个周,我已经要到自己实习的结尾了,此时再不为自己争取CREDIT,我又还有什么机会呢?还是我这次留下,生来就活该被埋没、永远没有出头之日?空气中弥漫着一股夏日正午一两点钟的骄阳炙烤着沥青马路的焦灼味道。

所以那天下午三四、四五点钟,当A真正帮我review的时候,我们就直接起火、吵起来了。就好像说,寒风呼啸的化冰天气的早上,一小人在床上打滚是无论如何也舍不得起床的。A就像一个拿了暖宝宝一一热水带给蜷缩地墙脚的叫化子取暖的好心人,可以取暖,别人当然心存感激;然而叫化子刚把暖宝宝拿到手,还没来得及好好体会一下它的温热,A却又瞬间把它抢了回去!换作别人、换作是你,你就真能舍得松手么?更何况原本脾气就不太好的自己了,这是逗人玩儿、好玩么?


\section{实习(43)}
\label{sec-5-46}

吵架归吵架,我们居然还吵出了几个问题来。

当我提到关于他TEST SUITE的一个什么问题,我感觉自己发现的问题(属于自己的CREDIT)已经被他CHECKIN的时候,我们就认认真真地炒出了细节。我的观点很明确,如果这是他的功劳,我绝不抢占;但如果不是他的,但因为他是我的MENTOR我也绝不会计较;A后来帮我指出,那是因为我用SUBVERSION还不太熟,他没有CHECKIN我的修改,我的功劳还是属于自己的。

其实那天我也不记得怎么吵得了,也居然提到了E。A说," If you can find HIS problem, you can help fix it. "我也很固执呀,"If that's his project, I don't want to step into somebody else's staff,"我接着强调,"I don't want to!"

那天要结束的时候,我想也经吵了这么半天,项目也没挑出什么毛病来,应该还是差不多的吧。便在A临走前问他,是否已经帮我想好一个新的项目。A说," I won't give you the new project until tomorrow. "那句话就成了那天警醒自己最深的话了!" Then what will happen tonight? "A不再说话,却以一种沉默的力量,迫使我去反醒自己。

那今晚,就让大家在风中凌乱吧\textasciitilde{}~:


\section{实习(44)}
\label{sec-5-47}

对于A刚刚不给我项目的原因,我还是傻傻想不清楚。以为自己还有什么做得不好的地方,便在自己的坐位上认真地检查可能存在的bug了,心想着等到第二天大家气都消了,也许A还是会再帮我看一下的吧。

我的计划是如此,然而变化也来得很突然。六点钟左右,离吃晚饭还有大半个小时呢,就听见那边B叫坐她正对面(同一CUBE对面,SENIOR坐她同一走道对面)的表姐说她CHECKIN了一个什么项目,要表姐帮她看一下。B说出来的话是谦虚的,说话的语气、态度却是傲慢的。表姐也就问了一下CHECKIN了没,便没再多说任何话。B这是耀武扬威、报复表姐、报复我们那天她的报告表姐没去吗?我瞬间登录到SUBVERSION账户,B果然是交了一个项目,同E的第一个项目类似,因为是app软件相关,所以交了可能有上百份文件吧。

回想起来,骄傲真是人类最大的敌人,而每个人最大的敌人却又是他自已。如果那些跟着B混的日子里,我没有滋长自己的骄傲,也许我就不会犯那个简单的编译错误(虽然我还是很需要去理解这些概念和原理的)。

我自己的人生也有很多遗憾。如果那天早上妈妈没有把我往死里打,今天的自己不出意外应该是流浪汉吧;如果2012年的春夏我听得进一位姐姐的友好帮助,去别的洲直接工作;如果2012年夏天我能得到导师的直接表明否绝,没有本科基础的硕士想两年毕业是完全没有可能的,我直接回国;如果爸爸出意外的时候我就直接回去,或许我还可以见到爸爸最后一面;如果我工作的所有积储是用来给妈妈治病了。。。


\section{实习(45)}
\label{sec-5-48}

曾经,我对自己高考的经历不愿对任何人提起;但后来再长大一些,经历过一些事情后,对于自己走过那样的弯路心存感激,因为它锻炼迫使我从内心去发掘力量。如果重新来过,我希望它发生在初一初二或是高一,而不是高三。前段时间C在B座位里对她讲与同事共事的工作方式,在她这样一个senior的年龄,她还能改、还有改的希望么?

另一面,表姐是惠质兰心、心无杂念的。之前上周同表姐在公园走路聊天的时候,表姐就劝过我,说我应该多把心思放在自己的学习上,别每天心里想那些乱七八糟的事情。我安慰表姐说,对于你来说,想这些事情会很烦;对于我来说,这些是直觉,不需要我花时间去想,我只要去感受一下就可以了。表姐说我不把自己的成绩弄好,想这些对自己有什么帮助,我反驳她说,我也就是自己理解了,也算是学习、见识一下眉高眼低,帮助自己成为一个更好的自己而已,我相信自己的明天会更好\textasciitilde{}~

有时候人太笨了,也实在是不服不行。明明这边下午就在吵架,就差要CHECKIN项目了,她还早早地把她自己的东西交上去,还表现出这样一个态度!表姐那段时间在忙一个攻艰项目,应该暂没有需要上交的任务,我的项目下午同A吵也吵过了,应该不会有太大问题。既然B已然这样了,site里官衔最大的manage也早就安慰过我,实习生不要害怕犯错误,那我就算是冒天下之大不韪,我也一定把自己的项目及时交上去。于是一个小时后,我的项目便也进subversion了,同样是关于一个软件(VDBench),连带添加和修改A的framework加入这样一个suite,这次我也同样地交了一百多个文件,一样也不比你少!\textasciitilde{}~


\section{实习(46)}
\label{sec-5-49}

第二天(周五)早上来到公司,第一件事便是上去看subversion了。对于昨天晚上的风吹草动,是迷途知返么,B真的是风中凌乱了。她把自己昨天交上去的项目,在昨晚我们走后,modified了再删,删了之后再重新添加,改了一个轮回;把MSTK连带我checkin进去的程序全部删除,再把她自己也不知道作了什么修改的、修改之后的程序添加了上去,subversion那天晚上简直是被她一个人给搞得乌烟瘴气!所以这天上午,就能够真正感受到善良的人们对B的同情,觉得她这样一个senior,怎么就被一个实习生给折腾得七晕八素、没了主意和方向?\textasciitilde{}~

而我在经历了A这一晚的隔离,不像往常那样review完就给我新项目之后,第二天A来office里,我再去要新项目时,我对A的态度就近乎要讨好起来,先乖乖地告诉他,我昨天晚上已经把项目交上去了,他提了一下到时可能还需要再改改;我就问他,还需要改什么,改.csv文件里VDBench的执行调用命令么(言下之意,那个不早就说过不是项目重点,一条命令就能满足现有需要的么)?他已经转换了话题(开始告诉我我的新项目了),倒是对我checkin项目没有任何责难;见这个test suite项目就这么风平浪静地要结束了,我就又问了一个项目里自己还有点儿迷糊的地方,为什么我从framework run VDBench出来的output文件夹在Windows下会出现在桌面上?他用半分钟时间确认了一遍我的问题,然后思考了一秒钟,给我的答案是:因为我是从desktop launch framework调用VDBench driver,所以output会default成这个路径。这样的常识当初的自己是没有的,但因为是常识,别人可以一秒内作答;再回想一下A mentor我后的第一个常识,我居然花了那么多个小时去翻代码(擦汗\textasciitilde{}),差点儿要同别人起火吵架,真是可笑。或许,I have learned it through a hard way \textasciitilde{} 同B(的bug,自己的idea行不通却压根就不知道)相比,什么叫黑白分明、天壤之别也该很清楚了吧!我欣赏的是这种头脑清醒、条理分明、有常识、懂原理、编程能力都很强大的mentor。


\section{实习(47)}
\label{sec-5-50}

那个周是我们学校开学前的最后一周,也是纠缠不清的。因为上一个周五的时候我已经同C打好招呼,希望她能帮我想想、同上级沟通一下看自己能否再多工作一个周,于是A周一不在、VDBench那个周和接下来的一周(实习的最后两周)就成了地震多发地带。 

我自己另一方面也同学校联系,学校那边我需要做的工作是,先同导师联系,取得系里同意,再取得学校IPO国际学生办公室里的同意,更新I-20;但因为新生开学,学校IPO很忙,所以我一直还没有拿到新的I-20。

这或许也是A周四傍晚不肯给我新项目的一个原因吧。但他还是信任我的,周五的早上我去找他,老实交待昨晚的罪行后,就真的给了我我的新项目:Windows系统下,用python 2.7.5的Tkinter module从可选择的地址路径中选择多个源文件,从源文件中选择两个变量,再plugin我之前overlap的程序,将比如说5个文件中的x, y变量的数据显示到一个统计图中。 题 目是mentor出的,思路都是自己的,这次的代码可是要自己一行一行地去敲的!可是A已经给了我新项目,我还是很开心的。 


\section{实习(48)}
\label{sec-5-51}

偏巧,C下周一也修假。C是一个blah-blah-talk-talk style的manager。她每周的一次的team会议,A同B的项目会议,偶尔早上来office,她都会有很多的talk。所有的talk里,我最欣赏她同A的谈话方式和内容,因为A很聪明,如果你仔细听,你可以学到他同老板的沟通方式,所谓指哪儿打哪儿,谈话的双方把对方的意思和企图揣摩得清清楚楚。她的talk这么多,我就会情不自禁地感觉到这个site这个manager领导下的team就有了某种程度上的怪异的夸张。她应该还是最欣赏A的,也很信任他。可那天下午当她们聊到C周一要出去玩的地方叫"Tahoe"时,A因为没听懂她的发音,重复了一遍,后来猜测到正确发音后,A下意识地重复了一遍这个地名。离A座位不远的我就能感觉到C的语气略变。

下午散步的时候我给小伙伴们讲,今天下午的时候C很怪,来到我  cube问完我必要的事情后说," I don't know if I can still see you next week. "我问她," Why can't you see me next week? "C不说话。 我对小伙伴们讲,我就搞不懂这个manager为什么会那么说话,搞得就好像我下周来不了了似的!有小伙伴宽慰manager说,她可能自己的什么事情吧。我很不服气地坚持了一句,"她就是那么想的!"

我早就已经同C讲过了,我之所以还没有拿到新I-20是因为学校新生开学,IPO负责这项工作的人员这周四、周五两天正给新生做orientation,非常忙。再后来,周五下午04:48分当我打通了相关人员的电话,他说明天周六他就算占用自己的私人时间也会帮我把新 I - 20 给弄出来。我的手机放下了,办公室的氛围跟着也变了,从C所在的manager区传来的某些人的大笑声。 


\section{实习(49)}
\label{sec-5-52}

前面没能介绍,这个夏天,因为有着A、D以及refer D来这个公司的朋友等若干臭味相投的小伙伴在一起,我们每个星期五出去打捞食物还是有着诸多乐趣的。第一次出去吃饭是大家响应我的愿望,去吃了海鲜(很害羞地说偶应该是吃了三盘,或者至少两大盘吧,倒是A因为打电话、接电话就只吃了简简单单的面条!虽然我的身体状况其实并不允许和承受多吃海鲜)。那天吃完饭后D帮忙出brain teaser问题让大家答,比如如何快速称分出混在药品瓶中的毒品瓶,比如有两个量筒量如何量出一个特定体积,像是幼小时候的脑筋急转弯,吃饱了美食,大家猜答案的兴致可不低!后来我们还吃过好几家,吃过很好吃的越南面包,某个风和日丽的中午听着旁边游乐场的背景音乐、还在公园喂过松鼠!

印象很深的是那次去TK noodles吃面条,面条本身很好吃不在话下。那段时间A的framework进展神速,所以A的项目汇报会议就成了偶滴一大享受。可是呢,C大概觉得我在会议上太享受了,于是她取消了那个周所有的三个会议:一个每周项目进展汇报例会、一个A的项目报告,还有一个是谁的什么review吧。对于C的这种做法,我心里还是很不爽的。不过因为A的坚持,只有他的会议在老板的杀伐夺断下得以侥幸存活。于是那天吃饭,我大概提到我以为A还是谁的会议是安排在周四,原来是周五哦!聊天的细节大多我都不大记得了,但还是记得E起头说他将来不想买房子,他喜欢住hotel!聊起hotel,接下来的话题就又一次地被他们谁给扭转到bed bug上去了。还有印象接下来的周二中午吃饭,饭桌上D的小伙伴直接问A宗教里关于性的问题。

白云苍狗、白驹过隙。学校开学前的最后一个周五该是我最后一次还能有机会同小伙伴们聚在一起吃东西了,下个周五还不知道会是什么状况呢。因为A是这个假期里给予自己帮助和肯定都最多的人,A很喜欢吃韩国菜,那天我们便整个假期里唯一一次去吃了韩国(泡)菜。

要出去吃饭,一走出office,我便问了A一个问题:"Why didn't you give me the new project last night? "我的个性比较明朗,我还是很想知道他的真实想法的,然而A仍然不说话,我只好自己找台阶下,"while thanks for your help, I lost five pounds last night! "小伙伴们忍不住想笑话我(哈哈\textasciitilde{}~)。


\section{实习(50)}
\label{sec-5-53}

那天同A的吵架,我想我应该只是太想要、太喜欢听别人的表扬,所以A一正一反来得太意外(虽然第一个项目review已经经历过一次逆压锻炼),我还是来不及反应,所以没能控制住自己的火爆脾气。如果我真心要同A吵架,那我就一定该下足火力去吵,我就该一针见血地指出,换mentor后这个假期里,我连测试的机器都没有,即便是周一作测试的A自己都没能来office,也同样地,实验室那些台机器我连个手指头都没沾上!实验室的机器,换mentor后,我就只用过 20 分钟\textasciitilde{}~

但毕竟这样的实习机会并不常有,为了表达自己情绪失控的歉意,接下来的周一,我招呼E我们两个要去headquarter总部吃东西。之前有一次我同E在总部那边排队买午餐的时候,排我前面的一位外国同事就向我推荐过一种他们国家很传统面条,据说是逢年过节的时候都会吃很好吃,但我一直都还没机会去吃。我同E约好,我们今天中午要抓住我最后一次绝无仅有的机会去吃他们国家的面条(E学校开学晚,这样他的实习会比我晚结束)! 

同E的聊天我总感觉没有主题,因为他也聪明(英语还是他的母语),我并不是总能知道聊天的内容和形式怎样会比较好一点儿。那就聊自己上过的课吧,我把我们上过的好玩儿的课、做过的好玩儿的项目都给讲他了一遍,E也讲了些他上课学到的。关于项目,E说他不太喜欢编程和调试,尤其不喜欢连续编上 10 个、 20 个文件的程序后再调试。我的观点是,只要自己思路是清晰的,哪怕编完 20 个modules再调试,虽然看起来是混乱了点儿,其实因为自己写过这些程序,还是可以快速地根据log trace back到各个源程序的!我强调说我也会优先考虑他的调试方法,但我对后一种方法不会有排斥心理。再没有什么可聊的,那我们还有一个共同的mentor!说起A的强大,E显得不以为然,他同意A的编程能力还是比较强大滴\textasciitilde{}~  然后在E这里就没有然后了\textasciitilde{}~ 难道在美国学生的眼里,mentor只有方方面面都得强大到可以深聊、交心、有得聊才能算是不错 mentor吗?那在偶眼里可就已然是知已、要成 soul mate 了呀\textasciitilde{}~!!!


\section{实习(51)}
\label{sec-5-54}

一天中午,我随小伙伴们取好饭菜已经坐下来,E就排在我后面,应该很快就来了吧。我们坐在一张不大不小的圆桌,差不多能刚刚容下五六个人。D同A以及他的小伙伴仿佛还在说着早上的会议或是项目的事情。我刚坐定,伴随着E的出现,就感觉到我们小伙伴们的眼珠眼镜撒了一地一一E今天极其夸张地拿了特别多的食物,不光纸盘里有富得流"油"的一顶盘,连盛纸盘的长方形大托盘里也被他在一侧堆放了很多蔬菜,这实在是太逆天、太阳从西边出来了!对呀,偶不是动辙就会穷凶极恶一番的么?今天的自已甘拜下风,孤独求败的境界是属于E的!

一秒钟后我明白,原来据称是他family friend的舅舅从远方出差回来了,他的靠山回来了!

就像我们两个的座位被安排得面对面,我们俩个之间无形中竞争还是最为直接和激烈的了。我是表姐帮忙refer进来的,可整个暑假,除了最开始发给C的一封邮件,表姐批评我邮件里写那么长的句子谁有时间来读之外,我没问过表姐任何问题,甚至中间很长一段时间因为表姐的疏离我们连话都很少说!我自己都从来不曾有过靠山、依靠过别人呀!而且,作为一个三十多岁的成年人,我也不可能去欺负他呀?

好吧,我承认有一次,上一次这个site party时我对他是有些冷淡的。

那公司里的大概每月左右一次的party我也介绍一下吧,我所知道的,共有三次这样的活动。

我来后参加过的第一次party是刚来不久的冰淇淋,那时刚认识大家,都还不熟,我、E、D同A我们几个因为中午一块儿吃饭也站在了一起。E同A正基情四射地热聊着他们各自锻炼身体的历史和体验。在锻炼身体的道路上我一向后知后觉。曾经有一份真挚的友情摆在我面前,我大学时同宿舍一位来自广西柳州的女孩子 offer 我她愿意教我打篮球,但我没有珍惜。现在,当一份渗透着竟争的友情摆在自己眼前时,我不想再错过锻炼和认识自己的机会!于是我买了个五磅重的呼啦圈,从那天起,这个暑假,我开始有意识地同自己较劲\textasciitilde{}~  


\section{实习(52)}
\label{sec-5-55}

小伙伴们第一次出去吃海鲜那次是team里的锐意新秀A开的车,小伙伴们都准备了现金,打算把钱给A这样一个人付现结账就可以了,E说他需要一些cash,这样那餐是E帮用信用卡结的账;TK Noodles那次大家都觉得味道很不错,只有E话里带刺、报怨嫌不好吃;每次出去吃东西基本从来都是E点得最多,也向D借过很多次钱。我从小并没有生活成长在海边,看得多了就对E慢慢形成了成见。小伙伴圈里,同一个team里的人,我同A和D的关系,那是极好极好滴\textasciitilde{}~,而工作了多年、A的成熟睿智以及事业有成所流露出的自然风采,都让自己深为叹服,希望明天的自己能够成为他今天这个样子,这是此时的D和我们实习生无论如何都达不到的。 

A同我的友好却会引起别人的注意和不满!华人同事放大了A帮助我的时间不再冗数;E的添油加醋、煽风点火亦不需多说;同E相处得多了,我总以为自己眼睛很毒,可我毒不过尖人。尖人给过我hard time,也同样为难过正式员工A。某天下午五六点钟,A已然准备好该回家了,尖人却很不识相地捡了个好时间问别人工作上的问题。A就直接说了那天他有事,等第二天早上上班后再帮尖人看一下到底是什么问题!第二天早上A就到得很早去帮他;刚跟着A作项目的某天,我去lab里找A问项目, lab 里我可能呆了五到十分钟的样子吧,在我从 lab 往外走返回 office 的路上就见C风风火火地往 lab 里赶,我不免狐疑,C难道会担心我俩打架?还有一次好玩的提一下就是,某天我准备去洗手间,在门口听见一位女性长辈年龄段的人同她的同事站在洗手间门口说,会考虑在lab里装camera,大概觉得很好玩儿吧,我云里雾里、不明所以的一阵兴奋!\textasciitilde{}~


\section{实习(53)}
\label{sec-5-56}

第二次party的时候,也是半下午,我大概刚从洗手间回来,从走道走过的时候,座位里就瞬间没了人影,已经拿了一大瓶水和一块大面包、坐到了自己座位上的E告诉我,他们在餐厅喝咖啡。当我去餐厅找到了他们,A、D以及D的小伙伴都在,我加入了他们成为四人方桌会议。后来D的小伙伴赶项目、较早地离开了,换来了E的意外再次前来。

这次的E不为任何食物、专为请教A问题而来!偶的记忆里清楚地记得,某天我和E从headquarter总部吃完午饭回来,傻傻的自己从后门进 office 门的时候,嘴巴上还在继续着回来路上的话题,忠心地表达和祝福着他,将来工作岗位上,manager的职位是属于他这样英语好、专业能力也很不错的年轻人的," It belongs to you! "后来再回想起这个镜头的时候,我都恨不得去狠狠地抽自己几个耳光!

E之前住的远,早上十点到公司,下午五六点钟走人;后来搬家住到附近,也没感觉到他工作作息时间上有任何变化;某个周五有事大半天不能来上班,大概心里还老大不愿意、不想向manager请假。那天的自己对于E这样的伎俩已经司空见惯,甚至有着某种程度上的本能地反感和厌恶。于是,我就静静地坐在那里,看着他同A对答他项目上的小问题\textasciitilde{}~ 我的脸上应该是没有任何表情的,如果有,那也一定是冷的!打个不太恰当的比喻,尖人一定是这个site的警犬,他的身影和智慧就像周遭的空气无处不在。但是那天,在我的冷漠和E的伎俩面前,坐在邻桌上的尖人终于是按兵没动!

好,就算E有他想不通的委屈;那我呢,就因为自己年龄大些,感悟和境界略高,那我就该去承受他board里senior舅舅的攻心战术么?内心里、专业道路上,我又何尝不是一个想要机遇、希望成长的孩子?我去帮忙打击和矫正他的恶习,谁又给了培养我的机会?在这样的环境里,公平是什么,谁又能真正主持仗义?更何况,整个暑假、自始自终,E都有着来自专人保镖一一尖人的保驾护航,他还想怎样?

既然E已经对我有这诸多的不满,那我还是老办法一一我惹不起,我躲,还不行么?于是,同往常一样,那天中午,E的family friend来坐到我们桌之前(还是来到座位我同他打声招呼便走了,我已经不记得了),我就早早撤退好给他滕位置!

我想,正如大家所看到的,我也并不成熟;如果我拥有强大的内心,我想我会试着去 behave 得优雅一点儿、大度一些\textasciitilde{}~


\section{实习(54)}
\label{sec-5-57}

高中物理早就学过,力的作用是相互的,我同B和E的嫌隙也是大家有目共睹的。后来同E,我们虽然坐得面对面、每天中午还是同小伙伴们一块儿去吃饭,但实际上我们话都说得很少;与B从月初分开后,我们基本就再也没有见过面。如果一定见到,我想我应该还是会礼貌性地点头表示一下的。 

快乐的时光总是过得飞快,追随A的有限几个周的日子转眼就过去了。站在暑假实习就要结束的船尾,我也禁不住会去问自己:这个暑假我收获得到了什么、学到了什么知识才干、又有怎样的经验和教训?我最在意的,因为在学校没法出头,我所有科目的最好成绩都只能得B,那过完了这个暑假,我的生存现状有什么改变吗?

这个暑假,我挣到了学费,但它终归都将会给交还给学校;这个实习的前一个半月是用来浪费光阴的;只在换了mentor后的几个周时间才真正有所斩获;几个项目,虽然上手很快、落后别人一个半月后仍能进展顺利,但毕竟别人不需要你快,别人需要的是你去等!那我的生存状况真的有变化么?

实习对我来说就只剩最后一个星期了,天气的骤然变化,我同A上周四架也吵过了,后续的表达对公司歉意的午饭周一也已经去吃过了。实习的大幕即将落下,我心上某个地方却总有一丝荫黎挥散不去、寂寥不已。


\section{实习(55)}
\label{sec-5-58}

早就听表姐说过,这个site里如果最大官衔的manager招人,他从来都一定会打电话给前雇主们去亲自省查新员工人品的。表姐、A和我我们三个的性格比较搭这点早就提到过了,天秤座的D和狮子座的自己相处起来也有着如同与多年来学校男闺密间的协调与融洽。很久之前同表姐的聊天也提到过,现在这个team里的人干的活,建一个team之前基本就是表姐一个人白天赶晚上、晚上赶白天一个人赶着就着应负过来的!如果说A和D真的是公司帮我找过来的小伙伴、公司帮聚集 collect 过来的易燃易爆品;表姐因着与自己的远亲血缘关系,整个夏天C的各种会议里、各种talk里,表姐一直都是那个被冷落和孤立的对象;那么这个team role里选项B的设置又是怎样一种存在,为什么会安排C与B组成一个亲密无间的team,来到这里?

一年前的国庆,表姐回加拿大同丈夫、孩子团聚的时候休假一周。当表姐返回、我从机场接到表姐回公司的路上,我就同表姐讲,表姐离开的这一周,B星期一一早就带着自家后园树上产的果子来公司讨好manager和大家,并在周一下午下班前,请教A、同A探讨了她的medusa python 项目的规划设定。她讲了10到15分钟的时候,大家没什么奇怪、很给她面子、伴奏般地给她配着背景语音音乐;当她一直讲一直讲、讲到30分钟、40分钟、45分钟的时候,同我一样所有注意到他们的人应该都能清楚地感觉到,B醉翁之意不在酒,班门弄斧么?B竭尽全力、试图向A展示的是她的professional,B向A传达的是或许来自于manager C、一种公司对待我这个实习生、欲要踩我的宗旨理念?!!

接表姐回公司的路上,我问表姐她手上有多少个项目、安不安全,我对表姐讲,这个女人极有野心,她是表姐帮忙refer进来的人,但她或许也是那个想要从team里挤掉表姐、取而代之的人!我曾经以为E上班时间比较晕乎,但那天E否定了我对他的印象。当B从A的cube里一返回(他俩座位原本就在隔壁,位于同一走道的同一侧,两人之间只隔着一层夹板),E就迎头接上去找到A。新接圣旨、揣度圣意的A对E很是热情!我对表姐讲,那一刻,我是生活在套子里、角落里的人,他们的team早就已经形成了!


\section{实习(56)}
\label{sec-5-59}

长江后浪推前浪,一代更比一代强。历史的车轮滚滚向前,我们一一两表姐、我自己,我们都已经全然超越了父辈们的生活;成长在这样一个伟大的时代,没有了父辈军师的引导指导,我们摸着石头过河、爬摸滚打中长大,体验经历着各自的烦恼,可以猜想,哪个人的内心没有经历过苦痛挣扎?

四年的大学生活我只对家住柳州的室友略为提及过高考的经历,这次请允许我放过高考。不管当年的自己对这位情敌多么妒忌排斥,时过境迁大幕落下,刻入记忆的却是竞争中的友情!记忆回塑到这里,恍惚中禁不住呢喃想问,当年的自己,究竟是爱着那个人,还是被生活中那个情敌吸引(生活中没有主见、看见别人男女朋友关系不错就跑过去抢的小三还少么?)?我是该冒险一试直接出国,还是稳妥地以国内研究生为跳板?大三下的生活就像阳光下空气中舞动的尘埃般浮躁不已!大四上学期我以极大的魄力勇敢报考了北京的研究生院,那时进退之间何其潇洒,伯仲之间见伊吕\textasciitilde{}~ 2002年春夏,我度日如年地等待315分的考研成绩、度日如年地等待划分数线、复试;12年后的这个春天,我再一次度日如年地祈盼幸运降临、刀锋上讨生活,以后再表,不叹也罢!2005年春夏,去留之间,一身冗肉也给瘦得用同学的话说"这么干净"!

从上周四晚开始,一段时间以来我一直休息不好。晚上之所以睡不着、休息不好,也是因为心有千千结。周二下午两点钟左右,像A所建议的,我最后一个项目从硬盘选择文件、先hard code一个变量、再选择一个变量的小任务已经完成;我去lab里找到A,他正type在忙着。前一段时间一直不知道E在忙什么,现在他的项目总算是进入到了测试阶段,正半天半天、整天整天地坐在lab里调试呢!我告诉A我的小任务写完了, A 轻描淡写地说,那就把另一个hard coded的字符串变量也改变为 button 自由选择吧\textasciitilde{}~ office里自上周五以来鸦雀无声,我的心里却翻江倒海、惊涛骇浪。在 A 的station边儿上趴了会儿,想到C还要来监督我们,我又何需予人话柄,于是两分钟不到,我站起来,感觉自己、像一口气提不上来、快要死掉,我独自出门到 block 里去转圈透气了\textasciitilde{}~


\section{实习(57)}
\label{sec-5-60}

我只跟着B作过一次high priority的任务,她说她有一件急任务,问我要不要跟她学,那时候的自己当然还是要的。当时的自己功力浅薄,尚辨不清B为何物。同B并排坐在她的cube里,作着一个关于Xgig的命令式界面操作任务。她连邮件里的任务问题都读得很草率,根本就真的读掉了关键词,还是我要求她读慢点儿我们才又读懂、读清楚的\textasciitilde{}~ 那天的B是真傻,还是装傻?真傻不奇怪,装傻也说得过去。

埋在她code里的bug一定是真的,因为她不曾、也不该指望我会去翻她的程序,那天组会后我对她指出这个bug时她一脸的错愕便是明证。那天C出差我们有困难的bug也是真的,她体会不到那个周二我的压力、傍晚时候还嘲笑过我,尖人也早就准备好了给我难堪。而且,那是整个假期第一次、也是影响最大的一次事件,案前、案中、案件结束后影响都非常大。Manager们事后为平息对B的不利影响,作了很多工作,绝不会是一场表演。

那二三十个小时的 power on hours 那一行代码呢?我太急功近利、急于求成、急于去帮忙解决问题了,甚至连B的表情我都没有去注意!也因为那个bug太简单了,简单到再回想这件事的自己已经不敢、没办法再去相信它是naturally发生的(表姐说过,有人就曾在lab里基本算是指着她的鼻子骂她"占着茅坑不拉屎",虽然表姐没有任何必要对我撒谎,我也丝毫不认为表姐会对我说这种无聊的谎言)!


\section{实习(58)}
\label{sec-5-61}

那上周四的checkin项目呢?周四下午我明明已经同mentor吵起来了,为什么她周四晚上傻到会在我之前去交项目?我的整个假期,或者甚至从换新manager以来(听表姐说的),B同C都有着深切的联系(她们俩个有极多的talk-talk,我没事儿时基本就见她俩在talk),舆论甚至为表姐出过头,认为B这样的奸佞小人为老板拍须留马,才得以平步青云、扶摇直上。也插一句那次舆论C的处理方案吧:以往的方案是C找B谈完话后紧接着就找 A 谈,一般不会找表姐谈话;而到底找不找表姐谈话、找不找D谈话就是要看情况、看心情不指望下文的事,这样A的CREDIT可以凭借C manager的威力部分地通感分配转移给B,因为A的优秀是大家都能够信服和认可的,B没有、也得不到这样的认可;那天C的应急方案是,先安排了B周五下午早走,这样C上午找B谈过了话;等下午B走之后,C再找A和表姐谈,以正视听(功臣依然是功臣一一A和表姐是),但原因也只是B那天下午不在(所以同以往谈话相比会显得反常)!她们俩个(B同C)究竟只是性格很像,还是别有企图和用心?那天晚上B风中凌乱、把subversion搅得一潭浑水,究竟是B个人成为,还是来自于C的主意和要求?

B是真傻,那就与我们一贯需要走T台去求人的作风吻合一致,这个人就实在是学得非常肤浅,没有能力和毅力去深入任何有挑战性的任务和项目!若B是假傻,那B装假的目的又是什么?site里manager招她这个天蝎座人(大概属龙不太记得了)就是要利用她的自私、善争好抢的个性来与我狮子座本能的aggressive相抗衡?

再者,直接导致换mentor的那次争抢怎么看怎么像是表演。组会上她提出的想法、一接到任务后我周五同她打招呼时她的有急事、她周日的抢,方方面面、点点滴滴"表演"(或者是"明抢")的痕迹昭然若著(除了(except)她自己当初不敢checkout一个版本来测试,我与她对质时看得见她脸上发懵)!再有,C那天下通知给我换mentor那个周五,下午(我中午吃完午饭回来)lab里station上显示的测试结束,是谁摆出来、等着我去看的,会是B吗,还是另有他人?是他们早就等着那一天到来、早就抓住了我的把柄,还是这所有的一切原本就只是一场歪打正着的意外?


\section{实习(59)}
\label{sec-5-62}

同E的相处里,我总觉得自己眼睛很毒,毒到自己都不想再去看某些人事,但我的眼睛真的毒么? 这个小时候几年里耳朵灌脓基本就没怎么好过、长大后才意识到自己耳朵略聋的人,却也是一直以来都是在用耳朵来谈恋爱(参见之前返校找表哥部分章节)、用耳朵和语言来工作的,却全然忘记了脖子上、脑袋上眼睛的存在,虽然多年的学校生活,同大部分学子一样,我的眼睛也早已近视,忘记了人与人之间的交流沟通,忘记了除了谈话聊天以外、对白独白之外的表现形式还可以有很多很多\textasciitilde{}~  

这个周自己严重睡眠不足、精神状态很差,因为以前自己没能梳理好的人际关系,因为一场实习即将结束我却没得到任何收获,心不甘情不愿,但大堂的钟声并不因为我闹情绪就将时间静止在这一刻,离了谁这个世界还不是在照样运转!

如果能够整合表姐与自己的想法,那么B的存在原本就是用来"对付"我的。只是,我想不明白,我只是一个普通得不能再普通的小老百姓,我有什么可以、需要、值得别人去对付的?如果说最开始我对B还有着对mentor应有的礼节,我还会偶尔坐在她的座位上虚心向她学习;那后来即使我对她讨好地开玩笑之后,我也不愿再坐她的座位,用她座位上的station,我们之间已经没能再剩下什么。如果说之前换了mentor我还对她心存感激,好歹别人教过我,现在如果我再见到这个人,我想烦燥的情绪会让自己直接起火,会冲上去直接问她十万个为什么\textasciitilde{}~

所谓,酒不醉人人自醉,工作不累人人杀人,说的该是我现在所处的情境吧。但这一切的原因也只是我miss掉了一部分情节。多想回到从前,多想像以前一样无忧无虑地type code,但时过境迁,一切都已经回不来了。多疑、忧虑或许也是成长必须经过的一个步骤吧。这或许也是自上周五以来office里出奇地安静的原因吧。毕竟,这里有个火山随时都有可能爆发呢。 


\section{实习(60) }
\label{sec-5-63}

表姐知道我这最近一段时间情绪不稳,便硬拉了我出去走路转圈。表姐当然知道我的委屈,她自始自终都知道的呀,但她打消我怒气的方式是,"你在这里工作,难道别人没有发你工资么?","别人发你工资就可以了,你还想要怎么样?"表姐是对我知根知底的人,很难想像她会说出这样的话。我也想要心底踏实、成就感、顺畅的人际关系呀!

偏巧在我最是急火功心想不清楚的时候,看见firmware的女manager正在陪B散步,还有一个另外的女同事。看见B的瞬间我说话的声音陡升10个分贝,仿佛同我说话的表姐已经变成了那个B,所以我已经是在迫不及待地渲泄着自己的怒火!表姐也是哪壶不开揭哪壶,明指着B问我,"哦她就从来没教过你?你就从来没从她那里学到任何东西?"可谁又是那个没有思维的存在随便她们如此折腾?人之所以会暴发也是因为总是有火上浇油的人来激发这一切的发生!我同表姐背道而行,180度对立!

如果说firmware的manager是公司为B指派的同一思维类型的人(也因为工作的需要,当初firmware的manager必须得根据公司的需要给B面子帮她找台阶下的呀),那55F的女士也一定是公司指派给我的,因为我俩还真算是比较搭,这大概也是那天是她来问我要不要同她们一起吃三明治的原因吧。Site里的那些头目们把人性这玩意儿也研究得挺透彻的嘛,同本狮子生涩的想法不谋而合! 


\section{实习(61) }
\label{sec-5-64}

这是整个假期里我精神状态最差的一周,虽然我的任务都还在可控范围内,但已然没有了先前的斗志,因为眼见着这个暑假就要结束了,所有的努力都让自己深觉无力。

好几天了,E都在lab里测试、调试着他的程序,我就搞不明白编程对美国学生来说真有这么难么?这期间一些人事上的舆论也影响着office的氛围,上周五A不是帮C纠正了"Tahoe"的正确发音么,当时 C的情绪、语气变化,加上A自工作以来就展现出的优秀就直接引导了大家的疑虑,一般可能都会去想A想要上位(但我知道小伙伴们出去散步的时候,他还帮助过我纠正过错误发音呢),C的blah-blah-talk-talk style可能与这个公司的风格真的不搭,这是一个technical公司,也许大家像我一样觉得B与C、C这样blah-blah-talk-talk很烦呢,可这时大家也都还没有足够的心理准备来承受这样一种潜在的转变。为了表白自己,也为帮助公司里内定的人选(以尖人为代表的、B对面的senior、firmware的女manager等人在C talk-talk的作工下也都逐渐地站好了队型),也为回答E的疑惑,某天下午两点钟左右,A在E到他座位上请求帮助时就直接来到E的座位上、直接告诉E coding上的答案、像放鞭炮般噼里啪啦地告知答案,帮E解答了五六个E原本不会的module上的问题,大概是因为E实在是太慢了吧,太久没有checkin项目了。 

我的项目也遇到了一个bug,事已至此、回天无力的自己对工作很是闲散(被表姐拉去外面、自己独自出去透气好多次)、以及A全然告诉E所有疑问的答案的情况,也直接导致了自己慵懒、泄气、想要放弃,便也接着E问了A一个bug。当时的A已经没有了回答E问题的直帅、清扫淡写、云淡风清,而是让我等他忙完(两个小时之后吧)。坐在cube里的自己就禁不住走神,反复琢磨着A待E同我这一正一反的对比,是出自于公司以及C的旨意呢,还是因为A真的只是希望我能自己解决问题?但那天下午那个我时时走神、渊源流长的等待还是让我觉得大家(至少公司、C、尖人、B以及B对面的senior吧)真心想看到的是我这个实习生不行!悲从中来,便强迫自己努力去独自解决问题。


\section{实习(62) }
\label{sec-5-65}

两个小时之后,A出现在我的座位上。我的不听使唤的脑子还在反复琢磨着,A对于在这样的时间与E相反地给了我这么长时间的等待或许也有着他自己的谨慎地考虑吧,我从来都是信任A的。 A大概想知道我自己对这段等待时间的看法吧,问到我这个问题,明知道自己一个小bug被这份无限等待给放大,却也并没有太在意这一点儿,这段时间里,至少认识到放大后我自己还是在努力去解决问题的,是死之将至、其言也善么?我表明自己的观点说" I need to spend some quality time to learn how to debug by myself! "A或者也想知道这个bug可能会对我造成的影响吧,我便直说了," If I cannot fix the bug, I will regret and get hurt for the rest of my whole full semester! "逼我承认了自己的确需要帮助,认识到这个bug对我可能造成的伤害,A开始动手帮我解决问题。

那个从来不用Emacs的A,连这种编辑器一个热键也不会,连复制、粘贴键都不会,居然也没有换个编辑器,也没见他type什么code,就只用鼠标便三下五去二地、以雷鸣闪电般的速度把我的问题给解决了,连我之前想要得到的图片图像都两分钟不到就已经全出来了!" Do you think if this will work?  "," Will it solve the problem? "A的语速并不快,只是解决问题的速度太快了,"Does this look like a graph to you?"A继续问我。这场景、这速度、或者说,这境界,在我颗昏昏乎乎、惶惶不可终日的脑袋里完全没有反应能力,这人早就傻掉了,于是便把他爆打一顿(在他离我最近的胳膊上使劲地打了几( 三 ?!)拳,这让实习生情何以堪、要肿么存活、还让不让人活的?实在是太逆天了!


\section{实习(63)}
\label{sec-5-66}

A走之后,真正懂了那个窃门后,便能明白,当时所谓的 bug ,也不过是自己第一次作命令式用户界面,对生疏的问题还是会稍微有点儿心理障碍,或者换句话说,也不过是因为 lost trace ofthe log 。如果两个小时前没有那份慵懒、如果自己能focus不走神,根据 log 是一定可以找到解决办法的。这件事过去快一年了。这一年里,我深深记住了这个教训,却全然忘记了当时那个 bug 本身到底是什么问题,大概bug本身真的不难(我记不住呀)、我也算是从教训中真正成长了吧。

在别人给过的难堪面前,我屈服了,而是非常积极正面地承认了spend quality time debugging的重要性。那天下午再接下来B对面的senior就站出来执行同样的理念了。B大概稍早的时候就有难题不懂去问过了senior,我们的座位都相隔不远,所以大家说话只要不是声音太低就基本都能听得见的。A帮我解决那一个问题之后,当B再去问senior她的难题时,senior就说要B自己那天接下来的时间、傍晚、晚上都再她自己先好好想想。万一晚上想不出来,万一到第二天早上还是想不出来,senior会第二天再去帮她解决问题。

这是发生在同一天下午的事情,前前后后不到一个下午的时间。B真是笨么,或许她早已猜测到senior这样的做法也不过是宽慰一下我过去几个小时所承受的委屈而已,同以往晚上会在公司加会儿班不一样,B从公司吃过晚饭后半小时不到就早早地回家了,大概是因为第二天早上自有答案会找上门来吧。 


\section{实习(64) }
\label{sec-5-67}

是周三还是周四下午,C有一次来我的位置上问了些什么问题,大概是确认这是我的最后一个周、周五是工作的最后一天吧。可以猜想,她应该会叫上组里的人一起,安排一次告别午餐吧。剩下的时间不多,我便也试探性地问了一下,"You won't push me to work too hard on my last day, will you?"C看着我笑而不语。

工作上我比表姐的common sense稍强一点儿。表姐上个周的时候还在说如果我想再工作一个周,只要跟manager说一声就可以了,学校那边就不用打招呼了(当时的自己讶异得不得了,但只要心里,嘴上并没有说什么)。表姐有她自己的工作上的追求,但这也不防碍她成为俗世里还算活得比较明白的人。

其实我就一小小实习生,从来不指望自己过生日的时候还能同大家一起出去庆祝;我就要回学校开学了,C会组织组里的人出去吃一餐午餐,就是再合适不过了的。可表姐还是以她SENIOR的资格,带着与她关系密切的人,我们先出去小打小闹地聚了一场。表姐、我、A、D和尖人,因为尖人当初是表姐帮忙REFER进来的人。正如大家所知道的,尖人的立场与我们是相冲突的,尖人之所以成为尖人,就不防碍他稍微为难一下我们,比如周五C带大家聚,周四午餐就该是极好的日子,但就算人家尖人周四周五不来上班,也可以帮给制造一点儿小尴尬、小GAP的不是么(这里任何的解释其实也都显得苍白,过往的经历不是也已经说明了一切问题吗),所以,我们大家小分队周三中午出去小聚的。 


\section{实习(65) }
\label{sec-5-68}

自己的项目眼见着就像往常一样轻轻松松地就要做完了(只不过这次不小心自己找了个BUG, 不过这也是A MENTOR我以来的唯一一个,他没有帮我解决过任何别的BUG的!),但阴郁的心情却不见有丝毫的减少,生存的出路始终是压在自己心头的一块千年的石头,挪动不得。表姐早就知道我心情不好,这个周陪我聊天也特别多。再下次我们在外面转的时候,表姐就又有一番话来劝我了。

表姐这次直接对我说,项目要是没有做完就没有做完,做不完也没有关系。MANAGER已经说了,甚至可以提前走,有事情就不来,最后这几天也完全可以不来,直接走人!我的那颗小脑袋听到这里就"嗡"地一声真炸了\textasciitilde{}~ 写到这里,深感自己语言贫乏,我的眼珠也一定又掉地上了;如果我的小伙伴们此时也在这里,他们也一定同我一样惊呆了!为什么这最后一个周,所有的人、所有的事、我都得去想它们的另一层意思?这哪里还是我那个极有魄力的表姐,这和俗世里的泼妇呈现出来的有什么区别?"什么BUG,HURT,REGRET,什么乱七八糟的,这种话都是矮情!"表姐的话说到这里,我终于是禁不住再次五雷轰顶,表姐的话在自己现在的脑袋里就给直接翻译成了,C要表姐对我传达这样的意思,真搞笑啊,这怎么可能?

\textbf{自己加的:} 写到这里,真的是写得很郁闷啊,写的心情也是此起彼伏

或许是没能建立好与读者间的联系,失去了大家的信任,非常遗憾\textasciitilde{}~

其实我是想写爱情的,大家如果想对骂对战,就早点儿骂早点儿战好了,这样子太难受了\textasciitilde{}~

周中有事,会写得慢些,周末多更新多一点儿,希望接下来的内容(两个学期)能两个星期写完

这样大家都能解脱了

今天晚饭前会再更新一次

\section{实习(66)}
\label{sec-5-69}

经过这段时间反反复复地考虑,结合表姐之前跟我提到过我的一些信息仔细去想,我还是能够想出个所以然来的。

之前表姐作MSTK、NUMv等项目的测试,都是手动人工去测,并没有automation;后来大概四五月份的时候,表姐接攻艰项目,从这些里撤出来,并且因为已经招了B,B向表姐要去了MSTK的项目她来做;B喜欢抱一堆项目在手里(感觉安全?),她还从别的TEAM里抢到至少一个项目。而她能从其它manager手里抢到项目本身也就开启了一场site里的领导对她特殊时期、特殊人才、特殊处理之程。

因为仔细观察就可以看见,因着site里头目们的影响,至少部分的manager或是组长之类的在很多事情上会给足B面子,而B在自己组里却得不到C之外(except)任何人的任何认可,甚至得不到我这个实习生的认可,一如C的blah-blah-talk-talk style在site里显得张扬而不得力。能看得见因为实习生的事,A时不时地(我实习期间至少两次)照顾怜悯他的组员B的情绪,但其余时间就算表姐、A都同B都只隔一层夹板,大家都绝不主动同她讲话!

在我到之前,D到的顺序我不太清楚(好像是比A晚到一个周);C在A前一周到岗;B是新manager到岗后迎合新人速度最快的人(原manager可能还正痛苦心伤呢),B迎合C速度之快、关系之好让组里几乎所有的人都生疑;而automation A的到来就直接站了表姐的队(他们属于有能力有" soul "的一组);我去第一天表姐与B就不合,但我当初并不知道她们的不合缘自哪里。刚到岗不久的D性格好也被安排随manager随意处置,当表姐与A的小分队太过强大,比如C出差B同我出问题下不来台的时候,site里相关manager也就几次强推D去站B的队以帮B取得平衡。


\section{实习(67)}
\label{sec-5-70}

我来实习后被分在C与B的小分队,但就像我onsite后的感谢信更敬重认可A的professional,对C和B我很难同她们建立良好的合作关系;我吃着碗里的白米饭望着E、A那边可口的面条,也很惨呀,终于在消磨完一个半月的时光后给转到A那边。

C和B存在的理由大概是打突击,对付我在那里实习的两三个月的时间。她们的确时常地表现得不够专业、敬业,但别人扉用她们的理由或许就是要欣赏她们的尘世属性呀。我去实习的第一天VP在、总部的market planning的代表也在。回想起来那实习第一天的午餐B或许并非不淡定,或许只是要彰显重要性,这样那餐午饭竟成了两个女人的颠峰对决!

在公司collect 的易燃易爆品面前(A和D的性格都极好),我经受住的了考验,没有同A与D中的任何人发展出爱情来;但公司或许是顺载学校的做法,强插C这个世俗的manager来镇压我的aggressive,坚绝不给我任何CREDIT,这从公司文化、企业文化的层面讲是不合理的,所以C终究必须走人,而B走不走、什么时候走大概要取决于她什么时候把她抢来的各大项目拖垮吧。

想明白这些,之前的方方面面、点点滴滴也就不那么矛盾了。这一个组原本就是两个team,两种strategy。大概也就明白了为什么C视我们为风,为空气,为不存在就是为了不给我们credit吧\textasciitilde{}~


\section{实习(68) }
\label{sec-5-71}

我的项目也如期顺利地做完了,A没时间,但他A答应周五帮我REVIEW。没有想到的是周五,A不仅帮我REVIEW了,而且也帮E REVIEW了,而且E排在我前。 

星期五早上11:00钟,A去了E的座位帮E REVIEW。我因为自己的项目已经做完,便也就坐在自己的座位上听着他们的讲话。大概是因为之前A是直接告诉E答案的吧,这次review A便有问到一些步骤的原因,奇怪的是,E答不出来;A需要听答案的意图还是很明显的,同样的问题问了第二篇,可E还是答 不出来。没办法只好继续,E大概应该能意识到他是应该需要去弄清楚那些步骤的原理的了吧!

没有想到,A的方式也同样地应用到了我身上。当他问我某个button的作用时,我还有点儿搞不清楚为什么这个button可以成为一个问题,愣着没说话;当他真也要起火问我第二遍的时候,我也不服啊,就带着戾气地答了,点这个button不就是为了调用一个叫什么什么的函数么?但A知道我是理解的就很满足了,问我刚刚为什么不答?我便说我刚刚还没有想到答案呢!

那天review的结果是,我有一个细节不满足他的要求,需要再修改一下。 因为之前的项目REVIEW前都是我先写一个我自己的解法,他再帮REVIEW IDEA先,最后有必要的话才帮我看CODE的。所以这最后一个项目,当我拿到项目的时候,我并不知道他有这样的细节要求,应该是稍微改一下就可以的吧。以前的项目一直都很顺利,我没什么压力,等A帮我REVIEW完,C就带大家我们该出去吃饭了。 


\section{实习(69) }
\label{sec-5-72}

那个夏天我锻炼得比较多,最多的还是在我家旁边的公园里hiking,那里的Trail还很长,连走带跑快速一圈过来差不多也得两个小时。 我这个夏天的周末都是在公园里度过的。 

我们开了两个车,A、D以及D的小伙伴和我我们熟,坐了一辆车;E去同表姐和C坐了另一辆(其实我已经不记得谁同谁坐了哪辆车了,记忆里大致是这样吧)。表姐帮推荐的一家素食馆,因为C是素食主义者。 我坐在一边角落里,A坐我对面,C坐我旁边,E坐C对面;再往那边是表姐,D以及D的小伙伴。 大家点了食物,吃得也还算热闹。

饭桌上大概D有提到接下来的长周末会同朋友一起坐 飞机到什么地方去HIKING\textasciitilde{}~;聊到E的少年老成(E特成熟),A一边作表演一边开玩笑同E在一起比着说,我真的比你大么,谁看得出来?C有两个孩子,然后也聊到C对她家孩子的教育方式。C说她不想不会去打击、或者阻碍了孩子们的热情和天性,所以所有的事情都会尽最大可能地以一种POSITIVE的思维去对待和处理。饭桌上C说的是她的孩子们,我脑袋里反映出来的就成了B就像那时她口中她的孩子(或者她的母亲)一样C对她那么亲切呵护,但我们其它人都不是,一定得不到她那样的对待!  饭桌上,我也有注意到C稍微尝了口某片鱼,但觉得不好吃就放弃了扔在盘子一边;后来我见A再去尝试那盘鱼的时候就也浅尝辄止、给扔到了一边,我心里纳闷着,这难道也是同 manager沟通的一种方式?嘻嘻\textasciitilde{}~ :)


\section{实习(70) }
\label{sec-5-73}

吃完中饭,我们几个熟悉的小伙伴打算坐一辆车一起走的。A原本说这旁边哪里有珍珠奶茶可以喝,但我们去看了一下,人太多队伍很长,我还有一个bug还没有改呢! 大家便陪我一起早点儿回来了。

坐在自己的坐位上时已经是一点半快两点钟了,折腾了一会儿,不对呀, 我怎么还是没有头绪呢,这可要怎么办?难道我们午饭时间哪里不对刺激到C了,D是被curse了么?下午一回来,C就去D的座位上去找D,感觉有重视和提升之意!  那天下午半下午的时间不到,两三个manager去D座位上去找D,先前我同A的第一个常识那行代码犯傻时那个离C不远比较BULLY的MANAGER也已经去找过D了,难道他们要transform D去干一件什么要务?其实Manager 去找D一点儿也不奇怪,奇怪的是吃完饭回来就去找、不止一个manager去找、好几个头在很短的时间里连续去找,难道C同我们小伙伴一起吃一顿午饭C就窥得了什么天机?

但,现在,我已经想不动这些了,我已经不想再去想这些问题了。现在的自己压力山大,为了给自己减压,我去lab里找到A,想问他万一我做不出来会有什么样的严重后果。" What happens if I cannot fix the bug, this detailed requirement? "我 的问话中略带着怒气,因为属于我的时间不多了。"Then the project is undone. "A答得很平静,气头上的自己觉得这话怎么听着怎么像他、他们就在等着这样的结果呢!" Congratulations! You are half way there! "我终于是到了跌到了生死悠关的关头,怎么可能没火气?


\section{实习(71) }
\label{sec-5-74}

我同A的关系是极好极好滴,以至于A作我mentor整个期间,在某些问题上(比如前面那个MSTK LOG项目),我从来就是完全忘记了我还是可以去反驳的!就像刚刚过去的一幕,其实我应该完全有理由请求mentor早点帮我review,以防万一review之后还有什么需要修改的,这样也可以给自己留足充足的时间。现在已经走到了这最差的一步,感觉自己快要只能望洋兴叹了!

亲爱的读者,如果此时我告诉你,因为我已经理清了这里的人际关系,所以为了满足一下manager C的愿望,所以我顺应她传达给表姐的旨意,直接放弃,下午早早地回家去整理东西,好第二天早上一大早便上路回学校,那一定是假的、骗人的,因为我的个性、computer science major学子的专业精神不允许任何人去作这种事情的啊!如果此时的我告诉你,我才高八斗,没有任何困难地两分钟就解决了问题,那也一定是骗人的。历史上真实发生过的事情是接下来这样子的。 

心里是万般憋屈窝火,可我没有任何理由可以对这样一个待自己也还不错的mentor发火,我能发出来的也只有自己的蛮横无理而已。这是自上周四晚瘦掉五磅肉以来已经是连续一个周休息不好了。课文上讲,"强弩之末,势不能穿薄绢也",高考的时候别人都在好好准备复习,我却还在自己一片泥泞心伤痛苦挣扎,难道我还要让这样的遗憾无休止地重复下去么?C说过今天下午五点钟她来收我的门牌卡,看看时间,两点过一点儿,这样我还有差不多接近三个小时的时间。我给自己下任务:这三个小时,我可以允许和接受自己因为经验有限、精神状态不够好而做不出来的最终结果,但我绝不接受自己不去努力的态度!于是,我强迫自己赶鸭子上架,硬着头皮上!\textasciitilde{}~


\section{实习(72) }
\label{sec-5-75}

我以为我会有一个轻松舒适的实习最后一天,却没想到我要接受一段时间里最残酷的考验。如果说世界上只剩下我(俩)存在是属于爱情的至高境界,那在这特殊的一天,我又真真切切地体验了一回。到那天,真的是实在是没一点儿精神了,跑到厨房里又灌了杯浓咖啡进去,强打精神开战。

因为精力不好,很难聚中精神,我开始think aloud,把自己的想法默念出来;这个时候天地之间就真只剩下我一个人了,我 think aloud, even talk aloud, who cares? (就算周遭的人受不了,可这是别人实习的最后不到三个小时,能够体验经历被打扰一回,也算是幸运吧! )

怎么说是一个什么问题呢,用户式界面,如果我说有一个对话框,当我调用x,y变量,把图画好,我的界面会显示x,y变量的值一次;当我再选变量调用 20 次 50 次,我的界面上会再出现 20 个 50 个显示x,y变量内容的对话框,而A要我界面上始终只有最后一次结果的对话框,大家就该明白我到底算是出什么问题了吧。

当时的自己是完全没有思路的,如果说之前我还学过Visual Basic编程,应该有过一点儿用户式界面的经验,但老大,那已经是我尚不是这个专业、半个世纪前的事了,当时学到的一点儿皮毛也早就还给老师了,现在我遇到这个问题就是两眼一抹黑、伸手不见五指,什么也看不见,什么都不知道!

但前两天A帮我fix那个bug的经验告诉自己,我应该去trace back log(smileface),根据log,我应该可以找到一个突破口;另一个突破点就是,前一两天的那个bug,A帮我fix的过程并没有很多coding,或许同我现在这个想要的界面处理有相通之处,我应该再回过头去好好学习一下的。 


\section{实习(73) }
\label{sec-5-76}

这样,在午饭饱餐一顿的困倦下,在一杯浓咖啡的药镇静下,一个半快两小时后,我总算是从阎王庙里逛一回、就差喝下孟婆汤、穷途末路地把思路给弄清楚了,再花了差不多一个小时左右的时间把code都调整好、整理好。中间我去问过一下A,A要求我把项目直接交到subversion上去,我剩下的时间不多,还要在交门卡之前至少写一两封感谢信吧,所以 04 : 42 左右再检查一遍自己的程序就就把项目交上去了。 

为时已晚,我所剩的时间并不多,我便开始写感谢信。打算写一封给整个team的,写 一封给 mentor A 个人的,因为这个假期,我们的关系由一封感谢信可以,顺理也该由一封感谢信结束的。因为之前浪费过一个半月的光阴,因为这期间有诸多的不愉快,我以为我会走得心平气和、没有多少值得留恋,可真正写信时,一想到A,想到这最后一个月自己心里感到过的踏实和有所满足、那些点点滴滴、实实在在的实习感受,眼泪就不受自己控制地扑塑而下,这里还是有自己应该深深感激的人啊。就要走了,就要离去,我以为自己很沉默很小心地不再制作出一丝的声响,可我控制、压抑不了那抽纸巾的声音;在想到A的这段时间里,坐在自己有限的座位空间里,竟然是控制不住自己眼泪滚滚落下、几度哽咽!

如果说初三的那位朋友是解救我于水火之中、一生都将值得回忆的朋友;如果说十六年前的舅舅为自己当时快要窒息的生活指出了一条生路、指了一条明路,从此阴郁的少女开始有了快乐;那这个假期,A的存在,便是用他大神般inspiring的智慧、设置了非常有效的learning curve,循序渐进、为我帮我真正开启了一条新的专业领域道路。因为这个暑假,借着良好的锻炼习惯和绝大部分时间优异的精神状态,在A的帮助诱导下,我已经开始懂得去"想"一些问题,这是我过去一年的新专业学习大锅饭式的教育一直都还达不到的、很迷糊的。常常是即使作业写出来,仍然不敢相信没有丝毫的自信,全是连蒙带骗、透着瞎猫抓着死老鼠般的辛酸和无奈。


\section{实习(74) }
\label{sec-5-77}

我把写给整个team和写给A的邮件都CC了一份留给自己,于是便有了后面的两封原稿邮件。待我把这些该写的邮件都写好,我剩下的时间非常有限,如果我还想去找谁见谁,那一定是A!珍惜自己剩下不多的时间,我就去找A了。

A看见我高压之下最终还是做了出来,感觉还是很为我开心的。时间就快要到5:00,邮件也已经发了出去,C就直接来我们这边找我了。A带我也就一起回到我的座位上(之前我们呆在A的位置上,我向他索要了邮件等联系方式)。因为C一直以来对我们实习生(尤其是对我)感觉很有些暴戾跋扈,我还是多多少少有些怕她的,便不知道该说什么该怎么办。

这时A就帮我解围,一边对C说,其实这个实习生做得还是很不错的,即便是在最后几个小时有限的时间里,项目也还是最终都满足了要求;也一边指挥我说要我demo给C看(E听说我需要demo他也过来我的座位来看了)。我就重新坐到了自己的座位上,按照A的要求,展示了从硬盘只从一个文件选择x,y变量,画出一个图来;对于这样的结果C没说话;A看了结果看过C后接着说,这个项目还可以选择多个源文件、把多个文件的数据整合到一起画图的,便要求我把选择多个文件的再demo一遍。我小心谨慎地都照做了,选择了 5 个源文件,结果也都如自己测试过的漂漂亮亮地出来了。A用桢查探求的目光望向C,我也顺着跟了过去,C的表情显得出乎意料、略带震撼和失望,这期间我有那么一秒钟的心理满足、暗自开心,可紧接着,这个manager嘴巴一张,就来了句:" I don't know if it can work from any machine, from any checkout verysion, or it can only work locally! "C这样的语音落下,E就回到了他自己的座位上去了。 A接着她的话说," As you have seen, it works great! "听到C的这句话,我就很不爽那个午饭时还说着(对她的两个孩子)她会所有的一切都尽最大努力、比较正面地去看待的C,对B一直都非常正面看待,百般呵护,为什么到了 我的头上就成了百般打压、抓住一切机会不惜一切代价地打击我?人与人之间的不同就这么明显和可悲么?不服啊不服,我很是不服,便直接问A,我不想删除自己的local files/directories,那我能不能用你的机器来测试一下给这个manager看看,A说他的系统缺一个文件,我没有可以测试的机器,时间也已经到了五点,C就等着把我送出去(赶走呢),我也就只好作罢。 


\section{实习(75) }
\label{sec-5-78}

\subsection{To: C, A, 表姐,D,B, E,me(personal account)\textasciitilde{}~}
\label{sec-5-78-1}

Hi Guys,  // Thank you!  Fri, Aug 30, 2013 at 4:49 PM

It is the moment that I have to say goodbye though I may not want to yet! My school started already\textasciitilde{}~\textasciitilde{}
I had great time working here as intern and I appreciate the chance working and hanging out with you guys. I have learned a lot as well as had lots of fun on Fridays, and interesting and unpredictable stories between mentors and me. I have learned a good lesson and I appreciate all the things whatever has happened in a good way.

Thank you so much, and wish you all the best!

mememe\textasciitilde{}~\textasciitilde{}

\subsection{Hi A,   // .uml outline 08/30/2013 at 04:44 PM}
\label{sec-5-78-2}

I have attached the required .uml outline for Medusa .csv results auto plot GUI, please check.
Please give me the chance to express with tears in eyes that, having you as my mentor this summer has been the most beautiful thing ever happend during my life. Thank you so much. I appreciate it a lot!

I wish you all the best!

mememe\textasciitilde{}~\textasciitilde{}

自已加的:还在赶工,今天的内容比较难把握一些

想把一个模块的事写完,写到回到学校

让我再多写写、晚上写完再一起更新吧\textasciitilde{}~:)

\section{实习(76) }
\label{sec-5-79}

于是那天下午接下来,就像从周五中午才review项目大家所看到的, C真的就像想赶人一样、收去了我的门卡、没有小伙伴们的陪伴,独自将我送出site。因为我同C之间越来越深的隔核,而我面对自己的缺点、脆弱和错误 ,我从来不不惧直面惨淡的人生,于是她单独把送我出来的路上,我就向她承认了那个换mentor的周五我已经认识到自己做错了的事实。" I don't know if you have learned the fact or not, I did realize that I made the technical mistake, and at least that once, B was correct! "我想我的内心是坦荡的,我没有什么见不得人的事需要躲着藏着。C便也开始安慰我说,希望我离开之后,能以一种积极正面的态度来看待这里所发生过的一切。 

前面提到,我所知道的site里的party有三次,我只经历了两次,第三次是属于E的殊荣。活动被安排在了九月十三,应该是E在这里工作的最后一天。而且不同于前两次活动在Local onsite在自家厨房的是,这次活动安排在户外,所有员工、实习生应该都是欢迎带家人亲朋好友可以参加的。而且,据说那次活动会有礼物,大家应该会玩得非常心。这些是送给E的贺礼。A早就说了,他会带他的妻子去参加活动。周三我们小分队出去吃午饭的车上并且他们有计划在那近期去visit 他的in-laws。

我也就要离去,我有些不方便带走的东西我还需要暂放在表姐那里,于是周五的晚上临走前我便去找了表姐。就像要远离家门的孩子,每逢别离,表姐也总是会有一番话要交待嘱咐给我的。就像那天表姐顺利完成他们业界在外面的测试后的那天傍晚我俩开开心心地聊天一样,这天晚上,表姐又把我带到了这个附近的公园。

他们会 受能做事有表现的伴侣吸引。他们仰慕触手可及者,并在生活的各方面寻求结果。 摩羯肖兔者很少会因容貌之故而喜欢某人,他们对人的个性本身感兴趣。

\section{实习(77) }
\label{sec-5-80}

如果说从前我只从舅舅舅母的故事里,和表姐对我讲过的事件里了解认识表姐,那这个暑假也让自己亲自见识一回表姐的待人处世、专业和厉害。曾经,甚至到实习中期的时候我还再认为表姐内心过于固执,即使有这样一次绝无仅有的实习机会,或许我们也不能珍惜、无法认识和了解彼此再深一点儿,我心灰意冷失望过,尤其是很长中间一段时间表姐一直不理我的时候。但等到实习结束,对表姐的认识多一点儿,表姐也看到我的干劲,也还真让我们相互间真正建立起了信任。这是我真真切切滴想往过的,看到这一天的到来,我恨不得手舞足蹈、心里美滋滋地、开心极了\textasciitilde{}~!

这是一位善良的女子,小的时候会爹妈分担这家务,照顾自己的妹妹长大;土相摩羯座,别人可能会认为他们固执,实际上他们很小心地提防可能的危险,在数里之外就能觉察到灾祸所在,所以他们几乎比任何其它的星座、比一般人都要实际。其实每次来到这里来到加州,我最先做的一件事情就是向表姐吐槽我所有的功课都只得了B,然后把自己这一个学期或者一个学年的上课情况、意外事件、受到过的委屈和伤害等等,我总是柔弱无力地把它们向表姐吐了又吐、吐了一滩又一滩,而表姐是耐心的也是善良的,我吐槽完毕表姐从来都是劝我,人与人之间是真诚的,要我将心比心,努力化解与老师同学间的冲突。对表姐苦口婆心般的劝导,我嘴上不说,但心里一直都是很感激的。而三四年前的 12 月,我究竟是赌得哪口气硬要认为表姐逼我还她两千块钱?想来那时的自己还是太不懂事了。 

\section{实习(78) }
\label{sec-5-81}

在我这里,表姐的人际关系也稍显迟顿(自己原本也是不个灵光的,要不然也不可能累积到最后)。表姐看我最后一个周挣扎在人际关系的漩涡里,她就一点儿也不难受么?她也难受,但她也无能为力、帮不上忙。因为表姐自己也看不清楚那些纷杂的人际关系啊!就像我能清楚地感觉到头目们在B出事时的宏观调控、A在B出各种事情时的几次微调,我能从与B相好的人的职位、处世方式猜测到可能的原因和蹊跷。表姐之前在美国这边最开始参加工作的几年里常常同事们闻得什么风吹草动、早走了,表姐总是那个后知后觉、各种后卫、打扫卫生却还被裁员下岗的人;另一边又有manager C各种打边鼓,而表姐也实实在在、从来就是一个manager让干什么就干什么的人,所以,最后一个周对我们俩个都是痛苦的,她看到我那样,也只能是徒增难受而已。

就像在B最为不顺的日子里,我仍能看见她不在site里吃中饭,但还很乐观地为自己找机会说她有很多Priority一样,相对而言,我其实是能够正面地看待问题的。我就把自己眼中D的乐天、任何时间都开开心心、请教问题别人都喜欢、同任何人总能站在别人的角度上替别人考虑、D小伙伴的向A问那个宗教问题帮我挡箭的好人品(以及我开玩笑时说他家儿子男孩一般同妈妈更亲时他说" Then my wife will be happy! "的开心)、E和B的各种小tricks、E的成熟稳重低调、B的乐观(除非自己打倒自己,别人没人可打倒她)和对待工作忘我、全然忘记掉team member的高度热情等优点,team里的各种会议方方面面、点点滴滴的感悟(比如会议上E的种种瞌睡;B的兴之所致、翻江倒海的各种新奇主意;有一次A说表姐NUMv一个什么东西是做不出来的,表姐事后同我说是A不懂那个technique,我对表姐解释A的背景里有这些版块的技能。A当时之所以那么说话,应该是因为manager已经明显表现出对B的呵护与偏好,任何一个新人都会顺应老板的意思而不是去忤逆她!)、尖人的各种神现身、以及A的专业能力上帮我fix一个bug所展现出的强大、工作能力上到岗不久,内敛低调(同面试manager职位、委屈在C之下)团结各色人(同尖人的交锋)等、借着automate其它team的测试,迅速升级为site风云人物、几乎可以架空current manager(我相信site里有很多像我一样忠心认可他的能力、祝福他的人,生日庆祝时也见识过了)、管理层面等等等都同表姐分享了一遍(A处理E与我的关系、帮B和平过度换mentor期等)。像上次用D的事例劝表姐一样,我还是希望能好言相劝表姐平易近人一点儿,同大家打成一片。

\section{实习(79) }
\label{sec-5-82}

我滴不隆冬、叽叽喳喳地向表姐说个不停。没想到表姐却打断打趣我说,"你这是情人眼里出西施,他的framework出问题了,你知道不?"见表姐如此反问、如此心惊,我便睁着眼睛说瞎话地问她,"谁的framework出问题了,A的会出问题,怎么可能?"表姐便解释、问我她那个NUMv早就交出去了,为什么那天C去找表姐,要她再去run一个NUMv几个test case? 

听到表姐充满惊恐、疑惑地问出这句话的时候,我的心就瞬间就融化了!这两年里我一定相信表姐的善良和无辜。表姐的问话多像高三那个不敢不愿意去上学,一定得去上学还巴巴地期待有人会鼓励一下当年的自己?当年的自己没能等到太多鼓励,但听到英语老师那句话的时候我心里温暖过;表姐是比我大 16 岁的人,听到这话的自己内心却像是一个母亲想要呵护自己的孩子,满满的都是理解宽容和爱意。我的那个从来都善解人意、听话的表姐,老板让干什么就干什么,甚至出去测试明明是B的MSTK出问题了,还委屈自己回来取更新的firmware,她何至于要如此屈尊?面对manager如此行事,她心里就真正没有过一丝一毫的委屈么?那时C找到表姐要她去作这些测试的时候,表姐内心里一定有过惊恐不安的吧。虽然表姐的项目早就已经交出去给A交了两三个月了,是A第一个automation的test sweet,可表姐对自己交出去的项目都还保持着非常谨慎的态度,揣摩着不知是A的架构出了什么问题,还是只是C想要她去核对一个什么内容(因为她相信自己当时的测试不出意外应该是没有任何问题的)。表姐说她当初测试的时候没出什么问题,A应该不会出问题才对,为什么C会安排她再去测一次呢?

我心里大致猜到是什么事儿了,便问表姐是哪天C要她去测试的,表姐答说是某个周一,我便问她是不是两周前的周 一,18号那天下午,表姐回想了一下说是。我便对表姐讲了那个周一A没有来,我以为lab里会空,所以那天下午去lab转告诉B对面的senior我想用机器的事。我安慰表姐不用担心,不但她自己原本的测试没有出错,A的framework也没有出问题,问题只是出在那天下午C要抓人去占机器占位置。我告诉表姐,现在知道,不光她这个攻艰项目的人被去当作苦力奴隶抓去占一台机器,另一个team里一个察言观色常同B搅在一起的那女的也给抓去测了,所以senior告诉C、C告诉site里其它manager,或者其它步骤,但一定是有一个人告知了site头那里,这也是为什么senior一两个小时之后才告诉我没有机器的原因,他们也需要反应、安排布置和分配的周转时间的呀。

\section{实习(80) }
\label{sec-5-83}

顺便地,我把自己生日他们庆祝、以及接下来周一的午餐等等一并讲给表姐听了。如果说属羊的女子内心是一个长不大的孩子(自己的个例就是中学时生长停滞五年),那摩羯座的内心其实也有着孩孩童般的纯真,我相信表姐是善良、清白无辜地被卷入到测试中去,进而还惶惶不可终日地担心了那么久!出于人本能的善良,我就把这些共有的关系、生活在同一片天空下的人事作为一种联结纽带帮表姐、也帮自己,把这一切的人际关系看得再清楚透彻一点儿吧。于是,我又向表姐讲了那天傍晚B同我checkin项目前前后后的经过,B的骄傲、我为表姐出气以及后果等等。表姐们是我中学黄金时代里爸妈缕缕拿来鼓励我的武器,我眼中的表姐是不可亵渎的,心高气傲的自己又怎么可能看着B让表姐那么受气?

我是非常佩服表姐是有着极大的魅力的。我是个迷信的人,便也不会免掉去查表姐的星座。星座上说,惊慌是表姐这个属相星座人的主要情绪特征,"他并不善于变通,所以当其遇到突然的转变或迅即的新事物时,他会立刻转向各种方向。摩羯肖兔者为无谓的事惊慌。"跑一下题,也写一句摩蝎座。这种个性的人在规避潜在的危险时所采用的作法就未必是他的真实想法与初衷了。那牵涉进来的人、事、性格存在,比如相对肤浅、出口必黄、走哪儿黄哪儿的水瓶座,还会那么理直气壮地认定一切的发生都是那么理所当然呢(是否,能够联系上下文环境,稍微深入地回想一下一些话、事情究竟是在什么境况下才说出来、得以发生)?(smileface)


\section{实习(81) }
\label{sec-5-84}

我想,如果时光倒流,那天傍晚、那种情境下(就算冒着项目可能会出问题的危险),我还是会为表姐出头去交项目的,因为我不愿看见惠质兰心、心平气和、潜心钻研的的表姐因为一颗石子崩入平静湖面而打破宁静、泛起惊恐滴涟漪。

如果借用那句著名的话,"因为懂得,所以慈悲",我想一定会有人说我自恋的,我也不敢承认自己能真正参透这句话,所以还是用自己的话说吧。一年365、366天,这个丫头偏偏就出生在了细心观察、分析以及评论人性的这一天。星座上说这一天出生的人对别人的评价令人可以真正看清自己的性格。所以用自己最真诚的话说,我想,我并没有先天下之忧而忧、悲天悯人的文人情怀和境界,我所有做过的能做的也只是当我能感受到、看清、明白别人在痛苦中挣扎时,希望自己的一个抚慰拥抱、一个举手之劳可以让别人解压释怀或者开窃、过得稍微舒坦一点儿而已。

不是说爱情里最重要的一点就是要能够理解、包容对方的缺点么,在我看来,友情不也一样吗?看到表姐这样一个小缺点的自己,真真切切地感觉到表姐不再是高傲不可一世的、不是那种大神般的存在近而只可远观、不可亲近的,那是一个有血有肉、实实在在滴肉身,有着你们都有的这样那样的缺点,但也就是这些这样那样的缺点,让这一切的发生来得这么自然、真实、充分感受爱情友情亲情的每一次脉动!smileface 我为自己在B的挑衅下为表姐出过气感到骄傲自豪!


\section{实习(82) }
\label{sec-5-85}

姐姐们都结婚很早,大姐家的孩子已经快要大学毕业了,二姐家的过去这一年正上高三。姐姐大概也受启迪于我当年高考的教训,也希望她的孩子在这样人生至关重要的一年里能够充分享受到父母的爱吧,姐姐休了一年的假,全程专业照顾她的孩子高三学习。小侄女早上上学时天已大亮,自己走路去学校,其余据说是早中晚一日三餐都由家里送到学校食堂,她会在那里、姐姐看着她吃完、陪她聊一聊学习生活中的事,等她吃完再回家;晚上太晚怕不安全,姐姐姐夫也会去学校接侄女回家。第一次听到妈妈给我提这事的时候,替小侄女高兴的自己眼泪就崩崩地滚了出来。

姐姐家离学校很近,这一年妈妈也一直在姐姐那里休养、顺便稍微帮帮姐姐的忙。姐姐会送早、晚餐加晚上接侄女回家,妈妈会帮着上午去菜市场买菜、外加送午餐到学校。因为姐姐也还是多多少少有点儿自己的事情需要去忙的,比如想学车,可能以后早晚还是会有想买车的打算吧,哪怕是买二手车呢?

这个暑假来加州前一个周时,我早就同姐姐约我,我想在自己出发前同大家视频一下看看大家、看看妈妈。很久不见,妈妈脸上的皱纹都早已爬上了额头。这一两年里,因着生活的变故,妈妈早已从那个很显年轻的中年母亲变成了位老太太,昔日臃肿的身财也在照顾爸爸意外的岁月里、也因着她自己的病故,现在瘦只剩不到90斤。我看着妈妈,心里很是感慨,摄像头前,怕掩饰不了自己,便挪到床头,借着扭头的间隙让枕巾沾去泪水。妈妈对我讲着给爸爸烧周年、清明回家时亲朋、姐姐们都回家去的情景境况。妈妈不紧不慢地讲着,我良久没有话说,只能任凭脑袋不停地扭动。


\section{实习(83) }
\label{sec-5-86}

同妈妈聊完后,姐姐自己陪我也拉过些家常。快要关视频下线了,我衷心地对姐姐感慨了一句,"有你们姊妹在家照顾妈妈,我在这边真的是心里好放心呀(实在是太感激你们了)!"这该是我这个不懂事的孩子这些年来第一次对任何一个姐姐讲这样的话,摄像头里的姐姐笑得花枝乱颤、忘乎所以,这边说这话的自己却禁不住再次眼框湿润。

唐立淇老师说,"回顾2013,狮子们真是经历了一场风雨啊。无助,慕名,看尽丑陋,也算是难得的一次"利空"。就当是能量平衡好了,让羡慕、嫉妒、恨自己的人,有机会发泄一下倒也不错。" 唐老师这句话是写2013年的,却很适合今天傍晚敲字时自己此刻的心情。

还是借用唐老师的话来说吧,回顾2012,这真是一段很值得纪念的成长史,上半年已经取得了一定成绩。火红的好运,让狮子看似无往不利,但变化的种子,也往往在无往不利下滋长。从五月开始,我便成了那被修理的受害者,因有了地位气势而致之。"上半年顺势斩断过去,下半年深沉的内部整理。"

回顾自己这新专业第一学年,落霞与孤骛齐飞,秋水共长天一色。这一年里,失魂落魄过、魂飞魄散过;受过老师最严厉的打击、也接的有事后的隐形表扬;默默地努力过,小有成绩;也有眼睁睁地期待着AI拿A却只得B的失落。过去的那个寒假是没有方向的,换来的是春天学期里,逃避去想任何结果的默默努力,因为天下所有的人都知道,努力不一定能取得成功,但不努力一定没法成功!


\section{实习(84)}
\label{sec-5-87}

如同早前一个周已经计划好了,我上一个周末的时候已经准备好了干粮、备好了药材、理好了头发,就等好好休息一晚第二天就可以开车返校了。这条路我已经来来回回走了很多遍很多遍,但唯有这一次的走得这么惆怅。开过了680、505、来到 5 号,我对于自己实习最后一个周、实习最后三个小时的战斗成果都还是很满意的,至少这一次我没再作缩头乌龟!回忆过了这一两年,也真心感慨,这一年对自己来说也真是一段值得纪念的成长史!但这一路上我心里其实始终都在想着一个问题:整个暑假状态都极好的自己,为什么最后一个周就过得那么疲惫、狼狈不堪?我知道是工作不累人人杀人的人际关系让自己困扰,可为什么我会把它们累积到这最后一个周呢?

我的脑海里过电影般反复重复着最后一两个周的细枝末节,去想自己表现得可圈可点的地方,也会去思考做得不足、需要努力的地方。想着最后关头,C的那句信口盖过来就是一句乌鸦嘴的话,还实在是禁不住心里有气;可A为什么就不帮我在他的机器上测试呢?去回想自己的项目,突然就想起,最后关头,因为急、因为赶,我还是犯了一个小错误。A让我把我最后一个项目放在什么什么目录下一个新创建的文件夹里,我测试时在sys.path("…")的路径里加了那个文件夹的文件名,但当我checkin交到subversion时,我忘了在路径里加那个文件夹的名字了(from …x import …)。其实真正诚实地说,当A要求我新建一个文件夹的当时,我当时就有想到我的module导入的时候,我是需要把新文件夹的名字带上的。但因为后来因为太忙太赶还是出错了\textasciitilde{}~ 心里万分后悔,在当时想到我要把它们装进一个文件夹的时候,思考想到这一点儿的时候,我就应该用笔把它写下来、记下来以免以防万一自己会忘记的,但自己当时的觉悟还没能到这一点!

就像后来的自己在QQ群里,其实感概小伙伴们的讨论还是很有些价值,能够帮助自己与同龄人去比较交流,找出自己比较弱需要加强的地方,所以从6月份我就开始存每天的QQ日志,但那天的日志我还是没能存下来。如果早点养成每次登录就能记录一部分的习惯,那我最后阶段同一位对我非常重要的人的聊天记录就不会丢失,但我还是失去了那份纪录,这对我是非常重要的,当然我也还是可以争取更大的胜利的smileface  \textasciitilde{}~当然,这是后话。

回想自己最后三个小时的状态,我以为自己没有遗憾,可还是有的啊,至少那个我是会做的,却失误掉了。回想到之前的失误,我开车的心情开始变得阴郁起来。


\section{实习(85) }
\label{sec-5-88}

其实我一直都还是没有想明白倒数第二个项目A说还有一个地方需要修改一下到底指的是什么?是因为.csv的那一条命令的话,它丝毫不会伤害到我,因为这是我们之前早就讨论过的话题,不是项目的不重要。那如果不是那一条命令,又会又可能是什么呢?

长途是漫长的,但是借由着时间的累积,还是不知不觉地开到了下午傍晚,一千miles的长途也就开到了中点。是我一直在想问题、分心了么,还是也是开得太久,人早就已经疲了,眼看旁边一个牌子写着45我瞬间就踩刹车迫降了下来,怎么后面就马上跟了辆警车?没办法,赶快把车停路边。

警察说根据他的纪录,我在第二个45限速牌后还开了61;我向他解释说这个小镇不明显,我是刚一看见这个45就降了下来(你可以测量现在停车处与这个45牌标处离得都不远),但我不知道你在停在这里等我;我解释我因为车的cruise系统坏了,我开是一直踩油门的,如果偶尔不小心速度升到61,根据车自带的计速表、根据GPS上的速度,一旦发现自己超速我都会很快降下来的;我甚至对警察解释说,我之前超速过,但这几年我已经彻底改变了,2012年7月底回学校时我全程没有超速;今年暑假我来加州时也没有超速过,这次是意外,希望他能帮忙理解通融一下\textasciitilde{}~

但是任凭我怎么说怎么解释,警察还是执着地给我开了罚单,他只对我说我可以选择交钱,也可以选择上庭或是写邮件,可是邮件管用能起作用么,上庭这前不着村后不着店离自己的学校到加州四五百迈的地方我要花怎样的代价来上庭,上庭还不如不上的呀?!!警察可不管这么多,给我开了罚单就走了。我除了开车上路,还能怎么样?开车前看了一下金额,\$160,他们以为我的钱是大水冲来的?


\section{实习(86) }
\label{sec-5-89}

我以为我可以坚持着边开车边调整好自己悲痛的心情,但这一张超速罚单就成了压倒骆驼的最后一根稻草,数日来地层里汩汩翻腾着岩浆在这一刻就真的火山暴发了。

滚滚落下的泪水已经严重扰乱了我的视线,我开不下去了,很容易出车祸会死掉的。我把车趴到路边,自己也趴在方向盘上,像一个走丢迷失的孩子、失声大哭起来。

我好歹也算是一个带点儿灵气的孩子,尤其是经历过高考的磨砺之后(它迫使我探索内在)。就像之前2012年5月有那么一刻,我会因为经受不了与舅舅关系的考验,觉得自己的人生就要跨掉了;在这一刻,随着一年来深沉的内部整理,成长的脚步早已逼近,就算是被迫成长,也不得不长一定得长!随着这个暑假最后两个项目也都一个一个带着瑕疵,这个实习的已经彻底落败,生存的压力就像刺骨的寒风再次侵袭。脆弱的人际关系,我要如何提高抵防?

这个警察不也一样么?我解释得这么清楚透彻了,为什么他还要那么执着地给我开罚单?在这45之前就是55的限速啊,正常情况下超5迈都不会抓,我第一次送开车来加州的路上舅舅对我说过这是不成文的规矩。那我61迈没什么不同,而且他自己就等在第二个45牌的有限距离内,这点儿分歧,他自己就没有意识的么,他给我开罚单他就安心、心安理得么(难道他也要受manager指挥)?可我那春天出庭结束后的\$1300的罚单呢?有谁会真正站在自已怕立场上替自己考虑呢?


\section{实习(87) }
\label{sec-5-90}

不是别人就消极就弱就出错,看人只能看到别人的缺点,昨天晚上,对看不转这些人际关系的表姐,我不是以早就已经以最大的善意向表姐讲述讲解了这来来往往、因果照应、讲完了几个轮回了么?可表姐的惊恐疑问让我没法不去重新斟酌评估评价C的各种捣乱。

有了C两个周前的捣乱,就不难理解一周前的周五她会说下个周见不到我的话;也就不难理解最后一个周表姐的各种神叨叨地劝说,不全是C的主意么?同自己向她的问话环环相扣,巧合得天衣无缝!有了来自policeman的一次又一次的罚款,有了来自于C的这层层罪恶,也早有了自己之前的两个team两种strategy的特殊关系的理解,至此我已不忌惮以最坏最大的恶意来揣度这这缤繁纷错的实习,这叵测的人生!

凭借最后两个周自己觉悟的恢复、意识逐渐清醒,以及周五A帮我fix一个bug前前后后的事情,此时的自己就该非常明白,上周四下午同A吵过之后,我checkin VDBench项目之前,B交她的项目就一定是C的主意了,B不会傻到自发地想在我先交,而且她们俩个无时无刻不在talk,那C要B交项目的旨意便正是逼我交项目!我以为自己赤胆忠心、设身处地的在为表姐着想,却不折不扣地走进了别人设计的陷井!B应该不明白C要她交项目的真实目的,但相对于表姐的固执可靠稳定,B便有当年交际花般的圆滑八面玲珑,与C沟通极好,就像国庆表姐时她会向C传信息!但人的傲慢是很容易滋长的,她原本按照C的要求交项目就该完成任务了,但后面的凌乱就该属于她个人行为,因为manager那天早就走了,她自己也一直就有这样的历史!C没有料到B会骄傲,B没料到我会为表姐出气,我想不到C的tricks,于是那天晚上大家就一起凌乱了!


\section{实习(88) }
\label{sec-5-91}

我知道自己是为表姐出气交的项目,还没有同A下这样的决定,所以第二天就表现得很乖,就很希望如果我有什么做错的地方A能帮我指出来,我还可以改的!前面忘了提,其实这天我有从subversion修改一个python文件,因为有一个indent我只空了三个space(正常应该需要 4 个),当时就有过舆论说我这是在暗讽B,就算别人只改一个空格也只修改了一个文件,哪里犯得着像B这么如此这般大刀阔斧地乱搅?也难怪site里善良的人们要去同情她了,但实际上我只是追求正确罢了。

那A呢,第二天的A为什么就只字不提具体那个要改的地方是什么呢?说起来也很奇怪,心理学家应该去研究这样一个现象,刚来美时在舅舅任职的学校官网上搜名字时我没找到,但一年后等我恢复,"eecs"就像一串神奇的密码蹦入脑海;同样的事情发生大哭时,这次的密码是"\textunderscore \textunderscore init\textunderscore \textunderscore .py"!回想A的framework里很多层文件夹可见的这个文件,对照C的随口一句恶咒我只能在local机器上行得通,外加A推诿说他的电脑缺一个文件,那我的这个文件夹里需不需要这个文件?

前面没有解释清楚,因为对生还有留恋、没有勇气去死,我高考前之所以想要离家出走,是因为认识到自己撒谎了,思想滑波,觉得自己一无是处,同任何亲人相处都会给别人带来痛苦,所以,为减轻对亲人、亲朋好友等关系至近的人的伤害,对我来说最好的办法就是远离他们,一个人到一个陌生的地方去流浪!当然这个想法是当时自己的秘密计划,若是那时有一个人挑明我想离开,也是因为我爱爸妈养育我的这个家,爱自己的亲人,我是否可以打开心结(这是后话)?今天大哭的自己险像极了当年的自己,还禁不住地泥足深陷、痛苦沉沦!


\section{实习(89) }
\label{sec-5-92}

好,我回去一定要仔细查一下这个细节。那就算我这最后一个项目还真缺一个文件,可我从来都不曾真正涉及到这个文件、这个版块的呀!这时骂人的心都一个头10000个大,痛苦呀痛苦、不服啊不服,那就让暴风雨来得更猛烈些吧!

那既然这个神奇的密码已经出现,它会成为别的项目的问题么?第一个不会,因为MSTK这个test suite是A早就搭好的,我也只是修改了一些他的文件,在现在的牌块中插入必要的文件,插入这样的部分内容就可以了,那其实项目呢?是的,除了最后一个项目,它就一样也成为了倒数第二个项目一一VDBench的问题!我的这个项目需要这个文件么,不需要么?它为什么不需要,不需要的话除了local机器上我指明了路径能测,到其它机器上真的能测么?C最后打击我的原话是怎么说的来着?我以为自己可以在自己Windows机器上测一下就可以了,我以为在自己机器上测过就可以,原来我真的需要在lab里好好仔细测一测的!

这个细节回去后一定要好好查一查!那现在的问题就是,上周五我checkin那个修改一个space的文件都交过一次,那么A为什么不帮我及时指出来一一尤其是当这个问题一定会出现在我下一个项目里的时候?再回想一下C随口而出的那句最后时刻还要打击我的话,这是C早就知道的?这是A早就知道的,这是A故意如此设计的?Ooh my God!  这里是平原,如果路边若有一堵墙,我一定去撞死,活在这个世界上太悲哀了!


\section{实习(90) }
\label{sec-5-93}

A怎么可以这么对我?A怎么可能这么对我?A不知道这一点,不,A不可能不知道这一点,C也不可能不知道这一点!如果如同我之前所猜测到的,如果背后真存在着一个时刻监督着我的队伍,那他们、C、site里的各大头目们一定是知道我上一个项目的问题的!那如此做法,就是为确保我最后一个项目一定接着出问题,无论如何,逃不出他们的手掌心?

A这么做是为了升官发财、平步青云?No! stop it,太恐怖了!我敢再往下想,心里有一个声音本能地告诉自己,不是A,一定不是A的本意;他这么做,一定是有苦衷的,是C授意B赶交项目的逼迫让A明白要那么做的!最最关键的,自己当时竟然不知道、意识不到我应该去反驳的!A是无辜的,那C呢?C从来都是就是这么对待自己的啊,这一环一环还有太多的步骤,C的做法有太多的不能理解,从最开始无视我们的存在,到最后想法设法加害于人!C是同B一样世俗的人,C又何至于能够把人掌控、操控得这么完美、天衣无缝?还是这个傀儡manager的背后,隐藏有真正的高人?我的想法吓到自己,我不敢再往下想下去\textasciitilde{}~

这一年里我得各种B就真的只值B,活该么?一年前的五月,我清清楚楚地知道感觉到舅舅有要我回去闹的意思,但我不敢、怕自己输不起;一年前我留下的时候各种犹豫,舅舅和表哥都不表态、不发表任何意见,那现在、一年后当自己毕业的时候,我同样没法去丝毫地依靠这样一个人,爱了又如何,不爱又如何,这个世界上我始终都只能一个人孤零零地去承担一切、依靠自己存活!

这里是横垮大平原的Freeway,车道上的车呼啸而过,我在车里放声大哭,思想思维走到这里,天无活人之路,梦想让我一次次去努力,却又让我一次次去挫败,我心里能感受的只有万籁俱静的恐怖,仿佛<电锯杀人狂>里的女子,如果QQ群里大家刷屏,"new grad哭昏在厕所",我凄切悲怯、哭天怆地、惊天地、泣鬼神!如果我生在早前的时候,我一定比孟姜女早上千年就已经哭倒了长城!这个人多想逃离、超越这平凡的世界!


\section{实习(91) }
\label{sec-5-94}

相比之下,E生得多么幸福!自始自终都有最好的mentor!始终都有着board里、善用各种攻心计的的舅舅罩着,有着尖人这样的贴身保镖全程护卫!在我还没有意识到自己有错误、犯过错误的时候,site里最大的官就安慰、抚慰过我说实习生suppose to make any kind of mistake they want! 表姐不愿意从technical层面帮我,可自己最困难的人际关系上表姐却无能为力、完全帮不上忙!而E不仅自己可以慢慢地做项目,还有着A与我各种失误教训的警醒,比如国庆节表姐走后B向A传信息的时候他就可抓住要点;那我最后一次DEMO的时候他一定可以千真万确清醒地认识到、明白别人是在抓把柄!这样即便A最后两个周只交他最后一个项目,他只要保证做对就可以了!

我到最后走的时候,连平时同自己最亲密的小伙伴都不敢送自己出来,就C一个人把自己赶出来了;小伙伴车里、吃中饭的时候还在开玩笑说的"bear hug"在这种情况下又怎么可能?别人还可以举国欢庆、到户外去欢庆,让所有的人都知道他们是粘了E的光才得以出去玩的\textasciitilde{}~ ;)

哭了大半个小时吧,哭过了最为悲怯的部分,终于哭累了,自己的情绪也开始慢慢平复。当我渐渐平息,那现在,对于自己的项目所出现的问题,他们又会怎么处理呢?因为自己在路上哭一场,他们就像最后一周自己的bug被无限放大一样,自已项目的问题是会被C无限放大,还是像上周四吵架时A提到过的,让E来发现我的问题,为E的提升提供有利电梯?这些,都已经与自己无关了!


\section{写了这么多,不容易…纯写实?写给自己看还是给大家看呢?如果给大家看的话给人物起个名字吧…}
\label{sec-5-95}

夏老师好,

是纯写实,主要还是吐槽吧,忍耐太久太痛苦会憋出毛病来的 smileface

实习的部分复杂点儿,后面的人物就简单多了,我去想想后面的人物该怎么称呼。

我只能讲故事,也很欢迎夏老师以这样的故事或者某些情节为原型,进行深层创作,写出更多更好的作品来。夏老师的作品、写的文我都很喜欢 :

\section{实习(92) }
\label{sec-5-96}

开过长途的人都知道开长途很累很累,很容易犯困,自己中间还又大哭过大半个小时,自已实在是疲惫交加,我不能拿自己的生命开玩笑,所以到半夜已经快到家了,离自己目的地小镇只差二三十迈的地方,还是找了个加油站靠边的位置停下来,把车门锁了,倒在车里休息,想着能睡着几个小时,再稍微开一开就到家了。

不清楚具体过了多久,大概一两个小时吧,有人把我叫醒,睁开眼睛,居然是警察!于是我打开车、摇下车窗,警察说得非常理解,说知道我大概已始开了很久的车了(从我的车牌可以下笔成章称是本地人),因为一般人不会在离家这么近的地方休息了,但他劝我不要把车停在这样一个靠什么墙边,建议我把车停到加油站的一个停车位上去。既然他已经这么说了,那我实际不上是完全可以照做停到那边停车位去的,但既然自己已经醒了,离家也不远,就开车回去吧!

我早就已经同自己的小伙伴联系好,我的新室友那天晚上不会锁门,所以我只要把门推开就可以回自己的卧室里舒舒服服地休息了。(当然,这个租房子的事情还是可以稍微解释一下的,草稿没打好,以后再补。)真正休息过一阵子后,脑袋就真的很清醒了,大几个小时前大哭的基本所有想法都已经灰飞烟灭,能告诉自己的只能是这个暑假这份实习是一块试金石,检验出了很多的问题,最大的问题当然就是学得不深、没扎进去。便也给自己暗下目标,希望新的一年里能有所进步。这也真的是到了成长的一个最黑暗的阶段,漫漫长路,暗无天日,这是接下来专业领域里最为黑暗的一年里我大哭三场的第一场,但专业上真正成长的脚步正在逼近!

\chapter{我与Emacs的不解情缘}
\label{sec-6}

\section{我与Emacs的不解情缘(1)}
\label{sec-6-1}

题目写的是我与"Emacs",实际上,我想写的是我与这个专业的最早接触,与Linux系统的各种小工具的接触与感悟。但司马昭之心,路人皆知,我想写的是Emacs啊!

我与现在这个Computer Science专业最早的接触是在上大一还是大二时学Visual Basic编程基础,分上下两学期。需要提一下的是上学期因为上机时间有限,我自己可能细心程度也还不够,所上上学期期末上机考试时自己该保存的一个文件没能保存下来,还是补考了一次的;这还是给自己敲下了警钟,所以下学期从一开始就开始了认认真真的学习,教我们的老师是从武大来的一年轻女老师,我的同班同学小伙伴们有好几个男生上课时同我一起抢着回答老师的问题,期末考试我的成绩是全班并列第一,大概有两三个同学得同样的成绩!再然后就是国内硕士研究生时老师有讲过数据库基础,但国内农科专业的科普课,讲得非常浅,我也就期末考试考完就将所学的所有知识就立刻全还给老师了!

来到美国这边后,与这个专业最早的接触是在2008年秋天上Multivariate Analysis课时,老师授课用R做数据分析。当时班上有位BCB的生物信息学的小美学生人很nice,常组织我们好几个同学一起写作业(当时还有另一位中国女孩学生与我同班)。他用一个我已经不记得是什么版本的Linux里的Open office写report,用那个系统里的Kite写R代码。同我们Windows系统下的软件界面代码完全没有颜色相比,他的代码从来就是colorful的,于是我就很羡慕。后来某天傍晚当时计算机专业的一位我们称他为晓慧姐的中国同胞在我的求救下,答应帮我帮一个Ubuntu系统,这样我可以写有颜色的R代码!


\section{我与Emacs的不解情缘(2)}
\label{sec-6-2}

前面应该已经提到过了,我用一个五百多元的厚重笔记本,买的时候连无线网的意识都不全,居然没有无线网卡。于是那天晚上我们这位姐就帮我装,装完系统装R,颇费了番周折,我总算是有了自己有生以来的第一个Linux系统了!

因为一个晚上的时间就这样过去,而且这位姐也早就说了他不太会configure wireless,我便拿着自己崭新的系统开开心心地回家了。那时没有其它人可以问,为了自己学习上网方便,就在网上使劲搜,我用的无线网卡是一个外置的Linksys的带加速的什么version的,大概这款无线网卡那时卖得还火、用得人多,所以先装driver什么的,费了九牛二虎之力之后,我居然真的就自己configure能连上无线网了!对我自己来说,在后来的自己看来,那也还算得上是一种搜索和解决问题的能力!

但那个系统我就真的只是用了gedit来编写R代码了,关于那个崭新系统的知识我当时学到的极少(可能那时还很贪玩,没有想学习和深入钻研的心吧)!那时用gedit是一个char一个char的type,一个空格一个空格的indent,可是因为编出来的代码是带颜色的,对当时的自己来说已经是非常知足、简直就是天大的安慰!后来等我转专业之后,这款编辑器就成了我的各种备胞一一只有打印代码的时候我才用它,因为用它打出来的文件比较有型!


\section{我与Emacs的不解情缘(3)}
\label{sec-6-3}

后来12年秋天转专业之后,因为上过一两个周我导师的system software的课,他建议大家使用Linux,所以我回家自己在网上搜索一番后,就把自己的台式机大电脑、笔记本小电脑全都装上了Ubuntu 12.04的版本,笔记本装的时候还crash掉一个月,寄回厂家去估价过,但因为charge太多不想修,后来在闺密朋友的帮助下才最终能让它再次正常运行! 

一开始我们的programming language课讲lisp,而且代课老师也建议说任何一个同学都应该学习使用至少一款coding editor,我就搜到Emacs lisp,因为可以compile。这样在从来没有接触学习过vi、vim之前,Emacs就这样误打误撞地成为了我一生的至爱!这么说,我真正想表达的是,后来我就再也没去学习过vi、vim的用法,Emacs真是我唯一的最爱!还记得早些年自己在网上写与表哥的所谓爱情的时候,不记得哪位网友在他自己的文章里(看见过是因为文章上了首页)提到过,"你当坚信,他永远是最适合你的,永远是你最好的选择,值到生死病死、死亡将人们分开!" 当时的自己读到这么句话的时候,回味自己的爱情,多么感动、多么确信这一点儿!现在岁月又走过这么些年,只有对Emacs,我还有着这样的爱与坚信,它是我唯一的至爱\textasciitilde{}~ smileface

学习使用热键后,我就用这款编辑器来完成我所有的编程工作,C、C++,Java,Python,Unicon,所有的所有的编程\textasciitilde{}~ 没有觉得有丝毫的不便,全然是乐在其中,流连忘返\textasciitilde{}~ 但也要提一句的是,因为我几乎只用这个系统下的编程器,terminal绝大部分时间也只用来compile C/C++代码,所以对Linux以及terminal的开发,对我来说来得比其它同学都要很晚一些。


\section{我与Emacs的不解情缘(4)}
\label{sec-6-4}

这个夏天,"天是蓝的,水是清的,我对你的爱是真的!"这句话如同大家所搜索过的,不是我原创,但我很欣赏那位美眉写她那篇文章所展现出的思维文学才情。

如果此时自己来承认,过去的那一年,即便自己用了一年的Emacs,我从来都是用M-x linum-mode来调出行号,敬请大家笑的时候稍微节制一点儿,笑掉大牙我可不会为你们作赔偿的!其实上CS121的年轻人TA在某堂lab课上有提到过home directory下关于Emacs configure的问题。但当时讲的是系里的linux server,我当时完全没有概念,没有行之有效的学习小伙伴,我的脑袋也就从来不曾开窃过,所以感觉不到自己有任何迟顿与不同,日子还是一样悠然悠然、悠哉悠哉地过\textasciitilde{}~ 再来回想一下刚过去春天的AI课,这位年轻人TA是与我们一起上这门课的,他若没有写decision tree,就应该再没有除我以外的别人会去写了,因为只有我们两个还算是有点儿统计背景的。回想起来,当时自己能够有勇气、敢去这样一个东西也还是很不错滴。

这个夏天的美了来自于 A 的启蒙,也来自于我自己思维的开窃。早在还是跟着B的时候,坐在舒适的office里,有着极好的状态,我就自发地悟到原来我还是可以hard code line number滴!进而那时接下来,我便configure了startup时inhibit startup window;我手动清道夫持续清了一年的三行内容,到这时startup buffer总算启动时可以自动清空了;设置了窗口全屏以及固定大小随我需求;装了autocomplete以及yasnippet自动补全;再后来接python项目的时候,用emacs-for-python IPython打造了了Emacs python IDE,这个IDE的打造因为之前有bug,很费了我一段时间,但到真正可以用C-c C-c execute代码的时候,还是非常开心的。


\section{我与Emacs的不解情缘(5)}
\label{sec-6-5}

忘了说,其实暑假的时候因为是用Visual Studio作测试,习惯了用emacs的再用任何其它的非key-based editor应该都有一个很不习惯的问题,所以就搜到一个revert-buffer,非常好用,windows下,linux下都可以用。我后来就用这个来结合所有的IDE,python netbeans,C\#的VS C#非常好用,几乎万能,所有系统通吃!

以后应该不会再写关于Emacs的内容,所以我就这次一次把想写的写完!暑假对emacs的意识清醒后,回到学校便开始新的color-theme,configure C/C++/C\#的IDE环境,装了GDB,ecb等等,以用后来也曾尝试过Java的IDE环境,但因为bug太多,常常会调整不好,并且真正很难的bug的时候我应该还是会进windows下的ide环境来debug的,所以对emacs ide慢慢的还是兴趣不大。 

后来接下来秋天一门network introduction课,老师说如果同学们作业写得看不清楚的话,他不会给分的,并且建议提倡我们去学用latex。其实这个东西早在上学期旁听导师算法课的时候他也提过,不过那个时候我只听不写作业,大概把这个软件安装了,但却还没能真正去学。那既然这次自己一定得写作业,就去学吧。用了AUCTex,也实现了自动补全。 

记得新学期这门课的老师向我推荐tex的时候提到说,这个东西会上瘾,会让人欲罢不能,后来随着自己越来越喜欢这个东西,就真的到了这个境界了,不仅费了很大的力气终于是configure出了可并存使用的英中文环境,接下来寒假的时候用它编了本连题目带解答的cc150的pdf,春天学期的时候,哈哈,就真滴又穷凶极恶了一回,花了一个周末的时间,从周五下午五点到周日晚上回家休息前,用tex编完了两本书,一本是linux midi function,一本是Windows Dev Center Audio devices的部分,方法是从website把内容复制下来,再在*.tex文件下把它们编成文档,最终生成pdf。现在回想起来都还有点儿不敢相信,太crazy了\textasciitilde{}~


\section{我与Emacs的不解情缘(6)}
\label{sec-6-6}

记得上AI课的时候五六个项目的report我也还都是code在linux下写,report在emacs下写个初稿,如果有图的话就最终再在windows下插入图片等最终完成,因为相比于openoffice,我还是喜欢emacs,习惯了这样一款编辑器。

再后来春天上evolutionary computation课时,代课老师还是AI的老师,授课内容还是类似的形式,要作四五个项目,项目同样不要求code,但需要写report。那时候是春天我已经开始在系里的CSAC里tutor学生,借助在那里工作的机会慢慢地就结识了不少的小伙伴,学习上真正交流的就多了起来。有一次一位我很欣赏的美国小伙伴就帮我推荐了升级版的latex-mode,用org-mode,可以直接在emacs里使用这个mode生成.tex文件,也可以直接生成.pdf文件,就像之前python我可以C-c C-c执行文件,只要C-x C-l p就可以生成pdf,知道有这样一个好东西存在,我好开心呀\textasciitilde{}~

因为该有的软件也早就基本就装了,但是configure这样一个中文环境还是很费了一翻波折。直到目前我也还是没能全面彻底地解决问题,但就像表姐说她之前赶时间尽量能把做、需要做的事情做完,我能做到的是生成一个自己的中英文环境模板后,应付我EC课的report,在emacs里直接生成pdf已经完全满足、绰绰有余了!后来,因为之在太喜欢这种生成的pdf,我把很多网面从网上复制下来后,用这个模板生成了pdf,以后我应该还可以分门别类,把这些积攒下来的pdf在装订成几本书吧。


\section{我与Emacs的不解情缘(7)}
\label{sec-6-7}

一两年的专业课学习,尤其是学了编译课明白了一些原理,下学期的EC课为自己编程给了足够的锻炼,我开始认为其实只要有一门语言自己能够学得深入,那我还是要稍微扩展一下广度来消除自己对陌生领域的恐惧。Python编程我还是得到了不错的锻炼,我欠缺的是java及前端的部分。

Java在上programming language的时候只写了一个关于halilulu的游戏编程;后来上EC课时看见网上一位网友强调java的重要性后便把自己的一个EC项目改编成了java,但这门语言,我也还需要更多的锻炼。这个夏天,本着自己消除广度的愿望,我开始用django搭建html网站,数据库用mysql,html、 css、bootstrap写html template,之前有过的对于html的陌生正在减少,当然,so far,我学得也并不深。如果说之前我所用到的所有的mode都可以自动补全,html里我却configure不了这样的feature,还是比较难受的,比较打击我学习这一块知识的热情(除了css-mode可以自动补全之外,css-mode用了rainbow-mode来显示以"#"打头的各种颜色也还比较好玩)。

Html-mode、sgml-mode的语法高亮、indent等我都喜欢,但试过了web-mode,对于前者在键入"</"后不能自动完成相应的tag还是很失望的,尤其是已经不能自动补全关键词、text等的前提下,于是我还是用了web-mode,虽然现有color-theme这个模式只有两种颜色!后来web-mode下color-theme用了ser  -mode,结合使用emmet-mode有很多很酷的feature,比如用固定的键,在已经输了标志符后,可以自动完成很多个标签,for example,输入"a"后用C-j调用emmet-mode,ENT,就会得到<a href=""></a>,并且此时光标可以定位在两引号之间需要type的位置,这个我还是很喜欢的。


\section{我与Emacs的不解情缘(8)}
\label{sec-6-8}

也是上学期同一个小伙伴帮推荐了zsh于是便自己安装了。当然对于Terminal下的command set我设置了Emacs键,但是觉悟不够,一直用得不多,花得的时间也不多,仅只move这一块,是.forward-word还是 .backward-word就移不通,于是就有了后来在QQ群汗流浃痛地问问题。只是uncheck一个选项的问题,不过还是很感谢那位姐耐心地帮我解答了问题。

后来,认识到自己虽然装了zsh,但对它的开发自己做得不够,后来就将以前自己想往过的复制、粘贴热键功能给做出来了。在Emacs中,C-spc是用来mark的,但Terminal下somehow,it doesn't work。后来自己设置了C-v用来mark,M-w复制,真的做了出来,做出来后带来的便利还是不同凡响的。后来想想,如果我用鼠标从google chrome复制命令来terminal下,我再用鼠标中键来粘贴还是可以忍受的,但如果我在Emacs下coding再转到terminal,我还得去动一下鼠标的话,就很难受了,最终也找到了解决办法。比如,"echo life is beautiful hello world"剪切"life is beautiful"贴到句子最后,以及也贴到Emacs buffer里,可以不用鼠标直接完成!设置完这些,很开心呀\textasciitilde{}~

以后,如果我有机会需要制作幻灯片,我想作为计算机major的专业人士,我不会去用power point,我会像之前那个小伙伴推荐的那样,去使用emacs beamer,相信它给人们带来的便利会像org-mode一样让自己欣喜若狂的\textasciitilde{}~


\section{我与Emacs的不解情缘(9)}
\label{sec-6-9}

差不多去年暑假后回学期也学会了自己写macro,贴个前段时间做题时写过的写lc的模板吧:M-x lc ENT :

\lstset{language=java,label= ,caption= ,numbers=none}
\begin{lstlisting}
(fset 'lc
   [?# ?i ?n ?c ?l ?u ?d ?e ?  ?< ?i ?o ?s ?t ?r ?e ?a ?m ?> return ?# ?i ?n
?c ?l ?u ?d ?e ?  ?< ?v ?e ?c ?t ?o ?r ?> return ?# ?i ?n ?c ?l ?u ?d ?e ? 
?< ?a ?l ?g ?o ?r ?i ?t ?h ?m ?> return ?# ?i ?n ?c ?l ?u ?d ?e ?  ?< ?c ?s
?t ?r ?i ?n ?g ?> return ?# ?i ?n ?c ?l ?u ?d ?e ?  ?< ?c ?m ?a ?t ?h ?> 
return ?# ?i ?n ?c ?l ?u ?d ?e ?  ?< ?s ?t ?a ?c ?k ?> return ?# ?i ?n ?c ?l
?u ?d ?e ?  ?< ?q ?u ?e ?u ?e ?> return ?u ?s ?i ?n ?g ?  ?n ?a ?m ?e ?s ?p
?a ?c ?e ?  ?s ?t ?d ?\; return return return return ?i ?n ?t ?  ?m ?a ?i ?
n ?\( ?\) ?\{ return return return tab ?r ?e ?t ?u ?r ?n ?  ?0 ?\; return ?\
} ?\C-p ?\C-p ?\C-p ?\C-p ?\C-p ?\C-p])
\end{lstlisting}

接下来两个是用来减速少行数的,把“\{”移到前一行的离最后一个有效字符空一个space的地方,其实两个可以合成了一个,写了运行了就那么用了,没有合并的:

\lstset{language=java,label= ,caption= ,numbers=none}
\begin{lstlisting}
(fset 'fo
   [?\M-g ?1 return ?\M-p ?\C-q ?\C-j ?\{ delete return ?  ?\{ delete return
?\M-g ?1 return ?\M-p ?\C-q ?\C-j ?  ?  ?  ?  ?\{ delete return ?  ?\{ 
delete return ?\M-g ?1 return ?\M-p ?\C-q ?\C-j ?  ?  ?  ?  ?  ?  ?  ?  ?\{ 
delete return ?  ?\{ delete return ?\M-g ?1 return ?\M-p ?\C-q ?\C-j ?  ?  ?
  ?  ?  ?  ?  ?  ?  ?  ?  ?  ?\{ delete return ?  ?\{ delete return ?\M-g ?1
return ?\M-p ?\C-q ?\C-j ?  ?  ?  ?  ?  ?  ?  ?  ?  ?  ?  ?  ?  ?  ?  ?  ?\
{ delete return ?  ?\{ delete return ?\M-g ?1 return ?\M-p ?\C-q ?\C-j ?  ? 
?  ?  ?  ?  ?  ?  ?  ?  ?  ?  ?  ?  ?  ?  ?  ?  ?  ?  ?\{ delete return ?  
?\{ delete return])
\end{lstlisting}

.emacs中org-mode configurations的部分:

先配语言环境:

has error here for codes, check \url{http://www.mitbbs.com/article_t1/WebRadio/31233349_0_11.html} 209楼。

\section{我与Emacs的不解情缘(10)}
\label{sec-6-10}

其实以上我已经基本上把自己用到过的比较好的package都列了一遍了,terminal里复制粘贴可以不用鼠标很开心,org-mode还是偶滴大爱啊\textasciitilde{}~

记得二三月份学会用这个东西,我对推荐自己这个模式的小伙伴承认感慨过,不敢相信自己居然曾经有一个周末是用来去干那事儿啦,早知道我就等到我学会了org-mode再去做,怎么也得提高效率四五位吧!对小伙伴承认这句话是在几个月前;现在如果让我重新再来说一遍这样的话,我大概会感慨说,不敢相信自己居然曾经有一个周末是用来去干那事儿啦,早知道我就直接去学python Beautiful Soup等等爬网方法、直接爬网去摘自己想要的果子了\textasciitilde{}~ smileface

我从来不曾认为自己是什么emacs大牛小牛甚至是在行,这也是为什么之前群里有人说过溢的话我不去理会的原因;但这也是一个学习的过程。到现在我的emacs只装了一遍,没有花时间和精力去整理,还有着多多少少的bug,或者某些自己想要的feature出不来;我现在只仅仅停留在把别人试过的packages自己能平平安安地装上用上,写写自己的macro,snippets的小打小闹上;等哪天bug多得不行了,我不得不、必须要重新去再装一遍的时候,我会考虑各方面的因素,比如分模块启动以减少startup开启等待时间,模板间的资源共享(比如不同的模式间可共享的snippets)与某些冲突等等。


\section{我与Emacs的不解情缘(11)}
\label{sec-6-11}

可就算是emacs有常常会有bug,但爱emacs的心从来不曾变过滴啊\textasciitilde{}~ 而且就像老师评说latex,用emacs也是一件很容易上瘾的事情,春天找工作时,想用emacs作类似于excel表格来记录找工作纪录,跑去搜一下就找到一个ses-mode用起来还挺不错的;想用emacs收发邮件,跑去搜一下就搜到成堆成堆的配置\textasciitilde{}~ 看来大家对它的瘾都是相通的,基本我能想到的大家都早就想到并且给出了行之有效的解决办法了\textasciitilde{}~ 可惜春天时功课紧张,收发邮件的配置没弄出来,但想征服它的心从来从来没减过,再过一两个月,新一季找工作来临前,我一定把它给整出来滴\textasciitilde{}~!! 同样受启发于emmet-mode的C-j调用出的诸多snippets,我其实应该把自己写过的几百个针对不同mode的macro改为snippets存放在自己定义的类似于emmet-mode的模块中,方便使用,但这也只是自己目前触角打开、思维的一个方向而已。

亲爱的小伙伴们,受限于自己的知识水平,我与emacs的情缘就只能写到这里了,和这里的任何人一样,在专业领域里,我也从来没有想过去倚赖谁,相信选择这个technical major的大家都有共识,我们最能依赖、依赖得最多的从来都是我们自己!对那些嫌别人笨的、劝退的,我也只能说省省吧!当learning curve到达一个拐点,当专业领域里的触角打开,对陌生领域的恐惧越来越少,我能做想做的也就是深入理解与发掘,把这个坑挖大挖深,深到自己可以有用武之地,目前的目标仅仅只是找到一份工作而已!在这个过程中,就算全世界都不相信我,我还会本能地去相信自己,没人能够真正阻止一个人前进的步伐越跨越大、越来越洋溢着自信\textasciitilde{}~!

就像我相信自己的明天会更好、更幸福,我相信与emacs我还有一段长长的征途要走,因为我还不太理解lisp的那些编程、不理解emacs那些浩如烟海的bug到底是怎么产生的,对某些bug我甚至到了像赤手去拿刺猬般无从下手、束手无策(比如一个select coding system,明明知道是语言、某些特殊中文字符出问题了,却不知道到底该从哪里去trace back这些特殊字符,从什么文件哪个log去找到这些特殊字符,目前尚在顽强探索搜录中\textasciitilde{}~!!),但这并不防碍我乐在其中,热情洋溢、信心满满地去认识和征服这块神秘的领域!相信我还能与Emacs谱出更美、更动人的恋曲\textasciitilde{}~ :)

\section{附}
\label{sec-6-12}

sample snippet

也贴一个snippet示范一下吧,偶从来就不是一个夸夸其谈的人,也省得别人嫌自己吹牛,而且仅仅也只是snippet而已,至于么?(看来偶与某前辈大神较上劲了,对不住了,这里,哈哈\textasciitilde{}~)

在yasnippet/snippets/cc-mode中有inc和inc.1两种,当"inc"时按TAB,会有两种选择:

一种是\#include <|>,
另一种是\#include "|" (其中"|"为光标所在位置)。

我不喜欢这种选择,便单一命名,把inc.1换为为ind,内容如下:

;; \# -*- mode: snippet -*-

;; \# name: \#include <\ldots{}>

;; \# key: ind

;; \# --

;; \#include <\$1> 

\chapter{关于软件工程项目}
\label{sec-7}
\section{关于软件工程项目: 邮件}
\label{sec-7-1}
\subsection{From: 小伙伴}
\label{sec-7-1-1}
Sent: Thursday, June 20, 2013 8:59 PM
To: me\textasciitilde{}~\textasciitilde{}
Subject: thoughpts on our projects 2- JS calendar code

Here is an example of calendar resource code written in JavaScript. It should be helpful.
\url{http://www.oschina.net/code/snippet_994981_22108}

Best,
小伙伴

\subsection{From: me\textasciitilde{}~\textasciitilde{}}
\label{sec-7-1-2}
Sent: Saturday, June 29, 2013 11:12 PM
To: 小伙伴
Subject: CS502 Weekly Reports

Hi 小伙伴,

I have been busy for a while, and now began to think about our projects for CS502 the software engineering.

I want to know if you have sent Jim the most recent documents to him already, or I still need to send both the documents and codes to him tomorrow as our starting point, and we go from this starting point reporting to him every week?

Please let me know your ideas about this and let's get started.

Thanks,
me\textasciitilde{}~\textasciitilde{}

\subsection{From: 小伙伴}
\label{sec-7-1-3}
Sent: Tuesday, July 02, 2013 9:46 PM
To: me\textasciitilde{}~\textasciitilde{}
Subject: RE: CS502 Weekly Reports

Also, I forgot to mention that there is a date picker that already exist. I think it is pretty handy for us to use.

小伙伴

\subsection{From: me\textasciitilde{}~\textasciitilde{}}
\label{sec-7-1-4}
Sent: Saturday, August 17, 2013 9:10 PM
To: 小伙伴
Subject: RE: CS502 Weekly Reports

Hi 小伙伴,

Fianlly got my email password back. Just want to check with you if you are still available for the CS502 project. I was too busy during the summer and have not done anything on it yet, but planning to do something this week. Just want to check if you would join me.

Thanks,
me\textasciitilde{}~\textasciitilde{}

\subsection{FW: Graduation: incompletes reverted, PhD awarded May 2013}
\label{sec-7-1-5}
From: 小伙伴
Sent: Tuesday, August 20, 2013 3:02 AM
To: me
From: XXX
Sent: Wednesday, June 05, 2013 4:12 PM
To: Wan, 小伙伴
Subject: Graduation: incompletes reverted, PhD awarded May 2013

Dear 小伙伴,

Your two CS 502 incomplete classes in Spring 2013 were reverted in order to award your Ph.D. degree for May 10, 2013. The reverted grades were included in the computation of your cumulative grade-point average for graduation.

You may make up the incomplete course work by December 20, 2013 in an effort to raise the reverted grades on your permanent record.

More details about incompletes and grade reversion at the time of graduation are available in Catalog Regulation F: Grades of Incomplete.

Thank you.

\subsection{Sent: Tuesday, August 20, 2013 1:05 AM}
\label{sec-7-1-6}
To: me\textasciitilde{}~\textasciitilde{}

Dear me\textasciitilde{}~\textasciitilde{},

I just moved to VA and I am super busy in preparing for the new semester. I won't be able to dive in to work on the projects at this moment. The email I received claimed that we have until the end of the fall semester to make up the grade. I will be more than happy to work with you in the coming semester once I settle down here. 

Best,
小伙伴

\section{关于软件工程项目:与导师联系}
\label{sec-7-2}
\subsection{From: me\textasciitilde{}~\textasciitilde{}}
\label{sec-7-2-1}
Sent: Monday, August 19, 2013 11:05 AM
To: 代课老师
Subject: CS502 project

Hi Dr. 代课老师, 

Wish you had a great summer. I have not got enough tuition fees yet for my fall semester, and I hope it will be ok for me to come back to school one week after school starts. 

The whole summer I have been busy working one those SAS SSD thing, and having not be able to take care of the summer projects yet. I know you have helped move the deadline late so that I can still spend dedicated time on it and get it done. Thanks. I will be working on the scheduler project for this week, and I will send the codes to you as soon as possible sometime this week. 

Thanks,
Heyan

\subsection{Sent : Monday, August 19, 2013 6:07 PM  (accentually it should be 11:07am because of the time system)}
\label{sec-7-2-2}
To: me

Coming 1 week late will cause some problems.  Are you registered for classes, and will just not be here (because you are spending another week earning money?)

\subsection{From: me\textasciitilde{}~\textasciitilde{}}
\label{sec-7-2-3}
Sent: Monday, August 19, 2013 11:34 AM
To: 代课老师
Subject: RE: CS502 project

Hi Dr. 代课老师, 

I am registering class online right now, and communicating with Dr. (目前导师) and Dr. (AI代课老师) and Dr. (后来导师) so that I can get my course selections as fixed as possible. Some minor changes can be made when I come back on campus. 

So the things for me this week: 
\begin{enumerate}
\item register for full time student;
\item Fix non-thesis research advisor, either Dr. Soule or Dr. Wrinkle
\item got CS502 and CS552 projects done
\item Pay tuition fees that I am affordable this week, some friends will help me on Monday to pay the rest
\end{enumerate}

Yeah i am registering classes to be a full time student, and I am just not be on campus in Moscow (because I am spending another week earning money and getting experience, right!)

Could this way avoid those important problems?

Thanks,
me\textasciitilde{}~\textasciitilde{}

\subsection{Sent: Monday, August 19, 2013 8:14 PM}
\label{sec-7-2-4}
To: me

This looks like a good plan, you seem to have it figured out.

\subsection{From: me\textasciitilde{}~\textasciitilde{}}
\label{sec-7-2-5}
Sent: Friday, August 23, 2013 10:05 AM
To: 代课老师
Subject: CS502 Project

Hi Dr. 代课老师, 

I am sorry to say that I am not able to finish the CS502 software engineering project by the end of summer. I have tried to download and read the scripts for JavaScript-based calendar API, but I still feel lost and I need your close direction to finish it. One maybe good excuse is that I have been working really hard this summer to build and initialize my career. I want to know if it be possible for you to help postpone the deadline slightly late so that I can get it done after I come back to school and have convenient communication between you and me? Thanks a lot!

I look forward to hearing from you. 
Thanks,
me\textasciitilde{}~\textasciitilde{}

\subsection{Sent: Friday, August 23, 2013 5:13 PM}
\label{sec-7-2-6}
To: me

OK. I will delay the deadline. Have you worked with Jie (小伙伴) on this?


\chapter{关于秋季选课与选research导师}
\label{sec-8}
\section{关于秋季选课}
\label{sec-8-1}
\subsection{On Aug 19, 2013, at 10:22 AM, "me\textasciitilde{}~\textasciitilde{}" <heyanh@vandals.uidaho.edu> wrote:}
\label{sec-8-1-1}

Hi Dr. 目前导师, 

I have not earned enough tuition fee for my full semester yet, and I am planning to come back on campus one week after school starts. And I would need the department's help for my spring semester I can not survive by myself. But right now, for fall semester course selections, I need your guide. 

I am interested in CS445(compiler), CS551(Computer Architecture), CS576(Data Mining), but I need to try to select as much 500 level as I can so that I can get graduated on time. And from this fall semester, I would need register 3 research CS580 for non-thesis graduation for Master (I am struggling right now between Terry and Bob which one I should follow). Do you have any suggestion which courses I should take?

I will keep close look into my Email this week, and I would be able to reach email conveniently. 

Thanks,
me\textasciitilde{}~\textasciitilde{}

\subsection{Re: Fall 2013 Course Selection}
\label{sec-8-1-2}
Sent: Monday, August 19, 2013 10:29 PM
To: me

You should take CS507 (Research Fundamentals).  Data mining will be done by long distance with Milos who is in Idaho Falls.  This tends to make the course a bit difficult to understand.  CS445 is a lot of work.  Was it on your list of deficiencies?

后来的导师或是AI的代课老师 would be good people to take a project under.   You might consider that AI代课老师 is also teaching CS507.

cheers,
ex-导师

\section{选research导师}
\label{sec-8-2}

\subsection{From: me\textasciitilde{}~\textasciitilde{}}
\label{sec-8-2-1}
Sent: Monday, August 19, 2013 10:31 AM
To: Soule, Terence
Subject: non-thesis research

Hi AI代课老师, 

I wish you had a good summer. 

In the end of spring semester, I was dedicated to AI and hoping I would have the chance to do my non-thesis research under your direction. But after a whole summer's time spending on SAS SSD, I began to feel worry about myself. 

On the one side, I am dedicated to study and work hard, and I have statistics master's degree, so it won't be too difficult for me to survive for master's degree. But for surviving in industry, I am a girl, and only a master, whether or not I would be able to survive in industry I feel slightly worried. On the other side, for surviving my master's degree, I would still need department or your help (if I work on AI) for spring semester. I can not do it by myself. 

I am trying hard to think a solution out. But I need your suggestion about the aspects that I missed or areas that I am not so confident right now. I need your suggestion and information to help me make a decision. 

Thanks in advance. 
me\textasciitilde{}~\textasciitilde{}

\subsection{Sent: Friday, August 23, 2013 5:31 AM}
\label{sec-8-2-2}
To: me\textasciitilde{}~\textasciitilde{}

Hi,

I don't think you should have too much trouble finishing the MS in computer science.  In terms of industry, I've spent all of my career in academia, so I don't know from personal experience how hard industry jobs are.  However, in general the demand for programmers is high enough that I don't think you will have too much trouble, especially if you can find a job that also involves a lot of statistics.  If you want we can certainly discuss options, and research, further.  My schedule is on-line at www2.cs.XXXX.edu/\textasciitilde{}yyyyy.

Hope you had a good summer,
AI代课老师

\subsection{From: me\textasciitilde{}~\textasciitilde{}}
\label{sec-8-2-3}
Sent: Monday, August 26, 2013 10:25 PM
To: AI代课老师
Subject: RE: non-thesis research

Hi Dr. AI代课老师, 

After another week's struggling, I think I will still go with Dr. 后来导师 for non-thesis graduation options. I don't think I am a research material, and that was the reason Dr. (以前统计专业时的导师) did not want me to do research with him when I was a master student for statistics major. The other reason is that I am in my thirties already, and I have my desire for graduation for life. I am sorry. I hope in not far future, both you and Dr. 目前导师 will be able to meet some talent students who enjoy doing research and would love to spend lots of time with projects. Please forgive me. 

Sincerely, regards, 
me\textasciitilde{}~\textasciitilde{} 

\subsection{RE: non-thesis research}
\label{sec-8-2-4}
AI代课老师
Sent:         Tuesday, August 27, 2013 2:21 PM
To:        
me\textasciitilde{}~\textasciitilde{}

Hi,

No need to apologize, you should absolutely pick the option that best fits your goals.  Good luck with your research and if I can help at some point please feel free to ask.

AI代课老师

\chapter{租房子}
\label{sec-9}

因为这段时间比较忙,来写这第三个学期内容的时候,我准备得并不是很充分,所以写起来时间上就会显得稍微混乱。

其实早在暑假的时候,因为自己需要晚回学校,就打电话回去给学校里的小伙伴请他帮我交学费,得知自己同一专业的一个同学正在找人租出他的房子。他同国内来的一位访问学者同住,但访问学者十月份第一周的时候回国,时间上有点儿不上不下,怕他走了丢下乱滩子不好处理,于是大家便商量着趁秋季入学的时候好找人入住早点儿解决问题。他们的房子离系里很近,只有不到十分钟的路程,LEASE正好也是到来年五月底结束,同自己的时间计划也还比较MATCH(我的暑假一般不在local过的),房租是贵了些,但也算是为自己的同胞解决问题吧,便答应他们让他们帮我找个室友,我帮租出一个房间,这样中国人互相帮忙、互相解决问题就挺好的,也算是与同专业的同学搞好了关系。

后来的事情就有点儿离谱了。首先这位同胞把自己几乎是个人信息贴到了学校公告栏(说学校计算机系一个女中国研究生,几乎可以唯一确定);电话里我就直接跟那位同学说了,明说了我是为帮他解决问题才答应去住的,他这样做不太妥当、有失情份!他在电话里态度很好,我也就没太计较;后来,我回学校前,他问我要一张CHECK,本来中国人这么处事,也没什么大义上的错,但总感觉这忙帮得些许赌心。不过也还算能理解吧,后来我就让帮自己交学费的小伙伴在自己回学校前就先给他写了张支票消除他的疑虑了。

\chapter{落寂归来}
\label{sec-10}

就像大家所看到的,带着缕缕伤痕、我开车回到了属于自己的小学校。见到自己的导师,我对他坦诚地说,"This summer, I have experienced all kinds of failures. "当然,我也并没有完全向这这位老师坦诚自己TECHNICAL、人际关系上的挫败,觉得好像我也还没有必要去向这样一个陌生人去谈论这一切细节的必要。刚回学校时,家里还没装网,周末学校系里各地方的门都没有开,我去书店上网、买水,那里同时也是一家咖啡厅,周围也有不少人在,可忽然感觉身处闹市、内心落寂,分外想念暑假里同自己一起度过欢快时光的小伙伴们\textasciitilde{}~

这个学期,经历了暑假换MENTOR前那么深刻的教训,编译课,就算老师说得再难,难于上青天,我也一定要去选的;这样的课,能有机会去选就已经算是运气了,哪里还会去考虑废掉的可能?再说了,就是因为没有退路的选了这门课,又怎么可能废掉?是的,这门课,就算自己的导师把它描绘成恐怖门、灭绝师太、电据杀人狂,我也一定会去选的!只是我也真没有意料到,这样一门课最终真的给自己带来莫以言状的痛苦心伤与信心上的挫败。

就像大家从邮件里所看到的,收到导师要我秋季学期选RESREARCH FUNDAMENTAL课邮件的自己就像接到一记警钟,这都什么时候了,他们还在打着要我去作RESEARCH的主意?后来我就直接跟了这位上春季给我上RTOS课的老师走NON-THESIS OPTION GRADUATION了。不是因为不喜欢AI的老师,喜欢欣赏这位老师的代课风格是一定的,只是害怕自己陷进了没法摆脱的泥潭。与其那样,不如还是规避这里,执行原计划,争取早点儿毕业吧\textasciitilde{}~!!

\chapter{秋季选课}
\label{sec-11}

随着意识的增长,这学期选课,我就慢慢地有了自己的向往与要求,不再会被因难轻易吓倒。这次回来,我对自己的导师提到过,我不喜欢对学生要求过于严格的老师,会因为害怕抑制抹杀我学习的热情;当然我也不喜欢对学生要求不严的老师,会滋长我的惰性、依赖性,让自己滑身堕落的深渊。最终这学期我选的课程是:Compiler design(4学分),data communication (3学分),computer hardware (3学分),non-thesis research (4学分)。

没有选那门统计相关的课是因为这门课与本校的这门硬件课在时间上是完全冲突的(完全一致);选课的时候我还在外州,只有他们在LOCAL的学生才有机会知道,老师发邮件给他们,说是因为时间上的冲突这门课的上课时间有可能会改变的;我是开学很久以后才了解到这样的信息,但也就因为这样的冲突,我就这样被自己最应该、最值得去上去选的课(因为自己有统计前景)拒绝在心门之外、深深地感到忧伤\textasciitilde{}~

选网络课是因为不想错过垫定这一块基础知识的机会;后来这门课的老师带给自己一样绝美的礼物一一就是LATEX啦\textasciitilde{}~ 选编译课是因为暑假的教训让自己认识到这块基础的重要性,走过了第一学期对新专业的各种恐惧,走过了第二学期写对了作业答案也不知道为什么的稀里糊途,现在原本就到了这样一个该挖深入的阶段,我没有任何理由不去选这门课。但选这课的结果却也是出乎意料地严格,几乎把自己整死\textasciitilde{}~ smileface

\chapter{关于一门统计相关课}
\label{sec-12}
\section{From: 后来导师@gmail.com [后来导师@gmail.com] on behalf of 后来导师 [后来导师@XXXXX.edu]}
\label{sec-12-1}
Sent: Tuesday, August 27, 2013 10:50 PM

To: all the CS551 classmates \& relative instructors

Cc: 后来代课老师

Subject: CS451/551, ECE441/541

Hi All -

I first of all apologize for all the confusion concerning this class. We had to make some last minute changes in teaching schedules, and so didn't have very good immediate solutions for covering all the classes. I wasn't very happy with the arrangement whereby we would re-use previous tapes, but it was the only solution that came to mind in the short time we had.

So, we are going to change things again - but I think this is a much better solution. Dr. 后来代课老师has agreed to take over the class. He is fully qualified to cover the material, having directed work involving computer architecture with the AFRL, and with designing specialized architectures for space applications in his own research with NASA. He will be doing live lectures, rather than replayed video.

I believe that as a result of this move we will go back to the original schedule for the class, meeting on TTh from 11:00-12:15 in the same room - assume that is the case, unless you hear otherwise.

Again, I am sorry for the inconvenience and confusion.

\begin{itemize}
\item 后来导师
\end{itemize}


\section{RE: CS451/551, ECE441/541}
\label{sec-12-2}

(me\textasciitilde{}~)

Sent:         Thursday, September 19, 2013 6:22 AM

To:         all the CS551 classmates \& relative instructors

Cc: 后来代课老师

Hi, Dear Instructors and classmates, 

Sorry for the interruption. My name is (me\textasciitilde{}~), and I am taking the course of cs451/551 this semester. 

I am writing to ask if we could still change the class meeting time for cs451/551. I know it was slightly late, but I want to take the chance to ask and see if it could happen for we have only several classmates for this course. I know it is slightly late, but it is really important to me. 

The reason I want to change the meeting time is because I want to audit the cs576 "Data Mining Topics and Techniques". I got my master's degree in Statistics from U of I in Dec 2009. And afterwards I spent 2.5 years in CA, around 1 year worked in Nielsen/Net Ratings as Research Analyst, 2.5 months in Ask.com as a statistician, and 3 months in PayPal titled Research Analyst III. For this fall semester I have registered cs551, cs445 compiler, cs520 "Data communication systems". (551 and 520 are required for my graduation to make the 9 credits focus.) Beside, I work around 15 hours/week in 后来导师's to cover my living expense. I may not even have any energy to register this data mining course at all. But I do need some inspiring ideas from the instructor to help combine my majors. 

How I wish I were as young as the classmates! But I have wasted too many years back and forth in this university already! I am not sure if it could make any difference by sending this email out, but it doesn't hurt to ask and share feelings. I will talk to Darby and Dr. 代课老师the first thing tomorrow morning, and hoping we could spend several minutes before class to discuss if it's feasible. 

Sorry for the inconvenience and thanks for reading this. 

(me\textasciitilde{}~)


\section{From: 后来导师 [后来导师@gmail.com]}
\label{sec-12-3}
Sent: Thursday, September 19, 2013 3:21 PM

To: (me\textasciitilde{}~)

Subject: Re: CS451/551, ECE441/541

(me\textasciitilde{}~) -

I have some advice - you are already trying to do too much. You have a full class load, and you haven't even finished last semester yet. You started this semester late, so you were behind from the very beginning. You are working in addition. You told me that you would finish the RTOS class incomplete, then start working on an as-yet unspecified graduate project. It is 1/4 of the way through the semester, and I haven't seen any progress on the incomplete, let alone the graduate project. I think you should focus on these projects, rather than trying to do something else.

On a slightly different subject, would it be possible to meet tomorrow instead of today? I could meet in the morning, either before or after my 11:30 class.

\begin{itemize}
\item 后来导师
\end{itemize}

\section{RE: CS451/551, ECE441/541}
\label{sec-12-4}
From: (me\textasciitilde{}~)

Sent:         Thursday, September 19, 2013 7:14 PM

To: 后来导师 [后来导师@gmail.com]        

Hi 后来导师, 

We could meet on Friday regularly at 10:30am. 

I will see you tomorrow. 

Thanks,
(me\textasciitilde{}~)

\chapter{我的室友}
\label{sec-13}
\section{我的室友(1)}
\label{sec-13-1}

因为已经同自已同胞解决好住房问题,从加州开车回去的当晚,我就住到了属于自己的房间。这房子是前房主们住的时候我从来不曾来拜访、不曾见过的;我的新室友是一个美国在校学生,也是之间我从来不曾见过,不知道将来到底会处得怎么样的。但就想着,我住这里,图明年lease结束得早(五月底),图离自己系里方便,可以安心学习;住校园里,只要大家友好相处,再怎么差也不会差得怎么样,差到哪里去吧\textasciitilde{}~

刚住进来的时候,我知道家里有这么个人存在;但他就像一个影子,我只看见蛛丝马迹,却从不曾见到真人;真正见到他已经是几天后的事了;美国人所以比我们要白些,很年轻,只有19、20岁吧。他说他以前住学校community里,像是出去玩集中营那种感觉(比如每天自己的洗刷用品会收起来放到房间里,后来才再习惯放外面)。

他也有好几个major,大概是有一个minor在数学还是统计。接下来有一次聊天我们就聊到我们系,是室友,我就直接对他说我不太喜欢我们系的一些作法,比如这们硬件课他们老师、同学换过上课时间,我当时根本就不知道,到自己统计相关的课都不能选了,完全就错过掉的自己应该好好努力的方向\textasciitilde{}~

\section{我的室友(2)}
\label{sec-13-2}

聊到这个学校里的学生,之前提到的那些个数字便是他告诉我的。他说,这个学校平均本科生毕业时间4.5年,研究生也得两三年吧。我就好奇问他,既然本科生都是平均时间4年,为什么这个学校会成为4.5\textasciitilde{}5年(关键也是那会儿我第一次听说这个数字)。便也就问他这里的学习就没有自己的life plan什么的吗?他们就心甘情愿在这里呆上4.5年甚至更久?室友告诉我说,要知道这个学校学生的入校成绩很低,他们有些人完全没有资格进入本科学习的!

室友的这些话让我想到了我自己。可以打保票的是:在我去年33岁的时候,如果让我知道,没有人能够两年完成这样一个30学分毕业的硕士学位,我会接受现实,回国!在被录取、同自己导师联系秋季选课的时间也还曾经一度想要放弃,我的邮件他4天才回;老师的邮件我也是10天才回,为什么,不就在考虑要不要放弃回国么?那十天里,同一个学校的小伙伴打过很多次电话,小伙伴也多方面解说劝我,我就是接受不了一万多的学费只能上7个不计入硕士毕业学分的这样一个事实!

被室友说他们本科生说得那么狠,我也禁不住去想自己的状况。当时,自己的舅舅、表哥都不曾为自己想读computer science作过任何表态,如果我真的那么差,没有任何基础,他们一一系里的graduate committee里的人怎么就那么放心大胆地接收了我,而且还是在别人33岁的高龄(还在别人至亲的人都不作任何表态的前提下)?这个学校的部分学生不懂life,那他们老师们也是这样,一点儿不懂?

\chapter{八月十五(小伙伴)}
\label{sec-14}
\section{八月十五(小伙伴)(1)}
\label{sec-14-1}

也很久没有同自己的小伙伴们联系了,回到学校,那天是一个什么机会来着,大概是他们non-thesis option的报告吧。那天正好也8月15日,大家也是好久没见,就说那咱大家一起去一个小伙伴家去找他来玩儿吧。到这里就必须得给小伙伴们起名字了。我租房间的前房主,他长得我们四人里最高最壮最大块的,叫他"板块"吧,帮我交学费的小伙伴就叫"(男)闺密"好了,还有另外一个读PHD的中国男生也块,但块头小了很多,就叫他"板砖"吧。所有我们四个人都是中国人,一起聚到了男闺密家里。

大概是去的路上吧,板块说让我花\$20买盒月饼大家吃;我心里不太舒服,于是就没有说话。虽然说我暑假挣了两分钱,但把自己的学费一交,基本就什么也不剩下了。而且男闺密帮我交了\$4000的学费,我都还没有还(我一直是到后来很晚的时候才还了\$1000,到来年春天退税后才再又还了\$1000,到现在我敲字的时候还久着别人\$2000块呢!),为什么他这么明确地要求我还习东西,而且还是在我帮忙租了他家房子之后?

哦,想起来了,回到学校后因为暑假自己租storage unit租得比较晚,我就被逼租到离自己小镇北边20迈的地方去了。所以这次回来以后,我自己去那里搬得搬好几趟,就请男闺密和板块以及板砖一起帮我搬。那次帮忙的事情好像也还是有那么点儿不愉快吧。板砖暑假的时候有帮我从IPO取出自已的CPT并用快递寄出来到加州给我,快递可能也得花\$19.50吧。后来帮我搬东西那天,原本他是答应帮一个车的,男闺密同板块已经同一辆车早到了,我自己开一辆车还在等板砖,板砖要先去加油,我们便去了加油站,不过那里我没有意识到别人大概是需要我出油费的。后来我原本跟他车后要去北方小镇的,他开得实在是太慢像爬一样,我便从车里打电话对他说,另外两个人应该早到了,我不等你了,我自己先开过去。后来我到目的地后,果然他们两人已经在那里等候多时了。装东西时我把所有东西装在两个车里,他们提到还有一辆车在路上时,我就直接拒绝告诉他们说,那辆车我已经不作指望了(后来还了板砖\$20,请他们夫妻吃了餐自家做的饭)。

\section{八月十五(小伙伴)(2)}
\label{sec-14-2}

那天在男闺密那里,同样的同一个专业的三个小伙伴男生和自己一个女生,四个中国人,原本该是多么团结、相互帮助和热闹的事情,但我们却话不投机半句多,没有了任何想再多说下去的欲望。

之前闺密已经因为我向他借钱的问题,我们讨论了半个小时,我同他好说好商量,我先还\$1000,这样剩下一个整数,等我有钱了再去还他;感觉这个说的过程我是那个没钱问别人借了钱的人,这整个过程还是很一段很痛苦的漫长过程;接着就成了板块批评我不感恩,因为我没有顺他的意花\$20买月饼;我就说了,你们帮忙搬家主要还是闺密开的车,你们出了些劳力,我是这段时间紧不功夫请大家吃饭,改天在我家吃饭不就挺好吗,我挣钱也不容易啊(后来我是圣诞节的时候请大家闺密、板块、板砖及其老婆,到我家去吃的饺子,住我楼下的女孩也一起在我家帮忙、吃饭)。我也略有不服啊,便也问他,我不是也帮了你的忙,把你的房子租出去了吗?他觉得我怎么都得租房子的,租他家的房子是顺水人情,我便说了,这一年的房子是我来美国后几年里住过的条件最好的房子了。我这么穷,我何苦何必要住你这么贵的房子,你有替我想想么?

\section{八月十五(小伙伴)(3)}
\label{sec-14-3}

我们聊的话题大概是想要打开我的心结,因为我觉得自己在选课上选得心里发堵。闺密和板块是持同一观点的,他们认为这样一门专业必须得沉得下心来、静静地钻研才能真正学到和掌握知识。沉下心来、潜心钻研我从来都是不反对的(相反,它成了我这个学期的目标,因为A在批评我时,也曾劝导过我同样的话),但我没能选到那门我想上的统计课我还是心里有着诸多遗憾的\textasciitilde{}~ 板块说系里奖学金啊一类的全是main office里的那个前系主任的老婆管事;这话在我这儿就成了一阵儿风,吹过了就没有了,因为我压根就不信这回事!他们俩个都是同其它系读书的时候第一年一学期选CS150 \& CS121,一学期选CS210、CS240 \& CS270选了一年选过来的;但我没有这样的美国时间,而且当初他们导师不让选的话我可以考虑回国的啊,那阵我同男闺密不知道打了多少回电话墨迹这事\textasciitilde{}~

这件话题讨论的结论就成了:他们两个一致认为,我有受迫害妄想症,他们认为老师们应该还是公正的。这已经算是我在学校里最为亲近的朋友了,他们的结论让自己哭笑不得,很是挫败;但他们的观点我无法认同啊。他们俩个,闺密是同实习时组里的D一样人品极好、很能替别人着想的人;就因为他好说话,这个计算机的硕士学位感觉他上的课非常有限,到时得自学多少内容才能真正满足找工作的要求?板块稍显谄媚,很懂得讨好老师,所以可以拿到很多A;我已经很是孤僻了,板砖表态表明观点说,"你的情况可能是特殊一点儿!"板砖大概是那个也经历过的明白人,但他不愿意去说重话刺激那两个人。我心里之前早奇怪过这个人的经历,国同体科计算机,根正苗红的专业人士,为什么硕士毕业后连OPT都不用要直接读博士?不过也想,或许别人年轻比较有理想吧。

今天很郁闷的自己又能怎么样,原本想同大家联络感情的,就只好扫兴自已回家了。记得这样一个日子,是因为到这一刻,我就已经是被这个整个世界给背叛、遗忘了。而且,是在八月十五这个传统节日\textasciitilde{}~

\chapter{关于爱情}
\label{sec-15}

我想我很挫败,讲了这么久的故事、写了这么久的东西,都还没法真正与读者之间建立起充分的信任,所以我所写的、写过的不经意的话都会轻易使风向转变,但比如像是昨天的,写出来的话原本就只是字面上的意思啊,那些都是事实啊(在这样一个国度,舅舅表哥原本就是我的距离三四代的远亲,但在这样一个陌生国度的土地上,他们也已经就是自己最亲的人了啊,在这片国土上,我还有比他们更远更亲的人么?),为什么风向就瞬间变了呢?

我是多么忌讳去写爱情,因为觉得自己太晚熟、真的是不懂,与同龄人相比落后太多了。但就像之前有网友建议过的,生活经历也告诉自己,我大概还是只能诚实如实地写出自己的真实想法应该就是最真诚真实的了,就像几年前写的时候真诚地写出自己对表哥家庭的种种不适应。

实习期间并不十分觉得,但真正回到学校一切都觉得陌生,无比怀念假期里同自己一起玩耍的小伙伴们。这个夏天的实习,我除了得到了充实的知识上、思想上的启迪,每天看着那个无比欣赏、春风得意、挥洒自如的A,我也有了一定程度上的迷失,我也被引入了一个更深的思考话题:这个表哥是那个内心里我真正爱的人么?还是,其实,我并没有去爱过这个人,这只是特殊情境下,亲情上向爱情的一个投影?还是,我真的爱过爱着?表哥的外形也是像着爸爸的,但同我那父爱如山的爸爸相比,表哥在我这里似乎缺少了些精神力量。表哥真的就是那个我想要去依赖的人么,还是这一切的发生不过是在那个环境下、自己依赖性的一个投影?

如果说法律上的隔离只是提供了让我去独立思考的社会环境,那这次暑假实习回来,我开始有了自己去独立思考、试图去努力发现、和认识自己的主观愿望。相信如果那个人是真心爱我的,他真心做出过那种牺牲,他就该宽忍、给自己这样探索认识自已的机会。这一年我34岁,据说这是一个女人最想入非非的年龄。

\chapter{MOUNTAIN BIKING}
\label{sec-16}
\section{MOUNTAIN BIKING (1)}
\label{sec-16-1}

好不容易,这么多年的锻炼愿望终于在这个暑假实现了!刚开学时,还是很有愿望想把自己暑假取得的锻炼习惯成果坚持下去的。结合当地的地理位置与环境(大家都知道我是从小就喜欢骑自行车的,小时候为从父母那里要一辆自行车是需要等很久的,其实至到现在也从来没有真的从家里要到一辆自己的自行车的),这个小镇多山,自已倒也是整个夏天都呆在山林里,再回到现在这个小镇,就很想MOUNTAIN BIKING。

我从CRAIGLIST上看到一辆可以塞进自已车里的小自行车(26还是24,很小很小),也不太贵花了\$25,但是很重。后来真正到山上骑一次才意识到这车好难骑,基本上快有点儿骑不动的,上坡的时候推得都很吃力的。因为自己没钱,重啊不好骑啊这些都还是可以忍受的,最后忍受不了的是,后轴有一个比较重要部位的油轮是塑料飞轮断了这样就真没法骑了。

后来来到小镇上的自行车铺里,别人有零件卖,\$15,但是装上要再花十块还是多少我不太记得了。因为回来买写作业的课桌装折桌子等,到这时我家里已经有些小工具了,后来我就买了零件试图回家自己装上,可还是难度太大了;折磨大半个小时后还是又重新回到了车铺里请别人帮忙装上。几乎就要顺利完成了,这时回来了车店里的老板。

\section{MOUNTAIN BIKING (2)}
\label{sec-16-2}

别人帮我装的时候,修车的师傅已经建议说这车在山上骑不安全,建议我给自行车作一下全身检查更新(人的全身检查尚且没钱,还车呢),我说我骑车时配备齐全,安全帽,手套都备上了,而且我骑的时候会穿长衣长裤这样即使摔倒,也不会太伤着自己,而且我的个性也还算谨慎的\textasciitilde{}~ 

店里的老板回来后,事情就发生了彻底的扭转:因为我花不起、不愿意出\$60给自行车作全身检查和更新(上山骑车只能算是自己刚刚兴起的一个小小爱好而已,万一这个爱好太昂贵,在我没有经济能力的情况下,我还是懂得放弃的),老板命令师傅把即将装好的车后飞轮折卸下来,而且基本对他店里的师傅们明说,我这辆自行车在他店里他们不得给修!

我也是很不理解,不就修辆自行车么,不修也没什么大不了,顶多自己上山跑步好了\textasciitilde{}~ 心里也略有不服,要走的时候便直接对老板说,要么退我零件钱,要么把零件还给我,这零件我可也是花了我\$15买来的。后来他就把零件钱退给我了。再后来他把community里的前学校IPO租自行车组织的"后身"(被另外一个人大概在这个关口附近接暂了)的人找来,希望她能帮我修找一辆安全的自行车。

\section{MOUNTAIN BIKING (3)}
\label{sec-16-3}

不知道同那个新接管的人到底该如何联系,也不知道前店老板那么做的目的和意图是什么,只是更多的觉得很意外、不可理喻。因为我还是想骑自行车,后来在镇上另一家店铺,我已经将自己的自行车修好了,别人大概也只要了我零件钱好像是(不太记得了,即使收了安装费应该也不超过\$5,基本就是像征性地收了一点儿)。

后来community里的人给我打电话留言说,她正好帮我找到一辆合适的自行车,问我什么时候有时间过去商量取。至于一个偏远小镇上的自行车行的老板是如何想到这一点、这一切的,还是学校某些组织早就联系、知会了这家车行老板,我并不清楚,但是到这个时候,我基本也就能够明白别人在建、在打造着文明community,从这方面来理解,就容易、润滑了很多。

而我对我们系里的选课多少还是有着怨念,别人自行车新身已经给出了好意,我愿不愿意接受就成为了一种选择。主观上我是不愿意的,我也就这么一个小人物,我的自己行车也已经修好,我只要能在山道上骑就可以了,我犯不着去扛一个精神文明的什么先锋受惠者。但我也不愿意也做那个恶人,在别人的好意面前显得寂寞猥琐、心怀鬼胎,不敢去欣然接受别人的好意也会成为一种狭隘,于是就接受、同她联系了。

IPO自行车租车新身决定说一辆换一辆,我把自己以前从IPO花\$20使用费(另\$30租金我还自行车的时候还会还给我的)租来的非mountain bike,在她那里更换成mountain bike,这样等我毕业时把这辆自行车再还给他们就可以了。这辆车租个头儿大点儿,但是碳化的就轻很多,骑起来比那辆我花\$25块钱从小镇上别人家里买的山地车要舒服很多。那辆小山地车我这次暑假来时还没有来得及处理,秋天回去后如果没有朋友需要,应该就会拿去Good will吧。

\chapter{打工生活}
\label{sec-17}
\section{打工生活(1)}
\label{sec-17-1}
记得一年前刚回到学校来,还不得不为自己的生活费在校园食堂打工的时候,很多人为自己鸣不平。很感激好心人对我生活的关心,但像我这样普普通通、平平凡凡的人,不打工,我的生活又何以为继?说起来还得感激学校食堂里给了这样的打工机会呢。如果学校里找不到可以打工的机会,国际全职学生又不能打校外的工,小地方,完全没有其它工作机会,又要怎么办才好?

是的,继去年十月份开始在校食堂打工以来,我还一直呆在这里。春天在这里打工的感觉很差,觉得自己在这里受到不少不公正与打击,对自己的形象、人格都造成了伤害;但凭良心说,随着暑假别人给我实习机会,我挣到学费,秋天再回到学校,系里的老师、同学我感觉没有任何变化,但食堂里就真的没有再像春天那么欺负人的呢(前两年出事出来澄清事情的时候应该写过这个,这次就不再多说了)。感觉食堂里,我的打工生存环境好了很多,这个学期,我基本最多的时候都是被安排在CLASSICS,就是美国学生比较喜欢来买东西吃,打扫起卫生等来也不会像沙啦那边事情多又繁忙收滩子的时候也很忙那么辛苦了,这点儿还是很欣慰的。

这个学期这里的打工,我作一三五晚上4 : 00 \textasciitilde{} 9 : 00pm,每周打工15个小时,加上前后走路来这里的时间,以及每次打完工后回家洗澡再重新回到系里浪费的时候,每周怎么也得花掉16、17个小时吧。

\section{打工生活(2)}
\label{sec-17-2}

九月底有一次在食堂里值班(是周一晚还是周三晚上,我不太记得了),那天我是在女生生理期间,但是基本上就快结束了。但是那晚却很是淋漓不尽、出奇地虚脱、手脚无力,站在那里站得很是勉强;自己平时搬得起的一络盘子到这天晚上就搬不动,只能一小叠一小叠地慢慢搬挪;浑身潮热,额头、脖子、身上都微微地冒着汗,很勉强地支撑到了晚上七点多钟(我们正常时间七点半钟shut down,不再服务学生,开始收拾滩位、打扫卫生),有位值班的美国学生帮忙cover了我,我去take break、吃了些东西,才又慢慢缓过劲儿来。

那天晚上是一位腿上略有残疾的女manager 负责管理close down。她还比较体谅,安排相对轻松的活给我,安排我去摆水果,从冷库里拿几箱水果来(她已经帮我把水果搬到了车里,我只推出去摆)摆到外面木箱里供学生们取食。

这次身体上的意外难受同将近一年前凌晨(清晨五六点钟)痛醒那次一样后来不了了之了,但也敲响我身体不好的警钟。后来我自己也上网去查过,大概那天坏在了我去上班时喝了些食堂里的pepsi苏打汽水导致身体虚脱。已经没有了正常人、相比于自己年轻时所拥有的健康,后来自己在这方面就比较注意,生理期间时,一定不喝苏打汽水、一定不喝咖啡、不吃香蕉,只喝热水、热血活血化瘀的食物,比如鸡骨架汤等,保证足够充足的睡眠\textasciitilde{}~

\chapter{第一次陪导师走路}
\label{sec-18}

正如还在暑假时写给老师的邮件里就早已点明,我接下来春天的学费是没有着落的,所以便与学校里的老师们都早早地打好了招呼,春天我可能需要系里资助。这个问题,便也就又一次地成为了我与导师交流的必要话题之一。

那天,我如约地准时出现在导师office里,导师却说他当时正要赶去ADMIN的楼上去开一个什么会,问我有没有时间陪他一起走过去边走边聊这个话题。我正在请教他有关春天自己学费的问题,有求于别人、需要系里帮忙,岂有不陪他走的道理?那天,我穿着比较正式的上班时穿的工作服,长袖长裤带跟凉鞋却略显裙副飘飘。于是我们走阶梯爬过一个小山坡、走过长长的校园林荫路,却感觉有那么几分招摇,又不敢说什么,也就没有多想!走到ADMIN,我大概也有再次提到我是女生岁数大了,不适合再呆在学校里做科研的话,但那天老师去开会前,说出来最重要的一句话是,你得学会去取舍、平衡!我有些发愣,自己懵懵懂懂地再一个人从ADMIN走回来,却没能真正想明白。

这次几乎可以放弃自己的一大爱好也不愿意花钱去修自行车,也正式证明了我没有钱修自行车一一也就更没有可以交春天学费的钱,学校那边大概新一轮的思想工作又开始了。或许也正是这样一次走路,直接导致了再接下来感觉到他们还想要我不毕业再拖一年的贼心不死。某天我在食堂打工的时候同系一个还比较熟的美国学生问我,什么时候毕业,这话题就像一颗不定时炸弹,那天就炸掉了那天食堂里打工的所有好心情。

\chapter{编译课介绍}
\label{sec-19}

过去的每个学期都有一门自己比较投入和喜欢的课,比如第一学期的<programming language>和cs121 C++编程;第二学期的AI课,RTOS也还不错;这学期的重点当然就是这编译课了。

编译课的老师是个身材高大的小老头,也是我的导师,其实春天旁听导师的算法课就是去听他的课了。春天已经旁听过他的算法课,就知道他编程能力其实还是瞒强大的,在系里老师们里面,他可能可以算是数一数二数三的,另外系主任是有钻研兴趣的那种,在工业界有着多年的项目经验,系主任这学期给我们上硬件课,可惜是偏理论,基本不编程;再来大牛自己近年来应该是不再编了,但相信年轻时候还是有着牢固的经验的,不然又何至于做到今天(其忽悠能力过人也是需要肯定的)?除去工业界,我CS210代课老师是搞UNICON的,编程绝对得强大呀\textasciitilde{}~ 但是要去评价我导师的讲课能力,frankly speaking,他没有AI老师、没有自己那种直觉的分析能力,所以很多时候难免就会说弯说绕,课讲得也还可以,但并不是很透彻,所以有些时候、有些比较难一点儿的问题我就难免会听得吃力。常常有的情况是,他自己在讲台上讲得得意、投入,苦口婆心、把他自己都讲成老太太嘴巴般成了破布,坐在下面的同学却麻木不仁,未必真正听明白了(至少我常常没听明白)。

也介绍一下接下来会登场的小伙伴们吧。我上课坐第二排中间。教室分三竖排,左右两竖排都只有两个坐位,认识的人里中国学生里只有Ph.d的板砖同我一起选了这门课,板块倒是这门课早就选过了(前一学年跟着UNICON的老师选过了),闺密没选过这门课。这位板砖同学就坐在教室最靠里排的位置。我们中间一排每排大概有六个坐位,我坐第二排靠里第二个位置。我的左边坐的是同第一学期同我一起选CS336我得了C的那个美国女生(那门课得C的记忆还是有着些许痛苦的),她这学期同时也与我一起上data communication课,老师建议用latex后她还是用手写的,不过她倒是字写得清晰干净\textasciitilde{}~ 学习不是很好,至少编程能力不够强,这是她第二次选编译课(前一学年跟UNICON的老师上过的,没过);坐我右边的是一位帅气的、金发碧眼的美国小伙子,个头适中,有着长长的睫毛和蓝色的忧郁的大眼睛,和很锻炼、很FIT的好身材\textasciitilde{}~ 有着这样的同桌也是对自己学习能力的一种激发和挑战啊\textasciitilde{}~! 忘了说,其实这位同桌就是上学期算法课我提到过很 sweet ,在考试前review导师不厌其烦地讲最基础的题目时装笨提问的那个坐我前排的小伙伴啊\textasciitilde{}~ 顺便一起强调一下,那次装笨第一个提问的同学也是位帅气的小伙子,比起同桌来显得更健壮、更muscle一些,后来他在这门课上帮过我不少忙,我们通过很多封邮件,也一起写过两三次作业。

\chapter{第二次大哭}
\label{sec-20}
\section{第二次大哭(1)}
\label{sec-20-1}

写到这个时候,写到最近,我发觉自己越来越写得像是断层岩,因为时间和事情先后顺序上的错乱。而且写到最近,因为自己也忙,几乎也是completely lose track of the feedback,只顾着自己埋头苦干地往前写。

前面提到过,那天在食堂里打工,一个同系同专业的学生问自己什么时候毕业,一句问话让自己那天晚上的所有好心情烟消云散,因为我已经清楚地感觉到系里逼我拖一年的动作正在逼近。不久后的一天,应该是九月底还是十月前几号前后,(小细节上具体什么小事情激发了这次大哭,我也已经不记得了)我就被那种受逼迫的情绪彻底击崩溃了,原本呆在学校学生office里的自己再也呆不下去,快速连走带跑地回家了,躺在床上滔滔大哭。

想我一介草根平民,餐风宿露地消耗着青春年华来到这里(原本想要争取本科毕业就出来,却在二姐姐夫的劝导激压下曲线救国,读完硕士再出来,又晚了四年),环境的彻底转换加上往日的情伤,一时间不知道该如何生活;一年后认了个舅舅,却又走向了另一个深渊;这次原本只是留下来读书,却又被小学校小系死死地压在这里,原本说好的两年毕业,从第一学期选课大牛、导师都知道这事,第一学期同UNICON老师达成默契,两年毕业、不争奖学金,即使老师压我,所有的课都得B我也都忍了,为什么到这个时候还再崩出来这么一出?这样的日子究竟什么时候才到头?我躺在床上被子蒙着头哭,哭半个时哭到自己崩溃的情绪彻底发泄完才慢慢平静下来;等自己平静下来就会去想,如果我下学期一定不延期,我下学期的学费问题该如何解决?

\section{第二次大哭(2)}
\label{sec-20-2}

这已经是自己来到美国这么多年来遇到过的最大的灾难了,这一两年的日子,老师的雪藏,一个接一个的B,我的日子已经是miserable enough了,说好的两年,还要再延期一年,这日子究竟要怎么才能过得下去?这个坎我迈不过去!在诺大的打击和痛苦面前,我还是又一次地想到了亲人,虽然我也还不至于对妈妈讲这种事情,但我还是需要听听姐姐们的意见的。

那时正好是下午傍晚时分,我的电话先打给了大姐夫,打给大姐夫比打给大姐有效,大姐多年来都不太管事儿。姐夫批评我说,有事有话好好说,一大清早就在电话里哭哭哭,哭个什么这是?我对姐夫说我春天读书没钱交学费了,姐夫怔了怔说,没钱交学费就回来,中国这么多人都好好地活着,就你一个人回来就活不去了?听着姐夫说这样的话,我很无语,电话里只能情绪更崩溃激动地大哭,后来电话是怎么挂断的我都不记得了,大概我也还是说了一两句万一没办法我就回去的话吧。

这样一个电话就是一记毒药,打完电话也只能一次比一次更崩溃\textasciitilde{}~ 哭啊哭,等到自己情绪再平静点儿后,就想我要不要给二姐打电话,我想打,因为有着一线希望;又不敢打,怕这亲情的肥皂泡破得一次比一次彻底。当年,考研的决定是我生病时二姐夫一句"你知道什么叫心比天高、命比纸薄么"给激出来的;拿到奖学金准备出国时又是二姐二姐夫坚绝阻止我出来,甚至我还在北京的时候电话里赌气对姐姐说,就算我问同学借钱我也要出去!现在,如果姐姐姐夫还有化腐朽为神奇的力量,那就请点化点化我吧\textasciitilde{}~ 

\section{第二次大哭(3)}
\label{sec-20-3}

打通电话,是二姐接的电话;我还是在哭,但说清事情的原委后,姐姐说,如果是你最后一个学期,那我们就算是把老家爸妈当年的房子给卖了(我四岁那年爸爸领他的沏将班子盖的房子),姐姐砸锅卖铁也得让你把这个学位拿到手啊。姐姐说,就像爸爸在你出国之前交待过你的,任何时候,遇到任何困难,只要保证人还在就是好的!姐姐说,到时候万一你没钱买机票,跟姐姐说,姐姐借钱也把钱给你攒足,只要你平平安安回来就好\textasciitilde{}~ 在上一个电话希望的肥皂泡破灭之后,虽然明明知道二姐二哥作不了主(一直以来都是大哥大姐留在家里给你母养老)去卖爸妈在老家村子里的房子,但还是被姐姐的话给感动的一踏糊涂,再一次激动地一把鼻涕一把泪、哭成个稀里哗啦\textasciitilde{}~

或许真的是老了吧,记得往昔哭过之后也不过第二天就完全好了,不着一丝痕迹;但这次,哭过后这天晚上早早地就睡着休息了,第二天很晚才起床,起床后居然还眼睑红肿,泪痕犹在。

\subsection{Are You Ready For the Career Fair?}
\label{sec-20-3-1}
Career Center [careercenter@XXXXXX.edu]
To help protect your privacy, some content in this message has been blocked. If you're sure this message is from a trusted sender and you want to re-enable the blocked features, click here.
Sent:        Thursday, September 26, 2013 10:04 PM
To:        
(me\textasciitilde{}~)
University of XXXXXX Career Center

Upcoming Events
Job Internship and Grad School Fair
Monday Sept. 30, 
2-6pm, Student XXXXX Center
Over 95 employers will be on campus to recruit for internships, full-time positions, part-time and seasonal jobs. Check out the participating employers.

Employer Information Session
Many employers who are attending the fair will host information sessions about their company and how to apply for job opportunities 
)
\chapter{毕业打算}
\label{sec-21}

既然现在这个关于毕业的问题已经变得如此迫切,我就去找系里的大件了,感觉他应该是系里的主心骨,有什么事情去找他,准是没错的。我就向大牛述说了我的经济情况:已经从闺密那里借了三千块钱还没有还,春天完全没有经济来源,希望看能不能够从系里得到什么经济上的帮助。大牛大概是说他的科研经费也被砍了好多吧,他也没钱,但是他建议我说春天可以part time,这样我按学分交学费,大概只要交两个学分的钱花\$2000多块就可以了,如果能继续在学校食堂里打工的话,生活费基本上也还有保障。而且考虑到我自己特殊的身份问题,大牛也是同意我八月份才毕业的,这样自己转身份就不会有GAP。

虽然自己只是转专业的学生,虽然自己只读了三个学期,但在自己没钱又不想延期的情况下,想来,这已然是最佳的解决方案了。没什么再多的可犹豫的,感谢大牛,同时顺了他的意思,将自己的导师从自己目前编译课的代课老师转换成现在导师一一也是春季学期给自己上RTOS的代课老师。 

这里我也还是有着疏忽的,我没有去问大牛我为什么需要转换导师,只是感激别人为自己指了条生路、留了一条活路,却没去细想他们要我换导师,这背后的动机又是什么?

\chapter{与导师邮件}
\label{sec-22}
\section{与导师邮件(1)}
\label{sec-22-1}

\subsection{hw2 help}
\label{sec-22-1-1}
(me\textasciitilde{}~)
Sent:        Friday, October 11, 2013 7:38 PM(同之前学校邮箱的问题一样,用的应该是北京时间,这里应该是减去7个小时?也就是发邮件的时间是12:38pm ? 参考<关于软件工程项目:与导师联系 >里的时间对比)
To:        
cs445代课老师@XXXXXX.edu
Hi Dr. cs445代课老师, 

My recognizer works for all four(这里应该是笔误,五个文件,因为这次作业总共有6个测试文件) files except strings.c- for CHARCONSTANT. I didn't work on that on any more. 
But after trying to build a syntax tree, I got tons of error/warning messages. Can you help give me some clue how to get the first several error msgs off?

I am thinking I could not completely finish this homework on time for hw2. Using EDT you mean deadline is 2:00pm locally(后来同同学们确认过,2:00pm locally makes nonsense. So the deadline should be 5:00pm locally.)? 
I have missed the non-terminal function definition part, and missed connectivity among all those parts. Can you help suggest a battery-plan for me so that I can make most points in the coming hours?

Thanks,
(me\textasciitilde{}~)

\subsection{Re: hw2 help}
\label{sec-22-1-2}
cs445代课老师 [captainbbbbbbb@gmail.com] on behalf of cs445代课老师 [cs445代课老师@XXXXXX.edu]
Sent:        Saturday, October 12, 2013 3:20 AM
To:        
(me\textasciitilde{}~)
Hi (me\textasciitilde{}~),
             You should probably make an appointment to come and talk about your code.   I can walk you
through it.   You need to get this to work because assignment 3 will rely on assignment 2 working.

\subsection{from:         cs445代课老师 <captainbbbbbbb@gmail.com>}
\label{sec-22-1-3}
to:         (me\textasciitilde{}~) <(me\textasciitilde{}~)@gmail.com>
date:         Mon, Oct 14, 2013 at 9:40 AM
subject:         are you dropping by my office?
mailed-by:         gmail.com
signed-by:         gmail.com
:         Important mainly because of the people in the conversation.

Are you dropping by my office to talk about CS445?

\subsection{from:         (me\textasciitilde{}~) <(me\textasciitilde{}~)@gmail.com>}
\label{sec-22-1-4}
to:         cs445代课老师 <captainbbbbbbb@gmail.com>
date:         Mon, Oct 14, 2013 at 10:30 AM
subject:         Re: are you dropping by my office?
mailed-by:         gmail.com

Hi, Dr. cs445代课老师, 

I will have cs520 exam at morning at 11:30-12:20pm, and after that before class, or after class, I will stop by so that you can help me walk through the (补:hw3) batteryplan. 

Thanks,
(me\textasciitilde{}~)

\subsection{from:         cs445代课老师 <captainbbbbbbb@gmail.com>}
\label{sec-22-1-5}
to:         (me\textasciitilde{}~) <(me\textasciitilde{}~)@gmail.com>
date:         Mon, Oct 14, 2013 at 11:21 AM
subject:         Re: are you dropping by my office?
mailed-by:         gmail.com
signed-by:         gmail.com
:         Important mainly because of the people in the conversation.

I am always unavailable the hour before class

Sent from my iPhone

\subsection{from:         (me\textasciitilde{}~) <(me\textasciitilde{}~)@gmail.com>}
\label{sec-22-1-6}
to:         cs445代课老师 <captainbbbbbbb@gmail.com>
date:         Mon, Oct 14, 2013 at 11:47 PM
subject:         Re: are you dropping by my office?
mailed-by:         gmail.com

Hi, Dr. cs445代课老师, 

I am writing to ask if you have any period open tomorrow so that I can schedule a meeting time with you so that you can help me walk through the battery plan for hw3. 

Thanks,
(me\textasciitilde{}~)

\subsection{from:         cs445代课老师 <captainbbbbbbb@gmail.com>}
\label{sec-22-1-7}
to:         (me\textasciitilde{}~) <(me\textasciitilde{}~)@gmail.com>
date:         Tue, Oct 15, 2013 at 9:18 AM
subject:         Re: are you dropping by my office?
mailed-by:         gmail.com
signed-by:         gmail.com
:         Important mainly because of the people in the conversation.

How about 2:30 on Tues (today)?
If not, how about 4:30?

\subsection{from:         (me\textasciitilde{}~) <(me\textasciitilde{}~)@gmail.com>}
\label{sec-22-1-8}
to:         cs445代课老师 <captainbbbbbbb@gmail.com>
date:         Tue, Oct 15, 2013 at 12:15 PM
subject:         Re: are you dropping by my office?
mailed-by:         gmail.com

Hi Dr. cs445代课老师, 

4:30pm Tues(today) works for me. I will be waiting for you at that time then. 

Thanks,
(me\textasciitilde{}~)

\subsection{from:         cs445代课老师 <captainbbbbbbb@gmail.com>}
\label{sec-22-1-9}
to:         (me\textasciitilde{}~) <(me\textasciitilde{}~)@gmail.com>
date:         Tue, Oct 15, 2013 at 1:14 PM
subject:         Re: are you dropping by my office?
mailed-by:         gmail.com
signed-by:         gmail.com
:         Important mainly because of the people in the conversation.

Too late.   The 4:30 meeting time is now gone.   How about at 5:00?

\subsection{from:         (me\textasciitilde{}~) <(me\textasciitilde{}~)@gmail.com>}
\label{sec-22-1-10}
to:         cs445代课老师 <captainbbbbbbb@gmail.com>
date:         Tue, Oct 15, 2013 at 1:23 PM
subject:         Re: are you dropping by my office?
mailed-by:         gmail.com

Dr. cs445代课老师, 

5:00pm works for me too today. I will see you in your office then. 

Thanks,
(me\textasciitilde{}~)

\subsection{from:         cs445代课老师 <captainbbbbbbb@gmail.com>}
\label{sec-22-1-11}
to:         (me\textasciitilde{}~) <(me\textasciitilde{}~)@gmail.com>
date:         Tue, Oct 15, 2013 at 2:26 PM
subject:         Re: are you dropping by my office?
mailed-by:         gmail.com
signed-by:         gmail.com
:         Important mainly because of the people in the conversation.

See you then

Sent from my iPhone

\section{与导师邮件(2)}
\label{sec-22-2}

关于这门编译课、关于这个学期,我应该会有很多与代课老师的邮件贴出来,那非专业读者就看个热闹,同专业的小伙伴们,敬请与我一起温习编译课、编译原理吧(build a C- compiler)。

之前的邮件都是从学样邮箱发出去收回来的,但不知道为什么从接下来某天的邮件起,老师给我发这封邮件、他自己的邮箱、我的接收邮箱就都变成了gmail的账户。想起来了,之前课堂上好像有一次他提到过他的学校邮箱好像是有什么问题,但具体是什么问题,我们他自己也不清楚。这样我的接收邮箱也就莫名其妙地换成了gmail的账户。而在之前的学校邮箱里,他回邮件后我没有再回,也是因为他把话说得太重了,我只是有小问题没有做完而已,怀疑我第二次作业有严重问题太过分了,所以不太想去理会他。

到这次邮件这里,我就招风耳地听到过导师的又一次洋洋自得。前面导师给出的那两个时间段我选择了后一个,他不是不available,而是他觉得我很很强的自尊心,所以他说4:30不行,但5:00可以这样可以把时间往后面推。知道他是这么想的,是因为那次我正好在main office那层楼,他正洋洋得意地对AI老师说着这样的话。

其实早就听说过这个导师fail掉过很多人,是系里很多学生眼中有名的变态老师。到这时,其实从初中化学老师那里那时就炼就好的好心态(化学老师在办公室里对其它老师说,我又想回答问题,结果给了我机会站起来回答问题又一不小心计算回答错了,他很是不满啊)还在持续也发挥着作用,我只是要上一门课理解一下那些个程序编译的原理,我只要顺利地过了这门课,真正学到知识、炼就了本领,谁去care你到底在胡说些什么啊?

\chapter{参见导师的后果}
\label{sec-23}

\section{from:         cs445代课老师 <captainbbbbbbb@gmail.com>}
\label{sec-23-1}
to:         (me\textasciitilde{}~) <(me\textasciitilde{}~)@gmail.com>
date:         Tue, Oct 15, 2013 at 6:05 PM
subject:         edits we made
mailed-by:         gmail.com
signed-by:         gmail.com
:         Important mainly because of the people in the conversation.

\section{from:         cs445代课老师 <captainbbbbbbb@gmail.com>}
\label{sec-23-2}
to:         (me\textasciitilde{}~) <(me\textasciitilde{}~)@gmail.com>
date:         Tue, Oct 22, 2013 at 3:25 PM
subject:         tar of c- work
mailed-by:         gmail.com
signed-by:         gmail.com
:         Important mainly because of the people in the conversation.

\section{from:         cs445代课老师 <captainbbbbbbb@gmail.com>}
\label{sec-23-3}
to:         (me\textasciitilde{}~) <(me\textasciitilde{}~)@gmail.com>
date:         Wed, Oct 23, 2013 at 3:26 PM
subject:         your code
mailed-by:         gmail.com
signed-by:         gmail.com
:         Important mainly because of the people in the conversation.

我一般都是周二、周三的office hour可以直接敲门去进他office里去问他问题。从第二次作业自己并没能按时完成所有任务,那一个变量的缩进问题就没有时间解决掉,所以后来我就不敢再大意,无论如何要先有一个battery plan出来先,要在能够保证完成任务的情况下再去独立思考吧。加上对我来说,很多问题这个老师也缺少AI老师那种分析、解决问题的锐利直觉,所以很多问题并不是上课就能听得懂、作业拿出来就知道解决办法的,所以去老师的office hour的情况也就多起来。

在早期的在office里见代课老师的结束后,老师总是在见面结束后把修改过后的code发邮件附件给我。

\chapter{编译课的烦恼}
\label{sec-24}
\section{编译课的烦恼(1)}
\label{sec-24-1}

这学期的这门课,除了我已经知道的导师有时候会讲到兴奋处他自己自high,把自己讲成老太太嘴巴,还有了异样的怪异。比如,他讲课是有一个速度问题的,正常情况下是按着同学们能够接受的速度来讲,但偶尔,他就不知道是哪根神经不太对劲,节奏会瞬间突然慢下来,出奇地慢、慢得如此怪异,根据以往的风向,大家就会把账算到我头上(比如第一学期的的CS210开学很长一段时间里,大家都觉得我奇笨无比,因为很久以后我才知道老师有张excel作业考试成绩表格,
给我的前两次作业一次4分、一次8分,满分分别是27分和33分,后来这个我特笨的效果漫延开后,他才把自我的成绩改回来)。

其实我早就说过,同CS210的老师达成不成文默契后,我处于被雪惨藏状态,很是低调,上课一般都不回答问题的,除了上学期快结束时最后一堂AI课向老师表达有想同他作课题的愿望提过一个问题。那现在当代课老师开始人为控制风向的时候,我当然不甘受欺压,就回答了一个问题。那堂课开课时是在讲grammar,我就提了两点,一是当他在白板上写grammar时,"',' should be defined as a token; "然后因为他的语法定义得不合理,有冗余,便装不懂地问到"Why do you need the third grammar rule?"

下课后有同学觉得我很聪明,我自己清楚地知道,我只是在别人施加的欺压面前正当防卫而已。

\section{编译课的烦恼(2)}
\label{sec-24-2}

我总感觉人际关系上,同板砖相比,我还是差了很多。这门课所有五次作业里第一次作业是最简单的。我deadline那天交完作业之后,还跑去老师的office里问了一下我有什么感觉不太确定的地方,板砖知道我有去过老师的office之后,也马上说他也有什么不确定的地方要去问一下老师,那天下午他应该也去过吧,这究竟是什么心理,我当时倒是没有多想。

后来第二次作业我写得相对比较慢,毕竟是一个转专业的学生,没有这么深入地去想过问题,整个暑假受鼓励于A的启发式教育,受鼓励于他支持鼓舞我去独立思考问题,我一方面真的很想自己独立思考去解决问题,另一方面也想从自己之前一贯的太过讲究效率以至于真是没能深入到知识的核心与原理中去的坏习惯中扭转回来,就真的希望像A曾经批评过我的那样,不是总是匆匆忙忙地总想着去完成任务、而是静下心来认真地去想问题,在理解的基础上再去解决问题,所以就慢了很多、慢得理所当然了。

这期间,hw2 deadline之前,老师有让小伙伴们举手表态第二次作业进度到他提出的某处步骤了,我是相对落后中的一个。后来deadline一过大概是接下来的周一吧,老师又强调一下说,有些同学第二次作业的语法还没弄好,建议他们抓紧时间赶快弄懂写好,因为第三次作业是在第二次作业的基础上进行的,第二次若进展不顺,势必影响到后面的作业。

老师说的是那些没做完的小伙伴们(因为到这里我已经独立完成了95\%的工作,只有打印树结构每行前的根据树LEVEL缩进的空白字符还没有想清楚、没能打出来),但因为老师之前追进度的时候我是相对落后的,便也谦虚问老师这第二次作业说,我所有的output都match他想要的结果了,除了(except,打印语法树的那前面根据树的level空的空格没有打印),这样可以了吗(还是我需要作一些其实必要的追踪以跟上进度)?我只是没有想好到底该如何根据树的层次去打印每行开头那些空格而已!亲,悲观的你可能觉得我还是那个作业不够完美的人;可乐观地去想,这只是一个变量根据层次递增递减速的问题,如果我去Google,凭借我强大的搜索能力(还远不是这个专业的学生时已经能把Linux ubuntu下无线网配置好、转专业第一个月内写出500行lisp写出game的一步move),我一定能搜出一种解决方法,而实际上这也仅仅只是设置一个变量的问题,而我谦虚地去提这样一个问题,也并非要寻求什么答案呀,我也只是向小伙伴们证明,我不是老师说的"那些同学"那么差的同学中的一员,我已经基本做完了,deadline交的时候,我就只差最后一步了\textasciitilde{}~ 在老师的高压手段下,我必须强势正当防卫来保护自己少受伤害!

\chapter{第二次走路}
\label{sec-25}
\section{第二次走路(1)}
\label{sec-25-1}

from:         cs445代课老师 <captainbbbbbbb@gmail.com>
to:         (me\textasciitilde{}~) <(me\textasciitilde{}~)@gmail.com>
date:         Mon, Oct 28, 2013 at 3:18 PM
subject:         can we meet at 4:30 not 3:30 so I can go to the seminar?
mailed-by:         gmail.com
signed-by:         gmail.com
:         Important mainly because of the words in the message.

随着老师时快时慢的上课进度,随着感觉自己一次次地被导师操控(比如被要求陪走去了一次ADMIN),随着时间的推移,我能够感觉到老师总是有意无意地暗示打击我学习不好,我的警惕性也都眼着在提高。到此时已然能够看得出,凭借着他自己强大的编程能力,这个老师是极为自负的,而且从他对上课进度、速度的快慢操控,也能够清楚地看得出,这个人也是极有手腕手段的。据说,这是一位在惠朴工作过15年的员工,后来是怎样的锲机让他决定回到学校读博士(进而一直呆在学校里工作),作出如此大的转变,背后的原因和动机是什么?学校的魅力对他来说到底在哪里?我一直很好奇。如此想来这人在惠朴工作了15年,他也未必只是普通人物,那工作15年后又是什么原因使得他要彻底转向呢?

我去自己使用过的两套邮件系统里找过,找不到我在收到这封邮件之前有发任何相关邮件给代课老师。那是白天上午上他的课快下课的时候同他一不小心约成了下午3:30在他office里见面问问题的?也过去快一年了,时间太久,仅凭记忆,我已经回想不起来了。

那天我去找导师之前,我并没有注意到他临时改变主意的邮件。这封我没能及时看见的邮件就直接导致了自己被迫陪着老师走了第二次的路。倒是后来事过境迁的自己禁不住去想,这真的是他"临时"改变的主意么?

\section{第二次走路(2)}
\label{sec-25-2}

所以那天,我按时去找导师,他在OFFICE里等我,他问我要不要陪他一起走去听seminar,这已经是第二次他要我陪他走了,虽然我心里有着老大的不情愿,但不知道为什么我总是不好直接去拒绝他,这样,我就又一次地被他拖着陪他走进了SEMINAR的教室(因为周一下午3:30是系里的cs401/cs501 seminar课),我既然来了,当然还是得听的,我找了靠门左边最后排方便的位置坐下,导师也坐了我前两排靠最左边的位置,侧身坐着方便打量课堂上的学生。我没有想到的是,代倮老师想要去听的一一那天的seminar课实际上是他们senior design课的项目报告,所以有那么四五组学生以小组为单位上讲台作报告汇报他们的项目。

到这时我也终于明白,难怪早上上data communication课时那个编译课坐我左边的女孩子表现得阴阳怪气,原本是今天她有上台表演的机会啊(其实我心里倍儿明白,她编程根本就不强,只是她的代课老师给她分在一个相对强大的组好让她得以完成项目而已,她却在早上的data communication课上兴奋得仿佛她是这个项目的功臣,衣着打扮得很漂亮、平时挽着好看的发赞今天披着甚至是拉直了头发)\textasciitilde{}~

我不太懂得美国人,确切地说,不太懂得的是美国学生。大概就像我从小到大对自行车有着近乎热切的向往(大概是因为从小到大看着姐姐们骑过拥有过新自行车,同学小伙伴骑过有过新自行车,但自己从来没有拥有过任何新自行车吧),他们美国学生,似乎对飞行有着难以名状的热切向往,几乎所有学生报告的项目都是关于飞行(,他们的这份热切,我就不懂了啊)\textasciitilde{}~

\section{第二次走路(3)}
\label{sec-25-3}

我欣赏他们的项目、欣赏他们的报告、表演,欣赏得蜻蜓点水、云淡风轻;犹记得大四下学期跟着微繁的导师作毕业课题,那时导师的老婆一一柳老师有一次聊天的时候说到过,不用去羡慕那些明星,要相信如果从小把你放在那样的环境,你要相信你也可以成为明星,甚至比他们作得更好!那时幼小的心灵对柳老师的这句话心生敬畏,却原来此刻,这份理论此时已经指导成为了我生活的现实。我欣赏他们的项目,但并不妄自菲薄,我相信把我放在同样的课堂上,就像暑假我随A完成的任何项目一样,我也可以、有能力作得出同样的结果,甚至比别人更好,我现在只要做到真诚地去欣赏别人就可以了就好了。

导师大概以为,那些、那份刻意安排(我这学期不上seminar课,几乎没有任何理由我会出现在这堂课的教室里)的同学们对他们项目的表演一定会打击到我吧,只可惜,我的反应吓着吓到自己的导师了,这堂被系里充分准备和安排的项目汇总报告,让导师真切地认识到这个学生与别人有着本质的不同,在他这个很自负的人的眼里,我有了太多过多的自信(汗,我也同样是从ubuntu装无线网卡、配无线网开始,从500行lisp开始,从装崩溃一个笔记本开始,从CS121被一个bug吓得魂飞魄散开始,经历着RTOS2514、AI课上的打击以及DECISION TREE的产生,经历着暑假的实习,一步一步、踏踏实实地建立起自己的自信来的啊,这一路走来,我又何尝容易?),大概也是从这一刻起,他一一我的导师,发誓要灭我的威风、给我点儿颜色看看\textasciitilde{}~!!

\chapter{我的导师}
\label{sec-26}

记得学期开始、前半学期的时候,大概是第二次交作业后我去问他第二次作业还不太确定的地方时的某次,在他office里他帮忙讲题,他有一次不小心碰到我的手!虽然,如果当初表哥常拉我的手,我会开心,如果我现在帅气的同桌来拉我的手,我也会高兴,但被这么一个(心里有着某种不舒服的)老头碰到自己的手,还是很心不甘、情不愿啊,便就瞬间把自己的手躲回来、藏起来!

我来到这里,只是想问问题,把自己的作业写出来、写完,我并没有其它非分想法,很是坦然;他,我的导师,大概也真的只是不小心碰到自己吧!加上心生反感、之后的自己都格个戒备,从来都是把自己的手尽量藏起来的或是保持安全距离的,那种事情就没再发生过。

后来第三次作业,他上课讲得不透彻,看课本也看不太懂,我写作业就盲人摸象了,对于该如何TRAVERSE一棵树,虽然课本上的例子是结合了pre-order traverse \& post-order traverse,但自己理解起来还是稍有些吃力,就这一点儿上,某天我去CSAC问板砖的时候,他倒是还有稍帮我请解,可惜我还是不太懂、似懂非懂,后来就真又跑去请教老师了。

那天,在他的office里,他说我既然不懂那pre-order traverse \& post-order traverse的结合的traverse方法,那就直接一个kind一个kind、一种类型一种类型的来写好了,“This way, you will get the power to control for each kind!”当他说到,我作为programmer程序员将拥有控制each kind的power、力量、权力的时候,他对power力量权利的执着强调,他说这话的欣然向往、如痴如醉的表情,就让自己突然开差恍惚,意识到一一POWER,对,意识到这样一位导师对权力有着无限执着向往的时候,感觉自己就像电影里思嘉丽忽然意识到认识到红土的可贵,我自己也好像就瞬间也拥有了programmer control programs的power。拥有这个power将是一种多么神奇的力量!只是,我拥有热爱着的是programmers ontrol programs的power,导师所向往的更多的是对人、对权利的掌控!

\chapter{第三次作业}
\label{sec-27}

这是一门4个学分的课,每周上四次周,周一到周四下午的某个固定时间段上课,共计五次作业,但每次后面一次作业都依赖前一次作业的结果,除了最后第五次作业不是那么依赖第四次以外(第四次作业更像是眼见着这个班就要在这如山的msgs面前全线崩溃了,代课老师自己赶快跳出来抢险救灾!)。所以任何时候,即使过了deadline也一定要加紧时间把所以能写、可以写的作业写完。

如果说第二次作业有6个test file还算正常,那么现在到第三次作业就有了20个左右的测试文件,有了至少500条warning或是error message。当测试文件的数量大到一定的程度,作业也就已经不再是对基础知识、基本能力的测试,而变成为了对学生作业、投入时间的考量。

如果说第二次作业我还在独立思考着到底该如何去写作业,第三次作业对我就彻底变成了盲人摸象,摸哪是哪儿、撞哪儿是哪儿!还好,后来是在那次“power”问题后,总算想清楚了怎么去架这个traverse down syntax tree的结构架构,到交作业时这个架子已经彻底搭好,塞入、generate了少量的msg,但离完成所有的msg还差得远。

后来代课老师大概也意识到他自己的“变态”,接下来的上课,就尽早赶快安慰我们说,他看了下同学们的第三次作业,不要灰心,还有一次makeup可以catch up,然后把hw3b的要求和评分标准给贴了出来。我,同其它绝大部分小伙伴们一样(像板砖这种土生土长、根正苗红的ph.d大概应该会比我们顺利很多吧),从来不曾学过这么深的东西(到CS121教完,假期VS IDE compile都出错过,可以想像,之前我学得到底有多浅!),吭哧吭哧地往前爬\textasciitilde{}~!!!

\chapter{hw3 \& hw3b与导师邮件}
\label{sec-28}

\section{from:         (me\textasciitilde{}~) <(me\textasciitilde{}~)@gmail.com>}
\label{sec-28-1}
to:         cs445代课老师 <captainbbbbbbb@gmail.com>
date:         Sun, Nov 3, 2013 at 12:23 AM
subject:         Hw3 problems 20131103001216
mailed-by:         gmail.com

Hi Dr. cs445代课老师, 

I have made some progress with my homework 3, but they are still very limited and right now, I am blocked 2 problems. 

\begin{enumerate}
\item How do I set scopes for function parameter declaration list? For example, 
\lstset{language=java,label= ,caption= ,numbers=none}
\begin{lstlisting}
int main (int x, char y){ }
\end{lstlisting}
\end{enumerate}
when x or y has been declared global before this func decl to avoid duplication decl error?

\begin{enumerate}
\item my DeclK :: VarK taverse through all the siblings, 
 my CompoundK traverse through children and potential siblings of each child in 
 (child[ 0 ], siblings of child[ 0 ], child[ 1 ], sibling is child[ 1 ]) order
I have tried to modify from either the VarK side or the CompoundK side, but each has other sets of problems. What is the magic here in order to avoid the duplication traversing or decls?
\end{enumerate}

The timestamp for this one is: 20131103001216

Thanks,
(me\textasciitilde{}~)

\section{from:         cs445代课老师 <captainbbbbbbb@gmail.com>}
\label{sec-28-2}
to:         (me\textasciitilde{}~) <(me\textasciitilde{}~)@gmail.com>
date:         Sun, Nov 3, 2013 at 8:04 AM
subject:         Re: Hw3 problems  20131103001216
mailed-by:         gmail.com
signed-by:         gmail.com

\subsection{Quote: On Sun, Nov 3, 2013 at 12:23 AM, (me\textasciitilde{}~) <(me\textasciitilde{}~)@gmail.com> wrote:}
\label{sec-28-2-1}
Hi Dr. cs445代课老师, 

I have made some progress with my homework 3, but they are still very limited and right now, I am blocked by 2 problems. 

\begin{enumerate}
\item How do I set scopes for function parameter declaration list? For example, 
\lstset{language=java,label= ,caption= ,numbers=none}
\begin{lstlisting}
int main (int x, char y){ }
\end{lstlisting}
\end{enumerate}

when x or y has been declared global before this func decl to avoid duplication decl error?

\subsection{Re: You need to enter a scope when you arrive at function node.}
\label{sec-28-2-2}
Remember that the name of the function has to be in the global scope but the parameters are in the local scope.

\subsection{Quote: 2. my DeclK :: VarK taverse through all the siblings,}
\label{sec-28-2-3}
my CompoundK traverse through children and potential siblings of each child in (child[ 0 ], siblings of child[ 0 ], child[ 1 ], sibling is child[ 1 ]) order

\subsection{Re: When you are at node N you should traverse the children of N}
\label{sec-28-2-4}
and the siblings of N not the siblings of the children of N   That is already handled by traversing the children of N. Each of them will get their siblings traversed after their children are traversed.

\section{from:         (me\textasciitilde{}~) <(me\textasciitilde{}~)@gmail.com>}
\label{sec-28-3}
to:         cs445代课老师 <captainbbbbbbb@gmail.com>
date:         Mon, Nov 4, 2013 at 2:42 PM
subject:         Hw3  20131104143735
mailed-by:         gmail.com

Hi Dr. cs445代课老师, 

I concerned about my flags used in the c-.y file. with timestamp 
20131104143735

, I got segmentation fault tested on c.c-, but not on a.c-. But when I run gdb trying to figure out why, c-.y line 1061 got my attention.  Why am I not able to pass a flag from symbol table to my current treenode pointer?
\lstset{language=java,label= ,caption= ,numbers=none}
\begin{lstlisting}
1061	  t->isArray = ((TreeNode*)(st->lookup(t->attr.name)))->isArray;  // t->isArray = false here ???
\end{lstlisting}

This may also be the reason for my ERROR(60) because of LoopFlag?

Please help me look into this line of code.  

Thanks,
(me\textasciitilde{}~)

\section{from:         cs445代课老师 <captainbbbbbbb@gmail.com>}
\label{sec-28-4}
to:         (me\textasciitilde{}~) <(me\textasciitilde{}~)@gmail.com>
date:         Mon, Nov 4, 2013 at 3:14 PM
subject:         Re: Hw3  20131104143735
mailed-by:         gmail.com
signed-by:         gmail.com

\subsection{Quote: On Mon, Nov 4, 2013 at 2:42 PM, (me\textasciitilde{}~) <(me\textasciitilde{}~)@gmail.com> wrote:}
\label{sec-28-4-1}
Hi Dr. cs445代课老师, 

I concerned about my flags used in the c-.y file. with timestamp 
20131104143735

, I got segmentation fault tested on c.c-,

\subsection{Re: The seg fault on c.c- is caused by the unary minus.}
\label{sec-28-4-2}

A unary minus does't have a child\footnote{DEFINITION NOT FOUND.}.   At line 1105 you ask it to look at child\footnotemark[1]{} without testing if it is NULL.

\subsection{Quote: but not on a.c-. But when I run gdb trying to figure out why,}
\label{sec-28-4-3}

c-.y line 1061 got my attention.  Why am I not able to pass a flag from symbol table to my current treenode pointer?
\lstset{language=java,label= ,caption= ,numbers=none}
\begin{lstlisting}
1061	  t->isArray = ((TreeNode*)(st->lookup(t->attr.name)))->isArray;
\end{lstlisting}

\subsection{Re: If it looks up the name and the name isn't in the list then it will return NULL and you will}
\label{sec-28-4-4}
try to do:  NULL->isArray.

\subsection{Quote: // t->isArray = false here ???}
\label{sec-28-4-5}

This may also be the reason for my ERROR(60) because of LoopFlag?

Please help me look into this line of code.  

Thanks,
(me\textasciitilde{}~)

cheers,

\section{from:         (me\textasciitilde{}~) <(me\textasciitilde{}~)@gmail.com>}
\label{sec-28-5}
to:         cs445代课老师 <captainbbbbbbb@gmail.com>
date:         Mon, Nov 4, 2013 at 3:24 PM
subject:         Re: Hw3  20131104143735
mailed-by:         gmail.com

I get the reason now after reading and checked back. I will write to you if I get blocked by any other issue. 

thanks,
(me\textasciitilde{}~)

\chapter{关于作业( hw3 \& hw3B)}
\label{sec-29}

\section{from:         (me\textasciitilde{}~) <(me\textasciitilde{}~)@gmail.com>}
\label{sec-29-1}
to:         cs445代课老师 <captainbbbbbbb@gmail.com>,
 profcs445代课老师@gmail.com
date:         Thu, Nov 14, 2013 at 9:34 PM
subject:         HW3B OpK
mailed-by:         gmail.com

Dr. cs445代课老师, 

I think I still didn't get the bug fixed yet with the OpK we have worked on this afternoon. 
Could you please help me look into it? I think I need some idea here. 

thanks,
(me\textasciitilde{}~)

\section{from:         cs445代课老师 <captainbbbbbbb@gmail.com>}
\label{sec-29-2}
to:         (me\textasciitilde{}~) <(me\textasciitilde{}~)@gmail.com>
date:         Fri, Nov 15, 2013 at 8:23 AM
subject:         Re: HW3B OpK
mailed-by:         gmail.com
signed-by:         gmail.com

Hi (me\textasciitilde{}~),
             I think the problem is that you didn't declare a new index variable for your for loop 
right after the OpK case:

case OpK:  // + - * / \% < > \texttt{= !} <= >= or not and
            //if (st->lookup(t->string) == NULL)
            //st->insert(t->string, (void*)t );
            printf("OP: \%s\n", t->string);
            for (int i = 0; i < MAXCHILDREN; i++) \{
                if (t->child[i] != NULL) \{
                    printf("CHILD: \%d\n", i);
                    insertCheckNode(t->child[i]);
                \}
            \}

The addition is changing for (i=0 to for (int i=0.

cheers,

\section{from:         (me\textasciitilde{}~) <(me\textasciitilde{}~)@gmail.com>}
\label{sec-29-3}
to:         cs445代课老师 <captainbbbbbbb@gmail.com>
date:         Fri, Nov 15, 2013 at 11:27 AM
subject:         Re: HW3B OpK
mailed-by:         gmail.com

Dr. cs445代课老师, 

This does not completely solve the problem. When I test it on tictactoe, the similar problems are still there. 
Please help me look into it. 

thanks,
(me\textasciitilde{}~)

\section{from:         (me\textasciitilde{}~) <(me\textasciitilde{}~)@gmail.com>}
\label{sec-29-4}
to:         cs445代课老师 <captainbbbbbbb@gmail.com>,
 Cs445代课老师n <profcs445代课老师@gmail.com>
date:         Sat, Nov 16, 2013 at 3:24 AM
subject:         HW3B 20131116031312 CallK walking down Parameter List \& Break \& Return
mailed-by:         gmail.com

Hi Dr. cs445代课老师, 

I think I pretty much have done all the parts that I understand with hw3B. But there are still quite some kind that I am not very clear how to solve it. 

For the most frequent appeared one, CallK for function calls, I don't know how to walk down the parameter list so that I can not only count the list length, but also compare parameter type and potentially recursively call itself. This one appeared several time in the test files, and produced a segmentation fault in ryantest.c-. 

The other miner issues includes break statement, return, and some constant initializations. The most recent one I turned in is with timestamp 20131116031312

Please help me take a look. And I will try my best work on it and turn it in on time. 

Thanks,
(me\textasciitilde{}~)

\section{from:         cs445代课老师 <captainbbbbbbb@gmail.com>}
\label{sec-29-5}
to:         (me\textasciitilde{}~) <(me\textasciitilde{}~)@gmail.com>
date:         Sat, Nov 16, 2013 at 10:11 AM
subject:         Re: HW3B 20131116031312 CallK walking down Parameter List \& Break \& Return
mailed-by:         gmail.com
signed-by:         gmail.com

At line 1310 in your code you have:

if ( (tmpt->sibling \texttt{= NULL) \&\&  (tmpp->sibling !} NULL) ) \{
1311                        printf("ERROR(\%d): Too few parameters passed for function '\%s' defined on line \%d.\n", 
1312                               tmpt->sibling->linnum, t->string, p->linnum);
1313                        ++numError;

which tests if tmpt->sibling is NULL and then line 1312 asks for tempt-sibling->linnum  and gets a seg fault

I think you want t->linenum  instead to return the line number of the calling function.

the same would go for the error message for the Too many parameters.

Good job, BTW, on getting all these errors messages in.

cheers,

\chapter{hw3 \& hw3b 最终结果}
\label{sec-30}

贴一个自己 hw3b 的 match 结果让小伙伴们感受一下我们作业需要的结果是什么样子的。

如果是同专业的小伙伴,最后拷贝到编辑器中仔细看一下,这个网页因为宽度有限,让并排两列的比较呈现出来的很变形。

\texttt{===============================================}
Output of Building User Code
Explode the tar
c-.l
c-.y
makefile
scanType.h
symtab.h
symtab.c
20131116163236-(me\textasciitilde{}~)-CS445-F13-A3B.tar: POSIX tar archive (GNU)
Tests:                                  directory
c-.l:                                   lex description text
c-.y:                                   lex description text
makefile:                               ASCII make commands text
scanType.h:                             ASCII C program text
symtab.c:                               ASCII C program text
symtab.h:                               ASCII C program text
Making compiler
bison -v -t -d c-.y  
flex c-.l
g++ -DCPLUSPLUS -g        -c -o lex.yy.o lex.yy.c
g++ -DCPLUSPLUS -g        -c -o c-.tab.o c-.tab.c
g++ -DCPLUSPLUS -g        -c -o symtab.o symtab.c
g++ -DCPLUSPLUS -g      lex.yy.o c-.tab.o symtab.o -lfl -lm  -o c- 
Extracting test files
allErrors.c-
basicAll.c-
basicAll2.c-
basicExtra.c-
bullsandcows.c-
call.c-
call2.c-
call3.c-
chars.c-
everything02.c-
factor.c-
factorial.c-
init.c-
moutest.c-
ryantest.c-
tictactoe.c-
undef.c-
while03.c-
\texttt{===============================================}
Output of Testing
Limited to 5 seconds total run time and 5000 lines of output

/* \texttt{==============================================} *
\begin{center}
\begin{tabular}{l}
Tests for CS445 Assignment 3B\\
Comparison with Expected Output\\
\end{tabular}
\end{center}
/* \texttt{==============================================} *

/export/home/nibbler/TestWorld
find makefile
makefile
a makefile is here
RUN: c- allErrors.c-
ERROR(6): Function 'dog' at line 2 is expecting to return type int but got type bool.
ERROR(13): Function 'cat' at line 9 is expecting to return type int but return has no return value.
ERROR(16): Symbol 'cat' is already defined at line 9.
ERROR(25): Function 'ox' at line 23 is expecting no return value, but return has return value.
WARNING(28): Expecting to return type int but function 'emu' has no return statement.
ERROR(35): Symbol 'x' is already defined at line 34.
ERROR(45): Expecting Boolean test condition in if statement but got type int.
ERROR(46): Expecting Boolean test condition in while statement but got type int.
ERROR(48): Cannot use function 'cat' as a simple variable.
ERROR(49): '=' requires operands of the same type but lhs is type int and rhs is type bool.
ERROR(50): '+=' requires operands of type int but rhs is of type bool.
ERROR(51): '-=' requires operands of type int but lhs is of type bool.
ERROR(53): Cannot use array as test condition in if statement.
ERROR(54): Cannot use array as test condition in while statement.
ERROR(59): Cannot have a break statement outside of loop.
ERROR(62): The operation '+' does not work with arrays.
ERROR(63): The operation '-' does not work with arrays.
ERROR(64): Unary 'not' requires an operand of type bool but was given type int.
ERROR(64): The operation 'not' does not work with arrays.
ERROR(66): '==' requires operands of the same type but lhs is type int and rhs is type bool.
ERROR(68): '+' requires operands of type int but lhs is of type bool.
ERROR(70): '*' requires operands of type int but rhs is of type bool.
ERROR(72): Unary 'not' requires an operand of type bool but was given type int.
ERROR(74): Cannot index nonarray 'x'.
ERROR(76): Array 'aa' should be indexed by type int but got type bool.
ERROR(78): Array index is the unindexed array 'zz'.
ERROR(80): Symbol 'xyzzy' is not defined.
ERROR(81): Symbol 'meerkat' is not defined.
ERROR(83): 'x' is a simple variable and cannot be called.
ERROR(85): Too many parameters passed for function 'dog' defined on line 2.
ERROR(86): Too few parameters passed for function 'ibex' defined on line 18.
ERROR(88): Expecting type int in parameter 1 of call to 'dog' defined on line 2 but got type bool.
ERROR(90): Not expecting array in parameter 1 of call to 'cat' defined on line 9.
ERROR(91): Expecting type int in parameter 1 of call to 'cat' defined on line 9 but got type void.
ERROR(93): Expecting array in parameter 1 of call to 'ibex' defined on line 18.
ERROR(94): Cannot use function 'ibex' as a simple variable.
ERROR(96): Cannot use function 'ibex' as a simple variable.
ERROR(96): '+' requires operands of type int but lhs is of type bool.
ERROR(98): The operation '*' only works with arrays.
ERROR(100): '*' requires operands of type int but lhs is of type bool.
ERROR(100): The operation '*' does not work with arrays.
ERROR(100): '+' requires operands of type int but lhs is of type bool.
ERROR(100): The operation '*' only works with arrays.
ERROR(100): The operation 'not' does not work with arrays.
ERROR(100): Unary 'not' requires an operand of type bool but was given type int.
ERROR(100): 'or' requires operands of type bool but lhs is of type int.
ERROR(100): Cannot have a break statement outside of loop.
ERROR(101): 'and' requires operands of type bool but lhs is of type int.
ERROR(101): 'and' requires operands of type bool but rhs is of type int.
ERROR(103): Cannot return an array.
ERROR(108): Symbol 'main' is already defined at line 32.
ERROR(122): '==' requires operands of the same type but lhs is type bool and rhs is type char.
ERROR(127): '=' requires operands of the same type but lhs is type bool and rhs is type char.
ERROR(131): '!=' requires operands of the same type but lhs is type bool and rhs is type char.
ERROR(137): Symbol 'z' is already defined at line 106.
ERROR(139): Variable 'a' is of type int but is being initialized with an expression of type bool.
ERROR(139): Variable 'b' is of type int but is being initialized with an expression of type bool.
ERROR(139): Variable 'zz' is of type int but is being initialized with an expression of type char.
ERROR(141): Initializer for variable 'd' is not a constant expression.
ERROR(142): '*' requires operands of type int but rhs is of type char.
ERROR(143): Initializer for variable 'e' is not a constant expression.
ERROR(144): Initializer for variable 'f' is not a constant expression.
ERROR(146): Symbol 'main' is already defined at line 32.
ERROR(159): '==' requires operands of the same type but lhs is type bool and rhs is type char.
ERROR(165): '=' requires operands of the same type but lhs is type bool and rhs is type char.
ERROR(183): In foreach statement the variable to the left of 'in' must not be an array.
ERROR(184): In foreach statement the variable to the left of 'in' must not be an array.
ERROR(184): Foreach requires operands of 'in' be the same type but lhs is type bool and rhs array is type int.
ERROR(187): Foreach requires operands of 'in' be the same type but lhs is type bool and rhs array is type int.
ERROR(189): If not an array, foreach requires rhs of 'in' be of type int but it is type bool.
ERROR(190): If not an array, foreach requires lhs of 'in' be of type int but it is type bool.
ERROR(196): Initializer for nonarray variable 'w' of type char cannot be initialized with an array.
ERROR(197): Variable 'u' is of type int but is being initialized with an expression of type char.
ERROR(197): Initializer for nonarray variable 'u' of type int cannot be initialized with an array.
ERROR(198): Array 't' must be of type char to be initialized, but is of type int.
ERROR(198): Initializer for array variable 't' must be a string, but is of nonarray type int.
ERROR(199): Initializer for array variable 's' must be a string, but is of nonarray type char.
ERROR(207): Symbol 'main' is already defined at line 32.
ERROR(209): Initializer for variable 'x' is not a constant expression.
ERROR(210): Initializer for nonarray variable 'c' of type char cannot be initialized with an array.
ERROR(211): Initializer for array variable 'd' must be a string, but is of nonarray type char.
ERROR(212): Variable 'e' is of type char but is being initialized with an expression of type int.
ERROR(213): Initializer for array variable 'f' must be a string, but is of nonarray type int.
ERROR(214): Array 'z' must be of type char to be initialized, but is of type int.
ERROR(216): '=' requires that if one operand is an array so must the other operand.
ERROR(217): '=' requires that if one operand is an array so must the other operand.
ERROR(218): '=' requires operands of the same type but lhs is type char and rhs is type int.
ERROR(219): '=' requires operands of the same type but lhs is type char and rhs is type int.
ERROR(219): '=' requires that if one operand is an array so must the other operand.
ERROR(220): '=' requires operands of the same type but lhs is type int and rhs is type char.
ERROR(222): '==' requires operands of the same type but lhs is type int and rhs is type char.
ERROR(223): '>' requires operands of type char or type int but rhs is of type bool.
ERROR(224): '>' requires operands of type char or type int but lhs is of type bool.
ERROR(225): The operation '>' does not work with arrays.
ERROR(226): The operation '>' does not work with arrays.
ERROR(227): The operation '>' does not work with arrays.
ERROR(229): Unary '-' requires an operand of type int but was given type char.
ERROR(230): The operation '*' only works with arrays.
ERROR(231): The operation '-' does not work with arrays.
ERROR(234): Unary '\sout{+' requires an operand of type int but was given type char.
ERROR(234): The operation '+}' does not work with arrays.
ERROR(235): The operation '++' does not work with arrays.
ERROR(236): Unary '--' requires an operand of type int but was given type char.
ERROR(236): The operation '--' does not work with arrays.
ERROR(237): The operation '--' does not work with arrays.
WARNING(207): Expecting to return type int but function 'main' has no return statement.
ERROR(242): Symbol 'fred' is already defined at line 241.
ERROR(244): Symbol 'fred' is already defined at line 243.
ERROR(245): Symbol 'fred' is already defined at line 243.
ERROR(246): Symbol 'fred' is already defined at line 243.
WARNING(242): Expecting to return type int but function 'fred' has no return statement.
Number of warnings: 3
Number of errors: 108
RUN: c- basicAll.c-
ERROR(16): Symbol 'cat' is not defined.
ERROR(18): Function 'dog' at line 12 is expecting to return type int but got type bool.
ERROR(25): Function 'cat' at line 21 is expecting to return type int but return has no return value.
ERROR(28): Symbol 'cat' is already defined at line 21.
ERROR(39): Function 'ox' at line 37 is expecting no return value, but return has return value.
WARNING(42): Expecting to return type int but function 'emu' has no return statement.
ERROR(54): Symbol 'x' is already defined at line 53.
ERROR(55): Symbol 'x' is already defined at line 53.
ERROR(63): Symbol 'v' is not defined.
ERROR(65): Expecting Boolean test condition in if statement but got type int.
ERROR(66): Expecting Boolean test condition in while statement but got type int.
ERROR(67): Expecting Boolean test condition in while statement but got type void.
ERROR(69): Cannot use function 'cat' as a simple variable.
ERROR(70): '=' requires operands of the same type but lhs is type int and rhs is type bool.
ERROR(71): '+=' requires operands of type int but rhs is of type bool.
ERROR(72): '-=' requires operands of type int but lhs is of type bool.
ERROR(73): '>' requires operands of type char or type int but lhs is of type bool.
ERROR(74): Unary '++' requires an operand of type int but was given type bool.
ERROR(76): Cannot use array as test condition in if statement.
ERROR(77): Cannot use array as test condition in while statement.
ERROR(79): Cannot have a break statement outside of loop.
ERROR(83): The operation '+' does not work with arrays.
ERROR(84): The operation '<' does not work with arrays.
ERROR(85): '<' requires operands of type char or type int but lhs is of type bool.
ERROR(85): '<' requires operands of type char or type int but rhs is of type bool.
ERROR(85): The operation '<' does not work with arrays.
ERROR(87): '==' requires operands of the same type but lhs is type int and rhs is type bool.
ERROR(91): '+' requires operands of type int but lhs is of type bool.
ERROR(93): '*' requires operands of type int but rhs is of type bool.
ERROR(95): Unary 'not' requires an operand of type bool but was given type int.
ERROR(96): '+' requires operands of type int but rhs is of type bool.
ERROR(96): Unary 'not' requires an operand of type bool but was given type int.
ERROR(97): Unary '-' requires an operand of type int but was given type bool.
ERROR(99): Cannot index nonarray 'x'.
ERROR(100): 'and' requires operands of type bool but lhs is of type int.
ERROR(100): 'and' requires operands of type bool but rhs is of type int.
ERROR(100): '*' requires operands of type int but rhs is of type bool.
ERROR(101): Symbol 'y' is not defined.
ERROR(101): 'and' requires operands of type bool but lhs is of type int.
ERROR(101): '*' requires operands of type int but rhs is of type bool.
ERROR(102): 'and' requires operands of type bool but lhs is of type int.
ERROR(102): 'and' requires operands of type bool but rhs is of type int.
ERROR(102): '*' requires operands of type int but rhs is of type bool.
ERROR(102): 'or' requires operands of type bool but rhs is of type int.
ERROR(104): The operation '*' only works with arrays.
ERROR(105): 'and' requires operands of type bool but lhs is of type int.
ERROR(109): Array 'aa' should be indexed by type int but got type bool.
ERROR(111): Cannot use function 'cat' as a simple variable.
ERROR(113): Array index is the unindexed array 'aa'.
ERROR(114): Symbol 'AA' is not defined.
ERROR(116): Symbol 'meerkat' is not defined.
ERROR(118): Symbol 'xyzzy' is not defined.
ERROR(119): Symbol 'meerkat' is not defined.
ERROR(120): Symbol 'xyzzy' is not defined.
ERROR(120): 'and' requires operands of type bool but rhs is of type int.
ERROR(122): Symbol 'flight' is not defined.
ERROR(122): Unary 'not' requires an operand of type bool but was given type int.
ERROR(122): Unary '-' requires an operand of type int but was given type bool.
ERROR(122): Symbol 'uu' is not defined.
ERROR(122): Symbol 'y' is not defined.
ERROR(122): '*' requires operands of type int but rhs is of type bool.
ERROR(124): 'x' is a simple variable and cannot be called.
ERROR(126): Too many parameters passed for function 'dog' defined on line 12.
ERROR(127): Too few parameters passed for function 'ibex' defined on line 31.
ERROR(129): Expecting type int in parameter 1 of call to 'dog' defined on line 12 but got type bool.
ERROR(131): Not expecting array in parameter 1 of call to 'cat' defined on line 21.
ERROR(133): Expecting array in parameter 1 of call to 'ibex' defined on line 31.
ERROR(134): Expecting type int in parameter 1 of call to 'ibex' defined on line 31 but got type bool.
ERROR(135): Cannot use function 'ibex' as a simple variable.
ERROR(137): Cannot use function 'ibex' as a simple variable.
ERROR(137): '+' requires operands of type int but lhs is of type bool.
ERROR(140): Array index is the unindexed array 'zz'.
ERROR(141): Expecting type int in parameter 1 of call to 'ox' defined on line 37 but got type bool.
ERROR(141): Array 'aa' should be indexed by type int but got type void.
ERROR(143): '=' requires operands of the same type but lhs is type int and rhs is type void.
ERROR(144): '+' requires operands of type int but rhs is of type void.
ERROR(145): Expecting type int in parameter 1 of call to 'cat' defined on line 21 but got type void.
ERROR(146): Expecting type int in parameter 1 of call to 'cat' defined on line 21 but got type bool.
ERROR(149): '=' requires operands of the same type but lhs is type int and rhs is type bool.
ERROR(150): '=' requires operands of the same type but lhs is type bool and rhs is type int.
ERROR(151): '=' requires operands of the same type but lhs is type int and rhs is type bool.
ERROR(151): '=' requires operands of the same type but lhs is type bool and rhs is type int.
ERROR(151): '=' requires operands of the same type but lhs is type int and rhs is type bool.
ERROR(153): '*' requires operands of type int but rhs is of type bool.
ERROR(153): '*' requires operands of type int but lhs is of type bool.
ERROR(153): 'and' requires operands of type bool but lhs is of type int.
ERROR(153): 'and' requires operands of type bool but rhs is of type int.
ERROR(155): '+' requires operands of type int but rhs is of type bool.
ERROR(155): Symbol 'parrot' is not defined.
ERROR(155): 'and' requires operands of type bool but rhs is of type int.
ERROR(155): Expecting type int in parameter 3 of call to 'emu' defined on line 42 but got type bool.
ERROR(156): Expecting type int in parameter 1 of call to 'emu' defined on line 42 but got type bool.
ERROR(156): '+' requires operands of type int but rhs is of type bool.
ERROR(156): Symbol 'parrot' is not defined.
ERROR(156): 'and' requires operands of type bool but rhs is of type int.
ERROR(156): Expecting type int in parameter 3 of call to 'emu' defined on line 42 but got type bool.
ERROR(156): 'and' requires operands of type bool but rhs is of type int.
ERROR(156): Expecting type int in parameter 2 of call to 'emu' defined on line 42 but got type bool.
ERROR(156): Expecting type int in parameter 3 of call to 'emu' defined on line 42 but got type bool.
ERROR(156): Too many parameters passed for function 'emu' defined on line 42.
ERROR(156): '*' requires operands of type int but rhs is of type bool.
ERROR(158): Cannot return an array.
Number of warnings: 1
Number of errors: 101
RUN: c- basicAll2.c-
ERROR(16): Symbol 'cat' is not defined.
ERROR(18): Function 'dog' at line 12 is expecting to return type char but got type int.
ERROR(25): Function 'cat' at line 21 is expecting to return type char but return has no return value.
ERROR(28): Symbol 'cat' is already defined at line 21.
ERROR(34): Function 'ibex' at line 31 is expecting to return type int but got type bool.
ERROR(39): Function 'ox' at line 37 is expecting no return value, but return has return value.
WARNING(42): Expecting to return type char but function 'emu' has no return statement.
ERROR(48): '+' requires operands of type int but lhs is of type char.
ERROR(48): '+' requires operands of type int but rhs is of type char.
ERROR(54): Symbol 'x' is already defined at line 53.
ERROR(55): Symbol 'x' is already defined at line 53.
ERROR(63): Symbol 'v' is not defined.
ERROR(65): Expecting Boolean test condition in if statement but got type char.
ERROR(66): Expecting Boolean test condition in while statement but got type char.
ERROR(67): Expecting type char in parameter 1 of call to 'ox' defined on line 37 but got type int.
ERROR(67): Expecting Boolean test condition in while statement but got type void.
ERROR(69): Cannot use function 'cat' as a simple variable.
ERROR(70): '=' requires operands of the same type but lhs is type char and rhs is type int.
ERROR(71): '+=' requires operands of type int but lhs is of type char.
ERROR(72): '-=' requires operands of type int but rhs is of type char.
ERROR(76): Expecting Boolean test condition in if statement but got type int.
ERROR(76): Cannot use array as test condition in if statement.
ERROR(77): Expecting Boolean test condition in while statement but got type int.
ERROR(77): Cannot use array as test condition in while statement.
ERROR(79): Cannot have a break statement outside of loop.
ERROR(81): Expecting Boolean test condition in while statement but got type int.
ERROR(83): '+' requires operands of type int but lhs is of type char.
ERROR(83): The operation '+' does not work with arrays.
ERROR(84): The operation '<' does not work with arrays.
ERROR(85): The operation '<' does not work with arrays.
ERROR(87): '\texttt{=' requires operands of the same type but lhs is type char and rhs is type bool.
ERROR(88): '=}' requires operands of the same type but lhs is type char and rhs is type int.
ERROR(89): '==' requires operands of the same type but lhs is type int and rhs is type bool.
ERROR(91): '+' requires operands of type int but lhs is of type bool.
ERROR(91): '+' requires operands of type int but rhs is of type char.
ERROR(93): '*' requires operands of type int but lhs is of type char.
ERROR(93): '*' requires operands of type int but rhs is of type bool.
ERROR(95): Unary 'not' requires an operand of type bool but was given type char.
ERROR(96): '+' requires operands of type int but lhs is of type char.
ERROR(96): Unary 'not' requires an operand of type bool but was given type int.
ERROR(99): Cannot index nonarray 'x'.
ERROR(100): 'and' requires operands of type bool but lhs is of type char.
ERROR(100): 'and' requires operands of type bool but rhs is of type char.
ERROR(100): '*' requires operands of type int but lhs is of type char.
ERROR(100): '*' requires operands of type int but rhs is of type bool.
ERROR(100): '+' requires operands of type int but lhs is of type char.
ERROR(100): '==' requires operands of the same type but lhs is type char and rhs is type int.
ERROR(100): '=' requires operands of the same type but lhs is type int and rhs is type bool.
ERROR(101): Symbol 'y' is not defined.
ERROR(101): 'and' requires operands of type bool but lhs is of type char.
ERROR(101): '*' requires operands of type int but lhs is of type char.
ERROR(101): '*' requires operands of type int but rhs is of type bool.
ERROR(101): '+' requires operands of type int but lhs is of type char.
ERROR(101): '==' requires operands of the same type but lhs is type char and rhs is type int.
ERROR(101): '=' requires operands of the same type but lhs is type int and rhs is type bool.
ERROR(102): 'and' requires operands of type bool but lhs is of type char.
ERROR(102): 'and' requires operands of type bool but rhs is of type char.
ERROR(102): '*' requires operands of type int but lhs is of type char.
ERROR(102): '*' requires operands of type int but rhs is of type bool.
ERROR(102): 'or' requires operands of type bool but rhs is of type int.
ERROR(102): '=' requires operands of the same type but lhs is type int and rhs is type bool.
ERROR(104): The operation '*' only works with arrays.
ERROR(105): 'and' requires operands of type bool but lhs is of type int.
ERROR(105): 'and' requires operands of type bool but rhs is of type int.
ERROR(111): Cannot use function 'cat' as a simple variable.
ERROR(111): Array 'aa' should be indexed by type int but got type char.
ERROR(113): Array 'aa' should be indexed by type int but got type char.
ERROR(113): Array index is the unindexed array 'aa'.
ERROR(114): Symbol 'AA' is not defined.
ERROR(116): Symbol 'meerkat' is not defined.
ERROR(118): Symbol 'xyzzy' is not defined.
ERROR(119): Symbol 'meerkat' is not defined.
ERROR(120): Symbol 'xyzzy' is not defined.
ERROR(120): 'and' requires operands of type bool but rhs is of type int.
ERROR(122): Symbol 'flight' is not defined.
ERROR(122): Unary 'not' requires an operand of type bool but was given type int.
ERROR(122): Unary '-' requires an operand of type int but was given type bool.
ERROR(122): Symbol 'uu' is not defined.
ERROR(122): Symbol 'y' is not defined.
ERROR(122): '*' requires operands of type int but lhs is of type char.
ERROR(122): '+' requires operands of type int but lhs is of type char.
ERROR(124): 'x' is a simple variable and cannot be called.
ERROR(126): Expecting type char in parameter 1 of call to 'dog' defined on line 12 but got type int.
ERROR(126): Too many parameters passed for function 'dog' defined on line 12.
ERROR(127): Too few parameters passed for function 'ibex' defined on line 31.
ERROR(129): Expecting type char in parameter 1 of call to 'dog' defined on line 12 but got type bool.
ERROR(131): Not expecting array in parameter 1 of call to 'cat' defined on line 21.
ERROR(133): Expecting array in parameter 1 of call to 'ibex' defined on line 31.
ERROR(134): Expecting type char in parameter 1 of call to 'ibex' defined on line 31 but got type int.
ERROR(135): Cannot use function 'ibex' as a simple variable.
ERROR(137): Cannot use function 'ibex' as a simple variable.
ERROR(139): Array 'aa' should be indexed by type int but got type char.
ERROR(140): Array 'aa' should be indexed by type int but got type char.
ERROR(140): Array index is the unindexed array 'zz'.
ERROR(141): Expecting type char in parameter 1 of call to 'ox' defined on line 37 but got type bool.
ERROR(141): Array 'aa' should be indexed by type int but got type void.
ERROR(143): Expecting type char in parameter 1 of call to 'ox' defined on line 37 but got type int.
ERROR(143): '=' requires operands of the same type but lhs is type char and rhs is type void.
ERROR(144): Expecting type char in parameter 1 of call to 'ox' defined on line 37 but got type int.
ERROR(144): '+' requires operands of type int but lhs is of type char.
ERROR(144): '+' requires operands of type int but rhs is of type void.
ERROR(144): '=' requires operands of the same type but lhs is type char and rhs is type int.
ERROR(145): Expecting type char in parameter 1 of call to 'ox' defined on line 37 but got type int.
ERROR(145): Expecting type char in parameter 1 of call to 'cat' defined on line 21 but got type void.
ERROR(146): Expecting type char in parameter 1 of call to 'cat' defined on line 21 but got type bool.
ERROR(147): Expecting type char in parameter 1 of call to 'cat' defined on line 21 but got type int.
ERROR(149): '=' requires operands of the same type but lhs is type char and rhs is type bool.
ERROR(151): '=' requires operands of the same type but lhs is type char and rhs is type int.
ERROR(151): '=' requires operands of the same type but lhs is type int and rhs is type char.
ERROR(151): '=' requires operands of the same type but lhs is type char and rhs is type int.
ERROR(153): '*' requires operands of type int but lhs is of type char.
ERROR(153): '*' requires operands of type int but rhs is of type char.
ERROR(153): 'and' requires operands of type bool but lhs is of type int.
ERROR(153): 'and' requires operands of type bool but rhs is of type int.
ERROR(155): '+' requires operands of type int but lhs is of type char.
ERROR(155): Expecting type char in parameter 1 of call to 'emu' defined on line 42 but got type int.
ERROR(155): Symbol 'parrot' is not defined.
ERROR(155): 'and' requires operands of type bool but lhs is of type int.
ERROR(155): 'and' requires operands of type bool but rhs is of type char.
ERROR(155): Expecting type char in parameter 3 of call to 'emu' defined on line 42 but got type bool.
ERROR(156): Expecting type char in parameter 1 of call to 'emu' defined on line 42 but got type bool.
ERROR(156): '+' requires operands of type int but lhs is of type char.
ERROR(156): Expecting type char in parameter 1 of call to 'emu' defined on line 42 but got type int.
ERROR(156): Symbol 'parrot' is not defined.
ERROR(156): 'and' requires operands of type bool but lhs is of type int.
ERROR(156): 'and' requires operands of type bool but rhs is of type char.
ERROR(156): Expecting type char in parameter 3 of call to 'emu' defined on line 42 but got type bool.
ERROR(156): 'and' requires operands of type bool but rhs is of type char.
ERROR(156): Expecting type char in parameter 2 of call to 'emu' defined on line 42 but got type bool.
ERROR(156): Expecting type char in parameter 3 of call to 'emu' defined on line 42 but got type bool.
ERROR(156): Too many parameters passed for function 'emu' defined on line 42.
ERROR(156): '*' requires operands of type int but lhs is of type char.
ERROR(158): Cannot return an array.
Number of warnings: 1
Number of errors: 132
RUN: c- basicExtra.c-
WARNING(3): Expecting to return type char but function 'toads' has no return statement.
ERROR(14): '<' requires operands of type char or type int but lhs is of type bool.
ERROR(14): '<' requires operands of type char or type int but rhs is of type bool.
ERROR(14): The operation '<' does not work with arrays.
ERROR(15): The operation '<' does not work with arrays.
ERROR(18): '<' requires operands of type char or type int but lhs is of type bool.
ERROR(18): '<' requires operands of type char or type int but rhs is of type bool.
ERROR(19): '/' requires operands of type int but lhs is of type bool.
ERROR(20): '/' requires operands of type int but rhs is of type bool.
ERROR(21): '\%' requires operands of type int but lhs is of type bool.
ERROR(22): '\%' requires operands of type int but rhs is of type bool.
ERROR(23): '-' requires operands of type int but lhs is of type bool.
ERROR(24): '-' requires operands of type int but rhs is of type bool.
ERROR(25): '<' requires operands of type char or type int but lhs is of type void.
ERROR(25): '<' requires operands of type char or type int but rhs is of type void.
ERROR(27): '<' requires operands of type char or type int but lhs is of type void.
ERROR(28): '<' requires operands of type char or type int but rhs is of type void.
ERROR(30): Symbol 'k' is not defined.
ERROR(31): Symbol 'k' is not defined.
ERROR(32): Symbol 'k' is not defined.
ERROR(32): Symbol 'k' is not defined.
ERROR(33): Symbol 'k' is not defined.
ERROR(34): 'i' is a simple variable and cannot be called.
Number of warnings: 1
Number of errors: 22
RUN: c- bullsandcows.c-
Number of warnings: 0
Number of errors: 0
RUN: c- call.c-
WARNING(1): Expecting to return type int but function 'showInt' has no return statement.
WARNING(7): Expecting to return type int but function 'showIntArray' has no return statement.
WARNING(18): Expecting to return type int but function 'showBool' has no return statement.
Number of warnings: 3
Number of errors: 0
RUN: c- call2.c-
Number of warnings: 0
Number of errors: 0
RUN: c- call3.c-
ERROR(4): Symbol 'fred' is not defined.
ERROR(4): Symbol 'x' is not defined.
ERROR(4): Symbol 'y' is not defined.
ERROR(4): 'and' requires operands of type bool but lhs is of type int.
ERROR(4): 'and' requires operands of type bool but rhs is of type int.
ERROR(6): Function 'main' at line 1 is expecting no return value, but return has return value.
ERROR(12): Too many parameters passed for function 'fred' defined on line 9.
ERROR(12): The operation '*' only works with arrays.
ERROR(14): Too few parameters passed for function 'fred' defined on line 9.
ERROR(13): Expecting array in parameter 1 of call to 'fred' defined on line 9.
ERROR(15): Expecting type char in parameter 1 of call to 'fred' defined on line 9 but got type int.
WARNING(9): Expecting to return type char but function 'fred' has no return statement.
ERROR(20): 'x' is a simple variable and cannot be called.
ERROR(20): '+' requires operands of type int but rhs is of type bool.
ERROR(20): The operation '*' only works with arrays.
ERROR(20): 'and' requires operands of type bool but lhs is of type int.
ERROR(20): 'and' requires operands of type bool but rhs is of type int.
ERROR(21): Symbol 'y' is not defined.
ERROR(21): '+' requires operands of type int but rhs is of type bool.
ERROR(21): The operation '*' only works with arrays.
ERROR(21): 'and' requires operands of type bool but lhs is of type int.
ERROR(21): 'and' requires operands of type bool but rhs is of type int.
ERROR(22): 'z' is a simple variable and cannot be called.
ERROR(22): '+' requires operands of type int but rhs is of type bool.
ERROR(22): The operation '*' only works with arrays.
ERROR(22): 'and' requires operands of type bool but lhs is of type int.
ERROR(22): 'and' requires operands of type bool but rhs is of type int.
ERROR(25): Symbol 'output' is already defined at line -1.
ERROR(26): Symbol 'outputb' is already defined at line -1.
ERROR(27): Symbol 'outputc' is already defined at line -1.
ERROR(29): Symbol 'input' is already defined at line -1.
ERROR(30): Symbol 'inputb' is already defined at line -1.
ERROR(31): Symbol 'inputc' is already defined at line -1.
ERROR(43): Symbol 'x' is already defined at line 42.
WARNING(42): Expecting to return type int but function 'AlanTuring' has no return statement.
ERROR(50): Symbol 'AlanTuring' is already defined at line 42.
ERROR(54): Symbol 'x' is already defined at line 52.
WARNING(50): Expecting to return type int but function 'AlanTuring' has no return statement.
Number of warnings: 3
Number of errors: 35
RUN: c- chars.c-
Number of warnings: 0
Number of errors: 0
RUN: c- everything02.c-
Number of warnings: 0
Number of errors: 0
RUN: c- factor.c-
Number of warnings: 0
Number of errors: 0
RUN: c- factorial.c-
Number of warnings: 0
Number of errors: 0
RUN: c- init.c-
ERROR(15): Symbol 'y' is already defined at line 4.
ERROR(17): Cannot index nonarray 'y'.
ERROR(17): '=' requires operands of the same type but lhs is type char and rhs is type int.
Number of warnings: 0
Number of errors: 3
RUN: c- moutest.c-
WARNING(5): Expecting to return type int but function 'emu' has no return statement.
ERROR(19): Array 'aa' should be indexed by type int but got type bool.
ERROR(20): Array index is the unindexed array 'dd'.
ERROR(17): Array 'cc' should be indexed by type int but got type bool.
ERROR(25): Array 'aa' should be indexed by type int but got type bool.
ERROR(23): Array 'cc' should be indexed by type int but got type bool.
ERROR(30): Array 'cc' should be indexed by type int but got type bool.
ERROR(39): The operation '*' only works with arrays.
ERROR(36): Array 'cc' should be indexed by type int but got type bool.
ERROR(42): Expecting type int in parameter 1 of call to 'emu' defined on line 5 but got type bool.
ERROR(43): '+' requires operands of type int but rhs is of type bool.
ERROR(44): '+' requires operands of type int but rhs is of type bool.
ERROR(45): '+' requires operands of type int but rhs is of type bool.
ERROR(46): Symbol 'parrot' is not defined.
ERROR(47): 'and' requires operands of type bool but rhs is of type int.
ERROR(45): Expecting type int in parameter 3 of call to 'emu' defined on line 5 but got type bool.
ERROR(48): 'and' requires operands of type bool but rhs is of type int.
ERROR(44): Expecting type int in parameter 3 of call to 'emu' defined on line 5 but got type bool.
ERROR(49): 'and' requires operands of type bool but rhs is of type int.
ERROR(43): Expecting type int in parameter 3 of call to 'emu' defined on line 5 but got type bool.
ERROR(43): 'and' requires operands of type bool but rhs is of type int.
ERROR(42): Expecting type int in parameter 2 of call to 'emu' defined on line 5 but got type bool.
ERROR(42): Expecting type int in parameter 3 of call to 'emu' defined on line 5 but got type bool.
ERROR(42): Too many parameters passed for function 'emu' defined on line 5.
ERROR(51): '*' requires operands of type int but rhs is of type bool.
Number of warnings: 1
Number of errors: 24
RUN: c- ryantest.c-
ERROR(7): Initializer for variable 'x' is not a constant expression.
ERROR(7): Symbol 'x' is already defined at line 5.
ERROR(18): '\texttt{=' requires operands of the same type but lhs is type int and rhs is type char.
ERROR(18): '=}' requires that if one operand is an array so must the other operand.
ERROR(19): 'and' requires operands of type bool but lhs is of type int.
ERROR(19): 'and' requires operands of type bool but rhs is of type char.
ERROR(19): The operation 'and' does not work with arrays.
ERROR(35): '=' requires operands of the same type but lhs is type bool and rhs is type int.
ERROR(34): '=' requires operands of the same type but lhs is type int and rhs is type bool.
ERROR(38): Symbol 'undef' is not defined.
ERROR(37): '=' requires operands of the same type but lhs is type int and rhs is type bool.
ERROR(41): Symbol 'undef' is not defined.
ERROR(44): Symbol 'undef' is not defined.
ERROR(47): '=' requires operands of the same type but lhs is type bool and rhs is type int.
ERROR(46): '=' requires operands of the same type but lhs is type int and rhs is type bool.
ERROR(46): '+' requires operands of type int but lhs is of type bool.
ERROR(50): Symbol 'undef' is not defined.
ERROR(49): '+' requires operands of type int but lhs is of type bool.
ERROR(53): Symbol 'undef' is not defined.
ERROR(52): '=' requires operands of the same type but lhs is type int and rhs is type bool.
ERROR(52): '+' requires operands of type int but lhs is of type bool.
ERROR(56): '=' requires operands of the same type but lhs is type bool and rhs is type void.
ERROR(55): '=' requires operands of the same type but lhs is type int and rhs is type bool.
ERROR(55): '+' requires operands of type int but lhs is of type bool.
ERROR(64): Symbol 'foo' is not defined.
ERROR(65): Symbol 'foo' is not defined.
ERROR(66): Symbol 'foo' is not defined.
ERROR(67): Symbol 'foo' is not defined.
ERROR(68): Symbol 'foo' is not defined.
ERROR(68): Symbol 'foo' is not defined.
ERROR(69): 'x' is a simple variable and cannot be called.
ERROR(69): '*' requires operands of type int but lhs is of type char.
ERROR(69): Cannot use function 'test' as a simple variable.
ERROR(71): Cannot use function 'check' as a simple variable.
ERROR(72): Cannot use function 'check' as a simple variable.
ERROR(72): 'x' is a simple variable and cannot be called.
ERROR(73): Symbol 'c' is not defined.
ERROR(74): Symbol 'c' is not defined.
ERROR(76): Symbol 'foo' is not defined.
ERROR(78): Cannot index nonarray 'x'.
ERROR(79): Cannot index nonarray 'x'.
ERROR(80): Symbol 'y' is not defined.
ERROR(82): Cannot use function 'check' as a simple variable.
ERROR(83): 'x' is a simple variable and cannot be called.
ERROR(86): '=' requires operands of the same type but lhs is type int and rhs is type char.
ERROR(87): '=' requires operands of the same type but lhs is type char and rhs is type int.
ERROR(87): Expecting type int in parameter 2 of call to 'check' defined on line 26 but got type char.
ERROR(88): Symbol 'y' is not defined.
ERROR(88): Too many parameters passed for function 'check' defined on line 26.
ERROR(88): Too few parameters passed for function 'check' defined on line 26.
ERROR(89): Expecting array in parameter 2 of call to 'func' defined on line 61.
ERROR(90): Not expecting array in parameter 1 of call to 'func' defined on line 61.
ERROR(92): '-' requires operands of type int but rhs is of type char.
ERROR(91): Expecting array in parameter 2 of call to 'func' defined on line 61.
ERROR(91): Too many parameters passed for function 'func' defined on line 61.
ERROR(94): Expecting type int in parameter 1 of call to 'func' defined on line 61 but got type char.
ERROR(94): Not expecting array in parameter 1 of call to 'func' defined on line 61.
ERROR(94): Too few parameters passed for function 'func' defined on line 61.
ERROR(99): Initializer for variable 'x' is not a constant expression.
ERROR(99): Symbol 'x' is already defined at line 97.
ERROR(100): Symbol 'x' is already defined at line 97.
ERROR(101): Variable 'x' is of type char but is being initialized with an expression of type int.
ERROR(101): Symbol 'x' is already defined at line 97.
ERROR(102): Symbol 'x' is already defined at line 97.
ERROR(104): Symbol 'z' is already defined at line 103.
ERROR(105): Initializer for variable 'y' is not a constant expression.
ERROR(106): Initializer for variable 'a' is not a constant expression.
ERROR(106): Initializer for nonarray variable 'a' of type int cannot be initialized with an array.
ERROR(107): Initializer for variable 'b' is not a constant expression.
ERROR(108): Initializer for variable 'd' is not a constant expression.
ERROR(109): Variable 'e' is of type int but is being initialized with an expression of type char.
ERROR(109): Initializer for nonarray variable 'e' of type int cannot be initialized with an array.
ERROR(110): Array 'f' must be of type char to be initialized, but is of type int.
ERROR(111): Initializer for nonarray variable 'g' of type char cannot be initialized with an array.
ERROR(113): Initializer for variable 'i' is not a constant expression.
ERROR(114): Initializer for variable 'j' is not a constant expression.
ERROR(115): Initializer for variable 'k' is not a constant expression.
ERROR(115): Variable 'k' is of type int but is being initialized with an expression of type bool.
ERROR(116): Initializer for variable 'j' is not a constant expression.
ERROR(116): Symbol 'j' is already defined at line 114.
ERROR(118): Initializer for variable 'j2' is not a constant expression.
ERROR(129): Expecting array in parameter 2 of call to 'retCheck' defined on line 125.
ERROR(130): Cannot return an array.
ERROR(127): Function 'retCheck' at line 125 is expecting to return type int but return has no return value.
Number of warnings: 0
Number of errors: 84
RUN: c- tictactoe.c-
Number of warnings: 0
Number of errors: 0
RUN: c- undef.c-
WARNING(3): Expecting to return type int but function 'intvalue' has no return statement.
ERROR(10): Symbol 'undef' is not defined.
ERROR(10): Initializer for variable 'x' is not a constant expression.
ERROR(10): Symbol 'x' is already defined at line 9.
ERROR(14): Symbol 'undef' is not defined.
ERROR(15): '+' requires operands of type int but rhs is of type void.
ERROR(16): Symbol 'undef' is not defined.
ERROR(17): '+' requires operands of type int but lhs is of type void.
ERROR(18): Symbol 'undef' is not defined.
ERROR(18): Symbol 'undef' is not defined.
ERROR(19): '+' requires operands of type int but lhs is of type void.
ERROR(19): '+' requires operands of type int but rhs is of type void.
ERROR(22): Symbol 'undef' is not defined.
ERROR(23): '+' requires operands of type int but rhs is of type void.
ERROR(24): Symbol 'undef' is not defined.
ERROR(25): '+' requires operands of type int but lhs is of type void.
ERROR(26): Symbol 'undef' is not defined.
ERROR(26): Symbol 'undef' is not defined.
ERROR(27): '+' requires operands of type int but lhs is of type void.
ERROR(27): '+' requires operands of type int but rhs is of type void.
ERROR(29): Symbol 'undef' is not defined.
ERROR(30): Symbol 'undef' is not defined.
ERROR(31): Symbol 'undef' is not defined.
ERROR(32): Symbol 'undef' is not defined.
ERROR(34): Symbol 'undef' is not defined.
ERROR(34): '+' requires operands of type int but rhs is of type bool.
ERROR(34): '==' requires operands of the same type but lhs is type int and rhs is type char.
ERROR(34): The operation '*' only works with arrays.
ERROR(35): 'x' is a simple variable and cannot be called.
ERROR(35): '+' requires operands of type int but rhs is of type bool.
ERROR(35): '==' requires operands of the same type but lhs is type int and rhs is type char.
ERROR(35): The operation '*' only works with arrays.
ERROR(37): Cannot use function 'novalue' as a simple variable.
ERROR(37): '=' requires operands of the same type but lhs is type void and rhs is type int.
ERROR(38): Cannot use function 'intvalue' as a simple variable.
ERROR(40): Expecting type int in parameter 1 of call to 'take1' defined on line 1 but got type void.
ERROR(41): Cannot use function 'novalue' as a simple variable.
ERROR(41): Expecting type int in parameter 1 of call to 'take1' defined on line 1 but got type void.
ERROR(42): Symbol 'undef' is not defined.
ERROR(43): Symbol 'undef' is not defined.
ERROR(45): Cannot use function 'intvalue' as a simple variable.
Number of warnings: 1
Number of errors: 40
RUN: c- while03.c-
Number of warnings: 0
Number of errors: 0

/* End of testing
\texttt{===============================================}
ztest zexpected differ: char 30492, line 442

Your output differs from the expected output.
This is a two column comparison with YOUR OUTPUT ON THE LEFT
with the EXPECTED OUTPUT ON THE RIGHT.
Limited to 5 seconds total run time and 5000 lines of ou        Limited to 5 seconds total run time and 5000 lines of ou

/* \texttt{==============================================} *                * \texttt{==============================================} *
\begin{center}
\begin{tabular}{lll}
Tests for CS445 Assignment 3B &  & Tests for CS445 Assignment 3B\\
Comparison with Expected Output &  & Comparison with Expected Output\\
\end{tabular}
\end{center}
/* \texttt{==============================================} *                * \texttt{==============================================} *

TestWorld                                                        TestWorld
find makefile                                                        find makefile
makefile                                                        makefile
a makefile is here                                                a makefile is here
RUN: c- allErrors.c-                                                RUN: c- allErrors.c-
ERROR(6): Function 'dog' at line 2 is expecting to retur        ERROR(6): Function 'dog' at line 2 is expecting to retur
ERROR(13): Function 'cat' at line 9 is expecting to retu        ERROR(13): Function 'cat' at line 9 is expecting to retu
ERROR(16): Symbol 'cat' is already defined at line 9.                ERROR(16): Symbol 'cat' is already defined at line 9.
ERROR(25): Function 'ox' at line 23 is expecting no retu        ERROR(25): Function 'ox' at line 23 is expecting no retu
WARNING(28): Expecting to return type int but function '        WARNING(28): Expecting to return type int but function '
ERROR(35): Symbol 'x' is already defined at line 34.                ERROR(35): Symbol 'x' is already defined at line 34.
ERROR(45): Expecting Boolean test condition in if statem        ERROR(45): Expecting Boolean test condition in if statem
ERROR(46): Expecting Boolean test condition in while sta        ERROR(46): Expecting Boolean test condition in while sta
ERROR(48): Cannot use function 'cat' as a simple variabl        ERROR(48): Cannot use function 'cat' as a simple variabl
ERROR(49): '=' requires operands of the same type but lh        ERROR(49): '=' requires operands of the same type but lh
ERROR(50): '+=' requires operands of type int but rhs is        ERROR(50): '+=' requires operands of type int but rhs is
ERROR(51): '-=' requires operands of type int but lhs is        ERROR(51): '-=' requires operands of type int but lhs is
ERROR(53): Cannot use array as test condition in if stat        ERROR(53): Cannot use array as test condition in if stat
ERROR(54): Cannot use array as test condition in while s        ERROR(54): Cannot use array as test condition in while s
ERROR(59): Cannot have a break statement outside of loop        ERROR(59): Cannot have a break statement outside of loop
ERROR(62): The operation '+' does not work with arrays.                ERROR(62): The operation '+' does not work with arrays.
ERROR(63): The operation '-' does not work with arrays.                ERROR(63): The operation '-' does not work with arrays.
ERROR(64): Unary 'not' requires an operand of type bool         ERROR(64): Unary 'not' requires an operand of type bool 
ERROR(64): The operation 'not' does not work with arrays        ERROR(64): The operation 'not' does not work with arrays
ERROR(66): '\texttt{=' requires operands of the same type but l	ERROR(66): '=}' requires operands of the same type but l
ERROR(68): '+' requires operands of type int but lhs is         ERROR(68): '+' requires operands of type int but lhs is 
ERROR(70): '*' requires operands of type int but rhs is         ERROR(70): '*' requires operands of type int but rhs is 
ERROR(72): Unary 'not' requires an operand of type bool         ERROR(72): Unary 'not' requires an operand of type bool 
ERROR(74): Cannot index nonarray 'x'.                                ERROR(74): Cannot index nonarray 'x'.
ERROR(76): Array 'aa' should be indexed by type int but         ERROR(76): Array 'aa' should be indexed by type int but 
ERROR(78): Array index is the unindexed array 'zz'.                ERROR(78): Array index is the unindexed array 'zz'.
ERROR(80): Symbol 'xyzzy' is not defined.                        ERROR(80): Symbol 'xyzzy' is not defined.
ERROR(81): Symbol 'meerkat' is not defined.                        ERROR(81): Symbol 'meerkat' is not defined.
ERROR(83): 'x' is a simple variable and cannot be called        ERROR(83): 'x' is a simple variable and cannot be called
ERROR(85): Too many parameters passed for function 'dog'        ERROR(85): Too many parameters passed for function 'dog'
ERROR(86): Too few parameters passed for function 'ibex'        ERROR(86): Too few parameters passed for function 'ibex'
ERROR(88): Expecting type int in parameter 1 of call to         ERROR(88): Expecting type int in parameter 1 of call to 
ERROR(90): Not expecting array in parameter 1 of call to        ERROR(90): Not expecting array in parameter 1 of call to
ERROR(91): Expecting type int in parameter 1 of call to         ERROR(91): Expecting type int in parameter 1 of call to 
ERROR(93): Expecting array in parameter 1 of call to 'ib        ERROR(93): Expecting array in parameter 1 of call to 'ib
ERROR(94): Cannot use function 'ibex' as a simple variab        ERROR(94): Cannot use function 'ibex' as a simple variab
ERROR(96): Cannot use function 'ibex' as a simple variab        ERROR(96): Cannot use function 'ibex' as a simple variab
ERROR(96): '+' requires operands of type int but lhs is         ERROR(96): '+' requires operands of type int but lhs is 
ERROR(98): The operation '*' only works with arrays.                ERROR(98): The operation '*' only works with arrays.
ERROR(100): '*' requires operands of type int but lhs is        ERROR(100): '*' requires operands of type int but lhs is
ERROR(100): The operation '*' does not work with arrays.        ERROR(100): The operation '*' does not work with arrays.
ERROR(100): '+' requires operands of type int but lhs is        ERROR(100): '+' requires operands of type int but lhs is
ERROR(100): The operation '*' only works with arrays.                ERROR(100): The operation '*' only works with arrays.
ERROR(100): The operation 'not' does not work with array        ERROR(100): The operation 'not' does not work with array
ERROR(100): Unary 'not' requires an operand of type bool        ERROR(100): Unary 'not' requires an operand of type bool
ERROR(100): 'or' requires operands of type bool but lhs         ERROR(100): 'or' requires operands of type bool but lhs 
ERROR(100): Cannot have a break statement outside of loo        ERROR(100): Cannot have a break statement outside of loo
ERROR(101): 'and' requires operands of type bool but lhs        ERROR(101): 'and' requires operands of type bool but lhs
ERROR(101): 'and' requires operands of type bool but rhs        ERROR(101): 'and' requires operands of type bool but rhs
ERROR(103): Cannot return an array.                                ERROR(103): Cannot return an array.
ERROR(108): Symbol 'main' is already defined at line 32.        ERROR(108): Symbol 'main' is already defined at line 32.
ERROR(122): '\texttt{=' requires operands of the same type but 	ERROR(122): '=}' requires operands of the same type but 
ERROR(127): '=' requires operands of the same type but l        ERROR(127): '=' requires operands of the same type but l
ERROR(131): '!=' requires operands of the same type but         ERROR(131): '!=' requires operands of the same type but 
ERROR(137): Symbol 'z' is already defined at line 106.                ERROR(137): Symbol 'z' is already defined at line 106.
ERROR(139): Variable 'a' is of type int but is being ini        ERROR(139): Variable 'a' is of type int but is being ini
ERROR(139): Variable 'b' is of type int but is being ini        ERROR(139): Variable 'b' is of type int but is being ini
ERROR(139): Variable 'zz' is of type int but is being in        ERROR(139): Variable 'zz' is of type int but is being in
ERROR(141): Initializer for variable 'd' is not a consta        ERROR(141): Initializer for variable 'd' is not a consta
ERROR(142): '*' requires operands of type int but rhs is        ERROR(142): '*' requires operands of type int but rhs is
ERROR(143): Initializer for variable 'e' is not a consta        ERROR(143): Initializer for variable 'e' is not a consta
ERROR(144): Initializer for variable 'f' is not a consta        ERROR(144): Initializer for variable 'f' is not a consta
ERROR(146): Symbol 'main' is already defined at line 32.        ERROR(146): Symbol 'main' is already defined at line 32.
ERROR(159): '\texttt{=' requires operands of the same type but 	ERROR(159): '=}' requires operands of the same type but 
ERROR(165): '=' requires operands of the same type but l        ERROR(165): '=' requires operands of the same type but l
ERROR(183): In foreach statement the variable to the lef        ERROR(183): In foreach statement the variable to the lef
ERROR(184): In foreach statement the variable to the lef        ERROR(184): In foreach statement the variable to the lef
ERROR(184): Foreach requires operands of 'in' be the sam        ERROR(184): Foreach requires operands of 'in' be the sam
ERROR(187): Foreach requires operands of 'in' be the sam        ERROR(187): Foreach requires operands of 'in' be the sam
ERROR(189): If not an array, foreach requires rhs of 'in        ERROR(189): If not an array, foreach requires rhs of 'in
ERROR(190): If not an array, foreach requires lhs of 'in        ERROR(190): If not an array, foreach requires lhs of 'in
ERROR(196): Initializer for nonarray variable 'w' of typ        ERROR(196): Initializer for nonarray variable 'w' of typ
ERROR(197): Variable 'u' is of type int but is being ini        ERROR(197): Variable 'u' is of type int but is being ini
ERROR(197): Initializer for nonarray variable 'u' of typ        ERROR(197): Initializer for nonarray variable 'u' of typ
ERROR(198): Array 't' must be of type char to be initial        ERROR(198): Array 't' must be of type char to be initial
ERROR(198): Initializer for array variable 't' must be a        ERROR(198): Initializer for array variable 't' must be a
ERROR(199): Initializer for array variable 's' must be a        ERROR(199): Initializer for array variable 's' must be a
ERROR(207): Symbol 'main' is already defined at line 32.        ERROR(207): Symbol 'main' is already defined at line 32.
ERROR(209): Initializer for variable 'x' is not a consta        ERROR(209): Initializer for variable 'x' is not a consta
ERROR(210): Initializer for nonarray variable 'c' of typ        ERROR(210): Initializer for nonarray variable 'c' of typ
ERROR(211): Initializer for array variable 'd' must be a        ERROR(211): Initializer for array variable 'd' must be a
ERROR(212): Variable 'e' is of type char but is being in        ERROR(212): Variable 'e' is of type char but is being in
ERROR(213): Initializer for array variable 'f' must be a        ERROR(213): Initializer for array variable 'f' must be a
ERROR(214): Array 'z' must be of type char to be initial        ERROR(214): Array 'z' must be of type char to be initial
ERROR(216): '=' requires that if one operand is an array        ERROR(216): '=' requires that if one operand is an array
ERROR(217): '=' requires that if one operand is an array        ERROR(217): '=' requires that if one operand is an array
ERROR(218): '=' requires operands of the same type but l        ERROR(218): '=' requires operands of the same type but l
ERROR(219): '=' requires operands of the same type but l        ERROR(219): '=' requires operands of the same type but l
ERROR(219): '=' requires that if one operand is an array        ERROR(219): '=' requires that if one operand is an array
ERROR(220): '=' requires operands of the same type but l        ERROR(220): '=' requires operands of the same type but l
ERROR(222): '\texttt{=' requires operands of the same type but 	ERROR(222): '=}' requires operands of the same type but 
ERROR(223): '>' requires operands of type char or type i        ERROR(223): '>' requires operands of type char or type i
ERROR(224): '>' requires operands of type char or type i        ERROR(224): '>' requires operands of type char or type i
ERROR(225): The operation '>' does not work with arrays.        ERROR(225): The operation '>' does not work with arrays.
ERROR(226): The operation '>' does not work with arrays.        ERROR(226): The operation '>' does not work with arrays.
ERROR(227): The operation '>' does not work with arrays.        ERROR(227): The operation '>' does not work with arrays.
ERROR(229): Unary '-' requires an operand of type int bu        ERROR(229): Unary '-' requires an operand of type int bu
ERROR(230): The operation '*' only works with arrays.                ERROR(230): The operation '*' only works with arrays.
ERROR(231): The operation '-' does not work with arrays.        ERROR(231): The operation '-' does not work with arrays.
ERROR(234): Unary '\sout{+' requires an operand of type int b        ERROR(234): Unary '+}' requires an operand of type int b
ERROR(234): The operation '\sout{+' does not work with arrays        ERROR(234): The operation '+}' does not work with arrays
ERROR(235): The operation '\sout{+' does not work with arrays        ERROR(235): The operation '+}' does not work with arrays
ERROR(236): Unary '--' requires an operand of type int b        ERROR(236): Unary '--' requires an operand of type int b
ERROR(236): The operation '--' does not work with arrays        ERROR(236): The operation '--' does not work with arrays
ERROR(237): The operation '--' does not work with arrays        ERROR(237): The operation '--' does not work with arrays
WARNING(207): Expecting to return type int but function         WARNING(207): Expecting to return type int but function 
ERROR(242): Symbol 'fred' is already defined at line 241        ERROR(242): Symbol 'fred' is already defined at line 241
ERROR(244): Symbol 'fred' is already defined at line 243        ERROR(244): Symbol 'fred' is already defined at line 243
ERROR(245): Symbol 'fred' is already defined at line 243        ERROR(245): Symbol 'fred' is already defined at line 243
ERROR(246): Symbol 'fred' is already defined at line 243        ERROR(246): Symbol 'fred' is already defined at line 243
WARNING(242): Expecting to return type int but function         WARNING(242): Expecting to return type int but function 
Number of warnings: 3                                                Number of warnings: 3
Number of errors: 108                                                Number of errors: 108
RUN: c- basicAll.c-                                                RUN: c- basicAll.c-
ERROR(16): Symbol 'cat' is not defined.                                ERROR(16): Symbol 'cat' is not defined.
ERROR(18): Function 'dog' at line 12 is expecting to ret        ERROR(18): Function 'dog' at line 12 is expecting to ret
ERROR(25): Function 'cat' at line 21 is expecting to ret        ERROR(25): Function 'cat' at line 21 is expecting to ret
ERROR(28): Symbol 'cat' is already defined at line 21.                ERROR(28): Symbol 'cat' is already defined at line 21.
ERROR(39): Function 'ox' at line 37 is expecting no retu        ERROR(39): Function 'ox' at line 37 is expecting no retu
WARNING(42): Expecting to return type int but function '        WARNING(42): Expecting to return type int but function '
ERROR(54): Symbol 'x' is already defined at line 53.                ERROR(54): Symbol 'x' is already defined at line 53.
ERROR(55): Symbol 'x' is already defined at line 53.                ERROR(55): Symbol 'x' is already defined at line 53.
ERROR(63): Symbol 'v' is not defined.                                ERROR(63): Symbol 'v' is not defined.
ERROR(65): Expecting Boolean test condition in if statem        ERROR(65): Expecting Boolean test condition in if statem
ERROR(66): Expecting Boolean test condition in while sta        ERROR(66): Expecting Boolean test condition in while sta
ERROR(67): Expecting Boolean test condition in while sta        ERROR(67): Expecting Boolean test condition in while sta
ERROR(69): Cannot use function 'cat' as a simple variabl        ERROR(69): Cannot use function 'cat' as a simple variabl
ERROR(70): '=' requires operands of the same type but lh        ERROR(70): '=' requires operands of the same type but lh
ERROR(71): '+=' requires operands of type int but rhs is        ERROR(71): '+=' requires operands of type int but rhs is
ERROR(72): '-=' requires operands of type int but lhs is        ERROR(72): '-=' requires operands of type int but lhs is
ERROR(73): '>' requires operands of type char or type in        ERROR(73): '>' requires operands of type char or type in
ERROR(74): Unary '\sout{+' requires an operand of type int bu        ERROR(74): Unary '+}' requires an operand of type int bu
ERROR(76): Cannot use array as test condition in if stat        ERROR(76): Cannot use array as test condition in if stat
ERROR(77): Cannot use array as test condition in while s        ERROR(77): Cannot use array as test condition in while s
ERROR(79): Cannot have a break statement outside of loop        ERROR(79): Cannot have a break statement outside of loop
ERROR(83): The operation '+' does not work with arrays.                ERROR(83): The operation '+' does not work with arrays.
ERROR(84): The operation '<' does not work with arrays.                ERROR(84): The operation '<' does not work with arrays.
ERROR(85): '<' requires operands of type char or type in        ERROR(85): '<' requires operands of type char or type in
ERROR(85): '<' requires operands of type char or type in        ERROR(85): '<' requires operands of type char or type in
ERROR(85): The operation '<' does not work with arrays.                ERROR(85): The operation '<' does not work with arrays.
ERROR(87): '\texttt{=' requires operands of the same type but l	ERROR(87): '=}' requires operands of the same type but l
ERROR(91): '+' requires operands of type int but lhs is         ERROR(91): '+' requires operands of type int but lhs is 
ERROR(93): '*' requires operands of type int but rhs is         ERROR(93): '*' requires operands of type int but rhs is 
ERROR(95): Unary 'not' requires an operand of type bool         ERROR(95): Unary 'not' requires an operand of type bool 
ERROR(96): '+' requires operands of type int but rhs is         ERROR(96): '+' requires operands of type int but rhs is 
ERROR(96): Unary 'not' requires an operand of type bool         ERROR(96): Unary 'not' requires an operand of type bool 
ERROR(97): Unary '-' requires an operand of type int but        ERROR(97): Unary '-' requires an operand of type int but
ERROR(99): Cannot index nonarray 'x'.                                ERROR(99): Cannot index nonarray 'x'.
ERROR(100): 'and' requires operands of type bool but lhs        ERROR(100): 'and' requires operands of type bool but lhs
ERROR(100): 'and' requires operands of type bool but rhs        ERROR(100): 'and' requires operands of type bool but rhs
ERROR(100): '*' requires operands of type int but rhs is        ERROR(100): '*' requires operands of type int but rhs is
ERROR(101): Symbol 'y' is not defined.                                ERROR(101): Symbol 'y' is not defined.
ERROR(101): 'and' requires operands of type bool but lhs        ERROR(101): 'and' requires operands of type bool but lhs
ERROR(101): '*' requires operands of type int but rhs is        ERROR(101): '*' requires operands of type int but rhs is
ERROR(102): 'and' requires operands of type bool but lhs        ERROR(102): 'and' requires operands of type bool but lhs
ERROR(102): 'and' requires operands of type bool but rhs        ERROR(102): 'and' requires operands of type bool but rhs
ERROR(102): '*' requires operands of type int but rhs is        ERROR(102): '*' requires operands of type int but rhs is
ERROR(102): 'or' requires operands of type bool but rhs         ERROR(102): 'or' requires operands of type bool but rhs 
ERROR(104): The operation '*' only works with arrays.                ERROR(104): The operation '*' only works with arrays.
ERROR(105): 'and' requires operands of type bool but lhs        ERROR(105): 'and' requires operands of type bool but lhs
ERROR(109): Array 'aa' should be indexed by type int but        ERROR(109): Array 'aa' should be indexed by type int but
ERROR(111): Cannot use function 'cat' as a simple variab        ERROR(111): Cannot use function 'cat' as a simple variab
ERROR(113): Array index is the unindexed array 'aa'.                ERROR(113): Array index is the unindexed array 'aa'.
ERROR(114): Symbol 'AA' is not defined.                                ERROR(114): Symbol 'AA' is not defined.
ERROR(116): Symbol 'meerkat' is not defined.                        ERROR(116): Symbol 'meerkat' is not defined.
ERROR(118): Symbol 'xyzzy' is not defined.                        ERROR(118): Symbol 'xyzzy' is not defined.
ERROR(119): Symbol 'meerkat' is not defined.                        ERROR(119): Symbol 'meerkat' is not defined.
ERROR(120): Symbol 'xyzzy' is not defined.                        ERROR(120): Symbol 'xyzzy' is not defined.
ERROR(120): 'and' requires operands of type bool but rhs        ERROR(120): 'and' requires operands of type bool but rhs
ERROR(122): Symbol 'flight' is not defined.                        ERROR(122): Symbol 'flight' is not defined.
ERROR(122): Unary 'not' requires an operand of type bool        ERROR(122): Unary 'not' requires an operand of type bool
ERROR(122): Unary '-' requires an operand of type int bu        ERROR(122): Unary '-' requires an operand of type int bu
ERROR(122): Symbol 'uu' is not defined.                                ERROR(122): Symbol 'uu' is not defined.
ERROR(122): Symbol 'y' is not defined.                                ERROR(122): Symbol 'y' is not defined.
ERROR(122): '*' requires operands of type int but rhs is        ERROR(122): '*' requires operands of type int but rhs is
ERROR(124): 'x' is a simple variable and cannot be calle        ERROR(124): 'x' is a simple variable and cannot be calle
ERROR(126): Too many parameters passed for function 'dog        ERROR(126): Too many parameters passed for function 'dog
ERROR(127): Too few parameters passed for function 'ibex        ERROR(127): Too few parameters passed for function 'ibex
ERROR(129): Expecting type int in parameter 1 of call to        ERROR(129): Expecting type int in parameter 1 of call to
ERROR(131): Not expecting array in parameter 1 of call t        ERROR(131): Not expecting array in parameter 1 of call t
ERROR(133): Expecting array in parameter 1 of call to 'i        ERROR(133): Expecting array in parameter 1 of call to 'i
ERROR(134): Expecting type int in parameter 1 of call to        ERROR(134): Expecting type int in parameter 1 of call to
ERROR(135): Cannot use function 'ibex' as a simple varia        ERROR(135): Cannot use function 'ibex' as a simple varia
ERROR(137): Cannot use function 'ibex' as a simple varia        ERROR(137): Cannot use function 'ibex' as a simple varia
ERROR(137): '+' requires operands of type int but lhs is        ERROR(137): '+' requires operands of type int but lhs is
ERROR(140): Array index is the unindexed array 'zz'.                ERROR(140): Array index is the unindexed array 'zz'.
ERROR(141): Expecting type int in parameter 1 of call to        ERROR(141): Expecting type int in parameter 1 of call to
ERROR(141): Array 'aa' should be indexed by type int but        ERROR(141): Array 'aa' should be indexed by type int but
ERROR(143): '=' requires operands of the same type but l        ERROR(143): '=' requires operands of the same type but l
ERROR(144): '+' requires operands of type int but rhs is        ERROR(144): '+' requires operands of type int but rhs is
ERROR(145): Expecting type int in parameter 1 of call to        ERROR(145): Expecting type int in parameter 1 of call to
ERROR(146): Expecting type int in parameter 1 of call to        ERROR(146): Expecting type int in parameter 1 of call to
ERROR(149): '=' requires operands of the same type but l        ERROR(149): '=' requires operands of the same type but l
ERROR(150): '=' requires operands of the same type but l        ERROR(150): '=' requires operands of the same type but l
ERROR(151): '=' requires operands of the same type but l        ERROR(151): '=' requires operands of the same type but l
ERROR(151): '=' requires operands of the same type but l        ERROR(151): '=' requires operands of the same type but l
ERROR(151): '=' requires operands of the same type but l        ERROR(151): '=' requires operands of the same type but l
ERROR(153): '*' requires operands of type int but rhs is        ERROR(153): '*' requires operands of type int but rhs is
ERROR(153): '*' requires operands of type int but lhs is        ERROR(153): '*' requires operands of type int but lhs is
ERROR(153): 'and' requires operands of type bool but lhs        ERROR(153): 'and' requires operands of type bool but lhs
ERROR(153): 'and' requires operands of type bool but rhs        ERROR(153): 'and' requires operands of type bool but rhs
ERROR(155): '+' requires operands of type int but rhs is        ERROR(155): '+' requires operands of type int but rhs is
ERROR(155): Symbol 'parrot' is not defined.                        ERROR(155): Symbol 'parrot' is not defined.
ERROR(155): 'and' requires operands of type bool but rhs        ERROR(155): 'and' requires operands of type bool but rhs
ERROR(155): Expecting type int in parameter 3 of call to        ERROR(155): Expecting type int in parameter 3 of call to
ERROR(156): Expecting type int in parameter 1 of call to        ERROR(156): Expecting type int in parameter 1 of call to
ERROR(156): '+' requires operands of type int but rhs is        ERROR(156): '+' requires operands of type int but rhs is
ERROR(156): Symbol 'parrot' is not defined.                        ERROR(156): Symbol 'parrot' is not defined.
ERROR(156): 'and' requires operands of type bool but rhs        ERROR(156): 'and' requires operands of type bool but rhs
ERROR(156): Expecting type int in parameter 3 of call to        ERROR(156): Expecting type int in parameter 3 of call to
ERROR(156): 'and' requires operands of type bool but rhs        ERROR(156): 'and' requires operands of type bool but rhs
ERROR(156): Expecting type int in parameter 2 of call to        ERROR(156): Expecting type int in parameter 2 of call to
ERROR(156): Expecting type int in parameter 3 of call to        ERROR(156): Expecting type int in parameter 3 of call to
ERROR(156): Too many parameters passed for function 'emu        ERROR(156): Too many parameters passed for function 'emu
ERROR(156): '*' requires operands of type int but rhs is        ERROR(156): '*' requires operands of type int but rhs is
ERROR(158): Cannot return an array.                                ERROR(158): Cannot return an array.
Number of warnings: 1                                                Number of warnings: 1
Number of errors: 101                                                Number of errors: 101
RUN: c- basicAll2.c-                                                RUN: c- basicAll2.c-
ERROR(16): Symbol 'cat' is not defined.                                ERROR(16): Symbol 'cat' is not defined.
ERROR(18): Function 'dog' at line 12 is expecting to ret        ERROR(18): Function 'dog' at line 12 is expecting to ret
ERROR(25): Function 'cat' at line 21 is expecting to ret        ERROR(25): Function 'cat' at line 21 is expecting to ret
ERROR(28): Symbol 'cat' is already defined at line 21.                ERROR(28): Symbol 'cat' is already defined at line 21.
ERROR(34): Function 'ibex' at line 31 is expecting to re        ERROR(34): Function 'ibex' at line 31 is expecting to re
ERROR(39): Function 'ox' at line 37 is expecting no retu        ERROR(39): Function 'ox' at line 37 is expecting no retu
WARNING(42): Expecting to return type char but function         WARNING(42): Expecting to return type char but function 
ERROR(48): '+' requires operands of type int but lhs is         ERROR(48): '+' requires operands of type int but lhs is 
ERROR(48): '+' requires operands of type int but rhs is         ERROR(48): '+' requires operands of type int but rhs is 
ERROR(54): Symbol 'x' is already defined at line 53.                ERROR(54): Symbol 'x' is already defined at line 53.
ERROR(55): Symbol 'x' is already defined at line 53.                ERROR(55): Symbol 'x' is already defined at line 53.
ERROR(63): Symbol 'v' is not defined.                                ERROR(63): Symbol 'v' is not defined.
ERROR(65): Expecting Boolean test condition in if statem        ERROR(65): Expecting Boolean test condition in if statem
ERROR(66): Expecting Boolean test condition in while sta        ERROR(66): Expecting Boolean test condition in while sta
ERROR(67): Expecting type char in parameter 1 of call to        ERROR(67): Expecting type char in parameter 1 of call to
ERROR(67): Expecting Boolean test condition in while sta        ERROR(67): Expecting Boolean test condition in while sta
ERROR(69): Cannot use function 'cat' as a simple variabl        ERROR(69): Cannot use function 'cat' as a simple variabl
ERROR(70): '=' requires operands of the same type but lh        ERROR(70): '=' requires operands of the same type but lh
ERROR(71): '+=' requires operands of type int but lhs is        ERROR(71): '+=' requires operands of type int but lhs is
ERROR(72): '-=' requires operands of type int but rhs is        ERROR(72): '-=' requires operands of type int but rhs is
ERROR(76): Expecting Boolean test condition in if statem        ERROR(76): Expecting Boolean test condition in if statem
ERROR(76): Cannot use array as test condition in if stat        ERROR(76): Cannot use array as test condition in if stat
ERROR(77): Expecting Boolean test condition in while sta        ERROR(77): Expecting Boolean test condition in while sta
ERROR(77): Cannot use array as test condition in while s        ERROR(77): Cannot use array as test condition in while s
ERROR(79): Cannot have a break statement outside of loop        ERROR(79): Cannot have a break statement outside of loop
ERROR(81): Expecting Boolean test condition in while sta        ERROR(81): Expecting Boolean test condition in while sta
ERROR(83): '+' requires operands of type int but lhs is         ERROR(83): '+' requires operands of type int but lhs is 
ERROR(83): The operation '+' does not work with arrays.                ERROR(83): The operation '+' does not work with arrays.
ERROR(84): The operation '<' does not work with arrays.                ERROR(84): The operation '<' does not work with arrays.
ERROR(85): The operation '<' does not work with arrays.                ERROR(85): The operation '<' does not work with arrays.
ERROR(87): '\texttt{=' requires operands of the same type but l	ERROR(87): '=}' requires operands of the same type but l
ERROR(88): '\texttt{=' requires operands of the same type but l	ERROR(88): '=}' requires operands of the same type but l
ERROR(89): '\texttt{=' requires operands of the same type but l	ERROR(89): '=}' requires operands of the same type but l
ERROR(91): '+' requires operands of type int but lhs is         ERROR(91): '+' requires operands of type int but lhs is 
ERROR(91): '+' requires operands of type int but rhs is         ERROR(91): '+' requires operands of type int but rhs is 
ERROR(93): '*' requires operands of type int but lhs is         ERROR(93): '*' requires operands of type int but lhs is 
ERROR(93): '*' requires operands of type int but rhs is         ERROR(93): '*' requires operands of type int but rhs is 
ERROR(95): Unary 'not' requires an operand of type bool         ERROR(95): Unary 'not' requires an operand of type bool 
ERROR(96): '+' requires operands of type int but lhs is         ERROR(96): '+' requires operands of type int but lhs is 
ERROR(96): Unary 'not' requires an operand of type bool         ERROR(96): Unary 'not' requires an operand of type bool 
ERROR(99): Cannot index nonarray 'x'.                                ERROR(99): Cannot index nonarray 'x'.
ERROR(100): 'and' requires operands of type bool but lhs        ERROR(100): 'and' requires operands of type bool but lhs
ERROR(100): 'and' requires operands of type bool but rhs        ERROR(100): 'and' requires operands of type bool but rhs
ERROR(100): '*' requires operands of type int but lhs is        ERROR(100): '*' requires operands of type int but lhs is
ERROR(100): '*' requires operands of type int but rhs is        ERROR(100): '*' requires operands of type int but rhs is
ERROR(100): '+' requires operands of type int but lhs is        ERROR(100): '+' requires operands of type int but lhs is
ERROR(100): '\texttt{=' requires operands of the same type but 	ERROR(100): '=}' requires operands of the same type but 
ERROR(100): '=' requires operands of the same type but l        ERROR(100): '=' requires operands of the same type but l
ERROR(101): Symbol 'y' is not defined.                                ERROR(101): Symbol 'y' is not defined.
ERROR(101): 'and' requires operands of type bool but lhs        ERROR(101): 'and' requires operands of type bool but lhs
ERROR(101): '*' requires operands of type int but lhs is        ERROR(101): '*' requires operands of type int but lhs is
ERROR(101): '*' requires operands of type int but rhs is        ERROR(101): '*' requires operands of type int but rhs is
ERROR(101): '+' requires operands of type int but lhs is        ERROR(101): '+' requires operands of type int but lhs is
ERROR(101): '\texttt{=' requires operands of the same type but 	ERROR(101): '=}' requires operands of the same type but 
ERROR(101): '=' requires operands of the same type but l        ERROR(101): '=' requires operands of the same type but l
ERROR(102): 'and' requires operands of type bool but lhs        ERROR(102): 'and' requires operands of type bool but lhs
ERROR(102): 'and' requires operands of type bool but rhs        ERROR(102): 'and' requires operands of type bool but rhs
ERROR(102): '*' requires operands of type int but lhs is        ERROR(102): '*' requires operands of type int but lhs is
ERROR(102): '*' requires operands of type int but rhs is        ERROR(102): '*' requires operands of type int but rhs is
ERROR(102): 'or' requires operands of type bool but rhs         ERROR(102): 'or' requires operands of type bool but rhs 
ERROR(102): '=' requires operands of the same type but l        ERROR(102): '=' requires operands of the same type but l
ERROR(104): The operation '*' only works with arrays.                ERROR(104): The operation '*' only works with arrays.
ERROR(105): 'and' requires operands of type bool but lhs        ERROR(105): 'and' requires operands of type bool but lhs
ERROR(105): 'and' requires operands of type bool but rhs        ERROR(105): 'and' requires operands of type bool but rhs
ERROR(111): Cannot use function 'cat' as a simple variab        ERROR(111): Cannot use function 'cat' as a simple variab
ERROR(111): Array 'aa' should be indexed by type int but        ERROR(111): Array 'aa' should be indexed by type int but
ERROR(113): Array 'aa' should be indexed by type int but        ERROR(113): Array 'aa' should be indexed by type int but
ERROR(113): Array index is the unindexed array 'aa'.                ERROR(113): Array index is the unindexed array 'aa'.
ERROR(114): Symbol 'AA' is not defined.                                ERROR(114): Symbol 'AA' is not defined.
ERROR(116): Symbol 'meerkat' is not defined.                        ERROR(116): Symbol 'meerkat' is not defined.
ERROR(118): Symbol 'xyzzy' is not defined.                        ERROR(118): Symbol 'xyzzy' is not defined.
ERROR(119): Symbol 'meerkat' is not defined.                        ERROR(119): Symbol 'meerkat' is not defined.
ERROR(120): Symbol 'xyzzy' is not defined.                        ERROR(120): Symbol 'xyzzy' is not defined.
ERROR(120): 'and' requires operands of type bool but rhs        ERROR(120): 'and' requires operands of type bool but rhs
ERROR(122): Symbol 'flight' is not defined.                        ERROR(122): Symbol 'flight' is not defined.
ERROR(122): Unary 'not' requires an operand of type bool        ERROR(122): Unary 'not' requires an operand of type bool
ERROR(122): Unary '-' requires an operand of type int bu        ERROR(122): Unary '-' requires an operand of type int bu
ERROR(122): Symbol 'uu' is not defined.                                ERROR(122): Symbol 'uu' is not defined.
ERROR(122): Symbol 'y' is not defined.                                ERROR(122): Symbol 'y' is not defined.
ERROR(122): '*' requires operands of type int but lhs is        ERROR(122): '*' requires operands of type int but lhs is
ERROR(122): '+' requires operands of type int but lhs is        ERROR(122): '+' requires operands of type int but lhs is
ERROR(124): 'x' is a simple variable and cannot be calle        ERROR(124): 'x' is a simple variable and cannot be calle
ERROR(126): Expecting type char in parameter 1 of call t        ERROR(126): Expecting type char in parameter 1 of call t
ERROR(126): Too many parameters passed for function 'dog        ERROR(126): Too many parameters passed for function 'dog
ERROR(127): Too few parameters passed for function 'ibex        ERROR(127): Too few parameters passed for function 'ibex
ERROR(129): Expecting type char in parameter 1 of call t        ERROR(129): Expecting type char in parameter 1 of call t
ERROR(131): Not expecting array in parameter 1 of call t        ERROR(131): Not expecting array in parameter 1 of call t
ERROR(133): Expecting array in parameter 1 of call to 'i        ERROR(133): Expecting array in parameter 1 of call to 'i
ERROR(134): Expecting type char in parameter 1 of call t        ERROR(134): Expecting type char in parameter 1 of call t
ERROR(135): Cannot use function 'ibex' as a simple varia        ERROR(135): Cannot use function 'ibex' as a simple varia
ERROR(137): Cannot use function 'ibex' as a simple varia        ERROR(137): Cannot use function 'ibex' as a simple varia
ERROR(139): Array 'aa' should be indexed by type int but        ERROR(139): Array 'aa' should be indexed by type int but
ERROR(140): Array 'aa' should be indexed by type int but        ERROR(140): Array 'aa' should be indexed by type int but
ERROR(140): Array index is the unindexed array 'zz'.                ERROR(140): Array index is the unindexed array 'zz'.
ERROR(141): Expecting type char in parameter 1 of call t        ERROR(141): Expecting type char in parameter 1 of call t
ERROR(141): Array 'aa' should be indexed by type int but        ERROR(141): Array 'aa' should be indexed by type int but
ERROR(143): Expecting type char in parameter 1 of call t        ERROR(143): Expecting type char in parameter 1 of call t
ERROR(143): '=' requires operands of the same type but l        ERROR(143): '=' requires operands of the same type but l
ERROR(144): Expecting type char in parameter 1 of call t        ERROR(144): Expecting type char in parameter 1 of call t
ERROR(144): '+' requires operands of type int but lhs is        ERROR(144): '+' requires operands of type int but lhs is
ERROR(144): '+' requires operands of type int but rhs is        ERROR(144): '+' requires operands of type int but rhs is
ERROR(144): '=' requires operands of the same type but l        ERROR(144): '=' requires operands of the same type but l
ERROR(145): Expecting type char in parameter 1 of call t        ERROR(145): Expecting type char in parameter 1 of call t
ERROR(145): Expecting type char in parameter 1 of call t        ERROR(145): Expecting type char in parameter 1 of call t
ERROR(146): Expecting type char in parameter 1 of call t        ERROR(146): Expecting type char in parameter 1 of call t
ERROR(147): Expecting type char in parameter 1 of call t        ERROR(147): Expecting type char in parameter 1 of call t
ERROR(149): '=' requires operands of the same type but l        ERROR(149): '=' requires operands of the same type but l
ERROR(151): '=' requires operands of the same type but l        ERROR(151): '=' requires operands of the same type but l
ERROR(151): '=' requires operands of the same type but l        ERROR(151): '=' requires operands of the same type but l
ERROR(151): '=' requires operands of the same type but l        ERROR(151): '=' requires operands of the same type but l
ERROR(153): '*' requires operands of type int but lhs is        ERROR(153): '*' requires operands of type int but lhs is
ERROR(153): '*' requires operands of type int but rhs is        ERROR(153): '*' requires operands of type int but rhs is
ERROR(153): 'and' requires operands of type bool but lhs        ERROR(153): 'and' requires operands of type bool but lhs
ERROR(153): 'and' requires operands of type bool but rhs        ERROR(153): 'and' requires operands of type bool but rhs
ERROR(155): '+' requires operands of type int but lhs is        ERROR(155): '+' requires operands of type int but lhs is
ERROR(155): Expecting type char in parameter 1 of call t        ERROR(155): Expecting type char in parameter 1 of call t
ERROR(155): Symbol 'parrot' is not defined.                        ERROR(155): Symbol 'parrot' is not defined.
ERROR(155): 'and' requires operands of type bool but lhs        ERROR(155): 'and' requires operands of type bool but lhs
ERROR(155): 'and' requires operands of type bool but rhs        ERROR(155): 'and' requires operands of type bool but rhs
ERROR(155): Expecting type char in parameter 3 of call t        ERROR(155): Expecting type char in parameter 3 of call t
ERROR(156): Expecting type char in parameter 1 of call t        ERROR(156): Expecting type char in parameter 1 of call t
ERROR(156): '+' requires operands of type int but lhs is        ERROR(156): '+' requires operands of type int but lhs is
ERROR(156): Expecting type char in parameter 1 of call t        ERROR(156): Expecting type char in parameter 1 of call t
ERROR(156): Symbol 'parrot' is not defined.                        ERROR(156): Symbol 'parrot' is not defined.
ERROR(156): 'and' requires operands of type bool but lhs        ERROR(156): 'and' requires operands of type bool but lhs
ERROR(156): 'and' requires operands of type bool but rhs        ERROR(156): 'and' requires operands of type bool but rhs
ERROR(156): Expecting type char in parameter 3 of call t        ERROR(156): Expecting type char in parameter 3 of call t
ERROR(156): 'and' requires operands of type bool but rhs        ERROR(156): 'and' requires operands of type bool but rhs
ERROR(156): Expecting type char in parameter 2 of call t        ERROR(156): Expecting type char in parameter 2 of call t
ERROR(156): Expecting type char in parameter 3 of call t        ERROR(156): Expecting type char in parameter 3 of call t
ERROR(156): Too many parameters passed for function 'emu        ERROR(156): Too many parameters passed for function 'emu
ERROR(156): '*' requires operands of type int but lhs is        ERROR(156): '*' requires operands of type int but lhs is
ERROR(158): Cannot return an array.                                ERROR(158): Cannot return an array.
Number of warnings: 1                                                Number of warnings: 1
Number of errors: 132                                                Number of errors: 132
RUN: c- basicExtra.c-                                                RUN: c- basicExtra.c-
WARNING(3): Expecting to return type char but function '        WARNING(3): Expecting to return type char but function '
ERROR(14): '<' requires operands of type char or type in        ERROR(14): '<' requires operands of type char or type in
ERROR(14): '<' requires operands of type char or type in        ERROR(14): '<' requires operands of type char or type in
ERROR(14): The operation '<' does not work with arrays.                ERROR(14): The operation '<' does not work with arrays.
ERROR(15): The operation '<' does not work with arrays.                ERROR(15): The operation '<' does not work with arrays.
ERROR(18): '<' requires operands of type char or type in        ERROR(18): '<' requires operands of type char or type in
ERROR(18): '<' requires operands of type char or type in        ERROR(18): '<' requires operands of type char or type in
ERROR(19): '/' requires operands of type int but lhs is         ERROR(19): '/' requires operands of type int but lhs is 
ERROR(20): '/' requires operands of type int but rhs is         ERROR(20): '/' requires operands of type int but rhs is 
ERROR(21): '\%' requires operands of type int but lhs is         ERROR(21): '\%' requires operands of type int but lhs is 
ERROR(22): '\%' requires operands of type int but rhs is         ERROR(22): '\%' requires operands of type int but rhs is 
ERROR(23): '-' requires operands of type int but lhs is         ERROR(23): '-' requires operands of type int but lhs is 
ERROR(24): '-' requires operands of type int but rhs is         ERROR(24): '-' requires operands of type int but rhs is 
ERROR(25): '<' requires operands of type char or type in        ERROR(25): '<' requires operands of type char or type in
ERROR(25): '<' requires operands of type char or type in        ERROR(25): '<' requires operands of type char or type in
ERROR(27): '<' requires operands of type char or type in        ERROR(27): '<' requires operands of type char or type in
ERROR(28): '<' requires operands of type char or type in        ERROR(28): '<' requires operands of type char or type in
ERROR(30): Symbol 'k' is not defined.                                ERROR(30): Symbol 'k' is not defined.
ERROR(31): Symbol 'k' is not defined.                                ERROR(31): Symbol 'k' is not defined.
ERROR(32): Symbol 'k' is not defined.                                ERROR(32): Symbol 'k' is not defined.
ERROR(32): Symbol 'k' is not defined.                                ERROR(32): Symbol 'k' is not defined.
ERROR(33): Symbol 'k' is not defined.                                ERROR(33): Symbol 'k' is not defined.
ERROR(34): 'i' is a simple variable and cannot be called        ERROR(34): 'i' is a simple variable and cannot be called
Number of warnings: 1                                                Number of warnings: 1
Number of errors: 22                                                Number of errors: 22
RUN: c- bullsandcows.c-                                                RUN: c- bullsandcows.c-
Number of warnings: 0                                                Number of warnings: 0
Number of errors: 0                                                Number of errors: 0
RUN: c- call.c-                                                        RUN: c- call.c-
WARNING(1): Expecting to return type int but function 's        WARNING(1): Expecting to return type int but function 's
WARNING(7): Expecting to return type int but function 's        WARNING(7): Expecting to return type int but function 's
WARNING(18): Expecting to return type int but function '        WARNING(18): Expecting to return type int but function '
Number of warnings: 3                                                Number of warnings: 3
Number of errors: 0                                                Number of errors: 0
RUN: c- call2.c-                                                RUN: c- call2.c-
Number of warnings: 0                                                Number of warnings: 0
Number of errors: 0                                                Number of errors: 0
RUN: c- call3.c-                                                RUN: c- call3.c-
ERROR(4): Symbol 'fred' is not defined.                                ERROR(4): Symbol 'fred' is not defined.
ERROR(4): Symbol 'x' is not defined.                                ERROR(4): Symbol 'x' is not defined.
ERROR(4): Symbol 'y' is not defined.                                ERROR(4): Symbol 'y' is not defined.
ERROR(4): 'and' requires operands of type bool but lhs i        ERROR(4): 'and' requires operands of type bool but lhs i
ERROR(4): 'and' requires operands of type bool but rhs i        ERROR(4): 'and' requires operands of type bool but rhs i
ERROR(6): Function 'main' at line 1 is expecting no retu        ERROR(6): Function 'main' at line 1 is expecting no retu
ERROR(12): Too many parameters passed for function 'fred        ERROR(12): Too many parameters passed for function 'fred
ERROR(12): The operation '*' only works with arrays.                ERROR(12): The operation '*' only works with arrays.
ERROR(14): Too few parameters passed for function 'fred'        ERROR(14): Too few parameters passed for function 'fred'
ERROR(13): Expecting array in parameter 1 of call to 'fr        ERROR(13): Expecting array in parameter 1 of call to 'fr
ERROR(15): Expecting type char in parameter 1 of call to        ERROR(15): Expecting type char in parameter 1 of call to
WARNING(9): Expecting to return type char but function '        WARNING(9): Expecting to return type char but function '
ERROR(20): 'x' is a simple variable and cannot be called        ERROR(20): 'x' is a simple variable and cannot be called
ERROR(20): '+' requires operands of type int but rhs is         ERROR(20): '+' requires operands of type int but rhs is 
ERROR(20): The operation '*' only works with arrays.                ERROR(20): The operation '*' only works with arrays.
ERROR(20): 'and' requires operands of type bool but lhs         ERROR(20): 'and' requires operands of type bool but lhs 
ERROR(20): 'and' requires operands of type bool but rhs         ERROR(20): 'and' requires operands of type bool but rhs 
ERROR(21): Symbol 'y' is not defined.                                ERROR(21): Symbol 'y' is not defined.
ERROR(21): '+' requires operands of type int but rhs is         ERROR(21): '+' requires operands of type int but rhs is 
ERROR(21): The operation '*' only works with arrays.                ERROR(21): The operation '*' only works with arrays.
ERROR(21): 'and' requires operands of type bool but lhs         ERROR(21): 'and' requires operands of type bool but lhs 
ERROR(21): 'and' requires operands of type bool but rhs         ERROR(21): 'and' requires operands of type bool but rhs 
ERROR(22): 'z' is a simple variable and cannot be called        ERROR(22): 'z' is a simple variable and cannot be called
ERROR(22): '+' requires operands of type int but rhs is         ERROR(22): '+' requires operands of type int but rhs is 
ERROR(22): The operation '*' only works with arrays.                ERROR(22): The operation '*' only works with arrays.
ERROR(22): 'and' requires operands of type bool but lhs         ERROR(22): 'and' requires operands of type bool but lhs 
ERROR(22): 'and' requires operands of type bool but rhs         ERROR(22): 'and' requires operands of type bool but rhs 
ERROR(25): Symbol 'output' is already defined at line -1        ERROR(25): Symbol 'output' is already defined at line -1
ERROR(26): Symbol 'outputb' is already defined at line -        ERROR(26): Symbol 'outputb' is already defined at line -
ERROR(27): Symbol 'outputc' is already defined at line -        ERROR(27): Symbol 'outputc' is already defined at line -
ERROR(29): Symbol 'input' is already defined at line -1.        ERROR(29): Symbol 'input' is already defined at line -1.
ERROR(30): Symbol 'inputb' is already defined at line -1        ERROR(30): Symbol 'inputb' is already defined at line -1
ERROR(31): Symbol 'inputc' is already defined at line -1        ERROR(31): Symbol 'inputc' is already defined at line -1
ERROR(43): Symbol 'x' is already defined at line 42.                ERROR(43): Symbol 'x' is already defined at line 42.
WARNING(42): Expecting to return type int but function '        WARNING(42): Expecting to return type int but function '
ERROR(50): Symbol 'AlanTuring' is already defined at lin        ERROR(50): Symbol 'AlanTuring' is already defined at lin
ERROR(54): Symbol 'x' is already defined at line 52.           <
WARNING(50): Expecting to return type int but function '        WARNING(50): Expecting to return type int but function '
Number of warnings: 3                                                Number of warnings: 3
Number of errors: 35                                           |        Number of errors: 34
RUN: c- chars.c-                                                RUN: c- chars.c-
Number of warnings: 0                                                Number of warnings: 0
Number of errors: 0                                                Number of errors: 0
RUN: c- everything02.c-                                                RUN: c- everything02.c-
Number of warnings: 0                                                Number of warnings: 0
Number of errors: 0                                                Number of errors: 0
RUN: c- factor.c-                                                RUN: c- factor.c-
Number of warnings: 0                                                Number of warnings: 0
Number of errors: 0                                                Number of errors: 0
RUN: c- factorial.c-                                                RUN: c- factorial.c-
Number of warnings: 0                                                Number of warnings: 0
Number of errors: 0                                                Number of errors: 0
RUN: c- init.c-                                                        RUN: c- init.c-
ERROR(15): Symbol 'y' is already defined at line 4.           <
ERROR(17): Cannot index nonarray 'y'.                           <
ERROR(17): '=' requires operands of the same type but lh   <
Number of warnings: 0                                                Number of warnings: 0
Number of errors: 3                                           |        Number of errors: 0
RUN: c- moutest.c-                                                RUN: c- moutest.c-
WARNING(5): Expecting to return type int but function 'e        WARNING(5): Expecting to return type int but function 'e
ERROR(19): Array 'aa' should be indexed by type int but         ERROR(19): Array 'aa' should be indexed by type int but 
ERROR(20): Array index is the unindexed array 'dd'.                ERROR(20): Array index is the unindexed array 'dd'.
ERROR(17): Array 'cc' should be indexed by type int but         ERROR(17): Array 'cc' should be indexed by type int but 
ERROR(25): Array 'aa' should be indexed by type int but         ERROR(25): Array 'aa' should be indexed by type int but 
ERROR(23): Array 'cc' should be indexed by type int but         ERROR(23): Array 'cc' should be indexed by type int but 
ERROR(30): Array 'cc' should be indexed by type int but         ERROR(30): Array 'cc' should be indexed by type int but 
ERROR(39): The operation '*' only works with arrays.                ERROR(39): The operation '*' only works with arrays.
ERROR(36): Array 'cc' should be indexed by type int but         ERROR(36): Array 'cc' should be indexed by type int but 
ERROR(42): Expecting type int in parameter 1 of call to         ERROR(42): Expecting type int in parameter 1 of call to 
ERROR(43): '+' requires operands of type int but rhs is         ERROR(43): '+' requires operands of type int but rhs is 
ERROR(44): '+' requires operands of type int but rhs is         ERROR(44): '+' requires operands of type int but rhs is 
ERROR(45): '+' requires operands of type int but rhs is         ERROR(45): '+' requires operands of type int but rhs is 
ERROR(46): Symbol 'parrot' is not defined.                        ERROR(46): Symbol 'parrot' is not defined.
ERROR(47): 'and' requires operands of type bool but rhs         ERROR(47): 'and' requires operands of type bool but rhs 
ERROR(45): Expecting type int in parameter 3 of call to         ERROR(45): Expecting type int in parameter 3 of call to 
ERROR(48): 'and' requires operands of type bool but rhs         ERROR(48): 'and' requires operands of type bool but rhs 
ERROR(44): Expecting type int in parameter 3 of call to         ERROR(44): Expecting type int in parameter 3 of call to 
ERROR(49): 'and' requires operands of type bool but rhs         ERROR(49): 'and' requires operands of type bool but rhs 
ERROR(43): Expecting type int in parameter 3 of call to         ERROR(43): Expecting type int in parameter 3 of call to 
ERROR(43): 'and' requires operands of type bool but rhs         ERROR(43): 'and' requires operands of type bool but rhs 
ERROR(42): Expecting type int in parameter 2 of call to         ERROR(42): Expecting type int in parameter 2 of call to 
ERROR(42): Expecting type int in parameter 3 of call to         ERROR(42): Expecting type int in parameter 3 of call to 
ERROR(42): Too many parameters passed for function 'emu'        ERROR(42): Too many parameters passed for function 'emu'
ERROR(51): '*' requires operands of type int but rhs is         ERROR(51): '*' requires operands of type int but rhs is 
Number of warnings: 1                                                Number of warnings: 1
Number of errors: 24                                                Number of errors: 24
RUN: c- ryantest.c-                                                RUN: c- ryantest.c-
ERROR(7): Initializer for variable 'x' is not a constant        ERROR(7): Initializer for variable 'x' is not a constant
ERROR(7): Symbol 'x' is already defined at line 5.                ERROR(7): Symbol 'x' is already defined at line 5.
ERROR(18): '\texttt{=' requires operands of the same type but l	ERROR(18): '=}' requires operands of the same type but l
ERROR(18): '\texttt{=' requires that if one operand is an array	ERROR(18): '=}' requires that if one operand is an array
ERROR(19): 'and' requires operands of type bool but lhs         ERROR(19): 'and' requires operands of type bool but lhs 
ERROR(19): 'and' requires operands of type bool but rhs         ERROR(19): 'and' requires operands of type bool but rhs 
ERROR(19): The operation 'and' does not work with arrays        ERROR(19): The operation 'and' does not work with arrays
ERROR(35): '=' requires operands of the same type but lh        ERROR(35): '=' requires operands of the same type but lh
ERROR(34): '=' requires operands of the same type but lh        ERROR(34): '=' requires operands of the same type but lh
ERROR(38): Symbol 'undef' is not defined.                        ERROR(38): Symbol 'undef' is not defined.
ERROR(37): '=' requires operands of the same type but lh        ERROR(37): '=' requires operands of the same type but lh
ERROR(41): Symbol 'undef' is not defined.                        ERROR(41): Symbol 'undef' is not defined.
ERROR(44): Symbol 'undef' is not defined.                        ERROR(44): Symbol 'undef' is not defined.
ERROR(47): '=' requires operands of the same type but lh        ERROR(47): '=' requires operands of the same type but lh
ERROR(46): '=' requires operands of the same type but lh        ERROR(46): '=' requires operands of the same type but lh
ERROR(46): '+' requires operands of type int but lhs is         ERROR(46): '+' requires operands of type int but lhs is 
ERROR(50): Symbol 'undef' is not defined.                        ERROR(50): Symbol 'undef' is not defined.
ERROR(49): '+' requires operands of type int but lhs is         ERROR(49): '+' requires operands of type int but lhs is 
ERROR(53): Symbol 'undef' is not defined.                        ERROR(53): Symbol 'undef' is not defined.
ERROR(52): '=' requires operands of the same type but lh        ERROR(52): '=' requires operands of the same type but lh
ERROR(52): '+' requires operands of type int but lhs is         ERROR(52): '+' requires operands of type int but lhs is 
ERROR(56): '=' requires operands of the same type but lh        ERROR(56): '=' requires operands of the same type but lh
ERROR(55): '=' requires operands of the same type but lh        ERROR(55): '=' requires operands of the same type but lh
ERROR(55): '+' requires operands of type int but lhs is         ERROR(55): '+' requires operands of type int but lhs is 
ERROR(64): Symbol 'foo' is not defined.                                ERROR(64): Symbol 'foo' is not defined.
ERROR(65): Symbol 'foo' is not defined.                                ERROR(65): Symbol 'foo' is not defined.
ERROR(66): Symbol 'foo' is not defined.                                ERROR(66): Symbol 'foo' is not defined.
ERROR(67): Symbol 'foo' is not defined.                                ERROR(67): Symbol 'foo' is not defined.
ERROR(68): Symbol 'foo' is not defined.                                ERROR(68): Symbol 'foo' is not defined.
ERROR(68): Symbol 'foo' is not defined.                                ERROR(68): Symbol 'foo' is not defined.
ERROR(69): 'x' is a simple variable and cannot be called        ERROR(69): 'x' is a simple variable and cannot be called
ERROR(69): '*' requires operands of type int but lhs is         ERROR(69): '*' requires operands of type int but lhs is 
ERROR(69): Cannot use function 'test' as a simple variab        ERROR(69): Cannot use function 'test' as a simple variab
ERROR(71): Cannot use function 'check' as a simple varia        ERROR(71): Cannot use function 'check' as a simple varia
ERROR(72): Cannot use function 'check' as a simple varia        ERROR(72): Cannot use function 'check' as a simple varia
ERROR(72): 'x' is a simple variable and cannot be called        ERROR(72): 'x' is a simple variable and cannot be called
ERROR(73): Symbol 'c' is not defined.                                ERROR(73): Symbol 'c' is not defined.
ERROR(74): Symbol 'c' is not defined.                                ERROR(74): Symbol 'c' is not defined.
ERROR(76): Symbol 'foo' is not defined.                                ERROR(76): Symbol 'foo' is not defined.
ERROR(78): Cannot index nonarray 'x'.                                ERROR(78): Cannot index nonarray 'x'.
ERROR(79): Cannot index nonarray 'x'.                                ERROR(79): Cannot index nonarray 'x'.
ERROR(80): Symbol 'y' is not defined.                                ERROR(80): Symbol 'y' is not defined.
ERROR(82): Cannot use function 'check' as a simple varia        ERROR(82): Cannot use function 'check' as a simple varia
ERROR(83): 'x' is a simple variable and cannot be called        ERROR(83): 'x' is a simple variable and cannot be called
ERROR(86): '=' requires operands of the same type but lh        ERROR(86): '=' requires operands of the same type but lh
ERROR(87): '=' requires operands of the same type but lh        ERROR(87): '=' requires operands of the same type but lh
ERROR(87): Expecting type int in parameter 2 of call to         ERROR(87): Expecting type int in parameter 2 of call to 
ERROR(88): Symbol 'y' is not defined.                                ERROR(88): Symbol 'y' is not defined.
ERROR(88): Too many parameters passed for function 'chec        ERROR(88): Too many parameters passed for function 'chec
ERROR(88): Too few parameters passed for function 'check        ERROR(88): Too few parameters passed for function 'check
ERROR(89): Expecting array in parameter 2 of call to 'fu        ERROR(89): Expecting array in parameter 2 of call to 'fu
ERROR(90): Not expecting array in parameter 1 of call to        ERROR(90): Not expecting array in parameter 1 of call to
ERROR(92): '-' requires operands of type int but rhs is         ERROR(92): '-' requires operands of type int but rhs is 
ERROR(91): Expecting array in parameter 2 of call to 'fu        ERROR(91): Expecting array in parameter 2 of call to 'fu
ERROR(91): Too many parameters passed for function 'func        ERROR(91): Too many parameters passed for function 'func
ERROR(94): Expecting type int in parameter 1 of call to         ERROR(94): Expecting type int in parameter 1 of call to 
ERROR(94): Not expecting array in parameter 1 of call to        ERROR(94): Not expecting array in parameter 1 of call to
ERROR(94): Too few parameters passed for function 'func'        ERROR(94): Too few parameters passed for function 'func'
ERROR(99): Initializer for variable 'x' is not a constan        ERROR(99): Initializer for variable 'x' is not a constan
ERROR(99): Symbol 'x' is already defined at line 97.                ERROR(99): Symbol 'x' is already defined at line 97.
ERROR(100): Symbol 'x' is already defined at line 97.                ERROR(100): Symbol 'x' is already defined at line 97.
ERROR(101): Variable 'x' is of type char but is being in        ERROR(101): Variable 'x' is of type char but is being in
ERROR(101): Symbol 'x' is already defined at line 97.                ERROR(101): Symbol 'x' is already defined at line 97.
ERROR(102): Symbol 'x' is already defined at line 97.                ERROR(102): Symbol 'x' is already defined at line 97.
ERROR(104): Symbol 'z' is already defined at line 103.                ERROR(104): Symbol 'z' is already defined at line 103.
ERROR(105): Initializer for variable 'y' is not a consta        ERROR(105): Initializer for variable 'y' is not a consta
ERROR(106): Initializer for variable 'a' is not a consta        ERROR(106): Initializer for variable 'a' is not a consta
ERROR(106): Initializer for nonarray variable 'a' of typ        ERROR(106): Initializer for nonarray variable 'a' of typ
ERROR(107): Initializer for variable 'b' is not a consta        ERROR(107): Initializer for variable 'b' is not a consta
ERROR(108): Initializer for variable 'd' is not a consta        ERROR(108): Initializer for variable 'd' is not a consta
ERROR(109): Variable 'e' is of type int but is being ini        ERROR(109): Variable 'e' is of type int but is being ini
ERROR(109): Initializer for nonarray variable 'e' of typ        ERROR(109): Initializer for nonarray variable 'e' of typ
ERROR(110): Array 'f' must be of type char to be initial        ERROR(110): Array 'f' must be of type char to be initial
ERROR(111): Initializer for nonarray variable 'g' of typ        ERROR(111): Initializer for nonarray variable 'g' of typ
ERROR(113): Initializer for variable 'i' is not a consta        ERROR(113): Initializer for variable 'i' is not a consta
ERROR(114): Initializer for variable 'j' is not a consta        ERROR(114): Initializer for variable 'j' is not a consta
ERROR(115): Initializer for variable 'k' is not a consta        ERROR(115): Initializer for variable 'k' is not a consta
ERROR(115): Variable 'k' is of type int but is being ini        ERROR(115): Variable 'k' is of type int but is being ini
ERROR(116): Initializer for variable 'j' is not a consta        ERROR(116): Initializer for variable 'j' is not a consta
ERROR(116): Symbol 'j' is already defined at line 114.                ERROR(116): Symbol 'j' is already defined at line 114.
ERROR(118): Initializer for variable 'j2' is not a const        ERROR(118): Initializer for variable 'j2' is not a const
                                                           >        ERROR(127): Function 'retCheck' at line 125 is expecting
ERROR(129): Expecting array in parameter 2 of call to 'r        ERROR(129): Expecting array in parameter 2 of call to 'r
ERROR(130): Cannot return an array.                                ERROR(130): Cannot return an array.
ERROR(127): Function 'retCheck' at line 125 is expecting   <
Number of warnings: 0                                                Number of warnings: 0
Number of errors: 84                                                Number of errors: 84
RUN: c- tictactoe.c-                                                RUN: c- tictactoe.c-
Number of warnings: 0                                                Number of warnings: 0
Number of errors: 0                                                Number of errors: 0
RUN: c- undef.c-                                                RUN: c- undef.c-
WARNING(3): Expecting to return type int but function 'i        WARNING(3): Expecting to return type int but function 'i
ERROR(10): Symbol 'undef' is not defined.                        ERROR(10): Symbol 'undef' is not defined.
ERROR(10): Initializer for variable 'x' is not a constan        ERROR(10): Initializer for variable 'x' is not a constan
ERROR(10): Symbol 'x' is already defined at line 9.                ERROR(10): Symbol 'x' is already defined at line 9.
ERROR(14): Symbol 'undef' is not defined.                        ERROR(14): Symbol 'undef' is not defined.
ERROR(15): '+' requires operands of type int but rhs is         ERROR(15): '+' requires operands of type int but rhs is 
ERROR(16): Symbol 'undef' is not defined.                        ERROR(16): Symbol 'undef' is not defined.
ERROR(17): '+' requires operands of type int but lhs is         ERROR(17): '+' requires operands of type int but lhs is 
ERROR(18): Symbol 'undef' is not defined.                        ERROR(18): Symbol 'undef' is not defined.
ERROR(18): Symbol 'undef' is not defined.                        ERROR(18): Symbol 'undef' is not defined.
ERROR(19): '+' requires operands of type int but lhs is         ERROR(19): '+' requires operands of type int but lhs is 
ERROR(19): '+' requires operands of type int but rhs is         ERROR(19): '+' requires operands of type int but rhs is 
ERROR(22): Symbol 'undef' is not defined.                        ERROR(22): Symbol 'undef' is not defined.
ERROR(23): '+' requires operands of type int but rhs is         ERROR(23): '+' requires operands of type int but rhs is 
ERROR(24): Symbol 'undef' is not defined.                        ERROR(24): Symbol 'undef' is not defined.
ERROR(25): '+' requires operands of type int but lhs is         ERROR(25): '+' requires operands of type int but lhs is 
ERROR(26): Symbol 'undef' is not defined.                        ERROR(26): Symbol 'undef' is not defined.
ERROR(26): Symbol 'undef' is not defined.                        ERROR(26): Symbol 'undef' is not defined.
ERROR(27): '+' requires operands of type int but lhs is         ERROR(27): '+' requires operands of type int but lhs is 
ERROR(27): '+' requires operands of type int but rhs is         ERROR(27): '+' requires operands of type int but rhs is 
ERROR(29): Symbol 'undef' is not defined.                        ERROR(29): Symbol 'undef' is not defined.
ERROR(30): Symbol 'undef' is not defined.                        ERROR(30): Symbol 'undef' is not defined.
ERROR(31): Symbol 'undef' is not defined.                        ERROR(31): Symbol 'undef' is not defined.
ERROR(32): Symbol 'undef' is not defined.                        ERROR(32): Symbol 'undef' is not defined.
ERROR(34): Symbol 'undef' is not defined.                        ERROR(34): Symbol 'undef' is not defined.
ERROR(34): '+' requires operands of type int but rhs is         ERROR(34): '+' requires operands of type int but rhs is 
ERROR(34): '\texttt{=' requires operands of the same type but l	ERROR(34): '=}' requires operands of the same type but l
ERROR(34): The operation '*' only works with arrays.                ERROR(34): The operation '*' only works with arrays.
ERROR(35): 'x' is a simple variable and cannot be called        ERROR(35): 'x' is a simple variable and cannot be called
ERROR(35): '+' requires operands of type int but rhs is         ERROR(35): '+' requires operands of type int but rhs is 
ERROR(35): '\texttt{=' requires operands of the same type but l	ERROR(35): '=}' requires operands of the same type but l
ERROR(35): The operation '*' only works with arrays.                ERROR(35): The operation '*' only works with arrays.
ERROR(37): Cannot use function 'novalue' as a simple var        ERROR(37): Cannot use function 'novalue' as a simple var
ERROR(37): '=' requires operands of the same type but lh        ERROR(37): '=' requires operands of the same type but lh
ERROR(38): Cannot use function 'intvalue' as a simple va        ERROR(38): Cannot use function 'intvalue' as a simple va
ERROR(40): Expecting type int in parameter 1 of call to         ERROR(40): Expecting type int in parameter 1 of call to 
ERROR(41): Cannot use function 'novalue' as a simple var        ERROR(41): Cannot use function 'novalue' as a simple var
ERROR(41): Expecting type int in parameter 1 of call to         ERROR(41): Expecting type int in parameter 1 of call to 
ERROR(42): Symbol 'undef' is not defined.                        ERROR(42): Symbol 'undef' is not defined.
ERROR(43): Symbol 'undef' is not defined.                        ERROR(43): Symbol 'undef' is not defined.
ERROR(45): Cannot use function 'intvalue' as a simple va        ERROR(45): Cannot use function 'intvalue' as a simple va
Number of warnings: 1                                                Number of warnings: 1
Number of errors: 40                                                Number of errors: 40
RUN: c- while03.c-                                                RUN: c- while03.c-
Number of warnings: 0                                                Number of warnings: 0
Number of errors: 0                                                Number of errors: 0

\textbf{/} End of testing                                                * End of testing

\chapter{罚单(Wrote on 20130916)}
\label{sec-31}

Dear Sir, 

The ticket situation back into that day was: I was taking a long drive from CA to my university, and I had been driving 450 miles already. It was a small town, I failed to notice the first "45" speed limit sign, and as far as noticed the second one, I slowed down immediately, but still the policeman chased me and gave me this ticket. 

I had been a bad driver years ago back in 2010 and 2011. I got a ticket from CA in Dec 2010 on my way back from ID to CA, and one in (Apr or May) 2011 in (some state) on my way from CA to my university. Those were all the long drives back and force between my university and CA. 

But after those two tickets I have changed. So on my way back from XX to CA in May 2011, I did not speed at all. And in the end of July 2012, when I decided to move back to XX from CA to continue for another master's degree, I drived safely all the way 60 mile / hour back. And on May 17th, 2013, I took all my time drive slowly and safely at the speed of 60 mile / hour a whole day from XX to CA trying to search for an internship, and I have no speeding neither. All the history since the second speeding ticket in my life proves that I have changed to be a good driver already.

I am an international student right now studying computer science in University of XXXXXX, and I am self-supported. The university did not provide any financial help for me at all, which means I have to pay more than \$10,000 per semester for my tuition fees. And these tuition fees have made me into trouble already. Besides, I have been forced to pay a more than \$1200 emergency ambulance fee from XXX and AAI this spring and recently, among which \$832.20 was for ambulence expence, and 50\% of it, \$416.10 for AAI collection fee. At that time I did't have health insurance, and the ambulence was called by the policeman, not me. My life has been made miserble enough by these government thing. And I have loans from my credit card and borrowed several thousand from my friend. Please help consider my situation. 

I mean I do realize I missed the first "45" speed limit sign, but as far as I noticed the second one, I slowed down immediately. Besides, all my cars, mitshibichi 96 before, and Buick 98 now, I stepped on the gas pad all the way even on high/free way. So it was possible that my speed came to be 61 mile on "55" IXX free way speed limit, but it won't be long. I would have reduced it immediately once I realized it from my meter or GPS. 

Please help consider my situation, and I will try my best continue to be a good driver with no speeding. You can reach me at (xxx) xxx-xxxx, (me\textasciitilde{}~)@gmail.com, or my current address is: 

XXX AAA str
XXXXX, XX 55555

Thank you in advance, and I look forward to hearing from you. 

(me\textasciitilde{}~)
\chapter{软件工程课结束}
\label{sec-32}
\section{软件工程课结束(1)}
\label{sec-32-1}

激情是什么,中考结束后,激情是放牛的孩子仰卧在青草地上,眼望着蔚蓝的天空中白云席卷而过,心中无限畅想着自已会考出怎样的成绩,自已真的会去那个县城里的一中么,还是远方会飞来一只野鹤,把我运往天外仙境,就是爸爸妈妈口中说过提到过的小表姐所在的美国、有个远亲的舅舅也在美国,他会管我么,他会乘鹤远来么?我钦点、细数、寻找着天空中飞鸟的痕迹。

后来再长大一点儿,就真正明白激情是每天早上醒来,都希望在接下来的一天能干成任何自己想要、喜欢做成的事情,是能够投入120\%的热情、集中精力投入这件事,我想我也还算是属于富有激情的性情中人吧,总希望自己能够兴奋地投入到自己有兴趣的工作中去\textasciitilde{}~

后来在新的专业领域,系里的大牛在某次他的课堂中提到,computer science的人要富有激情才对\textasciitilde{}~!大牛所提倡的激情钻研、探求科研高峰我一定会点赞的!想当年,lisp的那个作业我的500行代码,我也是在之前周五写了一晚上、交作业前的白天干了一整天,交作业前的晚上还在继续往前赶着写,一直写到了deadline前五小时的凌晨四点的啊;decision tree的时候我的office mate说我写得难写得痛苦就不要再写了,可只有坚持下去才有可能胜利啊\textasciitilde{}~ 同样的原理,考完编译课的那晚一一11月6日的晚上,当第二天我就需要向大牛汇报我的软件工程课项目时,我有技术上的难点,我就呆在网上使劲地搜,搜了一晚上,搜到了凌晨2:30才回家休息!第二天似乎校园里的人又都知道了我可能又写不出来作业了,居然干到了凌晨两点半,走在路上他们对我笑笑的\textasciitilde{}~

\section{软件工程课结束(2)}
\label{sec-32-2}

后来第二天,我去见了大牛,他是我们两个学生(另一个中国女孩已经去工作了,她大概已经不再需要这份成绩了)的软件工程课的代课老师。我告诉了他我目前技术上的难点,现有的两种解决方案为什么都达不到他的要求后,坐在大牛的办公室里,大牛用他的墙壁上的大屏幕,同我一起搜,搜了大概二三十分钟后,大牛充分肯定我的两点困难在这种语言里是没有解决方案的;他说他相信我编程序是没有问题的,只是我们一开始在选择这个项目的编程语时没有注意到这些个细节问题。大概也看我写编译课的作业写得痛苦吧,就要我不用再写这个项目的代码了,只要把我们春天没有写完的报告按照这样的计划实录写完就可以了。这样我的python代码就那么五六个嵌套的module就那么扔在那里了(程序过于简单,我就不再贴这里了)。

后来的期末考试是我从网上搜了一本软件工程的pdf,然后我对老师讲我想读这本pdf,也可以为将来的面试作准备;导师同意,我就爬回去读;可惜一直以来我对概念总是不太开窃,读得落花流水、稀里哗啦,结果最终见大牛review那天我把自己看过的有印象的概念、知识点漫天海聊地说了十分钟后,他又边翻边提问再讲解地把一些相对重要、容易混淆的概念帮我过了一遍这门课便算是结束了。倒是到下学期,准备面试的时候,那本pdf又被自己狠翻了几遍、还用latex作了一本术语索引手册,以备电话面试时不时之需!

这次见了大牛、这门课基本解脱之后,我还跑回去对同一个office里的小伙伴说,我把网上能搜到的解决方案都向大牛作过汇报,但一番半个小时的搜索后,就这种语言,我搜不到的解决方案他也搜不到没有办法,还很是为自己的搜索能力洋洋自得了一番\textasciitilde{}~ (汗!:)

此刻的自己也会去想,如果我当初是随大家一起选大课,我随大家一起作项目,我对陌生语言的恐惧会不会消失得早一些,我会不会成长得再快一点儿?现在这个项目对自己来说已经不是问题,但那时候的我们、自己何尝不是手足无措呢,要不然,怎么就偏偏选了个python来写,怎么看怎么不搭啊\textasciitilde{}~  smileface

\chapter{软件工程课邮件}
\label{sec-33}

\section{interview-question pdf}
\label{sec-33-1}
 (me\textasciitilde{}~) ((me\textasciitilde{}~)@XXXXXX.edu)
Sent:         Thursday, November 07, 2013 11:15 PM
To:        
cs502代课老师 [cs502代课老师@XXXXXX.edu]
Attachments:        
AC16\textunderscore sol.pdf‎ (730 KB‎)[Preview on web]
Hi, Dr. cs502代课老师, 

I have attached the pdf you just helped verified, and I will work on the design part documentations so that I  can get the writing part job done next week when we meet. 

thanks,
(me\textasciitilde{}~)


\section{UpdatedChofCmteform.doc}
\label{sec-33-2}
cs502代课老师
Sent:         Tuesday, October 22, 2013 11:28 PM
To:        
 (me\textasciitilde{}~)
Attachments:        
UpdatedChofCmteform.doc‎ (37 KB‎)[Preview on web]
Here is the form

-cs502代课老师

早就说过写得急的时候就难免会忘记内容,第二次大哭后的第二天,当我茫然无助地跑去找大牛的时候,因为别人为自己留了一条生路,所以为表达感激,我就对大牛说过,如果我有任何可能机会在这个国度生存下来,我会为系里、为那些真正有科研兴趣的老师捐点儿钱以示感谢\textasciitilde{}~ 大牛也还好说话,让我换导师后,还帮把换导师的表格发给了我。想来,那个学期的第二次哭应该是在这前一两天吧。


\section{From: cs502代课老师 (cs502代课老师@XXXXXX.edu)}
\label{sec-33-3}
Sent: Tuesday, December 10, 2013 5:16 PM
To: all his MS, ph.d students, course students
Subject: Reminder cs502代课老师 out of town

All,

Just a reminder that I am out of town this week. If you wish to meet with me during finals week, email me and we can pick a time; since final exams affect schedules.

-cs502代课老师


\section{cs502 meeting}
\label{sec-33-4}
 (me\textasciitilde{}~) ((me\textasciitilde{}~)@XXXXXX.edu)
Sent:         Wednesday, December 18, 2013 7:19 PM
To:        
cs502代课老师 (cs502代课老师@XXXXXX.edu)
Attachments:        
image001.gif‎ (356 B‎)

Dr. cs502代课老师, 

We have agreed to meet once more to make the final but I have not been able to write to you to schedule a meeting time yet. Tonight I will work in Bob's from 4pm, and on Friday 10:00-12:00am for cs520 exam, all other periods from today I am open. 

Please help pick an available period that works for you as well so that we can meet, and I will try my best to read more of that interview questions pdf to make the report. Please simply let me know when you have the schedule. 

thanks,
(me\textasciitilde{}~)


\section{RE: cs502 meeting}
\label{sec-33-5}
cs502代课老师 (cs502代课老师@XXXXXX.edu)
You replied on 12/19/2013 5:24 PM.
Sent:         Thursday, December 19, 2013 7:58 AM
To:        
 (me\textasciitilde{}~) ((me\textasciitilde{}~)@XXXXXX.edu)

How does 2pm Friday work for you?

-cs502代课老师


\section{RE: cs502 meeting}
\label{sec-33-6}
 (me\textasciitilde{}~) ((me\textasciitilde{}~)@XXXXXX.edu)
Sent:         Thursday, December 19, 2013 5:24 PM
To:        
cs502代课老师 (cs502代课老师@XXXXXX.edu)

Dr. cs502代课老师, 

2pm makes it so much convenient for me! By then I will have finished all my exams. Yeah I will meet you in your office at 2pm tomorrow then. 

thanks,
(me\textasciitilde{}~)

\chapter{hw2 \& hw3 Grades}
\label{sec-34}

编译课那个星期三期中考试,考完试的当晚,收到了代课老师发给我们的hw2、hw3的成绩。这成绩被老师压得太低了,低得完全超出了我的想像。我的第二次作业就只有一个变量设置的问题,第三次作业(hw3, not hw3b)我生成的msg是相对比较少,但搭好traverse framework的工作量也不该只是老师给出的这点儿分数啊?!!正如第二次作业后的邮件里、和教室里课堂上老师所表现出的,他试图自始自终就给我很低的成绩(为让我没有翻身的机会?),但都被我一再反水!现在这种情况下,我一收到邮件便就第一时间给老师发邮件,我想看成绩细节、知道评分标准,我要坚绝保证上完这门课时自己还是好好的(结果是,这门编译课结束后,我还是被“玩”残了\textasciitilde{}~!!)!

亲爱的读者,我很希望把自己的成绩都原封不动地贴出来,只可惜,很多我写过的邮件、local file都有至少一个版本,但那门课的成绩因为一直是贴在网上的,我竟然没有存,徒有指向成绩网址的空空如也的link而已!我想说,此时回忆这门课的自己,也感觉很遗憾\textasciitilde{}~

\chapter{hw2 \& hw3 Grades与代课老师的邮件}
\label{sec-35}

\section{Your CS445 Grades}
\label{sec-35-1}
cs445代课老师@XXXXXX.edu [cs445代课老师@XXXXXX.edu]
Sent:        Thursday, November 07, 2013 6:30 AM
To:        
 (me\textasciitilde{}~) ((me\textasciitilde{}~)@XXXXXX.edu)
Your email can be found at \url{http://ec2-54-200-16-181.us-west-2.compute.amazonaws.com/Results/msgs}-(me\textasciitilde{}~)-txt-eWwSe4MXrcii2vPSBpnDS

\section{Your CS445 Grades}
\label{sec-35-2}
cs445代课老师@XXXXXX.edu [cs445代课老师@XXXXXX.edu]
Sent:        Thursday, November 07, 2013 6:31 AM
To:        
 (me\textasciitilde{}~) ((me\textasciitilde{}~)@XXXXXX.edu)
Your email can be found at \url{http://ec2-54-200-16-181.us-west-2.compute.amazonaws.com/Results/msgs}-(me\textasciitilde{}~)-txt-fDMh1yp7EVJRDSkVucUXf

\section{from:         (me\textasciitilde{}~) <(me\textasciitilde{}~)@gmail.com>}
\label{sec-35-3}
to:         cs445代课老师 <captainbbbbbbb@gmail.com>
date:         Wed, Nov 6, 2013 at 10:44 PM
subject:         Hw2 \& hw3 grades  schedule appointment
mailed-by:         gmail.com

Hi Dr. cs445代课老师, 

I have been waiting for you during office hour today but I failed to see you. 

I received email regarding my grades for hw2 and hw3 a moment ago, and I feel the grades for me were way too low than I have expected. I want to schedule a time to meet you sometime tomorrow to discuss about the grades and potential make up to survive this course. 

I will be occupied tomorrow during 11:30-12:30 for cs520, 1:30-2:20pm for cs445, and 3:00-3:30pm for cs502. I will be open any other time from 9:00am- 6:00pm. Please let me know if you have any period open so that we can get good understanding about my grades. 

Thanks,
(me\textasciitilde{}~)

\section{from:         cs445代课老师 <captainbbbbbbb@gmail.com>}
\label{sec-35-4}
to:         (me\textasciitilde{}~) <(me\textasciitilde{}~)@gmail.com>
date:         Wed, Nov 6, 2013 at 11:18 PM
subject:         Re: Hw2 \& hw3 grades  schedule appointment
mailed-by:         gmail.com
signed-by:         gmail.com

\subsection{Quote: On Wed, Nov 6, 2013 at 10:44 PM, (me\textasciitilde{}~) <(me\textasciitilde{}~)@gmail.com> wrote:}
\label{sec-35-4-1}
Hi Dr. cs445代课老师, 

I have been waiting for you during office hour today but I failed to see you. 

I received email regarding my grades for hw2 and hw3 a moment ago, and I feel the grades for me were way too low than I have expected. I want to schedule a time to meet you sometime tomorrow to discuss about the grades and potential make up to survive this course. 

\subsection{Re: The only opening I have is at 4:30 tomorrow.}
\label{sec-35-4-2}

\section{from:         (me\textasciitilde{}~) <(me\textasciitilde{}~)@gmail.com>}
\label{sec-35-5}
to:         cs445代课老师 <captainbbbbbbb@gmail.com>
date:         Wed, Nov 6, 2013 at 11:25 PM
subject:         Re: Hw2 \& hw3 grades  schedule appointment
mailed-by:         gmail.com

Dr. cs445代课老师, 

Yes, I can do 4:30pm tomorrow. I will see you then in your office to discuss about hw2, hw3 grades, potential make up grades, and all possible solutions for surviving this course. 

If it is possible, please also help grade my exam for me so that we can have more information for me for this course. If you do not have time grading the exam by then, please just simple ignore this paragraph I wrote. 

I will see you tomorrow in your office at 4:30pm. 

thanks,
(me\textasciitilde{}~)

\chapter{期中考试后第一堂课}
\label{sec-36}
\section{期中考试后第一堂课(1)}
\label{sec-36-1}

小伙伴们或许还记得春季旁听导师算法课时,期中考试前对过于简单的知识点不厌其烦地讲了又讲,讲到聪明有悟性的学生需要装笨的程度\textasciitilde{}~ 那这门编译课的review呢?考前实在没有其它人提,后来坐我后面的一个美国男生问老师考前有review吗?老师对这个同学的提问显得鄙夷,同学们就不太敢讲话。所以最终考前是有review的,但是是以光前进的速度进行的,所以就半堂课的review时间等老师讲完了,我原本不懂的现在可能懂了个方向;我原本懂的大概已经不懂了\textasciitilde{}~ *0\^{} 而且真的的考试题目与review的内容并不是很相关。

那天是应该是周四吧(周三考试、周四还有课的),我们上午有编译课。这是期中考试考完后的第一堂课,老师上课第一句话是说,我就知道你们如果知道作业成绩后可能期中考试会考不好,所以等期中考试考完才把成绩发给你们!是啊,说得多么合情合理、人性化、尽显人文关怀,可实际上呢,我们也缺少了对自己两次作业的知情权,而且如果知道作业成绩,考试前会能准备得更充分的,我们同样也就缺少了准备期中考试的充分必要动力\textasciitilde{}~

\section{期中考试后第一堂课(2)}
\label{sec-36-2}

激情是什么,是在自己受到不公正打击时同敌人作最顽强的半争。想把别人的成绩就这么不明不白地压死也没有那么容易!哪怕最终我也还不得不死,那我也一定要尽自己的最大努力同敌人反抗到生命不息、战斗不止的最后一刻!我对自己说,从这一堂课起,我要主动,自己课堂上没听懂、听不懂的内容就当堂直接问老师,比如这第一堂课时我就问了,你前两堂课讲的framework,function return pointer, frame pointer的框架结构讲得不是很透彻,我还理解得不透,你能不能把这个框架结构再简单地REVIEW一遍?这是这学期这门编译课上自己第三次发言,虽然是以弱者的姿态。

我是班上仅有的两上活宝女生之一(另一个早提到过了,坐在我左边,基本上什么也不会,大概只能把她的第一次作业写得出来),老师这堂课是要主打提携女生、"宠爱"女生的牌么,还是借对我这样一个弱者姿态加以"宠爱"的方式来为他作这个学生很笨的广泛宣传增加力度和渗透性?我的这个问题提出后,代课老师像是保护国家一家保护动物般"呵护宠爱有加"地把这个compiler的大的架构结构从头到尾讲了整整一堂课。虽然我很清楚,这个我不太懂的问题我提问前班上真正理解的小伙伴未必过半,但像是一种情绪反溃,我还是直接顺了他的意,略带委屈地问他,你这堂课开始的时候说A不是某某状况,那你现在为什么又要去用A?言下之意,你现在讲的同你这堂课刚开始提出的那个什么什么观止是矛盾的呀?(对不起,上学期的上课笔记我没有带在身边,知识点的细节我已经忘记了,留在记忆里的更多的是自己心里那份不平,和这堂课前前后后提问的人、答题的人说话的语气腔调和同学们的情绪反应,大家一一包括我自己也都很讶异啊\textasciitilde{}~ 大概只有那个老头是个例外。)

\chapter{软件工程课最后步骤}
\label{sec-37}

\section{Login.py}
\label{sec-37-1}
\lstset{language=Python,label= ,caption= ,numbers=none}
\begin{lstlisting}
from Tkinter import *

import sys
sys.path.append('C:/502/')

class Login():
    def __init__(self, parent):
        top = self.top = Toplevel(parent)
        self.username = ""
        self.myLabel = Label(top, text = "Enter your UX Email address below:")
        self.myLabel.pack()
        self.entry = Entry(top, width = 30)
        self.entry.pack()
        self.button = Button(top, text="Submit", width = 10, command=self.on_button)
        self.button.pack()

    def getUsername(self):
        return self.username

    def on_button(self):
        self.username = self.entry.get()
        self.top.destroy()
\end{lstlisting}

\section{UserClass.py}
\label{sec-37-2}
\lstset{language=Python,label= ,caption= ,numbers=none}
\begin{lstlisting}
class UserClass():

    String_emailIdentifier = '@'
    String_emailVerify = 'XXXXXX.edu'
    String_studentIdentifier = '@XXXXXX.edu'
    String_instructorIdentifier = '@XXXXXX.edu'

    def __init__(self, String_email):
        self.String_email = String_email

    def isEmail(self):
        if ( self.String_email.find(UserClass.String_emailIdentifier) > -1 ):
            return True
        else:
            return False

    def isValidEmail(self):
        if self.isEmail():
            if ( self.String_email.find(UserClass.String_emailVerify)> -1 ):
                return True
            else:
                return False
        else:
            return False

    def isStudent(self):
        if self.isValidEmail():
            if (self.String_email.find(UserClass.String_studentIdentifier) > -1):
                return True
            else:
                return False
        else:
            return False

    def isInstructor(self):
        if self.isValidEmail():
            if (self.String_email.find(UserClass.String_instructorIdentifier) > -1):
                return True
            else:
                return False
        else:
            return False
\end{lstlisting}

\section{Dialog.py}
\label{sec-37-3}

\lstset{language=Python,label= ,caption= ,numbers=none}
\begin{lstlisting}
from Tkinter import *

import sys
sys.path.append('C:/502/')

class Dialog():
    def __init__(self, parent):
        top = self.top = Toplevel(parent)

        self.String_firstname = ""
        self.String_lastname = ""
        self.String_schedReason = ""

        self.Label_firstname = Label(top, text = "Firstname:")
        self.Label_firstname.pack()
        self.Entry_firstname = Entry(top, width = 30)
        self.Entry_firstname.pack()

        self.Label_lastname = Label(top, text = "Lastname:")
        self.Label_lastname.pack()
        self.Entry_lastname = Entry(top, width = 30)
        self.Entry_lastname.pack()

        self.Label_schedReason = Label(top, text = "Appointment Reason:")
        self.Label_schedReason.pack()
        self.Entry_schedReason = Entry(top, width = 100)
        self.Entry_schedReason.pack()

        self.button = Button(top, text="Submit", width = 10, command=self.on_button)
        self.button.pack()

    def getFirstname(self):
        return self.Entry_firstname

    def getLastname(self):
        return self.Entry_lastname

    def getSchedReason(self):
        return self.Entry_schedReason

    def on_button(self):
        self.Entry_firstname = self.Entry_firstname.get()
        self.Entry_lastname = self.Entry_lastname.get()
        self.Entry_schedReason = self.Entry_schedReason.get()
        self.top.destroy()
\end{lstlisting}

\section{CheckBox.py}
\label{sec-37-4}
\lstset{language=Python,label= ,caption= ,numbers=none}
\begin{lstlisting}
from Tkinter import *
import Tkinter

import sys
sys.path.append('C:/502/')

from Dialog import Dialog

class CheckBox(Frame):

    List_String_buttonText = ['8:00-8:30','8:30-9:00','9:00-9:30','9:30-10:00','10:00-10:30','10:30-11:00','11:00-11:30','11:30-12:00','12:00-12:30','12:30-1:00','1:00-1:30','1:30-2:00','2:00-2:30','2:30-3:00','3:00-3:30','3:30-4:00','4:00-4:30','4:30-5:00']

    def __init__(self, parent):
        top = self.top = Toplevel(parent)
        self.setDayCheckbutton('2013-09-23 Mon',0)
        self.setDayCheckbutton('2013-09-24 Tues',1)
        self.setDayCheckbutton('2013-09-25 Wed',2)
        self.setDayCheckbutton('2013-09-26 Thur',3)
        self.setDayCheckbutton('2013-09-27 Fri',4)

    def setDayCheckbutton(self, String_date, Int_col):
        myLabel = Label(self.top, text = String_date)
        myLabel.grid(row = 0, column = Int_col)
        myLabelWeekday = Label(self.top, text = 'Monday')
        myLabelWeekday.grid(row = 1, column = Int_col)

        Int_listLength = len(CheckBox.List_String_buttonText)
        Int_rowNum = 2
        for i in range(0, Int_listLength):
            String_buttonText = CheckBox.List_String_buttonText[i]
            if String_buttonText in ('12:00-12:30','12:30-1:00'):
                cb = Checkbutton(self.top, text=String_buttonText, state = DISABLED)
            else:
                cb = Checkbutton(self.top, text=String_buttonText, command = self.ClickOn)
            cb.grid(row = Int_rowNum, column = Int_col)
            Int_rowNum += 1

    def ClickOn(self):
        dialog = Dialog(self.top)
#        self.top.destroy()
\end{lstlisting}

\section{Application.py}
\label{sec-37-5}

\lstset{language=java,label= ,caption= ,numbers=none}
\begin{lstlisting}
# (me~~)@XXXXXX.edu

from Tkinter import *

import sys
sys.path.append('C:/502/')

from Login import Login
from UserClass import UserClass
from CheckBox import CheckBox

class Application(Frame):
    def login(self):
        login = Login(self.master)
        self.master.wait_window(login.top)
        self.String_emailAddress = login.getUsername()

        userClass = UserClass(self.String_emailAddress)

        if userClass.isEmail():
            if userClass.isValidEmail():
                self.canvas.delete(ALL)
                self.canvas.create_text(20, 30, anchor = W, text = "Welcome, " + self.String_emailAddress+"!")

                if userClass.isStudent():
                    checkbox_student = CheckBox(self.master)
                elif userClass.isInstructor():
                    pass
                    #pop instructor interface
            else:
               pass 

    def createWidgets(self):
        self.Login = Button(self)
        self.Login["text"] = "Login",
        self.Login["command"] = self.login
        self.Login.pack({"side": "top"})

        self.QUIT = Button(self)
        self.QUIT["text"] = "Quit"
        self.QUIT["fg"]   = "red"
        self.QUIT["command"] =  self.quit
        self.QUIT.pack({"side": "top"})

    def __init__(self, master=None):
        Frame.__init__(self, master)
        self.master = master
        self.pack()
        self.createWidgets()
        self.String_emailAddress = ""

        self.canvas = Canvas(self, width = 600, height = 100)
        self.canvas.pack(fill= BOTH, expand = 1)
\end{lstlisting}

\section{test.py}
\label{sec-37-6}

\lstset{language=java,label= ,caption= ,numbers=none}
\begin{lstlisting}
from Tkinter import *

import sys
sys.path.append('C:/tmp/')

from Application import Application

root = Tk()
app = Application(master=root)
app.mainloop()
\end{lstlisting}

\chapter{同写作业的小伙伴(hw3b)}
\label{sec-38}

\section{from:         (me\textasciitilde{}~) <(me\textasciitilde{}~)@gmail.com>}
\label{sec-38-1}
to:         (小伙伴)@XXXXX.edu
date:         Sat, Nov 16, 2013 at 2:42 PM
subject:         hw3b
mailed-by:         gmail.com
:         Important because you marked it as important.

Hi, 

Sorry for bothering. And I don't know your name yet. 

I have done all the parts I understood about hw3b, but I feel very confused with CallK t->child\footnote{DEFINITION NOT FOUND.} or ParamK. Could you please give me some idea how to deal with the calling function parameters walking down, do type check and potentially, how could I make it call recursively?

I may ask questions about hw4 probably in the late evening or next morning after I think about it, but before 5:00pm today I won't. 

Thank you!
同时我也把自己作业的代码原件发给了他

\section{from:         (小伙伴) ((小伙伴)@XXXXX.edu) <(小伙伴)@XXXXX.edu>}
\label{sec-38-2}
to:         (me\textasciitilde{}~) <(me\textasciitilde{}~)@gmail.com>
date:         Sat, Nov 16, 2013 at 4:54 PM
subject:         RE: hw3b
mailed-by:         XXXXX.edu
:         Important because you marked it as important.

No worries (me\textasciitilde{}~), my name is (小伙伴).

When you are processing the CallK you want to check t->child\footnotemark[2]{}, ParamK, against the function declaration ParamK's of the CallK's id (found in the symbol table).

Clarifying above statement with an example:

int dog(int x, y) \{
\ldots{}
\}

main() \{
dog(2,3);
\}

When you pass int dog in your semantic analysis you want to insert it in the symbol table.
Then when you reach dog(2,3); in main you will lookup "dog" in the symbol table, then walk through checking each sibling of child\footnotemark[2]{} (both the defined dog found in the symbol table and the one that you are currently processing).

So in effect your program, once it gets to dog(2,3);, will lookup "dog" in the symbol table, check type of 2 against int x and type of 3 against int y.

These parameters are siblings listed, so if you reach NULL on either the declaration parameters or the call parameters stop checking types and display an error saying that there are either too many or too few parameters.

Sorry for the late response, I hope it helps.

(小伙伴)

\section{from:         (me\textasciitilde{}~) <(me\textasciitilde{}~)@gmail.com>}
\label{sec-38-3}
to:         "(小伙伴) ((小伙伴)@XXXXX.edu)" <(小伙伴)@XXXXX.edu>
date:         Sat, Nov 16, 2013 at 5:08 PM
subject:         Re: hw3b
mailed-by:         gmail.com
:         Important because you marked it as important.

Hi (小伙伴), 

After having read your post, I feel I have done in similar way, but my error messages have duplicates. I can not differentiate between ParamK and CallK. All the type comparison and walking down lists I do it in CallK (process t->child\footnotemark[2]{}, and then walk down ->child\footnotemark[2]{}'s siblings list). Use basicAll.c- as the example, my erorr msg for ling 155, 156 are these ones: 

ERROR(155): '+' requires operands of type int but rhs is of type bool.
ERROR(155): Symbol 'parrot' is not defined.
ERROR(155): 'and' requires operands of type bool but rhs is of type int.
ERROR(155): Symbol 'parrot' is not defined.
ERROR(155): 'and' requires operands of type bool but rhs is of type int.
ERROR(155): 'and' requires operands of type bool but rhs is of type int.
ERROR(155): Expecting type int in parameter 3 of call to 'emu' defined on line 42 but got type bool.
ERROR(155): Too few parameters passed for function 'emu' defined on line 42.
ERROR(156): Expecting type int in parameter 1 of call to 'emu' defined on line 42 but got type bool.
ERROR(156): '+' requires operands of type int but rhs is of type bool.
ERROR(156): Symbol 'parrot' is not defined.
ERROR(156): 'and' requires operands of type bool but rhs is of type int.
ERROR(156): Symbol 'parrot' is not defined.
ERROR(156): 'and' requires operands of type bool but rhs is of type int.
ERROR(156): 'and' requires operands of type bool but rhs is of type int.
ERROR(156): Expecting type int in parameter 3 of call to 'emu' defined on line 42 but got type bool.
ERROR(156): Too few parameters passed for function 'emu' defined on line 42.
ERROR(156): 'and' requires operands of type bool but rhs is of type int.
ERROR(156): Expecting type int in parameter 2 of call to 'emu' defined on line 42 but got type bool.
ERROR(156): Expecting type int in parameter 3 of call to 'emu' defined on line 42 but got type bool.
ERROR(156): '*' requires operands of type int but rhs is of type bool.
ERROR(156): Too few parameters passed for function 'emu' defined on line 42.

My CallK part of my function is: 

\begin{lstlisting}[language=c++]
case CallK:

if (st->lookup(t->string) == NULL) {
        printf("ERROR(%d): Symbol '%s' is not defined.\n", t->linnum, t->string);
        ++numError;
        t->expType = Undefined; 
        if (t->child[0] != NULL)
        insertCheckNode(t->child[0]);
 } else {
        if ( (((TreeNode*)(st->lookup(t->string)))->kind.decl == VarK)
             || (((TreeNode*)(st->lookup(t->string)))->kind.decl == ParamK) ) {
        printf("ERROR(%d): '%s' is a simple variable and cannot be called.\n", t->linnum, t->string);
        ++numError;
        t->expType = Undefined;
        if ( t->child[0] )
            insertCheckNode(t->child[0]);
        } else { //FuncK
        TreeNode* p = (TreeNode*)(st->lookup(t->string));
        if ( (t->child[0] == NULL) && (p->child[0] != NULL) ) {
            printf("ERROR(%d): Too few parameters passed for function '%s' defined on line %d.\n", 
                   t->linnum, t->string, p->linnum);
            ++numError;
        }
        if (t->child[0] != NULL) {

            TreeNode* tmpt = t->child[0];
            int pcnt = 0;

            while (tmpt) {
                //if (tmpt) {
                insertCheckNode(tmpt);

                if (p->child[0] == NULL) {
                    printf("ERROR(%d): Too many parameters passed for function '%s' defined on line %d.\n", 
                           tmpt->linnum, t->string, p->linnum);
                    ++numError;
                } else {
                    TreeNode* tmpp = p->child[0];
                    if ((tmpt->expType != tmpp->expType) && (tmpt->expType != Undefined) ) {
                        printf("ERROR(%d): Expecting type %s in parameter %i of call to '%s' defined on line %d but got type %s.\n", 
                               tmpt->linnum, getType(tmpp->expType), pcnt+1, t->string, p->linnum, getType(tmpt->expType));
                        ++numError;
                    }
                    //if (tmpt->expType == tmpp->expType) 
                    //t->expType = p->expType;
                    if ( (tmpt->isArray == true) && (tmpp->isArray == false) ) {
                        printf("ERROR(%d): Not expecting array in parameter %i of call to '%s' defined on line %d.\n", 
                               t->linnum, pcnt+1, t->string, p->linnum);
                        ++numError;
                    }
                    if ( (tmpt->isArray == false) && (tmpp->isArray == true) ) {
                        printf("ERROR(%d): Expecting array in parameter %i of call to '%s' defined on line %d.\n", 
                               t->linnum, pcnt+1, t->string, p->linnum);
                        ++numError;
                    }
                    tmpt->expType = tmpp->expType;
                    tmpt->isArray = tmpp->isArray;

                    if ( (tmpt->sibling != NULL) &&  (tmpp->sibling == NULL) ) {
                        printf("ERROR(%d): Too many parameters passed for function '%s' defined on line %d.\n", 
                               tmpt->linnum, t->string, p->linnum);
                        ++numError;
                        //tmpt = tmpt->sibling;
                        //while (tmpt)
                        //insertCheckNode(tmpt);
                    }
                    if ( (tmpt->sibling == NULL) &&  (tmpp->sibling != NULL) ) {
                        printf("ERROR(%d): Too few parameters passed for function '%s' defined on line %d.\n", 
                               tmpt->linnum, t->string, p->linnum);
                        ++numError;
                    }
                
                    //while ( (tmpt->sibling != NULL) &&  (tmpp->sibling != NULL) ) {
                    if ( (tmpt->sibling != NULL) &&  (tmpp->sibling != NULL) ) 
                        pcnt++;
                    //insertCheckNode(tmpt->sibling);
                    /*                          
                                    if ( (tmpt->sibling->expType != tmpp->sibling->expType) && (tmpt->sibling->expType != Undefined) ) {
                                    printf("ERROR(%d): Expecting type %s in parameter %i of call to '%s' defined on line %d but got type %s.\n", 
                                    tmpt->sibling->linnum, getType(tmpp->sibling->expType), pcnt+1, t->string, 
                                    p->linnum, getType(tmpt->sibling->expType));
                                    ++numError;
                                    }
                                    if ( (tmpt->sibling->isArray == true) && (tmpp->sibling->isArray == false) ) {
                                    printf("ERROR(%d): Not expecting array in parameter %i of call to '%s' defined on line %d.\n", 
                                    tmpt->sibling->linnum, pcnt+1, t->string, p->linnum);
                                    ++numError;
                                    }
                                    if ( (tmpt->sibling->isArray == false) && (tmpp->sibling->isArray == true) ) {
                                    printf("ERROR(%d): Expecting array in parameter %i of call to '%s' defined on line %d.\n", 
                                    tmpt->sibling->linnum, pcnt+1, t->string, p->linnum);
                                    ++numError;
                                    } 
                                    tmpt->sibling->expType = tmpp->sibling->expType;
                                    tmpt->sibling->isArray = tmpp->sibling->isArray; 
                                    tmpt = tmpt->sibling;
                                    tmpp = tmpp->sibling;
                                    }
                    */
                    tmpp = tmpp->sibling;
                    tmpt = tmpt->sibling;
                } // end p->child[0] != NULL
                //while (tmpt) {
                //insertCheckNode(tmpt);
            } 
        } //   end t->child[0] != NULL
        t->expType = p->expType;

        } // FuncK
 } //symtab

if (t->sibling != NULL)
        insertCheckNode(t->sibling);
break;

\end{lstlisting}

Please help me look into it why I got 2 copies for sibling 1, and 3 copies for sibling 2, ? I have spent two hours on this stupid CallK, and got more and more confused. 
But after you finished your hw4, which I guess you have finished already. (Since it is due, it is low priority now. )

Most probably I will write to you on hw4 later tonight or midnight tomorrow morning. So tomorrow morning, when you get a chance, please help me a little bit with my hw4 questions. 

Thanks a lot!

\section{from:         (小伙伴) ((小伙伴)@XXXXX.edu) <(小伙伴)@XXXXX.edu>}
\label{sec-38-4}
to:         (me\textasciitilde{}~) <(me\textasciitilde{}~)@gmail.com>
date:         Sat, Nov 16, 2013 at 6:00 PM
subject:         RE: hw3b
mailed-by:         XXXXX.edu
:         Important because you marked it as important.

One problem that I can see that may be causing some of the issues is that you are checking both the current nodes and the current nodes' siblings in your while loop as you check FuncK. 

This will cause you to print an error if there is a problem on the current node, then an error (if there is one) on the next sibling of the current node, then make the next sibling the current node, which will do the same check.

(小伙伴)

\section{from:         (小伙伴) ((小伙伴)@XXXXX.edu) <(小伙伴)@XXXXX.edu>}
\label{sec-38-5}
to:         (me\textasciitilde{}~) <(me\textasciitilde{}~)@gmail.com>
date:         Sat, Nov 16, 2013 at 6:03 PM
subject:         RE: hw3b
mailed-by:         XXXXX.edu
:         Important because you marked it as important.

Ah I might be wrong\ldots{} I see you commented out the sibling check statements which would cause what I described in the last email. I can't see anything else immediately wrong though. I could see some of the commented out stuff causing your problem, or possibly other parts of your code but other than that I am not sure.

Sorry I couldn't help more.

(小伙伴)

\section{from:         (me\textasciitilde{}~) <(me\textasciitilde{}~)@gmail.com>}
\label{sec-38-6}
to:         "(小伙伴) ((小伙伴)@XXXXX.edu)" <(小伙伴)@XXXXX.edu>
date:         Sat, Nov 16, 2013 at 6:13 PM
subject:         Re: hw3b
mailed-by:         gmail.com
:         Important because you marked it as important.

Hi (小伙伴), 

You have helped a lot already. After you have checked my code, I feel relaxed that at least my direction is correct, and this is the right way to walk down the list. I will just focus on the duplicate parts to keep only one copy. 

Thank you so much! And I will try my best to get this one done after my hw4. 

Have a nice weekend!

\section{from:         (小伙伴) ((小伙伴)@XXXXX.edu) <(小伙伴)@XXXXX.edu>}
\label{sec-38-7}
to:         (me\textasciitilde{}~) <(me\textasciitilde{}~)@gmail.com>
date:         Sat, Nov 16, 2013 at 6:31 PM
subject:         RE: hw3b
mailed-by:         XXXXX.edu
:         Important because you marked it as important.

Thanks, you too. Happy to help!

\section{from:         (me\textasciitilde{}~) <(me\textasciitilde{}~)@gmail.com>}
\label{sec-38-8}
to:         "(小伙伴) ((小伙伴)@XXXXX.edu)" <(小伙伴)@XXXXX.edu>
date:         Sun, Nov 17, 2013 at 3:05 AM
subject:         Re: hw3b
mailed-by:         gmail.com

Hey (小伙伴), 

I had been worried about my hw4 cause I started my hw3 with Dr. cs445代课老师's help (I was not able to start independently yet). I feel the compiler homeworks for me just like trying so hard to catch a hedgehog and it always hurts. Now I feel slightly relaxed after I combined the parts together and generated some basic errors. Cause I have not start to "think" how to generate syntax error yet, tomorrow I will have to spend quality time on it. Thanks for all the help, please don't worry about my email tomorrow any more, and hopefully I can finish hw4 on time. 

\chapter{与代课老师邮件(hw4 \& hw3)}
\label{sec-39}
这篇是没有贴的
\section{from:         (me\textasciitilde{}~) <(me\textasciitilde{}~)@gmail.com>}
\label{sec-39-1}
to:         cs445代课老师 <captainbbbbbbb@gmail.com>,
 Cs445代课老师n <profcs445代课老师@gmail.com>
date:         Tue, Nov 26, 2013 at 2:19 PM
subject:         hw4 \& hw3
mailed-by:         gmail.com
:         Important because you marked it as important.

Dr. cs445代课老师, 

I have resubmitted hw4 \& hw3. For hw3, I have minor issues for ReturnK \& CompoundK. Do you think with hw3 I have so far, can I start to work on hw5 ?

thanks,

(me\textasciitilde{}~)

\section{from:         cs445代课老师 <captainbbbbbbb@gmail.com>}
\label{sec-39-2}
to:         (me\textasciitilde{}~) <(me\textasciitilde{}~)@gmail.com>
date:         Tue, Nov 26, 2013 at 9:37 PM
subject:         Re: hw4 \& hw3
mailed-by:         gmail.com
signed-by:         gmail.com
:         Important mainly because of your interaction with messages in the conversation.

Hi (me\textasciitilde{}~),
     I think you need to start on HW5 right away.   There is no reason to delay.   You know you have to 
traverse the tree in both the semantic and codegen sections.   I think you have been doing that.   I won't
have the hws graded until I have finished the the exams.

cheers,

\chapter{hw2 \& hw3 Grades}
\label{sec-40}

编译课那个星期三期中考试,考完试的当晚,收到了代课老师发给我们的hw2、hw3的成绩。这成绩被老师压得太低了,低得完全超出了我的想像。我的第二次作业就只有一个变量设置的问题,第三次作业(hw3, not hw3b)我生成的msg是相对比较少,但搭好traverse framework的工作量也不该只是老师给出的这点儿分数啊?!!正如第二次作业后的邮件里、和教室里课堂上老师所表现出的,他试图自始自终就给我很低的成绩(为让我没有翻身的机会?),但都被我一再反水!现在这种情况下,我一收到邮件便就第一时间给老师发邮件,我想看成绩细节、知道评分标准,我要坚绝保证上完这门课时自己还是好好的(结果是,这门编译课结束后,我还是被“玩”残了\textasciitilde{}~!!)!

亲爱的读者,我很希望把自己的成绩都原封不动地贴出来,只可惜,很多我写过的邮件、local file都有至少一个版本,但那门课的成绩因为一直是贴在网上的,我竟然没有存,徒有指向成绩网址的空空如也的link而已!我想说,此时回忆这门课的自己,也感觉很遗憾\textasciitilde{}~

\chapter{hw2 \& hw3 Grades与代课老师的邮件}
\label{sec-41}

\section{Your CS445 Grades}
\label{sec-41-1}
cs445代课老师@XXXXXX.edu [cs445代课老师@XXXXXX.edu]
Sent:        Thursday, November 07, 2013 6:30 AM
To:        
 (me\textasciitilde{}~) ((me\textasciitilde{}~)@XXXXXX.edu)
Your email can be found at \url{http://ec2-54-200-16-181.us-west-2.compute.amazonaws.com/Results/msgs}-(me\textasciitilde{}~)-txt-eWwSe4MXrcii2vPSBpnDS

\section{Your CS445 Grades}
\label{sec-41-2}
cs445代课老师@XXXXXX.edu [cs445代课老师@XXXXXX.edu]
Sent:        Thursday, November 07, 2013 6:31 AM
To:        
 (me\textasciitilde{}~) ((me\textasciitilde{}~)@XXXXXX.edu)
Your email can be found at \url{http://ec2-54-200-16-181.us-west-2.compute.amazonaws.com/Results/msgs}-(me\textasciitilde{}~)-txt-fDMh1yp7EVJRDSkVucUXf

\section{from:         (me\textasciitilde{}~) <(me\textasciitilde{}~)@gmail.com>}
\label{sec-41-3}
to:         cs445代课老师 <captainbbbbbbb@gmail.com>
date:         Wed, Nov 6, 2013 at 10:44 PM
subject:         Hw2 \& hw3 grades  schedule appointment
mailed-by:         gmail.com

Hi Dr. cs445代课老师, 

I have been waiting for you during office hour today but I failed to see you. 

I received email regarding my grades for hw2 and hw3 a moment ago, and I feel the grades for me were way too low than I have expected. I want to schedule a time to meet you sometime tomorrow to discuss about the grades and potential make up to survive this course. 

I will be occupied tomorrow during 11:30-12:30 for cs520, 1:30-2:20pm for cs445, and 3:00-3:30pm for cs502. I will be open any other time from 9:00am- 6:00pm. Please let me know if you have any period open so that we can get good understanding about my grades. 

Thanks,
(me\textasciitilde{}~)

\section{from:         cs445代课老师 <captainbbbbbbb@gmail.com>}
\label{sec-41-4}
to:         (me\textasciitilde{}~) <(me\textasciitilde{}~)@gmail.com>
date:         Wed, Nov 6, 2013 at 11:18 PM
subject:         Re: Hw2 \& hw3 grades  schedule appointment
mailed-by:         gmail.com
signed-by:         gmail.com

\subsection{Quote: On Wed, Nov 6, 2013 at 10:44 PM, (me\textasciitilde{}~) <(me\textasciitilde{}~)@gmail.com> wrote:}
\label{sec-41-4-1}
Hi Dr. cs445代课老师, 

I have been waiting for you during office hour today but I failed to see you. 

I received email regarding my grades for hw2 and hw3 a moment ago, and I feel the grades for me were way too low than I have expected. I want to schedule a time to meet you sometime tomorrow to discuss about the grades and potential make up to survive this course. 

\subsection{Re: The only opening I have is at 4:30 tomorrow.}
\label{sec-41-4-2}

\section{from:         (me\textasciitilde{}~) <(me\textasciitilde{}~)@gmail.com>}
\label{sec-41-5}
to:         cs445代课老师 <captainbbbbbbb@gmail.com>
date:         Wed, Nov 6, 2013 at 11:25 PM
subject:         Re: Hw2 \& hw3 grades  schedule appointment
mailed-by:         gmail.com

Dr. cs445代课老师, 

Yes, I can do 4:30pm tomorrow. I will see you then in your office to discuss about hw2, hw3 grades, potential make up grades, and all possible solutions for surviving this course. 

If it is possible, please also help grade my exam for me so that we can have more information for me for this course. If you do not have time grading the exam by then, please just simple ignore this paragraph I wrote. 

I will see you tomorrow in your office at 4:30pm. 

thanks,
(me\textasciitilde{}~)

\chapter{期中考试后第一堂课}
\label{sec-42}
\section{期中考试后第一堂课(1)}
\label{sec-42-1}

小伙伴们或许还记得春季旁听导师算法课时,期中考试前对过于简单的知识点不厌其烦地讲了又讲,讲到聪明有悟性的学生需要装笨的程度\textasciitilde{}~ 那这门编译课的review呢?考前实在没有其它人提,后来坐我后面的一个美国男生问老师考前有review吗?老师对这个同学的提问显得鄙夷,同学们就不太敢讲话。所以最终考前是有review的,但是是以光前进的速度进行的,所以就半堂课的review时间等老师讲完了,我原本不懂的现在可能懂了个方向;我原本懂的大概已经不懂了\textasciitilde{}~ smileface 而且真的的考试题目与review的内容并不是很相关。

那天是应该是周四吧(周三考试、周四还有课的),我们上午有编译课。这是期中考试考完后的第一堂课,老师上课第一句话是说,我就知道你们如果知道作业成绩后可能期中考试会考不好,所以等期中考试考完才把成绩发给你们!是啊,说得多么合情合理、人性化、尽显人文关怀,可实际上呢,我们也缺少了对自己两次作业的知情权,而且如果知道作业成绩,考试前会能准备得更充分的,我们同样也就缺少了准备期中考试的充分必要动力\textasciitilde{}~

\section{期中考试后第一堂课(2)}
\label{sec-42-2}

激情是什么,是在自己受到不公正打击时同敌人作最顽强的半争。想把别人的成绩就这么不明不白地压死也没有那么容易!哪怕最终我也还不得不死,那我也一定要尽自己的最大努力同敌人反抗到生命不息、战斗不止的最后一刻!我对自己说,从这一堂课起,我要主动,自己课堂上没听懂、听不懂的内容就当堂直接问老师,比如这第一堂课时我就问了,你前两堂课讲的framework,function return pointer, frame pointer的框架结构讲得不是很透彻,我还理解得不透,你能不能把这个框架结构再简单地REVIEW一遍?这是这学期这门编译课上自己第三次发言,虽然是以弱者的姿态。

我是班上仅有的两上活宝女生之一(另一个早提到过了,坐在我左边,基本上什么也不会,大概只能把她的第一次作业写得出来),老师这堂课是要主打提携女生、"宠爱"女生的牌么,还是借对我这样一个弱者姿态加以"宠爱"的方式来为他作这个学生很笨的广泛宣传增加力度和渗透性?我的这个问题提出后,代课老师像是保护国家一家保护动物般"呵护宠爱有加"地把这个compiler的大的架构结构从头到尾讲了整整一堂课。虽然我很清楚,这个我不太懂的问题我提问前班上真正理解的小伙伴未必过半,但像是一种情绪反溃,我还是直接顺了他的意,略带委屈地问他,你这堂课开始的时候说A不是某某状况,那你现在为什么又要去用A?言下之意,你现在讲的同你这堂课刚开始提出的那个什么什么观止是矛盾的呀?(对不起,上学期的上课笔记我没有带在身边,知识点的细节我已经忘记了,留在记忆里的更多的是自己心里那份不平,和这堂课前前后后提问的人、答题的人说话的语气腔调和同学们的情绪反应,大家一一包括我自己也都很讶异啊\textasciitilde{}~ 大概只有那个老头是个例外。)

\chapter{第三次大哭}
\label{sec-43}
\section{第三次大哭(1)}
\label{sec-43-1}

那天傍晚4:30我准时出现在代课老师的office门口,但我到后没过几秒钟,他的门口也站着一对可能是男女朋友的两学生,而且感觉他们也是来这里等老师的!他们是有些奇怪为什么我会出现在这里,我也很纳闷他们呀,这是老师在邮件里安排好的,老师又为什么要如此安排呢?

同样的三个学生站在门口等着,但是等到导师回到office,他就只把那个男生先领了进去,我很尴尬呀\textasciitilde{}~ 那个老师把那男生领进去的时候连招呼都没同自己打,我为什么要受这份委屈?至此,我已经很冲动想要骂人了,这算哪门子的事啊,时间是你安排的,你没时间就是没时间,你就算是记错自己时间表了,你也好歹跟人家提一下,对别人来说也是一种起码的尊重!你这样,难受就是你这个糟老老头子故意这么安排的?你一就差要进土的糟老头子,人家比你至少小上两个轮回,要不是为了学习为了成绩,谁稀罕向你问问题啊?可是自己的成绩已经被他变态地压低到难以忍受,我还是得等,带着委屈痛苦地等!心里也暗想,等到这门课结束了,我就再也不要同这老师有任何联系了\textasciitilde{}~!

我坐在系里main office前的沙发上等他们;想到被这个传说中的变态把自己成绩压这么低,想自己这一次留下也不过是要为自已的生活寻找一份出路,就早就有点儿情绪失控了,泪眼朦胧。这期间有老师一一比如之前cs210 programming的老师,从走道走过,AI老师的office就在走道对面,他们好几个人应该都注意到坐在那里的我表情情绪有些异样。

\section{第三次大哭(2)}
\label{sec-43-2}

后来大半个小时后,他们结束了,我去同导师谈,那天我终于还是没能控制住自己的情绪大哭一场。

其实,对系里给我的作业成绩,从cs336的C,从cs121的C换成B,从其它所有功课全拿B,我的理解是系里不支持两年毕业,但毕竟我们有"君子协议"在那里,看在系里其它同学也都不容易的份上,我其它所有功课都拿B我也都忍了,可为何就这个老师偏要与自己过不去?能够想得清楚的原理,只能是系里的老师鸡蛋里挑骨头、对我太苛刻了,想来我还是得从大的方向着手来沟通呀\textasciitilde{}~

也不知道什么时候、怎么弄的,好像可能是一开始眼泪就在往外掉了吧。

我对老师讲说,当年暑假写邮件讨论选课的时候,要不是你允许我选课,我压根就早就直接回中国去了;哪里还在这里受你们百般折磨?当初,就是因为你允许我选课,我才留下来的;就是因为你,you made my life miserable here, now!大概人处于悲伤状态,话也说得极其直接。老师强调说,不是因为他,是我自己作得决定!我心生好奇,就是因为你,怎么可能不是你,不是你,我的生活会与现在完全不同!但他说话的语气温和,不是要与我argue,而argue也不是今天的重点,暂且放他一马吧\textasciitilde{}~

\section{第三次大哭(3)}
\label{sec-43-3}

"I have agreement with Dr. cs210代课老师, why can't you guys just let me go, let me graduate smoothly?"为什么你们就不能让自己好好地毕业呢?我只是想毕业而已,为什么你就要这么硬把人逼迫着强留在这里,不允许别人正常毕业?导师说他没有逼我,没有要留我不让毕业,我就说了,你说的是没有,但你已经是在用成绩逼迫别人毕不了业!他只是说他没有,对压成绩的事不提。

这次返校,我能清楚地感觉到自己身体一年不如一年;第一学期就发作过一两次;这年早前四月份不太舒服的时候还去校医院看过医生,从2009年春夏有所觉察看医生开始,早就应该做一次超声波检查的,但受困于自己没有足够的经济来源还在一直拖着没做。因为经济上受困,医生说做一次那种仪器检查,如果只作一侧检查(而不是双侧的话),一个有效的检查大概要花\$1000左右,而我买的ISO的学生保险最多只能报\$400,所以我一直都没能好好看病。刚刚过去的九月底食堂里打工的那次身体不适,来得太急身体也虚脱得厉害,我很担心自己的身体健康\textasciitilde{}~ 而且这病对自己影响严重,会对自己造成严重的不良后果的!我想我的病已经这么严重了,在自己没有经济能力的情况下,不管系里、学校里能不能提供任何的帮助,知会系里的老师,对我来说,是必要的。别人只有在知情的情况下才可能提供有可能的帮助,如果你的经济状况别人根本就不知道,别人又有多少可能会跑来帮你?我哭着、强调着,心想着,这个老师应该会把我的实际情况反映到系里去的吧\textasciitilde{}~

\section{第三次大哭(4)}
\label{sec-43-4}

见我说得这么悲惨,老师试问我是什么病会有这么严重,我不方便细说也不知道学术用语的英文单词,就说是只有女生才会生的病,而且会与将来的生育能力紧密相关,千真万确会造成严重不良后果的,我可不想将来生不了小孩\textasciitilde{}~!

老师仿佛已经猜到是哪类病了,接着问我说,"Have you ever tried to touch it? How do you feel when you touch/push against it?""I feel it all the times, and I can feel/sense the difference compared to any other place when it comes back to its natural shape!" 不防去试想一下,一个吹鼓起来的气球,当你用一根指头把某个点按下去后,气球返弹回来时那份张力会是什么样子的?我向问话的老师竭力地描述着,我的肚子里就长有这样的东西呀。

让我非常意外的是,老师竟然接我的话说,他的wife也得过这样的病,"And that's the reason we don't have any child!"!我心里始终担心着的石头总算是落了地。人们都说同病相怜,得病的虽然不是他,也是他至关重要的人,与他休戚相关,很难说他就没有一点儿遗憾吧。知道了我现在的悲惨状况,他们一一代课老师的他,和系里其它老师们应该都不会再那么过分地为难我了吧!

\section{第三次大哭(5)}
\label{sec-43-5}

我就在家乡的省会城市上大学,记得上大学时候的某个五一期间,离家近放假刚从家里回来,班上的几个男生、学生班主任和宿舍里的女生,我们一起出去到校园里照相,也是借了梅雨季节雨过天晴的清新,与我同一宿舍的情敌竟然会受不了刚从家乡回来的自己眼神心思纯净,受不了我的呼吸"有家的气息"(这是她原话),竟然不敢再同我们一起照相,拉着她的同是体育生的同宿舍小伙伴就照了一张相就找借口逃跑躲掉了\textasciitilde{}~  smileface

为什么自己这半辈子都对雨过天晴大自然的清新有着如此深切真切执着的向往?是因为它浸透着我整个大学期间的年轻时候的黄金年华?是因为我出生成长在这样一方水土?还是那个幼年敏感的孩子内心里对那个自己坚定地认定错了的爸爸、对那个春夏交加的晚上一起出去田野里打鱼的爸爸有着深切的感情,只是那时的自己不知道而已?我想念、怀念自己出国前前半生的顺遂,而现在出门在外的自己这么多年来经历着种种磨难,我枯萎干涸在这异国他乡的荒郊野岭里\textasciitilde{}~  一如走过一年钢丝绳的惊险后那场惊心动魄的遇见,此刻的自己十分想念QQ群里偶滴大神和一起讨论、分享过的小伙伴们\textasciitilde{}~;)  在代课老师office里痛哭的时候,我希望天下雨,我想念家乡的梅雨季节,"I have never been home yet ever since I came here in US. I haven't been home for seven years"," I want to go home\textasciitilde{}~"

\section{第三次大哭(6)}
\label{sec-43-6}

我在老师的office里大哭着,哭干自己的每一滴血泪,哭回自己的家乡,哭回那个小时候的自己,我是怎么了,小时候的自己多么快乐!

可我还是不服啊,他说没有压低我的成绩就没有压低么?我不信,我压根儿就不信我的分数怎么可能那么低?我就还是对老师说,我要老师把班上所有同学的作业摆出来,公开公平地看待每一个学生的作业考试!我对老师说,我要查成绩,我要向学校报告老师偏心、给成绩不公平!老师说从他那里查成绩是不可以的,他也不可能把其它学生的作业、考试成绩公开给我看。但是,我可以报告给学校,学校还真有这样的机构!于是他自己上网找,给我写了个那个机构的电话号码。谁要他的破烂电话号码,有他这样的老师如此猖狂地镇压学生成绩,天下乌鸦一样黑,学校又何尝不会同流合污,会有公正、清白可言?说出那种话的自己也不过是想要发泄发泄自己心中的怒气罢了。

六点钟,老师说他要回家。我是五点多钟进老师office的吧,但现在也真的是晚了,虽然正事一件都还没干,但我也不能说不允许代课老师回家吧,于是撤退\textasciitilde{}~。
\chapter{与导师的绯闻 love affair gossip}
\label{sec-44}
\section{与导师的绯闻(1)}
\label{sec-44-1}

第二天周五,老师早就在之前的课堂上提到、安排了一个coding section,这样同学们写作业有困难的可以在这个课堂上问他。很奇怪的是这天来的小伙伴们的表情都很怪异,似乎他们瞬间都有了心思。坐我左边的女生来了,但她是落在最后面的,那些第二次作业的语法啊什么的,她好像都还没有听懂,没有写完。其它有几个男生,奇怪的是cs210坐我前面我们都分别写出了lisp tic-tac-toe的男生居然会来这个课堂;更奇怪的是一个留着大胡子的美国本科生男生"客串"(因为后来我们一起写作业的时候他表现得很聪明的,应该根本就不需要多问老师什么问题嘛)来听,而且他还带来了餐馆里打包的中国菜在课堂上吃!

后来接下来的周一还是周二上课课堂上,老师用笔记本显示一个什么内容,秒show了一下他的桌面,上面是一小男孩、一小女孩分别骑着一匹马并排站在山野里的卡通画面。这一秒,我也听到了课堂上有小伙伴情不自禁发出的唏嘘声,想来他们的表情也很惊奇!

到这时,我这颗榆木脑袋终于开始去想,在他们眼里,大概一一这个代课老师对我有那么点儿那个什么意思?回想他office里养着的那些树藤蔓,他这门课主页上的骑马照片,这是一个即将退休、行将入土的人啊,人一辈子他已经过完三分之二,这怎么可能?完全没有可能的事啊!

\section{与导师的绯闻(2)}
\label{sec-44-2}

而到这时,我也开始注意到,小镇上舆论早就已经炸开了。一起上课的同学们、系里的老师们甚至开始传说我这个学生粘着这个代课老师,总是下午很晚去找老师\textasciitilde{}~ 天地良心,我总是挑着周二周三office hour才去找的老师;下午傍晚就算是去找他也就那么两三次,而且还都是邮件里早就先预约好的。如果老师没有把时间定成那样,我也不可能去找他的呀!

我5月份去加州之前买的车,车title我一直都还没有收到;我还得去DMV再去申请一次(这个title我申请了三次,第三次让寄至DMV才算真正拿到手,后来知道前两份都被他们寄到前住址去了);在office里,帮我服务的妇人与我聊天说,这个小镇麻雀虽小、五脏俱全,发生过很多有趣的故事。妇人看着我的表情,让我觉得,我这个一直以为自己是那站在高高的山岗上看着日出日落、潮起潮落傻瓜笨蛋,到了这一刻,总算知道自己也是那个傻瓜笨蛋眼里所见到的周围那旋转的世界里的一颗沙子、一颗浮躁跳动的尘埃,也已然成为了这个小镇故事的一部分。

早前时候,我花很少的钱买了个微波炉,小镇上的人、学生似乎还很鄙视歧视;后来等我买的自己学生办公室里用的打印机到达的时候,他们终于不再说话了。那天我去邮局取网上买的打印机的墨盒,轮到我时,是一位四十岁左右的大叔serve for me,他让我等着,他去里面帮我去找。他回来过一次,同我再确认一个小细节,然后就又进去再帮我去找。那个下午四五点钟的时间段,小镇上来这里办事的人还挺多的,看着长长的等待队伍从排到门外一直变短变短再变短,短到最后就剩下了聊落的、不成队形的两三个人。当然最终,我还是取到了自己想要取回的墨盒。

这天这个世纪长的漫幽远绵长的等待,让事后的自己我感觉到,那个服务员、那位大叔那天是故意要我在那里等的。他希望用他自己的人生阅历、价值判断,和强加的这份等待,让我明白一一等下去,只要坚持等下去,我就一定会等得到我想要的!

\section{与导师的绯闻(3)}
\label{sec-44-3}

其实,我心里也有自己的答案。

当两条直线、两段截然不同的人生在这一刻相遇,碰撞出刹那芳华,但它注定短暂易逝,这一一是完全不可能的事!我与表哥感情很好,那场珍藏着的告别,他给了我这辈子最宝贵、温暖的感动和记忆。比自己大13岁的表哥,此时的自己尚且犹豫,初步估计一一这是又一匹比自己大上大概25岁的野马(自负、桀骜不驯,我大致猜得出属相,但我猜不出星座),我怎么可能去考虑他,这怎么可能?

早前我就说过,我这人心里长了草、谈起恋爱来,十匹骡子也把我拉不回来,其实这话也可以用到我的哭上去。我这人一旦一阵情绪上来,真是十匹骡子也拉不住,非得哭到地动山摇、彻底崩溃、累了再慢慢平静平息下来不可!虽然那天在那种情境下,换成任何另外一个代课老师,我都一定会情绪崩溃,但不一样的老师,我大概也还是不会崩溃、发泄到那么彻底吧!这还实在是一个处在中年危机的女人!

在这之前,早在暑假从加州回来后不久,在那份对A和组里小伙伴的无尽的想念里,我早就已经灵魂出窃了一一不管我多么想要留住那份珍贵的记忆,此时的感情却由不得我来作主。我自始自终都从来不曾想过与代课老师有过那种可能性,我是绣逗到这一两天才明白这事的,那为什么之前周四的课堂上,当老师对女生有所"宠爱"时,为什么我那堂课最后一个问题会提得带上了委屈的语气?我感到委屈的是对表哥的感情,是对这代课老师(虽然那时的自己并没有意识到),还是我给单得太久了,对别人的关心有着极度渴望?我还真是需要好好反醒一下自己!

\section{与导师的绯闻(4)}
\label{sec-44-4}

我对自己学习上的反省总是来得彻底洒脱、荡气回肠(QQ群里的小伙伴们应该是对我这一点儿再熟悉不过了smileface),然而要说感情上的反醒,我大概就只能"呵呵"了,因为我经历有限、不懂爱情的呀\textasciitilde{}~ 还是专家们的观点、意见更透彻权威点儿,那还是借用星象老师的预言来解惑吧。

有位星象老师在狮子座2014年年运中提到说:"至于本来就"不乖顺"型,则很容易受外界"积极爱慕者"之害,爱慕过头了害你被误解,或擦枪走火演变成不伦,都可能是婚姻的灾难。也许该去理解"背后动机",有太多"潜意识运作",也许不是因为风花雪月而爱,是太渴望人陪,所以谁靠近就把能量投射在对方身上,因为他是解救你于脆弱时的人,所以误以为是爱;或期待什么事发生可以改变无聊的生活,所以不想控制。"

到这里也就很清楚了,这不过是一个闹着中年危机的单身女人,被一个相对"变态"的老师折磨得有些喘不过气而已!如果这就说明我喜欢这个人,甚或是爱上一个人,那我一定是受虐狂,这个小镇、社会也紧接着就瞬间变得疯狂了\textasciitilde{}~!

对啊,这个小镇,又何至于会有这样的流言呢?假像都是老师自己制造的,当人们愿意去顺应这样一种假象的时候,小镇便也就屈服、臣服在大学教授这样的权威之下,舆论便失去了应有的分辨力!这是美国最为普通、平淡无奇的小镇,却以大海般博大的胸怀惊人地包容着我一一这样一个在网上书写完自己三生三世故事的异国他乡、有着曲折经历的女子,包容着这样一位同样有着人生遗憾的老师。法律管不了人类情感的事,道德也已然为这一切的发生进展一路绿灯,这一一是无言的结局?

\section{与导师的绯闻(5)}
\label{sec-44-5}

不!

一如第一学期我从代课老师的system software课吓跑逃跑后,这个老师试图以哄骗的方式(春季旁听算法课时)把我骗进编译课的课堂,这位代课老师、小镇上的善意的人们,他们都全然忽略、忘记了这两个截然不同的人生有着25岁年龄差距的事实,完全没意识到、忽略了,我一一这个在网上勤勤恳恳、夜以继日写完三生三世的人一一睡狮醒来,她的自我意识正在觉醒,她又岂肯容忍、臣服于这种被设计好、暗无天日的人生?No,this is mission impossible!!这与我想要的生活相差太远了\textasciitilde{}~ 

法律与道德之间的游走与平衡,包容的尺度如此难以把握,一不小心,它就变成了纵容一一君不见,庙堂之高、皇帝之远,多少学校、多少小镇上,凭借着大学教授的社会尊容,年迈老师诱惑、奸杀十八九岁刚入大学、涉世未深的少女,这种奸杀、枪杀还少么?是谁杀害了那些如花少女?罪恶单上有那万恶老师的一份,也该算上这小镇、社会愚昧的一份!

亲爱的读者,请原谅我的悲愤。一年365、366天,这姑娘偏偏就出生在"有两个主题深深吸引着这天出生的人,那就是个人的操守与道德问题,以及人类如蜉游般短促的生命问题。"的那一天。或者,再更私人一点儿,如果小镇上的人不是那般愚昧,如果小镇上的人有那么点儿操持与坚守,或许,写字这姑娘原本也不会经历那童年的不幸、不会经历这半世浮萍漂泊、风吹雨打、支离破碎的人生。

包容是一种态度,坚守的力量显得薄弱而难能可贵,值得邮局里的大叔、你和我一样的小市民、小伙伴们去思考、提倡、以及贡献一份微薄的力量。

\section{<东京爱情故事>}
\label{sec-44-6}

这两个太像的人是注定不能在一起,只能惺惺相惜的,所以他们之间的理解大于感情。从这里观者也大致能够猜测到,丽香在东京的多年里,遇到的这种男人居多,是聪明的同类,但无法互相归属。

然而人心的空洞,无法填满,一个女人最终要的不仅仅是懂她,而是全身心爱她。而这些,饱经世事的老男人通常是给不起的。于是,渴求着全身心爱的她在历经沧桑之后,加倍珍惜起完治质朴、简单、乡村少年般单纯的美好,她一度想把自己完全交付给这样的完治,也这样做了,但终究没有获得等量的情感共振,在能量耗尽后选择了自保性的离开。

因为爱得如此鲜明,每个人最终都成为了自己。
当多年后,在街上拥挤人群中,丽香看到里美蹲下来为完治系鞋带的一幕,也许释怀了,命运的选择就是让我们成为对的样子,如樱花一般开过一季绚烂绽放的激烈的青春,最终的作用就是让我们获得自己,因为爱得如此鲜明,每个人最终都成为了自己。

开端一个乡村少年下飞机来到东京生存开始,以在这里他见证城市也见证自己的蜕变为结局。这更像时代的隐喻,城市化进程也如人的成长一样,是褪去青涩,逐渐复杂化的过程,在繁华尽头、内心深处我们都是来自小镇的青年,怀念着过去农耕文明那些质朴的人事关系,当我们变得越成熟拥有聪明拥有一切,以为便获得了幸福,却发现越发想要追回失去的简单,想要一个心的归属,这就是丽香的心声。如果说丽香是当代繁华的都会,完治就是久远的清新乡野,我们一边成长,一边失去,一边追逐,一边怀念,它是爱情的故事,青春的故事,是东京的故事,也是所有城市和城市人在孤独中漂流、找寻精神理想和归属的故事。

丽香就是你身边微笑着,又独立又孤独的女子,在情感淡漠的都市,有早熟老成的爱情体验,有丰富练达的人情积淀;与聪明幽默的已婚大叔称得上棋逢对手,但暧昧点到为止,痛快但会清醒地保持距离;与睿智强势的优质精英称得上珠联璧合,但可惜无爱无感,堪称默契伙伴却做不成情郎;最难以释怀的是纯粹专一的美好,然而情深的男子往往只能匹配旧式女子里美,因为“让我靠一靠博得怜悯和保护”这样的戏码还没上演,她自己内心的防火墙就启动将脆弱一扫而光,朴实单纯的男人只会以为你懂事坚强,为你无懈可击的笑容,升起敬意却没有爱意;偶尔也会对爱的人随便宽衣解带――她做得了身体和自己的主人,不过,当你说你不需要为我负责时,也竟然就真的没有人为你负责。

这样的女子,不是弱女子,却也成为无法找到归属的女子。有时我也有些理解,丽香这么好,为什么总是爱得这么辛苦,因为像飞蛾扑火一样,她总算是找到了那团光明,那正是一种感情的宣泄,经过长时间淡漠,理智的情感压抑后的全身心付出,再知难而退。

坚强如她的女子,挣自由,挣到最后挣到的却是流浪。

爱情是人内心深处最简单的欲望,与某人融合,以此获得最深切真挚的情感共鸣,两个孤独的陌生人,要成为最紧密的联合体,这是人生中最不平凡的大事件。这样的安全与温暖,正是让我们飞蛾扑火,奋不顾身,又郁郁寡欢的理由。即使再聪明能干的人,在爱人面前也获得了软弱的权利,是安全感让我们软弱,让我们能够哭泣,能够索取,能够退行到孩童状态也不觉羞耻。如果有这样的一束光照亮生活,丽香这样的女人也许会活得不再需要那么努力。

电视剧里,坚强的丽香却一直在微笑,她在天台上看着东京的晨曦,旁观着这个世界的生机。城市好吗?它永远充满着希望!城市坏吗?为何却感觉总是孤身一人!也许,正是它让我们如此困惑。

\chapter{发信人: xiaan (今晚吃醋,谁家借点螃蟹), 信区: WebRadio}
\label{sec-45}
标  题: Re: 她,就是东京 【转载】 (转载)
发信站: BBS 未名空间站 (Fri Jul 25 17:50:30 2014, 美东)

不管"东爱"风靡大街小巷的时候,还是后来成为文艺青年口中经典爱情剧的时候,我都不耐烦看。对于这种四个挺好看的男女像小狗追尾巴一样一圈一圈地打转转的剧情,我年轻时候是很不屑的,俗套,琐碎,没出息。终于到了再也没有机会亲身尝试这些俗套,琐碎,和没出息的剧情的时候,我也变得有耐心来看这部剧了。

基本还算喜欢,比大部分韩剧要好,这种发达国家的都市爱情剧我觉得都挺能接受的,Bridget Jone's Diary, SATC,甚至Love Actually,对我来说都很有治愈力。我自己是乡下女人,在纽约逛半天就受不了了,可是偶尔还是会叶公好龙地向往城市生活,比如看到SATC里纽约的秋天,比如东爱里丽香一套一套的过时却永不落俗套的office lady的行头,我大概一辈子都不会这么穿,可是一直很喜欢。

至于爱情,四个青年都是很可爱的,虽然三上有点花,丸子有点呆,里美有点奸,丽香有点作,但是好就好在一点点,无伤大雅的一点点,既能推着剧情走,又不至于讨厌。长得好看又不惹人讨厌的四个青年的爱情故事,养眼程度是有起码的保障的。剧情进展虽然慢,但是细节还是很丰盈的,都市声不绝于耳,都市生活如在身畔,实在,真切,这个东京爱情故事里的东京两个字,并不是挂羊头卖狗肉。以及都市剧里面的乡愁,被大幅度地美化了,和城市生活相映成趣。丽香一再说她是在国外长大的,很不适应东京的生活,而丸子乡下人进城的背景,也是剧中很重要的元素。东京对他们来说,都不是"原乡"。他们对东京那种既陌生又熟悉,即想投入又盼望抽离的态度,不是这部剧的重点所在,却是不可更换的背景,而这样的背景让原本很俗套的故事,变得耐看很多。

他们之间的纠结也不招人烦,因为总的来说还是很干脆利落的。第一日剧对性的态度比韩剧开明很多,丽香和丸子上床那段处理的很干净,不需要那么多铺陈和慢动作,本来很自然的一件事,棒子对这些事情有点洁癖得过份;第二四个人之间的关系其实没那么复杂,三上跟里美果断分开后双方都没有走回头路,丽香跟三上有过倾心之谈可是并没有越雷池半步,虽然我觉得他们两个之间的chemistry挺强的,丽香和上司始终维持在艘美层次。这就是这部剧的好处,短短的10集,有铺垫有高潮有蓦然回首,结构上挺传统挺整齐的,再怎么纠结都不会累着观众。换作能够一年一年无休无止拍下去的美剧,四个人难免交叉睡来睡去不说,他们的七大姑八大姨肯定还会纷纷来打酱油,把原剧恰到好处的铺陈变成洒不完的狗血。

还有就是日本女人真是贤惠啊。看到丽香帮醉了酒的丸子脱鞋子盖被子,我明白为什么丽香是中国宅男经年不变的女神了。尼玛,有人如果能这样对我,那她也是我的女神 -- 漂亮,活泼,爱你爱得义无反顾无怨无悔,受了你的伤害不抱怨不诅咒不拉黑,还会帮你做饭脱鞋盖被子,一朵具有刘慧芳的品质的解语鲜花,要是真的,老子做牛做马也要追到手。

\chapter{狮子}
\label{sec-46}

上半年狮子仍被莫名的情绪困扰,与心头重担牵制着,唯一积极的出路,便是“以拔高的姿态面对”,
所以狮子们从去年至今,不断“努力超脱现实”,努力追求心灵美景,虽有收获,虽然彰显了自己的美好,但也同步彰显了现实丑恶:现实仍在,丑恶仍在,不可能不在意,就必须解决。 

苦守寒窑的computer science专业:
狮子们最有利的筹码,便是拥有“神等级的专业”,这是与现实世界最好的连结,就算彼此精神、目标不相容也没关系,在贡献专业方面,只要有机会狮子就愿意世俗些,因此才会“苦守寒窑”到现在。不过,到了今年,很多感觉都变了,如果现实再次证明“不值得守”,那狮子就会慎重考虑“换个方向,换个地方”了。 
只同同专业的小伙伴谈恋爱!

低调缺点:
木星仍是“幸运的”,你能得到别人妄想不到的平台去自我表现,重出江湖,展现贵气。尤其八月时,日、木同临命宫,气势之旺无人能敌,为精彩的下半年揭开华丽的序幕。 
要提醒狮子的是,木星不只放大你优异的部分,也放大了你的劣根性,所以不要被这样的机会,“勾出了坏品德”,好比就犯骄傲了,目中无人了,太喜孜孜或又开始得意忘形,在老师点名你独自表演的这段时间,这些正、反面的你都会被看个一清二楚,别人也就自有评价。

痛苦的享受
这真是“痛苦的享受”,因为你又专业执着,想精益求精,所以不得不手工,好追求精致、独一无二。而执着的结果,就是有更多做不完的工作,表象上,你得到的代价还算值得,你已占据行业的顶尖位置,没人敢轻忽你,专业度也带来“权势”,喊水会结冻,很有地位、气势,是业界之王,赫赫有名。 

出路:
先坚持“你最会的”,一步一步走,继续磨刀,继续保持业界领先地位,然后看看运势怎样撞出“全新角度运用你价值”的冒险机会吧。上半年与企图心有关的火星,不断在三宫运作,你终于有机会去表达、侃侃而谈、述说,这些表达不断刺激你的企图心,也让你做了很多特殊的尝试。机会可能很短小,但很精干,让狮子磨出“处变不惊”的能力,与泰山崩于前而色不改的气魄,越来越知道如何“切换频道”。这一切都在为下半年的“一飞冲天”铺路。 

哲学家的喟叹
因为狮子已感受到“某种极限”很难被突破,也开始思考工作“对个人巨大意义是什么”,这些体会,都让你不自觉用“超高远”的眼光审视自己处境,而有凡尘俗世却要与之起舞的无奈,这种“哲学家的喟叹”,一来怜惜自己无端受困,二来也明白环境就是这么肤浅,只好勉强在里面搅合,心境无比寥落,也无法被“爱戴你的人安慰”,你好想好想转变这一切。 

不但表现实力:
你面临较大的挑战是,总是最困难、最颠覆、最有挑战性的工作落到你身上,好处是你不断让人看到实力,坏处是因为“只许成功不许失败”所以压力有够大。你的好表现也得感谢职场同事,与你搭档的伙伴都是优秀、有实力的一时之选,因此能彼此加分,成果也不知不觉被提升了。 

\chapter{Financial Help Comes!}
\label{sec-47}

\section{From: (me\textasciitilde{}~)}
\label{sec-47-1}
Sent: Thursday, October 03, 2013 10:04 PM
To: (main office)
Subject: writing for financial help for Spring 2013

Hi (main office),

My name is (me\textasciitilde{}~), and I am a CS graduate student starting from fall 2012. This is my third semester and I am planning to graduate in August 2013. I am writing to you for some potential financial help from this program to support my spring semester tuition fees and living expense.

In June/July 2012, after I got accepted by this program, the department assigned Dr. cs445代课老师 as my adviser. And the department encouraged me to contact him regarding to course selection. Initially Dr. cs445代课老师 suggested me to select only cs121 and cs150. I had put the contact adviser and course selection things aside for couple of weeks (确切地说,是十天!) cause I had only around \$20,000 savings and I cannot bear to spend 3 years on campus at the age of 33(34 now, and 35 next year) and up to that point, I planned to go back to my home country. It was not until Dr. cs445代课老师 allowed me to skip cs121 that I changed my mind to stay here and continue with a CS master's degree.

Later on when I was on campus I learned it was Dr. cs445代课老师's kindness that allowed me to selection course that way. But for me, due to the lack of enough communication, it was one of the most important factors to support me make the decision to stay. By selection courses that way, I can almost financially survive within two years with one year help from the department. And I have took it for granted that after I pay for the first year tuition fees, I should be able to get TA or RA to support my next year cause I have always been good at study, and my master's program in statistics had allowed me to graduate even in three semesters(后来延一个学期是出于个人需要). Even though later on I didn't get any TA or RA from that program, they waved the out-of-state tuition fees for me for all the rest semesters except the very first one (所以我从来不曾想到,到同一学校另一个系里,形势会变得难以忍受的严厉,而这一切的信息我之前是不曾料想的\textasciitilde{}~).

So far after I came back to U of I fall 2012, besides having been worked at Bob's as a labor (which was an opportunity from the university) I have not got any financial help from the department. And with almost \$4000 borrowed from (男闺密) to pay my tuition fees for this fall semester, I have paid out every penny I have ever earned during my life and now, I am in urgent of the department's help to help me survive my last spring semester.

You can reach me writing to me using U of I email xxxxxx@yyyyyy.edu or my Gmail account aaaaaaaa@gmail.com, and you can call me (XXX) XXX-XXXX whenever it is convenient for you. And I will keep close contact with you so that I know the progress by trying to look for you at your office. Whenever you have information or concern, please do not hesitate to contact me.

Thanks in advance,

(me\textasciitilde{}~)


\section{From: (main office) ((main office)@XXXXXX.edu)}
\label{sec-47-2}
Sent: Friday, November 15, 2013 12:14 PM
To: '(me\textasciitilde{}~)'
Subject: RE: writing for financial help for Spring 2013

Dear (me\textasciitilde{}~),

If you are interested, we would like to offer you a CSAC/grader position with one of our faculty members for the spring semester.  The salary is \$4,000 to be paid out over the semester.  The appointment also requires 10 hours as a tutor in the CSAC.  Because this is a full-time appointment, your out-of-state fees are covered.

NOTE: you must be fully registered as a full-time student in order for me to put you on the payroll system.  Also, if you have never had a UX funded position, you will need to go down to Human Resources and fill out the necessary employment paperwork.

Please email me and let me know if you are interested in this position.

Thank you,

(main office)

Dept of Computer Science
University of XXXXX
djfhfkdjhf@dfhdjkfhjkdfh.edu


\section{from:         (main office) ((main office)@XXXXXX.edu) <(main office)@XXXXXX.edu>}
\label{sec-47-3}
to:         "(me\textasciitilde{}~)@gmail.com" <(me\textasciitilde{}~)@gmail.com>
date:         Fri, Nov 15, 2013 at 12:22 PM
subject:         FW: writing for financial help for Spring 2013
mailed-by:         XXXXXX.edu

Just in case you aren't reading your UI email.

(main office)


\section{from:         (me\textasciitilde{}~) <(me\textasciitilde{}~)@gmail.com>}
\label{sec-47-4}
to:         "(main office) ((main office)@XXXXXX.edu)" <(main office)@XXXXXX.edu>
date:         Fri, Nov 15, 2013 at 12:43 PM
subject:         Re: FW: writing for financial help for Spring 2013
mailed-by:         gmail.com

Yeah, I got the mail. Thank you so much!

收到来自系main office的邮件我原本该高兴,但我却高兴不起来;那个近乎变态的编译课的老师会让我过吗,我接下来两三次的作业变难吗?我能写得出来么?这个近乎变态的老师他会给我怎样的成绩?我第一个学期那门课的C会影响最终导致我拿不到这样的奖学金么?我心里堆了太多的疑问,我清楚地知道食堂里,如果接下来的学期不需要在那里打工,需要在旧学期结束前两周给出two weeks notice period,但因为自己的不安全感,我几乎是到学期结束也没有告诉任何人我春天可能不来了。

\chapter{第四次作业(代课老师出错)}
\label{sec-48}

\section{from:         (me\textasciitilde{}~) <(me\textasciitilde{}~)@gmail.com>}
\label{sec-48-1}
to:         cs445代课老师 <captainbbbbbbb@gmail.com>
date:         Sun, Nov 17, 2013 at 4:43 PM
subject:         hw4 processing nodekind order
mailed-by:         gmail.com

Hi Dr. cs445代课老师, 

For hw4, I just started and processed the logic, type and typearray. The intuition tells me I should process the nodekinds bottom up, but after typearray, I realized I lost my init errors. Will there be any possibility that your init.out has missed some error out?

Please help advise if it is correct to do bottom-up syntax check. And since I need to wait your reply, I will work on typefun or parms as the followed set. 

Thanks,
(me\textasciitilde{}~)


\section{from:         (me\textasciitilde{}~) <(me\textasciitilde{}~)@gmail.com>}
\label{sec-48-2}
to:         cs445代课老师 <captainbbbbbbb@gmail.com>
date:         Sun, Nov 17, 2013 at 5:36 PM
subject:         Re: hw4 processing nodekind order
mailed-by:         gmail.com

And, now I feel confused my grammar results matches the "tar of Assignment 4 preliminary tests", and my linux system results matches the wormulon system results, but I just could not match your Amazon online tests. 

Could you please help guide some appropriate testing environment, or let me know which one is the cretieria, the tarred output files, or the Amazon online?

Thanks. Feel confused. 


\section{from:         cs445代课老师 <captainbbbbbbb@gmail.com>}
\label{sec-48-3}
to:         (me\textasciitilde{}~) <(me\textasciitilde{}~)@gmail.com>
date:         Sun, Nov 17, 2013 at 7:25 PM
subject:         Re: hw4 processing nodekind order
mailed-by:         gmail.com
signed-by:         gmail.com
:         Important mainly because it was sent directly to you.

All you have to do for assignment 4 is put in the error productions and the yyerror code.
I am not sure what you mean by process the nodekinds bottom up.   Bison will do the
syntax error discovery.   You don't need to traverse the tree.


\section{from:         (me\textasciitilde{}~) <(me\textasciitilde{}~)@gmail.com>}
\label{sec-48-4}
to:         cs445代课老师 <captainbbbbbbb@gmail.com>
date:         Sun, Nov 17, 2013 at 7:28 PM
subject:         Re: hw4 processing nodekind order
mailed-by:         gmail.com

yeah, I used the wrong term. I mean for the test file, we have init, type, typearray, if, while etc. Should I process from bottom up, like type, init, then fun, params, while, foreach, then if etc? 

And, the tarred outputs are different from the outputs showed up after turnned in. Should we follow the tarred .c- run results .out files, or follow the outputs from Amazon run out ?

Thanks. 


\section{from:         cs445代课老师 <captainbbbbbbb@gmail.com>}
\label{sec-48-5}
to:         (me\textasciitilde{}~) <(me\textasciitilde{}~)@gmail.com>
date:         Sun, Nov 17, 2013 at 7:44 PM
subject:         Re: hw4 processing nodekind order
mailed-by:         gmail.com
signed-by:         gmail.com
:         Important mainly because it was sent directly to you.

\subsection{Quote: On Sun, Nov 17, 2013 at 7:28 PM, (me\textasciitilde{}~) <(me\textasciitilde{}~)@gmail.com> wrote:}
\label{sec-48-5-1}
yeah, I used the wrong term. I mean for the test file, we have init, type, typearray, if, while etc. Should I process from bottom up, like type, init, then fun, params, while, foreach, then if etc? 

\subsection{Re: I still don't understand.    What do you mean "bottom up"?    The test script runs your program on each file in the order it decides.   It should not effect your results.   I really am not sure what you are asking.}
\label{sec-48-5-2}

\subsection{Quote: And, the tarred outputs are different from the outputs showed up after turnned in. Should we follow the tarred .c- run results .out files, or follow the outputs from Amazon run out ?$\backslash$}
\label{sec-48-5-3}

\subsection{Re: The output from Amazon is the one to follow, but the .out files are the same as the ones on Amazon.    I just reran them and checked to be sure.   That is, I went to the test machine and pulled the .outs and then reran them on the web support machine and they seem to be the same.   I repushed them to the web server to be sure there is a clean copy there in case there was problem.}
\label{sec-48-5-4}

cheers,


\section{from:         (me\textasciitilde{}~) <(me\textasciitilde{}~)@gmail.com>}
\label{sec-48-6}
to:         cs445代课老师 <captainbbbbbbb@gmail.com>
date:         Sun, Nov 17, 2013 at 7:54 PM
subject:         Re: hw4 processing nodekind order
mailed-by:         gmail.com

Sorry I produced those trouble for you. But I mean from your website the tar file, named testDataA4.tar, the output files do not match the results from test machine. Please check the outputs for syntaxerr-typefun.out, syntaxerr-parms.out, they don't match the results from test machine. Please verifiy and confirm.

Thanks. 


\section{from:         cs445代课老师 <captainbbbbbbb@gmail.com>}
\label{sec-48-7}
to:         (me\textasciitilde{}~) <(me\textasciitilde{}~)@gmail.com>
date:         Mon, Nov 18, 2013 at 8:19 PM
subject:         Re: hw4 processing nodekind order
mailed-by:         gmail.com
signed-by:         gmail.com
:         Important mainly because it was sent directly to you.

In the attached file on the left is the expected output from the A4 assignment.   On the right is the concatenated output files taken from testDataA4.tar.   Excluding things like announcing what file is running the files are reported by sdiff to be identical.   See attached.

cheers,

\chapter{hw4:老师附件}
\label{sec-49}

\lstset{language=java,label= ,caption= ,numbers=none}
\begin{lstlisting}
/**** Limited to 5 seconds total run time and 5000 lines of output  <
							      <
/**** * ================================================ *	      <
|      Tests for CS445 Assignment 4                |	      <
|       Comparison with Expected Output            |	      <
/**** * ================================================ *	      <
							      <
/export/home/nibbler/TestWorld				      <
find makefile						      <
makefile						      <
a makefile is here					      <
RUN: c- allErrors.c-					      <
ERROR(ARGLIST): source file "allErrors.c-" could not be opene <
RUN: c- < syntaxerr-assign.c-				      <
ERROR(10): Syntax error.  Unexpected ';'.			ERROR(10): Syntax error.  Unexpected ';'.
ERROR(11): Syntax error.  Unexpected ';'.			ERROR(11): Syntax error.  Unexpected ';'.
ERROR(12): Syntax error.  Unexpected ';'.			ERROR(12): Syntax error.  Unexpected ';'.
ERROR(14): Syntax error.  Unexpected '='.			ERROR(14): Syntax error.  Unexpected '='.
ERROR(15): Syntax error.  Unexpected +=.			ERROR(15): Syntax error.  Unexpected +=.
ERROR(16): Syntax error.  Unexpected -=.			ERROR(16): Syntax error.  Unexpected -=.
ERROR(18): Syntax error.  Unexpected '='.			ERROR(18): Syntax error.  Unexpected '='.
ERROR(19): Syntax error.  Unexpected +=.			ERROR(19): Syntax error.  Unexpected +=.
ERROR(20): Syntax error.  Unexpected -=.			ERROR(20): Syntax error.  Unexpected -=.
ERROR(21): Syntax error.  Unexpected ++.			ERROR(21): Syntax error.  Unexpected ++.
ERROR(22): Syntax error.  Unexpected --.			ERROR(22): Syntax error.  Unexpected --.
ERROR(24): Syntax error.  Unexpected bool.			ERROR(24): Syntax error.  Unexpected bool.
ERROR(25): Syntax error.  Unexpected bool.			ERROR(25): Syntax error.  Unexpected bool.
ERROR(26): Syntax error.  Unexpected bool.			ERROR(26): Syntax error.  Unexpected bool.
ERROR(28): Syntax error.  Unexpected '='.			ERROR(28): Syntax error.  Unexpected '='.
ERROR(29): Syntax error.  Unexpected +=.			ERROR(29): Syntax error.  Unexpected +=.
ERROR(30): Syntax error.  Unexpected -=.			ERROR(30): Syntax error.  Unexpected -=.
ERROR(31): Syntax error.  Unexpected ++.			ERROR(31): Syntax error.  Unexpected ++.
ERROR(32): Syntax error.  Unexpected --.			ERROR(32): Syntax error.  Unexpected --.
Number of warnings: 0						Number of warnings: 0
Number of errors: 19						Number of errors: 19
RUN: c- < syntaxerr-call.c-				      <
ERROR(6): Syntax error.  Unexpected while.			ERROR(6): Syntax error.  Unexpected while.
ERROR(7): Syntax error.  Unexpected while.			ERROR(7): Syntax error.  Unexpected while.
ERROR(10): Syntax error.  Unexpected ';'.			ERROR(10): Syntax error.  Unexpected ';'.
ERROR(11): Syntax error.  Unexpected ')'.  Expecting ';'.	ERROR(11): Syntax error.  Unexpected ')'.  Expecting ';'.
Number of warnings: 0						Number of warnings: 0
Number of errors: 4						Number of errors: 4
RUN: c- < syntaxerr-call2.c-				      <
ERROR(5): Syntax error.  Unexpected id: u.  Expecting ']'.	ERROR(5): Syntax error.  Unexpected id: u.  Expecting ']'.
ERROR(7): Syntax error.  Unexpected id: u.  Expecting ']'.	ERROR(7): Syntax error.  Unexpected id: u.  Expecting ']'.
ERROR(10): Syntax error.  Unexpected ']'.  Expecting number.	ERROR(10): Syntax error.  Unexpected ']'.  Expecting number.
ERROR(22): Syntax error.  Unexpected '='.  Expecting ']'.	ERROR(22): Syntax error.  Unexpected '='.  Expecting ']'.
ERROR(23): Syntax error.  Unexpected ']'.			ERROR(23): Syntax error.  Unexpected ']'.
ERROR(23): Syntax error.  Unexpected ';'.  Expecting ')'.	ERROR(23): Syntax error.  Unexpected ';'.  Expecting ')'.
ERROR(27): Syntax error.  Unexpected '*'.  Expecting id.	ERROR(27): Syntax error.  Unexpected '*'.  Expecting id.
ERROR(27): Syntax error.  Unexpected ';'.  Expecting ']'.	ERROR(27): Syntax error.  Unexpected ';'.  Expecting ']'.
ERROR(27): Syntax error.  Unexpected ']'.  Expecting ')'.	ERROR(27): Syntax error.  Unexpected ']'.  Expecting ')'.
ERROR(27): Syntax error.  Unexpected number: 10.  Expecting '	ERROR(27): Syntax error.  Unexpected number: 10.  Expecting '
ERROR(27): Syntax error.  Unexpected number: 10.  Expecting '	ERROR(27): Syntax error.  Unexpected number: 10.  Expecting '
ERROR(27): Syntax error.  Unexpected number: 10.  Expecting '	ERROR(27): Syntax error.  Unexpected number: 10.  Expecting '
ERROR(28): Syntax error.  Unexpected character constant: 'x'.	ERROR(28): Syntax error.  Unexpected character constant: 'x'.
ERROR(29): Syntax error.  Unexpected '*'.  Expecting id.	ERROR(29): Syntax error.  Unexpected '*'.  Expecting id.
ERROR(30): Syntax error.  Unexpected '+'.			ERROR(30): Syntax error.  Unexpected '+'.
ERROR(30): Syntax error.  Unexpected ';'.  Expecting ')'.	ERROR(30): Syntax error.  Unexpected ';'.  Expecting ')'.
ERROR(31): Syntax error.  Unexpected '='.			ERROR(31): Syntax error.  Unexpected '='.
ERROR(34): Syntax error.  Unexpected ','.  Expecting ']'.	ERROR(34): Syntax error.  Unexpected ','.  Expecting ']'.
ERROR(34): Syntax error.  Unexpected int.  Expecting ')'.	ERROR(34): Syntax error.  Unexpected int.  Expecting ')'.
ERROR(35): Syntax error.  Unexpected '+'.			ERROR(35): Syntax error.  Unexpected '+'.
ERROR(35): Syntax error.  Unexpected ';'.  Expecting ')'.	ERROR(35): Syntax error.  Unexpected ';'.  Expecting ')'.
ERROR(36): Syntax error.  Unexpected '='.			ERROR(36): Syntax error.  Unexpected '='.
Number of warnings: 0						Number of warnings: 0
Number of errors: 22						Number of errors: 22
RUN: c- < syntaxerr-empty.c-				      <
ERROR(2): Syntax error.  Unexpected end of input.  Expecting 	ERROR(2): Syntax error.  Unexpected end of input.  Expecting 
Number of warnings: 0						Number of warnings: 0
Number of errors: 1						Number of errors: 1
RUN: c- < syntaxerr-foreach.c-				      <
ERROR(4): Syntax error.  Unexpected ')'.  Expecting in.		ERROR(4): Syntax error.  Unexpected ')'.  Expecting in.
ERROR(6): Syntax error.  Unexpected int.  Expecting in.		ERROR(6): Syntax error.  Unexpected int.  Expecting in.
ERROR(8): Syntax error.  Unexpected int.  Expecting id.		ERROR(8): Syntax error.  Unexpected int.  Expecting id.
ERROR(10): Syntax error.  Unexpected int.  Expecting id.	ERROR(10): Syntax error.  Unexpected int.  Expecting id.
ERROR(12): Syntax error.  Unexpected int.  Expecting '('.	ERROR(12): Syntax error.  Unexpected int.  Expecting '('.
ERROR(14): Syntax error.  Unexpected int.  Expecting '('.	ERROR(14): Syntax error.  Unexpected int.  Expecting '('.
ERROR(18): Syntax error.  Unexpected int.  Expecting '('.	ERROR(18): Syntax error.  Unexpected int.  Expecting '('.
ERROR(21): Syntax error.  Unexpected ')'.  Expecting in.	ERROR(21): Syntax error.  Unexpected ')'.  Expecting in.
ERROR(23): Syntax error.  Unexpected '*'.  Expecting in.	ERROR(23): Syntax error.  Unexpected '*'.  Expecting in.
ERROR(25): Syntax error.  Unexpected '*'.  Expecting id.	ERROR(25): Syntax error.  Unexpected '*'.  Expecting id.
ERROR(27): Syntax error.  Unexpected '*'.  Expecting id.	ERROR(27): Syntax error.  Unexpected '*'.  Expecting id.
ERROR(29): Syntax error.  Unexpected '*'.  Expecting '('.	ERROR(29): Syntax error.  Unexpected '*'.  Expecting '('.
ERROR(29): Syntax error.  Unexpected ')'.  Expecting ';'.	ERROR(29): Syntax error.  Unexpected ')'.  Expecting ';'.
ERROR(31): Syntax error.  Unexpected '*'.  Expecting '('.	ERROR(31): Syntax error.  Unexpected '*'.  Expecting '('.
ERROR(31): Syntax error.  Unexpected ';'.			ERROR(31): Syntax error.  Unexpected ';'.
ERROR(33): Syntax error.  Unexpected '*'.  Expecting '('.	ERROR(33): Syntax error.  Unexpected '*'.  Expecting '('.
ERROR(35): Syntax error.  Unexpected '*'.  Expecting '('.	ERROR(35): Syntax error.  Unexpected '*'.  Expecting '('.
ERROR(35): Syntax error.  Unexpected ';'.			ERROR(35): Syntax error.  Unexpected ';'.
ERROR(38): Syntax error.  Unexpected ')'.  Expecting in.	ERROR(38): Syntax error.  Unexpected ')'.  Expecting in.
ERROR(40): Syntax error.  Unexpected ';'.  Expecting in.	ERROR(40): Syntax error.  Unexpected ';'.  Expecting in.
ERROR(42): Syntax error.  Unexpected ')'.  Expecting id.	ERROR(42): Syntax error.  Unexpected ')'.  Expecting id.
ERROR(44): Syntax error.  Unexpected ';'.  Expecting id.	ERROR(44): Syntax error.  Unexpected ';'.  Expecting id.
ERROR(46): Syntax error.  Unexpected id: x.  Expecting '('.	ERROR(46): Syntax error.  Unexpected id: x.  Expecting '('.
ERROR(48): Syntax error.  Unexpected id: x.  Expecting '('.	ERROR(48): Syntax error.  Unexpected id: x.  Expecting '('.
ERROR(50): Syntax error.  Unexpected ')'.  Expecting '('.	ERROR(50): Syntax error.  Unexpected ')'.  Expecting '('.
ERROR(52): Syntax error.  Unexpected ';'.  Expecting '('.	ERROR(52): Syntax error.  Unexpected ';'.  Expecting '('.
Number of warnings: 0						Number of warnings: 0
Number of errors: 26						Number of errors: 26
RUN: c- < syntaxerr-if.c-				      <
ERROR(6): Syntax error.  Unexpected ++.				ERROR(6): Syntax error.  Unexpected ++.
ERROR(8): Syntax error.  Unexpected ++.				ERROR(8): Syntax error.  Unexpected ++.
ERROR(10): Syntax error.  Unexpected ++.			ERROR(10): Syntax error.  Unexpected ++.
ERROR(12): Syntax error.  Unexpected ++.			ERROR(12): Syntax error.  Unexpected ++.
ERROR(12): Syntax error.  Unexpected else.  Expecting ';'.	ERROR(12): Syntax error.  Unexpected else.  Expecting ';'.
ERROR(14): Syntax error.  Unexpected else.  Expecting ';'.	ERROR(14): Syntax error.  Unexpected else.  Expecting ';'.
ERROR(16): Syntax error.  Unexpected ++.			ERROR(16): Syntax error.  Unexpected ++.
ERROR(16): Syntax error.  Unexpected ++.  Expecting ';'.	ERROR(16): Syntax error.  Unexpected ++.  Expecting ';'.
ERROR(16): Syntax error.  Unexpected number: 222.  Expecting 	ERROR(16): Syntax error.  Unexpected number: 222.  Expecting 
ERROR(18): Syntax error.  Unexpected ++.			ERROR(18): Syntax error.  Unexpected ++.
ERROR(18): Syntax error.  Unexpected ++.  Expecting ';'.	ERROR(18): Syntax error.  Unexpected ++.  Expecting ';'.
ERROR(18): Syntax error.  Unexpected ++.  Expecting ';'.	ERROR(18): Syntax error.  Unexpected ++.  Expecting ';'.
ERROR(20): Syntax error.  Unexpected ++.  Expecting or or ')'	ERROR(20): Syntax error.  Unexpected ++.  Expecting or or ')'
ERROR(20): Syntax error.  Unexpected else.			ERROR(20): Syntax error.  Unexpected else.
ERROR(22): Syntax error.  Unexpected ++.  Expecting or or ')'	ERROR(22): Syntax error.  Unexpected ++.  Expecting or or ')'
ERROR(22): Syntax error.  Unexpected else.			ERROR(22): Syntax error.  Unexpected else.
ERROR(24): Syntax error.  Unexpected ++.  Expecting or or ')'	ERROR(24): Syntax error.  Unexpected ++.  Expecting or or ')'
ERROR(24): Syntax error.  Unexpected ++.			ERROR(24): Syntax error.  Unexpected ++.
ERROR(26): Syntax error.  Unexpected ++.  Expecting or or ')'	ERROR(26): Syntax error.  Unexpected ++.  Expecting or or ')'
ERROR(26): Syntax error.  Unexpected ++.			ERROR(26): Syntax error.  Unexpected ++.
ERROR(28): Syntax error.  Unexpected ++.  Expecting or or ')'	ERROR(28): Syntax error.  Unexpected ++.  Expecting or or ')'
ERROR(30): Syntax error.  Unexpected ++.  Expecting or or ')'	ERROR(30): Syntax error.  Unexpected ++.  Expecting or or ')'
ERROR(32): Syntax error.  Unexpected ++.  Expecting or or ')'	ERROR(32): Syntax error.  Unexpected ++.  Expecting or or ')'
ERROR(32): Syntax error.  Unexpected number: 222.  Expecting 	ERROR(32): Syntax error.  Unexpected number: 222.  Expecting 
ERROR(34): Syntax error.  Unexpected ++.			ERROR(34): Syntax error.  Unexpected ++.
ERROR(34): Syntax error.  Unexpected else.			ERROR(34): Syntax error.  Unexpected else.
ERROR(36): Syntax error.  Unexpected ++.			ERROR(36): Syntax error.  Unexpected ++.
ERROR(36): Syntax error.  Unexpected else.			ERROR(36): Syntax error.  Unexpected else.
ERROR(38): Syntax error.  Unexpected ++.			ERROR(38): Syntax error.  Unexpected ++.
ERROR(38): Syntax error.  Unexpected ++.			ERROR(38): Syntax error.  Unexpected ++.
ERROR(40): Syntax error.  Unexpected ++.			ERROR(40): Syntax error.  Unexpected ++.
ERROR(40): Syntax error.  Unexpected ++.			ERROR(40): Syntax error.  Unexpected ++.
ERROR(42): Syntax error.  Unexpected ++.			ERROR(42): Syntax error.  Unexpected ++.
ERROR(42): Syntax error.  Unexpected else.  Expecting ';'.	ERROR(42): Syntax error.  Unexpected else.  Expecting ';'.
ERROR(44): Syntax error.  Unexpected ++.			ERROR(44): Syntax error.  Unexpected ++.
ERROR(44): Syntax error.  Unexpected else.  Expecting ';'.	ERROR(44): Syntax error.  Unexpected else.  Expecting ';'.
ERROR(46): Syntax error.  Unexpected ++.			ERROR(46): Syntax error.  Unexpected ++.
ERROR(46): Syntax error.  Unexpected ++.  Expecting ';'.	ERROR(46): Syntax error.  Unexpected ++.  Expecting ';'.
ERROR(46): Syntax error.  Unexpected number: 222.  Expecting 	ERROR(46): Syntax error.  Unexpected number: 222.  Expecting 
ERROR(48): Syntax error.  Unexpected ++.			ERROR(48): Syntax error.  Unexpected ++.
ERROR(48): Syntax error.  Unexpected ++.  Expecting ';'.	ERROR(48): Syntax error.  Unexpected ++.  Expecting ';'.
ERROR(48): Syntax error.  Unexpected ++.  Expecting ';'.	ERROR(48): Syntax error.  Unexpected ++.  Expecting ';'.
ERROR(50): Syntax error.  Unexpected ++.			ERROR(50): Syntax error.  Unexpected ++.
ERROR(50): Syntax error.  Unexpected else.			ERROR(50): Syntax error.  Unexpected else.
ERROR(52): Syntax error.  Unexpected ++.			ERROR(52): Syntax error.  Unexpected ++.
ERROR(52): Syntax error.  Unexpected else.			ERROR(52): Syntax error.  Unexpected else.
ERROR(54): Syntax error.  Unexpected ++.			ERROR(54): Syntax error.  Unexpected ++.
ERROR(54): Syntax error.  Unexpected ++.			ERROR(54): Syntax error.  Unexpected ++.
ERROR(56): Syntax error.  Unexpected ++.			ERROR(56): Syntax error.  Unexpected ++.
ERROR(56): Syntax error.  Unexpected ++.			ERROR(56): Syntax error.  Unexpected ++.
ERROR(58): Syntax error.  Unexpected ++.			ERROR(58): Syntax error.  Unexpected ++.
ERROR(58): Syntax error.  Unexpected else.  Expecting ';'.	ERROR(58): Syntax error.  Unexpected else.  Expecting ';'.
ERROR(60): Syntax error.  Unexpected ++.			ERROR(60): Syntax error.  Unexpected ++.
ERROR(60): Syntax error.  Unexpected else.  Expecting ';'.	ERROR(60): Syntax error.  Unexpected else.  Expecting ';'.
ERROR(62): Syntax error.  Unexpected ++.  Expecting '('.	ERROR(62): Syntax error.  Unexpected ++.  Expecting '('.
ERROR(62): Syntax error.  Unexpected id: x.  Expecting ';'.	ERROR(62): Syntax error.  Unexpected id: x.  Expecting ';'.
ERROR(64): Syntax error.  Unexpected ++.  Expecting '('.	ERROR(64): Syntax error.  Unexpected ++.  Expecting '('.
ERROR(64): Syntax error.  Unexpected id: x.  Expecting ';'.	ERROR(64): Syntax error.  Unexpected id: x.  Expecting ';'.
ERROR(64): Syntax error.  Unexpected ++.			ERROR(64): Syntax error.  Unexpected ++.
ERROR(66): Syntax error.  Unexpected ++.  Expecting '('.	ERROR(66): Syntax error.  Unexpected ++.  Expecting '('.
ERROR(66): Syntax error.  Unexpected id: x.  Expecting ';'.	ERROR(66): Syntax error.  Unexpected id: x.  Expecting ';'.
ERROR(66): Syntax error.  Unexpected ++.			ERROR(66): Syntax error.  Unexpected ++.
ERROR(66): Syntax error.  Unexpected ++.			ERROR(66): Syntax error.  Unexpected ++.
ERROR(68): Syntax error.  Unexpected ++.  Expecting '('.	ERROR(68): Syntax error.  Unexpected ++.  Expecting '('.
ERROR(68): Syntax error.  Unexpected id: x.  Expecting ';'.	ERROR(68): Syntax error.  Unexpected id: x.  Expecting ';'.
ERROR(68): Syntax error.  Unexpected ++.			ERROR(68): Syntax error.  Unexpected ++.
ERROR(68): Syntax error.  Unexpected ++.			ERROR(68): Syntax error.  Unexpected ++.
ERROR(70): Syntax error.  Unexpected ++.  Expecting '('.	ERROR(70): Syntax error.  Unexpected ++.  Expecting '('.
ERROR(70): Syntax error.  Unexpected id: x.  Expecting ';'.	ERROR(70): Syntax error.  Unexpected id: x.  Expecting ';'.
ERROR(70): Syntax error.  Unexpected else.  Expecting ';'.	ERROR(70): Syntax error.  Unexpected else.  Expecting ';'.
ERROR(72): Syntax error.  Unexpected ++.  Expecting '('.	ERROR(72): Syntax error.  Unexpected ++.  Expecting '('.
ERROR(72): Syntax error.  Unexpected id: x.  Expecting ';'.	ERROR(72): Syntax error.  Unexpected id: x.  Expecting ';'.
ERROR(72): Syntax error.  Unexpected else.  Expecting ';'.	ERROR(72): Syntax error.  Unexpected else.  Expecting ';'.
ERROR(74): Syntax error.  Unexpected ++.  Expecting '('.	ERROR(74): Syntax error.  Unexpected ++.  Expecting '('.
ERROR(74): Syntax error.  Unexpected id: x.  Expecting ';'.	ERROR(74): Syntax error.  Unexpected id: x.  Expecting ';'.
ERROR(74): Syntax error.  Unexpected ++.  Expecting ';'.	ERROR(74): Syntax error.  Unexpected ++.  Expecting ';'.
ERROR(74): Syntax error.  Unexpected number: 222.  Expecting 	ERROR(74): Syntax error.  Unexpected number: 222.  Expecting 
ERROR(76): Syntax error.  Unexpected ++.  Expecting '('.	ERROR(76): Syntax error.  Unexpected ++.  Expecting '('.
ERROR(76): Syntax error.  Unexpected id: x.  Expecting ';'.	ERROR(76): Syntax error.  Unexpected id: x.  Expecting ';'.
ERROR(76): Syntax error.  Unexpected ++.  Expecting ';'.	ERROR(76): Syntax error.  Unexpected ++.  Expecting ';'.
ERROR(76): Syntax error.  Unexpected ++.  Expecting ';'.	ERROR(76): Syntax error.  Unexpected ++.  Expecting ';'.
ERROR(78): Syntax error.  Unexpected ++.  Expecting '('.	ERROR(78): Syntax error.  Unexpected ++.  Expecting '('.
ERROR(78): Syntax error.  Unexpected id: x.  Expecting ';'.	ERROR(78): Syntax error.  Unexpected id: x.  Expecting ';'.
ERROR(78): Syntax error.  Unexpected else.			ERROR(78): Syntax error.  Unexpected else.
ERROR(80): Syntax error.  Unexpected ++.  Expecting '('.	ERROR(80): Syntax error.  Unexpected ++.  Expecting '('.
ERROR(80): Syntax error.  Unexpected id: x.  Expecting ';'.	ERROR(80): Syntax error.  Unexpected id: x.  Expecting ';'.
ERROR(80): Syntax error.  Unexpected else.			ERROR(80): Syntax error.  Unexpected else.
ERROR(82): Syntax error.  Unexpected ++.  Expecting '('.	ERROR(82): Syntax error.  Unexpected ++.  Expecting '('.
ERROR(82): Syntax error.  Unexpected id: x.  Expecting ';'.	ERROR(82): Syntax error.  Unexpected id: x.  Expecting ';'.
ERROR(82): Syntax error.  Unexpected ++.			ERROR(82): Syntax error.  Unexpected ++.
ERROR(84): Syntax error.  Unexpected ++.  Expecting '('.	ERROR(84): Syntax error.  Unexpected ++.  Expecting '('.
ERROR(84): Syntax error.  Unexpected id: x.  Expecting ';'.	ERROR(84): Syntax error.  Unexpected id: x.  Expecting ';'.
ERROR(84): Syntax error.  Unexpected ++.			ERROR(84): Syntax error.  Unexpected ++.
ERROR(86): Syntax error.  Unexpected ++.  Expecting '('.	ERROR(86): Syntax error.  Unexpected ++.  Expecting '('.
ERROR(86): Syntax error.  Unexpected id: x.  Expecting ';'.	ERROR(86): Syntax error.  Unexpected id: x.  Expecting ';'.
ERROR(86): Syntax error.  Unexpected else.  Expecting ';'.	ERROR(86): Syntax error.  Unexpected else.  Expecting ';'.
ERROR(88): Syntax error.  Unexpected ++.  Expecting '('.	ERROR(88): Syntax error.  Unexpected ++.  Expecting '('.
ERROR(88): Syntax error.  Unexpected id: x.  Expecting ';'.	ERROR(88): Syntax error.  Unexpected id: x.  Expecting ';'.
ERROR(88): Syntax error.  Unexpected else.  Expecting ';'.	ERROR(88): Syntax error.  Unexpected else.  Expecting ';'.
ERROR(90): Syntax error.  Unexpected ++.  Expecting '('.	ERROR(90): Syntax error.  Unexpected ++.  Expecting '('.
ERROR(90): Syntax error.  Unexpected id: x.  Expecting ';'.	ERROR(90): Syntax error.  Unexpected id: x.  Expecting ';'.
ERROR(90): Syntax error.  Unexpected ++.  Expecting ';'.	ERROR(90): Syntax error.  Unexpected ++.  Expecting ';'.
ERROR(90): Syntax error.  Unexpected number: 222.  Expecting 	ERROR(90): Syntax error.  Unexpected number: 222.  Expecting 
ERROR(92): Syntax error.  Unexpected ++.  Expecting '('.	ERROR(92): Syntax error.  Unexpected ++.  Expecting '('.
ERROR(92): Syntax error.  Unexpected ++.  Expecting ';'.	ERROR(92): Syntax error.  Unexpected ++.  Expecting ';'.
ERROR(92): Syntax error.  Unexpected ')'.  Expecting ';'.	ERROR(92): Syntax error.  Unexpected ')'.  Expecting ';'.
ERROR(94): Syntax error.  Unexpected ++.  Expecting '('.	ERROR(94): Syntax error.  Unexpected ++.  Expecting '('.
ERROR(94): Syntax error.  Unexpected ++.  Expecting ';'.	ERROR(94): Syntax error.  Unexpected ++.  Expecting ';'.
ERROR(94): Syntax error.  Unexpected ')'.  Expecting ';'.	ERROR(94): Syntax error.  Unexpected ')'.  Expecting ';'.
ERROR(94): Syntax error.  Unexpected ++.			ERROR(94): Syntax error.  Unexpected ++.
ERROR(96): Syntax error.  Unexpected ++.  Expecting '('.	ERROR(96): Syntax error.  Unexpected ++.  Expecting '('.
ERROR(96): Syntax error.  Unexpected ++.  Expecting ';'.	ERROR(96): Syntax error.  Unexpected ++.  Expecting ';'.
ERROR(96): Syntax error.  Unexpected ')'.  Expecting ';'.	ERROR(96): Syntax error.  Unexpected ')'.  Expecting ';'.
ERROR(96): Syntax error.  Unexpected ++.			ERROR(96): Syntax error.  Unexpected ++.
ERROR(96): Syntax error.  Unexpected ++.			ERROR(96): Syntax error.  Unexpected ++.
ERROR(98): Syntax error.  Unexpected ++.  Expecting '('.	ERROR(98): Syntax error.  Unexpected ++.  Expecting '('.
ERROR(98): Syntax error.  Unexpected ++.  Expecting ';'.	ERROR(98): Syntax error.  Unexpected ++.  Expecting ';'.
ERROR(98): Syntax error.  Unexpected ')'.  Expecting ';'.	ERROR(98): Syntax error.  Unexpected ')'.  Expecting ';'.
ERROR(98): Syntax error.  Unexpected ++.			ERROR(98): Syntax error.  Unexpected ++.
ERROR(98): Syntax error.  Unexpected ++.			ERROR(98): Syntax error.  Unexpected ++.
ERROR(100): Syntax error.  Unexpected ++.  Expecting '('.	ERROR(100): Syntax error.  Unexpected ++.  Expecting '('.
ERROR(100): Syntax error.  Unexpected ++.  Expecting ';'.	ERROR(100): Syntax error.  Unexpected ++.  Expecting ';'.
ERROR(100): Syntax error.  Unexpected ')'.  Expecting ';'.	ERROR(100): Syntax error.  Unexpected ')'.  Expecting ';'.
ERROR(100): Syntax error.  Unexpected else.  Expecting ';'.	ERROR(100): Syntax error.  Unexpected else.  Expecting ';'.
ERROR(102): Syntax error.  Unexpected ++.  Expecting '('.	ERROR(102): Syntax error.  Unexpected ++.  Expecting '('.
ERROR(102): Syntax error.  Unexpected ++.  Expecting ';'.	ERROR(102): Syntax error.  Unexpected ++.  Expecting ';'.
ERROR(102): Syntax error.  Unexpected ')'.  Expecting ';'.	ERROR(102): Syntax error.  Unexpected ')'.  Expecting ';'.
ERROR(102): Syntax error.  Unexpected else.  Expecting ';'.	ERROR(102): Syntax error.  Unexpected else.  Expecting ';'.
ERROR(104): Syntax error.  Unexpected ++.  Expecting '('.	ERROR(104): Syntax error.  Unexpected ++.  Expecting '('.
ERROR(104): Syntax error.  Unexpected ++.  Expecting ';'.	ERROR(104): Syntax error.  Unexpected ++.  Expecting ';'.
ERROR(104): Syntax error.  Unexpected ')'.  Expecting ';'.	ERROR(104): Syntax error.  Unexpected ')'.  Expecting ';'.
ERROR(104): Syntax error.  Unexpected ++.  Expecting ';'.	ERROR(104): Syntax error.  Unexpected ++.  Expecting ';'.
ERROR(104): Syntax error.  Unexpected number: 222.  Expecting	ERROR(104): Syntax error.  Unexpected number: 222.  Expecting
ERROR(106): Syntax error.  Unexpected ++.  Expecting '('.	ERROR(106): Syntax error.  Unexpected ++.  Expecting '('.
ERROR(106): Syntax error.  Unexpected ++.  Expecting ';'.	ERROR(106): Syntax error.  Unexpected ++.  Expecting ';'.
ERROR(106): Syntax error.  Unexpected ')'.  Expecting ';'.	ERROR(106): Syntax error.  Unexpected ')'.  Expecting ';'.
ERROR(106): Syntax error.  Unexpected ++.  Expecting ';'.	ERROR(106): Syntax error.  Unexpected ++.  Expecting ';'.
ERROR(106): Syntax error.  Unexpected ++.  Expecting ';'.	ERROR(106): Syntax error.  Unexpected ++.  Expecting ';'.
ERROR(108): Syntax error.  Unexpected ++.  Expecting '('.	ERROR(108): Syntax error.  Unexpected ++.  Expecting '('.
ERROR(108): Syntax error.  Unexpected ++.  Expecting ';'.	ERROR(108): Syntax error.  Unexpected ++.  Expecting ';'.
ERROR(108): Syntax error.  Unexpected ++.  Expecting ';'.	ERROR(108): Syntax error.  Unexpected ++.  Expecting ';'.
ERROR(108): Syntax error.  Unexpected number: 111.  Expecting	ERROR(108): Syntax error.  Unexpected number: 111.  Expecting
ERROR(108): Syntax error.  Unexpected else.			ERROR(108): Syntax error.  Unexpected else.
ERROR(110): Syntax error.  Unexpected ++.  Expecting '('.	ERROR(110): Syntax error.  Unexpected ++.  Expecting '('.
ERROR(110): Syntax error.  Unexpected ++.  Expecting ';'.	ERROR(110): Syntax error.  Unexpected ++.  Expecting ';'.
ERROR(110): Syntax error.  Unexpected ++.  Expecting ';'.	ERROR(110): Syntax error.  Unexpected ++.  Expecting ';'.
ERROR(110): Syntax error.  Unexpected number: 111.  Expecting	ERROR(110): Syntax error.  Unexpected number: 111.  Expecting
ERROR(110): Syntax error.  Unexpected else.			ERROR(110): Syntax error.  Unexpected else.
ERROR(112): Syntax error.  Unexpected ++.  Expecting '('.	ERROR(112): Syntax error.  Unexpected ++.  Expecting '('.
ERROR(112): Syntax error.  Unexpected ++.  Expecting ';'.	ERROR(112): Syntax error.  Unexpected ++.  Expecting ';'.
ERROR(112): Syntax error.  Unexpected ++.  Expecting ';'.	ERROR(112): Syntax error.  Unexpected ++.  Expecting ';'.
ERROR(112): Syntax error.  Unexpected number: 111.  Expecting	ERROR(112): Syntax error.  Unexpected number: 111.  Expecting
ERROR(112): Syntax error.  Unexpected ++.			ERROR(112): Syntax error.  Unexpected ++.
ERROR(114): Syntax error.  Unexpected ++.  Expecting '('.	ERROR(114): Syntax error.  Unexpected ++.  Expecting '('.
ERROR(114): Syntax error.  Unexpected ++.  Expecting ';'.	ERROR(114): Syntax error.  Unexpected ++.  Expecting ';'.
ERROR(114): Syntax error.  Unexpected ++.  Expecting ';'.	ERROR(114): Syntax error.  Unexpected ++.  Expecting ';'.
ERROR(114): Syntax error.  Unexpected number: 111.  Expecting	ERROR(114): Syntax error.  Unexpected number: 111.  Expecting
ERROR(114): Syntax error.  Unexpected ++.			ERROR(114): Syntax error.  Unexpected ++.
ERROR(116): Syntax error.  Unexpected ++.			ERROR(116): Syntax error.  Unexpected ++.
ERROR(116): Syntax error.  Unexpected number: 111.  Expecting	ERROR(116): Syntax error.  Unexpected number: 111.  Expecting
ERROR(116): Syntax error.  Unexpected else.			ERROR(116): Syntax error.  Unexpected else.
ERROR(118): Syntax error.  Unexpected ++.			ERROR(118): Syntax error.  Unexpected ++.
ERROR(118): Syntax error.  Unexpected number: 111.  Expecting	ERROR(118): Syntax error.  Unexpected number: 111.  Expecting
ERROR(118): Syntax error.  Unexpected else.			ERROR(118): Syntax error.  Unexpected else.
ERROR(120): Syntax error.  Unexpected ++.			ERROR(120): Syntax error.  Unexpected ++.
ERROR(120): Syntax error.  Unexpected ++.  Expecting ';'.	ERROR(120): Syntax error.  Unexpected ++.  Expecting ';'.
ERROR(120): Syntax error.  Unexpected else.  Expecting ';'.	ERROR(120): Syntax error.  Unexpected else.  Expecting ';'.
ERROR(122): Syntax error.  Unexpected ++.			ERROR(122): Syntax error.  Unexpected ++.
ERROR(122): Syntax error.  Unexpected ++.  Expecting ';'.	ERROR(122): Syntax error.  Unexpected ++.  Expecting ';'.
ERROR(122): Syntax error.  Unexpected else.  Expecting ';'.	ERROR(122): Syntax error.  Unexpected else.  Expecting ';'.
ERROR(124): Syntax error.  Unexpected ++.			ERROR(124): Syntax error.  Unexpected ++.
ERROR(124): Syntax error.  Unexpected number: 111.  Expecting	ERROR(124): Syntax error.  Unexpected number: 111.  Expecting
ERROR(126): Syntax error.  Unexpected '('.  Expecting ')'.	ERROR(126): Syntax error.  Unexpected '('.  Expecting ')'.
ERROR(126): Syntax error.  Unexpected number: 111.  Expecting	ERROR(126): Syntax error.  Unexpected number: 111.  Expecting
ERROR(126): Syntax error.  Unexpected ';'.  Expecting ')'.	ERROR(126): Syntax error.  Unexpected ';'.  Expecting ')'.
ERROR(128): Syntax error.  Unexpected '('.  Expecting ')'.	ERROR(128): Syntax error.  Unexpected '('.  Expecting ')'.
ERROR(128): Syntax error.  Unexpected ++.  Expecting ')'.	ERROR(128): Syntax error.  Unexpected ++.  Expecting ')'.
ERROR(128): Syntax error.  Unexpected else.  Expecting ')'.	ERROR(128): Syntax error.  Unexpected else.  Expecting ')'.
ERROR(130): Syntax error.  Unexpected '('.  Expecting ')'.	ERROR(130): Syntax error.  Unexpected '('.  Expecting ')'.
ERROR(130): Syntax error.  Unexpected ++.  Expecting ')'.	ERROR(130): Syntax error.  Unexpected ++.  Expecting ')'.
ERROR(130): Syntax error.  Unexpected else.  Expecting ')'.	ERROR(130): Syntax error.  Unexpected else.  Expecting ')'.
ERROR(130): Syntax error.  Unexpected ';'.  Expecting ')'.	ERROR(130): Syntax error.  Unexpected ';'.  Expecting ')'.
ERROR(132): Syntax error.  Unexpected '('.  Expecting ')'.	ERROR(132): Syntax error.  Unexpected '('.  Expecting ')'.
ERROR(132): Syntax error.  Unexpected number: 111.  Expecting	ERROR(132): Syntax error.  Unexpected number: 111.  Expecting
ERROR(134): Syntax error.  Unexpected '('.  Expecting ')'.	ERROR(134): Syntax error.  Unexpected '('.  Expecting ')'.
ERROR(134): Syntax error.  Unexpected number: 111.  Expecting	ERROR(134): Syntax error.  Unexpected number: 111.  Expecting
ERROR(134): Syntax error.  Unexpected ';'.  Expecting ')'.	ERROR(134): Syntax error.  Unexpected ';'.  Expecting ')'.
ERROR(136): Syntax error.  Unexpected '('.  Expecting ')'.	ERROR(136): Syntax error.  Unexpected '('.  Expecting ')'.
ERROR(136): Syntax error.  Unexpected ++.  Expecting ')'.	ERROR(136): Syntax error.  Unexpected ++.  Expecting ')'.
ERROR(136): Syntax error.  Unexpected else.  Expecting ')'.	ERROR(136): Syntax error.  Unexpected else.  Expecting ')'.
ERROR(138): Syntax error.  Unexpected '('.  Expecting ')'.	ERROR(138): Syntax error.  Unexpected '('.  Expecting ')'.
ERROR(138): Syntax error.  Unexpected ++.  Expecting ')'.	ERROR(138): Syntax error.  Unexpected ++.  Expecting ')'.
ERROR(138): Syntax error.  Unexpected else.  Expecting ')'.	ERROR(138): Syntax error.  Unexpected else.  Expecting ')'.
ERROR(138): Syntax error.  Unexpected ';'.  Expecting ')'.	ERROR(138): Syntax error.  Unexpected ';'.  Expecting ')'.
ERROR(140): Syntax error.  Unexpected '('.  Expecting ')'.	ERROR(140): Syntax error.  Unexpected '('.  Expecting ')'.
ERROR(140): Syntax error.  Unexpected ++.  Expecting ')'.	ERROR(140): Syntax error.  Unexpected ++.  Expecting ')'.
ERROR(140): Syntax error.  Unexpected number: 111.  Expecting	ERROR(140): Syntax error.  Unexpected number: 111.  Expecting
ERROR(142): Syntax error.  Unexpected '('.  Expecting ')'.	ERROR(142): Syntax error.  Unexpected '('.  Expecting ')'.
ERROR(142): Syntax error.  Unexpected ++.  Expecting ')'.	ERROR(142): Syntax error.  Unexpected ++.  Expecting ')'.
ERROR(142): Syntax error.  Unexpected number: 111.  Expecting	ERROR(142): Syntax error.  Unexpected number: 111.  Expecting
ERROR(142): Syntax error.  Unexpected ';'.  Expecting ')'.	ERROR(142): Syntax error.  Unexpected ';'.  Expecting ')'.
ERROR(144): Syntax error.  Unexpected '('.  Expecting ')'.	ERROR(144): Syntax error.  Unexpected '('.  Expecting ')'.
ERROR(144): Syntax error.  Unexpected ++.  Expecting ')'.	ERROR(144): Syntax error.  Unexpected ++.  Expecting ')'.
ERROR(144): Syntax error.  Unexpected ++.  Expecting ')'.	ERROR(144): Syntax error.  Unexpected ++.  Expecting ')'.
ERROR(144): Syntax error.  Unexpected else.  Expecting ')'.	ERROR(144): Syntax error.  Unexpected else.  Expecting ')'.
ERROR(146): Syntax error.  Unexpected '('.  Expecting ')'.	ERROR(146): Syntax error.  Unexpected '('.  Expecting ')'.
ERROR(146): Syntax error.  Unexpected ++.  Expecting ')'.	ERROR(146): Syntax error.  Unexpected ++.  Expecting ')'.
ERROR(146): Syntax error.  Unexpected ++.  Expecting ')'.	ERROR(146): Syntax error.  Unexpected ++.  Expecting ')'.
ERROR(146): Syntax error.  Unexpected else.  Expecting ')'.	ERROR(146): Syntax error.  Unexpected else.  Expecting ')'.
ERROR(146): Syntax error.  Unexpected ';'.  Expecting ')'.	ERROR(146): Syntax error.  Unexpected ';'.  Expecting ')'.
ERROR(148): Syntax error.  Unexpected ++.  Expecting ')'.	ERROR(148): Syntax error.  Unexpected ++.  Expecting ')'.
ERROR(148): Syntax error.  Unexpected id: x.  Expecting ')'.	ERROR(148): Syntax error.  Unexpected id: x.  Expecting ')'.
ERROR(148): Syntax error.  Unexpected number: 111.  Expecting	ERROR(148): Syntax error.  Unexpected number: 111.  Expecting
ERROR(150): Syntax error.  Unexpected ++.  Expecting ')'.	ERROR(150): Syntax error.  Unexpected ++.  Expecting ')'.
ERROR(150): Syntax error.  Unexpected id: x.  Expecting ')'.	ERROR(150): Syntax error.  Unexpected id: x.  Expecting ')'.
ERROR(150): Syntax error.  Unexpected number: 111.  Expecting	ERROR(150): Syntax error.  Unexpected number: 111.  Expecting
ERROR(150): Syntax error.  Unexpected ';'.  Expecting ')'.	ERROR(150): Syntax error.  Unexpected ';'.  Expecting ')'.
ERROR(152): Syntax error.  Unexpected ++.  Expecting ')'.	ERROR(152): Syntax error.  Unexpected ++.  Expecting ')'.
ERROR(152): Syntax error.  Unexpected id: x.  Expecting ')'.	ERROR(152): Syntax error.  Unexpected id: x.  Expecting ')'.
ERROR(152): Syntax error.  Unexpected ++.  Expecting ')'.	ERROR(152): Syntax error.  Unexpected ++.  Expecting ')'.
ERROR(152): Syntax error.  Unexpected else.  Expecting ')'.	ERROR(152): Syntax error.  Unexpected else.  Expecting ')'.
ERROR(154): Syntax error.  Unexpected ++.  Expecting ')'.	ERROR(154): Syntax error.  Unexpected ++.  Expecting ')'.
ERROR(154): Syntax error.  Unexpected id: x.  Expecting ')'.	ERROR(154): Syntax error.  Unexpected id: x.  Expecting ')'.
ERROR(154): Syntax error.  Unexpected ++.  Expecting ')'.	ERROR(154): Syntax error.  Unexpected ++.  Expecting ')'.
ERROR(154): Syntax error.  Unexpected else.  Expecting ')'.	ERROR(154): Syntax error.  Unexpected else.  Expecting ')'.
ERROR(154): Syntax error.  Unexpected ';'.  Expecting ')'.	ERROR(154): Syntax error.  Unexpected ';'.  Expecting ')'.
ERROR(156): Syntax error.  Unexpected ++.  Expecting ')'.	ERROR(156): Syntax error.  Unexpected ++.  Expecting ')'.
ERROR(156): Syntax error.  Unexpected id: x.  Expecting ')'.	ERROR(156): Syntax error.  Unexpected id: x.  Expecting ')'.
ERROR(156): Syntax error.  Unexpected number: 111.  Expecting	ERROR(156): Syntax error.  Unexpected number: 111.  Expecting
ERROR(158): Syntax error.  Unexpected ++.  Expecting ')'.	ERROR(158): Syntax error.  Unexpected ++.  Expecting ')'.
ERROR(158): Syntax error.  Unexpected id: x.  Expecting ')'.	ERROR(158): Syntax error.  Unexpected id: x.  Expecting ')'.
ERROR(158): Syntax error.  Unexpected number: 111.  Expecting	ERROR(158): Syntax error.  Unexpected number: 111.  Expecting
ERROR(158): Syntax error.  Unexpected ';'.  Expecting ')'.	ERROR(158): Syntax error.  Unexpected ';'.  Expecting ')'.
ERROR(160): Syntax error.  Unexpected ++.  Expecting ')'.	ERROR(160): Syntax error.  Unexpected ++.  Expecting ')'.
ERROR(160): Syntax error.  Unexpected id: x.  Expecting ')'.	ERROR(160): Syntax error.  Unexpected id: x.  Expecting ')'.
ERROR(160): Syntax error.  Unexpected ++.  Expecting ')'.	ERROR(160): Syntax error.  Unexpected ++.  Expecting ')'.
ERROR(160): Syntax error.  Unexpected else.  Expecting ')'.	ERROR(160): Syntax error.  Unexpected else.  Expecting ')'.
ERROR(162): Syntax error.  Unexpected ++.  Expecting ')'.	ERROR(162): Syntax error.  Unexpected ++.  Expecting ')'.
ERROR(162): Syntax error.  Unexpected id: x.  Expecting ')'.	ERROR(162): Syntax error.  Unexpected id: x.  Expecting ')'.
ERROR(162): Syntax error.  Unexpected ++.  Expecting ')'.	ERROR(162): Syntax error.  Unexpected ++.  Expecting ')'.
ERROR(162): Syntax error.  Unexpected else.  Expecting ')'.	ERROR(162): Syntax error.  Unexpected else.  Expecting ')'.
ERROR(162): Syntax error.  Unexpected ';'.  Expecting ')'.	ERROR(162): Syntax error.  Unexpected ';'.  Expecting ')'.
ERROR(164): Syntax error.  Unexpected ++.  Expecting ')'.	ERROR(164): Syntax error.  Unexpected ++.  Expecting ')'.
ERROR(164): Syntax error.  Unexpected ++.  Expecting ')'.	ERROR(164): Syntax error.  Unexpected ++.  Expecting ')'.
ERROR(164): Syntax error.  Unexpected number: 111.  Expecting	ERROR(164): Syntax error.  Unexpected number: 111.  Expecting
ERROR(166): Syntax error.  Unexpected ++.  Expecting ')'.	ERROR(166): Syntax error.  Unexpected ++.  Expecting ')'.
ERROR(166): Syntax error.  Unexpected ++.  Expecting ')'.	ERROR(166): Syntax error.  Unexpected ++.  Expecting ')'.
ERROR(166): Syntax error.  Unexpected number: 111.  Expecting	ERROR(166): Syntax error.  Unexpected number: 111.  Expecting
ERROR(166): Syntax error.  Unexpected ';'.  Expecting ')'.	ERROR(166): Syntax error.  Unexpected ';'.  Expecting ')'.
ERROR(168): Syntax error.  Unexpected ++.  Expecting ')'.	ERROR(168): Syntax error.  Unexpected ++.  Expecting ')'.
ERROR(168): Syntax error.  Unexpected ++.  Expecting ')'.	ERROR(168): Syntax error.  Unexpected ++.  Expecting ')'.
ERROR(168): Syntax error.  Unexpected ++.  Expecting ')'.	ERROR(168): Syntax error.  Unexpected ++.  Expecting ')'.
ERROR(168): Syntax error.  Unexpected else.  Expecting ')'.	ERROR(168): Syntax error.  Unexpected else.  Expecting ')'.
ERROR(170): Syntax error.  Unexpected ++.  Expecting ')'.	ERROR(170): Syntax error.  Unexpected ++.  Expecting ')'.
ERROR(170): Syntax error.  Unexpected ++.  Expecting ')'.	ERROR(170): Syntax error.  Unexpected ++.  Expecting ')'.
ERROR(170): Syntax error.  Unexpected ++.  Expecting ')'.	ERROR(170): Syntax error.  Unexpected ++.  Expecting ')'.
ERROR(170): Syntax error.  Unexpected else.  Expecting ')'.	ERROR(170): Syntax error.  Unexpected else.  Expecting ')'.
ERROR(170): Syntax error.  Unexpected ';'.  Expecting ')'.	ERROR(170): Syntax error.  Unexpected ';'.  Expecting ')'.
Number of warnings: 0						Number of warnings: 0
Number of errors: 258						Number of errors: 258
RUN: c- < syntaxerr-ifminus.c-				      <
ERROR(10): Syntax error.  Unexpected else.			ERROR(10): Syntax error.  Unexpected else.
ERROR(16): Syntax error.  Unexpected number: 111.  Expecting 	ERROR(16): Syntax error.  Unexpected number: 111.  Expecting 
ERROR(16): Syntax error.  Unexpected else.			ERROR(16): Syntax error.  Unexpected else.
ERROR(18): Syntax error.  Unexpected number: 111.  Expecting 	ERROR(18): Syntax error.  Unexpected number: 111.  Expecting 
ERROR(18): Syntax error.  Unexpected else.			ERROR(18): Syntax error.  Unexpected else.
ERROR(20): Syntax error.  Unexpected else.  Expecting or or '	ERROR(20): Syntax error.  Unexpected else.  Expecting or or '
ERROR(22): Syntax error.  Unexpected else.  Expecting or or '	ERROR(22): Syntax error.  Unexpected else.  Expecting or or '
ERROR(24): Syntax error.  Unexpected number: 222.  Expecting 	ERROR(24): Syntax error.  Unexpected number: 222.  Expecting 
ERROR(26): Syntax error.  Unexpected ')'.			ERROR(26): Syntax error.  Unexpected ')'.
ERROR(26): Syntax error.  Unexpected number: 111.  Expecting 	ERROR(26): Syntax error.  Unexpected number: 111.  Expecting 
ERROR(26): Syntax error.  Unexpected else.			ERROR(26): Syntax error.  Unexpected else.
ERROR(28): Syntax error.  Unexpected ')'.			ERROR(28): Syntax error.  Unexpected ')'.
ERROR(28): Syntax error.  Unexpected number: 111.  Expecting 	ERROR(28): Syntax error.  Unexpected number: 111.  Expecting 
ERROR(28): Syntax error.  Unexpected else.			ERROR(28): Syntax error.  Unexpected else.
ERROR(30): Syntax error.  Unexpected ')'.			ERROR(30): Syntax error.  Unexpected ')'.
ERROR(30): Syntax error.  Unexpected else.  Expecting or or '	ERROR(30): Syntax error.  Unexpected else.  Expecting or or '
ERROR(32): Syntax error.  Unexpected ')'.			ERROR(32): Syntax error.  Unexpected ')'.
ERROR(32): Syntax error.  Unexpected else.  Expecting or or '	ERROR(32): Syntax error.  Unexpected else.  Expecting or or '
ERROR(34): Syntax error.  Unexpected ')'.			ERROR(34): Syntax error.  Unexpected ')'.
ERROR(34): Syntax error.  Unexpected number: 222.  Expecting 	ERROR(34): Syntax error.  Unexpected number: 222.  Expecting 
ERROR(36): Syntax error.  Unexpected ')'.			ERROR(36): Syntax error.  Unexpected ')'.
ERROR(36): Syntax error.  Unexpected ';'.  Expecting or or ')	ERROR(36): Syntax error.  Unexpected ';'.  Expecting or or ')
ERROR(38): Syntax error.  Unexpected ';'.  Expecting or or ')	ERROR(38): Syntax error.  Unexpected ';'.  Expecting or or ')
ERROR(38): Syntax error.  Unexpected else.			ERROR(38): Syntax error.  Unexpected else.
ERROR(40): Syntax error.  Unexpected ';'.  Expecting or or ')	ERROR(40): Syntax error.  Unexpected ';'.  Expecting or or ')
ERROR(40): Syntax error.  Unexpected else.			ERROR(40): Syntax error.  Unexpected else.
ERROR(42): Syntax error.  Unexpected else.			ERROR(42): Syntax error.  Unexpected else.
ERROR(44): Syntax error.  Unexpected else.			ERROR(44): Syntax error.  Unexpected else.
ERROR(46): Syntax error.  Unexpected id: x.  Expecting '('.	ERROR(46): Syntax error.  Unexpected id: x.  Expecting '('.
ERROR(48): Syntax error.  Unexpected id: x.  Expecting '('.	ERROR(48): Syntax error.  Unexpected id: x.  Expecting '('.
ERROR(50): Syntax error.  Unexpected id: x.  Expecting '('.	ERROR(50): Syntax error.  Unexpected id: x.  Expecting '('.
ERROR(52): Syntax error.  Unexpected id: x.  Expecting '('.	ERROR(52): Syntax error.  Unexpected id: x.  Expecting '('.
ERROR(54): Syntax error.  Unexpected id: x.  Expecting '('.	ERROR(54): Syntax error.  Unexpected id: x.  Expecting '('.
ERROR(56): Syntax error.  Unexpected id: x.  Expecting '('.	ERROR(56): Syntax error.  Unexpected id: x.  Expecting '('.
ERROR(58): Syntax error.  Unexpected id: x.  Expecting '('.	ERROR(58): Syntax error.  Unexpected id: x.  Expecting '('.
ERROR(58): Syntax error.  Unexpected else.			ERROR(58): Syntax error.  Unexpected else.
ERROR(60): Syntax error.  Unexpected id: x.  Expecting '('.	ERROR(60): Syntax error.  Unexpected id: x.  Expecting '('.
ERROR(60): Syntax error.  Unexpected else.			ERROR(60): Syntax error.  Unexpected else.
ERROR(62): Syntax error.  Unexpected id: x.  Expecting '('.	ERROR(62): Syntax error.  Unexpected id: x.  Expecting '('.
ERROR(68): Syntax error.  Unexpected number: 111.  Expecting 	ERROR(68): Syntax error.  Unexpected number: 111.  Expecting 
ERROR(68): Syntax error.  Unexpected else.			ERROR(68): Syntax error.  Unexpected else.
ERROR(70): Syntax error.  Unexpected ')'.  Expecting '('.	ERROR(70): Syntax error.  Unexpected ')'.  Expecting '('.
ERROR(72): Syntax error.  Unexpected ')'.  Expecting '('.	ERROR(72): Syntax error.  Unexpected ')'.  Expecting '('.
ERROR(74): Syntax error.  Unexpected ')'.  Expecting '('.	ERROR(74): Syntax error.  Unexpected ')'.  Expecting '('.
ERROR(76): Syntax error.  Unexpected ')'.  Expecting '('.	ERROR(76): Syntax error.  Unexpected ')'.  Expecting '('.
ERROR(78): Syntax error.  Unexpected ')'.  Expecting '('.	ERROR(78): Syntax error.  Unexpected ')'.  Expecting '('.
ERROR(80): Syntax error.  Unexpected number: 111.  Expecting 	ERROR(80): Syntax error.  Unexpected number: 111.  Expecting 
ERROR(80): Syntax error.  Unexpected else.			ERROR(80): Syntax error.  Unexpected else.
ERROR(82): Syntax error.  Unexpected number: 111.  Expecting 	ERROR(82): Syntax error.  Unexpected number: 111.  Expecting 
ERROR(82): Syntax error.  Unexpected else.			ERROR(82): Syntax error.  Unexpected else.
ERROR(84): Syntax error.  Unexpected ';'.  Expecting '('.	ERROR(84): Syntax error.  Unexpected ';'.  Expecting '('.
ERROR(86): Syntax error.  Unexpected number: 111.  Expecting 	ERROR(86): Syntax error.  Unexpected number: 111.  Expecting 
ERROR(86): Syntax error.  Unexpected else.			ERROR(86): Syntax error.  Unexpected else.
ERROR(88): Syntax error.  Unexpected number: 111.  Expecting 	ERROR(88): Syntax error.  Unexpected number: 111.  Expecting 
ERROR(88): Syntax error.  Unexpected else.			ERROR(88): Syntax error.  Unexpected else.
ERROR(90): Syntax error.  Unexpected else.  Expecting ';'.	ERROR(90): Syntax error.  Unexpected else.  Expecting ';'.
ERROR(92): Syntax error.  Unexpected else.  Expecting ';'.	ERROR(92): Syntax error.  Unexpected else.  Expecting ';'.
ERROR(94): Syntax error.  Unexpected number: 111.		ERROR(94): Syntax error.  Unexpected number: 111.
ERROR(94): Syntax error.  Unexpected ';'.  Expecting ')'.	ERROR(94): Syntax error.  Unexpected ';'.  Expecting ')'.
ERROR(94): Syntax error.  Unexpected else.			ERROR(94): Syntax error.  Unexpected else.
ERROR(96): Syntax error.  Unexpected number: 111.		ERROR(96): Syntax error.  Unexpected number: 111.
ERROR(96): Syntax error.  Unexpected ';'.  Expecting ')'.	ERROR(96): Syntax error.  Unexpected ';'.  Expecting ')'.
ERROR(96): Syntax error.  Unexpected else.			ERROR(96): Syntax error.  Unexpected else.
ERROR(98): Syntax error.  Unexpected else.			ERROR(98): Syntax error.  Unexpected else.
ERROR(102): Syntax error.  Unexpected number: 111.  Expecting	ERROR(102): Syntax error.  Unexpected number: 111.  Expecting
ERROR(104): Syntax error.  Unexpected number: 111.  Expecting	ERROR(104): Syntax error.  Unexpected number: 111.  Expecting
ERROR(106): Syntax error.  Unexpected else.  Expecting ')'.	ERROR(106): Syntax error.  Unexpected else.  Expecting ')'.
ERROR(108): Syntax error.  Unexpected else.  Expecting ')'.	ERROR(108): Syntax error.  Unexpected else.  Expecting ')'.
ERROR(118): Syntax error.  Unexpected number: 111.  Expecting	ERROR(118): Syntax error.  Unexpected number: 111.  Expecting
ERROR(120): Syntax error.  Unexpected number: 111.  Expecting	ERROR(120): Syntax error.  Unexpected number: 111.  Expecting
ERROR(122): Syntax error.  Unexpected else.  Expecting ')'.	ERROR(122): Syntax error.  Unexpected else.  Expecting ')'.
ERROR(124): Syntax error.  Unexpected else.  Expecting ')'.	ERROR(124): Syntax error.  Unexpected else.  Expecting ')'.
ERROR(134): Syntax error.  Unexpected number: 111.  Expecting	ERROR(134): Syntax error.  Unexpected number: 111.  Expecting
ERROR(136): Syntax error.  Unexpected number: 111.  Expecting	ERROR(136): Syntax error.  Unexpected number: 111.  Expecting
ERROR(138): Syntax error.  Unexpected else.  Expecting ')'.	ERROR(138): Syntax error.  Unexpected else.  Expecting ')'.
ERROR(140): Syntax error.  Unexpected else.  Expecting ')'.	ERROR(140): Syntax error.  Unexpected else.  Expecting ')'.
Number of warnings: 0						Number of warnings: 0
Number of errors: 76						Number of errors: 76
RUN: c- < syntaxerr-init.c-				      <
ERROR(2): Syntax error.  Unexpected ';'.			ERROR(2): Syntax error.  Unexpected ';'.
ERROR(3): Syntax error.  Unexpected ':'.  Expecting ',' or ';	ERROR(3): Syntax error.  Unexpected ':'.  Expecting ',' or ';
ERROR(6): Syntax error.  Unexpected number: 10.  Expecting ',	ERROR(6): Syntax error.  Unexpected number: 10.  Expecting ',
ERROR(8): Syntax error.  Unexpected ':'.  Expecting id.		ERROR(8): Syntax error.  Unexpected ':'.  Expecting id.
ERROR(9): Syntax error.  Unexpected ':'.  Expecting id.		ERROR(9): Syntax error.  Unexpected ':'.  Expecting id.
Number of warnings: 0						Number of warnings: 0
Number of errors: 5						Number of errors: 5
RUN: c- < syntaxerr-logic.c-				      <
ERROR(6): Syntax error.  Unexpected false.  Expecting ';'.	ERROR(6): Syntax error.  Unexpected false.  Expecting ';'.
ERROR(8): Syntax error.  Unexpected or.				ERROR(8): Syntax error.  Unexpected or.
ERROR(9): Syntax error.  Unexpected ';'.			ERROR(9): Syntax error.  Unexpected ';'.
ERROR(10): Syntax error.  Unexpected or.			ERROR(10): Syntax error.  Unexpected or.
ERROR(11): Syntax error.  Unexpected or.			ERROR(11): Syntax error.  Unexpected or.
ERROR(15): Syntax error.  Unexpected and.			ERROR(15): Syntax error.  Unexpected and.
ERROR(16): Syntax error.  Unexpected ';'.			ERROR(16): Syntax error.  Unexpected ';'.
ERROR(17): Syntax error.  Unexpected and.			ERROR(17): Syntax error.  Unexpected and.
ERROR(18): Syntax error.  Unexpected and.			ERROR(18): Syntax error.  Unexpected and.
ERROR(21): Syntax error.  Unexpected and.			ERROR(21): Syntax error.  Unexpected and.
Number of warnings: 0						Number of warnings: 0
Number of errors: 10						Number of errors: 10
RUN: c- < syntaxerr-parens.c-				      <
ERROR(5): Syntax error.  Unexpected ')'.  Expecting ';'.	ERROR(5): Syntax error.  Unexpected ')'.  Expecting ';'.
ERROR(6): Syntax error.  Unexpected ';'.  Expecting ')'.	ERROR(6): Syntax error.  Unexpected ';'.  Expecting ')'.
ERROR(7): Syntax error.  Unexpected ';'.  Expecting ')'.	ERROR(7): Syntax error.  Unexpected ';'.  Expecting ')'.
Number of warnings: 0						Number of warnings: 0
Number of errors: 3						Number of errors: 3
RUN: c- < syntaxerr-parms.c-				      <
ERROR(4): Syntax error.  Unexpected ')'.  Expecting id.		ERROR(4): Syntax error.  Unexpected ')'.  Expecting id.
ERROR(5): Syntax error.  Unexpected id: z.  Expecting bool or	ERROR(5): Syntax error.  Unexpected id: z.  Expecting bool or
ERROR(6): Syntax error.  Unexpected ')'.  Expecting bool or c	ERROR(6): Syntax error.  Unexpected ')'.  Expecting bool or c
ERROR(8): Syntax error.  Unexpected bool.  Expecting ')'.	ERROR(8): Syntax error.  Unexpected bool.  Expecting ')'.
ERROR(9): Syntax error.  Unexpected id: z.  Expecting ')'.	ERROR(9): Syntax error.  Unexpected id: z.  Expecting ')'.
ERROR(11): Syntax error.  Unexpected ';'.  Expecting id.	ERROR(11): Syntax error.  Unexpected ';'.  Expecting id.
ERROR(12): Syntax error.  Unexpected ';'.  Expecting id.	ERROR(12): Syntax error.  Unexpected ';'.  Expecting id.
ERROR(12): Syntax error.  Unexpected ')'.  Expecting id.	ERROR(12): Syntax error.  Unexpected ')'.  Expecting id.
ERROR(13): Syntax error.  Unexpected ';'.  Expecting id.	ERROR(13): Syntax error.  Unexpected ';'.  Expecting id.
ERROR(13): Syntax error.  Unexpected id: z.  Expecting bool o	ERROR(13): Syntax error.  Unexpected id: z.  Expecting bool o
ERROR(14): Syntax error.  Unexpected ';'.  Expecting id.	ERROR(14): Syntax error.  Unexpected ';'.  Expecting id.
ERROR(14): Syntax error.  Unexpected ')'.  Expecting bool or 	ERROR(14): Syntax error.  Unexpected ')'.  Expecting bool or 
ERROR(16): Syntax error.  Unexpected bool.  Expecting id.	ERROR(16): Syntax error.  Unexpected bool.  Expecting id.
ERROR(16): Syntax error.  Unexpected bool.  Expecting ')'.	ERROR(16): Syntax error.  Unexpected bool.  Expecting ')'.
ERROR(18): Syntax error.  Unexpected ')'.  Expecting id.	ERROR(18): Syntax error.  Unexpected ')'.  Expecting id.
ERROR(19): Syntax error.  Unexpected id: y.  Expecting ')'.	ERROR(19): Syntax error.  Unexpected id: y.  Expecting ')'.
ERROR(20): Syntax error.  Unexpected id: y.  Expecting ')'.	ERROR(20): Syntax error.  Unexpected id: y.  Expecting ')'.
ERROR(21): Syntax error.  Unexpected id: y.  Expecting ')'.	ERROR(21): Syntax error.  Unexpected id: y.  Expecting ')'.
ERROR(22): Syntax error.  Unexpected id: y.  Expecting ')'.	ERROR(22): Syntax error.  Unexpected id: y.  Expecting ')'.
ERROR(23): Syntax error.  Unexpected id: y.  Expecting ')'.	ERROR(23): Syntax error.  Unexpected id: y.  Expecting ')'.
ERROR(24): Syntax error.  Unexpected id: y.  Expecting ')'.	ERROR(24): Syntax error.  Unexpected id: y.  Expecting ')'.
ERROR(25): Syntax error.  Unexpected id: y.  Expecting ')'.	ERROR(25): Syntax error.  Unexpected id: y.  Expecting ')'.
ERROR(26): Syntax error.  Unexpected id: y.  Expecting ')'.	ERROR(26): Syntax error.  Unexpected id: y.  Expecting ')'.
ERROR(28): Syntax error.  Unexpected ')'.  Expecting id.	ERROR(28): Syntax error.  Unexpected ')'.  Expecting id.
ERROR(29): Syntax error.  Unexpected id: z.  Expecting bool o	ERROR(29): Syntax error.  Unexpected id: z.  Expecting bool o
ERROR(30): Syntax error.  Unexpected ')'.  Expecting bool or 	ERROR(30): Syntax error.  Unexpected ')'.  Expecting bool or 
ERROR(32): Syntax error.  Unexpected bool.  Expecting ')'.	ERROR(32): Syntax error.  Unexpected bool.  Expecting ')'.
ERROR(33): Syntax error.  Unexpected id: z.  Expecting ')'.	ERROR(33): Syntax error.  Unexpected id: z.  Expecting ')'.
ERROR(36): Syntax error.  Unexpected ','.  Expecting id.	ERROR(36): Syntax error.  Unexpected ','.  Expecting id.
ERROR(37): Syntax error.  Unexpected ','.  Expecting id.	ERROR(37): Syntax error.  Unexpected ','.  Expecting id.
ERROR(37): Syntax error.  Unexpected ')'.  Expecting id.	ERROR(37): Syntax error.  Unexpected ')'.  Expecting id.
ERROR(38): Syntax error.  Unexpected ','.  Expecting id.	ERROR(38): Syntax error.  Unexpected ','.  Expecting id.
ERROR(38): Syntax error.  Unexpected id: z.  Expecting bool o	ERROR(38): Syntax error.  Unexpected id: z.  Expecting bool o
ERROR(39): Syntax error.  Unexpected ','.  Expecting id.	ERROR(39): Syntax error.  Unexpected ','.  Expecting id.
ERROR(39): Syntax error.  Unexpected ')'.  Expecting bool or 	ERROR(39): Syntax error.  Unexpected ')'.  Expecting bool or 
ERROR(41): Syntax error.  Unexpected ','.  Expecting id.	ERROR(41): Syntax error.  Unexpected ','.  Expecting id.
ERROR(41): Syntax error.  Unexpected bool.  Expecting ')'.	ERROR(41): Syntax error.  Unexpected bool.  Expecting ')'.
ERROR(42): Syntax error.  Unexpected ','.  Expecting id.	ERROR(42): Syntax error.  Unexpected ','.  Expecting id.
ERROR(42): Syntax error.  Unexpected id: z.  Expecting ')'.	ERROR(42): Syntax error.  Unexpected id: z.  Expecting ')'.
ERROR(43): Syntax error.  Unexpected ','.  Expecting id.	ERROR(43): Syntax error.  Unexpected ','.  Expecting id.
ERROR(44): Syntax error.  Unexpected ','.  Expecting id.	ERROR(44): Syntax error.  Unexpected ','.  Expecting id.
Number of warnings: 0						Number of warnings: 0
Number of errors: 41						Number of errors: 41
RUN: c- < syntaxerr-small.c-				      <
WARNING(5): Invalid input character: ^.  Character ignored.	WARNING(5): Invalid input character: ^.  Character ignored.
ERROR(5): Syntax error.  Unexpected id: k.  Expecting ']'.	ERROR(5): Syntax error.  Unexpected id: k.  Expecting ']'.
ERROR(6): Syntax error.  Unexpected id: x.			ERROR(6): Syntax error.  Unexpected id: x.
ERROR(7): Syntax error.  Unexpected '='.  Expecting '('.	ERROR(7): Syntax error.  Unexpected '='.  Expecting '('.
ERROR(8): Syntax error.  Unexpected id: x0.  Expecting ';'.	ERROR(8): Syntax error.  Unexpected id: x0.  Expecting ';'.
ERROR(9): Syntax error.  Unexpected '+'.			ERROR(9): Syntax error.  Unexpected '+'.
WARNING(14): Invalid input character: @.  Character ignored.	WARNING(14): Invalid input character: @.  Character ignored.
WARNING(14): Invalid input character: #.  Character ignored.	WARNING(14): Invalid input character: #.  Character ignored.
ERROR(14): Syntax error.  Unexpected ';'.  Expecting ')'.	ERROR(14): Syntax error.  Unexpected ';'.  Expecting ')'.
ERROR(15): Syntax error.  Unexpected ';'.  Expecting ')'.	ERROR(15): Syntax error.  Unexpected ';'.  Expecting ')'.
ERROR(16): Syntax error.  Unexpected ';'.  Expecting ')'.	ERROR(16): Syntax error.  Unexpected ';'.  Expecting ')'.
Number of warnings: 3						Number of warnings: 3
Number of errors: 8						Number of errors: 8
RUN: c- < syntaxerr-summul.c-				      <
ERROR(6): Syntax error.  Unexpected number: 222.  Expecting '	ERROR(6): Syntax error.  Unexpected number: 222.  Expecting '
ERROR(8): Syntax error.  Unexpected '+'.			ERROR(8): Syntax error.  Unexpected '+'.
ERROR(9): Syntax error.  Unexpected ';'.			ERROR(9): Syntax error.  Unexpected ';'.
ERROR(10): Syntax error.  Unexpected '+'.			ERROR(10): Syntax error.  Unexpected '+'.
ERROR(11): Syntax error.  Unexpected '+'.			ERROR(11): Syntax error.  Unexpected '+'.
ERROR(15): Syntax error.  Unexpected ';'.			ERROR(15): Syntax error.  Unexpected ';'.
ERROR(16): Syntax error.  Unexpected ';'.			ERROR(16): Syntax error.  Unexpected ';'.
ERROR(18): Syntax error.  Unexpected ';'.			ERROR(18): Syntax error.  Unexpected ';'.
ERROR(21): Syntax error.  Unexpected '+'.			ERROR(21): Syntax error.  Unexpected '+'.
ERROR(22): Syntax error.  Unexpected '+'.			ERROR(22): Syntax error.  Unexpected '+'.
Number of warnings: 0						Number of warnings: 0
Number of errors: 10						Number of errors: 10
RUN: c- < syntaxerr-type.c-				      <
ERROR(2): Syntax error.  Unexpected ';'.  Expecting id.		ERROR(2): Syntax error.  Unexpected ';'.  Expecting id.
ERROR(3): Syntax error.  Unexpected id: b.  Expecting ',' or 	ERROR(3): Syntax error.  Unexpected id: b.  Expecting ',' or 
ERROR(4): Syntax error.  Unexpected int.  Expecting id.		ERROR(4): Syntax error.  Unexpected int.  Expecting id.
ERROR(5): Syntax error.  Unexpected int.  Expecting id.		ERROR(5): Syntax error.  Unexpected int.  Expecting id.
Number of warnings: 0						Number of warnings: 0
Number of errors: 4						Number of errors: 4
RUN: c- < syntaxerr-typearray.c-			      <
ERROR(4): Syntax error.  Unexpected id: x.  Expecting number.	ERROR(4): Syntax error.  Unexpected id: x.  Expecting number.
ERROR(6): Syntax error.  Unexpected '*'.  Expecting ']'.	ERROR(6): Syntax error.  Unexpected '*'.  Expecting ']'.
ERROR(11): Syntax error.  Unexpected ';'.  Expecting ']'.	ERROR(11): Syntax error.  Unexpected ';'.  Expecting ']'.
ERROR(13): Syntax error.  Unexpected ']'.  Expecting number.	ERROR(13): Syntax error.  Unexpected ']'.  Expecting number.
ERROR(15): Syntax error.  Unexpected ';'.  Expecting number.	ERROR(15): Syntax error.  Unexpected ';'.  Expecting number.
ERROR(17): Syntax error.  Unexpected number: 33.  Expecting '	ERROR(17): Syntax error.  Unexpected number: 33.  Expecting '
ERROR(19): Syntax error.  Unexpected number: 33.  Expecting '	ERROR(19): Syntax error.  Unexpected number: 33.  Expecting '
ERROR(21): Syntax error.  Unexpected ']'.  Expecting ',' or '	ERROR(21): Syntax error.  Unexpected ']'.  Expecting ',' or '
ERROR(25): Syntax error.  Unexpected '['.  Expecting id.	ERROR(25): Syntax error.  Unexpected '['.  Expecting id.
ERROR(27): Syntax error.  Unexpected '['.  Expecting id.	ERROR(27): Syntax error.  Unexpected '['.  Expecting id.
ERROR(29): Syntax error.  Unexpected '['.  Expecting id.	ERROR(29): Syntax error.  Unexpected '['.  Expecting id.
ERROR(31): Syntax error.  Unexpected '['.  Expecting id.	ERROR(31): Syntax error.  Unexpected '['.  Expecting id.
ERROR(33): Syntax error.  Unexpected number: 33.  Expecting i	ERROR(33): Syntax error.  Unexpected number: 33.  Expecting i
ERROR(35): Syntax error.  Unexpected number: 33.  Expecting i	ERROR(35): Syntax error.  Unexpected number: 33.  Expecting i
ERROR(37): Syntax error.  Unexpected ']'.  Expecting id.	ERROR(37): Syntax error.  Unexpected ']'.  Expecting id.
ERROR(41): Syntax error.  Unexpected bool.  Expecting ']'.	ERROR(41): Syntax error.  Unexpected bool.  Expecting ']'.
ERROR(43): Syntax error.  Unexpected ']'.  Expecting number.	ERROR(43): Syntax error.  Unexpected ']'.  Expecting number.
ERROR(45): Syntax error.  Unexpected bool.  Expecting number.	ERROR(45): Syntax error.  Unexpected bool.  Expecting number.
ERROR(47): Syntax error.  Unexpected number: 33.  Expecting '	ERROR(47): Syntax error.  Unexpected number: 33.  Expecting '
ERROR(49): Syntax error.  Unexpected number: 33.  Expecting '	ERROR(49): Syntax error.  Unexpected number: 33.  Expecting '
ERROR(51): Syntax error.  Unexpected ']'.  Expecting ',' or '	ERROR(51): Syntax error.  Unexpected ']'.  Expecting ',' or '
ERROR(53): Syntax error.  Unexpected bool.  Expecting ',' or 	ERROR(53): Syntax error.  Unexpected bool.  Expecting ',' or 
ERROR(55): Syntax error.  Unexpected '['.  Expecting id.	ERROR(55): Syntax error.  Unexpected '['.  Expecting id.
ERROR(57): Syntax error.  Unexpected '['.  Expecting id.	ERROR(57): Syntax error.  Unexpected '['.  Expecting id.
ERROR(59): Syntax error.  Unexpected '['.  Expecting id.	ERROR(59): Syntax error.  Unexpected '['.  Expecting id.
ERROR(61): Syntax error.  Unexpected '['.  Expecting id.	ERROR(61): Syntax error.  Unexpected '['.  Expecting id.
ERROR(63): Syntax error.  Unexpected number: 33.  Expecting i	ERROR(63): Syntax error.  Unexpected number: 33.  Expecting i
ERROR(65): Syntax error.  Unexpected number: 33.  Expecting i	ERROR(65): Syntax error.  Unexpected number: 33.  Expecting i
ERROR(67): Syntax error.  Unexpected ']'.  Expecting id.	ERROR(67): Syntax error.  Unexpected ']'.  Expecting id.
ERROR(69): Syntax error.  Unexpected bool.  Expecting id.	ERROR(69): Syntax error.  Unexpected bool.  Expecting id.
Number of warnings: 0						Number of warnings: 0
Number of errors: 30						Number of errors: 30
RUN: c- < syntaxerr-typefun.c-				      <
ERROR(9): Syntax error.  Unexpected '*'.  Expecting bool or c	ERROR(9): Syntax error.  Unexpected '*'.  Expecting bool or c
ERROR(14): Syntax error.  Unexpected ';'.  Expecting bool or 	ERROR(14): Syntax error.  Unexpected ';'.  Expecting bool or 
ERROR(20): Syntax error.  Unexpected '('.  Expecting id.	ERROR(20): Syntax error.  Unexpected '('.  Expecting id.
ERROR(22): Syntax error.  Unexpected '('.  Expecting id.	ERROR(22): Syntax error.  Unexpected '('.  Expecting id.
ERROR(24): Syntax error.  Unexpected ')'.  Expecting id.	ERROR(24): Syntax error.  Unexpected ')'.  Expecting id.
ERROR(26): Syntax error.  Unexpected ';'.  Expecting id.	ERROR(26): Syntax error.  Unexpected ';'.  Expecting id.
ERROR(30): Syntax error.  Unexpected ';'.  Expecting bool or 	ERROR(30): Syntax error.  Unexpected ';'.  Expecting bool or 
ERROR(32): Syntax error.  Unexpected id: dog.  Expecting bool	ERROR(32): Syntax error.  Unexpected id: dog.  Expecting bool
ERROR(34): Syntax error.  Unexpected ';'.  Expecting '('.	ERROR(34): Syntax error.  Unexpected ';'.  Expecting '('.
ERROR(38): Syntax error.  Unexpected '('.			ERROR(38): Syntax error.  Unexpected '('.
ERROR(42): Syntax error.  Unexpected ';'.			ERROR(42): Syntax error.  Unexpected ';'.
ERROR(47): Syntax error.  Unexpected '*'.  Expecting bool or 	ERROR(47): Syntax error.  Unexpected '*'.  Expecting bool or 
ERROR(51): Syntax error.  Unexpected '*'.  Expecting ',' or '	ERROR(51): Syntax error.  Unexpected '*'.  Expecting ',' or '
ERROR(53): Syntax error.  Unexpected '('.  Expecting id.	ERROR(53): Syntax error.  Unexpected '('.  Expecting id.
ERROR(55): Syntax error.  Unexpected '('.  Expecting id.	ERROR(55): Syntax error.  Unexpected '('.  Expecting id.
ERROR(57): Syntax error.  Unexpected '*'.  Expecting id.	ERROR(57): Syntax error.  Unexpected '*'.  Expecting id.
ERROR(59): Syntax error.  Unexpected '*'.  Expecting id.	ERROR(59): Syntax error.  Unexpected '*'.  Expecting id.
ERROR(61): Syntax error.  Unexpected id: dog.  Expecting bool	ERROR(61): Syntax error.  Unexpected id: dog.  Expecting bool
ERROR(63): Syntax error.  Unexpected id: dog.  Expecting bool	ERROR(63): Syntax error.  Unexpected id: dog.  Expecting bool
ERROR(65): Syntax error.  Unexpected id: dog.  Expecting bool	ERROR(65): Syntax error.  Unexpected id: dog.  Expecting bool
ERROR(67): Syntax error.  Unexpected id: dog.  Expecting bool	ERROR(67): Syntax error.  Unexpected id: dog.  Expecting bool
ERROR(69): Syntax error.  Unexpected '('.  Expecting bool or 	ERROR(69): Syntax error.  Unexpected '('.  Expecting bool or 
ERROR(71): Syntax error.  Unexpected '('.  Expecting bool or 	ERROR(71): Syntax error.  Unexpected '('.  Expecting bool or 
ERROR(73): Syntax error.  Unexpected '*'.  Expecting bool or 	ERROR(73): Syntax error.  Unexpected '*'.  Expecting bool or 
ERROR(75): Syntax error.  Unexpected '*'.  Expecting bool or 	ERROR(75): Syntax error.  Unexpected '*'.  Expecting bool or 
ERROR(78): Syntax error.  Unexpected '('.  Expecting ')'.	ERROR(78): Syntax error.  Unexpected '('.  Expecting ')'.
ERROR(80): Syntax error.  Unexpected '('.  Expecting ')'.	ERROR(80): Syntax error.  Unexpected '('.  Expecting ')'.
ERROR(82): Syntax error.  Unexpected bool.  Expecting ')'.	ERROR(82): Syntax error.  Unexpected bool.  Expecting ')'.
ERROR(84): Syntax error.  Unexpected bool.  Expecting ')'.	ERROR(84): Syntax error.  Unexpected bool.  Expecting ')'.
ERROR(86): Syntax error.  Unexpected '('.  Expecting id.	ERROR(86): Syntax error.  Unexpected '('.  Expecting id.
ERROR(88): Syntax error.  Unexpected '('.  Expecting id.	ERROR(88): Syntax error.  Unexpected '('.  Expecting id.
ERROR(90): Syntax error.  Unexpected bool.  Expecting id.	ERROR(90): Syntax error.  Unexpected bool.  Expecting id.
ERROR(92): Syntax error.  Unexpected bool.  Expecting id.	ERROR(92): Syntax error.  Unexpected bool.  Expecting id.
ERROR(94): Syntax error.  Unexpected id: dog.  Expecting bool	ERROR(94): Syntax error.  Unexpected id: dog.  Expecting bool
ERROR(96): Syntax error.  Unexpected id: dog.  Expecting bool	ERROR(96): Syntax error.  Unexpected id: dog.  Expecting bool
ERROR(98): Syntax error.  Unexpected id: dog.  Expecting bool	ERROR(98): Syntax error.  Unexpected id: dog.  Expecting bool
ERROR(100): Syntax error.  Unexpected id: dog.  Expecting boo	ERROR(100): Syntax error.  Unexpected id: dog.  Expecting boo
ERROR(102): Syntax error.  Unexpected '('.  Expecting bool or	ERROR(102): Syntax error.  Unexpected '('.  Expecting bool or
ERROR(104): Syntax error.  Unexpected '('.  Expecting bool or	ERROR(104): Syntax error.  Unexpected '('.  Expecting bool or
Number of warnings: 0						Number of warnings: 0
Number of errors: 39						Number of errors: 39
RUN: c- < syntaxerr-unary.c-				      <
ERROR(5): Syntax error.  Unexpected while.			ERROR(5): Syntax error.  Unexpected while.
ERROR(6): Syntax error.  Unexpected ';'.			ERROR(6): Syntax error.  Unexpected ';'.
ERROR(8): Syntax error.  Unexpected while.			ERROR(8): Syntax error.  Unexpected while.
ERROR(9): Syntax error.  Unexpected ';'.			ERROR(9): Syntax error.  Unexpected ';'.
ERROR(11): Syntax error.  Unexpected while.			ERROR(11): Syntax error.  Unexpected while.
ERROR(12): Syntax error.  Unexpected ';'.			ERROR(12): Syntax error.  Unexpected ';'.
Number of warnings: 0						Number of warnings: 0
Number of errors: 6						Number of errors: 6
RUN: c- < syntaxerr-while.c-				      <
ERROR(6): Syntax error.  Unexpected int.  Expecting or or ')'	ERROR(6): Syntax error.  Unexpected int.  Expecting or or ')'
ERROR(8): Syntax error.  Unexpected int.			ERROR(8): Syntax error.  Unexpected int.
ERROR(10): Syntax error.  Unexpected int.			ERROR(10): Syntax error.  Unexpected int.
ERROR(12): Syntax error.  Unexpected int.  Expecting '('.	ERROR(12): Syntax error.  Unexpected int.  Expecting '('.
ERROR(14): Syntax error.  Unexpected int.  Expecting '('.	ERROR(14): Syntax error.  Unexpected int.  Expecting '('.
ERROR(18): Syntax error.  Unexpected int.  Expecting '('.	ERROR(18): Syntax error.  Unexpected int.  Expecting '('.
ERROR(23): Syntax error.  Unexpected ';'.			ERROR(23): Syntax error.  Unexpected ';'.
ERROR(25): Syntax error.  Unexpected ')'.			ERROR(25): Syntax error.  Unexpected ')'.
ERROR(25): Syntax error.  Unexpected ';'.  Expecting or or ')	ERROR(25): Syntax error.  Unexpected ';'.  Expecting or or ')
ERROR(27): Syntax error.  Unexpected ';'.			ERROR(27): Syntax error.  Unexpected ';'.
ERROR(29): Syntax error.  Unexpected '*'.  Expecting '('.	ERROR(29): Syntax error.  Unexpected '*'.  Expecting '('.
ERROR(29): Syntax error.  Unexpected ')'.  Expecting ';'.	ERROR(29): Syntax error.  Unexpected ')'.  Expecting ';'.
ERROR(31): Syntax error.  Unexpected '*'.  Expecting '('.	ERROR(31): Syntax error.  Unexpected '*'.  Expecting '('.
ERROR(31): Syntax error.  Unexpected ';'.			ERROR(31): Syntax error.  Unexpected ';'.
ERROR(33): Syntax error.  Unexpected '*'.  Expecting '('.	ERROR(33): Syntax error.  Unexpected '*'.  Expecting '('.
ERROR(35): Syntax error.  Unexpected '*'.  Expecting '('.	ERROR(35): Syntax error.  Unexpected '*'.  Expecting '('.
ERROR(35): Syntax error.  Unexpected ';'.			ERROR(35): Syntax error.  Unexpected ';'.
ERROR(40): Syntax error.  Unexpected ';'.  Expecting or or ')	ERROR(40): Syntax error.  Unexpected ';'.  Expecting or or ')
ERROR(42): Syntax error.  Unexpected ')'.			ERROR(42): Syntax error.  Unexpected ')'.
ERROR(42): Syntax error.  Unexpected ';'.  Expecting or or ')	ERROR(42): Syntax error.  Unexpected ';'.  Expecting or or ')
ERROR(44): Syntax error.  Unexpected ';'.			ERROR(44): Syntax error.  Unexpected ';'.
ERROR(46): Syntax error.  Unexpected id: x.  Expecting '('.	ERROR(46): Syntax error.  Unexpected id: x.  Expecting '('.
ERROR(48): Syntax error.  Unexpected id: x.  Expecting '('.	ERROR(48): Syntax error.  Unexpected id: x.  Expecting '('.
ERROR(50): Syntax error.  Unexpected ')'.  Expecting '('.	ERROR(50): Syntax error.  Unexpected ')'.  Expecting '('.
ERROR(52): Syntax error.  Unexpected ';'.  Expecting '('.	ERROR(52): Syntax error.  Unexpected ';'.  Expecting '('.
Number of warnings: 0						Number of warnings: 0
Number of errors: 25						Number of errors: 25
							      <
End of testing					      <
\end{lstlisting}

\chapter{风向}
\label{sec-50}
\section{风向(1)}
\label{sec-50-1}

我是那个这学期一边上着编译课、同时还每周在学校食堂里打着十五六小时工的学生,于是,对于编译课这个代课老师的风向,还是稍有把握的。

前面提到过了,这个学期食堂里,我的打工环境已经比上学期好了很多。实际上,更精准的话应该是,这学期的打工人文环境是上学期的取反(-),因为这学期,多了太多想要讨好我的manager一一绝大部分时间被安排在classics这个相对轻松的station暂且不说,有一阵儿还有年轻的小美校园大学生shut down时在我耳边唱歌\textasciitilde{}~ 而与这样的环境相对应的是学校校园里的活动,比如九月底的career fair,我甚至根本就没有去,连进去逛也一下也没有,因为那时他们正在打我跟编译课老师作research的牌。说得好听的舆论打法是,这个在普朴工作过15年的人,如果我research作得好,等我毕业了,别人可以帮我推荐工作!亲,我还没有开口说话,大概你们也已经猜到我想说什么了一一这话在我就是一阵风,吹过了就什么也没有了!别说research我能不能真正作得好还成问题,就算是自己真正作得好了,能发几篇文章么,不想要学术圈混的人文章要来是作什么用的?这会儿舆论风口上,别人说的话如此好听动人,到时候呢,一年以后、两年以后,谁还记得谁,别人让不让毕业、会不会再拖,我拖得起么,这话我能信吗?而且这个"变态"老师名下,多少冤魂野鬼,半途而废的、中途退学的、直接放弃博士出去工作的,有相当高的比例,难道我一定要明知山有虎、偏向虎山行,自已同自己过不去?不,这个导师,他多我一个不多,少我一个不少,我可不想去作那冤魂野鬼\textasciitilde{}~ ;)

后来,我基本上把食堂里的manager都快得罪光了,因为此人顽固不化,太不支持配合他们的活动了!

后来,观察得多了,慢慢也就看明白了他们一一食堂、学校的活动,与代课老师三方合作的牌到底在怎么打了!

\section{风向(2)}
\label{sec-50-2}

这个编译课的老师,在打着我与表哥的爱情关系牌。他时而,他又会故意以讲assembly language这样大家都会发懵、都会头大、都会对这门课的作业缺少信心和安全感的课堂内容,而且光速扫进、速度奇快。他以这种方式在课堂上打击鄙视我(们,主要是针对我),嫌我学习不好,大概还是想要极尽所能地压我在这所残破的小学校多呆一年吧\textasciitilde{}~

他时而也在上课的课堂上提拔、鼓励我们一一滚滚长江东逝水,浪花滔尽英雄一一以讲compiler的framework,讲他之前快速跳过的细节来增加我们的信心。当然,当老师们对我进行可控范围内的提拔的时候,这种提拔也并不仅只来自同一个老师,比如硬件课的老师就也有一次课堂上,公开否定过另一位印度女生的意见,来鼓励提携我。那个代课老师课堂上这么说过:"上过编译课、懂得assembly language,懂得代码优化,甚至过一些实习经验,那找工作就很容易了(还是别的一句什么,最后这一句记得不是很牢固)!"

而编译课老师这上课的节奏的快慢与否,也与食堂、学校活动的结果紧密相关。有一次学校食堂里又在开一个什么活动,根据个人判断,那会儿他们在打我与表哥分手一一从我与同学们的互动他们大概也会看得出来,在与表哥的爱情里,我的灵魂正在游走!虽然在网上、分好几次,我费了九牛二虎之力写完了自己的三生三世,写完了自己那场爱情;但这并不防碍别人想要以食堂里活动的形式将这种分手放大。

\section{风向(3)}
\label{sec-50-3}

我一一的状态在哪里?我不知道\textasciitilde{}~(后来,当偶遇见QQ群里滴大神,答案就揭晓了吧!当然,这是后话)!

如果有一天,表哥恋爱了,或者真如当年舅母所说,表哥与韩国哪位那时的美女(前女友)订婚了,我想我一定不会恨一一表哥与我之间永远不该有恨、不会有恨;舅舅和表哥都是在我漫长、支离破碎的成长岁月中给过我最真切关怀与温暖的人,这些年来,这份感情里,有血浓于水的亲情、有过最真切的感动与爱,有踏踏实实、实实在在的温暖,要我去恨,我做不到!我应该会为表哥送上默默祝福的吧\textasciitilde{}~ 

如果有一天,我遇到了让自己动心的人,我想我大概也该会一一没有遗憾,不再眷恋,大踏步迈向新生活吧\textasciitilde{}~ smileface

但无论如何,舅舅、表哥这一家、我与表哥的爱情,绝对不deserve、不能也不应该,以这种敲锣打鼓、庆祝的方式送别离开!

彼时,与这个家庭千丝万缕的联系,那个我曾经心心念念牵挂着的人,在表哥获得确定的幸福之前,抑或在我遇到让自己动心的人之前,它都像自己身上的一个肿瘤、一块肉般全然成为自己身体的一部分,水浮交融,任何勉强的外力拉扭、强行分开,都必将会血肉模糊、生离死别、痛彻心菲\textasciitilde{}~ 关于这场爱情的结局,还是把它留给缘份、留给时间来解决吧\textasciitilde{}~

那晚,我在沙位吧表情冷峻;manager过来问我,是不是饿了想take break还是回家休息时,我的反应很诧异一一早前九月底我身体那么虚弱的时候,我也坚持把那天的shift值完,怎么可能因为饿不好好工作呢?!我受不了、不能容忍的应该是一种蝇头小利、鼠目寸光的作法吧\textasciitilde{}~
% Emacs 24.3.1 (Org mode 8.2.7c)
\end{document}